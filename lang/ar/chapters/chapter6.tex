\qrchapter{https://forgottenpillar.com/rsc/en-fp-chapter6}{Examining the test}


\qrchapter{https://forgottenpillar.com/rsc/en-fp-chapter6}{فحص الاختبار}


In Sister White's reply, to Dr. Kellogg's belief on the Trinity doctrine and his attempts to \textit{patch up} the Living Temple, we see that she viewed the Trinity doctrine as contradicting the light given her regarding \emcap{the personality of God}. If she had actually embraced the Trinity doctrine, we would expect her to carefully separate it from pantheism and preserve its legitimate aspects. However, this is not what we see in her response. Instead, her response was to contrast the Trinity doctrine with the truth about the \emcap{personality of God}, recalling her past visions which showed that this doctrine would rob God's people of their past experiences. In her reactive recalling of how God established the \emcap{fundamental principles}, she indicated that the Trinity doctrine \textit{tears down the pillars of our faith} and \textit{leads us astray from the foundation principles}. This stark difference can be clearly seen by comparing our current Fundamental Beliefs with the \emcap{Fundamental Principles} held in the past.


في رد الأخت وايت على معتقد الدكتور كيلوغ حول عقيدة الثالوث ومحاولاته \textit{ترقيع} كتاب ذا ليفينغ تمبل، نرى أنها اعتبرت عقيدة الثالوث متناقضة مع النور المعطى لها بخصوص \emcap{شخصانية الله}. لو كانت قد تبنت بالفعل عقيدة الثالوث، لتوقعنا منها أن تفصلها بعناية عن وحدة الوجود وتحافظ على جوانبها المشروعة. ومع ذلك، هذا ليس ما نراه في ردها. بدلاً من ذلك، كان ردها هو مقارنة عقيدة الثالوث بالحق حول \emcap{شخصانية الله}، مستذكرة رؤاها السابقة التي أظهرت أن هذه العقيدة ستسلب شعب الله تجاربهم السابقة. في استذكارها التفاعلي لكيفية تأسيس الله \emcap{للمبادئ الجوهرية}، أشارت إلى أن عقيدة الثالوث \textit{تهدم أعمدة إيماننا} و\textit{تقودنا بعيدًا عن المبادئ الأساسية}. يمكن رؤية هذا الاختلاف الصارخ بوضوح من خلال مقارنة معتقداتنا الأساسية الحالية مع \emcap{المبادئ الأساسية} التي كانت موجودة في الماضي.


Keeping in mind Sister White’s reply to Dr. Kellogg's belief on the Trinity doctrine, let us review the characteristics of the theories she described in the chapter “\textit{The Foundation of our Faith}”. When Sister White is speaking of Kellogg’s theories of God, our question should be, “do her quotations make sense if the Trinity doctrine is applied to their context?” Let’s examine each characteristic.


مع الأخذ في الاعتبار رد الأخت وايت على معتقد الدكتور كيلوغ حول عقيدة الثالوث، دعونا نراجع خصائص النظريات التي وصفتها في فصل “\textit{أساس إيماننا}”. عندما تتحدث الأخت وايت عن نظريات كيلوغ عن الله، يجب أن يكون سؤالنا، “هل اقتباساتها منطقية إذا تم تطبيق عقيدة الثالوث على سياقها؟“ دعونا نفحص كل خاصية.


\subsection*{Does the Trinity “rob the people of God of their past experience”?}


\subsection*{هل الثالوث “يسلب شعب الله تجربتهم السابقة”؟}


\egw{They \normaltext{[the spiritualistic theories]} make of no effect the truth of heavenly origin, and \textbf{rob the people of God of their past experience}, giving them instead a false science.}[SpTB02 54.1; 1904][https://egwwritings.org/read?panels=p417.275]


\egw{إنها \normaltext{[النظريات الروحانية]} تجعل الحق ذا الأصل السماوي بلا تأثير، و\textbf{تسلب شعب الله تجربتهم السابقة}، وتعطيهم بدلاً منها علمًا زائفًا.}[SpTB02 54.1; 1904][https://egwwritings.org/read?panels=p417.275]


\egw{This foundation was built by the Masterworker, and will stand storm and tempest. Will they permit this man \normaltext{[Kellogg]} to present \textbf{doctrines that deny the past experience of the people of God}? The time has come to take decided action.}[SpTB02 54.2; 1904][https://egwwritings.org/read?panels=p417.276]


\egw{هذا الأساس بناه المعلم الرئيسي، وسيصمد أمام العاصفة والزوبعة. هل سيسمحون لهذا الرجل \normaltext{[كيلوغ]} بتقديم \textbf{عقائد تنكر التجربة السابقة لشعب الله}؟ لقد حان الوقت لاتخاذ إجراء حاسم.}[SpTB02 54.2; 1904][https://egwwritings.org/read?panels=p417.276]


\egw{\textbf{What influence is it that would lead men at this stage of our history to work in an underhanded, powerful way to \underline{tear down the foundation of our faith},—the foundation that was laid \underline{at the beginning of our work} by prayerful study of the word and by revelation? Upon this foundation \underline{we have been building for the past fifty years}}. Do you wonder that when I see the beginning of a work \textbf{that would \underline{remove some of the pillars of our faith},} I have something to say? I must obey the command, ‘Meet it!’}[SpTB02 58.1; 1904][https://egwwritings.org/read?panels=p417.295]


\egw{\textbf{ما هو التأثير الذي يقود الناس في هذه المرحلة من تاريخنا للعمل بطريقة خفية وقوية \underline{لهدم أساس إيماننا}،—الأساس الذي وُضع \underline{في بداية عملنا} من خلال الدراسة المصلية للكلمة وبالوحي؟ على هذا الأساس \underline{كنا نبني طوال الخمسين سنة الماضية}}. هل تتعجبون أنني عندما أرى بداية عمل \textbf{من شأنه أن \underline{يزيل بعض أعمدة إيماننا}،} يكون لدي ما أقوله؟ يجب أن أطيع الأمر، ‘واجهه!’}[SpTB02 58.1; 1904][https://egwwritings.org/read?panels=p417.295]


According to Sister White’s testimony, the foundation of our faith was the \emcap{Fundamental Principles}. Currently, these do not represent our beliefs. Most objectionable is the first point, concerning who God is. Instead of the belief that there is one God—the Father, a personal spiritual being, we have a new belief that there is one God—Father, Son, and Holy Spirit, a unity of three Persons. From the light and the experiences of how God established the first point of the \emcap{Fundamental Principles}, does the newly formed doctrine about who God is and what He is, has robbed the people of God of their past experience?


وفقًا لشهادة الأخت وايت، فإن أساس إيماننا كان \emcap{المبادئ الأساسية}. حاليًا، هذه لا تمثل معتقداتنا. الأكثر اعتراضًا هو النقطة الأولى، المتعلقة بمن هو الله. بدلاً من الاعتقاد بأن هناك إله واحد—الآب، كائن روحي شخصي، لدينا معتقد جديد بأن هناك إله واحد—الآب والابن والروح القدس، وحدة من ثلاثة أشخاص. من النور والتجارب حول كيفية تأسيس الله للنقطة الأولى من \emcap{المبادئ الأساسية}، هل العقيدة المشكلة حديثًا حول من هو الله وما هو، قد سلبت شعب الله تجربتهم السابقة؟


\subsection*{Does the Trinity tear down the pillars of our faith, or lead astray from foundation principles?}


\subsection*{هل الثالوث يهدم أعمدة إيماننا، أو يقود بعيدًا عن المبادئ الأساسية؟}


\egw{I have been instructed by the heavenly messenger that some of the reasoning in the book, ‘Living Temple,’ is unsound and that \textbf{this reasoning would lead astray the minds of those who are not thoroughly established on the foundation principles of present truth.}}[SpTB02 51.3; 1904][https://egwwritings.org/read?panels=p417.262]


\egw{لقد تلقيت تعليمات من الرسول السماوي بأن بعض المنطق في كتاب ‘ذا ليفينغ تمبل’ غير سليم وأن \textbf{هذا المنطق سيقود عقول أولئك الذين ليسوا راسخين تمامًا على المبادئ الأساسية للحق الحاضر إلى الضلال.}}[SpTB02 51.3; 1904][https://egwwritings.org/read?panels=p417.262]


\egw{About the time that ‘Living Temple’ was published, there passed before me in the night season, representations indicating that some \textbf{danger was approaching}, and that I must prepare for it by writing out the things God has revealed to me \textbf{regarding the foundation principles of our faith}.}[SpTB02 52.3; 1904][https://egwwritings.org/read?panels=p417.267]


\egw{في الوقت الذي نُشر فيه كتاب ‘ذا ليفينغ تمبل’، مرت أمامي في رؤيا ليلية، تمثيلات تشير إلى أن \textbf{خطرًا كان يقترب}، وأنه يجب أن أستعد له بكتابة الأشياء التي كشفها الله لي \textbf{بخصوص المبادئ الأساسية لإيماننا}.}[SpTB02 52.3; 1904][https://egwwritings.org/read?panels=p417.267]


\egw{\textbf{The enemy of souls has sought to bring in the supposition that a great reformation was to take place among Seventh-day Adventists, and that this reformation would consist in \underline{giving up the doctrines which stand as the pillars of our faith,} and engaging in a process of reorganization}. Were this reformation to take place, what would result? \textbf{The principles of truth} that God in His wisdom has given to the remnant church, \textbf{would be discarded}. Our religion would be changed. \textbf{The fundamental principles} that have sustained the work for the last fifty years \textbf{would be accounted as error}. A new organization would be established. Books of a new order would be written. A system of intellectual philosophy would be introduced.}[SpTB02 54.3; 1904][https://egwwritings.org/read?panels=p417.277]


\egw{\textbf{لقد سعى عدو النفوس إلى إدخال الافتراض بأن إصلاحًا عظيمًا كان سيحدث بين الأدفنتست السبتيين، وأن هذا الإصلاح سيتمثل في \underline{التخلي عن العقائد التي تقف كأعمدة إيماننا،} والانخراط في عملية إعادة تنظيم}. لو حدث هذا الإصلاح، ماذا سيكون النتيجة؟ \textbf{مبادئ الحق} التي أعطاها الله في حكمته للكنيسة البقية، \textbf{ستُرفض}. ستتغير ديانتنا. \textbf{المبادئ الجوهرية} التي دعمت العمل خلال الخمسين سنة الماضية \textbf{ستُعتبر خطأ}. ستُؤسس منظمة جديدة. ستُكتب كتب من نوع جديد. سيُقدم نظام من الفلسفة الفكرية.}[SpTB02 54.3; 1904][https://egwwritings.org/read?panels=p417.277]


Dr. Kellogg’s theories on the \emcap{personality of God}, if accepted, would ignite a reformation within the Seventh-day Adventist Church. Based on intellectual philosophy, they would cause us to renounce some of the doctrines that stand as the pillars of our faith, condemning the \emcap{Fundamental Principles} as error. Could it be that by adhering to the Trinity doctrine we entered into a new organization?


نظريات الدكتور كيلوغ حول \emcap{شخصانية الله}، إذا قُبلت، ستشعل إصلاحًا داخل كنيسة الأدفنتست السبتيين. استنادًا إلى الفلسفة الفكرية، ستجعلنا ننبذ بعض العقائد التي تقف كأعمدة إيماننا، مدينة \emcap{المبادئ الأساسية} كخطأ. هل يمكن أن نكون بالتمسك بعقيدة الثالوث قد دخلنا في منظمة جديدة؟


\egw{Shortly before I sent out the testimonies \textbf{regarding the efforts of the enemy to undermine the foundation of our faith through the dissemination of seductive theories}, I had read an incident about a ship in a fog meeting an iceberg…}[SpTB02 55.3; 1904][https://egwwritings.org/read?panels=p417.282]


\egw{قبل وقت قصير من إرسالي للشهادات \textbf{المتعلقة بجهود العدو لتقويض أساس إيماننا من خلال نشر نظريات مغرية}، كنت قد قرأت حادثة عن سفينة في الضباب تصطدم بجبل جليدي...}[SpTB02 55.3; 1904][https://egwwritings.org/read?panels=p417.282]


\egw{Messages of every order and kind have been \textbf{urged upon Seventh-day Adventists, to take the place of the truth which, \underline{point by point}, has been sought out by prayerful study, and testified to by the miracle-working power of the Lord}. \textbf{But the way-marks which have made us what we are, are to be preserved, and they will be preserved}, as God has signified through His word and the testimony of His Spirit. \textbf{He calls upon us to hold firmly}, with the grip of faith, \textbf{to \underline{the fundamental principles} that are based upon \underline{unquestionable authority}}.}[SpTB02 59.1; 1904][https://egwwritings.org/read?panels=p417.299]


\egw{لقد تم \textbf{فرض رسائل من كل نوع وصنف على الأدفنتست السبتيين، لتحل محل الحق الذي، \underline{نقطة بنقطة}، تم البحث عنه بدراسة صلاتية، وشُهد له بقوة الله العاملة للمعجزات}. \textbf{لكن المعالم التي جعلتنا ما نحن عليه، يجب أن تُحفظ، وستُحفظ}، كما أشار الله من خلال كلمته وشهادة روحه. \textbf{إنه يدعونا للتمسك بثبات}، بقبضة الإيمان، \textbf{بـ \underline{المبادئ الجوهرية} التي تستند إلى \underline{سلطة لا تقبل الشك}}.}[SpTB02 59.1; 1904][https://egwwritings.org/read?panels=p417.299]


The \emcap{personality of God} was the pillar of our faith\footnote{\href{https://egwwritings.org/?ref=en_Ms62-1905.14}{EGW, Ms62-1905.14; 1905}}. The \emcap{personality of God} was expressed in the first point of the \emcap{Fundamental Principles}. Could it be that by adhering to the Trinity doctrine we have torn down this particular pillar of our faith? Is it possible that by accepting the Trinity doctrine we were led astray from this foundation principle—the \emcap{personality of God}?


كانت \emcap{شخصانية الله} عمود إيماننا\footnote{\href{https://egwwritings.org/?ref=en_Ms62-1905.14}{EGW, Ms62-1905.14; 1905}}. تم التعبير عن \emcap{شخصانية الله} في النقطة الأولى من \emcap{المبادئ الأساسية}. هل يمكن أن نكون بالتمسك بعقيدة الثالوث قد هدمنا هذا العمود المحدد من إيماننا؟ هل من الممكن أننا بقبول عقيدة الثالوث قد ضللنا عن هذا المبدأ الأساسي—\emcap{شخصانية الله}؟


\subsection*{Does the Trinity do away with the personality of God?}


\subsection*{هل يلغي الثالوث شخصانية الله؟}


\egw{\textbf{It \normaltext{[The Living Temple]} introduces that which is naught but \underline{speculation} in regard to \underline{the personality of God} and where His presence is.}}[SpTB02 51.3; 1904][https://egwwritings.org/read?panels=p417.262]


\egw{\textbf{إنه \normaltext{[ذا ليفينغ تمبل]} يقدم ما هو إلا \underline{تكهنات} فيما يتعلق \underline{بشخصانية الله} وأين يوجد حضوره.}}[SpTB02 51.3; 1904][https://egwwritings.org/read?panels=p417.262]


\egw{\textbf{The spiritualistic theories \underline{regarding the personality of God}, followed to their logical conclusion, sweep away the whole Christian economy.}}[SpTB02 54.1; 1904][https://egwwritings.org/read?panels=p417.275]


\egw{\textbf{النظريات الروحانية \underline{المتعلقة بشخصانية الله}، إذا اتبعت إلى نتائجها المنطقية، تجرف النظام المسيحي بأكمله.}}[SpTB02 54.1; 1904][https://egwwritings.org/read?panels=p417.275]


\egw{‘Living Temple’ contains the alpha of these theories. I knew that the omega would follow in a little while; and I trembled for our people. I knew that \textbf{I must warn our brethren and sisters not to enter into controversy over \underline{the presence} and \underline{personality of God}. The statements made in ‘Living Temple’ \underline{in regard to this point are incorrect}. The scripture used to substantiate the doctrine there set forth, is scripture misapplied}.}[SpTB02 53.2; 1904][https://egwwritings.org/read?panels=p417.271]


\egw{يحتوي ‘ذا ليفينغ تمبل’ على ألفا هذه النظريات. كنت أعلم أن الأوميغا ستتبع بعد قليل؛ وارتعدت لأجل شعبنا. كنت أعلم أنه \textbf{يجب أن أحذر إخوتنا وأخواتنا من الدخول في جدال حول \underline{حضور} و\underline{شخصانية الله}. البيانات الواردة في ‘ذا ليفينغ تمبل’ \underline{فيما يتعلق بهذه النقطة غير صحيحة}. الكتاب المقدس المستخدم لدعم العقيدة المطروحة هناك، هو كتاب مقدس مطبق بشكل خاطئ}.}[SpTB02 53.2; 1904][https://egwwritings.org/read?panels=p417.271]


The theories Kellogg presented in the Living Temple are speculative in regard to the \emcap{personality of God} and where His presence is. These theories deal with the question of the quality or state of God being a person\footnote{The Merriam-Webster definition of ‘\textit{personality}’ - “\textit{the quality or state of being a person}”}. God has given us definite light regarding this issue in our \emcap{Fundamental Principles}. Could it be that the Trinity doctrine is casting doubt on this definite light regarding the \emcap{personality of God}?


تعتبر النظريات التي قدمها كيلوغ في ذا ليفينغ تمبل تخمينية فيما يتعلق بـ\emcap{شخصانية الله} وأين يوجد حضوره. تتعامل هذه النظريات مع مسألة الصفة أو الحالة التي يكون بها الله شخصًا\footnote{تعريف قاموس ميريام-ويبستر لـ’\textit{شخصانية}’ - “\textit{الصفة أو الحالة التي يكون بها الكائن شخصًا.}”} لقد أعطانا الله نورًا محددًا بخصوص هذه المسألة في \emcap{المبادئ الجوهرية} الخاصة بنا. هل يمكن أن تكون عقيدة الثالوث تلقي بظلال من الشك على هذا النور المحدد المتعلق بـ\emcap{شخصانية الله}؟


\subsection*{Is the Trinity doctrine presented as if Mrs. White supported it?}


\subsection*{هل تُقدَّم عقيدة الثالوث كما لو أن السيدة وايت كانت تدعمها؟}


\egw{In the controversy that arose among our brethren \textbf{regarding the teachings of this book,} those in favor of giving it a wide circulation \textbf{declared: ‘It contains the very sentiments that Sister White has been teaching.’ This assertion struck right to my heart. I felt heart-broken; for I knew that this representation of the matter was not true}.}[SpTB02 53.1; 1904][https://egwwritings.org/read?panels=p417.270]


\egw{في الصراع الذي نشأ بين إخوتنا \textbf{بخصوص تعاليم هذا الكتاب،} أعلن المؤيدون لإعطائه انتشارًا واسعًا: ‘إنه يحتوي على نفس الآراء التي كانت الأخت وايت تُعلِّمها.’ هذا التصريح ضرب قلبي مباشرة. شعرت بانكسار قلبي؛ لأنني كنت أعلم أن هذا التمثيل للأمر لم يكن صحيحًا.}[SpTB02 53.1; 1904][https://egwwritings.org/read?panels=p417.270]


\egw{\textbf{I am compelled to speak in denial of the claim that the teachings of ‘Living Temple’ can be sustained by statements from my writings}. There may be in this book expressions and sentiments that are in harmony with my writings. And there may be in my writings many statements which, taken from their connection, and interpreted according to the mind of the writer of ‘Living Temple,’ would seem to be in harmony with the teachings of this book. This may give apparent support to the assertion that the sentiments in ‘Living Temple’ are in harmony with my writings. \textbf{But God forbid that this sentiment should prevail}.}[SpTB02 53.3; 1904][https://egwwritings.org/read?panels=p417.272]


\egw{\textbf{أنا مضطرة للتحدث لإنكار الادعاء بأن تعاليم ‘ذا ليفينغ تمبل’ يمكن دعمها بتصريحات من كتاباتي}. قد تكون في هذا الكتاب تعبيرات وآراء متناغمة مع كتاباتي. وقد تكون في كتاباتي العديد من التصريحات التي، إذا أُخذت من سياقها، وفُسرت وفقًا لعقل كاتب ‘ذا ليفينغ تمبل’، قد تبدو متناغمة مع تعاليم هذا الكتاب. هذا قد يعطي دعمًا ظاهريًا للتأكيد بأن الآراء في ‘ذا ليفينغ تمبل’ متناغمة مع كتاباتي. \textbf{لكن حاشا لله أن يسود هذا الرأي}.}[SpTB02 53.3; 1904][https://egwwritings.org/read?panels=p417.272]


At this point, we have many unanswered questions. But, as we continue to study the first point of the \emcap{Fundamental Principles}, we will find answers to all of these questions. So far, in light of the \emcap{Fundamental Principles}, belief in the Trinity doctrine—as a Seventh-day Adventist—becomes very questionable. In order to defend the Trinity doctrine, the authority of the \emcap{Fundamental Principles} must be compromised. In what follows, we will briefly study their authority, context in Adventist history, and God’s purpose in giving them. We will also look at the true authorship of the \emcap{Fundamental Principles} and their role in present days.


في هذه النقطة، لدينا العديد من الأسئلة التي لم تتم الإجابة عليها. لكن، بينما نواصل دراسة النقطة الأولى من \emcap{المبادئ الجوهرية}، سنجد إجابات لكل هذه الأسئلة. حتى الآن، في ضوء \emcap{المبادئ الجوهرية}، يصبح الإيمان بعقيدة الثالوث—كأدفنتست سبتي—مشكوكًا فيه للغاية. من أجل الدفاع عن عقيدة الثالوث، يجب المساس بسلطة \emcap{المبادئ الجوهرية}. فيما يلي، سندرس بإيجاز سلطتها، وسياقها في تاريخ الأدفنتست، وغرض الله في إعطائها. سننظر أيضًا في المؤلف الحقيقي لـ\emcap{المبادئ الجوهرية} ودورها في الأيام الحاضرة.


% Examining Test

\begin{titledpoem}
    
    \stanza{
        Sister White's vision stands against the tide, \\
        As Trinity's doctrine she firmly denied. \\
        Her words a warning, clear and bright, \\
        Against teachings that dimmed revealed light.
    }

    \stanza{
        The pillars of faith, established with care, \\
        Now face a challenge, a doctrine to beware. \\
        For what was built through prayer and revelation, \\
        Faces change through doctrinal innovation.
    }

    \stanza{
        The personality of God, once clearly known, \\
        Now wrapped in theories not heaven's own. \\
        Past experiences of God's people at stake, \\
        When foundations new doctrines would break.
    }

    \stanza{
        The waymarks that made us what we are, \\
        Should guide us still, like a guiding star. \\
        Hold firmly to principles with faith's strong grip, \\
        Lest in confusion's fog we lose our ship.
    }
\end{titledpoem}