\qrchapter{https://forgottenpillar.com/rsc/en-fp-chapter22}{The bottom of the issue}


\qrchapter{https://forgottenpillar.com/rsc/en-fp-chapter22}{قاع المسألة}


Today, when we compare our current Fundamental Beliefs with the previous \emcap{Fundamental Principles} we see the change in the foundation of Seventh-day Adventist faith. This change has occurred in the understanding of God’s person, or the \emcap{personality of God}. Particular to the \emcap{personality of God}, Sister White wrote that the track of truth lies close beside the track of error:


اليوم، عندما نقارن معتقداتنا الأساسية الحالية مع \emcap{المبادئ الأساسية} السابقة، نرى التغيير في أساس إيمان الأدفنتست السبتيين. حدث هذا التغيير في فهم شخص الله، أو \emcap{شخصانية الله}. وفيما يتعلق بـ \emcap{شخصانية الله}، كتبت الأخت وايت أن مسار الحق يقع بالقرب من مسار الخطأ:


\egw{\textbf{The track of truth lies close beside the track of error}, and both tracks may \textbf{seem }to be one to minds which are not worked by the Holy Spirit, and which, therefore, \textbf{are not quick to discern the difference between truth and error}.}[SpTB02 52.2; 1904][https://egwwritings.org/read?panels=p417.266]


\egw{\textbf{إن مسار الحق يقع بالقرب من مسار الخطأ}، وكلا المسارين قد \textbf{يبدوان} واحدًا للعقول التي لا يعمل فيها الروح القدس، والتي، بالتالي، \textbf{ليست سريعة في تمييز الفرق بين الحق والخطأ}.}[SpTB02 52.2; 1904][https://egwwritings.org/read?panels=p417.266]


We ask ourselves, how can we draw a clear line between these two tracks? In order to do that we need to get to the bottom of the issue. We need to find a distinguishing principle that separates these two tracks.


نسأل أنفسنا، كيف يمكننا رسم خط واضح بين هذين المسارين؟ للقيام بذلك، نحتاج إلى الوصول إلى قاع المسألة. نحتاج إلى إيجاد مبدأ مميز يفصل بين هذين المسارين.


By studying our current Trinitarian belief and the works of our pioneers regarding the \emcap{personality of God}, we have found one characterizing principle that distinguishes the truth on the \emcap{personality of God}, as held by our pioneers, from our current Trinitarian belief. Both sides claim the Bible to be their ultimate authority, yet differences can be discerned by the interpretation of the Bible. In the following, we are talking about understanding and interpreting Scripture concerning God’s person. Understanding God’s person can be presented in two distinct, mutually exclusive understandings, which clearly draw a line between the two opposing camps.


من خلال دراسة معتقدنا الثالوثي الحالي وأعمال روادنا فيما يتعلق بـ \emcap{شخصانية الله}، وجدنا مبدأً مميزًا يميز الحق حول \emcap{شخصانية الله}، كما تمسك به روادنا، عن معتقدنا الثالوثي الحالي. يدعي كلا الجانبين أن الكتاب المقدس هو سلطتهم النهائية، ومع ذلك يمكن تمييز الاختلافات من خلال تفسير الكتاب المقدس. فيما يلي، نتحدث عن فهم وتفسير الكتاب المقدس فيما يتعلق بشخص الله. يمكن تقديم فهم شخص الله في فهمين متميزين ومتعارضين، يرسمان بوضوح خطًا بين المعسكرين المتعارضين.


One, more popular, view is that God presented Himself in a language that is familiar to us in order to explain only the concepts of salvation. So, God presented Himself in words such as ‘\textit{father}’, ‘\textit{son}’, and ‘\textit{spirit}’, to describe the relationships between these concepts. This makes none of these words interpretable by their obvious meaning; rather, they hold symbolic or metaphoric value. The principle behind this reasoning is: \textbf{God adjusted Himself to man}.


وجهة النظر الأولى، والأكثر شيوعًا، هي أن الله قدم نفسه بلغة مألوفة لنا لشرح مفاهيم الخلاص فقط. لذلك، قدم الله نفسه بكلمات مثل ‘\textit{أب}’ و’\textit{ابن}’ و’\textit{روح}’، لوصف العلاقات بين هذه المفاهيم. هذا يجعل أيًا من هذه الكلمات غير قابلة للتفسير بمعناها الواضح؛ بل تحمل قيمة رمزية أو مجازية. المبدأ وراء هذا المنطق هو: \textbf{الله كيّف نفسه مع الإنسان}.


The other, opposing, view is that \textbf{God adjusted man to Himself}; \textit{He created man in His own image}. Therefore, words like ‘\textit{father}’, ‘\textit{son}’, and ‘\textit{spirit}’, as they address God, imply their obvious meaning. This is the fundamental difference.


وجهة النظر الأخرى، المعارضة، هي أن \textbf{الله كيّف الإنسان مع نفسه}؛ \textit{خلق الإنسان على صورته}. لذلك، فإن كلمات مثل ‘\textit{أب}’ و’\textit{ابن}’ و’\textit{روح}’، عندما تشير إلى الله، تعني معناها الواضح. هذا هو الاختلاف الأساسي.


When we come to understand biblical terms like ‘\textit{person}’, ‘\textit{father}’, ‘\textit{son}’ and ‘\textit{spirit}’, we must choose which view we support and apply it accordingly. Either these terms are understood in their obvious meaning, or symbolically or metaphorically. There is no middle ground between these two; we must choose one. The following quotation should settle any dilemma.


عندما نأتي لفهم المصطلحات الكتابية مثل ‘\textit{شخص}’ و’\textit{أب}’ و’\textit{ابن}’ و’\textit{روح}’، يجب أن نختار أي وجهة نظر ندعمها ونطبقها وفقًا لذلك. إما أن تُفهم هذه المصطلحات بمعناها الواضح، أو بشكل رمزي أو مجازي. لا يوجد حل وسط بين هذين الاثنين؛ يجب أن نختار واحدًا. الاقتباس التالي يجب أن يحسم أي معضلة.


\egw{\textbf{\underline{The language of the Bible should be explained according to its obvious meaning, unless a symbol or figure is employed}}.}[GC 598.3; 1888][https://egwwritings.org/read?panels=p133.2717]


\egw{\textbf{\underline{ينبغي شرح لغة الكتاب المقدس وفقًا لمعناها الواضح، ما لم يتم استخدام رمز أو صورة}}.}[GC 598.3; 1888][https://egwwritings.org/read?panels=p133.2717]


We believe that it is impossible for the Bible to be its own interpreter and not explain its own symbols. If the Bible applies the word ‘father’ to God, but never explains this term, then it should be accepted in its obvious meaning. The same applies to the words ‘son’ and ‘spirit’. Man is created in the image of God. God adjusted man to Himself. The obvious meaning is derived from the experience of man. We understand the obvious meaning of the word ‘father’ through regular, human fathership. But our fathership is the image of our God Who is the Father to His Son. Paul testified:


نعتقد أنه من المستحيل أن يكون الكتاب المقدس مفسرًا لنفسه وألا يشرح رموزه الخاصة. إذا كان الكتاب المقدس يطبق كلمة “أب” على الله، ولكنه لا يشرح هذا المصطلح أبدًا، فينبغي قبوله بمعناه الواضح. وينطبق الشيء نفسه على كلمتي “ابن” و”روح”. الإنسان مخلوق على صورة الله. الله كيّف الإنسان مع نفسه. المعنى الواضح مستمد من تجربة الإنسان. نفهم المعنى الواضح لكلمة “أب” من خلال الأبوة البشرية العادية. لكن أبوتنا هي صورة إلهنا الذي هو الأب لابنه. شهد بولس:


\bible{For this cause I bow my knees unto \textbf{the Father of our Lord Jesus Christ, Of whom the whole \underline{family} in heaven and earth is \underline{named}}}[Ephesians 3:14-15].


\bible{لِهَذَا السَّبَبِ أَحْنِي رُكْبَتَيَّ لَدَى \textbf{أَبِي رَبِّنَا يَسُوعَ الْمَسِيحِ، الَّذِي مِنْهُ تُسَمَّى كُلُّ \underline{عَشِيرَةٍ} فِي السَّمَاوَاتِ وَعَلَى الأَرْضِ}}[أفسس ٣: ١٤-١٥].


In Greek, the word ‘\textit{family}’ is the word ‘\textit{patria}’, derived from the word ‘\textit{pater}’, which means ‘\textit{father}’. Some translations even render this verse with \bible{Of whom all \textbf{paternity} in heaven and earth is named} (DRB), which is a more literal translation. The Father of our Lord Jesus Christ is truly the father to His Son, just as truly as we are fathers to our children on Earth. Our paternity on Earth is named according to Paternity in Heaven, where God is the Father of our Lord Jesus Christ. Our earthly paternity is an image of Heavenly Paternity, where God is the Father to His Son. This supports the obvious meaning that Jesus is truly the Son of our God.


في اليونانية، كلمة ‘\textit{عشيرة}’ هي كلمة ‘\textit{باتريا}’، مشتقة من كلمة ‘\textit{باتير}’، والتي تعني ‘\textit{أب}’. بعض الترجمات تقدم هذه الآية بصيغة \bible{الَّذِي مِنْهُ تُسَمَّى كُلُّ \textbf{أُبُوَّةٍ} فِي السَّمَاوَاتِ وَعَلَى الأَرْضِ} (DRB)، وهي ترجمة أكثر حرفية. إن أبا ربنا يسوع المسيح هو حقًا أب لابنه، تمامًا كما نحن آباء لأطفالنا على الأرض. أبوتنا على الأرض مسماة وفقًا للأبوة في السماء، حيث الله هو أبو ربنا يسوع المسيح. أبوتنا الأرضية هي صورة للأبوة السماوية، حيث الله هو أب لابنه. هذا يدعم المعنى الواضح بأن يسوع هو حقًا ابن إلهنا.


The same underlying principle applies to the understanding behind the word ‘\textit{spirit}’ and the word ‘\textit{being}’. God adjusted man to Himself; He created man in His own image. Man is a being, possessing body and spirit, just like God—and in saying this, we are not saying that man and God possess the same nature. God formed man from the dust of the ground. His physical nature is confined to the elements found on the earth. We do not pry into the nature of God. That will forever remain a mystery to us; it is not revealed unto us. But what is revealed to us is that He has a form, and the form of a man is an image of the form of God. The Bible plainly approves this understanding when describing God sitting upon His throne:


ينطبق نفس المبدأ الأساسي على فهم كلمة ‘\textit{روح}’ وكلمة ‘\textit{كائن}’. لقد كيّف الله الإنسان ليتوافق معه؛ خلقه على صورته. الإنسان كائن، يمتلك جسدًا وروحًا، تمامًا مثل الله—وبقولنا هذا، لا نقول إن الإنسان والله يمتلكان نفس الطبيعة. شكّل الله الإنسان من تراب الأرض. طبيعته المادية محصورة في العناصر الموجودة على الأرض. نحن لا نتطفل على طبيعة الله. ستظل تلك سرًا لنا إلى الأبد؛ فهي غير مكشوفة لنا. لكن ما كُشف لنا هو أن له هيئة، وهيئة الإنسان هي صورة لهيئة الله. يؤيد الكتاب المقدس بوضوح هذا الفهم عند وصف الله جالسًا على عرشه:


\bible{\textbf{upon the likeness of the throne was the likeness as \underline{the appearance of a man}} above upon it}[Ezekiel 1:26].


\bible{\textbf{وَعَلَى شِبْهِ الْعَرْشِ شِبْهٌ \underline{كَمَنْظَرِ إِنْسَانٍ}} عَلَيْهِ مِنْ فَوْقُ}[حزقيال ١: ٢٦].


The obvious meaning of the word ‘\textit{spirit}’, applied to the Spirit of God, is derived from the understanding of “\textit{the spirit of man}”. God adjusted man to Himself; He created man in His own image. Just as man possesses a spirit, God possesses a Spirit. The spirit of man has the nature of man, and the spirit of God has the nature of God. With respect to their nature, they are not the same, but respective of their relation to their inner being, they are the same; the Bible puts them on the same level. \bible{\textbf{The \underline{Spirit} itself beareth witness with \underline{our spirit}}, that we are the children of God:}[Romans 8:16]; \bible{For what \textbf{man knoweth the things of a man}, save the \textbf{\underline{spirit of man} which is in him}? \textbf{\underline{even so}} the things of \textbf{God knoweth} no man, \textbf{but \underline{the Spirit of God}}.}[1 Corinthians 2:11].


المعنى الواضح لكلمة ‘\textit{روح}’، المطبقة على روح الله، مشتق من فهم “\textit{روح الإنسان}”. كيّف الله الإنسان ليتوافق معه؛ خلقه على صورته. كما يمتلك الإنسان روحًا، يمتلك الله روحًا. روح الإنسان لها طبيعة الإنسان، وروح الله لها طبيعة الله. فيما يتعلق بطبيعتهما، هما ليسا متماثلين، ولكن فيما يتعلق بعلاقتهما بكيانهما الداخلي، هما متماثلان؛ يضعهما الكتاب المقدس على نفس المستوى. \bible{\textbf{اَلرُّوحُ نَفْسُهُ أَيْضًا يَشْهَدُ لأَرْوَاحِنَا} أَنَّنَا أَوْلاَدُ اللهِ}[رومية ٨: ١٦]؛ \bible{لأَنْ \textbf{مَنْ مِنَ النَّاسِ يَعْرِفُ أُمُورَ الإِنْسَانِ} إِلاَّ \textbf{\underline{رُوحُ الإِنْسَانِ} الَّذِي فِيهِ}؟ \textbf{\underline{هَكَذَا أَيْضًا}} أُمُورُ \textbf{اللهِ لاَ يَعْرِفُهَا} أَحَدٌ \textbf{إِلاَّ \underline{رُوحُ اللهِ}}.}[١ كورنثوس ٢: ١١].


In terms of family relationships and the quality or state of being a person, man and God are alike, because God created man in His own image. God adjusted man unto Himself. But in their nature, God and man are not alike. God is divine and man is earthly.


من حيث العلاقات العائلية والصفة أو الحالة التي يكون بها الكائن شخصًا، الإنسان والله متشابهان، لأن الله خلق الإنسان على صورته. كيّف الله الإنسان ليتوافق معه. لكن في طبيعتهما، الله والإنسان ليسا متشابهين. الله إلهي والإنسان أرضي.


The Trinity doctrine adheres to the understanding that God adjusted Himself to man, and that God merely used the terms ‘\textit{father}’, ‘\textit{son}’ and ‘\textit{spirit}’ so that we might understand Him better. This idea underpins and drives the trinitarian paradigm. In what follows, we will not extensively examine our Trinitarian literature, but will support our claim by a few official statements from the Seventh-day Adventist Church.


تلتزم عقيدة الثالوث بفهم أن الله كيّف نفسه للإنسان، وأن الله استخدم فقط مصطلحات ‘\textit{أب}’ و’\textit{ابن}’ و’\textit{روح}’ حتى نفهمه بشكل أفضل. هذه الفكرة تدعم وتقود النموذج الثالوثي. فيما يلي، لن نفحص بشكل مكثف أدبياتنا الثالوثية، ولكن سندعم ادعاءنا ببعض البيانات الرسمية من كنيسة الأدفنتست السبتيين.


The first statement comes from the Biblical Research Institute, the official institution of the General Conference, which promotes the teachings and doctrines of the Seventh-day Adventist Church. They openly negate the parental relationship between the Father and His Son, in favour of a metaphorical understanding.


البيان الأول يأتي من معهد البحوث الكتابية، المؤسسة الرسمية للمجمع العام، التي تروج لتعاليم وعقائد كنيسة الأدفنتست السبتيين. إنهم ينفون علنًا العلاقة الأبوية بين الآب وابنه، لصالح فهم مجازي.


\others{The father-son image \textbf{cannot be literally applied to the divine Father-Son relationship} within the Godhead. \textbf{The Son is not the natural, literal Son of the Father} ... \textbf{The term ‘Son’ is used metaphorically} when applied to the Godhead.}[Adventist Biblical Research Institute; also published in the official ‘Adventist World’ magazine][https://www.adventistbiblicalresearch.org/materials/a-question-of-sonship/]


\others{صورة الأب والابن \textbf{لا يمكن تطبيقها حرفيًا على علاقة الأب والابن الإلهية} داخل اللاهوت. \textbf{الابن ليس الابن الطبيعي، الحرفي للآب} ... \textbf{مصطلح ‘ابن’ يُستخدم مجازيًا} عند تطبيقه على اللاهوت.}[معهد البحوث الكتابية الأدفنتستي؛ نُشر أيضًا في مجلة ‘عالم الأدفنتست’ الرسمية][https://www.adventistbiblicalresearch.org/materials/a-question-of-sonship/]


Regarding the \emcap{personality of God}, in the context of the trinitarian paradigm, the Seventh-day Adventist church issued the following statements in a Sabbath school lesson:


فيما يتعلق بـ \emcap{شخصانية الله}، في سياق النموذج الثالوثي، أصدرت كنيسة الأدفنتست السبتيين البيانات التالية في درس مدرسة السبت:


\others{\textbf{The \underline{word persons} used in the title of today's lesson \underline{must be understood in a theological sense}}. \textbf{If we equate human personality with God, we would say that three persons means three individuals. But then we would have three Gods, or tritheism}. \textbf{But \underline{historic Christianity} has given to the word person, when used of God, \underline{a special meaning}}: a personal self-distinction, which gives distinctiveness in the Persons of the Godhead without destroying the concept of oneness. \textbf{This idea is not easy to grasp or to explain! \underline{It is part of the mystery of the Godhead}}.}[“Lesson 3.” Ssnet.org, 2025, \href{http://www.ssnet.org/qrtrly/eng/98d/less03.html}{www.ssnet.org/qrtrly/eng/98d/less03.html}. Accessed 3 Feb. 2025.]


\others{\textbf{يجب فهم \underline{كلمة أشخاص} المستخدمة في عنوان درس اليوم \underline{بمعنى لاهوتي}}. \textbf{إذا ساوينا الشخصانية البشرية بالله، فسنقول إن ثلاثة أشخاص تعني ثلاثة أفراد. لكن حينئذ سيكون لدينا ثلاثة آلهة، أو عقيدة الآلهة الثلاثة}. \textbf{لكن \underline{المسيحية التاريخية} أعطت لكلمة شخص، عند استخدامها مع الله، \underline{معنى خاصًا}}: تمييز ذاتي شخصي، يعطي تميزًا في أقانيم اللاهوت دون تدمير مفهوم الوحدانية. \textbf{هذه الفكرة ليست سهلة الفهم أو الشرح! \underline{إنها جزء من سر اللاهوت}}.}[“Lesson 3.” Ssnet.org, 2025, \href{http://www.ssnet.org/qrtrly/eng/98d/less03.html}{www.ssnet.org/qrtrly/eng/98d/less03.html}. Accessed 3 Feb. 2025.]


\others{These texts and others lead us to believe that \textbf{our wonderful God is \underline{three Persons in one},} a mind-boggling \textbf{mystery }but a truth we accept by faith because Scripture reveals it.}[Ibid.]


\others{تقودنا هذه النصوص وغيرها إلى الاعتقاد بأن \textbf{إلهنا الرائع هو \underline{ثلاثة أشخاص في واحد}،} وهو \textbf{سر} مذهل ولكنه حقيقة نقبلها بالإيمان لأن الكتاب المقدس يكشفها.}[Ibid.]


According to official statements presented in the Sabbath School Lesson, the word \textit{‘persons’},\textit{ }in regard to God, should not be equated with human personality, but should be applied in the theological sense. This is in sharp contrast to the vision Sister White had regarding the \emcap{personality of God}. \egwinline{‘I have often seen the lovely Jesus, that \textbf{He is a person}. I asked Him if \textbf{His Father was a person}, and \textbf{had \underline{a form} like Himself}. Said Jesus, ‘\textbf{I am the express image of My Father’s person!}’ [Hebrews 1:3.]}[Lt253-1903.12; 1903][https://egwwritings.org/read?panels=p9980.18] Her understanding of the quality or state of God being a person is that God is a person in an obvious way—He possesses a form. In the same way she recognized Jesus to be a person, Jesus testified that God is a person, having a form just as He has. Contrary to the obvious and literal view is a spiritual view. She continues to address the error of the spiritual view. \egwinline{\textbf{I have often seen that \underline{the spiritual view} took away all the glory of heaven, and that in many minds the throne of David and the lovely person of Jesus have been burned up in the fire of spiritualism}. I have seen that some who have been deceived and led into this error, will be brought out into the light of truth, \textbf{but it will be almost impossible for them to get entirely rid of the deceptive power of spiritualism. Such should make thorough work in confessing their errors, and leaving them forever}.}[Lt253-1903.13; 1903][https://egwwritings.org/read?panels=p9980.19] According to the Sabbath School Lesson, the obvious understanding of the term \textit{‘person’ }is incorrect because this would \others{\textbf{equate human personality with God}}, meaning that \others{\textbf{three persons means three individuals}}. Opposite to the obvious view is the theological view. For Sister White, the opposite is the spiritual view. This view takes \egwinline{away all the glory of heaven, and that in many minds the throne of David and the lovely person of Jesus have been burned up in the fire of spiritualism}. In the writings of our pioneers, previously examined, we recognize the truthfulness of her claim. The presented theological view of God’s person does away with the truth on the \emcap{personality of God} that Sister White received in a vision. The theological view is explained as one God, Who is a person, yet three persons, made up of three distinct Gods—God the Father, God the Son, and God the Holy Ghost. The Bible never explains God with such a quality or state of being a person. It is simply presumed by trinitarian believers and, because it is never explained, is deemed a mystery of God, but in fact—it is an error.


وفقًا للبيانات الرسمية المقدمة في درس مدرسة السبت، فإن كلمة \textit{‘أشخاص’},\textit{ }فيما يتعلق بالله، لا ينبغي أن تُساوى بالشخصانية البشرية، بل ينبغي تطبيقها بالمعنى اللاهوتي. وهذا يتناقض بشكل حاد مع الرؤيا التي رأتها الأخت وايت بخصوص \emcap{شخصانية الله}. \egwinline{‘لقد رأيت يسوع الحبيب مرارًا، وأن \textbf{يسوع هو شخص}. سألته إذا كان \textbf{أبوه شخصًا}، \textbf{وله \underline{هيئة} مثله}. قال يسوع، ‘\textbf{أنا صورة جوهره!}’ [عبرانيين 1:3.]}[Lt253-1903.12; 1903][https://egwwritings.org/read?panels=p9980.18] فهمها للصفة أو الحالة التي يكون بها الله شخصًا هو أن الله شخص بطريقة واضحة—لديه هيئة. وبنفس الطريقة التي اعترفت بها أن يسوع شخص، شهد يسوع أن الله شخص، له هيئة كما له هو. وعلى النقيض من الرؤية الواضحة والحرفية توجد النظرة الروحانية. وتواصل معالجة خطأ النظرة الروحانية. \egwinline{\textbf{لقد رأيت مرارًا أن \underline{النظرة الروحانية} أزالت كل مجد السماء، وأنه في أذهان كثيرين تم حرق عرش داود وشخص يسوع الجميل في نار الروحانية}. لقد رأيت أن بعض الذين خُدعوا وقادوا إلى هذا الخطأ، سيُخرجون إلى نور الحق، \textbf{لكن سيكون من المستحيل تقريبًا عليهم التخلص تمامًا من القوة المخادعة للروحانية. على هؤلاء أن يقوموا بعمل شامل في الاعتراف بأخطائهم، وتركها إلى الأبد}.}[Lt253-1903.13; 1903][https://egwwritings.org/read?panels=p9980.19] وفقًا لدرس مدرسة السبت، فإن الفهم الواضح لمصطلح \textit{‘شخص’ }غير صحيح لأن هذا \others{\textbf{يساوي الشخصانية البشرية بالله}}، مما يعني أن \others{\textbf{ثلاثة أشخاص تعني ثلاثة أفراد}}. وعكس النظرة الواضحة هي النظرة اللاهوتية. بالنسبة للأخت وايت، النقيض هو النظرة الروحانية. هذه النظرة \egwinline{تزيل كل مجد السماء، وفي أذهان كثيرين تم حرق عرش داود وشخص يسوع الجميل في نار الروحانية}. في كتابات روادنا، التي تمت دراستها سابقًا، ندرك صدق ادعائها. النظرة اللاهوتية المقدمة لشخص الله تلغي الحق عن \emcap{شخصانية الله} الذي تلقته الأخت وايت في رؤيا. النظرة اللاهوتية تُفسر على أنها إله واحد، وهو شخص، ومع ذلك ثلاثة أشخاص، مكونة من ثلاثة آلهة متميزين—الله الآب، والله الابن، والله الروح القدس. الكتاب المقدس لا يشرح أبدًا الله بمثل هذه الصفة أو حالة كونه شخصًا. إنه ببساطة افتراض من قبل المؤمنين بالثالوث، ولأنه لم يتم شرحه أبدًا، يُعتبر سرًا من أسرار الله، ولكنه في الواقع—خطأ.


When we draw the line between truth and error, we also need to draw the line between the things that are mystery and those that are revealed. Regarding the nature of God, silence is eloquence. Unfortunately, many who are advocating the Trinity doctrine fail to draw this line in the proper place. We protest that the \emcap{personality of God}, that is the quality or state of God being a person, is a mystery. Our pioneers understood it and they clearly explained it from the Bible. If they did not read and accept the Bible in its plain and simple language, they wouldn’t be able to explain the \emcap{personality of God}.


عندما نرسم الخط الفاصل بين الحق والخطأ، نحتاج أيضًا إلى رسم الخط بين الأشياء التي هي سر وتلك التي تم الكشف عنها. فيما يتعلق بطبيعة الله، الصمت هو البلاغة. لسوء الحظ، كثير من الذين يدافعون عن عقيدة الثالوث يفشلون في رسم هذا الخط في المكان المناسب. نحن نحتج على أن \emcap{شخصانية الله}، أي الصفة أو الحالة التي يكون بها الله شخصًا، هي سر. فهمها روادنا وشرحوها بوضوح من الكتاب المقدس. لو لم يقرأوا ويقبلوا الكتاب المقدس بلغته البسيطة والواضحة، لما تمكنوا من شرح \emcap{شخصانية الله}.


There are brethren who completely agree with the \emcap{personality of God} laid out in the \emcap{Fundamental Principles}. They agree that the terms ‘\textit{father}’, ‘\textit{son}’ and ‘\textit{spirit}’ should be interpreted by their obvious meaning, yet they continue to advocate the Trinity doctrine because they fail to correctly draw the line between what is being revealed by God and what is not. The argument goes something like this: yes, God is a personal, spiritual being; He does have a body of some sort, Christ is His only begotten Son, and the Holy Spirit is Their representative, but that all applies to our physical universe, which is cumbered by space and time; beyond space and time, God is Trinity.


هناك إخوة يتفقون تمامًا مع \emcap{شخصانية الله} الموضحة في \emcap{المبادئ الأساسية}. يتفقون على أن مصطلحات ‘\textit{الأب}’ و’\textit{الابن}’ و’\textit{الروح}’ يجب تفسيرها بمعناها الواضح، ومع ذلك يستمرون في الدفاع عن عقيدة الثالوث لأنهم يفشلون في رسم الخط بشكل صحيح بين ما تم الكشف عنه من قبل الله وما لم يتم. الحجة تسير على النحو التالي: نعم، الله كائن روحي شخصي؛ لديه نوع من الجسد، المسيح هو ابنه الوحيد، والروح القدس هو ممثلهما، لكن كل ذلك ينطبق على كوننا المادي، الذي يثقله المكان والزمان؛ وما وراء المكان والزمان، الله هو الثالوث.


Such a view fails to draw the line between what is revealed and what is a mystery. One consequence of such a conception of God is that it casts doubt on the things which are revealed unto us. To recognize that takes honesty because it is very enticing to conceptualize God beyond space and time, but it is, ultimately, unjustifiable because we are finite and bound to space and time. In his book, the Living Temple, Dr. Kellogg conceptualized God beyond \others{the bounds of space and time}. Dr. Kellogg objected to the conception of God depicted by the \emcap{Fundamental Principles}, because God, in His personality, was bound to His body and thus “\textit{circumscribed}” to one locality, say the temple, or the throne in Heaven\footnote{\href{https://archive.org/details/J.H.Kellogg.TheLivingTemple1903/page/n31/mode/2up}{John H. Kellogg, The Living Temple, p. 31}}. This was unprofitable for Dr. Kellogg, and he advocated that God is far beyond our comprehension as are the bounds of space and time.


مثل هذه النظرة تفشل في رسم الخط بين ما هو مكشوف وما هو سر. إحدى نتائج مثل هذا التصور لله هو أنه يلقي بالشك على الأشياء التي كُشفت لنا. إدراك ذلك يتطلب الصدق لأنه من المغري جدًا تصور الله خارج المكان والزمان، لكنه في النهاية غير مبرر لأننا محدودون ومقيدون بالمكان والزمان. في كتابه، ذا ليفينغ تمبل، تصور الدكتور كيلوغ الله خارج \others{حدود المكان والزمان}. اعترض الدكتور كيلوغ على تصور الله الموصوف في \emcap{المبادئ الأساسية}، لأن الله، في شخصانيته، كان مقيدًا بجسده وبالتالي “\textit{محصورًا}” في مكان واحد، مثل الهيكل، أو العرش في السماء\footnote{\href{https://archive.org/details/J.H.Kellogg.TheLivingTemple1903/page/n31/mode/2up}{John H. Kellogg, The Living Temple, p. 31}}. كان هذا غير مجدٍ للدكتور كيلوغ، ودافع عن أن الله أبعد بكثير من فهمنا كما هي حدود المكان والزمان.


\others{\textbf{\underline{Discussions respecting the form of God are utterly unprofitable}, and serve only to belittle our conceptions of him who is above all things}, \textbf{and hence not to be compared in form or size or glory or majesty with anything which man has ever seen or which it is within his power to conceive}. In the presence of questions like these, we have only to acknowledge our foolishness and incapacity, and bow our heads with awe and reverence \textbf{in the presence of a Personality, an Intelligent Being} to the existence of which all nature bears definite and positive testimony, \textbf{but which is as far beyond our comprehension \underline{as are the bounds of space and time}}.}[Ibid, p. 33][https://archive.org/details/J.H.Kellogg.TheLivingTemple1903/page/n33/mode/2up]


\others{\textbf{\underline{المناقشات المتعلقة بشكل الله عديمة الفائدة تمامًا}، وتخدم فقط لتقليل تصوراتنا عنه الذي هو فوق كل الأشياء}، \textbf{وبالتالي لا يمكن مقارنته في الشكل أو الحجم أو المجد أو العظمة بأي شيء رآه الإنسان أو ضمن قدرته على تصوره}. في حضرة أسئلة مثل هذه، ليس لدينا سوى الاعتراف بحماقتنا وعجزنا، وخفض رؤوسنا بخشية وتوقير \textbf{في حضرة شخصانية، كائن ذكي} لوجوده تشهد كل الطبيعة شهادة محددة وإيجابية، \textbf{لكنه أبعد من فهمنا \underline{كما هي حدود المكان والزمان}}.}[Ibid, p. 33][https://archive.org/details/J.H.Kellogg.TheLivingTemple1903/page/n33/mode/2up]


Dr. Kellogg was reproved for his conceptions of God. His conception of God was God beyond the bounds of space and time. This conception is problematic because it is beyond the bounds of the Scriptures; it is pure conjecture, which casts doubt on the revelation of the Scripture. If the Scriptures testify that God is a definite, tangible being, being present in one place more than another, then any discussions regarding God being beyond space are utterly unprofitable. Such discussions tend to lead toward skepticism on the very conceptions of God that the Scriptures plainly testify of. As we can recall, this was the main problem with Dr. Kellogg, and Sister White gave us many warnings regarding this issue.


تم توبيخ الدكتور كيلوغ على تصوراته عن الله. كان تصوره لله هو الله خارج حدود المكان والزمان. هذا التصور إشكالي لأنه خارج حدود الكتاب المقدس؛ إنه تخمين محض، يلقي بالشك على وحي الكتاب المقدس. إذا شهد الكتاب المقدس أن الله كائن محدد وملموس، موجود في مكان واحد أكثر من آخر، فإن أي مناقشات تتعلق بكون الله خارج المكان هي عديمة الفائدة تمامًا. مثل هذه المناقشات تميل إلى أن تؤدي إلى الشك في تصورات الله نفسها التي يشهد بها الكتاب المقدس بوضوح. كما يمكننا أن نتذكر، كانت هذه هي المشكلة الرئيسية مع الدكتور كيلوغ، وقدمت لنا الأخت وايت العديد من التحذيرات بخصوص هذه المسألة.


\egw{‘The secret things belong unto the Lord our God: but those things which are revealed belong unto us and to our children forever.’ Deuteronomy 29:29. \textbf{The revelation of Himself that God has given in His word is for our study}. \textbf{This we may seek to understand}. \textbf{\underline{But beyond this we are not to penetrate}}. \textbf{The highest intellect may tax itself until it is wearied out in \underline{conjectures}\footnote{\href{https://www.merriam-webster.com/dictionary/conjectures}{Merriam Webster Dictionary} - ‘\textit{conjecture}’ - “\textit{a: inference formed without proof or sufficient evidence; b: a conclusion deduced by surmise or guesswork}”} \underline{regarding the nature of God}, but the effort will be fruitless}. \textbf{This problem has not been given us to solve. No human mind can comprehend God.} \textbf{None are to indulge in speculation regarding His nature. Here silence is eloquence. The Omniscient One is above discussion}.}[MH 429.3; 1905][https://egwwritings.org/read?panels=p135.2227]


\egw{‘السرائر للرب إلهنا والمعلنات لنا ولبنينا إلى الأبد.’ تثنية 29:29. \textbf{إن إعلان الله عن نفسه الذي أعطاه في كلمته هو لدراستنا}. \textbf{هذا يمكننا أن نسعى لفهمه}. \textbf{\underline{لكن ما وراء ذلك لا يجب أن نخترقه}}. \textbf{قد يُجهد أعلى ذكاء نفسه حتى يتعب في \underline{التخمينات}\footnote{\href{https://www.merriam-webster.com/dictionary/conjectures}{Merriam Webster Dictionary} - ‘\textit{conjecture}’ - “\textit{a: استنتاج تم تكوينه بدون دليل أو أدلة كافية؛ b: استنتاج مستنبط بالتخمين أو الحدس}”} \underline{بشأن طبيعة الله}، لكن الجهد سيكون عديم الجدوى}. \textbf{هذه المشكلة لم تُعط لنا لحلها. لا يمكن لأي عقل بشري أن يفهم الله.} \textbf{لا ينبغي لأحد أن ينغمس في التكهنات بشأن طبيعته. هنا الصمت هو البلاغة. العليم فوق النقاش}.}[MH 429.3; 1905][https://egwwritings.org/read?panels=p135.2227]


\egw{I say, and have ever said, \textbf{that I will not engage in controversy with any one in regard to \underline{the nature} and personality of God}. \textbf{Let those who try to describe God know that on such a subject silence is eloquence}. \textbf{\underline{Let the Scriptures be read in simple faith, and let each one form his conceptions of God from His inspired Word}}.}[Lt214-1903.9; 1903][https://egwwritings.org/read?panels=p10700.15]


\egw{أقول، وقلت دائمًا، \textbf{أنني لن أدخل في جدال مع أي شخص فيما يتعلق \underline{بطبيعة} وشخصانية الله}. \textbf{دع أولئك الذين يحاولون وصف الله يعلمون أنه في مثل هذا الموضوع الصمت هو البلاغة}. \textbf{\underline{ليُقرأ الكتاب المقدس بإيمان بسيط، وليشكل كل واحد تصوراته عن الله من كلمته الموحى بها}}.}[Lt214-1903.9; 1903][https://egwwritings.org/read?panels=p10700.15]


\egw{No human mind can comprehend God. No man hath seen Him at any time. We are as ignorant of God as little children. But as little children we may love and obey Him. \textbf{Had this been understood, such sentiments as are in this book would never have been expressed}.}[Lt214-1903.10; 1903][https://egwwritings.org/read?panels=p10700.16]


\egw{لا يمكن لعقل بشري أن يدرك الله. لم يره أحد قط. نحن جاهلون بالله كالأطفال الصغار. ولكن كالأطفال الصغار يمكننا أن نحبه ونطيعه. \textbf{لو كان هذا مفهوماً، لما تم التعبير أبداً عن آراء كتلك الموجودة في هذا الكتاب}.}[Lt214-1903.10; 1903][https://egwwritings.org/read?panels=p10700.16]


You might wonder why Sister White said that she will not engage in controversy with anyone concerning the nature and \emcap{personality of God}, while she was heavily engaged in the controversy over the \emcap{personality of God}, and wrote many different testimonies regarding it. Discussions regarding the \emcap{personality of God}, to some degree, touch the nature of God; yet, those regarding the nature of God, in connection to the \emcap{personality of God}, Sister White did not engage in. She knew where to draw the line. She pointed out that the Bible should draw this line for us. \egw{\textbf{\underline{Let the Scriptures be read in simple faith, and let each one form his conceptions of God from His inspired Word.}}} The \emcap{Fundamental Principles} obey this rule. Sister White told us that we must not try to explain in regard to the \emcap{personality of God} any further than the Bible has done.


قد تتساءل لماذا قالت الأخت وايت أنها لن تنخرط في جدال مع أي شخص بخصوص طبيعة و\emcap{شخصانية الله}، بينما كانت منخرطة بشدة في الصراع حول \emcap{شخصانية الله}، وكتبت العديد من الشهادات المختلفة بشأنها. المناقشات المتعلقة بـ\emcap{شخصانية الله}، إلى حد ما، تمس طبيعة الله؛ ومع ذلك، تلك المتعلقة بطبيعة الله، فيما يتصل بـ\emcap{شخصانية الله}، لم تنخرط فيها الأخت وايت. كانت تعرف أين ترسم الخط الفاصل. أشارت إلى أن الكتاب المقدس ينبغي أن يرسم هذا الخط لنا. \egw{\textbf{\underline{لتُقرأ الكتب المقدسة بإيمان بسيط، وليشكل كل واحد تصوراته عن الله من كلمته الموحى بها.}}} تطيع \emcap{المبادئ الأساسية} هذه القاعدة. أخبرتنا الأخت وايت أننا يجب ألا نحاول أن نشرح فيما يتعلق بـ\emcap{شخصانية الله} أكثر مما فعله الكتاب المقدس.


\egw{Keep your eyes fixed on the Lord Jesus Christ, and by beholding Him you will be changed into His likeness. \textbf{Talk not of these spiritualistic theories. Let them find no place in your mind.} Let our papers be kept free from everything of the kind. Publish the truth; do not publish error. \textbf{Do not try to explain in regard to the personality of God. \underline{You cannot give any further explanation than the Bible has given}}. \textbf{Human theories regarding Him are good for nothing}. Do not soil your minds by studying the misleading theories of the enemy. Labor to draw minds away from everything of this character. It will be better to keep these subjects out of our papers. Let the doctrines of present truth be put into our papers, but give no room to a repeating of erroneous theories.}[Lt179-1904.4; 1904][https://egwwritings.org/read?panels=p7751.11]


\egw{ثبتوا أعينكم على الرب يسوع المسيح، وبالنظر إليه ستتغيرون إلى صورته. \textbf{لا تتحدثوا عن هذه النظريات الروحانية. لا تدعوها تجد مكاناً في عقولكم.} لتُحفظ صحفنا خالية من كل شيء من هذا النوع. انشروا الحق؛ لا تنشروا الخطأ. \textbf{لا تحاولوا أن تشرحوا فيما يتعلق بشخصانية الله. \underline{لا يمكنكم تقديم أي شرح إضافي أكثر مما قدمه الكتاب المقدس}}. \textbf{النظريات البشرية المتعلقة به لا قيمة لها}. لا تلوثوا عقولكم بدراسة النظريات المضللة للعدو. اعملوا على إبعاد العقول عن كل شيء من هذا النوع. سيكون من الأفضل إبقاء هذه المواضيع خارج صحفنا. لتوضع عقائد الحق الحاضر في صحفنا، ولكن لا تعطوا مجالاً لتكرار النظريات الخاطئة.}[Lt179-1904.4; 1904][https://egwwritings.org/read?panels=p7751.11]


Let the Bible form our conceptions of God. We cannot give any further explanation of the \emcap{personality of God} than the Bible has given. If the Bible speaks of God that, in His person, He is bound to one locality, like His temple, the sanctuary, and His throne, we should accept that regardless of whether it sounds limiting to God. God is limited in space, in His body, but His presence is not limited, for He is everywhere present by His representative, the Holy Spirit.


لندع الكتاب المقدس يشكل تصوراتنا عن الله. لا يمكننا تقديم أي شرح إضافي لـ\emcap{شخصانية الله} أكثر مما قدمه الكتاب المقدس. إذا تحدث الكتاب المقدس عن الله بأنه، في شخصه، مرتبط بمكان واحد، مثل هيكله، والمقدس، وعرشه، فعلينا أن نقبل ذلك بغض النظر عما إذا كان يبدو مقيداً لله. الله محدود في المكان، في جسده، لكن حضوره غير محدود، لأنه موجود في كل مكان بواسطة ممثله، الروح القدس.


The revelation of God does express some limitations of His, and some of them are of a salvational matter. For instance, the Bible clearly says that God is omnipotent (Revelation 19:6), He can do all, yet we find that He could save men by no other means than giving His only begotten Son for us. In the garden of Gethsemane, when God handed the cup of His wrath to His Son, Christ prayed for the possibility that this cup could pass from Him, but ultimately for God's will to be done. Here we see all of the available options the Father had in order to save men. It was not possible to save fallen men, other than for God’s Son to die in their stead. Many protest the idea that something was impossible for God. But if it was possible for God to save men, without His Son drinking the cup of His wrath, surely God would have done it. Some protest this idea of God being limited to only one option of saving men, while He might have infinite options—He is omnipotent, after all. With this thinking, God’s salvation of lost men by the sacrifice of His own begotten Son is enshrouded with doubt, and essentially rejected, even scorned, depicting God as a child murderer. But the revelation is clear in the face of these skeptics. It is not God who is heinous for giving His Son for us; it is sin that is heinous. Sin had demanded this infinite sacrifice to be laid, and there was no other way. That was not roleplay\footnote{The Week of Prayer issue by the Adventist Review, October 31, 1996}, but a reality, that caused infinite grief and suffering to our heavenly Father in giving His own begotten\footnote{Read about God’s gift of His \egwinline{own begotten Son} in \href{https://egwwritings.org/?ref=en_Lt13-1894.18&para=5486.24}{{EGW, Lt13-1894.18; 1894}}}, obedient Son to die in our stead.


إن إعلان الله يعبر عن بعض حدوده، وبعضها من أمور الخلاص. على سبيل المثال، يقول الكتاب المقدس بوضوح إن الله قادر على كل شيء (رؤيا 19:6)، يمكنه أن يفعل كل شيء، ومع ذلك نجد أنه لم يستطع أن يخلص البشر بأي وسيلة أخرى غير تقديم ابنه الوحيد المولود من أجلنا. في بستان جثسيماني، عندما قدم الله كأس غضبه لابنه، صلى المسيح من أجل إمكانية أن تعبر عنه هذه الكأس، ولكن في النهاية لتكن مشيئة الله. هنا نرى جميع الخيارات المتاحة للآب من أجل خلاص البشر. لم يكن من الممكن خلاص البشر الساقطين، إلا بموت ابن الله بدلاً منهم. يعترض الكثيرون على فكرة أن شيئاً ما كان مستحيلاً على الله. ولكن لو كان من الممكن لله أن يخلص البشر، دون أن يشرب ابنه كأس غضبه، لكان الله قد فعل ذلك بالتأكيد. يعترض البعض على فكرة أن الله مقيد بخيار واحد فقط لخلاص البشر، بينما قد تكون لديه خيارات لا حصر لها - فهو قادر على كل شيء، بعد كل شيء. بهذا التفكير، يتم إحاطة خلاص الله للبشر الضالين بتضحية ابنه المولود بالشك، ورفضه أساساً، بل والسخرية منه، مصوراً الله كقاتل أطفال. لكن الإعلان واضح في وجه هؤلاء المشككين. ليس الله هو البشع لأنه قدم ابنه من أجلنا؛ بل الخطية هي البشعة. لقد تطلبت الخطية هذه التضحية اللامتناهية، ولم يكن هناك طريق آخر. لم يكن ذلك تمثيلية\footnote{عدد أسبوع الصلاة من مجلة الأدفنتست ريفيو، 31 أكتوبر 1996}، بل واقعاً، تسبب في حزن ومعاناة لا متناهية لأبينا السماوي في تقديم ابنه المولود\footnote{اقرأ عن عطية الله \egwinline{ابنه المولود} في \href{https://egwwritings.org/?ref=en_Lt13-1894.18&para=5486.24}{{EGW, Lt13-1894.18; 1894}}} المطيع ليموت بدلاً منا.


Let our conceptions of who God is, what God is, and of what character He is, be molded by plain Scripture, and let us not doubt it.


لتتشكل تصوراتنا عن من هو الله، وما هو الله، وما هي صفاته، بواسطة الكتاب المقدس الواضح، ولا نشك فيه.


% The bottom of the issue

\begin{titledpoem}
    \stanza{
        Beside the track of truth, error does tread, \\
        A line so fine, where the Holy Spirit lead. \\
        In words familiar, God's personas blend, \\
        Father and son, where meanings extend.
    }

    \stanza{
        Two views diverge on this sacred script, \\
        One symbolic, the other clearly depicted. \\
        As mirrors of man, in His image cast, \\
        God forms our essence, from the first to the last.
    }

    \stanza{
        Father and Son, in literal hues, \\
        Or metaphors for the divine clues? \\
        Truth's narrow path, so closely lain, \\
        Beside the error, we strive to explain.
    }

    \stanza{
        For through the Bible, meanings unfold, \\
        In God's own language, bold and told. \\
        Not just in symbols, but in our frame, \\
        His likeness, His nature, forever the same.
    }

    \stanza{
        As earthly fathers reflect His ways, \\
        So too our spirit His nature portrays. \\
        In discussions of God, where mysteries thrive, \\
        Let scriptures speak, and in faith we dive.
    }

    \stanza{
        Keep to what's revealed, in the Word abide, \\
        Where human theories and errors collide. \\
        God in His fullness, a mystery remains, \\
        Yet in His Word, His truth sustains.
    }
\end{titledpoem}