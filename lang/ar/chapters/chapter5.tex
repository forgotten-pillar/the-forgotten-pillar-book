\qrchapter{https://forgottenpillar.com/rsc/en-fp-chapter5}{The patchwork theories - Lt253-1903}


\qrchapter{https://forgottenpillar.com/rsc/en-fp-chapter5}{نظريات الترقيع - Lt253-1903}


\egw{Dear Brother,—}


\egw{أخي العزيز،—}


\egwnogap{\textbf{I must tell you that your ideas in regard to some things \underline{have been decidedly wrong}.} I would that you could see your errors. \textbf{The book Living Temple \underline{is not to be patched up}, a few changes made in it, and then advertised and praised as a valuable production}. It would be better to present the physiological parts in another book under another title. \textbf{When you wrote that book}, \textbf{you were not under the inspiration of God}. There was by your side the one who inspired Adam to look at God in a false light. Your whole heart needs to be changed, thoroughly and entirely cleansed.}[Lt253-1903.1; 1903][https://egwwritings.org/read?panels=p9980.7]


\egwnogap{\textbf{يجب أن أخبرك أن أفكارك فيما يتعلق ببعض الأمور \underline{كانت خاطئة بشكل واضح}.} أتمنى لو أنك تستطيع أن ترى أخطاءك. \textbf{كتاب ذا ليفينغ تمبل \underline{لا ينبغي أن يُرقَّع}، مع إجراء بعض التغييرات فيه، ثم الإعلان عنه والثناء عليه كإنتاج قيم}. سيكون من الأفضل تقديم الأجزاء الفسيولوجية في كتاب آخر تحت عنوان آخر. \textbf{عندما كتبت ذلك الكتاب}، \textbf{لم تكن تحت إلهام الله}. كان بجانبك ذلك الذي ألهم آدم لينظر إلى الله بطريقة خاطئة. قلبك بأكمله يحتاج إلى تغيير، وتنظيف شامل وكامل.}[Lt253-1903.1; 1903][https://egwwritings.org/read?panels=p9980.7]


\egwnogap{\textbf{My brother, do not allow yourself to be alienated from your ministering brethren who tell you of your dangers. Those who faithfully and frankly tell you of your errors are your best friends.} I am sorry, very sorry, for your medical associates. They have been unfaithful to God and untrue to you in failing to tell you kindly but firmly where you were not working righteously.}[Lt253-1903.2; 1903][https://egwwritings.org/read?panels=p9980.8]


\egwnogap{\textbf{أخي، لا تسمح لنفسك بالابتعاد عن إخوتك الخدام الذين يخبرونك بمخاطرك. أولئك الذين يخبرونك بأخطائك بأمانة وصراحة هم أفضل أصدقائك.} أنا آسفة، آسفة جدًا، لزملائك الطبيين. لقد كانوا غير أمناء لله وغير صادقين معك في فشلهم بإخبارك بلطف ولكن بحزم أين كنت لا تعمل بصلاح.}[Lt253-1903.2; 1903][https://egwwritings.org/read?panels=p9980.8]


\egwnogap{There are many things that you must overcome before you can be saved. In the heart that is not led by God, there is a something that leads it to desire to be sustained in its wrong course. The men who faithfully tell you the truth, pointing out your mistakes, you have regarded as your enemies. But often they are your best friends and, in telling you wherein you were walking in strange paths, were doing a very disagreeable duty. The Lord’s servants are not to flatter your pride; they are not to stand silent, fearing to say, ‘Why do ye thus?’ They are faithfully to warn you of your danger.}[Lt253-1903.3; 1903][https://egwwritings.org/read?panels=p9980.9]


\egwnogap{هناك أشياء كثيرة يجب أن تتغلب عليها قبل أن تخلص. في القلب الذي لا يقوده الله، هناك شيء ما يدفعه إلى الرغبة في أن يُدعم في مساره الخاطئ. الرجال الذين يخبرونك بالحقيقة بأمانة، مشيرين إلى أخطائك، اعتبرتهم أعداءك. لكنهم غالبًا أفضل أصدقائك، وعندما أخبروك بأنك كنت تسير في طرق غريبة، كانوا يؤدون واجبًا غير مريح للغاية. خدام الرب ليسوا هنا ليتملقوا كبرياءك؛ إنهم ليسوا ليقفوا صامتين، خائفين من أن يقولوا، “لماذا تفعل هكذا؟“ إنهم يحذرونك بأمانة من خطرك.}[Lt253-1903.3; 1903][https://egwwritings.org/read?panels=p9980.9]


\egwnogap{\textbf{My husband, Elder Joseph Bates, Father Pierce, Elder Edson, and many others who were keen, noble, and true were among those who, after the passing of the time in 1844, searched for truth}. \textbf{At our important meetings, these men would meet together and search for the truth as for hidden treasure}. I met with them, and we studied and prayed earnestly; for we felt that we must learn God’s truth. Often we remained together until late at night, and sometimes through the entire night, praying for light and studying the Word. As we fasted and prayed, great power came upon us. But I could not understand the reasoning of the brethren. My mind was locked, as it were, and I could not comprehend what we were studying. Then the Spirit of God would come upon me, I would be taken off in vision, and a clear explanation of the passages we had been studying would be given me with instruction as to the position we were to take regarding truth and duty. Again and again this happened. \textbf{A line of truth extending from that time to the time when we shall enter the city of God was plainly marked out before me}, and I gave my brethren and sisters the instruction that the Lord had given me. They knew that when not in vision, I could not understand these matters, and they accepted as light direct from heaven the revelations given me. \textbf{Thus the leading points of our faith as we hold them today were firmly established}. \textbf{\underline{Point after point} was clearly defined, and all the brethren came into harmony}.}[Lt253-1903.4; 1903][https://egwwritings.org/read?panels=p14068.9980010]


\egwnogap{\textbf{زوجي، الشيخ جوزيف بيتس، الأب بيرس، الشيخ إدسون، والعديد من الآخرين الذين كانوا أذكياء، نبلاء، وصادقين كانوا من بين أولئك الذين، بعد مرور الوقت في عام 1844، بحثوا عن الحق}. \textbf{في اجتماعاتنا المهمة، كان هؤلاء الرجال يجتمعون معًا ويبحثون عن الحق كما لو كان كنزًا مخفيًا}. اجتمعت معهم، ودرسنا وصلينا بجدية؛ لأننا شعرنا أنه يجب علينا أن نتعلم حق الله. غالبًا ما بقينا معًا حتى وقت متأخر من الليل، وأحيانًا طوال الليل، نصلي من أجل النور وندرس الكلمة. بينما كنا نصوم ونصلي، جاءت علينا قوة عظيمة. لكنني لم أستطع فهم منطق الإخوة. كان عقلي مغلقًا، إن جاز التعبير، ولم أستطع فهم ما كنا ندرسه. ثم كانت روح الله تأتي عليّ، وكنت أُؤخذ في رؤيا، ويُعطى لي شرح واضح للمقاطع التي كنا ندرسها مع تعليمات حول الموقف الذي يجب أن نتخذه فيما يتعلق بالحق والواجب. حدث هذا مرارًا وتكرارًا. \textbf{خط من الحق يمتد من ذلك الوقت إلى الوقت الذي سندخل فيه مدينة الله كان مرسومًا بوضوح أمامي}، وأعطيت إخوتي وأخواتي التعليمات التي أعطاني إياها الرب. كانوا يعلمون أنني عندما لا أكون في رؤيا، لم أستطع فهم هذه الأمور، وقبلوا الإعلانات المعطاة لي كنور مباشر من السماء. \textbf{وهكذا تم تأسيس النقاط الرئيسية لإيماننا كما نتمسك بها اليوم بثبات}. \textbf{\underline{نقطة تلو الأخرى} تم تحديدها بوضوح، وجميع الإخوة أصبحوا في انسجام}.}[Lt253-1903.4; 1903][https://egwwritings.org/read?panels=p14068.9980010]


\egwnogap{\textbf{The whole company of believers were united in the truth}. \textbf{There were those who came in with strange doctrines, but we were never afraid to meet them. Our experience was wonderfully established by the revelations of the Holy Spirit}.}[Lt253-1903.5; 1903][https://egwwritings.org/read?panels=p9980.11]


\egwnogap{\textbf{كانت جماعة المؤمنين بأكملها متحدة في الحق}. \textbf{كان هناك من جاءوا بتعاليم غريبة، لكننا لم نخف أبدًا من مواجهتهم. تم تأسيس خبرتنا بشكل رائع من خلال إعلانات الروح القدس}.}[Lt253-1903.5; 1903][https://egwwritings.org/read?panels=p9980.11]


\egwnogap{For two or three years my mind continued to be locked to the Scriptures. In 1846 I was married to Elder James White. It was some time after my second son was born that we were in great perplexity regarding certain points of doctrine. I was praying to the Lord to unlock my mind, that I might understand His Word. Suddenly I seemed to be enshrouded in clear, beautiful light, and ever since, \textbf{the Scriptures have been an open book to me}.}[Lt253-1903.6; 1903][https://egwwritings.org/read?panels=p14068.9980012]


\egwnogap{لمدة عامين أو ثلاثة استمر عقلي مغلقًا أمام الكتب المقدسة. في عام 1846 تزوجت من الشيخ جيمس وايت. كان ذلك بعد فترة من ولادة ابني الثاني عندما كنا في حيرة كبيرة بشأن نقاط معينة من العقيدة. كنت أصلي إلى الرب ليفتح عقلي، حتى أفهم كلمته. فجأة بدا وكأنني محاطة بنور جميل وواضح، ومنذ ذلك الحين، \textbf{أصبحت الكتب المقدسة كتابًا مفتوحًا بالنسبة لي}.}[Lt253-1903.6; 1903][https://egwwritings.org/read?panels=p14068.9980012]


\egwnogap{I was at that time in Paris, Maine. Old Father Andrews was very sick. For some time he had been a great sufferer from inflammatory rheumatism. He could not move without intense pain. We prayed for him. I laid my hands on his head, and said, “Father Andrews, the Lord Jesus maketh thee whole.” He was healed instantly. He got up and walked about the room, praising God, and saying, “I never saw it on this wise before. Angels of God are in this room.” The glory of God was revealed. \textbf{Light seemed to shine all through the house, and an angel’s hand was laid upon my head. From that time to this I have been able to understand the Word of God.}}[Lt253-1903.7; 1903][https://egwwritings.org/read?panels=p9980.13]


\egwnogap{كنت في ذلك الوقت في باريس، مين. كان الأب أندروز العجوز مريضًا جدًا. لبعض الوقت كان يعاني كثيرًا من الروماتيزم الالتهابي. لم يستطع التحرك دون ألم شديد. صلينا من أجله. وضعت يدي على رأسه، وقلت: “أيها الأب أندروز، الرب يسوع يشفيك.” شُفي على الفور. نهض ومشى في الغرفة، مسبحًا الله، وقائلاً: “لم أر قط مثل هذا من قبل. ملائكة الله في هذه الغرفة.” تجلى مجد الله. \textbf{بدا النور يشع في جميع أنحاء المنزل، ووُضعت يد ملاك على رأسي. من ذلك الوقت إلى الآن تمكنت من فهم كلمة الله.}}[Lt253-1903.7; 1903][https://egwwritings.org/read?panels=p9980.13]


\egwnogap{\textbf{After the passing of the time, we were opposed and cruelly falsified. Erroneous theories were pressed in upon us by men and women who had gone into fanaticism}. I was directed to go to the places where these people were advocating these erroneous theories, and as I went, the power of the Spirit was wonderfully displayed in rebuking the errors that were creeping in. \textbf{\underline{Satan himself, in the person of a man}, was working to make of no effect my testimony regarding the position that we now know to be substantiated by Scripture.}}[Lt253-1903.8; 1903][https://egwwritings.org/read?panels=p9980.14]


\egwnogap{\textbf{بعد مرور الوقت، تعرضنا للمعارضة والتزييف القاسي. تم فرض نظريات خاطئة علينا من قبل رجال ونساء وقعوا في التعصب}. تم توجيهي للذهاب إلى الأماكن التي كان هؤلاء الناس يدعون فيها هذه النظريات الخاطئة، وعندما ذهبت، تجلت قوة الروح بشكل رائع في توبيخ الأخطاء التي كانت تتسلل. \textbf{\underline{الشيطان نفسه، في شخص إنسان}، كان يعمل ليجعل شهادتي بلا تأثير فيما يتعلق بالموقف الذي نعرف الآن أنه مدعوم بالكتاب المقدس.}}[Lt253-1903.8; 1903][https://egwwritings.org/read?panels=p9980.14]


\egwnogap{\textbf{Just such theories as you have presented in Living Temple were presented then}. \textbf{These subtle, deceiving sophistries have again and again sought to find place amongst us. \underline{But I have ever had the same testimony to bear which I now bear regarding the personality of God}}.}[Lt253-1903.9; 1903][https://egwwritings.org/read?panels=p9980.15]


\egwnogap{\textbf{مثل هذه النظريات التي قدمتها في ذا ليفينغ تمبل تم تقديمها آنذاك}. \textbf{هذه السفسطات الخادعة الدقيقة سعت مرارًا وتكرارًا للعثور على مكان بيننا. \underline{لكن كان لدي دائمًا نفس الشهادة لأقدمها والتي أقدمها الآن بخصوص شخصانية الله}}.}[Lt253-1903.9; 1903][https://egwwritings.org/read?panels=p9980.15]


\egwnogap{In (Early Writings, 60, 66, 67)\footnote{It appears that the pages are incorrect. The mentioned paragraphs can be found in Early Writings on pages \href{https://egwwritings.org/read?panels=p28.462&index=0}{70.2}, \href{https://egwwritings.org/read?panels=p28.490&index=0}{77}, and \href{https://egwwritings.org/read?panels=p28.390&index=0}{54.2}.}, are the following statements:}[Lt253-1903.10; 1903][https://egwwritings.org/read?panels=p9980.16]


\egwnogap{في (كتابات مبكرة، 60، 66، 67)\footnote{يبدو أن الصفحات غير صحيحة. يمكن العثور على الفقرات المذكورة في كتابات مبكرة في الصفحات \href{https://egwwritings.org/read?panels=p28.462&index=0}{70.2}، \href{https://egwwritings.org/read?panels=p28.490&index=0}{77}، و \href{https://egwwritings.org/read?panels=p28.390&index=0}{54.2}.}، توجد البيانات التالية:}[Lt253-1903.10; 1903][https://egwwritings.org/read?panels=p9980.16]


\egwnogap{‘May 14, 1851, I saw the beauty and loveliness of Jesus. As I beheld His glory, the thought did not occur to me that I should ever be separated from His presence. \textbf{I saw a light coming from the glory that encircled the Father}, and as it approached near to me, my body shook and trembled like a leaf. I thought that if it should come near me, I would be struck out of existence; but the light passed me. \textbf{Then could I have some sense of the great and terrible \underline{God} with whom we have to do}.’}[Lt253-1903.11; 1903][https://egwwritings.org/read?panels=p9980.17]


\egwnogap{‘في 14 مايو 1851، رأيت جمال ولطف يسوع. وبينما كنت أتأمل مجده، لم يخطر ببالي أنني سأنفصل يومًا عن حضوره. \textbf{رأيت نورًا قادمًا من المجد الذي يحيط بالآب}، وعندما اقترب مني، اهتز جسدي وارتعش كورقة. ظننت أنه إذا اقترب مني، سأُمحى من الوجود؛ لكن النور مر بي. \textbf{حينئذ استطعت أن أدرك شيئًا عن عظمة ورهبة \underline{الله} الذي نتعامل معه}.’}[Lt253-1903.11; 1903][https://egwwritings.org/read?panels=p9980.17]


\egwnogap{‘I have often seen \textbf{the lovely Jesus, that He is a person}. \textbf{I asked Him if His Father was a person, and had \underline{a form} like Himself}. Said Jesus, ‘\textbf{I am the express image of My Father’s person!}’ [Hebrews 1:3.]}[Lt253-1903.12; 1903][https://egwwritings.org/read?panels=p9980.18]


\egwnogap{‘لقد رأيت كثيرًا \textbf{يسوع المحبوب، وأنه شخص}. \textbf{سألته إذا كان أبوه شخصًا، وله \underline{هيئة} مثله}. قال يسوع، ‘\textbf{أنا صورة جوهره}!’ [عبرانيين 1:3.]}[Lt253-1903.12; 1903][https://egwwritings.org/read?panels=p9980.18]


\egwnogap{‘\textbf{I have often seen that the spiritual view took away all the glory of heaven, and that in many minds the throne of David and the lovely person of Jesus have been burned up in the fire of spiritualism}. I have seen that some who have been deceived and led into this error, will be brought out into the light of truth, \textbf{but it will be almost impossible for them to get entirely rid of the deceptive power of spiritualism. Such should make thorough work in confessing their errors, and leaving them forever}.’}[Lt253-1903.13; 1903][https://egwwritings.org/read?panels=p9980.19]


\egwnogap{‘\textbf{لقد رأيت كثيرًا أن النظرة الروحانية أزالت كل مجد السماء، وأنه في أذهان كثيرة تم حرق عرش داود وشخص يسوع المحبوب في نار الروحانية}. لقد رأيت أن بعض الذين خُدعوا وقادوا إلى هذا الخطأ، سيُخرجون إلى نور الحق، \textbf{لكن سيكون من شبه المستحيل عليهم التخلص تمامًا من القوة الخادعة للروحانية. على هؤلاء أن يقوموا بعمل شامل في الاعتراف بأخطائهم، وتركها إلى الأبد}.’}[Lt253-1903.13; 1903][https://egwwritings.org/read?panels=p9980.19]


\egwnogap{\textbf{There is a strain of spiritualism \underline{coming in} among our people, and \underline{it will undermine the faith} of those who give place to it, leading them to give heed to seducing spirits and doctrines of devils}. Errors will be presented in a pleasing and flattering manner. The enemy desires to divert the minds of our brethren and sisters from the work of preparing a people to stand in these last days.}[Lt253-1903.14; 1903][https://egwwritings.org/read?panels=p9980.21]


\egwnogap{\textbf{هناك تيار من الروحانية \underline{يدخل} بين شعبنا، و\underline{سيقوض إيمان} أولئك الذين يفسحون له المجال، مما يؤدي بهم إلى الإصغاء إلى أرواح مضللة وتعاليم شياطين}. ستُقدم الأخطاء بطريقة مُرضية ومُطرية. يرغب العدو في تحويل أذهان إخوتنا وأخواتنا عن عمل إعداد شعب ليقف في هذه الأيام الأخيرة.}[Lt253-1903.14; 1903][https://egwwritings.org/read?panels=p9980.21]


\egwnogap{I am instructed to warn our brethren and sisters \textbf{not to discuss the nature of our God}. Many of the curious who attempted to open the ark of the testament, to see what was inside, were punished for their presumption. \textbf{We are not to say that the Lord God of heaven is in a leaf, or in a tree; for He is not there. \underline{He sitteth upon His throne in the heavens}.}}[Lt253-1903.15; 1903][https://egwwritings.org/read?panels=p9980.22]


\egwnogap{لقد أُمرت بتحذير إخوتنا وأخواتنا \textbf{من مناقشة طبيعة إلهنا}. كثير من الفضوليين الذين حاولوا فتح تابوت العهد، ليروا ما بداخله، عوقبوا على تجاسرهم. \textbf{لا يجب أن نقول إن الرب إله السماء موجود في ورقة، أو في شجرة؛ لأنه ليس هناك. \underline{إنه يجلس على عرشه في السماوات}.}}[Lt253-1903.15; 1903][https://egwwritings.org/read?panels=p9980.22]


\egwnogap{The work of the Creator as seen in nature reveals His power. But nature is not above God, nor is God in nature as some represent Him to be. God made the world, but the world is not God; it is but the work of His hands. \textbf{Nature reveals the work of a positive, \underline{personal God}, showing that God is, and that He is a rewarder of those who diligently seek Him}.}[Lt253-1903.16, 1903][https://egwwritings.org/read?panels=p9980.23]


\egwnogap{عمل الخالق كما يُرى في الطبيعة يكشف قوته. لكن الطبيعة ليست فوق الله، ولا الله في الطبيعة كما يصوره البعض. الله صنع العالم، لكن العالم ليس الله؛ إنه مجرد عمل يديه. \textbf{الطبيعة تكشف عمل \underline{إله شخصي} إيجابي}، مُظهرة أن الله موجود، وأنه يجازي الذين يجتهدون في طلبه.}[Lt253-1903.16, 1903][https://egwwritings.org/read?panels=p9980.23]


\egwnogap{I could say much regarding the sanctuary; the ark containing the law of God; the cover of the ark, which is the mercy seat; the angels at either end of the ark; and other things connected with the heavenly sanctuary and with the great day of atonement. I could say much regarding the mysteries of heaven; but my lips are closed. I have no inclination to try to describe them.}[Lt253-1903.17; 1903][https://egwwritings.org/read?panels=p9980.25]


\egwnogap{يمكنني أن أقول الكثير بخصوص المقدس؛ التابوت الذي يحتوي على شريعة الله؛ غطاء التابوت، الذي هو كرسي الرحمة؛ الملائكة عند طرفي التابوت؛ وأشياء أخرى مرتبطة بالمقدس السماوي وبيوم الكفارة العظيم. يمكنني أن أقول الكثير بخصوص أسرار السماء؛ لكن شفتاي مغلقتان. ليس لدي رغبة في محاولة وصفها.}[Lt253-1903.17; 1903][https://egwwritings.org/read?panels=p9980.25]


\egwnogap{\textbf{I would not dare to speak of God as you have spoken of Him}. He is high and lifted up, and His glory fills the heavens. “The voice of the Lord is mighty; it shaketh the cedars of Lebanon. \textbf{The Lord is in His holy temple}; let all the earth keep silence before Him.” [See Psalm 29:5; Habakkuk 2:20.]}[Lt253-1903.18; 1903][https://egwwritings.org/read?panels=p9980.26]


\egwnogap{\textbf{لا أجرؤ على التحدث عن الله كما تحدثت أنت عنه}. إنه عالٍ ومرتفع، ومجده يملأ السماوات. “صوت الرب قوي؛ يكسر أرز لبنان. \textbf{الرب في هيكل قدسه}؛ فلتصمت الأرض كلها أمامه.” [انظر مزمور 29:5؛ حبقوق 2:20.]}[Lt253-1903.18; 1903][https://egwwritings.org/read?panels=p9980.26]


\egwnogap{\textbf{My brother, when you are tempted to speak of God, \underline{where He is, or what He is}, remember that on this point silence is eloquence}. Take off your shoes from off your feet; for the ground on which you are placing your careless, unsanctified feet is holy ground.}[Lt253-1903.19; 1903][https://egwwritings.org/read?panels=p14068.9980027]


\egwnogap{\textbf{أخي، عندما تُجرَّب للتحدث عن الله، \underline{أين هو، أو ما هو}، تذكر أن الصمت في هذه النقطة هو البلاغة}. اخلع حذاءك من قدميك؛ لأن الأرض التي تضع عليها قدميك المستهترتين غير المقدستين هي أرض مقدسة.}[Lt253-1903.19; 1903][https://egwwritings.org/read?panels=p14068.9980027]


\egwnogap{\textbf{I am instructed to say that there is nothing in the Word of God to substantiate your spiritualistic theories. Will you not renounce these theories at once? Upon them your mind has been dwelling for a long time, but they have had no sanctifying, refining, ennobling influence upon your life. The Lord has no use for these theories, and He would not have His people vindicate or propagate them.}}[Lt253-1903.20; 1903][https://egwwritings.org/read?panels=p9980.28]


\egwnogap{\textbf{لقد أُمرت أن أقول إنه لا يوجد شيء في كلمة الله يدعم نظرياتك الروحانية. ألن تتخلى عن هذه النظريات على الفور؟ لقد كان ذهنك منشغلاً بها لفترة طويلة، لكنها لم يكن لها أي تأثير مقدس أو منقٍّ أو مهذب على حياتك. الرب ليس له حاجة بهذه النظريات، وهو لا يريد لشعبه أن يدافع عنها أو ينشرها.}}[Lt253-1903.20; 1903][https://egwwritings.org/read?panels=p9980.28]


\egwnogap{\textbf{The Father, the omniscient One, created the world \underline{through} Christ Jesus}. Christ is the light of the world, the way to eternal life. He, the anointed One, God gave to make an atonement for the sins of the world. You need to understand that unless you believe \textbf{in that atonement}, and know that you are bought with the price of the blood of \textbf{the only begotten Son of God}, you will assuredly be bound up with the wicked one. \textbf{If you continue to cherish the theories that you have been cherishing, you will be left to become the sport of Satan’s temptations}. He is playing the game of life for your soul. Remain for a little longer linked up with him, and be assured that you will lose your soul.}[Lt253-1903.21; 1903][https://egwwritings.org/read?panels=p9980.29]


\egwnogap{\textbf{الآب، العليم بكل شيء، خلق العالم \underline{من خلال} المسيح يسوع}. المسيح هو نور العالم، الطريق إلى الحياة الأبدية. هو، الممسوح، الذي أعطاه الله ليقدم كفارة عن خطايا العالم. عليك أن تفهم أنه ما لم تؤمن \textbf{بتلك الكفارة}، وتعلم أنك اشتُريت بثمن دم \textbf{ابن الله الوحيد}، فستكون حتماً مرتبطاً بالشرير. \textbf{إذا استمررت في التمسك بالنظريات التي كنت تتمسك بها، فستُترك لتصبح ألعوبة لتجارب الشيطان}. إنه يلعب لعبة الحياة من أجل روحك. ابقَ مرتبطاً به لفترة أطول قليلاً، وتأكد أنك ستخسر روحك.}[Lt253-1903.21; 1903][https://egwwritings.org/read?panels=p9980.29]


\egwnogap{By declaring that our institutions are undenominational, you have put our people and our work in a false position. You have been led over a terrible path, the dangers of which you have not known, but may sometime see. It is not yet too late for wrongs to be righted. There is hope for you. \textbf{You have followed the enemy step by step, striving to look into mysteries too high and holy for your comprehension}. \textbf{Then in your teaching the Holy One has been brought down to man’s \underline{scientific, spiritualistic ideas}}. You have been walking in crooked paths. You have lost the moral image of God. But there is hope for you. You may still turn your feet into the right path. Will you not now make straight paths for your feet, lest the lame be turned out of the way? Will you now refuse to sow one more seed of skepticism and sophistry in the minds of others? Will you now come to Christ and be healed?}[Lt253-1903.22; 1903][https://egwwritings.org/read?panels=p14068.9980030]


\egwnogap{بإعلانك أن مؤسساتنا غير طائفية، فقد وضعت شعبنا وعملنا في موقف خاطئ. لقد قُدت على طريق رهيب، لم تكن تعرف مخاطره، ولكن قد تراها في وقت ما. لم يفت الأوان بعد لتصحيح الأخطاء. هناك أمل لك. \textbf{لقد اتبعت العدو خطوة بخطوة، مجتهداً في النظر إلى أسرار عالية ومقدسة تفوق فهمك}. \textbf{ثم في تعليمك تم إنزال القدوس إلى مستوى \underline{أفكار الإنسان العلمية والروحانية}}. لقد كنت تسير في طرق ملتوية. لقد فقدت الصورة الأخلاقية لله. لكن هناك أمل لك. لا يزال بإمكانك تحويل قدميك إلى الطريق الصحيح. ألن تجعل الآن مسارات مستقيمة لقدميك، لئلا ينحرف العرج عن الطريق؟ ألن ترفض الآن زرع بذرة أخرى من الشك والسفسطة في عقول الآخرين؟ ألن تأتي الآن إلى المسيح وتُشفى؟}[Lt253-1903.22; 1903][https://egwwritings.org/read?panels=p14068.9980030]


\egwnogap{\textbf{I have hesitated and delayed about the sending out of that which the Spirit of the Lord has impelled me to write}. I did not want to be compelled to present the satanic influence of these sophistries. But unless there is a decided change, in yourself and your associates, I shall have to do this, to save others from following the path that you have been following. I shall have to obey the command given me of God, “\textbf{Meet it}.” This is the only thing that I can do.}[Lt253-1903.23; 1903][https://egwwritings.org/read?panels=p9980.31]


\egwnogap{\textbf{لقد ترددت وتأخرت في إرسال ما دفعني روح الرب لكتابته}. لم أكن أريد أن أُجبر على تقديم التأثير الشيطاني لهذه السفسطات. ولكن ما لم يكن هناك تغيير حاسم، فيك وفي زملائك، سيتعين علي القيام بذلك، لإنقاذ الآخرين من اتباع المسار الذي كنت تتبعه. سيتعين علي أن أطيع الأمر الذي أعطاني إياه الله، “\textbf{واجهه}.” هذا هو الشيء الوحيد الذي يمكنني فعله.}[Lt253-1903.23; 1903][https://egwwritings.org/read?panels=p9980.31]


\egwnogap{I present to you the things that the Lord has presented to me. There is a great work to be done. We are to take hold of the work understandingly, praying, believing, and receiving the Holy Spirit. Thus only can we do the work given us. \textbf{I am required by God to bear testimony against Living Temple}. Whatever your associates may say concerning this book,\textbf{ I take the position now and forever that it is a snare}. \textbf{No union will be formed by our people as a whole upon the \underline{theories} that you have begun to present in that book}. \textbf{You may regard this as forever decided}. \textbf{As a people we shall stand firm \underline{on the platform that has withstood test and trial}. We shall hold to the \underline{sure pillars of our faith}. \underline{The principles of truth} that God has revealed to us are our only foundation. They have made us what we are. These new, fanciful theories are fascinating and misleading. They endanger the eternal interests of the soul. The Scriptures do not sustain them}. Clothed with the Christian armor, shod with the preparation of the gospel of peace, we shall stand \textbf{firm against these misleading theories}. You may turn and wrest the Word of God to your own destruction, but I entreat you not to do this.}[Lt253-1903.24; 1903][https://egwwritings.org/read?panels=p9980.32]


\egwnogap{أقدم لك الأشياء التي قدمها لي الرب. هناك عمل عظيم يجب القيام به. علينا أن نتولى العمل بفهم، مصلين، مؤمنين، ومستقبلين الروح القدس. هكذا فقط يمكننا القيام بالعمل المعطى لنا. \textbf{الله يطلب مني أن أشهد ضد ذا ليفينغ تمبل}. مهما قال زملاؤك عن هذا الكتاب، \textbf{فإنني أتخذ الموقف الآن وإلى الأبد بأنه فخ}. \textbf{لن يتشكل اتحاد بين شعبنا ككل على \underline{النظريات} التي بدأت في تقديمها في ذلك الكتاب}. \textbf{يمكنك اعتبار هذا مقرراً إلى الأبد}. \textbf{كشعب سنقف بثبات \underline{على المنصة التي صمدت أمام الاختبار والتجربة}. سنتمسك \underline{بأعمدة إيماننا} الثابتة. \underline{مبادئ الحق} التي كشفها الله لنا هي أساسنا الوحيد. لقد جعلتنا ما نحن عليه. هذه النظريات الجديدة الخيالية ساحرة ومضللة. إنها تعرض المصالح الأبدية للروح للخطر. الكتاب المقدس لا يدعمها}. متسلحين بدرع المسيحي، منتعلين استعداد إنجيل السلام، سنقف \textbf{بثبات ضد هذه النظريات المضللة}. قد تلتفت وتحرف كلمة الله لهلاكك، لكنني أتوسل إليك ألا تفعل ذلك.}[Lt253-1903.24; 1903][https://egwwritings.org/read?panels=p9980.32]


\egwnogap{\textbf{Heaven is not a vapor. It is a place}. \textbf{Christ has gone to prepare mansions for those who love Him}, those who, in obedience to His commands, come out from the world and are separate. The principles of heaven must be brought into our experience, that we may be distinguished from the world. \textbf{There must be a marked contrast between us and the world; for we are God’s denominated people}.}[Lt253-1903.25; 1903][https://egwwritings.org/read?panels=p9980.33]


\egwnogap{\textbf{السماء ليست بخاراً. إنها مكان}. \textbf{المسيح ذهب ليعد منازل للذين يحبونه}، أولئك الذين، في طاعة لوصاياه، يخرجون من العالم ويكونون منفصلين. يجب إدخال مبادئ السماء في تجربتنا، حتى نتميز عن العالم. \textbf{يجب أن يكون هناك تباين واضح بيننا وبين العالم؛ لأننا شعب الله المعين}.}[Lt253-1903.25; 1903][https://egwwritings.org/read?panels=p9980.33]


\egwnogap{The Lord has given you an opportunity to make things right. \textbf{I rejoice that you have made a beginning. Do not think that we have no right to try to correct your errors and the results of these errors. As long as God gives me breath, and commissions me to use pen and voice in beating back this evil thing that has come in among us, I shall act my part in the warfare. Ever since I was seventeen years old, I have had to fight this battle against false theories, in defense of the truth}. \textbf{The history of our past experience is indelibly fixed in my mind, and I am determined that \underline{no theories of the order that you have been accepting} shall come into our ranks}. If you refuse to change, and labor to lead your associates after you, and they venture to follow your leading, the accountability rests with you and with them, not on my soul.}[Lt253-1903.26, 1903][https://egwwritings.org/read?panels=p9980.34]


\egwnogap{لقد أعطاك الرب فرصة لتصحيح الأمور. \textbf{أفرح لأنك بدأت. لا تظن أنه ليس لنا الحق في محاولة تصحيح أخطائك ونتائج هذه الأخطاء. طالما أن الله يعطيني نفساً، ويكلفني باستخدام القلم والصوت في صد هذا الشر الذي دخل بيننا، سأقوم بدوري في المعركة. منذ أن كنت في السابعة عشرة من عمري، كان علي أن أخوض هذه المعركة ضد النظريات الكاذبة، دفاعاً عن الحق}. \textbf{إن تاريخ تجربتنا الماضية مثبت بشكل لا يمحى في ذهني، وأنا مصممة على ألا \underline{تدخل نظريات من النوع الذي كنت تقبله} إلى صفوفنا}. إذا رفضت التغيير، وسعيت لقيادة زملائك وراءك، وتجرأوا على اتباع قيادتك، فإن المسؤولية تقع عليك وعليهم، وليس على روحي.}[Lt253-1903.26, 1903][https://egwwritings.org/read?panels=p9980.34]


\egwnogap{\textbf{I speak decidedly, in order that you may know, that unless there is a decided change in you, there can be no hope of a union between you and those who are holding the beginning of their confidence firm unto the end.} You have made the division. \textbf{\underline{We must stand firm for the truths that the Lord has given us as the pillars of our faith}}.}[Lt253-1903.27; 1903][https://egwwritings.org/read?panels=p9980.35]


\egwnogap{\textbf{أتحدث بحزم، لكي تعلم أنه ما لم يكن هناك تغيير حاسم فيك، فلا يمكن أن يكون هناك أمل في اتحاد بينك وبين أولئك الذين يتمسكون ببداية ثقتهم بثبات إلى النهاية.} أنت من صنع الانقسام. \textbf{\underline{يجب أن نقف بثبات من أجل الحقائق التي أعطانا إياها الرب كأعمدة إيماننا}}.}[Lt253-1903.27; 1903][https://egwwritings.org/read?panels=p9980.35]


\egwnogap{I entreat you to turn to the Lord with full purpose of heart, before it is forever too late. Separate yourself from the influences which have separated you from your brethren who are engaged in the gospel ministry and from the people whom God is leading. \textbf{\underline{Patchwork theories} cannot be accepted by those who are loyal to the faith and to \underline{the principles} that have withstood all the opposition of satanic influences}.}[Lt253-1903.28; 1903][https://egwwritings.org/read?panels=p9980.36]


\egwnogap{أتوسل إليك أن تتوجه إلى الرب بعزم كامل في القلب، قبل أن يفوت الأوان إلى الأبد. افصل نفسك عن التأثيرات التي فصلتك عن إخوتك المنخرطين في خدمة الإنجيل وعن الناس الذين يقودهم الله. \textbf{\underline{نظريات الترقيع} لا يمكن أن يقبلها أولئك المخلصون للإيمان و\underline{للمبادئ} التي صمدت أمام كل معارضة التأثيرات الشيطانية}.}[Lt253-1903.28; 1903][https://egwwritings.org/read?panels=p9980.36]


\egwnogap{If you will empty yourself of all that has separated you from Christ, and receive the Saviour into your heart, you will be transformed in character. Lay off responsibilities for a time, and go away somewhere with a few of your brethren, and with them search the Scriptures. Humble your heart before the Lord, and make thorough work for repentance. \textbf{The religion of Christ is the spiritual leaven that is to be introduced into the heart. This changes the life and character}. This religion is a heavenly principle, seen in the Christian’s life and conversation. It is revealed in Christian purity. The love of Christ is seen in the tenderness and grace of sanctified humanity. It is by the Word made flesh that we are saved. Our redemption was wrought out, \textbf{not by the Son of God’s remaining in heaven, but by the Son of God’s becoming incarnate—taking humanity upon Him and coming to this world}. Thus eternal life was brought to us. That which authority, commands, and promises could not do, God did by coming to this world in the likeness of sinful flesh.}[Lt253-1903.29; 1903][https://egwwritings.org/read?panels=p9980.37]


\egwnogap{إذا أفرغت نفسك من كل ما فصلك عن المسيح، واستقبلت المخلص في قلبك، فستتحول في الشخصية. ضع المسؤوليات جانبًا لفترة من الوقت، واذهب إلى مكان ما مع بعض إخوتك، وابحثوا معًا في الكتاب المقدس. تواضع قلبك أمام الرب، واعمل عملاً شاملاً للتوبة. \textbf{إن دين المسيح هو الخميرة الروحية التي يجب إدخالها إلى القلب. هذا يغير الحياة والشخصية}. هذا الدين هو مبدأ سماوي، يُرى في حياة المسيحي وحديثه. إنه يتجلى في الطهارة المسيحية. محبة المسيح تُرى في اللطف والنعمة للإنسانية المقدسة. إننا نخلص بالكلمة التي صارت جسدًا. لقد تم إنجاز فدائنا، \textbf{ليس ببقاء ابن الله في السماء، بل بتجسد ابن الله - بأخذه الإنسانية عليه ومجيئه إلى هذا العالم}. هكذا أُحضرت لنا الحياة الأبدية. ما لم تستطع السلطة والأوامر والوعود أن تفعله، فعله الله بمجيئه إلى هذا العالم في شبه الجسد الخاطئ.}[Lt253-1903.29; 1903][https://egwwritings.org/read?panels=p9980.37]


\egwnogap{Christ came to the earth to live as a man among men, not to be spoiled by human frailty, but to place in the minds of men principles of truth that could never be obliterated, because they are eternally true. He came to bring a new life to fallen human beings—an excellence that could not be stained or deteriorated by sin.}[Lt253-1903.30; 1903][https://egwwritings.org/read?panels=p9980.38]


\egwnogap{جاء المسيح إلى الأرض ليعيش كإنسان بين الناس، ليس ليُفسد بالضعف البشري، بل ليضع في أذهان الناس مبادئ الحق التي لا يمكن أن تُمحى أبدًا، لأنها حقيقية أبدية. لقد جاء ليجلب حياة جديدة للبشر الساقطين - تميزًا لا يمكن أن تلوثه الخطية أو تدهوره.}[Lt253-1903.30; 1903][https://egwwritings.org/read?panels=p9980.38]


\egwnogap{\textbf{My brother, I must tell you that you have little realization of whither your feet have been tending}. You have been binding yourself up with those who belong to the army of the great apostate. \textbf{Your mind has been as dark as Egypt}. \textbf{If you will fall on the Rock and be broken}, Christ will accept you. But you have been standing on the enemy’s ground, doing his work. \textbf{The religious world is fast going over the same road that you have been following. If you continue to follow this road, you will have plenty of company. But what will the end be?}}[Lt253-1903.31; 1903][https://egwwritings.org/read?panels=p14068.9980039]


\egwnogap{\textbf{أخي، يجب أن أخبرك أن لديك إدراكًا ضئيلًا لأين كانت قدماك تتجهان}. لقد كنت تربط نفسك مع أولئك الذين ينتمون إلى جيش المرتد العظيم. \textbf{عقلك كان مظلمًا كمصر}. \textbf{إذا سقطت على الصخرة وانكسرت}، فإن المسيح سيقبلك. لكنك كنت تقف على أرض العدو، تقوم بعمله. \textbf{العالم الديني يسير بسرعة على نفس الطريق الذي كنت تتبعه. إذا استمررت في هذا الطريق، سيكون لديك الكثير من الرفقة. ولكن ماذا ستكون النهاية؟}}[Lt253-1903.31; 1903][https://egwwritings.org/read?panels=p14068.9980039]


\egwnogap{So long have you been walking in darkness, so long have you followed your own way, that you may be strongly tempted to resist this appeal that I make. If it were not that your \textbf{eternal interests are involved}, I would not speak to you on this subject. It would seem that I have written enough, that there is no need of my urging this subject upon you further. \textbf{But I tell you in truth that I clearly understand what I am doing}. Sufficient light has been given you. But for several years you have not heeded this light. If you had wished to know what the Lord has said, you could have known; \textbf{for you have the books that have been written under the guidance of His Spirit}. You have had all the directions that could be asked for to point out the right way. Direct light has been sent you. But you have looked upon this as of less importance than your own plans and devisings. If you had heeded the testimonies sent you, Living Temple would never have been written.}[Lt253-1903.32; 1903][https://egwwritings.org/read?panels=p9980.40]


\egwnogap{لقد سرت في الظلام لفترة طويلة، واتبعت طريقك الخاص لفترة طويلة، لدرجة أنك قد تُجرب بشدة لمقاومة هذا النداء الذي أقدمه. لولا أن \textbf{مصالحك الأبدية متورطة}، لما تحدثت إليك حول هذا الموضوع. قد يبدو أنني كتبت ما يكفي، وأنه لا حاجة لي لحثك على هذا الموضوع أكثر. \textbf{لكنني أقول لك بالحق أنني أفهم بوضوح ما أفعله}. لقد أُعطيت لك نورًا كافيًا. ولكن لعدة سنوات لم تلتفت إلى هذا النور. لو كنت ترغب في معرفة ما قاله الرب، لكان بإمكانك أن تعرف؛ \textbf{لأن لديك الكتب التي كُتبت تحت إرشاد روحه}. لقد تلقيت كل التوجيهات التي يمكن أن تُطلب لتشير إلى الطريق الصحيح. لقد أُرسل إليك نور مباشر. لكنك اعتبرت هذا أقل أهمية من خططك وتدبيراتك. لو كنت انتبهت إلى الشهادات المرسلة إليك، لما كُتب كتاب “ذا ليفينغ تمبل” أبدًا.}[Lt253-1903.32; 1903][https://egwwritings.org/read?panels=p9980.40]


\egwnogap{Will you not make a thorough, determined, Christlike effort to break the spell that Satan has cast over you? He has had great power over your mind and has swayed you in wrong lines. He thinks that he can hold you now. Will you not defeat and disappoint him?}[Lt253-1903.33; 1903][https://egwwritings.org/read?panels=p9980.41]


\egwnogap{ألن تبذل جهدًا شاملًا وحاسمًا على طريقة المسيح لكسر السحر الذي ألقاه الشيطان عليك؟ لقد كان له قوة عظيمة على عقلك وقد دفعك في اتجاهات خاطئة. إنه يعتقد أنه يستطيع أن يمسك بك الآن. ألن تهزمه وتخيب أمله؟}[Lt253-1903.33; 1903][https://egwwritings.org/read?panels=p9980.41]


\egwnogap{I write to you as I would to a son. Break away from the enemy—the accuser of the brethren. Say to him, “Get thee behind me Satan. I have committed a grievous sin in heeding your suggestions. I will no longer listen to them.” I beg of you, for your soul’s sake, to resist the tempter, that he may flee from you. Draw near to God, and He will draw near to you. \textbf{You will lose heaven unless you fall on the Rock and are broken}.}[Lt253-1903.34; 1903][https://egwwritings.org/read?panels=p9980.42]


\egwnogap{أكتب إليك كما لو كنت ابنًا. انفصل عن العدو - المشتكي على الإخوة. قل له: “اذهب عني يا شيطان. لقد ارتكبت خطيئة فادحة بالإصغاء إلى اقتراحاتك. لن أستمع إليها بعد الآن.” أتوسل إليك، من أجل خلاص نفسك، أن تقاوم المجرب، حتى يهرب منك. اقترب من الله، وهو سيقترب منك. \textbf{ستخسر السماء ما لم تسقط على الصخرة وتنكسر}.}[Lt253-1903.34; 1903][https://egwwritings.org/read?panels=p9980.42]


Many things in this letter to Dr. Kellogg go without being said, yet are explained when the context is understood. Ellen White read the letter from Brother Daniells expressing how Dr. Kellogg wanted to revise the Living Temple because he\others{had been thinking the matter over, and began to see that he had made a slight mistake in \textbf{expressing }his views}, and\others{that within a short time \textbf{he had come to believe in the trinity} and could now see pretty clearly where all the difficulty was, and believed that he could clear the matter up satisfactorily}. Kellogg confessed,\others{that he now believed \textbf{in God the Father, God the Son, and God the Holy Ghost}}. In answer to that, Sister White personally wrote to him:\egwinline{The book Living Temple \textbf{is not to be patched up}, a few changes made in it, and then advertised and praised as a valuable production}. How did Kellogg want to patch up his book? According to A. G. Daniells’ testimony, he thought to change a few expressions by explicitly stating his trinitarian sentiment. But the expression of the views was not the real problem—it was the views themselves. Sister White did not spare rebuking him for his views of God, which were \textit{trinitarian} views. She told him that she is\egwinline{\textbf{determined that \underline{no theories of the order that you have been accepting} shall come into our ranks}}. This is a very strong statement. Could it be that, since Kellogg confessed that he was accepting the Trinity doctrine, Sister White was also including it in her statement? It seems unthinkable because this doctrine is in our ranks today. But her statement actually pinpoints the Trinity when she said:\egwinline{\textbf{Patchwork theories} cannot be accepted by those who are loyal \textbf{to the faith and to the principles} that have withstood all the opposition of satanic influences}. Kellogg wanted to patch up “\textit{Living Temple}” by explicitly mentioning the Trinity doctrine. Why was Sister White determined to keep this doctrine out of our ranks, yet it is in our ranks today? It is fair to point out that the Trinity was not part of Seventh-day Adventist faith in her time and it came into our ranks later. Today, many argue that it was because of her works that the Trinity is a part of our beliefs, but Ellen White’s reaction, and her answer to Kellogg’s belief in it, showcases how she dealt with such doctrine. What can we learn from that?


هناك العديد من الأمور في هذه الرسالة إلى الدكتور كيلوغ تمر دون أن تُقال، ولكنها تُفسر عندما يُفهم السياق. قرأت إلين وايت الرسالة من الأخ دانيلز التي تعبر عن كيف أراد الدكتور كيلوغ مراجعة كتاب “ذا ليفينغ تمبل” لأنه\others{كان يفكر في الأمر، وبدأ يرى أنه ارتكب خطأً بسيطًا في \textbf{التعبير عن} آرائه}، وأنه\others{خلال فترة قصيرة \textbf{أصبح يؤمن بالثالوث} ويمكنه الآن أن يرى بوضوح أين كانت المشكلة، ويعتقد أنه يمكنه توضيح الأمر بشكل مُرضٍ}. اعترف كيلوغ،\others{أنه يؤمن الآن \textbf{بالله الآب، والله الابن، والله الروح القدس}}. ردًا على ذلك، كتبت الأخت وايت إليه شخصيًا:\egwinline{كتاب “ذا ليفينغ تمبل” \textbf{لا ينبغي ترقيعه}، وإجراء بعض التغييرات فيه، ثم الإعلان عنه والثناء عليه كإنتاج قيم}. كيف أراد كيلوغ ترقيع كتابه؟ وفقًا لشهادة أ. ج. دانيلز، فكر في تغيير بعض التعبيرات من خلال ذكر آرائه الثالوثية بشكل صريح. لكن التعبير عن الآراء لم يكن المشكلة الحقيقية - بل الآراء نفسها. لم تتردد الأخت وايت في توبيخه على آرائه عن الله، والتي كانت آراء \textit{ثالوثية}. أخبرته أنها\egwinline{\textbf{مصممة على أن \underline{لا تدخل نظريات من النوع الذي كنت تقبله} إلى صفوفنا}}. هذا بيان قوي جدًا. هل يمكن أن يكون، بما أن كيلوغ اعترف بأنه كان يقبل عقيدة الثالوث، أن الأخت وايت كانت تشمله أيضًا في بيانها؟ يبدو هذا أمرًا لا يمكن تصوره لأن هذه العقيدة موجودة في صفوفنا اليوم. لكن بيانها في الواقع يشير إلى الثالوث عندما قالت:\egwinline{\textbf{نظريات الترقيع} لا يمكن أن يقبلها أولئك الذين هم مخلصون \textbf{للإيمان وللمبادئ} التي صمدت أمام كل معارضة التأثيرات الشيطانية}. أراد كيلوغ ترقيع “ذا ليفينغ تمبل” من خلال ذكر عقيدة الثالوث بشكل صريح. لماذا كانت الأخت وايت مصممة على إبقاء هذه العقيدة خارج صفوفنا، ومع ذلك فهي في صفوفنا اليوم؟ من الإنصاف الإشارة إلى أن الثالوث لم يكن جزءًا من إيمان الأدفنتست السبتيين في زمنها وأنه دخل إلى صفوفنا لاحقًا. اليوم، يجادل الكثيرون بأنه بسبب أعمالها أصبح الثالوث جزءًا من معتقداتنا، لكن رد فعل إلين وايت، وإجابتها على إيمان كيلوغ به، توضح كيف تعاملت مع مثل هذه العقيدة. ماذا يمكننا أن نتعلم من ذلك؟


Taken in its context, this letter sheds new light on Kellogg’s controversy and demonstrates how we should deal with the Trinity doctrine. The first thing we question is why Sister White never used the word “Trinity” in her writings, even when she was directly dealing with this doctrine? Elsewhere, she answers:


إذا أخذناها في سياقها، تلقي هذه الرسالة ضوءًا جديدًا على صراع كيلوغ وتوضح كيف ينبغي أن نتعامل مع عقيدة الثالوث. أول ما نتساءل عنه هو لماذا لم تستخدم الأخت وايت كلمة “ثالوث” في كتاباتها، حتى عندما كانت تتعامل مباشرة مع هذه العقيدة؟ في مكان آخر، تجيب:


\egw{I was cautioned not to enter into controversy \textbf{regarding the question} that \textbf{\underline{will come up}} over \textbf{these things, because controversy \underline{might lead men to resort to subterfuges, and their minds would be led away from the truth of the Word of God to assumption and guesswork}}. \textbf{The more that fanciful theories are discussed, the \underline{less men will know of God and of the truth that sanctifies the soul}}.}[Lt232-1903.41; 1903][https://egwwritings.org/read?panels=p14068.10197050]


\egw{لقد حُذرت من الدخول في جدال \textbf{بخصوص المسألة} التي \textbf{\underline{ستظهر}} حول \textbf{هذه الأمور، لأن الجدال \underline{قد يؤدي بالناس إلى اللجوء إلى الحيل، وستبتعد أذهانهم عن حق كلمة الله إلى الافتراض والتخمين}}. \textbf{كلما نوقشت النظريات الخيالية أكثر، \underline{قل ما يعرفه الناس عن الله وعن الحق الذي يقدس النفس}}.}[Lt232-1903.41; 1903][https://egwwritings.org/read?panels=p14068.10197050]


This is a very important lesson and principle that Sister White is teaching us here. When the controversy over Kellogg’s theories arose, she did not venture into the theories themselves, because this would lead the minds of men away from the truth of the Word of God to assumption and guesswork. Rather, she led the minds of men into the truth, which sanctifies the soul. She led by example, evident here in her letter to Dr. Kellogg. This truth that she led the minds of men to, was the truth on the \emcap{personality of God}. She rebuked Kellogg for his theories but, very importantly, we properly identify these theories by their context and her implicit expression of them.


هذا درس ومبدأ مهم جدًا تعلمنا إياه الأخت وايت هنا. عندما نشأ الجدل حول نظريات كيلوغ، لم تخض في النظريات نفسها، لأن هذا سيبعد أذهان الناس عن حق كلمة الله إلى الافتراض والتخمين. بل قادت أذهان الناس إلى الحق، الذي يقدس النفس. قادت بالقدوة، وهذا واضح هنا في رسالتها إلى الدكتور كيلوغ. هذا الحق الذي قادت أذهان الناس إليه، كان الحق حول \emcap{شخصانية الله}. لقد وبخت كيلوغ على نظرياته ولكن، من المهم جدًا، أن نحدد هذه النظريات بشكل صحيح من خلال سياقها وتعبيرها الضمني عنها.


We see that she made a contrast between the Trinity and the \emcap{personality of God}. She made a contrast between the old principles of our faith and the new theories. First, she drew our minds back to the beginning of our spiritual heritage,\egwinline{after the passing of the time in 1844}, when her husband James White, Joseph Bates, Father Pierce, Elder Edson, and many others who were keen, noble, and true, searched for truth. She pointed back to the wonderful and mighty experiences of how the leading points of our faith, held in 1903, were firmly established. \egwinline{\textbf{Thus \underline{the leading points of our faith}} as we hold them today were firmly established.} \egwinline{\textbf{\underline{Point after point} was clearly defined, and all the brethren came into harmony}.} \egwinline{\textbf{The whole company of believers were united in the truth}}. Obviously, from the context of chapter 10 of the Special Testimonies, we know that these experiences explain \egwinline{\textbf{how firmly the foundation of our faith has been laid}}[SpTB02 56.4; 1904][https://egwwritings.org/read?panels=p417.288]. This foundation is expressed in the \emcap{Fundamental Principles}\footnote{\href{https://static1.squarespace.com/static/554c4998e4b04e89ea0c4073/t/59d17e24c027d84167e17617/1506901547915/SDA-YB1905+\%28P.+188-192\%29.pdf}{Yearbook Of Seventh-day Adventist denomination 1905, p. 188-192}}. This foundation is the truth which,\egwinline{\textbf{\underline{point by point}}, \textbf{has been sought out by prayerful study, and testified to by the miracle-working power of the Lord}}. God \egwinline{\textbf{calls upon us to \underline{hold firmly}, with the grip of faith, to \underline{the fundamental principles} that are \underline{based upon unquestionable authority}}.}[SpTB02 59.1; 1904][https://egwwritings.org/read?panels=p417.299] In light of these experiences and the truth expressed in the \emcap{fundamental principles}, \egwinline{\textbf{\underline{Patchwork theories} cannot be accepted by those who are loyal \underline{to the faith} and \underline{to the principles} that have withstood all the opposition of satanic influences}}[Lt253-1903.28; 1903][https://egwwritings.org/read?panels=p14068.9980036]. From the historical record of these brethren who were keen, noble and true, we have evidence that they, too, have contrasted the Trinity doctrine with the truth on the \emcap{personality of God}. James White, in the Review and Herald article, listed \others{some of the popular fables of the age}, saying: \others{Here we might mention \textbf{the Trinity, which \underline{does away the personality of God, and of his Son Jesus Christ}}}[James White, Review \& Herald, December 11, 1855, p. 85.15][http://documents.adventistarchives.org/Periodicals/RH/RH18551211-V07-11.pdf]. J. N. Andrews said, \others{\textbf{The doctrine of the Trinity which was established in the church by the council of Nicea, A. D. 325}. \textbf{This doctrine \underline{destroys the personality of God, and his Son Jesus Christ our Lord}}...}[J. N. Andrews, Review \& Herald, March 6, 1855, p. 185][http://documents.adventistarchives.org/Periodicals/RH/RH18550306-V06-24.pdf] J. B. Frisbie, in his article “\textit{Seventh-day Sabbath not abolished}”, compares the Sabbath God to the Sunday god; he describes the Sabbath God in light of the \emcap{personality of God} expressed in the first point of the \emcap{Fundamental Principles}. The Sunday god is described by the \others{unity of this God-head, there are three persons of one substance, power and eternity; the Father, the Son, and the Holy Ghost}[J. B. Frisbie, Review \& Herald March 7, 1854. p. 50][http://documents.adventistarchives.org/Periodicals/RH/RH18540307-V05-07.pdf]. He explained how the doctrine on the \emcap{personality of God} stands in conflict with the doctrine of Trinity, in the same way the Holy Sabbath stands in conflict with pagan Sunday worship. Also, brother J. N. Loughborough wrote the objections to the Trinity doctrine in the Adventist Review and Sabbath Herald\footnote{\href{https://adventistdigitallibrary.org/adl-349160/advent-review-and-sabbath-herald-november-5-1861}{J. N. Loughborough, November 5, 1861, Review \& Herald, vol. 18, p. 184, par. 1-11}}. In the other publication of the Review and Herald, he published the article “\textit{Is God a person?}”, explaining the position of Seventh-day Adventist belief on the \emcap{personality of God}, expressed in the first point of the \emcap{Fundamental Principles}\footnote{\href{http://documents.adventistarchives.org/Periodicals/RH/RH18550918-V07-06.pdf}{J. N. Loughborough, September 18. 1855, Review \& Herald, vol. 7, p. 6.}}. James White was also explaining the same position in his multiple print pamphlet, “\textit{The Personality of God}”\footnote{\href{https://egwwritings.org/?ref=en_PERGO.1.1&para=1471.3}{J. White, The Personality of God, June 18. 1861.}}. These are just a few examples where the Adventist pioneers explained the position on the \emcap{personality of God} expressed by the first point of the \emcap{fundamental principles}.


نرى أنها قامت بعمل مقارنة بين الثالوث و\emcap{شخصانية الله}. لقد قارنت بين المبادئ القديمة لإيماننا والنظريات الجديدة. أولاً، وجهت أذهاننا إلى بداية تراثنا الروحي،\egwinline{بعد مرور الوقت في عام 1844}، عندما كان زوجها جيمس وايت، وجوزيف بيتس، والأب بيرس، والشيخ إدسون، والعديد من الآخرين الذين كانوا حريصين، نبلاء، وصادقين، يبحثون عن الحقيقة. أشارت إلى التجارب الرائعة والقوية حول كيفية تأسيس النقاط الرئيسية لإيماننا، التي كانت موجودة في عام 1903، بشكل راسخ. \egwinline{\textbf{وهكذا تم تأسيس \underline{النقاط الرئيسية لإيماننا} كما نتمسك بها اليوم بشكل راسخ.}} \egwinline{\textbf{تم تحديد \underline{نقطة تلو الأخرى} بوضوح، وجميع الإخوة أصبحوا في انسجام.}} \egwinline{\textbf{كانت جماعة المؤمنين بأكملها متحدة في الحق}}. من الواضح، من سياق الفصل العاشر من الشهادات الخاصة، نعلم أن هذه التجارب توضح \egwinline{\textbf{مدى رسوخ أساس إيماننا}}[SpTB02 56.4; 1904][https://egwwritings.org/read?panels=p417.288]. يتم التعبير عن هذا الأساس في \emcap{المبادئ الأساسية}\footnote{\href{https://static1.squarespace.com/static/554c4998e4b04e89ea0c4073/t/59d17e24c027d84167e17617/1506901547915/SDA-YB1905+\%28P.+188-192\%29.pdf}{كتاب السنة لطائفة الأدفنتست السبتيين 1905، ص. 188-192}}. هذا الأساس هو الحق الذي،\egwinline{\textbf{\underline{نقطة بنقطة}، \textbf{تم البحث عنه من خلال الدراسة المصحوبة بالصلاة، وشهدت له قوة الرب العاملة بالمعجزات}}. الله \egwinline{\textbf{يدعونا إلى \underline{التمسك بقوة}، بقبضة الإيمان، \underline{بالمبادئ الأساسية} التي \underline{تستند إلى سلطة لا تقبل الشك}}.}[SpTB02 59.1; 1904][https://egwwritings.org/read?panels=p417.299] في ضوء هذه التجارب والحق المعبر عنه في \emcap{المبادئ الأساسية}، \egwinline{\textbf{لا يمكن قبول \underline{نظريات الترقيع} من قبل أولئك المخلصين \underline{للإيمان} و\underline{للمبادئ} التي صمدت أمام كل معارضة التأثيرات الشيطانية}}[Lt253-1903.28; 1903][https://egwwritings.org/read?panels=p14068.9980036]. من السجل التاريخي لهؤلاء الإخوة الذين كانوا حريصين ونبلاء وصادقين، لدينا دليل على أنهم، أيضًا، قارنوا عقيدة الثالوث مع الحق حول \emcap{شخصانية الله}. ذكر جيمس وايت، في مقال مجلة الريفيو آند هيرالد، \others{بعض الخرافات الشائعة في العصر}، قائلاً: \others{هنا يمكننا أن نذكر \textbf{الثالوث، الذي \underline{يلغي شخصانية الله، وابنه يسوع المسيح}}}[James White, Review \& Herald, December 11, 1855, p. 85.15][http://documents.adventistarchives.org/Periodicals/RH/RH18551211-V07-11.pdf]. قال ج. ن. أندروز، \others{\textbf{عقيدة الثالوث التي تأسست في الكنيسة من قبل مجلس نيقية، عام 325 م}. \textbf{هذه العقيدة \underline{تدمر شخصانية الله، وابنه يسوع المسيح ربنا}}...}[J. N. Andrews, Review \& Herald, March 6, 1855, p. 185][http://documents.adventistarchives.org/Periodicals/RH/RH18550306-V06-24.pdf] ج. ب. فريسبي، في مقالته “\textit{سبت اليوم السابع لم يُلغَ}”، يقارن إله السبت بإله الأحد؛ يصف إله السبت في ضوء \emcap{شخصانية الله} المعبر عنها في النقطة الأولى من \emcap{المبادئ الأساسية}. يتم وصف إله الأحد بـ \others{وحدة هذا الإله، هناك ثلاثة أشخاص من جوهر واحد، وقوة وأبدية؛ الآب، والابن، والروح القدس}[J. B. Frisbie, Review \& Herald March 7, 1854. p. 50][http://documents.adventistarchives.org/Periodicals/RH/RH18540307-V05-07.pdf]. شرح كيف أن العقيدة حول \emcap{شخصانية الله} تتعارض مع عقيدة الثالوث، بنفس الطريقة التي يتعارض بها السبت المقدس مع عبادة الأحد الوثنية. أيضًا، كتب الأخ ج. ن. لوبورو الاعتراضات على عقيدة الثالوث في مجلة أدفنتست ريفيو آند سباث هيرالد\footnote{\href{https://adventistdigitallibrary.org/adl-349160/advent-review-and-sabbath-herald-november-5-1861}{J. N. Loughborough, November 5, 1861, Review \& Herald, vol. 18, p. 184, par. 1-11}}. في منشور آخر من مجلة ريفيو آند هيرالد، نشر مقالة “\textit{هل الله شخص؟}”، موضحًا موقف معتقد الأدفنتست السبتيين حول \emcap{شخصانية الله}، المعبر عنه في النقطة الأولى من \emcap{المبادئ الأساسية}\footnote{\href{http://documents.adventistarchives.org/Periodicals/RH/RH18550918-V07-06.pdf}{J. N. Loughborough, September 18. 1855, Review \& Herald, vol. 7, p. 6.}}. كان جيمس وايت أيضًا يشرح نفس الموقف في كتيبه المطبوع المتعدد، “\textit{شخصانية الله}”\footnote{\href{https://egwwritings.org/?ref=en_PERGO.1.1&para=1471.3}{J. White, The Personality of God, June 18. 1861.}}. هذه مجرد أمثلة قليلة حيث شرح رواد الأدفنتست الموقف حول \emcap{شخصانية الله} المعبر عنه في النقطة الأولى من \emcap{المبادئ الأساسية}.


Sister White rebuked Kellogg:\egwinline{\textbf{But I tell you in truth that I clearly understand what I am doing}. \textbf{Sufficient light has been given you}. But for several years you have not heeded this light. If you had wished to know what the Lord has said, you could have known; \textbf{for \underline{you have the books} that have been written under the guidance of His Spirit}. You have had all the directions that could be asked for to point out the right way. Direct light has been sent you. But you have looked upon this as of less importance than your own plans and devisings. If you had heeded the testimonies sent you, Living Temple would never have been written.}[Lt253-1903.32; 1903][https://egwwritings.org/read?panels=p14068.9980040] The core issue of Dr. Kellogg’s controversy was \egwinline{the personality of God and where His presence is}[SpTB02 51.3; 1904][https://egwwritings.org/read?panels=p417.262]. Dr. Kellogg had access to the pioneer writings, books and the church's \emcap{Fundamental Principles} that were testified to by the miracle working power of the Holy Spirit.


وبخت الأخت وايت كيلوغ:\egwinline{\textbf{لكنني أقول لك بالحق أنني أفهم بوضوح ما أفعله}. \textbf{لقد أُعطيت نورًا كافيًا}. ولكن لعدة سنوات لم تلتفت إلى هذا النور. إذا كنت ترغب في معرفة ما قاله الرب، كان بإمكانك أن تعرف؛ \textbf{لأن \underline{لديك الكتب} التي كُتبت تحت إرشاد روحه}. لقد حصلت على كل التوجيهات التي يمكن أن تُطلب لتشير إلى الطريق الصحيح. تم إرسال نور مباشر إليك. لكنك اعتبرت هذا أقل أهمية من خططك وتدابيرك. لو كنت قد انتبهت إلى الشهادات المرسلة إليك، لما كُتب كتاب ليفينغ تمبل أبدًا.}[Lt253-1903.32; 1903][https://egwwritings.org/read?panels=p14068.9980040] كانت القضية الأساسية في صراع الدكتور كيلوغ هي \egwinline{شخصانية الله وأين يوجد حضوره}[SpTB02 51.3; 1904][https://egwwritings.org/read?panels=p417.262]. كان الدكتور كيلوغ لديه إمكانية الوصول إلى كتابات الرواد والكتب و\emcap{المبادئ الأساسية} للكنيسة التي شهدت لها قوة الروح القدس العاملة بالمعجزات.


Sister White recalled the experiences of how the \textit{leading points of our faith}, as were held in former times, were firmly established.\egwinline{\textbf{\underline{Point after point} was clearly defined, and all the brethren came into harmony}}[Lt253-1903.4; 1903][https://egwwritings.org/read?panels=p14068.9980010]. These leading points were the \emcap{Fundamental Principles}, of which the \emcap{personality of God} was one. This point, and Sister White’s testimony of it, remained the same during the course of her life.  She said\egwinline{\textbf{\underline{I have ever had the same testimony to bear which I now bear regarding the personality of God}}}[Lt253-1903.9; 1903][https://egwwritings.org/read?panels=p14068.9980015]. From Early Writings, she then quoted her visions of the Heavenly reality. She recalled how she had had the privilege to be in the presence of God, how God, encircled by the light of His glory, passed by her side. She did not see God from the light He was encircled by; she was afraid of Him, thinking that if He were to approach her she\egwinline{would be struck out of existence}. Then she saw\egwinline{\textbf{the lovely Jesus, that He is a person}. \textbf{I asked Him if His Father was a person, and had \underline{a form like} Himself}. Said Jesus, ‘\textbf{I am the express image of My Father’s person!}’}[Lt253-1903.12; 1903][https://egwwritings.org/read?panels=p14068.9980018]. The question she had was: \textit{is God a person, having a form like Jesus}? The answer was affirmative—with a strong biblical foundation. Her visions were not the source of the truth on the \emcap{personality of God}; rather, they confirmed the truth the pioneers had discovered through diligent study of God’s word.


تذكرت الأخت وايت تجارب كيف تم تأسيس \textit{النقاط الرئيسية لإيماننا}، كما كانت في الأزمنة السابقة، بشكل راسخ.\egwinline{\textbf{تم تحديد \underline{نقطة تلو الأخرى} بوضوح، وجميع الإخوة أصبحوا في انسجام}}[Lt253-1903.4; 1903][https://egwwritings.org/read?panels=p14068.9980010]. هذه النقاط الرئيسية كانت \emcap{المبادئ الأساسية}، التي كانت \emcap{شخصانية الله} واحدة منها. هذه النقطة، وشهادة الأخت وايت عنها، ظلت كما هي خلال حياتها. قالت\egwinline{\textbf{كانت لدي دائمًا \underline{نفس الشهادة} التي أقدمها الآن بخصوص شخصانية الله}}[Lt253-1903.9; 1903][https://egwwritings.org/read?panels=p14068.9980015]. من كتاب “الكتابات المبكرة”، اقتبست رؤاها عن الواقع السماوي. تذكرت كيف كان لها امتياز أن تكون في حضرة الله، كيف أن الله، محاطًا بنور مجده، مر بجانبها. لم تر الله من النور الذي كان محاطًا به؛ كانت خائفة منه، معتقدة أنه إذا اقترب منها\egwinline{ستُضرب من الوجود}. ثم رأت\egwinline{\textbf{يسوع المحبوب، أنه شخص}. \textbf{سألته إذا كان أبوه شخصًا، وله \underline{شكل مثله}}. قال يسوع، ‘\textbf{أنا صورة جوهره!}’}[Lt253-1903.12; 1903][https://egwwritings.org/read?panels=p14068.9980018]. السؤال الذي كان لديها هو: \textit{هل الله شخص، له شكل مثل يسوع}؟ كانت الإجابة إيجابية - مع أساس كتابي قوي. لم تكن رؤاها مصدر الحق حول \emcap{شخصانية الله}؛ بل أكدت الحق الذي اكتشفه الرواد من خلال الدراسة الجادة لكلمة الله.


Therefore, their final conclusion on the \emcap{personality of God} was,\others{That there is \textbf{one God}, \textbf{a personal, spiritual \underline{being}}, \textbf{the creator of all things}, omnipotent, omniscient, and eternal, infinite in wisdom, holiness, justice, goodness, truth, and mercy; unchangeable, and \textbf{everywhere present by his representative, the Holy Spirit}. Ps. 139:7; That there is one Lord Jesus Christ, \textbf{the Son of the Eternal Father, the one by whom he created all things, and by whom they do consist} …and as the closing portion of his work as priest, before he takes his throne as king, he will make \textbf{the great atonement} for the sins of all such, and their sins will then be blotted out (Acts 3:19) and borne away from the sanctuary, as shown in the service of the Levitical priesthood, which foreshadowed and prefigured the ministry of our Lord in heaven. See Lev. 16; Heb. 8: 4, 5; 9: 6, 7; etc.}[The first, and part of the second, point of the Fundamental Principles, 1905.]


لذلك، كان استنتاجهم النهائي حول \emcap{شخصانية الله} هو،\others{أن هناك \textbf{إله واحد}، \textbf{\underline{كائن} روحي شخصي}، \textbf{خالق كل الأشياء}، قدير، عليم، وأبدي، لا نهائي في الحكمة، والقداسة، والعدل، والصلاح، والحق، والرحمة؛ لا يتغير، و\textbf{موجود في كل مكان بواسطة ممثله، الروح القدس}. مز 139: 7؛ أن هناك رب واحد يسوع المسيح، \textbf{ابن الآب الأبدي، الذي به خلق كل الأشياء، وبه تقوم} ... وكجزء ختامي من عمله ككاهن، قبل أن يأخذ عرشه كملك، سيقوم \textbf{بالكفارة العظيمة} عن خطايا كل هؤلاء، وستُمحى خطاياهم حينئذ (أعمال 3: 19) وتُحمل بعيدًا عن المقدس، كما هو موضح في خدمة الكهنوت اللاوي، الذي سبق وصور خدمة ربنا في السماء. انظر لاويين 16؛ عبرانيين 8: 4، 5؛ 9: 6، 7؛ إلخ.}[النقطة الأولى، وجزء من النقطة الثانية، من المبادئ الأساسية، 1905.]


Ellen White reminded Dr. Kellogg on this point of the \emcap{fundamental principles} by stating:\egwinline{\textbf{The Father, the omniscient One, created the world \underline{through} Christ Jesus}. Christ is the light of the world, the way to eternal life. \textbf{He, the anointed One, God gave to make an atonement for the sins of the world}...}[Lt253-1903.21; 1903][https://egwwritings.org/read?panels=p14068.9980029]


ذكّرت إلين وايت الدكتور كيلوغ بهذه النقطة من \emcap{المبادئ الأساسية} بقولها:\egwinline{\textbf{الآب، العليم، خلق العالم \underline{من خلال} المسيح يسوع}. المسيح هو نور العالم، الطريق إلى الحياة الأبدية. \textbf{هو، الممسوح، أعطاه الله ليقدم كفارة عن خطايا العالم}...}[Lt253-1903.21; 1903][https://egwwritings.org/read?panels=p14068.9980029]


The question on the \emcap{personality of God} deals with the quality or state of God being a person. The Adventist pioneers gave an answer to it and God approved it through the writings of Ellen White: God is a \textit{personal spiritual Being} and He is our heavenly Father. Where is His presence?\egwinline{\textbf{We are not to say that the Lord God of heaven is in a leaf, or in a tree; for He is not there. \underline{He sitteth upon His throne in the heavens}}.}[Lt253-1903.15; 1903][https://egwwritings.org/read?panels=p14068.9980022] \\
His presence is on the throne in heaven. \\
\egwinline{\textbf{Heaven is not a vapor. It is a place}. \textbf{Christ has gone to prepare mansions for those who love Him}, those who, in obedience to His commands, come out from the world and are separate...}[EGW, Lt253-1903.25; 1903][https://egwwritings.org/read?panels=p14068.9980033]. \\
“...\egwinline{‘The voice of the Lord is mighty; it shaketh the cedars of Lebanon. \textbf{The Lord is in His holy temple}; let all the earth keep silence before Him.’ [See Psalm 29:5; Habakkuk 2:20.]}[Lt253-1903.18; 1903][https://egwwritings.org/read?panels=p14068.9980026]


السؤال حول \emcap{شخصانية الله} يتعلق بالصفة أو الحالة التي يكون بها الله شخصًا. قدم رواد الأدفنتست إجابة لذلك وأقرها الله من خلال كتابات إلين وايت: الله هو \textit{كائن روحي شخصي} وهو أبونا السماوي. أين يوجد حضوره؟\egwinline{\textbf{لا يجب أن نقول إن الرب إله السماء موجود في ورقة، أو في شجرة؛ لأنه ليس هناك. \underline{إنه يجلس على عرشه في السماوات}}.}[Lt253-1903.15; 1903][https://egwwritings.org/read?panels=p14068.9980022] \\
حضوره على العرش في السماء. \\
\egwinline{\textbf{السماء ليست بخارًا. إنها مكان}. \textbf{ذهب المسيح ليعد منازل لأولئك الذين يحبونه}، أولئك الذين، في طاعة لوصاياه، يخرجون من العالم ويكونون منفصلين...}[EGW, Lt253-1903.25; 1903][https://egwwritings.org/read?panels=p14068.9980033]. \\
“...\egwinline{‘صوت الرب قوي؛ يهز أرز لبنان. \textbf{الرب في هيكله المقدس}؛ فلتصمت كل الأرض أمامه.’ [انظر مزمور 29: 5؛ حبقوق 2: 20.]}[Lt253-1903.18; 1903][https://egwwritings.org/read?panels=p14068.9980026]


According to Adventist pioneers and Sister White, our heavenly Father is one God. He is a personal Spiritual Being, present in heaven, on His throne. The throne of heaven is a real, physical throne, upon which sits a real Person (Being, having a form, just like Jesus)—our heavenly Father. That place is a real place; it is not a vapor, or any other spiritual view.


وفقًا لرواد الأدفنتست والأخت وايت، أبونا السماوي هو إله واحد. هو كائن روحي شخصي، موجود في السماء، على عرشه. عرش السماء هو عرش حقيقي، مادي، يجلس عليه شخص حقيقي (كائن، له شكل، تمامًا مثل يسوع) - أبونا السماوي. ذلك المكان هو مكان حقيقي؛ ليس بخارًا، أو أي نظرة روحانية أخرى.


\egwinline{\textbf{I have often seen that the spiritual view took away all the glory of heaven, and that in many minds the throne of David and the lovely person of Jesus have been burned up in the fire of spiritualism}. I have seen that some who have been deceived and led into this error, will be brought out into the light of truth, \textbf{but it will be almost impossible for them to get entirely rid of the deceptive power of spiritualism. Such should make thorough work in confessing their errors, and leaving them forever}.}[Lt253-1903.13; 1903][https://egwwritings.org/read?panels=p14068.9980019]


\egwinline{\textbf{لقد رأيت غالبًا أن النظرة الروحانية أزالت كل مجد السماء، وأنه في أذهان كثيرة تم حرق عرش داود وشخص يسوع المحبوب في نار الروحانية}. لقد رأيت أن بعض الذين خُدعوا وقادوا إلى هذا الخطأ، سيُخرجون إلى نور الحق، \textbf{لكن سيكون من شبه المستحيل عليهم التخلص تمامًا من القوة المخادعة للروحانية. على هؤلاء أن يقوموا بعمل شامل في الاعتراف بأخطائهم، وتركها إلى الأبد}.}[Lt253-1903.13; 1903][https://egwwritings.org/read?panels=p14068.9980019]


The spiritual view of God’s person is an erroneous view. In the Bible we have testimonies of heaven, the heavenly throne, and God who is sitting upon it. If we accept these testimonies in their obvious meaning, then the Trinity doctrine cannot be sustained. The Bible and Spirit of Prophecy present one God in heaven, as a personal being, having a body and form just as Jesus has. This view is not in harmony with the doctrine of the Triune God, since it requires the Holy Spirit to be a Being\footnote{Please look at \hyperref[appendix:unauthenticated-reports]{the appendix} for more quotations which exclude the Holy Spirit to be a being, possessing physical body and form.}, having a body and form—this idea would compromise the Holy Spirit to be a means of the Father and Son by which They are everywhere present. In order to sustain the Trinity doctrine, the testimonies regarding the throne of God and of God’s person, need to be understood by some spiritual view. Here we have seen that Sister White contrasted the truth of the \emcap{personality of God} with the doctrine of Trinity. She contrasted the doctrine of Trinity with the first two points of the \emcap{Fundamental Principles}, which were the results of our pioneers studying the Word of God. Referring to the pioneers and the \emcap{Fundamental Principles}, she said: \egwinline{\textbf{\underline{Patchwork theories} cannot be accepted by those who are \underline{loyal to the faith and to the principles} that have withstood all the opposition of satanic influences.}}[Lt253-1903.28; 1903][https://egwwritings.org/read?panels=p14068.9980036]


النظرة الروحانية لشخص الله هي نظرة خاطئة. في الكتاب المقدس لدينا شهادات عن السماء، والعرش السماوي، والله الذي يجلس عليه. إذا قبلنا هذه الشهادات بمعناها الواضح، فلا يمكن دعم عقيدة الثالوث. يقدم الكتاب المقدس وروح النبوة إلهًا واحدًا في السماء، ككائن شخصي، له جسد وشكل تمامًا كما ليسوع. هذه النظرة ليست متوافقة مع عقيدة الإله الثالوثي، لأنها تتطلب أن يكون الروح القدس كائنًا\footnote{يرجى النظر إلى \hyperref[appendix:unauthenticated-reports]{الملحق} للمزيد من الاقتباسات التي تستبعد أن يكون الروح القدس كائنًا، يمتلك جسدًا ماديًا وشكلًا.}، له جسد وشكل - هذه الفكرة من شأنها أن تضر بكون الروح القدس وسيلة للآب والابن التي بها يكونان موجودين في كل مكان. من أجل دعم عقيدة الثالوث، يجب فهم الشهادات المتعلقة بعرش الله وشخص الله، من خلال بعض النظرة الروحانية. هنا رأينا أن الأخت وايت قارنت حق \emcap{شخصانية الله} مع عقيدة الثالوث. قارنت عقيدة الثالوث مع النقطتين الأوليين من \emcap{المبادئ الأساسية}، اللتين كانتا نتيجة دراسة روادنا لكلمة الله. مشيرة إلى الرواد و\emcap{المبادئ الأساسية}، قالت: \egwinline{\textbf{لا يمكن قبول \underline{نظريات الترقيع} من قبل أولئك \underline{المخلصين للإيمان وللمبادئ} التي صمدت أمام كل معارضة التأثيرات الشيطانية.}}[Lt253-1903.28; 1903][https://egwwritings.org/read?panels=p14068.9980036]


The conclusion is straightforward and simple. Those who are loyal to the faith, and to the principles received in the beginning of the work, cannot accept patchwork theories. Put into context, the patchwork theory, which is the Trinity doctrine, cannot be accepted by those who are holding fast \egwinline{\textbf{to \underline{the fundamental principles} that are \underline{based upon unquestionable authority}}}[SpTB02 59.1; 1904][https://egwwritings.org/read?panels=p417.299]. This conclusion leads us back to our first proposed test of the foundation of our faith.


الاستنتاج واضح وبسيط. أولئك المخلصون للإيمان، وللمبادئ التي تم تلقيها في بداية العمل، لا يمكنهم قبول نظريات الترقيع. في السياق، نظرية الترقيع، التي هي عقيدة الثالوث، لا يمكن قبولها من قبل أولئك الذين يتمسكون بقوة \egwinline{\textbf{\underline{بالمبادئ الأساسية} التي \underline{تستند إلى سلطة لا تقبل الشك}}}[SpTB02 59.1; 1904][https://egwwritings.org/read?panels=p417.299]. هذا الاستنتاج يعيدنا إلى اختبارنا الأول المقترح لأساس إيماننا.


% The Patchwork Theories

\begin{titledpoem}
    
    \stanza{
        Truth established through earnest prayer, \\
        Points of faith discovered with care. \\
        Principles tested by time and trial, \\
        Stand firm against Satan's denial.
    }

    \stanza{
        Patchwork theories seek to sway, \\
        Those from the ancient, proven way. \\
        No revisions of truth we'll accept, \\
        The faithful path must be kept.
    }

    \stanza{
        The personality of God, a sacred revelation, \\
        Not subject to human innovation. \\
        Loyal hearts stand on ground that's sure, \\
        Where foundations eternal endure.
    }
    
\end{titledpoem}