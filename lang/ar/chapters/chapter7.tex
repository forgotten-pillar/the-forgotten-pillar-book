\qrchapter{https://forgottenpillar.com/rsc/en-fp-chapter7}{The authority of the Fundamental Principles} \label{chap:authority}


\qrchapter{https://forgottenpillar.com/rsc/en-fp-chapter7}{سلطة المبادئ الأساسية} \label{chap:authority}


In the 10th chapter of the Special Testimonies, we read how God established the foundation of our faith. Sister White used several different expressions for the foundation of our faith. Her references included: “\textit{a platform of eternal truth}, \textit{“pillars of our faith”}, \textit{“principles of truth”}, \textit{“principal points”}, \textit{“waymarks”}, and “\textit{foundation principles}—all of these refer to the \emcap{Fundamental Principles}. At the end of the chapter, she affirmed the will of God that \egwinline{He calls upon us to hold firmly, with the grip of faith, to \textbf{the fundamental principles} that are \textbf{based upon unquestionable \underline{authority}}.}[SpTB02 59.1; 1904][https://egwwritings.org/read?panels=p417.299]


في الفصل العاشر من الشهادات الخاصة، نقرأ كيف أسس الله أساس إيماننا. استخدمت الأخت وايت عدة تعبيرات مختلفة لأساس إيماننا. تضمنت إشاراتها: “\textit{منصة الحق الأبدي}، \textit{“أعمدة إيماننا”}، \textit{“مبادئ الحق”}، \textit{“النقاط الرئيسية”}، \textit{“معالم”}، و”\textit{المبادئ الأساسية}—كل هذه تشير إلى \emcap{المبادئ الأساسية}. في نهاية الفصل، أكدت إرادة الله بأن \egwinline{يدعونا للتمسك بقوة، بقبضة الإيمان، بـ \textbf{المبادئ الأساسية} التي \textbf{تستند إلى \underline{سلطة} لا تقبل الشك}.}[SpTB02 59.1; 1904][https://egwwritings.org/read?panels=p417.299]


The authority on which the \emcap{fundamental principles} are established is unquestionable. They were the result of deep, earnest study in the time of great disappointment, when \egwinline{\textbf{\underline{point by point}, has been sought out by prayerful study, and testified to by the \underline{miracle-working power of the Lord}}}\footnote{Ibid.}. \egwinline{\textbf{Thus \underline{the leading points of our faith} as we hold them today were firmly established}. \textbf{\underline{Point after point} was clearly defined, and all the brethren came into harmony}.}[Lt253-1903.4; 1903][https://egwwritings.org/read?panels=p14068.9980010]


السلطة التي تأسست عليها \emcap{المبادئ الأساسية} لا تقبل الشك. كانت نتيجة دراسة عميقة وجادة في وقت خيبة الأمل الكبيرة، عندما \egwinline{\textbf{\underline{نقطة بنقطة}، تم البحث عنها بالدراسة المصحوبة بالصلاة، وشهدت عليها \underline{قوة الرب العاملة للمعجزات}}}\footnote{المرجع نفسه}. \egwinline{\textbf{وهكذا تم تأسيس \underline{النقاط الرئيسية لإيماننا} كما نتمسك بها اليوم بثبات}. \textbf{\underline{نقطة بعد نقطة} تم تحديدها بوضوح، وجميع الإخوة أصبحوا في انسجام}.}[Lt253-1903.4; 1903][https://egwwritings.org/read?panels=p14068.9980010]


They were the result of the earnest Bible studies of our pioneers, after the passing of time in 1844. As the Seventh-day Adventist movement progressed, there came a need for instituting the organization, which was realized in 1863. In 1872, the Seventh-day Adventist Church issued the document called “\textit{A Declaration of the Fundamental Principles, Taught and Practiced by the Seventh-Day Adventists}. This was the first written document declaring the \emcap{fundamental principles} as public statements of the Seventh-day Adventist faith. This document was the public synopsis of Seventh-day Adventist faith and it declared \others{what is, and has been, with great unanimity, held by} the Seventh-day Adventist people. It was written \others{to meet inquiries} as to what was believed by Seventh-day Adventists, \others{to correct false statements circulated} and to \others{remove erroneous impressions}[FP1872 3.1; 1872][https://egwwritings.org/read?panels=p928.8].


كانت نتيجة دراسات الكتاب المقدس الجادة لروادنا، بعد مرور الوقت في عام 1844. مع تقدم حركة الأدفنتست السبتيين، ظهرت الحاجة لتأسيس التنظيم، الذي تحقق في عام 1863. في عام 1872، أصدرت كنيسة الأدفنتست السبتيين وثيقة تسمى “\textit{إعلان المبادئ الأساسية التي يعلمها ويمارسها الأدفنتست السبتيون}. كانت هذه أول وثيقة مكتوبة تعلن \emcap{المبادئ الأساسية} كبيانات عامة لإيمان الأدفنتست السبتيين. كانت هذه الوثيقة ملخصًا عامًا لإيمان الأدفنتست السبتيين وأعلنت \others{ما هو، وما كان، بإجماع كبير، يتمسك به} شعب الأدفنتست السبتيين. كُتبت \others{لتلبية الاستفسارات} حول ما كان يؤمن به الأدفنتست السبتيون، \others{لتصحيح البيانات الكاذبة المتداولة} و\others{لإزالة الانطباعات الخاطئة}[FP1872 3.1; 1872][https://egwwritings.org/read?panels=p928.8].


Today it is still debated who authored the synopsis because originally, in 1872, it was left anonymous. In 1874, James White issued it in Signs of the Times\footnote{\href{https://adventistdigitallibrary.org/adl-364148/signs-times-june-4-1874}{Signs of the Times, June 4, 1874}} and Uriah Smith in the Review and Herald\footnote{\href{http://documents.adventistarchives.org/Periodicals/RH/RH18741124-V44-22.pdf}{The Advent Review and Herald of the Sabbath, November 24, 1874}}—both signing with their own signatures. In 1889, Uriah Smith revised it by adding three points; it was issued in the Adventist Yearbook with his signature on it. Uriah Smith died in 1903 and all successive printings of the \emcap{Fundamental Principles} were printed under his name. They were printed in the Yearbooks—each year from 1905 until 1914\footnote{For more detailed timeline of Fundamental Principles, see \hyperref[appendix:timeline]{Appendix: Fundamental Principles - Timeline}}. Sister White died in 1915 and, for the next 17 years, the \emcap{fundamental principles} were not printed. Their next appearance was in the 1931 Yearbook when they received significant changes.


لا يزال هناك جدل اليوم حول من ألف الملخص لأنه في الأصل، في عام 1872، تُرك مجهول المؤلف. في عام 1874، أصدره جيمس وايت في ساينز أوف ذا تايمز\footnote{\href{https://adventistdigitallibrary.org/adl-364148/signs-times-june-4-1874}{ساينز أوف ذا تايمز، 4 يونيو 1874}} وأصدره يوريا سميث في ريفيو آند هيرالد\footnote{\href{http://documents.adventistarchives.org/Periodicals/RH/RH18741124-V44-22.pdf}{ذا أدفنت ريفيو آند هيرالد أوف ذا ساباث، 24 نوفمبر 1874}}—كلاهما وقع بتوقيعه الخاص. في عام 1889، قام يوريا سميث بمراجعتها بإضافة ثلاث نقاط؛ وتم إصدارها في كتاب السنة الأدفنتستي بتوقيعه عليها. توفي يوريا سميث في عام 1903 وتمت طباعة جميع الطبعات اللاحقة من \emcap{المبادئ الأساسية} تحت اسمه. تمت طباعتها في كتب السنة—كل عام من 1905 حتى 1914\footnote{لمزيد من التفاصيل حول الجدول الزمني للمبادئ الأساسية، انظر \hyperref[appendix:timeline]{الملحق: المبادئ الأساسية - الجدول الزمني}}. توفيت الأخت وايت في عام 1915، ولمدة 17 عامًا التالية، لم تتم طباعة \emcap{المبادئ الأساسية}. ظهورها التالي كان في كتاب السنة 1931 عندما شهدت تغييرات كبيرة.


In 1971, LeRoy Froom wrote about a statement from 1872: \others{Though appearing anonymously, it was actually composed by Smith}[Edwin Froom, LeRoy. Movement of Destiny. 1971., p. 160]. Unfortunately, he didn’t provide any data to support his claim. It is unfortunate to see how pro-trinitarian scholars consider the \emcap{Fundamental Principles} to be of very little importance. Their true value is starkly diminished by attributing these beliefs to those of a small group of people, mostly to James White’s or Uriah Smith’s personal belief, rather than belief which was \others{with great unanimity, held by}[Preface of the Fundamental Principles 1872] the Seventh-day Adventist people. In 1958, Ministry Magazine described the \emcap{Fundamental Principles} as follows:


في عام 1971، كتب ليروي فروم عن بيان من عام 1872: \others{على الرغم من ظهوره مجهول المؤلف، فقد تم تأليفه فعليًا بواسطة سميث}[إدوين فروم، ليروي. حركة المصير. 1971، ص. 160]. للأسف، لم يقدم أي بيانات لدعم ادعائه. من المؤسف أن نرى كيف يعتبر العلماء المؤيدون للثالوث \emcap{المبادئ الأساسية} ذات أهمية ضئيلة جدًا. تم تقليل قيمتها الحقيقية بشكل صارخ من خلال نسب هذه المعتقدات إلى مجموعة صغيرة من الناس، معظمهم إلى المعتقد الشخصي لجيمس وايت أو يوريا سميث، بدلاً من المعتقد الذي كان \others{بإجماع كبير، يتمسك به}[مقدمة المبادئ الأساسية 1872] شعب الأدفنتست السبتيين. في عام 1958، وصفت مجلة مينيستري \emcap{المبادئ الأساسية} على النحو التالي:


\others{It is true that in 1872 a ‘Declaration of the Fundamental Principles Taught and Practiced by Seventhday Adventists’ was printed, \textbf{but it was never adopted by the denomination and therefore cannot be considered official}. Evidently a small group, \textbf{perhaps even one or two, endeavored to put into words what they thought were the views of the entire church…}}[Ministry Magazine “\textit{Our Declaration of Fundamental Beliefs}”, January 1958, Roy Anderson, J. Arthur Buckwalter, Louise Kleuser, Earl Cleveland and Walter Schubert]


\others{صحيح أنه في عام 1872 تمت طباعة ‘إعلان المبادئ الأساسية التي يعلمها ويمارسها الأدفنتست السبتيون’، \textbf{لكنه لم يتم تبنيه أبدًا من قبل الطائفة وبالتالي لا يمكن اعتباره رسميًا}. من الواضح أن مجموعة صغيرة، \textbf{ربما حتى واحد أو اثنين، سعوا لوضع في كلمات ما اعتقدوا أنها كانت آراء الكنيسة بأكملها...}}[مجلة مينيستري “\textit{إعلاننا للمعتقدات الأساسية}”، يناير 1958، روي أندرسون، ج. آرثر باكوولتر، لويز كليوسر، إيرل كليفلاند ووالتر شوبرت]


Problematically, there is no evidence to support the claim that the \emcap{Fundamental Principles} were not the representation of faith of the whole body. We certainly know that Sister White endorsed them and, from her influence alone, we know that these beliefs were indeed accepted by the denomination—this is in addition to the fact that they were printed multiple times over the course of 42 years, during the life of Ellen White.


المشكلة أنه لا يوجد دليل يدعم الادعاء بأن \emcap{المبادئ الأساسية} لم تكن تمثل إيمان الجسد بأكمله. نحن نعلم بالتأكيد أن الأخت وايت أيدتها، ومن تأثيرها وحده، نعلم أن هذه المعتقدات كانت بالفعل مقبولة من قبل الطائفة—هذا بالإضافة إلى حقيقة أنها طُبعت عدة مرات على مدار 42 عامًا، خلال حياة إلين وايت.


But there should be no controversy over the authorship of the \emcap{Fundamental Principles}. We have a quotation from Sister White about who authored them. When speaking of Uriah Smith, Sister White wrote:


لكن لا ينبغي أن يكون هناك جدل حول تأليف \emcap{المبادئ الأساسية}. لدينا اقتباس من الأخت وايت حول من ألفها. عندما تحدثت عن يوريا سميث، كتبت الأخت وايت:


\egw{\textbf{Brother Smith was with us in the rise of this work. He understands how \underline{we—my husband and myself}—have carried the work forward and upward step by step and have borne the hardships, the poverty, and the want of means. With us were those early workers. Elder Smith, especially, was one with my husband in his early manhood}. …}[Ms54-1890.6; 1890][https://egwwritings.org/read?panels=p7213.15]


\egw{\textbf{كان الأخ سميث معنا في بداية هذا العمل. يفهم كيف \underline{نحن—زوجي وأنا}—قمنا بدفع العمل إلى الأمام والأعلى خطوة بخطوة وتحملنا المشقات والفقر ونقص الوسائل. كان معنا أولئك العمال الأوائل. الشيخ سميث، على وجه الخصوص، كان واحدًا مع زوجي في شبابه المبكر}. ...}[Ms54-1890.6; 1890][https://egwwritings.org/read?panels=p7213.15]


\egwnogap{\textbf{\underline{We have stood shoulder to shoulder with Elder Smith in this work while the Lord was laying the foundation principles}}. \textbf{We had to work constantly against one-idea men}, who thought correct business relations in regard to the work which had to be done were an evidence of worldly-mindedness, and the cranky ones who would present themselves as capable of bearing responsibilities, but could not be trusted to be connected with the work lest they swing it in wrong lines. \textbf{Step after step has had to be taken, \underline{not after the wisdom of men} but after the wisdom and instruction of One who is too wise to err and too good to do us harm}. \textbf{There have been so many elements that would have to be proved and tried. I thank the Lord that Elders Smith, Amadon, and Batchellor still live. They composed the members of our family in the most trying parts of our history}.}[Ms54-1890.7; 1890][https://egwwritings.org/read?panels=p7213.16]


\egwnogap{\textbf{\underline{لقد وقفنا جنبًا إلى جنب مع الشيخ سميث في هذا العمل بينما كان الرب يضع المبادئ الأساسية}}. \textbf{كان علينا أن نعمل باستمرار ضد الرجال ذوي الفكرة الواحدة}، الذين اعتقدوا أن العلاقات التجارية الصحيحة فيما يتعلق بالعمل الذي كان يجب القيام به كانت دليلاً على العقلية الدنيوية، والأشخاص المتقلبون الذين كانوا يقدمون أنفسهم على أنهم قادرون على تحمل المسؤوليات، ولكن لا يمكن الوثوق بهم ليكونوا مرتبطين بالعمل خشية أن يحولوه إلى مسارات خاطئة. \textbf{كان لا بد من اتخاذ خطوة تلو الأخرى، \underline{ليس وفقًا لحكمة البشر} بل وفقًا لحكمة وتعليمات ذاك الذي هو أحكم من أن يخطئ وأطيب من أن يؤذينا}. \textbf{كان هناك الكثير من العناصر التي كان لا بد من اختبارها وتجربتها. أشكر الرب أن الشيوخ سميث وأمادون وباتشيلور ما زالوا على قيد الحياة. لقد كانوا أعضاء عائلتنا في أصعب أجزاء تاريخنا}.}[Ms54-1890.7; 1890][https://egwwritings.org/read?panels=p7213.16]


According to this quotation, who laid down the foundation principles?


وفقًا لهذا الاقتباس، من الذي وضع المبادئ الأساسية؟


\egwinline{\textbf{\underline{We have stood shoulder to shoulder with Elder Smith in this work while the Lord was laying the foundation principles}}.} \textbf{It was the Lord}! But who wrote them down as a declaration of our faith? It was Elder Smith with James White and Sister White; we see that where Sister White says\egwinline{\textbf{we} have stood shoulder to shoulder with Elder Smith}. This \textit{‘we’} is explained in the previous paragraph: \egwinline{He \normaltext{[Elder Smith]} understands how\textbf{ we—my husband and myself}—have carried the work forward}. With this quotation, Sister White was clearly involved when the Lord was laying the \emcap{Fundamental Principles}.


\egwinline{\textbf{\underline{لقد وقفنا جنبًا إلى جنب مع الشيخ سميث في هذا العمل بينما كان الرب يضع المبادئ الأساسية}}.} \textbf{كان الرب}! ولكن من الذي دوّنها كإعلان لإيماننا؟ كان الشيخ سميث مع جيمس وايت والأخت وايت؛ نرى ذلك حيث تقول الأخت وايت \egwinline{\textbf{نحن} وقفنا جنبًا إلى جنب مع الشيخ سميث}. هذا \textit{‘نحن’} تم شرحه في الفقرة السابقة: \egwinline{إنه \normaltext{[الشيخ سميث]} يفهم كيف \textbf{قمنا نحن—زوجي وأنا} بدفع العمل إلى الأمام}. بهذا الاقتباس، كانت الأخت وايت مشاركة بوضوح عندما كان الرب يضع \emcap{المبادئ الجوهرية}.


It is true that the Declaration of the \emcap{Fundamental Principles} was written by a small group of people, namely Elder Smith, James White and Ellen White, but they endeavored to put into words what was the true view of the entire church body. They accurately represented the \emcap{fundamental principles}—the truths received in the beginning of our work. If that were not the case, then this declaration is the very opposite of what it claims to be. They were written \others{to meet inquiries} as to what was believed by Seventh-day Adventists, \others{to correct false statements circulated} and to \others{remove erroneous impressions.}[FP1872 3.1; 1872][https://egwwritings.org/read?panels=p928.8] If this document misrepresented the Adventist position, why was its continual reprinting, over the course of 42 years, permitted? It was reprinted until the death of Ellen White. If this document misrepresented the church’s position, wouldn’t Ellen White have raised her voice against it? She always raised her voice against the misrepresentation of the Seventh-day Adventist position, as she did with D. M. Canright and Dr. Kellogg. If the \emcap{Fundamental Principles} were misrepresenting the Seventh-day Adventist position, then all subsequent reprinting should be attributed to a conspiracy theory. That would be the greatest conspiracy theory within the Seventh-day Adventist Church. Ever. The harmony between the writings of Ellen White, Adventist pioneers, and the claims made in the Declaration of the \emcap{Fundamental Principles}, testify of the fact that this declaration is an accurate \others{summary of the principal features of} Seventh-day Adventist \others{faith, upon which there is, so far as we know, entire unanimity throughout the body}[The preface of the Fundamental Principles 1889].


صحيح أن إعلان \emcap{المبادئ الجوهرية} كُتب بواسطة مجموعة صغيرة من الأشخاص، وهم الشيخ سميث وجيمس وايت وإلين وايت، لكنهم سعوا لوضع في كلمات ما كانت النظرة الحقيقية للكنيسة بأكملها. لقد مثلوا بدقة \emcap{المبادئ الجوهرية}—الحقائق التي تم استلامها في بداية عملنا. إذا لم يكن الأمر كذلك، فإن هذا الإعلان هو عكس ما يدعي أنه تمامًا. تمت كتابتها \others{لتلبية الاستفسارات} حول ما كان يؤمن به الأدفنتست السبتيون، \others{لتصحيح البيانات الكاذبة المتداولة} و\others{لإزالة الانطباعات الخاطئة.}[FP1872 3.1; 1872][https://egwwritings.org/read?panels=p928.8] إذا كانت هذه الوثيقة تسيء تمثيل موقف الأدفنتست، فلماذا سُمح بإعادة طباعتها المستمرة، على مدى 42 عامًا؟ تمت إعادة طباعتها حتى وفاة إلين وايت. إذا كانت هذه الوثيقة تسيء تمثيل موقف الكنيسة، ألم تكن إلين وايت سترفع صوتها ضدها؟ لقد رفعت دائمًا صوتها ضد سوء تمثيل موقف الأدفنتست السبتيين، كما فعلت مع د. م. كانرايت والدكتور كيلوغ. إذا كانت \emcap{المبادئ الجوهرية} تسيء تمثيل موقف الأدفنتست السبتيين، فيجب أن تُعزى جميع عمليات إعادة الطباعة اللاحقة إلى نظرية مؤامرة. ستكون تلك أكبر نظرية مؤامرة داخل كنيسة الأدفنتست السبتيين. على الإطلاق. إن التناغم بين كتابات إلين وايت ورواد الأدفنتست والادعاءات المقدمة في إعلان \emcap{المبادئ الجوهرية}، يشهد على حقيقة أن هذا الإعلان هو \others{ملخص دقيق للسمات الرئيسية} لإيمان الأدفنتست السبتيين، \others{والتي هناك، على حد علمنا، إجماع تام عليها في جميع أنحاء الجسد}[مقدمة المبادئ الجوهرية 1889].


With the death of Sister White in 1915, printing of the \emcap{Fundamental Principles} ceased. From 1915 onward, the Yearbook did not print any statement of belief until 1931. At this time, the \emcap{Fundamental Principles} received substantial changes. For the first time, the Trinity was introduced to the \emcap{fundamental principles}. In points’ 2 and 3 we read:


مع وفاة الأخت وايت في عام 1915، توقفت طباعة \emcap{المبادئ الجوهرية}. من عام 1915 فصاعدًا، لم يطبع الكتاب السنوي أي بيان للإيمان حتى عام 1931. في هذا الوقت، تلقت \emcap{المبادئ الجوهرية} تغييرات جوهرية. لأول مرة، تم إدخال الثالوث إلى \emcap{المبادئ الجوهرية}. في النقطتين 2 و3 نقرأ:


\others{2. \textbf{That the Godhead, or Trinity, consists of the Eternal Father, a \underline{personal, spiritual Being}}, omnipotent, \textbf{\underline{omnipresent}}, omniscient, infinite in wisdom and love; \textbf{the Lord Jesus Christ, the Son of the Eternal Father}, \textbf{through whom all things were created} and through whom the salvation of the redeemed hosts will be accomplished; \textbf{the Holy Spirit, the third person of the Godhead}, the great regenerating power in the work of redemption. Matt. 28:19}


\others{2. \textbf{أن اللاهوت، أو الثالوث، يتكون من الآب الأبدي، \underline{كائن شخصي روحي}}، قدير، \textbf{\underline{كلي الوجود}}، كلي العلم، غير محدود في الحكمة والمحبة؛ \textbf{الرب يسوع المسيح، ابن الآب الأبدي}، \textbf{الذي من خلاله خُلقت جميع الأشياء} ومن خلاله سيتم خلاص جموع المفديين؛ \textbf{الروح القدس، الأقنوم الثالث من اللاهوت}، القوة العظيمة المجددة في عمل الفداء. متى 28:19}


\others{3. \textbf{That Jesus Christ is very God, being of the same nature and essence as the Eternal Father}…}[Yearbook of the Seventh-day Adventist Denomination, 1931, page. 377][https://static1.squarespace.com/static/554c4998e4b04e89ea0c4073/t/59d17eec12abd9c6194cd26d/1506901758727/SDA-YB1931-22+\%28P.+377-380\%29.pdf]


\others{3. \textbf{أن يسوع المسيح هو الله حقًا، كونه من نفس طبيعة وجوهر الآب الأبدي}...}[الكتاب السنوي لطائفة الأدفنتست السبتيين، 1931، صفحة. 377][https://static1.squarespace.com/static/554c4998e4b04e89ea0c4073/t/59d17eec12abd9c6194cd26d/1506901758727/SDA-YB1931-22+\%28P.+377-380\%29.pdf]


This change, in favor of the Trinity, appeared sixteen years after the death of Sister White. A comparison of this statement with the original \emcap{Fundamental Principles} presents several striking differences. The Father is still a personal, spiritual Being, the creator of all things, but is not addressed as “\textit{one God}” any longer. Jesus Christ is still the Son of the Eternal Father, through whom the Father created all things; Jesus is, also, of the very same nature and essence of the Father. Although these were the same terms to describe the doctrine on the \emcap{personality of God} in the original \emcap{Fundamental Principles}, we ask about the meaning of the term “\textit{personal, spiritual being}” applied to the Father, if He is, by new statement, omnipresent by Himself? The Holy Spirit is not an instrument, or means of the Father’s omnipresence anymore. Although this statement uses similar rhetoric of the original \emcap{Fundamental Principles}, it steps away from the original doctrine on the presence and the \emcap{personality of God}.


ظهر هذا التغيير، لصالح الثالوث، بعد ستة عشر عامًا من وفاة الأخت وايت. مقارنة هذا البيان مع \emcap{المبادئ الجوهرية} الأصلية تقدم عدة اختلافات ملفتة. لا يزال الآب كائنًا شخصيًا روحيًا، خالق كل الأشياء، لكنه لم يعد يُخاطب بـ “\textit{إله واحد}” بعد الآن. لا يزال يسوع المسيح هو ابن الآب الأبدي، الذي من خلاله خلق الآب كل الأشياء؛ يسوع أيضًا، من نفس طبيعة وجوهر الآب. على الرغم من أن هذه كانت نفس المصطلحات لوصف عقيدة \emcap{شخصانية الله} في \emcap{المبادئ الجوهرية} الأصلية، نسأل عن معنى مصطلح “\textit{كائن شخصي روحي}” المطبق على الآب، إذا كان هو، بحسب البيان الجديد، كلي الوجود بنفسه؟ الروح القدس ليس أداة، أو وسيلة لكلية وجود الآب بعد الآن. على الرغم من أن هذا البيان يستخدم بلاغة مماثلة للمبادئ الجوهرية الأصلية، إلا أنه يبتعد عن العقيدة الأصلية حول وجود و\emcap{شخصانية الله}.


According to LeRoy Froom, this statement was written entirely by Francis Wilcox, with the approval of three other brothers (C.H. Watson, M.E. Kern and E.R. Palmer).\footnote{Edwin Froom, LeRoy. Movement of Destiny. 1971., p. 411, 413, 414} In the unpublished paper of \textit{The Seventh-day Adventist Church in Mission: 1919-1979}, we read how Elder Wilcox made this statement contrary to the belief of the church body and published it without their approval.


وفقًا لليروي فروم، كُتب هذا البيان بالكامل بواسطة فرانسيس ويلكوكس، بموافقة ثلاثة إخوة آخرين (سي.إتش. واتسون، إم.إي. كيرن وإي.آر. بالمر).\footnote{إدوين فروم، ليروي. حركة المصير. 1971.، ص. 411، 413، 414} في الورقة غير المنشورة \textit{كنيسة الأدفنتست السبتيين في المهمة: 1919-1979}، نقرأ كيف قام الشيخ ويلكوكس بصياغة هذا البيان خلافًا لمعتقد جسد الكنيسة ونشره دون موافقتهم.


\others{\textbf{Realizing that the General Conference Committee or any other church body would never accept the document in the form in which it was written}, Elder Wilcox, with full knowledge of the group \normaltext{[C.H. Watson, M.E. Kern and E.R. Palmer]}, handed the Statement directly to Edson Rogers, the General Conference statistician, who published it in the 1931 edition of the Yearbook, where it has appeared ever since. It was without the official approval of the General Conference Committee, therefore, and without any formal denominational adoption, that Elder Wilcox's statement became the accepted declaration of our faith.}[Dwyer, Bonnie. “A New Statement of Fundamental Beliefs (1980) - Spectrum Magazine.” \textit{Spectrum Magazine}, 7 June 2009, \href{https://spectrummagazine.org/news/new-statement-fundamental-beliefs-1980/}{spectrummagazine.org/news/new-statement-fundamental-beliefs-1980/}. Accessed 30 Jan. 2025.]


\others{\textbf{إدراكًا أن لجنة المؤتمر العام أو أي هيئة كنسية أخرى لن تقبل أبدًا الوثيقة بالشكل الذي كُتبت به}، سلم الشيخ ويلكوكس، بمعرفة كاملة من المجموعة \normaltext{[سي.إتش. واتسون، إم.إي. كيرن وإي.آر. بالمر]}، البيان مباشرة إلى إدسون روجرز، إحصائي المؤتمر العام، الذي نشره في طبعة 1931 من الكتاب السنوي، حيث ظهر منذ ذلك الحين. كان ذلك بدون الموافقة الرسمية للجنة المؤتمر العام، وبالتالي، وبدون أي تبني طائفي رسمي، أصبح بيان الشيخ ويلكوكس الإعلان المقبول لإيماننا.}[دواير، بوني. “بيان جديد للمعتقدات الأساسية (1980) - مجلة سبكتروم.” \textit{مجلة سبكتروم}، 7 يونيو 2009، \href{https://spectrummagazine.org/news/new-statement-fundamental-beliefs-1980/}{spectrummagazine.org/news/new-statement-fundamental-beliefs-1980/}. تم الوصول إليه في 30 يناير 2025.]


In 1980, the final change to the public synopsis of the Seventh-day Adventist faith was made. The General Conference voted to adopt today’s official statement:


في عام 1980، تم إجراء التغيير النهائي للملخص العام لإيمان كنيسة الأدفنتست السبتيين. صوّت المجمع العام على تبني البيان الرسمي المعتمد اليوم:


\others{\textbf{There is one God: Father, Son and Holy Spirit, a unity of three coeternal Persons}. God is immortal, all-powerful, all-knowing, above all, and \textbf{ever present}. He is infinite and beyond human comprehension, yet known through His self-revelation. He is forever worthy of worship, adoration, and service by the whole creation.}[Seventh-day Adventists Believe: A Biblical Exposition of 27 Fundamental Doctrines, p. 16]


\others{\textbf{هناك إله واحد: الآب والابن والروح القدس، وحدة ثلاثة أقانيم أزلية}. الله خالد، كلي القدرة، كلي المعرفة، فوق الكل، و\textbf{حاضر دائمًا}. هو لا محدود وفوق الإدراك البشري، ومع ذلك معروف من خلال إعلانه عن ذاته. هو مستحق للعبادة والتمجيد والخدمة من قبل الخليقة كلها إلى الأبد.}[Seventh-day Adventists Believe: A Biblical Exposition of 27 Fundamental Doctrines, p. 16]


In this brief historical overview we see that the 1931 statement is a “middle step” between the original Adventist belief to the full trinitarian belief.


في هذه النظرة التاريخية الموجزة نرى أن بيان عام 1931 هو “خطوة وسطى” بين المعتقد الأدفنتستي الأصلي والمعتقد الثالوثي الكامل.


The change in our beliefs has occurred over time with many discussions. Our Adventist history has left a trace of these changes. If we are honest truth seekers we should study this matter in detail. Can we see, in our Adventist history, why we have left the first point of the \emcap{Fundamental Principles} in favor of the Trinity doctrine? Most certainly! In the following studies we will look at some of the historical documents that show why we have moved from the first point of the \emcap{Fundamental Principles}, held in the early years, to accept the Trinity doctrine. During these studies, we bid you to prayerfully evaluate the changes with your own beliefs.


حدث التغيير في معتقداتنا على مر الزمن مع العديد من المناقشات. لقد تركت تاريخنا الأدفنتستي أثرًا لهذه التغييرات. إذا كنا باحثين صادقين عن الحقيقة فعلينا دراسة هذا الأمر بالتفصيل. هل يمكننا أن نرى، في تاريخنا الأدفنتستي، لماذا تركنا النقطة الأولى من \emcap{المبادئ الأساسية} لصالح عقيدة الثالوث؟ بالتأكيد! في الدراسات التالية سننظر إلى بعض الوثائق التاريخية التي توضح لماذا انتقلنا من النقطة الأولى من \emcap{المبادئ الأساسية}، التي كانت معتمدة في السنوات الأولى، إلى قبول عقيدة الثالوث. خلال هذه الدراسات، ندعوك لتقييم التغييرات بصلاة مع معتقداتك الخاصة.


% The authority of the Fundamental Principles

\begin{titledpoem}
\stanza{
    Our principles of faith stand firm and true, \\
    Established by the Lord through chosen few. \\
    A platform built on unquestionable might, \\
    Waymarks that guide us through the darkest night.
}

\stanza{
    The pioneers sought truth with earnest prayer, \\
    Point after point laid down with godly care. \\
    Yet modern minds have altered what was clear, \\
    Changing foundations held for many a year.
}

\stanza{
    Return, O church, to truths that God ordained, \\
    Not to revised beliefs that men have claimed. \\
    Stand firm upon the rock that cannot move, \\
    In Fundamental Principles approved.
}

\stanza{
    Let not new scholars lead your faith astray, \\
    From paths our founders walked in heaven's way. \\
    The Lord Himself laid down these truths of old, \\
    Embrace their power with faith both strong and bold.
}
\end{titledpoem}