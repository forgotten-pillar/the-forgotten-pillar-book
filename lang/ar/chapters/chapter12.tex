\qrchapter{https://forgottenpillar.com/rsc/en-fp-chapter12}{Heaven's reality}


\qrchapter{https://forgottenpillar.com/rsc/en-fp-chapter12}{حقيقة السماء}


The \emcap{personality of God} deals with the quality or state of God being a person. Whenever we look at the pioneer's work on the \emcap{personality of God}, we see that they were all in harmony with the view that God is a tangible \textit{being}, possessing both body and parts. We always see the same underlying reasoning, which differentiates the term ‘\textit{spirit}’ and the term ‘\textit{being}’. By differentiating these terms, they explain the quality or state of God being a person\footnote{\href{https://www.merriam-webster.com/dictionary/personality}{Merriam-Webster Dictionary} defines the word ‘\textit{personality}’ as “\textit{quality or state of being a person}”.}—a \emcap{personality of God}. All their conclusions are summed up in the first point of the \emcap{Fundamental Principles}. \others{There is \textbf{one God}, a \textbf{personal}, \textbf{spiritual being}, the Creator of all things, omnipotent, omniscient, … and \textbf{every-where present by his representative, the Holy Spirit}. Psalm 139:7.}[FPSDA 1.2][https://egwwritings.org/read?panels=p1299.6]


تتعلق \emcap{شخصانية الله} بالصفة أو الحالة التي يكون بها الله شخصًا. عندما ننظر إلى عمل الرواد حول \emcap{شخصانية الله}، نرى أنهم كانوا جميعًا في انسجام مع وجهة النظر التي تقول إن الله \textit{كائن} ملموس، يمتلك جسدًا وأجزاء. نرى دائمًا نفس المنطق الأساسي، الذي يميز بين مصطلح ‘\textit{روح}’ ومصطلح ‘\textit{كائن}’. من خلال التمييز بين هذه المصطلحات، يشرحون الصفة أو الحالة التي يكون بها الله شخصًا\footnote{\href{https://www.merriam-webster.com/dictionary/personality}{قاموس مريام-ويبستر} يعرّف كلمة ‘\textit{شخصانية}’ بأنها “\textit{الصفة أو الحالة التي يكون بها الكائن شخصًا.}”}—\emcap{شخصانية الله}. تُلخص جميع استنتاجاتهم في النقطة الأولى من \emcap{المبادئ الأساسية}. \others{هناك \textbf{إله واحد}، \textbf{شخصي}، \textbf{كائن روحي}، خالق كل الأشياء، كلي القدرة، كلي العلم، ... و\textbf{حاضر في كل مكان بواسطة ممثله، الروح القدس}. مزمور 139:7.}[FPSDA 1.2][https://egwwritings.org/read?panels=p1299.6]


So far, in the pioneers’ work, we have seen that the \emcap{personality of God} is tightly connected to the reality of God’s presence. God is a personal spiritual being, having a body and shape; as such, His presence is cumbered to one locality—as the Bible says, in His temple, at His throne where He is surrounded with unapproachable glory. But He is everywhere present by His representative, the Holy Spirit. Obviously, the Holy Spirit is a spirit, and not a being, \bible{for a spirit hath not flesh and bones as ye see me have}, said Jesus (Luke 24:39). Christ is also a Being, like His Father. He is an express image of the Father’s person; therefore, He bears the same personality, or the quality or state of being a person, as His Father does.


حتى الآن، في عمل الرواد، رأينا أن \emcap{شخصانية الله} مرتبطة ارتباطًا وثيقًا بواقع حضور الله. الله كائن روحي شخصي، له جسد وشكل؛ وبالتالي، فإن حضوره محدود بمكان واحد—كما يقول الكتاب المقدس، في هيكله، عند عرشه حيث يحيط به مجد لا يُقترب منه. لكنه حاضر في كل مكان بواسطة ممثله، الروح القدس. من الواضح أن الروح القدس هو روح، وليس كائنًا، \bible{لأن الروح ليس له لحم وعظام كما ترون لي}، قال يسوع (لوقا 24:39). المسيح أيضًا كائن، مثل أبيه. هو صورة جوهر الآب؛ لذلك، فهو يحمل نفس الشخصانية، أو الصفة أو الحالة التي يكون بها الكائن شخصًا، كما يفعل أبوه.


In our experience, when we present the original Seventh-day Adventist beliefs on the \emcap{personality of God} to our trinitarian brothers, as expressed in the first two points of the \emcap{Fundamental Principles}, they often claim that the statements in the \emcap{Fundamental Principles} are correct in some way, but the understanding attributed to the terms “\textit{personal spiritual being}” are false. They usually attempt to harmonize the \emcap{Fundamental Principles} with the Trinity doctrine by twisting the words “\textit{spiritual being}”, as if the word ‘\textit{spiritual}’ means something mysterious, suitable to equalize the \emcap{personality of God} and of Christ with the personality of the Holy Ghost\footnote{The quality or state of the Holy Spirit being a person is bearing witness, not having the form of a person. \egw{\textbf{The Holy Spirit has a personality}, \textbf{\underline{else} He could not \underline{bear witness} to our spirits} and with our spirits that we are the children of God. \textbf{He must also be a divine person}, \textbf{\underline{else} He could not \underline{search out} the secrets which lie hidden in the mind of God}. ‘For what man knoweth the things of a man save the spirit of man, which is in him; even so the things of God knoweth no man, but the Spirit of God.’ [1 Corinthians 2:11.]}[21LtMs, Ms 20, 1906, par. 32][https://egwwritings.org/read?panels=p14071.10296041&index=0]. It is crystal clear that the Holy Spirit is a person, yet not in the same way as the Father and the Son, as the Holy Spirit does not possess the quality of an outward physical personage like the Father and the Son do.}. The underlying problem comes down to the understanding of the heavenly realities. The Bible is not silent about heaven, and its realities, and our pioneers understood it well. Below we read about the explanation of the terms “\textit{spiritual being}” from James White and Uriah Smith in their book, “\textit{The Biblical Institute}”. The Bible explains these terms using the example of angels, which are “\textit{spiritual beings}”.


في تجربتنا، عندما نقدم معتقدات الأدفنتست السبتيين الأصلية حول \emcap{شخصانية الله} لإخوتنا المؤمنين بالثالوث، كما هو معبر عنها في النقطتين الأوليين من \emcap{المبادئ الأساسية}، غالبًا ما يدعون أن البيانات الواردة في \emcap{المبادئ الأساسية} صحيحة بطريقة ما، لكن الفهم المنسوب إلى مصطلحات “كائن روحي شخصي” خاطئة. عادة ما يحاولون مواءمة \emcap{المبادئ الأساسية} مع عقيدة الثالوث من خلال تحريف كلمات “كائن روحي”، كما لو أن كلمة ‘روحي’ تعني شيئًا غامضًا، مناسبًا لمعادلة \emcap{شخصانية الله} والمسيح مع شخصانية الروح القدس\footnote{إن الصفة أو الحالة التي يكون بها الروح القدس شخصًا هي الشهادة، وليس امتلاك شكل شخص. \egw{\textbf{الروح القدس له شخصانية}، \textbf{\underline{وإلا} لما استطاع أن \underline{يشهد} لأرواحنا} ومع أرواحنا أننا أبناء الله. \textbf{يجب أن يكون أيضًا شخصًا إلهيًا}، \textbf{\underline{وإلا} لما استطاع أن \underline{يفحص} الأسرار المخفية في عقل الله}. ‘لأن من من الناس يعرف أمور الإنسان إلا روح الإنسان الذي فيه؟ هكذا أيضًا أمور الله لا يعرفها أحد إلا روح الله.’ [1 كورنثوس 2:11.]}[21LtMs, Ms 20, 1906, par. 32][https://egwwritings.org/read?panels=p14071.10296041&index=0]. من الواضح تمامًا أن الروح القدس هو شخص، ولكن ليس بنفس الطريقة التي يكون بها الآب والابن، حيث أن الروح القدس لا يمتلك صفة الشخصية الخارجية المادية كما يمتلكها الآب والابن.}. المشكلة الأساسية تعود إلى فهم الحقائق السماوية. الكتاب المقدس ليس صامتًا عن السماء وحقائقها، وقد فهم روادنا ذلك جيدًا. أدناه نقرأ عن شرح مصطلحات “كائن روحي” من جيمس وايت ويوريا سميث في كتابهما، “المعهد الكتابي”. يشرح الكتاب المقدس هذه المصطلحات باستخدام مثال الملائكة، التي هي “كائنات روحية”.


\begin{figure}[hp]
    \centering
    \includegraphics[width=1\linewidth]{images/uriah-smith.jpg}
    \caption*{Uriah Smith (1832-1903)}
    \label{fig:uriah-smith}
\end{figure}


\begin{figure}[hp]
    \centering
    \includegraphics[width=1\linewidth]{images/uriah-smith.jpg}
    \caption*{يوريا سميث (1832-1903)}
    \label{fig:uriah-smith}
\end{figure}


\others{\textbf{Angels are real beings}. They are described in the Bible as \textbf{possessing face, feet, wings} \&x. Ezekiel says of the cherubim, ‘\textbf{Their whole \underline{body} and their backs and their hands and their wings},’ \&c. Eze. 10:12. Angels \textbf{appeared }unto Abraham. Gen. 18:1-8. They talked and ate with him. They went on to Sodom and communed with Lot, who, entering into his house baked unleavened bread for them and they did eat. \textbf{These person were called angels}. David speaks of the manna as the corn of Heaven and angel’s food. Ps. 78:23-25.}


\others{\textbf{الملائكة كائنات حقيقية}. يتم وصفهم في الكتاب المقدس \textbf{بامتلاكهم وجهًا وأقدامًا وأجنحة} وغيرها. يقول حزقيال عن الكروبيم، ‘\textbf{كل \underline{أجسامهم} وظهورهم وأيديهم وأجنحتهم}،‘ وغيرها. حزقيال 10:12. \textbf{ظهر} الملائكة لإبراهيم. تكوين 18:1-8. تحدثوا وأكلوا معه. ذهبوا إلى سدوم وتحدثوا مع لوط، الذي دخل بيته وخبز لهم فطيرًا وأكلوا. \textbf{هؤلاء الأشخاص دُعوا ملائكة}. يتحدث داود عن المن كحنطة السماء وطعام الملائكة. مزمور 78:23-25.}


\othersnogap{The case of Balaam, Num. 22:22-31, is an interesting incident. The angel \textbf{appeared }to Balaam with a sword \textbf{drawn in his hand}. The question is sometimes asked \textbf{how angels can be \underline{material beings since we cannot see them}. This case illustrates it}. The record says the \textbf{Lord opened the eyes of Balaam and he saw the angel}. \textbf{The angel did not create a body for that occasion}.\textbf{ He was just the same as he was before Balaam saw him; \underline{but the change took place in Balaam}. His eyes were opened, then he beheld the angel}. It was the same with the servant of Elisha when he and his master were brought into a straight place, surrounded by the army of the king of Syria. 2 Kings 6:17. Elisha prayed that \textbf{the eyes of his servant might be opened}; and he immediately saw the whole mountain full of horses and chariots round about Elisha.}


\othersnogap{حالة بلعام، عدد 22:22-31، هي حادثة مثيرة للاهتمام. \textbf{ظهر} الملاك لبلعام وسيفه \textbf{مسلول في يده}. يُطرح السؤال أحيانًا \textbf{كيف يمكن للملائكة أن تكون \underline{كائنات مادية بينما لا يمكننا رؤيتها}. توضح هذه الحالة ذلك}. يقول السجل إن \textbf{الرب فتح عيني بلعام فرأى الملاك}. \textbf{لم يخلق الملاك جسدًا لتلك المناسبة}. \textbf{كان هو نفسه كما كان قبل أن يراه بلعام؛ \underline{لكن التغيير حدث في بلعام}. انفتحت عيناه، ثم رأى الملاك}. كان الأمر نفسه مع خادم أليشع عندما وُضع هو وسيده في مكان ضيق، محاطين بجيش ملك سوريا. 2 ملوك 6:17. صلى أليشع \textbf{أن تُفتح عينا خادمه}؛ وعلى الفور رأى الجبل كله مملوءًا خيلًا ومركبات حول أليشع.}


\othersnogap{\textbf{This may be further illustrated referring to things which we know are material and yet which we cannot see}. Air is material, light is material, even thought itself is only the result of material organizations — matter acting upon matter — and yet we can see none of these things. \textbf{Just so with the angels}.}


\othersnogap{\textbf{يمكن توضيح هذا أكثر بالإشارة إلى أشياء نعلم أنها مادية ومع ذلك لا يمكننا رؤيتها}. الهواء مادي، الضوء مادي، حتى الفكر نفسه هو فقط نتيجة تنظيمات مادية — المادة تعمل على المادة — ومع ذلك لا يمكننا رؤية أي من هذه الأشياء. \textbf{هكذا الحال مع الملائكة}.}


\othersnogap{\textbf{It is further objected to the materiality of the angels that they are called spirits. }Heb. 1:13, 14.\textbf{\underline{But this is no objection to their being literal beings}}. \textbf{They are simply spiritual beings organized differently from these earthly bodies which we possess}. Paul says, 1 Cor. 15:44, ‘\textbf{There is a natural body and there is \underline{a spiritual body}}.’ \textbf{The natural body we now have; the spiritual body we shall have in the resurrection}. ‘\textbf{It is raised a spiritual body}.’ Verse 44. \textbf{But then we are equal unto the angels}, Luke 20:36; \textbf{then we have bodies like unto Christ’s most glorious body}. Phil. 3:4\footnote{Typo: It should be Philippians 3:21} \textbf{and Christ is no less a spirit than the angels}. \textbf{We read that God is a spirit, that is, simply \underline{a spiritual being}}.}[James White and Uriah Smith, The Biblical Institute (Kindle Locations 2537-2553). Kindle Edition.]


\othersnogap{\textbf{هناك اعتراض آخر على مادية الملائكة وهو أنهم يُدعون أرواحًا.} عبرانيين 1:13، 14. \textbf{\underline{لكن هذا ليس اعتراضًا على كونهم كائنات حرفية}}. \textbf{إنهم ببساطة كائنات روحية منظمة بشكل مختلف عن هذه الأجسام الأرضية التي نمتلكها}. يقول بولس، 1 كورنثوس 15:44، ‘\textbf{يوجد جسم طبيعي ويوجد \underline{جسم روحي}}’. \textbf{الجسم الطبيعي لدينا الآن؛ الجسم الروحي سيكون لدينا في القيامة}. ‘\textbf{يُقام جسمًا روحيًا}’. الآية 44. \textbf{ولكن حينئذ نكون مساوين للملائكة}، لوقا 20:36؛ \textbf{حينئذ تكون لنا أجسام مثل جسد المسيح المجيد}. فيلبي 3:4\footnote{خطأ مطبعي: يجب أن تكون فيلبي 3:21} \textbf{والمسيح ليس أقل روحًا من الملائكة}. \textbf{نقرأ أن الله روح، أي ببساطة \underline{كائن روحي}}.}[James White and Uriah Smith, The Biblical Institute (Kindle Locations 2537-2553). Kindle Edition.]


The Bible gives us the insight that angels are spiritual beings that possess material bodies, but are still unseen to us, unless the Lord opens our eyes to see them. When the righteous will rise up in their new glorified bodies, they will rise in a spiritual body, an incorruptible one. This body will be tangible and material just as the new Earth will be tangible and material. And with our spiritual bodies we will possess the renewed Earth, we will replenish it \bible{and subdue it: and have dominion over the fish of the sea, and over the fowl of the air, and over every living thing that moveth upon the earth}[Genesis 1:28].


يعطينا الكتاب المقدس نظرة ثاقبة بأن الملائكة هي كائنات روحية تمتلك أجسامًا مادية، لكنها لا تزال غير مرئية لنا، ما لم يفتح الرب أعيننا لنراها. عندما يقوم الأبرار في أجسادهم الممجدة الجديدة، سيقومون في جسد روحي، غير قابل للفساد. هذا الجسد سيكون ملموسًا وماديًا تمامًا كما ستكون الأرض الجديدة ملموسة ومادية. وبأجسادنا الروحية سنمتلك الأرض المتجددة، وسنملأها \bible{وأخضعوها، وتسلطوا على سمك البحر وعلى طير السماء وعلى كل حيوان يدب على الأرض}[تكوين 1:28].


% Heaven's reality

\begin{titledpoem}
    
    \stanza{
        God is not a vapor unknown, \\
        Not as a mystery on His throne. \\
        In just one place as beings are, \\
        Yet by His Spirit spread afar.
    }

    \stanza{
        Christ bears God’s image as His Son, \\
        Two beings divine, not joined as one. \\
        Angels have bodies, yet unseen, \\
        Physical forms with heav’nly sheen.
    }

    \stanza{
        We cannot see their spirit frame, \\
        Until immortal life we claim. \\
        We’ll resurrect like them to be, \\
        And dwell with them eternally.
    }

    \stanza{
        God is not three in mystic blend, \\
        But Father, Son, Their Spirit send. \\
        The Father, Son are not obscure, \\
        They’re personal, we know for sure.
    }
    
\end{titledpoem}