\qrchapter{https://forgottenpillar.com/rsc/en-fp-chapter26}{The steps to Omega}


\qrchapter{https://forgottenpillar.com/rsc/en-fp-chapter26}{خطوات نحو أوميغا}


In our study so far, we have seen evidence that Kellogg’s controversy was connected to the Trinity doctrine and the \emcap{personality of God} expressed in the first point of the \emcap{Fundamental Principles}. Unfortunately, today we do not stand on that foundation regarding the \emcap{personality of God}; we have built another foundation that has changed the truth on the \emcap{personality of God} to a mysterious Triune God. Sister White was clearly against this reorganization and she prophesied that in the closing of His work, God will rehearse the history of the Advent movement and re-establish every pillar of our faith that was held in the beginning.


في دراستنا حتى الآن، رأينا أدلة على أن خلاف كيلوغ كان مرتبطًا بعقيدة الثالوث و\emcap{شخصانية الله} المعبر عنها في النقطة الأولى من \emcap{المبادئ الجوهرية}. للأسف، اليوم نحن لا نقف على ذلك الأساس فيما يتعلق بـ\emcap{شخصانية الله}؛ لقد بنينا أساسًا آخر غيّر الحق عن \emcap{شخصانية الله} إلى إله ثالوثي غامض. كانت الأخت وايت بوضوح ضد هذه إعادة التنظيم وتنبأت بأنه في ختام عمله، سيعيد الله سرد تاريخ حركة المجيء ويعيد تأسيس كل عمود من أعمدة إيماننا التي كانت موجودة في البداية.


\egw{\textbf{\underline{The Lord has declared that the history of the past shall be rehearsed as we enter upon the closing work}. \underline{Every truth} that He has given for these last days is to be proclaimed to the world. \underline{Every pillar} that He has established \underline{is to be strengthened}. We cannot now step off the foundation that God has established. We cannot now enter into any new organization; for this would mean apostasy from the truth}.}[Ms129-1905.6; 1905][https://egwwritings.org/read?panels=p9797.13]


\egw{\textbf{\underline{لقد أعلن الرب أن تاريخ الماضي سيُعاد سرده عندما ندخل العمل الختامي}. \underline{كل حق} أعطاه لهذه الأيام الأخيرة يجب أن يُعلن للعالم. \underline{كل عمود} أقامه \underline{يجب أن يُقوّى}. لا يمكننا الآن أن نخطو بعيدًا عن الأساس الذي وضعه الله. لا يمكننا الآن أن ندخل في أي تنظيم جديد؛ لأن هذا سيعني الارتداد عن الحق}.}[Ms129-1905.6; 1905][https://egwwritings.org/read?panels=p9797.13]


Comparing the \emcap{Fundamental Principles} with the current Fundamental Beliefs of Seventh-day Adventists, we see that we have entered into a new organization. God’s warning, given through Sister White, to re-establish all pillars of our faith in these last days, is becoming imperative. As we traced the Trinity doctrine from Kellogg's controversy, we came across Ellen White’s warnings against alpha and omega apostasy, which will enter into our church.


بمقارنة \emcap{المبادئ الجوهرية} مع المعتقدات الأساسية الحالية للأدفنتست السبتيين، نرى أننا دخلنا في تنظيم جديد. إن تحذير الله، المعطى من خلال الأخت وايت، لإعادة تأسيس جميع أعمدة إيماننا في هذه الأيام الأخيرة، أصبح ضروريًا. وبينما تتبعنا عقيدة الثالوث من خلاف كيلوغ، واجهنا تحذيرات إلين وايت ضد ارتداد ألفا وأوميغا، الذي سيدخل كنيستنا.


\egw{\textbf{‘Living Temple’ contains the alpha of these theories. I knew that \underline{the omega would follow in a little while}; and I trembled for our people}. I knew that \textbf{I must warn our brethren and sisters not to enter into controversy \underline{over the presence and personality of God}. The statements made in ‘Living Temple’ \underline{in regard to this point are incorrect}. }The scripture used to substantiate the doctrine there set forth, is scripture misapplied.}[SpTB02 53.2; 1904][https://egwwritings.org/read?panels=p417.271]


\egw{\textbf{يحتوي ‘ذا ليفينغ تمبل’ على ألفا هذه النظريات. كنت أعلم أن \underline{أوميغا ستتبع بعد قليل}؛ وارتعدت لأجل شعبنا}. كنت أعلم أنه \textbf{يجب علي أن أحذر إخوتنا وأخواتنا من الدخول في جدال \underline{حول وجود وشخصانية الله}. البيانات الواردة في ‘ذا ليفينغ تمبل’ \underline{فيما يتعلق بهذه النقطة غير صحيحة}. }إن الكتاب المقدس المستخدم لدعم العقيدة المطروحة هناك، هو كتاب مقدس مطبق بشكل خاطئ.}[SpTB02 53.2; 1904][https://egwwritings.org/read?panels=p417.271]


In the context of Seventh-day Adventist reorganization, we identify several steps that were necessary to accomplish this reorganization and are necessary to uphold it.


في سياق إعادة تنظيم الأدفنتست السبتيين، نحدد عدة خطوات كانت ضرورية لإنجاز إعادة التنظيم هذه وضرورية للحفاظ عليها.


\subsection*{Step 1: Deny the Fundamental Principles to be the foundation of our faith and the official, and accurate, representation of Seventh-day Adventist beliefs}


\subsection*{الخطوة 1: إنكار المبادئ الجوهرية كأساس لإيماننا والتمثيل الرسمي والدقيق لمعتقدات الأدفنتست السبتيين}


The first step necessary is to hide the original foundation of our faith by unlinking it with the \emcap{Fundamental Principles}.


الخطوة الأولى الضرورية هي إخفاء الأساس الأصلي لإيماننا من خلال فك ارتباطه بـ\emcap{المبادئ الجوهرية}.


\egw{\textbf{As a people, we are to \underline{stand firm on the platform of eternal truth} that has withstood test and trial. We are to \underline{hold to the sure pillars of our faith}. \underline{The principles of truth} that God has revealed to us \underline{are our only true foundation}. They have made us what we are. The lapse of time has not lessened their value. \underline{It is the constant effort of the enemy to remove these truths from their setting}, and to put in their place \underline{spurious theories}. He \underline{will bring in} everything that he possibly can to carry out his deceptive designs.}}[SpTB02 51.2; 1904][https://egwwritings.org/read?panels=p417.261]


\egw{\textbf{كشعب، يجب أن \underline{نقف بثبات على منصة الحق الأبدي} التي صمدت أمام الاختبار والتجربة. يجب أن \underline{نتمسك بأعمدة إيماننا الثابتة}. \underline{مبادئ الحق} التي كشفها الله لنا \underline{هي أساسنا الحقيقي الوحيد}. لقد جعلتنا ما نحن عليه. مرور الوقت لم يقلل من قيمتها. \underline{إنه الجهد المستمر للعدو لإزالة هذه الحقائق من إطارها}، ووضع \underline{نظريات زائفة} مكانها. سوف \underline{يُدخل} كل ما يمكنه لتنفيذ مخططاته الخادعة.}}[SpTB02 51.2; 1904][https://egwwritings.org/read?panels=p417.261]


\egw{\textbf{Messages of every order and kind have been urged upon Seventh-day Adventists, to take the place of the truth which, \underline{point by point}, has been sought out by prayerful study, and testified to by the miracle-working power of the Lord}. \textbf{But \underline{the way-marks} \underline{which have made us what we are}, \underline{are to be preserved}, and they \underline{will be preserved}, as God has signified through His word and the testimony of His Spirit}. \textbf{He calls upon us to \underline{hold firmly}, with the grip of faith, to \underline{the fundamental principles} that are \underline{based upon unquestionable authority}}.}[SpTB02 59.1; 1904][https://egwwritings.org/read?panels=p417.299]


\egw{\textbf{لقد تم حث رسائل من كل نوع وصنف على الأدفنتست السبتيين، لتحل محل الحق الذي، \underline{نقطة بنقطة}، تم البحث عنه بالدراسة المصحوبة بالصلاة، وشُهد له بقوة الرب العاملة للمعجزات}. \textbf{لكن \underline{المعالم} \underline{التي جعلتنا ما نحن عليه}، \underline{يجب أن تُحفظ}، وسوف \underline{تُحفظ}، كما أشار الله من خلال كلمته وشهادة روحه}. \textbf{إنه يدعونا إلى \underline{التمسك بقوة}، بقبضة الإيمان، بـ\underline{المبادئ الجوهرية} التي \underline{تستند إلى سلطة لا تقبل الشك}}.}[SpTB02 59.1; 1904][https://egwwritings.org/read?panels=p417.299]


The \emcap{Fundamental Principles} were the truths God revealed to the pioneers after the passing of time in 1844. We have seen the testimonies of our pioneers, including Ellen White, regarding the first point of the \emcap{Fundamental Principles}. All of them were in harmony regarding these particular points of our faith. In 1863, Seventh-day Adventists organized themselves into a church, as an organized body. Since then, many were misrepresenting the position of the Seventh-day Adventist Church and the pioneers found it necessary to meet inquiries, \others{and sometimes to correct false statements circulated against} the church’s beliefs and practices. Consequently, in 1872, the pioneers issued the document called “\textit{A Declaration of the Fundamental Principles, Taught and Practiced by the Seventh-Day Adventists}”\footnote{“A Declaration of the Fundamental Principles, Taught and Practiced by the Seventh-Day Adventists (1872) : MVT : Free Download, Borrow, and Streaming : Internet Archive.” Internet Archive, 2025, \href{https://archive.org/details/ADeclarationOfTheFundamentalPrinciplesTaughtAndPracticedByThe}{archive.org/details/ADeclarationOfTheFundamentalPrinciplesTaughtAndPracticedByThe}. Accessed 3 Feb. 2025.}. This declaration presented the public with \others{a brief statement of what is, and has been, with great unanimity, held by}[The preface of the Fundamental Principles in 1872.] Seventh-day Adventists.


المبادئ الجوهرية كانت الحقائق التي كشفها الله للرواد بعد مرور الوقت في عام 1844. لقد رأينا شهادات روادنا، بما في ذلك إلين وايت، فيما يتعلق بالنقطة الأولى من المبادئ الجوهرية. كان جميعهم في انسجام فيما يتعلق بهذه النقاط المحددة من إيماننا. في عام 1863، نظم الأدفنتست السبتيون أنفسهم في كنيسة، كهيئة منظمة. منذ ذلك الحين، كان الكثيرون يسيئون تمثيل موقف كنيسة الأدفنتست السبتيين ووجد الرواد أنه من الضروري الرد على الاستفسارات، \others{وأحيانًا لتصحيح البيانات الكاذبة المتداولة ضد} معتقدات وممارسات الكنيسة. ونتيجة لذلك، في عام 1872، أصدر الرواد وثيقة تسمى “\textit{إعلان المبادئ الأساسية التي يعلمها ويمارسها الأدفنتست السبتيون}”\footnote{“A Declaration of the Fundamental Principles, Taught and Practiced by the Seventh-Day Adventists (1872) : MVT : Free Download, Borrow, and Streaming : Internet Archive.” Internet Archive, 2025, \href{https://archive.org/details/ADeclarationOfTheFundamentalPrinciplesTaughtAndPracticedByThe}{archive.org/details/ADeclarationOfTheFundamentalPrinciplesTaughtAndPracticedByThe}. Accessed 3 Feb. 2025.}. قدم هذا الإعلان للجمهور \others{بيانًا موجزًا عما هو، وما كان، بإجماع كبير، يتمسك به}[مقدمة المبادئ الأساسية في عام 1872.] الأدفنتست السبتيون.


In the chapter “\hyperref[chap:authority]{The Authority of the Fundamental Principles}”, we discussed how pro-Trinitarian scholars have been compromising the authority of the \emcap{Fundamental Principles}, denying their true value in our Adventist history.


في الفصل “\hyperref[chap:authority]{سلطة المبادئ الجوهرية}”، ناقشنا كيف أن العلماء المؤيدين للثالوث كانوا يتنازلون عن سلطة المبادئ الجوهرية، منكرين قيمتها الحقيقية في تاريخنا الأدفنتستي.


Pro-trinitarian scholars argue that this declaration was not what it claims to be—a declaration of the \emcap{fundamental principles}, taught and practiced by the Seventh-day Adventists. This declaration was a summary of the principal features of Adventist’s faith, and no point is really as problematic or objectionable as the first point, dealing with the \emcap{personality of God} and where His presence is. But the evidence in favor of the \emcap{Fundamental Principles}, especially to the first point, is overwhelming.


يجادل العلماء المؤيدون للثالوث بأن هذا الإعلان لم يكن ما يدعي أنه - إعلان المبادئ الجوهرية، التي يعلمها ويمارسها الأدفنتست السبتيون. كان هذا الإعلان ملخصًا للسمات الرئيسية لإيمان الأدفنتست، ولا توجد نقطة إشكالية أو مرفوضة حقًا مثل النقطة الأولى، التي تتناول شخصانية الله وأين يوجد حضوره. لكن الأدلة المؤيدة للمبادئ الجوهرية، خاصة للنقطة الأولى، ساحقة.


All of these claims are easily refuted by the fact that the \emcap{Fundamental Principles} have been regularly issued and reprinted over the course of the entire life of Sister White, until 1914. If they were mere private opinions of a few individuals, as claimed by scholars\footnote{Ministry Magazine “Our Declaration of Fundamental Beliefs”: January 1958, Roy Anderson, J. Arthur Buckwalter, Louise Kleuser, Earl Cleveland and Walter Schubert}, would they have been consistently reprinted over the course of 42 years\footnote{For a detailed list of publications throughout these years, see the Appendix.}, publicly claiming to represent the synopsis of Seventh-day Adventist faith? If they had been issued only once, we could deem it a conspiracy by some individuals to purposely misrepresent Seventh-day Adventist faith. On the contrary, the \emcap{Fundamental Principles} were regularly reprinted, and they truly represented the official Seventh-day Adventist faith and practice.


يمكن دحض كل هذه الادعاءات بسهولة من خلال حقيقة أن المبادئ الجوهرية قد تم إصدارها وإعادة طباعتها بانتظام على مدار حياة الأخت وايت بأكملها، حتى عام 1914. إذا كانت مجرد آراء خاصة لبعض الأفراد، كما يدعي العلماء\footnote{Ministry Magazine “Our Declaration of Fundamental Beliefs”: January 1958, Roy Anderson, J. Arthur Buckwalter, Louise Kleuser, Earl Cleveland and Walter Schubert}، فهل كانت ستتم إعادة طباعتها باستمرار على مدار 42 عامًا\footnote{للحصول على قائمة مفصلة بالمنشورات خلال هذه السنوات، انظر الملحق.}، مدعية علنًا أنها تمثل موجزًا لإيمان الأدفنتست السبتيين؟ لو كانت قد صدرت مرة واحدة فقط، لكان بإمكاننا اعتبارها مؤامرة من بعض الأفراد لتحريف إيمان الأدفنتست السبتيين عمدًا. على العكس من ذلك، تمت إعادة طباعة المبادئ الجوهرية بانتظام، وكانت تمثل حقًا إيمان وممارسة الأدفنتست السبتيين الرسمية.


Another argument is that Sister White approved the \emcap{Fundamental Principles} in her writings by explicitly referring to them, and also by teaching the same truths taught in the \emcap{Fundamental Principles}. The works of our pioneers are also in harmony with the statements in this Declaration of the \emcap{Fundamental Principles}. Considering all of these facts, it is inevitable that this declaration was truthful in its claims. This document was indeed a declaration of the \emcap{fundamental principles}, taught and practiced by the Seventh-day Adventist Church, representing a public \others{synopsis of our faith}, \others{a brief statement of what is, and has been, with great unanimity, held by} Seventh-day Adventists.\footnote{The preface of the Fundamental Principles in 1872.} As such, it accurately represents the Seventh-day Adventist belief and practice, and represents the foundation of Seventh-day Adventist faith in the time of Ellen White.


حجة أخرى هي أن الأخت وايت وافقت على المبادئ الجوهرية في كتاباتها من خلال الإشارة إليها صراحة، وأيضًا من خلال تعليم نفس الحقائق التي تُعلَّم في المبادئ الجوهرية. كما أن أعمال روادنا تتوافق مع البيانات الواردة في هذا الإعلان عن المبادئ الجوهرية. بالنظر إلى كل هذه الحقائق، فمن الحتمي أن هذا الإعلان كان صادقًا في ادعاءاته. كانت هذه الوثيقة بالفعل إعلانًا عن المبادئ الجوهرية، التي تعلمها وتمارسها كنيسة الأدفنتست السبتيين، وتمثل \others{موجزًا عامًا لإيماننا}، \others{بيانًا موجزًا عما هو، وما كان، بإجماع كبير، يتمسك به} الأدفنتست السبتيون.\footnote{مقدمة المبادئ الأساسية في عام 1872.} وبهذه الصفة، فإنها تمثل بدقة معتقدات وممارسات الأدفنتست السبتيين، وتمثل أساس إيمان الأدفنتست السبتيين في زمن إلين وايت.


Today, in defense of the Trinity doctrine, Adventist historians boldly claim that when our pioneers were studying Adventist truths such as the sanctuary, investigative judgment, the Sabbath and other doctrines, they \others{did not study the subject of the doctrine of God}. These Adventist historians falsely claim that the doctrine of God \others{was not the question that they dealt at that time}[Denis Kaiser. “From Antitrinitarianism to Trinitarianism: The Adventist story” and Panelist. The God We Worship: A Godhead Symposium. Central California Conference, Dinuba, CA. March 23-24, 2018.]. Following this false claim, they present historical data on how Adventist doctrine gradually moved toward Trinitarian understanding. The truth is, there are some instances early on\footnote{The earliest mention of the Trinity doctrine, in a positive sense, was when M.C. Wilcox reprinted a non-Adventist article by Samuel Spear in Signs of the Times, December 7th, 1891 and December 14th, 1891} when the Trinity doctrine is mentioned in a positive light in our literature. But when you consider the fact that the Adventist church did have a positive position on the subject of the doctrine of God, as it was expressed in the \emcap{Fundamental Principles}, these instances cannot be interpreted as progressiveness in understanding, but rather an intrusion of the Trinity doctrine into the Seventh-day Adventist Church.


اليوم، دفاعًا عن عقيدة الثالوث، يدعي مؤرخو الأدفنتست بجرأة أنه عندما كان روادنا يدرسون حقائق الأدفنتست مثل المقدس، والدينونة التحقيقية، والسبت وغيرها من العقائد، \others{لم يدرسوا موضوع عقيدة الله}. يدعي هؤلاء المؤرخون الأدفنتست زورًا أن عقيدة الله \others{لم تكن المسألة التي تناولوها في ذلك الوقت}[Denis Kaiser. “From Antitrinitarianism to Trinitarianism: The Adventist story” and Panelist. The God We Worship: A Godhead Symposium. Central California Conference, Dinuba, CA. March 23-24, 2018.]. بعد هذا الادعاء الكاذب، يقدمون بيانات تاريخية حول كيف تحولت عقيدة الأدفنتست تدريجيًا نحو فهم الثالوث. الحقيقة هي أن هناك بعض الحالات المبكرة\footnote{أقدم ذكر لعقيدة الثالوث، بمعنى إيجابي، كان عندما أعاد إم. سي. ويلكوكس طباعة مقال غير أدفنتستي لصموئيل سبير في ساينز أوف ذا تايمز، 7 ديسمبر 1891 و14 ديسمبر 1891} عندما تم ذكر عقيدة الثالوث بطريقة إيجابية في أدبياتنا. ولكن عندما تأخذ في الاعتبار حقيقة أن كنيسة الأدفنتست كان لديها موقف إيجابي حول موضوع عقيدة الله، كما تم التعبير عنه في المبادئ الجوهرية، لا يمكن تفسير هذه الحالات على أنها تقدم في الفهم، بل بالأحرى تسلل لعقيدة الثالوث إلى كنيسة الأدفنتست السبتيين.


It is easy to refute the claim that Adventist pioneers did not understand the doctrine of God. If they did not understand it, they would have failed to proclaim the first angel’s message. We discussed this point in detail in the chapter “\hyperref[chap:remembering-the-beginning]{Remembering the beginning}”. The Seventh-day Adventist movement was not a failure, but a God-led, prophetic movement.


من السهل دحض الادعاء بأن رواد الأدفنتست لم يفهموا عقيدة الله. لو لم يفهموها، لفشلوا في إعلان رسالة الملاك الأول. ناقشنا هذه النقطة بالتفصيل في الفصل “\hyperref[chap:remembering-the-beginning]{تذكر البداية}”. لم تكن حركة الأدفنتست السبتيين فاشلة، بل كانت حركة نبوية بقيادة الله.


\subsection*{Step 2: Ignore the warnings of building a new foundation}


\subsection*{الخطوة 2: تجاهل التحذيرات من بناء أساس جديد}


When the \emcap{Fundamental Principles} are removed from the equation, many of Ellen White’s warnings fail to shine in their true light and their true meaning does not resonate with the reader.


عندما تتم إزالة المبادئ الجوهرية من المعادلة، فإن العديد من تحذيرات إلين وايت تفشل في التألق بنورها الحقيقي ولا يتردد معناها الحقيقي لدى القارئ.


We have cited many quotations where Sister White warned the church not to step off the \emcap{Fundamental Principles}. We dealt with them in the chapter “\hyperref[chap:apostasy]{The great apostasy is soon to be realized}”, but we will mention one of the most prominent quotations again.


لقد اقتبسنا العديد من الاقتباسات التي حذرت فيها الأخت وايت الكنيسة من الخروج عن المبادئ الجوهرية. تناولناها في الفصل “\hyperref[chap:apostasy]{الارتداد العظيم سيتحقق قريبًا}”، لكننا سنذكر واحدًا من أبرز الاقتباسات مرة أخرى.


\egw{\textbf{The enemy of souls has sought to bring in the supposition that a great reformation was to take place among Seventh-day Adventists, and that this reformation would \underline{consist in giving up the doctrines which stand as the pillars of our faith} and engaging in a process of reorganization}. Were this reformation to take place, what would result? \textbf{The principles of truth that God in His wisdom has given to the remnant church would be discarded. Our religion would be changed. \underline{The fundamental principles that have sustained the work for the last fifty years would be accounted as error}}. \textbf{A new organization would be established. Books of a new order would be written. A system of intellectual philosophy would be introduced}...}[Lt242-1903.13; 1903][https://egwwritings.org/read?panels=p7767.20]


\egw{\textbf{سعى عدو النفوس إلى إدخال الافتراض بأن إصلاحًا عظيمًا كان سيحدث بين الأدفنتست السبتيين، وأن هذا الإصلاح \underline{سيتمثل في التخلي عن العقائد التي تقف كأعمدة إيماننا} والانخراط في عملية إعادة تنظيم}. لو حدث هذا الإصلاح، فماذا ستكون النتيجة؟ \textbf{ستُهمل مبادئ الحق التي أعطاها الله بحكمته للكنيسة البقية. ستتغير ديانتنا. \underline{المبادئ الأساسية التي دعمت العمل خلال الخمسين سنة الماضية ستُعتبر خطأ}}. \textbf{ستُؤسس منظمة جديدة. ستُكتب كتب من نوع جديد. سيُقدم نظام من الفلسفة الفكرية}...}[Lt242-1903.13; 1903][https://egwwritings.org/read?panels=p7767.20]


\egwnogap{Who has authority to begin such a movement? \textbf{We have our Bibles. We have our experience, attested to by the miraculous working of the Holy Spirit}. \textbf{We have a truth that admits of no compromise.} \textbf{\underline{Shall we not repudiate everything that is not in harmony with this truth}?}}[Lt242-1903.14; 1903][https://egwwritings.org/read?panels=p7767.21]


\egwnogap{من لديه السلطة لبدء مثل هذه الحركة؟ \textbf{لدينا كتبنا المقدسة. لدينا خبرتنا، التي يشهد لها العمل المعجزي للروح القدس}. \textbf{لدينا حق لا يقبل أي مساومة.} \textbf{\underline{ألا يجب أن نرفض كل ما ليس في انسجام مع هذا الحق}؟}}[Lt242-1903.14; 1903][https://egwwritings.org/read?panels=p7767.21]


\subsection*{Step 3: Deny that the personality of God was the pillar of our faith and a part of the foundation of our faith}


\subsection*{الخطوة 3: إنكار أن شخصانية الله كانت عمود إيماننا وجزءًا من أساس إيماننا}


There is one Ellen White statement that apparently supports the claim that the \emcap{personality of God} was not a pillar of our faith. Another expression for “\textit{pillars of our faith}” is “\textit{landmarks}”. In the following quotations, Sister White lists several landmarks: the cleansing of the sanctuary, the three angels’ messages, the temple of God, the Sabbath and the non-immortality of the wicked.


هناك تصريح واحد لإلين وايت يدعم على ما يبدو الادعاء بأن \emcap{شخصانية الله} لم تكن عمودًا لإيماننا. تعبير آخر عن “\textit{أعمدة إيماننا}” هو “\textit{معالم}”. في الاقتباسات التالية، تسرد الأخت وايت عدة معالم: تطهير المقدس، رسائل الملائكة الثلاثة، هيكل الله، السبت وعدم خلود الأشرار.


\egw{The passing of the time in 1844 was a period of great events, opening to our astonished eyes \textbf{the cleansing of the sanctuary transpiring in heaven}, and having decided relation to God’s people upon the earth, [also] \textbf{the first and second angels’ messages and the third}, unfurling the banner on which was inscribed, ‘The commandments of God and the faith of Jesus.’ [Revelation 14:12.] One of the landmarks under this message was \textbf{the temple of God}, seen by His truth-loving people in heaven, and the ark containing the law of God. The light of \textbf{the Sabbath} of the fourth commandment flashed its strong rays in the pathway of the transgressors of God’s law. The \textbf{non-immortality of the wicked} is an old landmark. \textbf{I can call to mind nothing more that can come under the head of the old landmarks}. All this cry about changing the old landmarks is all imaginary.}[Ms13-1889.9; 1889][https://egwwritings.org/read?panels=p4179.14]


\egw{كان مرور الوقت في عام 1844 فترة أحداث عظيمة، فتحت أمام أعيننا المندهشة \textbf{تطهير المقدس الذي يحدث في السماء}، وله علاقة حاسمة بشعب الله على الأرض، [أيضًا] \textbf{رسالتي الملاك الأول والثاني والثالث}، رافعين الراية التي كُتب عليها، ‘وصايا الله وإيمان يسوع.’ [رؤيا 14:12.] أحد المعالم تحت هذه الرسالة كان \textbf{هيكل الله}، الذي رآه شعبه المحب للحق في السماء، والتابوت الذي يحتوي على شريعة الله. نور \textbf{السبت} في الوصية الرابعة أرسل أشعته القوية في طريق المتعدين على شريعة الله. \textbf{عدم خلود الأشرار} هو معلم قديم. \textbf{لا يمكنني أن أتذكر شيئًا آخر يمكن أن يندرج تحت عنوان المعالم القديمة}. كل هذا الصراخ حول تغيير المعالم القديمة هو خيال محض.}[Ms13-1889.9; 1889][https://egwwritings.org/read?panels=p4179.14]


At the end of this list of landmarks, or pillars of our faith, she states that she can recall nothing else that would fall under the category of the old landmarks. For many, this quotation serves as proof that the \emcap{personality of God} was neither an old landmark nor a pillar. It is true that in this quotation, Sister White did not explicitly mention the \emcap{personality of God}, but it would be implicitly included under the first angel’s message, as well as being an underlying doctrine of the Sanctuary message. Furthermore, there are other quotations from Sister White that explicitly include the \emcap{personality of God} as an old landmark or pillar of our faith.


في نهاية هذه القائمة من المعالم، أو أعمدة إيماننا، تذكر أنها لا تستطيع أن تتذكر شيئًا آخر يمكن أن يندرج تحت فئة المعالم القديمة. بالنسبة للكثيرين، يعتبر هذا الاقتباس دليلاً على أن \emcap{شخصانية الله} لم تكن معلمًا قديمًا ولا عمودًا. صحيح أنه في هذا الاقتباس، لم تذكر الأخت وايت صراحةً \emcap{شخصانية الله}، لكنها ستكون مشمولة ضمنيًا تحت رسالة الملاك الأول، وكذلك كونها عقيدة أساسية لرسالة المقدس. علاوة على ذلك، هناك اقتباسات أخرى من الأخت وايت تشمل صراحةً \emcap{شخصانية الله} كمعلم قديم أو عمود لإيماننا.


\egw{Those who seek to remove the \textbf{old landmarks} are not holding fast; they \textbf{are not remembering how they have received and heard}. Those who try to \textbf{\underline{bring in} theories that would remove \underline{the pillars of our faith}} \textbf{concerning the sanctuary}, \textbf{\underline{or concerning the personality of God or of Christ}, are working as blind men}. They are seeking to bring in uncertainties and to set the people of God \textbf{adrift}, without an anchor.}[Ms62-1905.14; 1905][https://egwwritings.org/read?panels=p10026.20]


\egw{أولئك الذين يسعون لإزالة \textbf{المعالم القديمة} لا يتمسكون بها؛ إنهم \textbf{لا يتذكرون كيف استلموا وسمعوا}. أولئك الذين يحاولون \textbf{\underline{إدخال} نظريات من شأنها أن تزيل \underline{أعمدة إيماننا}} \textbf{المتعلقة بالمقدس}، \textbf{\underline{أو المتعلقة بشخصانية الله أو المسيح}، يعملون كرجال عميان}. إنهم يسعون لإدخال الشكوك وجعل شعب الله \textbf{يطفو}، بدون مرساة.}[Ms62-1905.14; 1905][https://egwwritings.org/read?panels=p10026.20]


Sister White also teaches us that the pillars of our faith constitute the foundation of our faith.


تعلمنا الأخت وايت أيضًا أن أعمدة إيماننا تشكل أساس إيماننا.


\egw{\textbf{What influence is it that would lead men at this stage of our history to work in an underhanded, powerful way \underline{to tear down the foundation of our faith},—the foundation that was laid at the beginning of our work by prayerful study of the word and by revelation? Upon \underline{this foundation} we have been building for \underline{the past fifty years}. Do you wonder that when I see the beginning of a work that would \underline{remove some of the pillars of our faith}, I have something to say? I must obey the command, ‘Meet it!’}}[SpTB02 58.1; 1904][https://egwwritings.org/read?panels=p417.295]


\egw{\textbf{ما هو التأثير الذي يقود الناس في هذه المرحلة من تاريخنا للعمل بطريقة خفية وقوية \underline{لهدم أساس إيماننا}،—الأساس الذي وُضع في بداية عملنا من خلال الدراسة المصلية للكلمة وبالوحي؟ على \underline{هذا الأساس} كنا نبني \underline{خلال الخمسين سنة الماضية}. هل تتعجبون أنني عندما أرى بداية عمل من شأنه أن \underline{يزيل بعض أعمدة إيماننا}، يكون لدي ما أقوله؟ يجب أن أطيع الأمر، ‘واجهه!’}}[SpTB02 58.1; 1904][https://egwwritings.org/read?panels=p417.295]


Removing some of the pillars of our faith means tearing down the foundation of our faith. Elsewhere, Sister White said that tearing down or undermining the foundation of our faith is done by indoctrination of the sentiments regarding the \emcap{personality of God}.


إزالة بعض أعمدة إيماننا تعني هدم أساس إيماننا. في مكان آخر، قالت الأخت وايت إن هدم أو تقويض أساس إيماننا يتم من خلال تلقين الآراء المتعلقة بـ \emcap{شخصانية الله}.


\egw{The college was taken out of Battle Creek; yet students are still called there, and there they \textbf{become indoctrinated with the very sentiments regarding the personality of God and Christ that would undermine the foundation of our faith}.}[Lt72-1906.5; 1906][https://egwwritings.org/read?panels=p10013.11]


\egw{تم إخراج الكلية من باتل كريك؛ ومع ذلك لا يزال الطلاب يُدعون إلى هناك، وهناك \textbf{يتم تلقينهم بالآراء المتعلقة بشخصانية الله والمسيح التي من شأنها أن تقوض أساس إيماننا}.}[Lt72-1906.5; 1906][https://egwwritings.org/read?panels=p10013.11]


In light of these quotations we see positive testimony that the \emcap{personality of God} was part of the foundation of our faith. Furthermore, in chapter 10 of the special testimonies, entitled “\textit{The foundation of our faith}”, Sister White mentioned “\textit{Fundamental Principles}” using the synonyms “\textit{pillars of our faith}”, “\textit{waymarks}”, and “\textit{landmarks}”, when addressing the foundation of our faith.


في ضوء هذه الاقتباسات نرى شهادة إيجابية بأن \emcap{شخصانية الله} كانت جزءًا من أساس إيماننا. علاوة على ذلك، في الفصل العاشر من الشهادات الخاصة، بعنوان “\textit{أساس إيماننا}”، ذكرت الأخت وايت “\textit{المبادئ الجوهرية}” مستخدمة المرادفات “\textit{أعمدة إيماننا}”، “\textit{معالم}”، و”\textit{معالم بارزة}”، عند تناول أساس إيماننا.


\subsection*{Step 4: Alter the meaning of the term “the personality of God”}


\subsection*{الخطوة 4: تغيير معنى مصطلح “شخصانية الله”}


The term ‘\textit{personality}’ has two different applications and the most common definition in everyday use is in the area of psychology. ‘\textit{Personality}’ is defined as “\textit{the characteristic sets of behaviors, cognitions, and emotional patterns that evolve from biological and environmental factors}”\footnote{Wikipedia Contributors. “Personality.” Wikipedia, Wikimedia Foundation, 19 Apr. 2019, \href{https://en.wikipedia.org/wiki/Personality}{en.wikipedia.org/wiki/Personality}.}. It is of utmost importance to recognize that when we are dealing with the pillar of our faith—“\textit{the personality of God}”—we are not in the realms of psychology. The accurate application of the word ‘\textit{personality}’ within the doctrine on the \emcap{personality of God} is found in the Merriam-Webster Dictionary: “\textit{the quality or state of being a person}”\footnote{\href{https://www.merriam-webster.com/dictionary/personality}{Merriam-Webster Dictionary} - ‘\textit{personality}’}. According to the Merriam-Webster Dictionary, this definition has been in use since the 15th century\footnote{See “\href{https://www.merriam-webster.com/dictionary/personality\#word-history}{First known use}” of the word ‘personality’ in Merriam Webster Dictionary}. In the 1828 edition of the Merriam Webster Dictionary we read definition of the word ‘\textit{personality}’ as: “\textit{that which constitutes an individual a distinct person}”\footnote{\href{https://archive.org/details/americandictiona02websrich/page/272/mode/2up}{Merriam-Webster Dictionary, 1828 edition} - ‘\textit{personality}’} \footnote{\href{https://archive.org/details/websterscomplete00webs/page/974/mode/2up}{The 1886 edition of Merriam-Webster Dictionary} defines the word ‘\textit{personality}’ as: “\textit{that which constitutes, or pertains to, a person}”}. Both of the definitions are found in The Encyclopaedic Dictionary, by Hunter Robert\footnote{\href{https://babel.hathitrust.org/cgi/pt?id=mdp.39015050663213&view=1up&seq=780}{Hunter Robert, The Encyclopaedic Dictionary} - ‘\textit{personality}’}—dictionary owned by Ellen White. The use of these definitions can be seen from the articles written on the \emcap{personality of God}.


لمصطلح ‘\textit{شخصانية}’ تطبيقان مختلفان والتعريف الأكثر شيوعًا في الاستخدام اليومي هو في مجال علم النفس. تُعرَّف ‘\textit{الشخصانية}’ بأنها “\textit{مجموعات السلوكيات والإدراكات والأنماط العاطفية المميزة التي تتطور من العوامل البيولوجية والبيئية}”\footnote{Wikipedia Contributors. “Personality.” Wikipedia, Wikimedia Foundation, 19 Apr. 2019, \href{https://en.wikipedia.org/wiki/Personality}{en.wikipedia.org/wiki/Personality}.}. من الأهمية بمكان أن ندرك أنه عندما نتعامل مع عمود إيماننا—“\textit{شخصانية الله}”—فنحن لسنا في مجال علم النفس. التطبيق الدقيق لكلمة ‘\textit{شخصانية}’ ضمن عقيدة \emcap{شخصانية الله} موجود في قاموس ميريام-ويبستر: “\textit{الصفة أو الحالة التي يكون بها الكائن شخصًا.}”\footnote{\href{https://www.merriam-webster.com/dictionary/personality}{Merriam-Webster Dictionary} - ‘\textit{personality}’}. وفقًا لقاموس ميريام-ويبستر، فإن هذا التعريف كان قيد الاستخدام منذ القرن الخامس عشر\footnote{See “\href{https://www.merriam-webster.com/dictionary/personality\#word-history}{First known use}” of the word ‘personality’ in Merriam Webster Dictionary}. في طبعة عام 1828 من قاموس ميريام ويبستر نقرأ تعريف كلمة ‘\textit{شخصانية}’ على أنها: “\textit{ما يجعل الفرد شخصًا متميزًا}”\footnote{\href{https://archive.org/details/americandictiona02websrich/page/272/mode/2up}{Merriam-Webster Dictionary, 1828 edition} - ‘\textit{personality}’} \footnote{\href{https://archive.org/details/websterscomplete00webs/page/974/mode/2up}{The 1886 edition of Merriam-Webster Dictionary} defines the word ‘\textit{personality}’ as: “\textit{that which constitutes, or pertains to, a person}”}. كلا التعريفين موجودان في القاموس الموسوعي، لهنتر روبرت\footnote{\href{https://babel.hathitrust.org/cgi/pt?id=mdp.39015050663213&view=1up&seq=780}{Hunter Robert, The Encyclopaedic Dictionary} - ‘\textit{personality}’}—القاموس الذي كانت تملكه إلين وايت. يمكن رؤية استخدام هذه التعريفات من المقالات المكتوبة عن \emcap{شخصانية الله}.


In 1903, when Sister White wrote to Dr. Kellogg, \egwinline{I have \textbf{ever }had the same testimony to bear which I now bear \textbf{regarding the personality of God}}[Lt253-1903.9; 1903][https://egwwritings.org/read?panels=p9980.15], she recalled her vision when she saw the Father and the Son.


في عام 1903، عندما كتبت الأخت وايت إلى الدكتور كيلوغ، \egwinline{كان لدي \textbf{دائمًا }نفس الشهادة التي أقدمها الآن \textbf{بخصوص شخصانية الله}}[Lt253-1903.9; 1903][https://egwwritings.org/read?panels=p9980.15]، تذكرت رؤيتها عندما رأت الآب والابن.


\egw{‘I have often seen the lovely Jesus, that\textbf{ He is a person}.\textbf{ I asked Him if His Father was a person, }and \textbf{had \underline{a form} like Himself}. Said Jesus, ‘\textbf{I am the express image of My Father’s person!}’ [Hebrews 1:3.]}[Lt253-1903.12; 1903][https://egwwritings.org/read?panels=p9980.18]


\egw{‘لقد رأيت كثيرًا يسوع الجميل، وأنه\textbf{ شخص}.\textbf{ سألته إذا كان أبوه شخصًا، }و\textbf{له \underline{هيئة} مثله}. قال يسوع، ‘\textbf{أنا صورة جوهره!}’ [عبرانيين 1:3.]}[Lt253-1903.12; 1903][https://egwwritings.org/read?panels=p9980.18]


The quality or state that Sister White defines God to be a person is to have \textit{a form}—\textit{a physical appearance}. Dr. Kellogg follows the same application of the word \textit{‘personality’}, although through speculation.


الصفة أو الحالة التي تعرّف بها الأخت وايت أن الله شخص هي أن يكون له \textit{هيئة}—\textit{مظهر جسدي}. يتبع الدكتور كيلوغ نفس تطبيق كلمة \textit{‘شخصانية’}، وإن كان من خلال التكهن.


\others{The fact that God is so great that we cannot form a clear mental picture of \textbf{his physical appearance} need not lessen in our minds the reality of \textbf{His personality}...}[John H. Kellogg, The Living Temple, p. 31][https://archive.org/details/J.H.Kellogg.TheLivingTemple1903/page/n31/mode/2up]


\others{إن حقيقة أن الله عظيم لدرجة أننا لا نستطيع تكوين صورة ذهنية واضحة عن \textbf{مظهره الجسدي} لا ينبغي أن تقلل في أذهاننا من واقع \textbf{شخصانيته}...}[John H. Kellogg, The Living Temple, p. 31][https://archive.org/details/J.H.Kellogg.TheLivingTemple1903/page/n31/mode/2up]


As we have previously seen, our Adventist pioneers also pinpointed the physical appearance as a quality that makes God a person. James White wrote, \others{Those who deny \textbf{the personality of God}, say that ‘image’ here does not mean \textbf{physical form}, but moral image...}[James S. White, PERGO 1.1; 1861][https://egwwritings.org/read?panels=p1471.3]. J. B. Frisbie wrote, \others{Some seem to suppose it argues against \textbf{the personality of God}, because he is a Spirit, and say that he is without \textbf{body, or parts}...}[\href{https://documents.adventistarchives.org/Periodicals/RH/RH18540307-V05-07.pdf}{Adventist Review and Sabbath Herald, March 7, 1854}, J. B. Frisbie, “The Seventh-Day Sabbath Not Abolished”, p. 50]


كما رأينا سابقًا، حدد رواد الأدفنتست أيضًا المظهر الجسدي كصفة تجعل الله شخصًا. كتب جيمس وايت، \others{أولئك الذين ينكرون \textbf{شخصانية الله}، يقولون إن ‘الصورة’ هنا لا تعني \textbf{الشكل الجسدي}، بل الصورة الأخلاقية...}[James S. White, PERGO 1.1; 1861][https://egwwritings.org/read?panels=p1471.3]. كتب ج. ب. فريسبي، \others{يبدو أن البعض يفترض أنه يحتج ضد \textbf{شخصانية الله}، لأنه روح، ويقولون إنه بدون \textbf{جسد، أو أجزاء}...}[\href{https://documents.adventistarchives.org/Periodicals/RH/RH18540307-V05-07.pdf}{Adventist Review and Sabbath Herald, March 7, 1854}, J. B. Frisbie, “The Seventh-Day Sabbath Not Abolished”, p. 50]


In light of the facts, we recognize the application of the word ‘\textit{personality}’. When the subject on the \emcap{personality of God} is presented in its connection to the Trinity doctrine, there is often a tendency to alter the meaning of the word ‘\textit{personality}’. It is also important to mention that the subject on the \emcap{personality of God} deals with the personality of the Father. This is clearly seen from the presented data.


في ضوء الحقائق، ندرك تطبيق كلمة ‘\textit{شخصانية}’. عندما يتم تقديم موضوع \emcap{شخصانية الله} في ارتباطه بعقيدة الثالوث، غالبًا ما يكون هناك ميل لتغيير معنى كلمة ‘\textit{شخصانية}’. من المهم أيضًا أن نذكر أن موضوع \emcap{شخصانية الله} يتعامل مع شخصانية الآب. هذا واضح من البيانات المقدمة.


\subsection*{Step 5: In examining the Kellogg crisis, shifting the main focus from the personality of God to pantheism}


\subsection*{الخطوة 5: في دراسة أزمة كيلوغ، تحويل التركيز الرئيسي من شخصانية الله إلى وحدة الوجود}


The data on the Kellogg crisis, in connection with the Trinity doctrine, is overwhelming if the \emcap{personality of God} is accounted for in the equation. The only way to not connect the dots is to ignore the \emcap{personality of God} and shift focus to pantheism exclusively. We do not deny the pantheistic nature of Kellogg's controversy. We believe that the pantheistic nature of Kellogg's controversy cannot be rightly understood if it is not examined in the true light of the \emcap{personality of God}. But, unfortunately, in examination of the Kellogg crisis, the attention that pantheism receives supersedes the examination of the truth on the \emcap{personality of God}.


إن البيانات المتعلقة بأزمة كيلوغ، فيما يتصل بعقيدة الثالوث، هائلة إذا أُخذت \emcap{شخصانية الله} في الحسبان. الطريقة الوحيدة لعدم ربط النقاط هي تجاهل \emcap{شخصانية الله} وتحويل التركيز إلى وحدة الوجود حصريًا. نحن لا ننكر الطبيعة الوحدوية (البانثيستية) لصراع كيلوغ. نحن نؤمن أن الطبيعة الوحدوية لصراع كيلوغ لا يمكن فهمها بشكل صحيح إذا لم يتم فحصها في ضوء الحقيقة حول \emcap{شخصانية الله}. لكن، للأسف، في دراسة أزمة كيلوغ، فإن الاهتمام الذي تحظى به وحدة الوجود يفوق فحص الحقيقة حول \emcap{شخصانية الله}.


You can do a search of Ellen White’s compilations to see just how much more attention pantheism received than the \emcap{personality of God}. If you were to search her writings for ‘pantheism’ or ‘pantheistic’, excluding the compilations after her death, you would find 36 occurrences. Among them are several repetitive quotations that Sister White copied from one letter to another, or to the special testimonies for the church. If you were to count the distinct occurrences you would only find 12 distinct quotations containing words like ‘\textit{pantheism}’ or ‘\textit{pantheistic}’\footnote{On the \href{https://egwwritings.org/}{https://egwwritings.org/} search bar, input the word “\textit{pantheis*} ”; this will include all words beginning with the ‘\textit{pantheis...}’, (including ‘\textit{pantheism}’ and ‘\textit{pantheistic}’). The results can be compared in subsetting the corpus of Ellen White writings by including or excluding compilations after her death. This option is available in the dropdown menu under the search bar.}. If you conducted the same search, but only in the compilations issued after her death, you would find 140 occurrences! All of these fall into one of the twelve distinct instances Sister White wrote on the subject of pantheism.


يمكنك إجراء بحث في مجموعات كتابات إلين وايت لترى كم من الاهتمام الإضافي حظيت به وحدة الوجود مقارنة بـ \emcap{شخصانية الله}. إذا بحثت في كتاباتها عن ‘وحدة الوجود’ أو ‘وحدوي’، مستثنيًا المجموعات التي جُمعت بعد وفاتها، ستجد 36 ظهورًا. من بينها عدة اقتباسات متكررة نسختها الأخت وايت من رسالة إلى أخرى، أو إلى الشهادات الخاصة للكنيسة. إذا قمت بعد الظهورات المميزة فقط ستجد 12 اقتباسًا مميزًا فقط تحتوي على كلمات مثل ‘\textit{وحدة الوجود}’ أو ‘\textit{وحدوي}’\footnote{في شريط البحث على \href{https://egwwritings.org/}{https://egwwritings.org/}، أدخل كلمة “\textit{pantheis*} “؛ وهذا سيشمل جميع الكلمات التي تبدأ بـ ‘\textit{pantheis...}’، (بما في ذلك ‘\textit{pantheism}’ و ‘\textit{pantheistic}’). يمكن مقارنة النتائج في تقسيم مجموعة كتابات إلين وايت من خلال تضمين أو استبعاد المجموعات بعد وفاتها. هذا الخيار متاح في القائمة المنسدلة أسفل شريط البحث.}. إذا أجريت نفس البحث، ولكن فقط في المجموعات الصادرة بعد وفاتها، ستجد 140 ظهورًا! كل هذه تندرج ضمن واحدة من الحالات المميزة الاثنتي عشرة التي كتبت فيها الأخت وايت عن موضوع وحدة الوجود.


In a search of Ellen White writings on the phrase “\textit{personality of God}”, excluding the compilations after her death, you would find 58 occurrences. Among them are also several repetitive quotations that Sister White copied to several different letters and to the testimonies for the church. Yet, if you were to search this phrase within the compilations that were issued after her death you would only find 52 occurrences.


في بحث عن كتابات إلين وايت حول عبارة “\textit{شخصانية الله}”، مستثنيًا المجموعات بعد وفاتها، ستجد 58 ظهورًا. من بينها أيضًا عدة اقتباسات متكررة نسختها الأخت وايت إلى عدة رسائل مختلفة وإلى شهادات للكنيسة. ومع ذلك، إذا بحثت عن هذه العبارة داخل المجموعات التي صدرت بعد وفاتها فستجد فقط 52 ظهورًا.


These simple statistics demonstrate the focus of the compilators after the death of Sister White. Such emphasis on pantheism changed our public opinion regarding Kellogg’s crisis. Forty-three, out of fifty-eight, quotations on the phrase “\textit{personality of God}” are found in letters and manuscripts, available to the public from 2015 onwards. This means that three quarters (\textit{74 percent}) of the quotation regarding the \emcap{personality of God}, prior to 2015, was not available to the public. Prior to 2015 we did not have much available data to study Kellogg's crisis in light of the \emcap{personality of God} and in its context.


توضح هذه الإحصائيات البسيطة تركيز المجمعين بعد وفاة الأخت وايت. مثل هذا التركيز على وحدة الوجود غيّر رأينا العام بخصوص أزمة كيلوغ. ثلاثة وأربعون، من أصل ثمانية وخمسين، اقتباسًا حول عبارة “\textit{شخصانية الله}” موجودة في رسائل ومخطوطات، متاحة للجمهور من عام 2015 فصاعدًا. هذا يعني أن ثلاثة أرباع (\textit{74 بالمائة}) من الاقتباسات المتعلقة بـ \emcap{شخصانية الله}، قبل عام 2015، لم تكن متاحة للجمهور. قبل عام 2015 لم يكن لدينا الكثير من البيانات المتاحة لدراسة أزمة كيلوغ في ضوء \emcap{شخصانية الله} وفي سياقها.


% Steps to Omega

\begin{titledpoem}
    
    \stanza{
        On pillars now, the shadows cast— \\
        A truth forsaken, from the past. \\
        In steps they chart the silent drift, \\
        Five marks of change, through sacred rift.
    }

    \stanza{
        Denial blooms when once truth stood, \\
        Foundations are not understood, \\
        The fundamentals, once held dear \\
        Obscured, as new creeds appear.
    }

    \stanza{
        Prophetic warnings have been dimmed, \\
        Pioneers are shunned, old hymns are trimmed. \\
        The testimonies once rang out \\
        But now they’re often tinged with doubt.
    }

    \stanza{
        “God is a person” cast aside, \\
        And now His essence they deride. \\
        Forgotten pillar once was strong \\
        Now a new pillar, which is wrong!
    }

    \stanza{
        Scholars now twist the sacred term, \\
        Words redefined, they now affirm. \\
        Gone is the quest to see God’s face, \\
        Dim the desire for His embrace.
    }

    \stanza{
        The Kellogg crisis point is missed, \\
        The alpha given untrue twist \\
        And thus, the lessons are not learned \\
        The church toward omega turned.
    }

    \stanza{
        Confusion reigns, we can’t perceive \\
        It is not clear what we believe \\
        Our history has been revised \\
        We wanted truth, but then they lied.
    }
    
\end{titledpoem}