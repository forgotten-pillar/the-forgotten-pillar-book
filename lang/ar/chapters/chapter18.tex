\qrchapter{https://forgottenpillar.com/rsc/en-fp-chapter18}{The Heavenly Trio}


\qrchapter{https://forgottenpillar.com/rsc/en-fp-chapter18}{الثلاثي السماوي}


So far we have seen the evidence that Ellen White knew about Dr. Kellogg's trinitarian sentiments, and we have seen how she responded to it. She always uplifted the truth on the presence and the \emcap{personality of God}, and called to come back to the foundation of our faith—\emcap{Fundamental Principles}. However, when Adventist scholars discuss the doctrine of the Trinity and Ellen White, they do not approach it in the same manner as Ellen White did. The \emcap{Fundamental Principles} together with the doctrine on the \emcap{personality of God} is downplayed, and the twisted story is presented that Ellen White was trinitarian and responsible for the church's acceptance of the Trinity doctrine into our ranks. We want to challenge this twisted story by looking at the evidence that is often used to support this false narrative.


حتى الآن رأينا الأدلة على أن إلين وايت كانت على علم بآراء الدكتور كيلوغ الثالوثية، ورأينا كيف ردت عليها. لقد رفعت دائمًا الحق حول وجود و\emcap{شخصانية الله}، ودعت للعودة إلى أساس إيماننا—\emcap{المبادئ الجوهرية}. ومع ذلك، عندما يناقش علماء الأدفنتست عقيدة الثالوث وإلين وايت، فإنهم لا يتناولونها بنفس الطريقة التي تناولتها إلين وايت. يتم التقليل من شأن \emcap{المبادئ الجوهرية} مع عقيدة \emcap{شخصانية الله}، ويتم تقديم قصة محرفة مفادها أن إلين وايت كانت ثالوثية ومسؤولة عن قبول الكنيسة لعقيدة الثالوث في صفوفنا. نريد أن نتحدى هذه القصة المحرفة من خلال النظر في الأدلة التي غالبًا ما تُستخدم لدعم هذه الرواية الكاذبة.


One of the most prominent quotations to support the claim that Sister White was responsible for accepting the Trinity doctrine into our ranks is her writings and comments on Matthew 28:19\footnote{\bible{Go ye therefore, and teach all nations, baptizing them in the name of the Father, and of the Son, and of the Holy Ghost}[Matthew 28:19]}. The most prominent quotation to stand out in defense of the Trinity doctrine is “\textit{the Heavenly Trio}” quotation:


أحد الاقتباسات الأكثر بروزًا لدعم الادعاء بأن الأخت وايت كانت مسؤولة عن قبول عقيدة الثالوث في صفوفنا هو كتاباتها وتعليقاتها على متى 28:19\footnote{\bible{فَاذْهَبُوا وَتَلْمِذُوا جَمِيعَ الأُمَمِ وَعَمِّدُوهُمْ بِاسْمِ الآبِ وَالابْنِ وَالرُّوحِ الْقُدُسِ}[متى 28:19]}. الاقتباس الأكثر بروزًا للدفاع عن عقيدة الثالوث هو اقتباس “\textit{الثلاثي السماوي}”:


\egw{\textbf{There are \underline{three living persons} of the \underline{heavenly trio}}; in the name of these three great powers—\textbf{the Father, the Son, and the Holy Spirit}—those who receive Christ by living faith are baptized, and these powers will co-operate with the obedient subjects of heaven in their efforts to live the new life in Christ...}[Ev 615.1; 1946][https://egwwritings.org/read?panels=p30.3407]


\egw{\textbf{هناك \underline{ثلاثة أشخاص أحياء} في \underline{الثلاثي السماوي}}؛ باسم هذه القوى العظيمة الثلاث—\textbf{الآب والابن والروح القدس}—يتعمد الذين يقبلون المسيح بالإيمان الحي، وستتعاون هذه القوى مع رعايا السماء المطيعين في جهودهم للعيش الحياة الجديدة في المسيح...}[Ev 615.1; 1946][https://egwwritings.org/read?panels=p30.3407]


To reiterate, this quotation is often cited to argue that Sister White defended and advocated the Trinity doctrine. But, if we take a look at this quotation in its literary context, we see that within the quotation itself she actually \textit{refuted} this doctrine and exalted the truth on the \emcap{personality of God}. To some this is a ludicrous claim, but we invite you to make your judgment based on presented data. Let us examine the context of this quotation.


للتأكيد، غالبًا ما يُستشهد بهذا الاقتباس للجدال بأن الأخت وايت دافعت عن عقيدة الثالوث ودعت إليها. لكن، إذا نظرنا إلى هذا الاقتباس في سياقه الأدبي، نرى أنها في الواقع \textit{رفضت} هذه العقيدة وأعلت من شأن الحق حول \emcap{شخصانية الله} في الاقتباس نفسه. قد يبدو هذا الادعاء سخيفًا للبعض، لكننا ندعوك لإصدار حكمك بناءً على البيانات المقدمة. دعونا نفحص سياق هذا الاقتباس.


\egw{I am instructed to say, \textbf{The sentiments} of those who are searching for advanced scientific ideas \textbf{\underline{are not to be trusted}}. Such representations as the following are made: ‘\textbf{The Father is as the light invisible; the Son is as the light embodied; the Spirit as the light shed abroad.}’ ‘\textbf{The Father is like the dew, invisible vapor; the Son is like the dew gathered in beauteous form; the Spirit is like the dew fallen to the seat of life.}’ Another representation: ‘\textbf{The Father is like the invisible vapor. The Son is like the leaden cloud. The Spirit is rain fallen and working in refreshing power.}’}[Ms21-1906.8; 1906][https://egwwritings.org/read?panels=p9754.15]


\egw{لقد أُرشدت أن أقول، \textbf{إن آراء} أولئك الذين يبحثون عن أفكار علمية متقدمة \textbf{\underline{لا يمكن الوثوق بها}}. يتم تقديم تمثيلات مثل ما يلي: ‘\textbf{الآب هو كالنور غير المرئي؛ الابن هو كالنور المتجسد؛ والروح هو كالنور المنتشر في الخارج.}’ ‘\textbf{الآب مثل الندى، بخار غير مرئي؛ الابن مثل الندى المتجمع في شكل جميل؛ والروح مثل الندى الساقط على مقعد الحياة.}’ تمثيل آخر: ‘\textbf{الآب مثل البخار غير المرئي. الابن مثل السحابة الرصاصية. والروح هو المطر الساقط والعامل بقوة منعشة.}’}[Ms21-1906.8; 1906][https://egwwritings.org/read?panels=p9754.15]


What sentiments are not to be trusted? The data suggest that those sentiments are trinitarian ideas of \textit{one God in three persons}. How do we know that? We see in the literary context of the representations Sister White was quoting. Contrary to the popular belief that she was referencing the “\textit{false}” trinity expressed by Dr. Kellogg,\footnote{Whidden, Woodrow W, et al. \textit{The Trinity : Understanding God’s Love, His Plan of Salvation, and Christian Relationships}. Hagerstown, Md, Review And Herald Pub. Association, 2002, p. 216.} she was actually referencing trinitarian idea of \textit{three living persons of one living God}, advocated by William Boardman, in his book “Higher Christian Life”, which she quoted. The context matters. The context of the quotations she quoted, shows that the representations of the Father, the Son, and the Holy Spirit are serving to illustrate the sentiment of three living persons of one God. That is the sentiment we have been clearly instructed by God, not to trust. Let the data be its own interpreter.


ما هي الآراء التي لا يمكن الوثوق بها؟ تشير البيانات إلى أن تلك الآراء هي أفكار ثالوثية عن \textit{إله واحد في ثلاثة أشخاص}. كيف نعرف ذلك؟ نرى في السياق الأدبي للتمثيلات التي كانت الأخت وايت تقتبسها. خلافًا للاعتقاد الشائع بأنها كانت تشير إلى “الثالوث \textit{الزائف}” الذي عبر عنه الدكتور كيلوغ،\footnote{Whidden, Woodrow W, et al. \textit{The Trinity : Understanding God's Love, His Plan of Salvation, and Christian Relationships}. Hagerstown, Md, Review And Herald Pub. Association, 2002, p. 216.} كانت في الواقع تشير إلى الفكرة الثالوثية عن \textit{ثلاثة أشخاص أحياء لإله واحد حي}، التي دعا إليها ويليام بوردمان، في كتابه “الحياة المسيحية الأسمى”، الذي اقتبست منه. السياق مهم. سياق الاقتباسات التي اقتبستها، يظهر أن تمثيلات الآب والابن والروح القدس تخدم لتوضيح فكرة ثلاثة أشخاص أحياء لإله واحد. هذا هو الرأي الذي أُرشدنا بوضوح من الله، بعدم الوثوق به. دع البيانات تكون مفسرة نفسها.


\section*{The Higher Christian Life, William Boardman}


\section*{الحياة المسيحية الأسمى، ويليام بوردمان}


Ellen White owned William Boardman's book “Higher Christian Life.” It was a good book about Christian sanctification, but in it there was trinitarian sentiment, which Sister White was particularly instructed by God to call out. This is another instance of evidence where we see that Ellen White was familiar with the trinitarian stance, and she was addressing it directly. Let's get familiar with the trinitarian sentiments promoted by William Boardman.


امتلكت إلين وايت كتاب ويليام بوردمان “الحياة المسيحية الأسمى”. كان كتابًا جيدًا عن التقديس المسيحي، لكن كان فيه رأي ثالوثي، أرشدها الله بشكل خاص للتنبيه إليه. هذا مثال آخر من الأدلة حيث نرى أن إلين وايت كانت على دراية بالموقف الثالوثي، وكانت تتناوله مباشرة. دعونا نتعرف على الآراء الثالوثية التي روج لها ويليام بوردمان.


Speaking of Triune God, William Boardman writes:


متحدثًا عن الإله الثالوثي، يكتب ويليام بوردمان:


\othersQuote{And then, again, the Father is the author and planner of salvation through faith in his Son; and when we trust in his Son we honor the Father, because we accept of his plan of salvation for us, justify his wisdom, and act in accordance with his will in the matter. \textbf{A glance at the official and essential relations of the persons of the Holy Trinity to each other and to us, may throw additional light upon our pathway}. Upon this subject flippancy would border upon blasphemy. It is holy ground. He who ventures upon it may well tread with unshod foot, and uncovered head bowed low.}[William Boardman, The Higher Christian Life, p. 99; 1858][https://archive.org/details/higherchristian02boargoog/page/n106/]


\othersQuote{ثم، مرة أخرى، الآب هو مؤلف ومخطط الخلاص من خلال الإيمان بابنه؛ وعندما نثق في ابنه نكرم الآب، لأننا نقبل خطته للخلاص من أجلنا، ونبرر حكمته، ونتصرف وفقًا لإرادته في هذا الأمر. \textbf{إن نظرة على العلاقات الرسمية والأساسية لأشخاص الثالوث المقدس مع بعضهم البعض ومعنا، قد تلقي ضوءًا إضافيًا على طريقنا}. في هذا الموضوع، الاستخفاف قد يقترب من التجديف. إنها أرض مقدسة. من يجرؤ على الاقتراب منها عليه أن يمشي بقدم حافية، ورأس مكشوف منحنٍ.}[William Boardman, The Higher Christian Life, p. 99; 1858][https://archive.org/details/higherchristian02boargoog/page/n106/]


Brother Boardman wants us to take \others{a glance at the official and essential relations} of the three persons of the Holy Trinity. He asserts that \textit{God is one but also three}–\textit{Triune}–by presenting official and essential relations of the persons of the Holy Trinity. His fundamental statement and outline for his thesis is as follows:


يريدنا الأخ بوردمان أن نلقي \others{نظرة على العلاقات الرسمية والأساسية} للأشخاص الثلاثة في الثالوث المقدس. وهو يؤكد أن \textit{الله واحد ولكنه أيضًا ثلاثة}–\textit{ثالوث}–من خلال تقديم العلاقات الرسمية والأساسية لأشخاص الثالوث المقدس. بيانه الأساسي ومخطط أطروحته هو كما يلي:


\othersQuote{\textbf{The Father is fullness of the Godhead \underline{invisibly}, without form, whom no creature hath seen or can see}. \\
\textbf{The Son is the fullness of the Godhead \underline{embodied}, that his creatures may see him, and know him, and trust him}. \\
\textbf{The Spirit is the fullness of the Godhead \underline{in all active workings}, whether of creation, providence, revelation, or salvation, by which God manifests himself to and through the universe}.}[William Boardman, The Higher Christian Life, p. 100][https://archive.org/details/higherchristian02boargoog/page/n108/]


\othersQuote{\textbf{الآب هو ملء اللاهوت \underline{بشكل غير مرئي}، بدون شكل، الذي لم يره أي مخلوق ولا يستطيع أن يراه}. \\
\textbf{الابن هو ملء اللاهوت \underline{متجسدًا}، حتى يتمكن مخلوقاته من رؤيته، ومعرفته، والثقة به}. \\
\textbf{الروح هو ملء اللاهوت \underline{في كل الأعمال النشطة}، سواء في الخلق، أو العناية الإلهية، أو الوحي، أو الخلاص، التي من خلالها يُظهر الله نفسه للكون ومن خلاله}.}[William Boardman, The Higher Christian Life, p. 100][https://archive.org/details/higherchristian02boargoog/page/n108/]


This statement is foundational to his following statements and illustrations. In the following paragraphs, William Boardman gives the biblical motives to illustrate \others{the official and essential relations of the Holy Trinity}—\textit{that is, God being one, but yet three}. He writes:


هذا البيان أساسي لبياناته وتوضيحاته اللاحقة. في الفقرات التالية، يقدم ويليام بوردمان الدوافع الكتابية لتوضيح \others{العلاقات الرسمية والأساسية للثالوث المقدس}—\textit{أي أن الله واحد، ولكنه ثلاثة}. يكتب:


\othersQuote{Another of the names of Jesus will give the same analogies in a light not less striking - \textbf{The Sun of Righteousness}. \\
All the light of the sun in the heavens was once hidden in the invisibility of primal darkness; and after this, the light now blazing in the orb of day was, when first the command when forth, Let light be! and light was, at most only the diffused haze of the gray dawn of the morn of creation out of the darkness of chaotic night, without form, or body, or centre, or radiance, or glory. But when separated from the darkness and centered in the sun, then in its glorious glitter it became so resplendent that none but the eagle eye could bear to look it in the face. \\
But then again its rays falling aslant through earth’s atmosphere and vapors, gladdens all the world with the same light, dispelling the winter, and the cold, and the darkness; starting Spring forth in floral beauty, and Summer in vernal luxuriance, and Autumn laden with golden treasures for the garner.
\textbf{The Father is as the Light invisible}. \\
\textbf{The Son is as the Light embodied}. \\
\textbf{The Spirit is as the Light shed down}.}[William Boardman, The Higher Christian Life, p. 101,102][https://archive.org/details/higherchristian02boargoog/page/n108/]


\othersQuote{اسم آخر من أسماء يسوع سيعطي نفس التشابهات في ضوء لا يقل إثارة - \textbf{شمس البر}. \\
كل نور الشمس في السماوات كان مخفيًا ذات مرة في عدم رؤية الظلام الأولي؛ وبعد ذلك، كان النور المتوهج الآن في قرص النهار، عندما صدر الأمر أولاً، ليكن نور! وكان النور، على الأكثر مجرد ضباب منتشر من الفجر الرمادي لصباح الخليقة من ظلام الليل الفوضوي، بدون شكل، أو جسد، أو مركز، أو إشعاع، أو مجد. ولكن عندما انفصل عن الظلام وتمركز في الشمس، أصبح في بريقه المجيد متألقًا لدرجة أن لا أحد سوى عين النسر يمكنها أن تنظر إليه مباشرة. \\
ولكن بعد ذلك مرة أخرى أشعته تسقط مائلة عبر الغلاف الجوي للأرض والأبخرة، تبهج العالم كله بنفس النور، مبددة الشتاء، والبرد، والظلام؛ مطلقة الربيع في جمال زهري، والصيف في خضرة وفيرة، والخريف محملاً بالكنوز الذهبية للمخزن.
\textbf{الآب هو كالنور غير المرئي}. \\
\textbf{الابن هو كالنور المتجسد}. \\
\textbf{الروح هو كالنور المنسكب}.}[William Boardman, The Higher Christian Life, p. 101,102][https://archive.org/details/higherchristian02boargoog/page/n108/]


This illustration of the Sun of Righteousness shows that God the Father, who is \textit{the fullness of the Godhead invisible,} can be symbolically illustrated as a Light that \others{was once hidden in the invisibility of primal darkness}. The Son, who is \textit{the fullness of the Godhead embodied}, is like a Light that is embodied in \others{the morn of creation}. The Holy Spirit, who is \textit{the fullness of the Godhead in all active workings}, is like a \others{Light shed down}. William Boardman gives us another similar illustration to clarify the \others{official relations of the persons of the Godhead}:


هذا التوضيح لشمس البر يُظهر أن الله الآب، الذي هو \textit{ملء اللاهوت غير المرئي،} يمكن تصويره رمزيًا كنور \others{كان مخفيًا ذات مرة في عدم رؤية الظلام الأولي}. الابن، الذي هو \textit{ملء اللاهوت المتجسد}، هو مثل النور المتجسد في \others{صباح الخليقة}. الروح القدس، الذي هو \textit{ملء اللاهوت في كل الأعمال النشطة}، هو مثل \others{النور المنسكب}. يقدم لنا ويليام بوردمان توضيحًا آخر مشابهًا لتوضيح \others{العلاقات الرسمية لأشخاص اللاهوت}:


\othersQuote{One of the similies for blessed influences of the Spirit, \textbf{while giving the self-same official relations of the persons of the Godhead, to each other and to us}, may illustrate them still further,—\textbf{The Dew},—\textbf{The dew of Hermon} - the dew on the mown meadow. Before the dew gathers at all in drops, it hangs over all the landscape in visible vapor, omnipresent but unseen. By and by as the light wanes into morning, and as the temperature sinks and touches the dew point the invisible becomes the visible, the embodied; and, as the sun rises, it stands in diamond drops trembling and glittering in the sun’s young beams in pearly beauty upon leaf and flower, over all the face of nature. \\
But now again, a breeze springs up, the breath of heaven is wafted gently along, shaking leaf and flower, and in a moment the pearly drops are invisible angina. But where now? Fallen at the root of herb and flower to impart new life, freshness, vigor to all it touches. \\
\textbf{The Father is like the dew in invisible vapor}. \\
\textbf{The Son is like the dew gathered in beauteous form}. \\
\textbf{The Spirit is like the dew fallen to the seat of life}.}[William Boardman, The Higher Christian Life, p. 102,103][https://archive.org/details/higherchristian02boargoog/page/n110/]


\othersQuote{أحد التشبيهات للتأثيرات المباركة للروح، \textbf{بينما يعطي نفس العلاقات الرسمية لأشخاص اللاهوت، لبعضهم البعض ولنا}، قد يوضحها أكثر،—\textbf{الندى}،—\textbf{ندى حرمون} - الندى على المرج المحصود. قبل أن يتجمع الندى على الإطلاق في قطرات، يتعلق فوق كل المناظر الطبيعية في بخار غير مرئي، موجود في كل مكان ولكن غير مرئي. وبعد ذلك عندما يتلاشى الضوء إلى الصباح، وعندما تنخفض درجة الحرارة وتلامس نقطة الندى، يصبح غير المرئي مرئيًا، متجسدًا؛ ومع شروق الشمس، يقف في قطرات ماسية ترتعش وتتلألأ في أشعة الشمس الفتية بجمال لؤلؤي على الأوراق والزهور، على كل وجه الطبيعة. \\
ولكن الآن مرة أخرى، تهب نسمة، يتم حمل نفس السماء برفق على طول، مهتزة الأوراق والزهور، وفي لحظة تصبح القطرات اللؤلؤية غير مرئية مرة أخرى. ولكن أين الآن؟ سقطت عند جذر العشب والزهرة لتمنح حياة جديدة، نضارة، قوة لكل ما تلمسه. \\
\textbf{الآب مثل الندى في البخار غير المرئي}. \\
\textbf{الابن مثل الندى المتجمع في شكل جميل}. \\
\textbf{الروح مثل الندى الساقط إلى مقر الحياة}.}[William Boardman, The Higher Christian Life, p. 102,103][https://archive.org/details/higherchristian02boargoog/page/n110/]


The Father, who is \textit{the fullness of the Godhead invisible,} is illustrated by the \others{dew in invisible vapor}. The Son, who is \textit{the fullness of the Godhead embodied}, is illustrated by \others{the dew gathered in beauteous form}. The Spirit, who is \textit{the fullness of the Godhead in all active works}, is illustrated by \others{the dew fallen to the seat of life}. The next illustration that exemplifies the official relations of the three personalities of one God is by another Bible likening—the Rain.


الآب، الذي هو \textit{ملء اللاهوت غير المرئي،} يتم توضيحه بـ \others{الندى في البخار غير المرئي}. الابن، الذي هو \textit{ملء اللاهوت المتجسد}، يتم توضيحه بـ \others{الندى المتجمع في شكل جميل}. الروح، الذي هو \textit{ملء اللاهوت في كل الأعمال النشطة}، يتم توضيحه بـ \others{الندى الساقط إلى مقر الحياة}. التوضيح التالي الذي يمثل العلاقات الرسمية للشخصيات الثلاث لإله واحد هو من خلال تشبيه كتابي آخر—المطر.


\othersQuote{\textbf{Yet one more of these Bible likenings} – by no means exhausting them – will not be unwelcome, or useless, - \textbf{the Rain}. \\
Rain, like the dew, floats in invisibility, and omnipresence at the first, over all, around all. Seen by none. While it remains in its invisibility, the earth parches, clods cleave together, the ground cracks open, the sun pours down his burning heat, the winds lift up the dust in circling whirls, and rolling clouds, and famine gaunt and greedy stalks through the land, followed by pestilence and death. By and by, the eager watcher sees the little hand-like cloud rising far out over the sea. It gathers, gathers, gathers; comes and spreads as it comes, in majesty over the whole heavens: - But all is parched and dry and dead yet, upon earth. \\
But now comes a drop, and drop after drop, quicker, faster – the shower, the rain – sweeping on, and giving to earth all the treasures of the clouds – clods open, furrows soften, springs, rivulets, rivers, swell and fill, and all the land is gladdened again with restored abundance. \\
\textbf{The Father is like to the invisible vapor}. \\
\textbf{The Son is as the laden cloud and falling rain}. \\
\textbf{The Spirit is the Rain – fallen and working in refreshing power}.}[William Boardman, The Higher Christian Life, p. 103,104][https://archive.org/details/higherchristian02boargoog/page/n110/]


\othersQuote{\textbf{واحد آخر من هذه التشبيهات الكتابية} - التي لا تستنفد بأي حال من الأحوال - لن يكون غير مرحب به، أو غير مفيد، - \textbf{المطر}. \\
المطر، مثل الندى، يطفو في عدم الرؤية، وموجود في كل مكان في البداية، فوق الكل، حول الكل. لا يراه أحد. بينما يبقى في حالة عدم رؤيته، تجف الأرض، تلتصق الكتل معًا، تتشقق الأرض، تصب الشمس حرارتها الحارقة، ترفع الرياح الغبار في دوامات دائرية، وسحب متدحرجة، والمجاعة النحيلة والجشعة تجوب الأرض، يتبعها الوباء والموت. بعد فترة، يرى المراقب المتلهف السحابة الصغيرة التي تشبه اليد ترتفع بعيدًا فوق البحر. تتجمع، تتجمع، تتجمع؛ تأتي وتنتشر أثناء قدومها، في عظمة فوق السماء كلها: - لكن كل شيء جاف وميت حتى الآن، على الأرض. \\
ولكن الآن تأتي قطرة، وقطرة بعد قطرة، أسرع، أسرع - الوابل، المطر - يجتاح، ويعطي للأرض كل كنوز السحب - تنفتح الكتل، تلين الأخاديد، تتضخم الينابيع، والجداول، والأنهار، وتمتلئ، وكل الأرض تبتهج مرة أخرى بالوفرة المستعادة. \\
\textbf{الآب مثل البخار غير المرئي}. \\
\textbf{الابن مثل السحابة المحملة والمطر المتساقط}. \\
\textbf{الروح هو المطر - الساقط والعامل بقوة منعشة}.}[William Boardman, The Higher Christian Life, p. 103,104][https://archive.org/details/higherchristian02boargoog/page/n110/]


Let's give William Boardman a fair hearing. He is not saying that the Father is \others{invisible vapor}; rather, he uses a metaphor of rain and \others{invisible vapor} to illustrate his main point that the Father is the invisible fullness of the Godhead. So it is with the Son, who, just like rain manifested in leaden clouds, is all the fullness of the Godhead manifested. To ensure his sentiments are not potentially misrepresented, William Boardman clarified his sentiment. This was the very sentiment that Ellen White was instructed by God not to trust:


دعونا نعطي ويليام بوردمان استماعًا عادلًا. هو لا يقول إن الآب هو \others{بخار غير مرئي}؛ بل يستخدم استعارة المطر و\others{البخار غير المرئي} لتوضيح نقطته الرئيسية وهي أن الآب هو ملء اللاهوت غير المرئي. وكذلك الأمر مع الابن، الذي، تمامًا مثل المطر المتجلي في السحب الرصاصية، هو كل ملء اللاهوت المتجلي. لضمان عدم إساءة تمثيل آرائه المحتملة، أوضح ويليام بوردمان رأيه. كان هذا هو الرأي نفسه الذي أُرشدت إلين وايت من قبل الله بعدم الثقة به:


\othersQuote{\textbf{These likenings are all imperfect. They rather hide than illustrate \underline{the tri-personality of the one God}, for they are not persons but things, poor and earthly at best, to represent the living personalities of the living God. So much they may do, however, as to illustrate the official relations of each to the others and of each and all to us. And more. They may also illustrate the truth that all the fulness of Him who filleth all in all, dwells in each person of \underline{the Triune God}}. \\
\textbf{The Father is all the fulness of the Godhead INVISIBLE}. \\
\textbf{The Son is all the fulness of the Godhead MANIFESTED}. \\
\textbf{The Spirit is all the fulness of the Godhead MAKING MANIFEST}. \\
\textbf{The persons are not mere offices, or modes of revelation, but living persons of the living God}.}[William Boardman, The Higher Christian Life, p. 104,105][https://archive.org/details/higherchristian02boargoog/page/n112/]


\othersQuote{\textbf{هذه التشبيهات كلها غير كاملة. إنها تخفي بدلاً من أن توضح \underline{شخصانية الله الثلاثية الواحدة}، لأنها ليست أشخاصًا بل أشياء، فقيرة وأرضية في أحسن الأحوال، لتمثل الشخصيات الحية للإله الحي. ومع ذلك، قد تفعل هذه التشبيهات ما يكفي لتوضيح العلاقات الرسمية لكل منها مع الآخرين ولكل منها وجميعها معنا. وأكثر من ذلك. قد توضح أيضًا حقيقة أن كل ملء الذي يملأ الكل في الكل، يسكن في كل شخص من \underline{الله الثالوثي}}. \\
\textbf{الآب هو كل ملء اللاهوت غير المرئي}. \\
\textbf{الابن هو كل ملء اللاهوت المُعلَن}. \\
\textbf{الروح هو كل ملء اللاهوت الذي يُعلِن}. \\
\textbf{الأشخاص ليسوا مجرد مناصب، أو أنماط للإعلان، بل أشخاص أحياء للإله الحي}.}[William Boardman, The Higher Christian Life, p. 104,105][https://archive.org/details/higherchristian02boargoog/page/n112/]


It is crucial to emphasize that when Boardman uses these Bible likenings from nature, he speaks of the illustrations, and not reality. These representations are illustrating his sentiments. In his own admission, that was the sentiment of three \others{living personalities of the living God.} Though these illustrations are imperfect, they may \others{illustrate the official relations} of \others{the tri-personality of the one God} and \others{the truth that all the fullness of Him who filleth all in all dwells in each person of the Triune God.} One God in three persons is the sentiment in question, and that sentiment is common to all types and versions of the trinity doctrine—including our current trinitarian stance in the second point of the Fundamental Beliefs.\footnote{\others{There is \textbf{one God}: Father, Son, and Holy Spirit, \textbf{a unity of three} coeternal \textbf{Persons}…} 2nd point of the Fundamental Beliefs}


من المهم التأكيد على أنه عندما يستخدم بوردمان هذه التشبيهات الكتابية من الطبيعة، فإنه يتحدث عن التوضيحات، وليس الواقع. هذه التمثيلات توضح آراءه. وباعترافه الخاص، كان ذلك رأي ثلاثة \others{شخصيات حية للإله الحي.} على الرغم من أن هذه التوضيحات غير كاملة، إلا أنها قد \others{توضح العلاقات الرسمية} لـ \others{شخصانية الله الثلاثية الواحدة} و \others{حقيقة أن كل ملء الذي يملأ الكل في الكل يسكن في كل شخص من الله الثالوثي.} إله واحد في ثلاثة أشخاص هو الرأي المطروح، وهذا الرأي مشترك بين جميع أنواع وإصدارات عقيدة الثالوث - بما في ذلك موقفنا الثالوثي الحالي في النقطة الثانية من المعتقدات الأساسية.\footnote{\others{هناك \textbf{إله واحد}: الآب والابن والروح القدس، \textbf{وحدة من ثلاثة} \textbf{أشخاص} أزليين...} النقطة الثانية من المعتقدات الأساسية}


In this brief look at William Boardman's sentiments, it is clear that the sentiments in question which Ellen White was instructed by God to call out, were the sentiments of the Triune God, or \textit{three living persons in the Trinity}. With that data in mind, let's examine Ellen White's response.


في هذه النظرة الموجزة لآراء وليام بوردمان، من الواضح أن الآراء المعنية التي أمر الله إلين وايت بالتنديد بها، كانت آراء الله الثالوثي، أو \textit{ثلاثة أشخاص أحياء في الثالوث}. مع وضع تلك البيانات في الاعتبار، دعونا نفحص رد إلين وايت.


\section*{Ellen White on William Boardman’s sentiment}


\section*{إلين وايت عن رأي وليام بوردمان}


With the Heavenly Trio quotation, it has been asserted that Ellen White was trinitarian. This is done by ignorantly or sometimes purposely ignoring the context of this valuable quotation. When reading Ellen White’s response, in which she defends our perceptions of God, try to recognize whom she is addressing when she speaks of God. Was the God she defended the Trinity or the Father? Referencing William Boardmans illustrations she said:


من خلال اقتباس الثالوث السماوي، تم التأكيد على أن إلين وايت كانت تؤمن بالثالوث. يتم ذلك عن طريق تجاهل سياق هذا الاقتباس القيم بجهل أو أحيانًا عن قصد. عند قراءة رد إلين وايت، الذي تدافع فيه عن تصوراتنا عن الله، حاول أن تتعرف على من تخاطب عندما تتحدث عن الله. هل كان الله الذي دافعت عنه هو الثالوث أم الآب؟ بالإشارة إلى توضيحات وليام بوردمان قالت:


\egw{\textbf{All these \underline{spiritualistic} representations are simply nothingness}. They are imperfect, untrue. They weaken and diminish the Majesty which no earthly likeness can be compared to. \textbf{God cannot be compared with the things His hands have made}. These are mere earthly things, suffering under the curse of God because of the sins of man. \textbf{The Father cannot be described by the things of earth}. \textbf{The Father is all the fulness of the Godhead \underline{bodily} and is \underline{invisible to mortal sight}}.}[Ms21-1906.9; 1906][https://egwwritings.org/read?panels=p9754.15]


\egw{\textbf{كل هذه التمثيلات \underline{الروحانية} هي ببساطة لا شيء}. إنها غير كاملة وغير صحيحة. إنها تضعف وتقلل من الجلال الذي لا يمكن مقارنته بأي تشبيه أرضي. \textbf{لا يمكن مقارنة الله بالأشياء التي صنعتها يداه}. هذه مجرد أشياء أرضية، تعاني تحت لعنة الله بسبب خطايا الإنسان. \textbf{لا يمكن وصف الآب بأشياء الأرض}. \textbf{الآب هو كل ملء اللاهوت \underline{جسديًا} وهو \underline{غير مرئي للنظر البشري}}.}[Ms21-1906.9; 1906][https://egwwritings.org/read?panels=p9754.15]


By observing the context, it is obvious that Sister White follows Boardman’s line of reasoning and corrects the mistakes. For better comparison, let us look at their writings side by side:


من خلال ملاحظة السياق، من الواضح أن الأخت وايت تتبع خط تفكير بوردمان وتصحح الأخطاء. للمقارنة بشكل أفضل، دعونا ننظر إلى كتاباتهما جنبًا إلى جنب:


\begin{table}[H]
\centering
\renewcommand{\arraystretch}{1.5}
\setlength{\tabcolsep}{15pt}
\resizebox{\textwidth}{!}{
\begin{tabular}{|p{0.4\textwidth}|p{0.4\textwidth}|}
\hline
\multicolumn{1}{|c|}{\textbf{William Boardman}} & \multicolumn{1}{c|}{\textbf{Ellen G. White}} \\ \hline
\othersQuote{These likenings are all imperfect. They rather hide than \textbf{illustrate the tri-personality of the \underline{one God}}, for they are not persons but things, poor and earthly at best, to represent \textbf{the living personalities of the living God}. \textbf{So much they may do, however, as to illustrate the official relations of each to the other and of each and all to us. And more. They may also illustrate the truth that all the fulness of Him who filleth all in all, dwells in \underline{each person of Triune God}}.}[p. 104,105][https://archive.org/details/higherchristian02boargoog/page/n112] & 
\egw{\textbf{All these \underline{spiritualistic} representations are simply nothingness}. They are imperfect, untrue. They weaken and diminish the Majesty which no earthly likeness can be compared to. \textbf{God cannot be compared with the things His hands have made}. These are mere earthly things, suffering under the curse of God because of the sins of man. \textbf{The Father cannot be described by the things of earth}.}[Ms21-1906.9; 1906][https://egwwritings.org/read?panels=p9754.15] \\ \hline
\end{tabular}
}
\end{table}


\begin{table}[H]
\centering
\renewcommand{\arraystretch}{1.5}
\setlength{\tabcolsep}{15pt}
\resizebox{\textwidth}{!}{
\begin{tabular}{|p{0.4\textwidth}|p{0.4\textwidth}|}
\hline
\multicolumn{1}{|c|}{\textbf{وليام بوردمان}} & \multicolumn{1}{c|}{\textbf{إلين ج. وايت}} \\ \hline
\othersQuote{هذه التشبيهات كلها غير كاملة. إنها تخفي بدلاً من أن \textbf{توضح شخصانية \underline{الله الواحد} الثلاثية}، لأنها ليست أشخاصًا بل أشياء، فقيرة وأرضية في أحسن الأحوال، لتمثل \textbf{الشخصيات الحية للإله الحي}. \textbf{ومع ذلك، قد تفعل هذه التشبيهات ما يكفي لتوضيح العلاقات الرسمية لكل منها مع الآخر ولكل منها وجميعها معنا. وأكثر من ذلك. قد توضح أيضًا حقيقة أن كل ملء الذي يملأ الكل في الكل، يسكن في \underline{كل شخص من الله الثالوثي}}.}[p. 104,105][https://archive.org/details/higherchristian02boargoog/page/n112] & 
\egw{\textbf{كل هذه التمثيلات \underline{الروحانية} هي ببساطة لا شيء}. إنها غير كاملة وغير صحيحة. إنها تضعف وتقلل من الجلال الذي لا يمكن مقارنته بأي تشبيه أرضي. \textbf{لا يمكن مقارنة الله بالأشياء التي صنعتها يداه}. هذه مجرد أشياء أرضية، تعاني تحت لعنة الله بسبب خطايا الإنسان. \textbf{لا يمكن وصف الآب بأشياء الأرض}.}[Ms21-1906.9; 1906][https://egwwritings.org/read?panels=p9754.15] \\ \hline
\end{tabular}
}
\end{table}


In this comparison, it is clear who God is for William Boardman, and who He is for Sister White. For Boardman, God is the Triune God, a tri-personality of the one God. For Sister White, God is the Father. For Boardman, these representations are imperfect because they \others{rather hide than illustrate the tri-personality of the one God}, and for Sister White these representations are imperfect because \egw{The Father cannot be described by the things of earth}. For Boardman, God is the \textit{Triune God}; for Sister White, God is \textit{the Father}.


في هذه المقارنة، من الواضح من هو الله بالنسبة لوليام بوردمان، ومن هو بالنسبة للأخت وايت. بالنسبة لبوردمان، الله هو الله الثالوثي، شخصانية ثلاثية للإله الواحد. بالنسبة للأخت وايت، الله هو الآب. بالنسبة لبوردمان، هذه التمثيلات غير كاملة لأنها \others{تخفي بدلاً من أن توضح شخصانية الإله الواحد الثلاثية}، وبالنسبة للأخت وايت هذه التمثيلات غير كاملة لأن \egw{الآب لا يمكن وصفه بأشياء الأرض}. بالنسبة لبوردمان، الله هو \textit{الله الثالوثي}؛ بالنسبة للأخت وايت، الله هو \textit{الآب}.


Boardman’s only point that Ellen White affirms is that these representations are imperfect. Surely, William Boardman would not agree with Ellen White that these representations are \textit{spiritualistic} and \textit{untrue}. On the contrary, he believes that these illustrations \others{illustrate the truth that all the fulness of Him who filleth all in all, dwells in each person of Triune God}. To say that Ellen White agreed with such sentiment is gross misrepresentation.


النقطة الوحيدة التي تؤكدها إلين وايت من كلام بوردمان هي أن هذه التمثيلات غير كاملة. بالتأكيد، لن يوافق وليام بوردمان إلين وايت على أن هذه التمثيلات \textit{روحانية} و\textit{غير صحيحة}. على العكس من ذلك، فهو يعتقد أن هذه التوضيحات \others{توضح حقيقة أن كل ملء الذي يملأ الكل في الكل، يسكن في كل شخص من الله الثالوثي}. القول بأن إلين وايت وافقت على مثل هذا الرأي هو تحريف فادح.


The context of this important quotation prompts important questions. Why does the prophet of God refer to the representations that illustrate the \others{tri-personality of the one God} as \egwinline{spiritualistic representations}, which illustrate the sentiment that \egwinline{is not to be trusted}? Or why does the prophet of God refer to the representations that \others{represent the living personalities of the living God} as \egwinline{spiritualistic representations}? Or why does the prophet of God, when referring to the representations that \others{illustrate the truth that all the fullness of Him who filleth all in all, dwells in each person of Triune God}, refer to them as \egwinline{spiritualistic representations}? All of these spiritualistic representations illustrate the sentiment that \egwinline{is not to be trusted}. This sentiment is clearly the trinitarian sentiment.


يثير سياق هذا الاقتباس المهم أسئلة مهمة. لماذا تشير نبية الله إلى التمثيلات التي توضح \others{شخصانية الله الثلاثية الواحد} بأنها \egwinline{تمثيلات روحانية}، والتي توضح الرأي الذي \egwinline{لا يمكن الوثوق به}؟ أو لماذا تشير نبية الله إلى التمثيلات التي \others{تمثل الشخصيات الحية للإله الحي} بأنها \egwinline{تمثيلات روحانية}؟ أو لماذا تشير نبية الله، عند الإشارة إلى التمثيلات التي \others{توضح حقيقة أن كل ملء الذي يملأ الكل في الكل، يسكن في كل شخص من الله الثالوث}، إليها بأنها \egwinline{تمثيلات روحانية}؟ كل هذه التمثيلات الروحانية توضح الرأي الذي \egwinline{لا يمكن الوثوق به}. هذا الرأي هو بوضوح الرأي الثالوثي.


Sister White continues to follow Boardman’s line of reasoning and corrects the error.


تواصل الأخت وايت اتباع خط تفكير بوردمان وتصحح الخطأ.


\begin{table}[H]
\centering
\renewcommand{\arraystretch}{1.5}
\setlength{\tabcolsep}{15pt}
\resizebox{\textwidth}{!}{
\begin{tabular}{|p{0.4\textwidth}|p{0.4\textwidth}|}
\hline
\multicolumn{1}{|c|}{\textbf{William Boardman}} & \multicolumn{1}{c|}{\textbf{Ellen G. White}} \\ \hline
\othersQuote{The Father is fullness of the Godhead \textbf{invisibly}, \textbf{\underline{without form}}, whom \textbf{no creature hath seen \underline{or can see}}.}[p.100][https://archive.org/details/higherchristian02boargoog/page/n108/]


\begin{table}[H]
\centering
\renewcommand{\arraystretch}{1.5}
\setlength{\tabcolsep}{15pt}
\resizebox{\textwidth}{!}{
\begin{tabular}{|p{0.4\textwidth}|p{0.4\textwidth}|}
\hline
\multicolumn{1}{|c|}{\textbf{ويليام بوردمان}} & \multicolumn{1}{c|}{\textbf{إلين وايت}} \\ \hline
\othersQuote{الآب هو ملء اللاهوت \textbf{بشكل غير مرئي}، \textbf{\underline{بدون شكل}}، الذي \textbf{لم يره أي مخلوق \underline{ولا يمكن أن يراه}}.}[p.100][https://archive.org/details/higherchristian02boargoog/page/n108/]


\othersQuote{The Father is all the fullness of the Godhead \textbf{INVISIBLE}.}[p.105][https://archive.org/details/higherchristian02boargoog/page/n112/] & 
\egw{The Father is all the fulness of the Godhead \textbf{\underline{bodily}}, and is \textbf{invisible to mortal sight}.}[Ms21-1906.9; 1906][https://egwwritings.org/read?panels=p9754.15] \\ \hline
\end{tabular}
}
\end{table}


\othersQuote{الآب هو كل ملء اللاهوت \textbf{غير المرئي}.}[p.105][https://archive.org/details/higherchristian02boargoog/page/n112/] & 
\egw{الآب هو كل ملء اللاهوت \textbf{\underline{جسديًا}}، وهو \textbf{غير مرئي للنظر البشري الفاني}.}[Ms21-1906.9; 1906][https://egwwritings.org/read?panels=p9754.15] \\ \hline
\end{tabular}
}
\end{table}


For Boardman, the Father does not have a form nor body and is invisible to all creatures. For Sister White, the Father has a form and body and is invisible only to mortal human beings.\footnote{When Sister White talks about mortals, she talks about sin polluted humanity. After the restoration of humanity, at the resurrection, Christ will give His immortal life to His children. For more information read \href{https://egwwritings.org/?ref=en_RH.July.5.1887.par.5}{EGW, RH July 5, 1887, par. 5; 1887}.}


بالنسبة لبوردمان، الآب ليس له شكل ولا جسد وهو غير مرئي لجميع المخلوقات. بالنسبة للأخت وايت، الآب له شكل وجسد وهو غير مرئي فقط للبشر الفانين.\footnote{عندما تتحدث الأخت وايت عن الفانين، فإنها تتحدث عن البشرية الملوثة بالخطيئة. بعد استعادة البشرية، عند القيامة، سيمنح المسيح حياته الخالدة لأبنائه. لمزيد من المعلومات اقرأ \href{https://egwwritings.org/?ref=en_RH.July.5.1887.par.5}{EGW, RH July 5, 1887, par. 5; 1887}.}


This quotation is one of the most direct quotations regarding the \emcap{personality of God}. \egwinline{The Father is all the fullness of the Godhead \textbf{bodily}}[Ms21-1906.9; 1906][https://egwwritings.org/read?panels=p9754.16].


هذا الاقتباس هو أحد أكثر الاقتباسات المباشرة المتعلقة بـ \emcap{شخصانية الله}. \egwinline{الآب هو كل ملء اللاهوت \textbf{جسديًا}}[Ms21-1906.9; 1906][https://egwwritings.org/read?panels=p9754.16].


It might be confusing to someone that the Father is all the fullness of the Godhead bodily because in \textit{Colossians 2:9}, when referring to Jesus, it is written that \bible{in him dwelleth all the fulness of the Godhead bodily.} Scripture does not contradict itself. \textit{Colossians 2:9} does not exclude the Father to be all the fulness of the Godhead bodily. Various places in the Bible describe the Father having a body (\textit{a form: Daniel 7:9,10; Revelation 4:2,3; 1 Kings 22:19-22; a shape: John 5:37}). He has the appearance of a man (\textit{Ezekiel 1:26-28}). He has a face (\textit{Exodus 33:20; Matthew 18:10; Revelation 22:3, 4}). However, the Bible is completely silent about the nature of its substance. The Bible teaches us that \bible{\textbf{The secret things belong unto the LORD our God}: \textbf{but those things which \underline{are revealed} belong unto us and to our children for ever}, that we may do all the words of this law}[Deuteronomy 29:29]. It is revealed to us that the Father has body, He is all the fulness of the Godhead bodily. Also, it is revealed that in Jesus also dwells all the fulness of the Godhead bodily, because \bible{it pleased the Father that in him should all fulness dwell}[Colossians 1:19]. This is not a contradiction whatsoever because the Son is \bible{the \textbf{express image of \underline{His person}}}[Hebrews 1:3].


قد يكون مربكًا لشخص ما أن الآب هو كل ملء اللاهوت جسديًا لأنه في \textit{كولوسي 2:9}، عند الإشارة إلى يسوع، كُتب أنه \bible{فيه يحل كل ملء اللاهوت جسديًا.} الكتاب المقدس لا يناقض نفسه. \textit{كولوسي 2:9} لا يستثني الآب من أن يكون كل ملء اللاهوت جسديًا. تصف أماكن مختلفة في الكتاب المقدس الآب بأن له جسد (\textit{شكل: دانيال 7:9،10؛ رؤيا 4:2،3؛ 1 ملوك 22:19-22؛ هيئة: يوحنا 5:37}). له مظهر إنسان (\textit{حزقيال 1:26-28}). له وجه (\textit{خروج 33:20؛ متى 18:10؛ رؤيا 22:3، 4}). ومع ذلك، فإن الكتاب المقدس صامت تمامًا حول طبيعة جوهره. يعلمنا الكتاب المقدس أن \bible{\textbf{السرائر للرب إلهنا}: \textbf{والمعلنات لنا ولبنينا إلى الأبد}، لنعمل بجميع كلمات هذه الشريعة}[تثنية 29:29]. لقد أُعلن لنا أن الآب له جسد، وهو كل ملء اللاهوت جسديًا. كما أُعلن أن في يسوع أيضًا يسكن كل ملء اللاهوت جسديًا، لأنه \bible{فيه سر أن يحل كل الملء}[كولوسي 1:19]. هذا ليس تناقضًا على الإطلاق لأن الابن هو \bible{\textbf{رسم \underline{جوهره}}}[عبرانيين 1:3].


\begin{table}[H]
\centering
\renewcommand{\arraystretch}{1.5}
\setlength{\tabcolsep}{15pt}
\resizebox{\textwidth}{!}{
\begin{tabular}{|p{0.4\textwidth}|p{0.4\textwidth}|}
\hline
\multicolumn{1}{|c|}{\textbf{William Boardman}} & \multicolumn{1}{c|}{\textbf{Ellen G. White}} \\ \hline
\othersQuote{The Son is the fullness of the Godhead \textbf{embodied, that his creatures may see him, and know him, and trust him}.}[p.100][https://archive.org/details/higherchristian02boargoog/page/n108/]


\begin{table}[H]
\centering
\renewcommand{\arraystretch}{1.5}
\setlength{\tabcolsep}{15pt}
\resizebox{\textwidth}{!}{
\begin{tabular}{|p{0.4\textwidth}|p{0.4\textwidth}|}
\hline
\multicolumn{1}{|c|}{\textbf{ويليام بوردمان}} & \multicolumn{1}{c|}{\textbf{إلين وايت}} \\ \hline
\othersQuote{الابن هو ملء اللاهوت \textbf{متجسدًا، حتى يمكن لمخلوقاته أن يروه، ويعرفوه، ويثقوا به}.}[p.100][https://archive.org/details/higherchristian02boargoog/page/n108/]


\othersQuote{The Son is all the fulness of the Godhead \textbf{MANIFESTED}.}[p.105][https://archive.org/details/higherchristian02boargoog/page/n112/] & 
\egw{The Son is all the fulness of the Godhead \textbf{manifested}. The Word of God declares Him to be ‘\textbf{the express image of His person}’. ‘God so loved the world that He gave \textbf{His only begotten Son}, that whosoever believeth in Him should not perish, but have everlasting life’. \textbf{Here is shown \underline{the personality of the Father}}.}[Ms21-1906.10; 1906][https://egwwritings.org/read?panels=p9754.17] \\ \hline
\end{tabular}
}
\end{table}


\othersQuote{الابن هو كل ملء اللاهوت \textbf{المُعلَن}.}[p.105][https://archive.org/details/higherchristian02boargoog/page/n112/] & 
\egw{الابن هو كل ملء اللاهوت \textbf{المُعلَن}. كلمة الله تعلن أنه ‘\textbf{رسم جوهره}’. ‘لأنه هكذا أحب الله العالم حتى بذل \textbf{ابنه الوحيد}، لكي لا يهلك كل من يؤمن به بل تكون له الحياة الأبدية’. \textbf{هنا تظهر \underline{شخصانية الآب}}.}[Ms21-1906.10; 1906][https://egwwritings.org/read?panels=p9754.17] \\ \hline
\end{tabular}
}
\end{table}


Sister White focused on the \emcap{personality of God}, which is the personality of the Father. In Christ, who is \egwinline{begotten in the express image of the Father’s person}[ST May 30, 1895, par. 3; 1895][https://egwwritings.org/read?panels=p820.12891], is shown the personality of the Father. In the same way that Jesus is a person, so is the Father. The quality or state of Christ being a person is the same quality or state of the Father being a person. As Christ is a personal being, so is the Father. Just as all the fullness of the Godhead bodily dwells in Christ, so it does in the Father, because Christ is begotten in the express image of the Father’s person. In Him is shown the personality of the Father. These simple conclusions have been asserted by Scripture in John 3:16 and Hebrews 1:3.


ركزت الأخت وايت على \emcap{شخصانية الله}، التي هي شخصانية الآب. في المسيح، الذي هو \egwinline{مولود على صورة شخص الآب}[ST May 30, 1895, par. 3; 1895][https://egwwritings.org/read?panels=p820.12891]، تظهر شخصانية الآب. بنفس الطريقة التي يكون فيها يسوع شخصًا، كذلك الآب. الصفة أو الحالة التي يكون بها المسيح شخصًا هي نفس الصفة أو الحالة التي يكون بها الآب شخصًا. كما أن المسيح كائن شخصي، كذلك الآب. تمامًا كما يسكن كل ملء اللاهوت جسديًا في المسيح، كذلك في الآب، لأن المسيح مولود على صورة شخص الآب. فيه تظهر شخصانية الآب. هذه الاستنتاجات البسيطة تم تأكيدها من قبل الكتاب المقدس في يوحنا 3:16 وعبرانيين 1:3.


Does the same reasoning, of the personality of the Father and Son, apply to the Holy Spirit? Speaking of the Holy Spirit, Sister White continues:


هل ينطبق نفس المنطق، المتعلق بشخصانية الآب والابن، على الروح القدس؟ عندما تتحدث عن الروح القدس، تواصل الأخت وايت:


\egw{\textbf{The Comforter that Christ} promised to send after He ascended to heaven, \textbf{is the Spirit \underline{in} all the fulness of the Godhead}, making manifest the power of divine grace to all who receive and believe in Christ as a personal Saviour.}[Ms21-1906.11; 1906][https://egwwritings.org/read?panels=p9754.18]


\egw{\textbf{المعزي الذي وعد المسيح} بإرساله بعد صعوده إلى السماء، \textbf{هو الروح \underline{في} كل ملء اللاهوت}، الذي يُظهر قوة النعمة الإلهية لكل من يقبل المسيح ويؤمن به كمخلص شخصي.}[Ms21-1906.11; 1906][https://egwwritings.org/read?panels=p9754.18]


Sister White draws a distinction between Father and Son who \textbf{are}, individually, \textbf{all} the fullness of the Godhead, and the Spirit that is \textbf{in all} the fullness of the Godhead. This is a marked contrast to William Boardman’s reasoning, where all three are the fullness of the Godhead. Sister White does not follow this trinitarian fashion. The explanation is simple in light of the \emcap{personality of God} and of Christ. The Holy Spirit is a spirit, and the spirit dwells \textbf{in} the flesh/body. The Holy Spirit is \textbf{in all} the fullness of the Godhead\footnote{Take a look at the quotation from \href{https://egwwritings.org/?ref=en_Ms128-1897.13&para=5426.19}{{EGW, Ms128-1897.13; 1897}}, where Sister White states that the Father and the Son are the absolute Godhead.}.


ترسم الأخت وايت تمييزاً بين الآب والابن اللذين \textbf{هما}، بشكل فردي، \textbf{كل} ملء اللاهوت، والروح الذي هو \textbf{في كل} ملء اللاهوت. هذا تناقض واضح مع منطق ويليام بوردمان، حيث الثلاثة جميعاً هم ملء اللاهوت. الأخت وايت لا تتبع هذا النمط الثالوثي. التفسير بسيط في ضوء \emcap{شخصانية الله} والمسيح. الروح القدس هو روح، والروح يسكن \textbf{في} الجسد. الروح القدس هو \textbf{في كل} ملء اللاهوت\footnote{انظر إلى الاقتباس من \href{https://egwwritings.org/?ref=en_Ms128-1897.13&para=5426.19}{{EGW, Ms128-1897.13; 1897}}، حيث تذكر الأخت وايت أن الآب والابن هما اللاهوت المطلق.}.


Finally, the quotation continues to its most renowned part:


أخيراً، يستمر الاقتباس إلى جزئه الأكثر شهرة:


\begin{table}[H]
    \centering
    \renewcommand{\arraystretch}{1.5}
    \setlength{\tabcolsep}{15pt}
    \resizebox{\textwidth}{!}{
    \begin{tabular}{|p{0.4\textwidth}|p{0.4\textwidth}|}
    \hline
    \multicolumn{1}{|c|}{\textbf{William Boardman}} & \multicolumn{1}{c|}{\textbf{Ellen G. White}} \\ \hline
    \othersQuote{\textbf{The Father} is all the fulness of the Godhead INVISIBLE.}


\begin{table}[H]
    \centering
    \renewcommand{\arraystretch}{1.5}
    \setlength{\tabcolsep}{15pt}
    \resizebox{\textwidth}{!}{
    \begin{tabular}{|p{0.4\textwidth}|p{0.4\textwidth}|}
    \hline
    \multicolumn{1}{|c|}{\textbf{ويليام بوردمان}} & \multicolumn{1}{c|}{\textbf{إلين وايت}} \\ \hline
    \othersQuote{\textbf{الآب} هو كل ملء اللاهوت غير المرئي.}


\othersQuote{\textbf{The Son} is all the fulness of the Godhead MANIFESTED.}


\othersQuote{\textbf{الابن} هو كل ملء اللاهوت المُعلَن.}


\othersQuote{\textbf{The Spirit} is all the fulness of the Godhead MAKING MANIFEST.}


\othersQuote{\textbf{الروح} هو كل ملء اللاهوت الذي يُعلِن.}


\othersQuote{\textbf{The persons} are not mere offices, or modes of revelation, \textbf{but living persons of the living God}.}[p.105][https://archive.org/details/higherchristian02boargoog/page/n112/] & 
    \egw{There are \textbf{three living persons of the heavenly trio}; in the name of these three great powers—\textbf{the Father, the Son, and the Holy Spirit}—those who receive Christ by living faith are baptized, and these powers will co-operate with the obedient subjects of heaven in their efforts to live the new life in Christ.}[Ms21-1906.11; 1906][https://egwwritings.org/read?panels=p9754.18] \\ \hline
    \end{tabular}
    }
    \end{table}


\othersQuote{\textbf{الأشخاص} ليسوا مجرد مناصب، أو أساليب للإعلان، \textbf{بل أشخاص أحياء للإله الحي}.}[p.105][https://archive.org/details/higherchristian02boargoog/page/n112/] & 
    \egw{هناك \textbf{ثلاثة أشخاص أحياء في الثلاثي السماوي}؛ باسم هذه القوى العظيمة الثلاث—\textbf{الآب والابن والروح القدس}—يتعمد الذين يقبلون المسيح بالإيمان الحي، وستتعاون هذه القوى مع الخاضعين المطيعين للسماء في جهودهم للعيش الحياة الجديدة في المسيح.}[Ms21-1906.11; 1906][https://egwwritings.org/read?panels=p9754.18] \\ \hline
    \end{tabular}
    }
    \end{table}


In light of the context of William Boardman’s book, we see a marked contrast between \others{three living persons of \textbf{one living God}}, which is the trinitarian sentiment, and \egwinline{the three living persons of \textbf{the heavenly trio}}, which is in accordance with the truth on the \emcap{personality of God}.


في ضوء سياق كتاب ويليام بوردمان، نرى تناقضاً واضحاً بين \others{ثلاثة أشخاص أحياء \textbf{لإله حي واحد}}، وهو الرأي الثالوثي، و\egwinline{الثلاثة أشخاص أحياء \textbf{للثلاثي السماوي}}، وهو ما يتوافق مع الحق حول \emcap{شخصانية الله}.


The word ‘\textit{trio}’ simply indicates the group of three. The \textit{“heavenly trio}” is represented by the Father, the Son, and the Holy Spirit. But, contrary to popular assumption, they do not make one living God. Three-in-one and one-in-three are concepts that do away with the \emcap{personality of God}. This is why Sister White referred to trinitarian sentiments as sentiments that \egwinline{are not to be trusted}[Ms21-1906.8; 1906][https://egwwritings.org/read?panels=p9754.15].


كلمة ‘\textit{ثلاثي}’ تشير ببساطة إلى مجموعة من ثلاثة. \textit{“الثلاثي السماوي”} يمثله الآب والابن والروح القدس. لكن، خلافاً للافتراض الشائع، هم لا يشكلون إلهاً حياً واحداً. ثلاثة في واحد وواحد في ثلاثة هي مفاهيم تتخلص من \emcap{شخصانية الله}. لهذا السبب أشارت الأخت وايت إلى الآراء الثالوثية بأنها آراء \egwinline{لا يمكن الوثوق بها}[Ms21-1906.8; 1906][https://egwwritings.org/read?panels=p9754.15].


Sister White never followed any trinitarian fashion—neither in words and expressions, nor in sentiments. There is an almost effortless research endeavor we encourage you to take: in the writings of Ellen White, search for standard trinitarian terms like “\textit{three are one},” “\textit{one are three},” “\textit{one in three},” “\textit{three in one},” or any of the permutations possible. In her impressive oeuvre you will not find a single occurrence of any of these, let alone the word ‘\textit{trinity}’ describing our God\footnote{There is but one occurrence, in the writings of Ellen White, of the word ‘\textit{trinity}’ referring to \egw{the lust of the flesh, the lust of the eyes and the pride of life}[Lt43-1898.25; 1898][https://egwwritings.org/read?panels=p4806.31]}. She never used these phrases that are necessary to explain the trinitarian sentiment. Examining the following quote, we can see why she never said that God is trinity.


لم تتبع الأخت وايت أبدًا أي موضة ثالوثية—لا في الكلمات والتعبيرات، ولا في الآراء. هناك مسعى بحثي سهل نشجعك على القيام به: في كتابات إلين وايت، ابحث عن المصطلحات الثالوثية القياسية مثل “\textit{الثلاثة هم واحد}،“ “\textit{الواحد هم ثلاثة}،“ “\textit{واحد في ثلاثة}،“ “\textit{ثلاثة في واحد}،“ أو أي من التبديلات الممكنة. في أعمالها المثيرة للإعجاب لن تجد حالة واحدة من أي من هذه، ناهيك عن كلمة ‘\textit{ثالوث}’ تصف إلهنا\footnote{هناك حالة واحدة فقط، في كتابات إلين وايت، لكلمة ‘\textit{ثالوث}’ تشير إلى \egw{شهوة الجسد، وشهوة العيون، وتعظم المعيشة}[Lt43-1898.25; 1898][https://egwwritings.org/read?panels=p4806.31]}. لم تستخدم أبدًا هذه العبارات الضرورية لشرح الشعور الثالوثي. من خلال فحص الاقتباس التالي، يمكننا أن نرى لماذا لم تقل أبدًا أن الله هو ثالوث.


\egw{The subject of \textbf{\underline{speculation} regarding \underline{God’s personality} \underline{we will not venture} to express}, \textbf{\underline{except in the language of the Word which represents His personality}}. There is to be no discussion over this question \textbf{lest God would give unmistakable revelation of \underline{what He is}} that would extinguish the one who dares venture on the holy ground in \textbf{his speculative theories}, as some ventured to do in opening the ark to see what was in it as its power and how God was manifested. The men were slain for their curiosity science.}[17LtMs, Ms 223, 1902, par. 16][https://egwwritings.org/read?panels=p14067.9124037&index=0]


\egw{موضوع \textbf{\underline{التكهن} بخصوص \underline{شخصانية الله} \underline{لن نجرؤ على التعبير عنه}}، \textbf{\underline{إلا بلغة الكلمة التي تمثل شخصانيته}}. يجب ألا تكون هناك مناقشة حول هذه المسألة \textbf{لئلا يعطي الله كشفًا لا لبس فيه عن \underline{ماهيته}} الذي من شأنه أن يفني من يجرؤ على المغامرة على الأرض المقدسة في \textbf{نظرياته التكهنية}، كما تجرأ البعض على فعل ذلك في فتح التابوت لمعرفة ما كان فيه وما هي قوته وكيف تجلى الله. قُتل الرجال بسبب فضولهم العلمي.}[17LtMs, Ms 223, 1902, par. 16][https://egwwritings.org/read?panels=p14067.9124037&index=0]


Did you catch that? There is to be no discussion over the question of what God is, \egwinline{lest God would give unmistakable revelation} of \egwinline{what He is}. To say “God is \_\_\_\_\_\_\_”, the blank must be filled with \egwinline{the language of the Word which represents His personality.} The Bible clearly teaches that God is a personal, spiritual being—a truth confirmed by Christ Himself in His revelations to Ellen White. This fits within the biblical language that describes God’s personality. However, according to above statement, can we say “\textit{God is trinity}?” No! That is not \egwinline{the language of the Word which represents His personality.} Therefore, within explored context, we can safely conclude that, the Trinitarian view of God is part of \egwinline{speculative theories} of \egwinline{what He is}.


هل لاحظت ذلك؟ يجب ألا تكون هناك مناقشة حول مسألة ماهية الله، \egwinline{لئلا يعطي الله كشفًا لا لبس فيه} عن \egwinline{ماهيته}. لنقول “الله هو \_\_\_\_\_\_\_“، يجب ملء الفراغ بـ \egwinline{لغة الكلمة التي تمثل شخصانيته.} يعلّم الكتاب المقدس بوضوح أن الله كائن شخصي روحي—وهي حقيقة أكدها المسيح نفسه في إعلاناته لإلين وايت. هذا يتناسب مع لغة الكتاب المقدس التي تصف شخصانية الله. ومع ذلك، وفقًا للبيان أعلاه، هل يمكننا أن نقول “\textit{الله هو ثالوث}؟“ لا! هذه ليست \egwinline{لغة الكلمة التي تمثل شخصانيته.} لذلك، ضمن السياق المستكشف، يمكننا أن نستنتج بأمان أن النظرة الثالوثية لله هي جزء من \egwinline{النظريات التكهنية} حول \egwinline{ماهيته}.


This being said, the phrase \egwinline{Heavenly Trio} is not a definition of what God is. Our God is the Father—not \egwinline{the Heavenly Trio.} The term Heavenly Trio does not serve as a replacement for the Trinitarian idea of \textit{three living persons of one God}. This becomes obvious, when we examine the context. Ellen White was instructed to warn us against Trinitarian sentiments, not to trust them. She was not endorsing them.


بعد قول هذا، فإن عبارة \egwinline{الثلاثي السماوي} ليست تعريفًا لماهية الله. إلهنا هو الآب—وليس \egwinline{الثلاثي السماوي.} مصطلح الثلاثي السماوي لا يعمل كبديل للفكرة الثالوثية عن \textit{ثلاثة أشخاص أحياء لإله واحد}. يصبح هذا واضحًا عندما ندرس السياق. تم توجيه إلين وايت لتحذيرنا من الآراء الثالوثية، وليس للثقة بها. لم تكن تؤيدها.


Although the illustrations Ellen White quoted were not from Dr. Kellogg, it seems that Kellogg's proponents, if not Kellogg himself, were defending him with William Boardman's sentiments. We do not have direct data to confirm this, but we do know that Dr. Kellogg raised \others{the theological side of questions of \textbf{the trinity and all that sort of things}.}[Interview, J. H. Kellogg, G. W. Amadon and A. C. Bourdeau, October 7th 1907 held at Kellogg’s residence][https://archive.org/details/KelloggVs.TheBrethrenHisLastInterviewAsAnAdventistoct71907/page/n37] The last three paragraphs in the heavenly trio manuscript \href{https://egwwritings.org/?ref=en_Ms21-1906&para=9754.1}{(Ms21-1906; 1906)} reveal the connection with Dr. Kellogg, which is another “smoking gun” of Dr. Kellogg's trinitarian stance.


على الرغم من أن التوضيحات التي اقتبستها إلين وايت لم تكن من الدكتور كيلوغ، يبدو أن مؤيدي كيلوغ، إن لم يكن كيلوغ نفسه، كانوا يدافعون عنه بآراء ويليام بوردمان. ليس لدينا بيانات مباشرة لتأكيد هذا، لكننا نعلم أن الدكتور كيلوغ أثار \others{الجانب اللاهوتي من أسئلة \textbf{الثالوث وكل ذلك النوع من الأشياء}.}[مقابلة، ج. هـ. كيلوغ، ج. و. أمادون و أ. س. بوردو، 7 أكتوبر 1907 عقدت في منزل كيلوغ][https://archive.org/details/KelloggVs.TheBrethrenHisLastInterviewAsAnAdventistoct71907/page/n37] تكشف الفقرات الثلاث الأخيرة في مخطوطة الثلاثي السماوي \href{https://egwwritings.org/?ref=en_Ms21-1906&para=9754.1}{(Ms21-1906; 1906)} عن الصلة مع الدكتور كيلوغ، وهو “دليل دامغ” آخر على موقف الدكتور كيلوغ الثالوثي.


\egw{I write this because any moment my life may be ended. \textbf{Unless there is a breaking away from the influence that Satan has prepared, and a \underline{reviving of the testimonies that God has given, souls will perish in their delusion}. They will accept fallacy after fallacy and will thus keep up a disunion that will always exist until those who have been deceived take \underline{their stand on the right platform}}. All this higher education that is being planned will be extinguished; for it is spurious. The more simple the education of our workers, the less connection they have with the men whom God is not leading, the more will be accomplished. \textbf{Work will be done in the \underline{simplicity} of true godliness, and the old, old times will be back when, under the Holy Spirit’s guidance, thousands were converted in a day. When the truth in its simplicity is lived in every place, then God will work through His angels as He worked on the day of Pentecost, and hearts will be changed so decidedly that there will be a manifestation of the influence of genuine truth, as is represented in the descent of the Holy Spirit}.}[Ms21-1906.18; 1906][https://egwwritings.org/read?panels=p9754.25]


\egw{أكتب هذا لأن حياتي قد تنتهي في أي لحظة. \textbf{ما لم يكن هناك ابتعاد عن التأثير الذي أعده الشيطان، و\underline{إحياء للشهادات التي أعطاها الله، ستهلك النفوس في ضلالها}. سيقبلون الباطل تلو الباطل وبذلك سيحافظون على انقسام سيظل موجودًا دائمًا حتى يتخذ المخدوعون \underline{موقفهم على المنصة الصحيحة}}. كل هذا التعليم العالي الذي يتم التخطيط له سينطفئ؛ لأنه مزيف. كلما كان تعليم عمالنا أبسط، وكلما قلت صلتهم بالرجال الذين لا يقودهم الله، كلما تم إنجاز المزيد. \textbf{سيتم العمل في \underline{بساطة} التقوى الحقيقية، وستعود الأزمنة القديمة القديمة عندما، تحت إرشاد الروح القدس، تم تحويل آلاف في يوم واحد. عندما يتم عيش الحق في بساطته في كل مكان، سيعمل الله من خلال ملائكته كما عمل في يوم الخمسين، وستتغير القلوب بشكل حاسم لدرجة أنه سيكون هناك إظهار لتأثير الحق الحقيقي، كما هو ممثل في نزول الروح القدس}.}[Ms21-1906.18; 1906][https://egwwritings.org/read?panels=p9754.25]


\egwnogap{The Holy Spirit never has and never will in the future divorce the medical missionary work from the gospel ministry. They cannot be divorced. Bound up with Jesus Christ, the ministry of the Word and the healing of the sick are one.}[Ms21-1906.19; 1906][https://egwwritings.org/read?panels=p9754.26]


\egwnogap{الروح القدس لم ولن يفصل أبدًا في المستقبل العمل الطبي الإرسالي عن خدمة الإنجيل. لا يمكن فصلهما. مرتبطان بيسوع المسيح، خدمة الكلمة وشفاء المرضى هما واحد.}[Ms21-1906.19; 1906][https://egwwritings.org/read?panels=p9754.26]


\egwnogap{The fifty-eighth chapter of Isaiah contains instruction for today. \textbf{‘Cry aloud, spare not, lift up thy voice like a trumpet, and show My people their transgression, and the house of Jacob their sin.’ God does not accept \underline{Dr. Kellogg as His laborer}, unless he will now break with Satan}. The work would not have been hindered, as it has been for the past several years, \textbf{if Dr. Kellogg were a converted man. ‘Come,’ I call, ‘come ye out and be separate from him and his associates whom he has leavened.’ I am now giving the message God has given me, to give to all who claim to believe the truth, \underline{‘Come ye out from among them, and be separate},’ else their sin in justifying wrongs and framing deceits will continue to be the ruin of souls. We cannot afford to be on the wrong side. We cannot afford to cover the truth with scientific problems. We urge that decided changes be made and no more stumbling blocks be placed before the feet of the people of God}. Let every soul put on the gospel shoes. \textbf{Let every soul pray and work, placing their feet upon \underline{the foundation Christ laid} in giving His life for the life of the world}.}[Ms21-1906.20; 1906][https://egwwritings.org/read?panels=p9754.27]


\egwnogap{يحتوي الإصحاح الثامن والخمسون من إشعياء على تعليمات لليوم. \textbf{‘نادِ بصوت عالٍ، لا تمسك، ارفع صوتك كبوق، وأخبر شعبي بمعصيتهم، وبيت يعقوب بخطيئتهم.’ الله لا يقبل \underline{الدكتور كيلوغ كعامل له}، ما لم ينفصل الآن عن الشيطان}. لم يكن العمل سيُعاق، كما حدث في السنوات القليلة الماضية، \textbf{لو كان الدكتور كيلوغ رجلاً متحولاً. ‘تعالوا،‘ أنادي، ‘اخرجوا وانفصلوا عنه وعن شركائه الذين خمّرهم.’ أنا الآن أعطي الرسالة التي أعطاني الله إياها، لأعطيها لكل من يدّعي أنه يؤمن بالحق، \underline{‘اخرجوا من بينهم، وكونوا منفصلين}،‘ وإلا فإن خطيئتهم في تبرير الأخطاء وتشكيل الخداع ستستمر في أن تكون سبب هلاك النفوس. لا يمكننا أن نتحمل أن نكون في الجانب الخطأ. لا يمكننا أن نتحمل تغطية الحق بالمشاكل العلمية. نحث على إجراء تغييرات حاسمة وعدم وضع المزيد من حجارة العثرة أمام أقدام شعب الله}. ليضع كل نفس حذاء الإنجيل. \textbf{ليصلي ويعمل كل نفس، واضعين أقدامهم على \underline{الأساس الذي وضعه المسيح} ببذل حياته من أجل حياة العالم}.}[Ms21-1906.20; 1906][https://egwwritings.org/read?panels=p9754.27]


The heavenly trio quotation was part of Kellogg's controversy. This is evidence that Kellogg’s controversy included the Trinity doctrine. We are told to break \egwinline{away from the influence of Satan} and to revive the \egw{testimony that God has given} us, or else our souls will perish in delusions. These influences and delusions come from trinitarians such as \textit{William Boardman} and \textit{Dr. John H. Kellogg}. She is pointing us back to place our feet upon the foundation that was built by the Masterworker.\footnote{\href{https://egwwritings.org/?ref=en_SpTB02.54.2&para=417.276}{EGW, SpTB02 54.2; 1904}}


كان اقتباس الثلاثي السماوي جزءًا من جدال كيلوغ. هذا دليل على أن جدال كيلوغ تضمن عقيدة الثالوث. نحن مدعوون للابتعاد \egwinline{عن تأثير الشيطان} وإحياء \egw{الشهادة التي أعطاها الله} لنا، وإلا ستهلك نفوسنا في الأوهام. هذه التأثيرات والأوهام تأتي من الثالوثيين مثل \textit{ويليام بوردمان} و\textit{الدكتور جون هـ. كيلوغ}. إنها تشير إلينا للعودة لوضع أقدامنا على الأساس الذي بناه المعلم الرئيسي.\footnote{\href{https://egwwritings.org/?ref=en_SpTB02.54.2&para=417.276}{EGW, SpTB02 54.2; 1904}}


We hope that this context exposes the false narrative of Ellen White's endorsement of the Trinity doctrine, propagated by our Adventist scholars. Dr. Kellogg was in apostasy for stepping off from the foundation of our faith, and the Trinity doctrine was his justification. With such data in mind, one must ask: If the Trinity was true, and Ellen White endorsed it, and this “true” Trinity was mixed with Dr. Kellogg's error, we should expect her to separate the Trinity from error. But this is not what she did. Instead, she consistently pointed us back to the foundation of our faith, where we had a clear teaching on the presence and the \emcap{personality of God}. But for the case of Trinity, she faithfully bore the message from Heaven: “\textit{\textbf{I am instructed to say}, the sentiments of those who are searching for \textbf{trinitarian ideas are not to be trusted}}.”


نأمل أن يكشف هذا السياق الرواية الكاذبة عن تأييد إلين وايت لعقيدة الثالوث، التي روج لها علماؤنا الأدفنتست. كان الدكتور كيلوغ في ارتداد لخروجه عن أساس إيماننا، وكانت عقيدة الثالوث تبريره. مع وجود مثل هذه البيانات في الاعتبار، يجب أن يسأل المرء: إذا كان الثالوث حقيقيًا، وأيدته إلين وايت، وكان هذا الثالوث “الحقيقي” مختلطًا مع خطأ الدكتور كيلوغ، فيجب أن نتوقع منها أن تفصل الثالوث عن الخطأ. لكن هذا ليس ما فعلته. بدلاً من ذلك، أشارت باستمرار إلى أساس إيماننا، حيث كان لدينا تعليم واضح عن وجود و\emcap{شخصانية الله}. ولكن في حالة الثالوث، حملت بأمانة الرسالة من السماء: “\textit{\textbf{أنا مكلفة بأن أقول}، إن آراء أولئك الذين يبحثون عن \textbf{أفكار ثالوثية لا يمكن الوثوق بها}}.”


% The Heavenly Trio

\begin{titledpoem}
    
    \stanza{
        In heaven’s realm, where truths unfold, \\
        A message clear, so brave and bold. \\
        God spoke through Ellen, clear and bright, \\
        Revealing depths of heavenly light.
    }

    \stanza{
        Misunderstood by some who read, \\
        Her words of God that all must heed. \\
        Not as triune, but trio three \\
        Distinct as persons, heavenly.
    }

    \stanza{
        The Father, not a formless feel, \\
        Invisible to us, yet real. \\
        He is the fullness, all complete, \\
        The Godhead, bodily, concrete.
    }

    \stanza{
        The Son, God’s fullness, manifest \\
        In Him, divinity does rest. \\
        God’s character, seen in His face, \\
        In Christ, we see His Father’s grace.
    }

    \stanza{
        The Spirit, in all fullness dwells, \\
        A mystery nature, Ellen tells. \\
        With forms, the Father and His Son \\
        With Them, in Spirit, we are one.
    }

    \stanza{
        Distinct and clear, Their roles unfold, \\
        The Father, Son, in form behold. \\
        Yet present everywhere we find, \\
        Their Spirit shows Their heart and mind.
    }

    \stanza{
        God’s message true, from up above. \\
        Reveals to us the Father’s love. \\
        To know this truth about our God— \\
        It lights the path that we must trod.
    }

    \stanza{
        Dear Ellen’s words, in context found, \\
        Reveal a truth that’s so profound \\
        Not trinity did she embrace, \\
        But trio persons in their place.
    }

    \stanza{
        The pillar stands, our platform firm, \\
        God’s personality we learn. \\
        The trio that is heavenly, \\
        Exposes falsehood—trinity.
    }
    
\end{titledpoem}