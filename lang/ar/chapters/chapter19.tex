\qrchapter{https://forgottenpillar.com/rsc/en-fp-chapter19}{Ellen White and Matthew 28:19}


\qrchapter{https://forgottenpillar.com/rsc/en-fp-chapter19}{إلين وايت ومتى 28:19}


Many assert that Ellen White promoted the Trinity doctrine, and that she is the one responsible for accepting it into our ranks. These claims do not consider that she defended the \emcap{personality of God} expressed in the first point of the \emcap{Fundamental Principles}. To support the claims that Ellen White was trinitarian, quotations are presented to her comment on Matthew 28:19:


يؤكد الكثيرون أن إلين وايت روجت لعقيدة الثالوث، وأنها هي المسؤولة عن قبولها في صفوفنا. هذه الادعاءات لا تأخذ بعين الاعتبار أنها دافعت عن \emcap{شخصانية الله} المعبر عنها في النقطة الأولى من \emcap{المبادئ الأساسية}. لدعم الادعاءات بأن إلين وايت كانت تؤمن بالثالوث، يتم تقديم اقتباسات لتعليقها على متى 28:19:


\bible{Go ye therefore, and teach all nations, \textbf{baptizing them in the name of \underline{the Father}, and of \underline{the Son}, and of \underline{the Holy Ghost}}.}[Matthew 28:19]


\bible{فَاذْهَبُوا وَتَلْمِذُوا جَمِيعَ الأُمَمِ \textbf{وَعَمِّدُوهُمْ بِاسْمِ \underline{الآبِ} وَ\underline{الاِبْنِ} وَ\underline{الرُّوحِ الْقُدُسِ}}.}[متى 28:19]


This verse has been most prominent in support of the Trinity doctrine. The Trinity doctrine has propositions about the \emcap{personality of God} of which this text says nothing to support. This verse itself does not teach that the Father, the Son, and the Holy Ghost, comprise \textit{one} God, the God of the Bible. There are other explicit verses in the Bible that exclude such interpretation of the text, i.e. 1 Corinthians 8:4-6; John 17:3; Ephesians 4:4-6; 1 Timothy 2:5.


كانت هذه الآية الأكثر بروزًا في دعم عقيدة الثالوث. تحتوي عقيدة الثالوث على مقترحات حول \emcap{شخصانية الله} لا يقول هذا النص شيئًا لدعمها. هذه الآية نفسها لا تعلّم أن الآب والابن والروح القدس يشكلون إلهًا \textit{واحدًا}، إله الكتاب المقدس. هناك آيات صريحة أخرى في الكتاب المقدس تستبعد مثل هذا التفسير للنص، مثل 1 كورنثوس 8:4-6؛ يوحنا 17:3؛ أفسس 4:4-6؛ 1 تيموثاوس 2:5.


Unfortunately, the same unsupported assumptions made about Matthew 28:19 are made about Sister White’s quotations dealing with this verse. For example, Sister White uses terms like \egwinline{three highest powers in heaven}[Lt253a-1903.18; 1903][https://egwwritings.org/read?panels=p10143.25], \egwinline{three great powers of heaven}[8T 254.1; 1904][https://egwwritings.org/read?panels=p112.1450], \egwinline{the three holy dignitaries of heaven}[Ms92-1901.26: 1901][https://egwwritings.org/read?panels=p10732.32] and similar expressions—none of these quotations justify the assumption that these three (the Father, the Son, and the Holy Spirit) make \textit{one} God. On the contrary, as discussed in the previous chapter, keeping William Boardman’s sentiments and \egwinline{the heavenly trio} in context, “\textit{three-in-one}” sentiments \egwinline{should not be trusted}[Ms21-1906.8; 1906][https://egwwritings.org/read?panels=p9754.15].


للأسف، نفس الافتراضات غير المدعومة التي تُطرح حول متى 28:19 تُطرح أيضًا حول اقتباسات الأخت وايت التي تتناول هذه الآية. على سبيل المثال، تستخدم الأخت وايت مصطلحات مثل \egwinline{ثلاث أعلى قوى في السماء}[Lt253a-1903.18; 1903][https://egwwritings.org/read?panels=p10143.25]، \egwinline{ثلاث قوى عظيمة في السماء}[8T 254.1; 1904][https://egwwritings.org/read?panels=p112.1450]، \egwinline{الشخصيات الثلاث المقدسة في السماء}[Ms92-1901.26: 1901][https://egwwritings.org/read?panels=p10732.32] وتعبيرات مماثلة - لا يوجد في أي من هذه الاقتباسات ما يبرر افتراض أن هؤلاء الثلاثة (الآب والابن والروح القدس) يشكلون إلهًا \textit{واحدًا}. على العكس من ذلك، كما نوقش في الفصل السابق، فإن الحفاظ على آراء ويليام بوردمان و\egwinline{الثلاثي السماوي} في سياقها، فإن آراء “\textit{ثلاثة في واحد}” \egwinline{لا ينبغي الوثوق بها}[Ms21-1906.8; 1906][https://egwwritings.org/read?panels=p9754.15].


The heavenly trio (the group of three: the Father, the Son and the Holy Spirit) are also present in other Bible verses, in addition to Matthew 28:19. There are several other instances in the New Testament where the Father, the Son and the Holy Spirit are mentioned, and these verses should be used to interpret the meaning behind the heavenly trio. None of the verses on the heavenly trio prove a three-in-one God; rather, all of them refer to the Father as one God. In the following verses, the heavenly trio is bolded in order to better distinguish the Father, the Son and the Holy Spirit.


الثلاثي السماوي (مجموعة الثلاثة: الآب والابن والروح القدس) موجود أيضًا في آيات كتابية أخرى، بالإضافة إلى متى 28:19. هناك عدة حالات أخرى في العهد الجديد حيث يُذكر الآب والابن والروح القدس، وينبغي استخدام هذه الآيات لتفسير معنى الثلاثي السماوي. لا تثبت أي من الآيات عن الثلاثي السماوي وجود إله ثلاثة في واحد؛ بل تشير جميعها إلى الآب كإله واحد. في الآيات التالية، تم تمييز الثلاثي السماوي بالخط العريض لتمييز الآب والابن والروح القدس بشكل أفضل.


\bible{There is one body, and \textbf{one Spirit}, even as ye are called in one hope of your calling; \textbf{One Lord}, one faith, one baptism, \textbf{One God and Father} of all, who is above all, and through all, and in you all.}[Ephesians 4:4-6]


\bible{جَسَدٌ وَاحِدٌ، \textbf{وَرُوحٌ وَاحِدٌ}، كَمَا دُعِيتُمْ أَيْضًا فِي رَجَاءِ دَعْوَتِكُمُ الْوَاحِدِ. \textbf{رَبٌّ وَاحِدٌ}، إِيمَانٌ وَاحِدٌ، مَعْمُودِيَّةٌ وَاحِدَةٌ، \textbf{إِلهٌ وَآبٌ وَاحِدٌ} لِلْكُلِّ، الَّذِي عَلَى الْكُلِّ وَبِالْكُلِّ وَفِي كُلِّكُمْ.}[أفسس 4:4-6]


\bible{Now there are diversities of gifts, but the \textbf{same Spirit}. And there are differences of administrations, but the \textbf{same Lord}. And there are diversities of operations, but it is \textbf{the same God} which worketh all in all.}[1 Corinthians 12:4-6]


\bible{فَأَنْوَاعُ مَوَاهِبَ مَوْجُودَةٌ، وَلكِنَّ \textbf{الرُّوحَ وَاحِدٌ}. وَأَنْوَاعُ خِدَمٍ مَوْجُودَةٌ، وَلكِنَّ \textbf{الرَّبَّ وَاحِدٌ}. وَأَنْوَاعُ أَعْمَال مَوْجُودَةٌ، وَلكِنَّ \textbf{اللهَ وَاحِدٌ}، الَّذِي يَعْمَلُ الْكُلَّ فِي الْكُلِّ.}[1 كورنثوس 12:4-6]


\bible{The grace of \textbf{the Lord Jesus Christ}, and the love of \textbf{God}, and the communion of \textbf{the Holy Ghost}, be with you all. Amen.}[2 Corinthians 13:14]


\bible{نِعْمَةُ \textbf{رَبِّنَا يَسُوعَ الْمَسِيحِ}، وَمَحَبَّةُ \textbf{اللهِ}، وَشَرِكَةُ \textbf{الرُّوحِ الْقُدُسِ} مَعَ جَمِيعِكُمْ. آمِينَ.}[2 كورنثوس 13:14]


\bible{For through \textbf{him} \normaltext{[Christ]} we both have access by one \textbf{Spirit} unto the \textbf{Father}.}[Ephesians 2:18]


\bible{لأَنَّ \textbf{بِهِ} \normaltext{[المسيح]} لَنَا كِلَيْنَا قُدُومًا فِي \textbf{رُوحٍ} وَاحِدٍ إِلَى \textbf{الآبِ}.}[أفسس 2:18]


\bible{But we are bound to give thanks alway to \textbf{God} for you, brethren beloved of \textbf{the Lord}, because \textbf{God} hath from the beginning chosen you to salvation through sanctification of \textbf{the Spirit} and belief of the truth.}[2 Thessalonians 2:13]


\bible{وَأَمَّا نَحْنُ فَيَنْبَغِي لَنَا أَنْ نَشْكُرَ \textbf{اللهَ} كُلَّ حِينٍ لأَجْلِكُمْ أَيُّهَا الإِخْوَةُ الْمَحْبُوبُونَ مِنَ \textbf{الرَّبِّ}، أَنَّ \textbf{اللهَ} اخْتَارَكُمْ مِنَ الْبَدْءِ لِلْخَلاَصِ، بِتَقْدِيسِ \textbf{الرُّوحِ} وَتَصْدِيقِ الْحَقِّ.}[2 تسالونيكي 2:13]


\bible{How much more shall the blood of \textbf{Christ}, who through the eternal \textbf{Spirit} offered himself without spot to \textbf{God}, purge your conscience from dead works to serve \textbf{the living God}?}[Hebrews 9:14]


\bible{فَكَمْ بِالْحَرِيِّ يَكُونُ دَمُ \textbf{الْمَسِيحِ}، الَّذِي بِرُوحٍ أَزَلِيٍّ قَدَّمَ نَفْسَهُ للهِ بِلاَ عَيْبٍ، يُطَهِّرُ ضَمَائِرَكُمْ مِنْ أَعْمَالٍ مَيِّتَةٍ لِتَخْدِمُوا \textbf{اللهَ الْحَيَّ}!}[عبرانيين 9:14]


\bible{Elect according to the foreknowledge of \textbf{God the Father}, through sanctification of \textbf{the Spirit}, unto obedience and sprinkling of the blood of \textbf{Jesus Christ}: Grace unto you, and peace, be multiplied.}[1 Peter 1:2]


\bible{بِمُقْتَضَى عِلْمِ \textbf{اللهِ الآبِ} السَّابِقِ، فِي تَقْدِيسِ \textbf{الرُّوحِ} لِلطَّاعَةِ، وَرَشِّ دَمِ \textbf{يَسُوعَ الْمَسِيحِ}. لِتُكْثَرْ لَكُمُ النِّعْمَةُ وَالسَّلاَمُ.}[1 بطرس 1:2]


All of the above verses talk about the heavenly trio (the Father, the Son and the Holy Spirit), and all of them consistently testify that the Father is the one referred to as God.
The same reasoning holds ground for Ellen White’s interpretation of Matthew 28:19.


جميع الآيات المذكورة أعلاه تتحدث عن الثلاثي السماوي (الآب والابن والروح القدس)، وكلها تشهد باستمرار أن الآب هو المشار إليه بالله.
ينطبق نفس المنطق على تفسير إلين وايت لمتى 28:19.


\egw{Christ gave His followers a positive promise that after His ascension He would send them His Spirit. ‘Go ye therefore,’ He said, ‘and teach all nations, baptizing them in the name of \textbf{the Father (a personal God),} and of \textbf{the Son (a personal Prince and Saviour),} and of \textbf{the Holy Ghost (sent from heaven to represent Christ);} teaching them to observe all things whatsoever I have commanded you, and, lo, I am with you alway, even unto the end of the world.’ Matthew 28:19, 20.}[RH October 26, 1897, par. 9; 1897][https://egwwritings.org/read?panels=p821.16317]


\egw{أعطى المسيح لأتباعه وعدًا إيجابيًا بأنه بعد صعوده سيرسل لهم روحه. “فاذهبوا” قال “وتلمذوا جميع الأمم وعمدوهم باسم \textbf{الآب (إله شخصي)،} وباسم \textbf{الابن (أمير ومخلص شخصي)،} وباسم \textbf{الروح القدس (المرسل من السماء ليمثل المسيح)؛} وعلموهم أن يحفظوا جميع ما أوصيتكم به. وها أنا معكم كل الأيام إلى انقضاء الدهر”. متى 28:19، 20.}[RH October 26, 1897, par. 9; 1897][https://egwwritings.org/read?panels=p821.16317]


The brackets in this quotation are in the original manuscript written by Ellen White. Here, she gives her own interpretation of Matthew 28:19. The Father is a personal God, the Son is a personal Prince and Saviour, and the Holy Spirit is Christ’s representative. This interpretation is in harmony with the \emcap{personality of God} expressed in the first point of the \emcap{Fundamental Principles}. Matthew 28:19 is a matter of interpretation. The interpretation which makes the Heavenly Trio one God is not inspired. This is not what the text indicates. Rather, let's read Matthew 28:19 within inspired compound: “\textit{Go ye therefore, and teach all nations, baptizing them in the name of a personal God, a personal Prince and Savior, and of the Holy Ghost}.” If one would read the text as such, no one would ever assume that one God is a unity of three persons. Therefore, let's stick to the inspiration, rather than subterfuge\footnote{\href{https://egwwritings.org/?ref=en\_Lt232-1903.41&para=10197.50}{{EGW, Lt232-1903.41; 1903}}}.


الأقواس في هذا الاقتباس موجودة في المخطوطة الأصلية التي كتبتها إلين وايت. هنا، تقدم تفسيرها الخاص لمتى 28:19. الآب هو إله شخصي، والابن هو أمير ومخلص شخصي، والروح القدس هو ممثل المسيح. هذا التفسير يتوافق مع \emcap{شخصانية الله} المعبر عنها في النقطة الأولى من \emcap{المبادئ الأساسية}. متى 28:19 هي مسألة تفسير. التفسير الذي يجعل الثلاثي السماوي إلهًا واحدًا ليس موحى به. هذا ليس ما يشير إليه النص. بل لنقرأ متى 28:19 ضمن المركب الموحى به: “\textit{فاذهبوا وتلمذوا جميع الأمم وعمدوهم باسم إله شخصي، وأمير ومخلص شخصي، والروح القدس}.” لو قرأ أحد النص على هذا النحو، لما افترض أبدًا أن إلهًا واحدًا هو وحدة من ثلاثة أشخاص. لذلك، دعونا نلتزم بالوحي، بدلاً من الحيلة\footnote{\href{https://egwwritings.org/?ref=en\_Lt232-1903.41&para=10197.50}{{EGW, Lt232-1903.41; 1903}}}.


\egw{Let them be thankful to God for His manifold mercies and be kind to one another. \textbf{They have \underline{one God} and \underline{one Saviour}; and \underline{one Spirit}—\underline{the Spirit of Christ}—is to bring unity into their ranks}.}[9T 189.3; 1909][https://egwwritings.org/read?panels=p115.1057]


\egw{ليكونوا شاكرين لله على مراحمه المتعددة وليكونوا لطفاء بعضهم نحو بعض. \textbf{لديهم \underline{إله واحد} و\underline{مخلص واحد}؛ و\underline{روح واحد}—\underline{روح المسيح}—ليجلب الوحدة إلى صفوفهم}.}[9T 189.3; 1909][https://egwwritings.org/read?panels=p115.1057]


In light of the presented evidence, we see that simply numbering the Father, the Son and the Holy Spirit, does not prove the \textit{three-in-one} assumption, nor is it in conflict with the \emcap{personality of God} expressed in the \emcap{Fundamental Principles}. There is no denial of three persons of the Godhead, but only a denial of the assumption that these Three Great Worthies make one God.


في ضوء الأدلة المقدمة، نرى أن مجرد تعداد الآب والابن والروح القدس، لا يثبت افتراض \textit{الثلاثة في واحد}، ولا يتعارض مع \emcap{شخصانية الله} المعبر عنها في \emcap{المبادئ الأساسية}. لا يوجد إنكار لثلاثة أشخاص في اللاهوت، ولكن فقط إنكار للافتراض بأن هؤلاء الثلاثة العظماء الجديرين يشكلون إلهًا واحدًا.


Matthew 28:19 is a valuable verse and it opens a new field of study within the Bible and the Spirit of Prophecy. In the context of the Living Temple, and referring to its sentiments, Sister White wrote that this verse should be studied most earnestly because it is not half understood.


متى 28:19 هي آية قيمة وتفتح مجالًا جديدًا للدراسة في الكتاب المقدس وروح النبوة. في سياق كتاب ذا ليفينغ تمبل، وبالإشارة إلى آرائه، كتبت الأخت وايت أن هذه الآية يجب دراستها بجدية شديدة لأنها ليست مفهومة تمامًا.


\egw{Just before His ascension, Christ gave His disciples a wonderful presentation, \textbf{as recorded in the twenty-eighth chapter of Matthew}. \textbf{This chapter contains instruction} that our ministers, our \textbf{physicians}, our youth, and all our church members need to \textbf{study most \underline{earnestly}}. \textbf{Those who study this instruction as they should will \underline{not dare to advocate theories that have no foundation in the Word of God}}. My brethren and sisters, make the Scriptures, which contain the alpha and omega of knowledge, your study. \textbf{All through the Old Testament and the New, there are things \underline{that are not half understood}}. ‘And Jesus came and spake unto them, saying, All power is given unto Me in heaven and in earth. Go ye therefore, and teach all nations, \textbf{baptizing them in the name of the Father, and of the Son, and of the Holy Ghost}; teaching them to observe all things whatsoever I have commanded you; and, lo, I am with you alway, even unto the end of the world.’ [Verses 18-20.]}[Lt214-1906.10; 1906][https://egwwritings.org/read?panels=p10171.16]


\egw{قبل صعوده مباشرة، قدم المسيح لتلاميذه عرضًا رائعًا، \textbf{كما هو مسجل في الفصل الثامن والعشرين من إنجيل متى}. \textbf{يحتوي هذا الفصل على تعليمات} يحتاج وزراؤنا و\textbf{أطباؤنا} وشبابنا وجميع أعضاء كنيستنا إلى \textbf{دراستها \underline{بجدية}}. \textbf{أولئك الذين يدرسون هذه التعليمات كما ينبغي \underline{لن يجرؤوا على الدفاع عن نظريات ليس لها أساس في كلمة الله}}. إخوتي وأخواتي، اجعلوا الكتب المقدسة، التي تحتوي على ألفا وأوميغا المعرفة، دراستكم. \textbf{في جميع أنحاء العهد القديم والجديد، هناك أشياء \underline{ليست مفهومة تمامًا}}. “فتقدم يسوع وكلمهم قائلاً: دُفع إليّ كل سلطان في السماء وعلى الأرض. فاذهبوا وتلمذوا جميع الأمم، \textbf{وعمدوهم باسم الآب والابن والروح القدس}. وعلموهم أن يحفظوا جميع ما أوصيتكم به. وها أنا معكم كل الأيام إلى انقضاء الدهر”. [الآيات 18-20].}[Lt214-1906.10; 1906][https://egwwritings.org/read?panels=p10171.16]


There is a reason why Ellen White pinpointed to Matthew 28:19 as a Scripture which is \egwinline{not half understood.} This statement is made in the context of 1906, where many ministers, and physicians were advocating the trinity doctrine. As we have seen, the understanding of God as a trinity, was not something Ellen White supported, and for this reason, herself, she dared not \egwinline{to advocate theories that have no foundation in the Word of God.}


هناك سبب لماذا أشارت إلين وايت إلى متى 28:19 كنص كتابي \egwinline{غير مفهوم بشكل كامل.} هذا التصريح تم في سياق عام 1906، حيث كان العديد من الوعاظ والأطباء يدعون إلى عقيدة الثالوث. كما رأينا، فإن فهم الله كثالوث لم يكن شيئًا دعمته إلين وايت، ولهذا السبب، هي نفسها، لم تجرؤ \egwinline{على الدعوة إلى نظريات ليس لها أساس في كلمة الله.}


\egw{The great Teacher held in His hand \textbf{the entire map of truth. In \underline{simple} language He \underline{made plain} to His disciples} the way to heaven and \textbf{the endless subjects of divine power}. \textbf{The question of \underline{the essence of God} was a subject on which He maintained a wise reserve}, for their entanglements and specifications would bring in science which could not be dwelt upon by unsanctified minds without confusion. \textbf{In regard to God and in regard to His personality, the Lord Jesus said}, ‘Have I been so long time with you, and yet hast thou not known Me, Philip? He that hath seen Me hath seen the Father.’ [John 14:9.] \textbf{Christ was the express image of His Father’s person}.}[19LtMs, Ms 45, 1904, par. 15][https://egwwritings.org/read?panels=p14069.9381023&index=0]


\egw{المعلم العظيم كان يحمل في يده \textbf{خريطة الحق بأكملها. بلغة \underline{بسيطة} \underline{أوضح} لتلاميذه} الطريق إلى السماء و\textbf{المواضيع اللانهائية للقوة الإلهية}. \textbf{كان سؤال \underline{جوهر الله} موضوعًا حافظ فيه على تحفظ حكيم}، لأن تشابكاتهم ومواصفاتهم ستجلب علمًا لا يمكن للعقول غير المقدسة التفكر فيه دون ارتباك. \textbf{فيما يتعلق بالله وفيما يتعلق بشخصانيته، قال الرب يسوع}، ‘أنا معكم زمانًا هذه مدته ولم تعرفني يا فيلبس؟ الذي رآني فقد رأى الآب.’ [يوحنا 14:9.] \textbf{كان المسيح صورة جوهره.}}[19LtMs, Ms 45, 1904, par. 15][https://egwwritings.org/read?panels=p14069.9381023&index=0]


\egwnogap{The open path, the safe path of walking in the way of His commandments, is a path from which there is no safe departing. \textbf{And when men follow their own human theories dressed up in soft, fascinating representations, they make a snare in which to catch souls}. \textbf{\underline{In the place of devoting your powers to theorizing}}, Christ has given you a work to do. His commission is, Go <throughout the world> and make disciples of all nations, \textbf{baptizing them in the name of the Father, and of the Son, and of the Holy Ghost}. Before the disciples shall compass the threshold, there is to be the imprint of \textbf{the sacred name, baptizing the believers in \underline{the name of the threefold powers} in the heavenly world}. The human mind is impressed in this ceremony, the beginning of the Christian life. It means very much. The work of salvation is not a small matter, but so vast that \textbf{the highest authorities} are taken hold of by the expressed faith of the human agency. \textbf{The Father, the Son, and the Holy Ghost, \underline{the eternal Godhead} is involved in the action required to make assurance to the human agent to unite \underline{all heaven} to contribute to the exercise of human faculties to reach and embrace the fulness of \underline{the threefold powers} to unite in the great work appointed, confederating the heavenly powers with the human, that men may become, through heavenly efficiency, partakers of the divine nature and workers together with Christ}.}[19LtMs, Ms 45, 1904, par. 16][https://egwwritings.org/read?panels=p14069.9381024&index=0]


\egwnogap{الطريق المفتوح، الطريق الآمن للسير في طريق وصاياه، هو طريق لا يوجد منه ابتعاد آمن. \textbf{وعندما يتبع الناس نظرياتهم البشرية المزينة بتمثيلات ناعمة وفاتنة، فإنهم يصنعون فخًا لاصطياد النفوس}. \textbf{\underline{بدلاً من تكريس قواك للتنظير}}، أعطاك المسيح عملاً لتقوم به. تكليفه هو، اذهبوا <إلى العالم أجمع> وتلمذوا جميع الأمم، \textbf{معمدين إياهم باسم الآب والابن والروح القدس}. قبل أن يجتاز التلاميذ العتبة، يجب أن يكون هناك طبعة \textbf{الاسم المقدس، معمدين المؤمنين \underline{باسم القوى الثلاثية} في العالم السماوي}. العقل البشري يتأثر في هذه المراسم، بداية الحياة المسيحية. هذا يعني الكثير. عمل الخلاص ليس أمرًا صغيرًا، بل واسع جدًا لدرجة أن \textbf{أعلى السلطات} تؤخذ بالإيمان المعبر عنه من قبل الوكالة البشرية. \textbf{الآب والابن والروح القدس، \underline{اللاهوت الأبدي} متورط في العمل المطلوب لتقديم ضمان للوكيل البشري لتوحيد \underline{كل السماء} للمساهمة في ممارسة القدرات البشرية للوصول واحتضان ملء \underline{القوى الثلاثية} للاتحاد في العمل العظيم المعين، موحدين القوى السماوية مع البشرية، حتى يصبح الناس، من خلال الكفاءة السماوية، مشاركين في الطبيعة الإلهية وعاملين مع المسيح}.}[19LtMs, Ms 45, 1904, par. 16][https://egwwritings.org/read?panels=p14069.9381024&index=0]


This quotation is yet another often misrepresented statement. It has been often used to argue that Ellen White advocated for the Trinity by referencing the Father, the Son and the Holy Spirit by term \egwinline{eternal Godhead.} However, we must peel back the layers of its context. Ellen White was explaining the meaning behind Matthew 28:19. She stated: \egwinline{In the place of devoting your powers to theorizing,} fulfill the commission given by Christ. Theorizing about what? Theorizing about \egwinline{the essence of God.} This is another “smoking gun” for the Trinity doctrine, especially when she referenced the \emcap{personality of God} by stating: \egwinline{\textbf{In regard to God and in regard to His personality}, the Lord Jesus said…[John 14:9.] Christ was the express image of His \textbf{Father’s person}.} John 14:9 does not mean that seeing the Father in Christ implies they are one and the same person, all part of one God. Rather, it affirms that Christ is the express image of the Father’s person. The “God” she referred to was the Father. Indeed, Jesus taught the truth about who and what God is. This is what He \egwinline{made plain} \egwinline{in the simple language.} To claim that by the term \egwinline{eternal Godhead} Ellen White was endorsing the Trinity would contradict the very caution she expressed in the context of this passage.


هذا الاقتباس هو بيان آخر كثيرًا ما يُساء تمثيله. غالبًا ما استُخدم للجدال بأن إلين وايت دعمت الثالوث من خلال الإشارة إلى الآب والابن والروح القدس بمصطلح \egwinline{اللاهوت الأبدي.} ومع ذلك، يجب علينا كشف طبقات سياقه. كانت إلين وايت تشرح معنى متى 28:19. وقالت: \egwinline{بدلاً من تكريس قواك للتنظير،} أتمم التكليف الذي أعطاه المسيح. التنظير حول ماذا؟ التنظير حول \egwinline{جوهر الله.} هذا “دليل دامغ” آخر لعقيدة الثالوث، خاصة عندما أشارت إلى \emcap{شخصانية الله} بقولها: \egwinline{\textbf{فيما يتعلق بالله وفيما يتعلق بشخصانيته}، قال الرب يسوع... [يوحنا 14:9.] كان المسيح صورة \textbf{جوهر الآب}.} يوحنا 14:9 لا يعني أن رؤية الآب في المسيح تعني أنهما شخص واحد، كلهم جزء من إله واحد. بل يؤكد أن المسيح هو صورة جوهر الآب. “الله” الذي أشارت إليه كان الآب. في الواقع، علّم يسوع الحقيقة حول من وما هو الله. هذا ما \egwinline{أوضحه} \egwinline{بلغة بسيطة.} إن الادعاء بأنها كانت تؤيد الثالوث بمصطلح \egwinline{اللاهوت الأبدي} سيتناقض مع التحذير الذي عبرت عنه في سياق هذا المقطع.


Unfortunately, the desperate desire of Trinitarians to paint Ellen White as a Trinitarian advocate has overshadowed the true, inspired meaning of Matthew 28:19. Her message was: \egwinline{In the place of devoting your powers to theorizing} about \egwinline{the essence of God,} Christ has given us the commission in Matthew 28:19. And she explained the meaning of Matthew 28:19. Her point was: The Father, Son, and Holy Spirit unite all of heaven’s resources with human effort so that, through divine power, people may share in God’s nature and work alongside Christ. That is the meaning of this \egwinline{threefold name.} She continued explaining:


للأسف، فإن الرغبة اليائسة للثالوثيين في تصوير إلين وايت كمؤيدة للثالوث قد طغت على المعنى الحقيقي الموحى به لمتى 28:19. كانت رسالتها: \egwinline{بدلاً من تكريس قواك للتنظير} حول \egwinline{جوهر الله،} أعطانا المسيح التكليف في متى 28:19. وشرحت معنى متى 28:19. كانت نقطتها: الآب والابن والروح القدس يوحدون كل موارد السماء مع الجهد البشري حتى يتمكن الناس، من خلال القوة الإلهية، من المشاركة في طبيعة الله والعمل جنبًا إلى جنب مع المسيح. هذا هو معنى هذا \egwinline{الاسم الثلاثي.} واستمرت في الشرح:


\egw{\textbf{Man’s capabilities can multiply through the connection of human agencies with divine agencies}. \textbf{United with the heavenly powers}, the human capabilities increase according to that faith that works by love and purifies, sanctifies, and ennobles the whole man. \textbf{\underline{The heavenly powers} have \underline{pledged themselves} to minister to human agents to make the name of God and of Christ and of the Holy Spirit their living efficiency, working and energizing the sanctified man, to make this name above every other name}. \textbf{All the treasures of heaven are under obligation to do for man} infinitely more than human beings can comprehend by multiplying threefold the human with the heavenly agencies.}[19LtMs, Ms 45, 1904, par. 17][https://egwwritings.org/read?panels=p14069.9381026&index=0]


\egw{\textbf{قدرات الإنسان يمكن أن تتضاعف من خلال اتصال الوكالات البشرية بالوكالات الإلهية}. \textbf{متحدة مع القوى السماوية}، تزداد القدرات البشرية وفقًا لذلك الإيمان الذي يعمل بالمحبة ويطهر ويقدس ويرفع الإنسان بأكمله. \textbf{\underline{القوى السماوية} \underline{تعهدت} بالخدمة للوكلاء البشريين لجعل اسم الله والمسيح والروح القدس كفاءتهم الحية، تعمل وتنشط الإنسان المقدس، لجعل هذا الاسم فوق كل اسم آخر}. \textbf{كل كنوز السماء تحت التزام للقيام من أجل الإنسان} بما هو أكثر بما لا يمكن للبشر فهمه من خلال مضاعفة ثلاثة أضعاف الوكالات البشرية مع السماوية.}[19LtMs, Ms 45, 1904, par. 17][https://egwwritings.org/read?panels=p14069.9381026&index=0]


\egwnogap{\textbf{\underline{The three great and glorious heavenly characters} are present on the occasion of baptism. All the human capabilities are to be henceforth consecrated powers to do service for God in representing the Father, the Son, and the Holy Ghost upon whom they depend. \underline{All heaven is represented by these three} in covenant relation with the new life}. ‘If ye then be risen with Christ, seek those things that are above, where Christ sitteth at \textbf{the right hand of God}.’ [Colossians 3:1.]}[19LtMs, Ms 45, 1904, par. 18][https://egwwritings.org/read?panels=p14069.9381027&index=0]


\egwnogap{\textbf{\underline{الشخصيات السماوية الثلاث العظيمة والمجيدة} حاضرة في مناسبة المعمودية. كل القدرات البشرية ستكون من الآن فصاعدًا قوى مكرسة للقيام بالخدمة لله في تمثيل الآب والابن والروح القدس الذين يعتمدون عليهم. \underline{كل السماء ممثلة بهؤلاء الثلاثة} في علاقة عهد مع الحياة الجديدة}. ‘فإن كنتم قد قمتم مع المسيح فاطلبوا ما فوق، حيث المسيح جالس \textbf{عن يمين الله}.’ [كولوسي 3:1.]}[19LtMs, Ms 45, 1904, par. 18][https://egwwritings.org/read?panels=p14069.9381027&index=0]


Many claim that Matthew 28:19 is uninspired because it was inserted by the Catholic Church\footnote{Note, 1 John 5:7 \bible{For there are three that bear record in heaven, the Father, the Word, and the Holy Ghost: and these three are one.} is an interpolation known as “\textit{Johannine Comma}”. Ellen White never used that verse. This was not the case with Matthew 28:19.}. Yet, here we have divine inspiration revealing its true meaning—the significance of baptism in the threefold name as a pledge made by these \egwinline{three great and glorious heavenly characters.} Their pledge is that \egwinline{\textbf{all the treasures of heaven are under obligation to do for man} infinitely more than human beings can comprehend by multiplying threefold the human with the heavenly agencies.}


يدعي الكثيرون أن متى 28:19 غير موحى به لأنه تم إدراجه من قبل الكنيسة الكاثوليكية\footnote{ملاحظة، 1 يوحنا 5:7 \bible{فإن الذين يشهدون في السماء هم ثلاثة: الآب، والكلمة، والروح القدس. وهؤلاء الثلاثة هم واحد.} هو إقحام معروف باسم “\textit{فاصلة يوحنا}”. لم تستخدم إلين وايت تلك الآية أبدًا. هذا لم يكن الحال مع متى 28:19.}. ومع ذلك، لدينا هنا الوحي الإلهي يكشف معناه الحقيقي - أهمية المعمودية بالاسم الثلاثي كتعهد قدمته هذه \egwinline{الشخصيات السماوية الثلاث العظيمة والمجيدة.} تعهدهم هو أن \egwinline{\textbf{كل كنوز السماء تحت التزام للقيام من أجل الإنسان} بما هو أكثر بما لا يمكن للبشر فهمه من خلال مضاعفة ثلاثة أضعاف الوكالات البشرية مع السماوية.}


Ellen White frequently quoted Matthew 28:19, explaining the pledge of the Father, the Son, and the Holy Spirit. This pledge serves as a wonderful encouragement and a promise upheld by Heaven. A detailed study of this pledge is beyond the scope of this book, as it does not directly address the presence and \emcap{personality of God}. However, we encourage you to explore this topic for yourself. When you delve deeper into its meaning, you will come to understand the reality of the ministry of heavenly angels.


اقتبست إلين وايت كثيرًا من متى 28:19، موضحة تعهد الآب والابن والروح القدس. هذا التعهد يعمل كتشجيع رائع ووعد تدعمه السماء. دراسة مفصلة لهذا التعهد تتجاوز نطاق هذا الكتاب، لأنه لا يتناول مباشرة وجود و\emcap{شخصانية الله}. ومع ذلك، نشجعك على استكشاف هذا الموضوع بنفسك. عندما تتعمق في معناه، ستفهم واقع خدمة الملائكة السماوية.


Sister White stated that \egwinline{all heaven is represented by these three in covenant relation with the new life.} These three are the Father, the Son, and the Holy Spirit. In another instance, she said:


ذكرت الأخت وايت أن \egwinline{كل السماء ممثلة بهؤلاء الثلاثة في علاقة عهد مع الحياة الجديدة.} هؤلاء الثلاثة هم الآب والابن والروح القدس. وفي حالة أخرى، قالت:


\egw{\textbf{All heaven is interested in your home}. \textbf{God and Christ and \underline{the heavenly angels}} are intensely desirous that you shall so train your children that they will be prepared to enter the family of the redeemed.}[17LtMs, Ms 161, 1902, par. 11][https://egwwritings.org/read?panels=p14067.9877018&index=0]


\egw{\textbf{كل السماء مهتمة ببيتك}. \textbf{الله والمسيح و\underline{الملائكة السماوية}} يرغبون بشدة أن تدرب أولادك بحيث يكونون مستعدين للدخول إلى عائلة المفديين.}[17LtMs, Ms 161, 1902, par. 11][https://egwwritings.org/read?panels=p14067.9877018&index=0]


This is not a contradiction. All of heaven is represented by the Father, the Son, and the Holy Spirit, and in this quote, she specifically mentioned \egwinline{God and Christ and \textbf{the heavenly angels}.} There is a close connection between the workings of the Holy Spirit and the ministry of angels. The Inspiration testifies:


هذا ليس تناقضًا. كل السماء ممثلة بالآب والابن والروح القدس، وفي هذا الاقتباس، ذكرت بالتحديد \egwinline{الله والمسيح و\textbf{الملائكة السماوية}.} هناك صلة وثيقة بين عمل الروح القدس وخدمة الملائكة. يشهد الوحي:


\egw{A measure of \textbf{the Spirit} is given to every man to profit withal. \textbf{Through the ministry of the angels \underline{the Holy Spirit is enabled} to work upon the mind and heart of the human agent}, and draw him to Christ who has paid the ransom money for his soul, that the sinner may be rescued from the slavery of sin and Satan.}[8LtMs, Lt 71, 1893, par. 10][https://egwwritings.org/read?panels=p14058.6086016&index=0]


\egw{يُعطى مقدار من \textbf{الروح} لكل إنسان للاستفادة منه. \textbf{من خلال خدمة الملائكة \underline{يتمكن الروح القدس} من العمل على عقل وقلب الإنسان}، وجذبه إلى المسيح الذي دفع فدية نفسه، حتى يُنقذ الخاطئ من عبودية الخطية والشيطان.}[8LtMs, Lt 71, 1893, par. 10][https://egwwritings.org/read?panels=p14058.6086016&index=0]


This angelic ministry is one of the elements in the baptismal pledge of Matthew 28:19. When Ellen White said, \egwinline{\textbf{The heavenly powers} have \textbf{pledged themselves} to minister to human agents…,} she was referring to the holy angels. The connection between the Holy Spirit and the holy angels is beyond the scope of this book, but you can explore this topic further in the sequel, \textit{Rediscovering the Pillar}\footnote{Download for free: \href{https://forgottenpillar.com/book/rediscovering-the-pillar}{https://forgottenpillar.com/book/rediscovering-the-pillar}}, in the section on the Holy Spirit\footnote{Also, see the study on the angels \href{https://notefp.link/angels}{https://notefp.link/angels}}.


هذه الخدمة الملائكية هي أحد العناصر في عهد المعمودية في متى 28:19. عندما قالت إلين وايت، \egwinline{\textbf{القوات السماوية} قد \textbf{تعهدت نفسها} بخدمة البشر...،} كانت تشير إلى الملائكة القديسين. العلاقة بين الروح القدس والملائكة القديسين تتجاوز نطاق هذا الكتاب، ولكن يمكنك استكشاف هذا الموضوع بشكل أعمق في الجزء التالي، \textit{إعادة اكتشاف العمود}\footnote{تنزيل مجاني: \href{https://forgottenpillar.com/book/rediscovering-the-pillar}{https://forgottenpillar.com/book/rediscovering-the-pillar}}، في القسم الخاص بالروح القدس\footnote{انظر أيضًا الدراسة عن الملائكة \href{https://notefp.link/angels}{https://notefp.link/angels}}.


% Ellen White and Matthew 28:19

\begin{titledpoem}
    
    \stanza{
        In threefold name we’re baptized true, \\
        Not trinity as some construe. \\
        The Father, Son, and Spirit’s role, \\
        Not one God formed of triple whole.
    }

    \stanza{
        Dear Ellen’s words make clear the case, \\
        This pledge assures us heaven’s grace. \\
        The powers three have pledged their might, \\
        To guide the faithful to the light.
    }

    \stanza{
        Not proof of essence three-in-one, \\
        But heaven’s promise, freely done. \\
        A covenant of help divine, \\
        As new believers cross the line.
    }

    \stanza{
        The Father – God, in person real, \\
        The Son – our Prince, our wounds to heal, \\
        The Spirit – representative, \\
        Through Him Christ does in us now live.
    }
    
\end{titledpoem}