\qrchapter{https://forgottenpillar.com/rsc/en-fp-chapter27}{Steps to apostasy}


\qrchapter{https://forgottenpillar.com/rsc/en-fp-chapter27}{خطوات الارتداد}


In the following quotation, brother J. N. Loughborough, who was one of the pioneers of the Seventh-day Adventist Church, warned us about the five steps to apostasy.


في الاقتباس التالي، حذرنا الأخ ج. ن. لوبورو، الذي كان أحد رواد كنيسة الأدفنتست السبتيين، من خمس خطوات للارتداد.


\others{\textbf{The} \textbf{first step} of apostasy is to \textbf{get up a creed}, telling us what we shall believe. \textbf{The second} is to \textbf{make that creed a test of fellowship}. \textbf{The third} is to \textbf{try members by that creed}. \textbf{The fourth} is to \textbf{denounce as heretics those who do not believe that creed}. And \textbf{fifth}, to \textbf{commence persecution against such}. I plead that we are not patterning after the churches in any unwarrantable sense in the step proposed.}[John N. Loughborough, Review and Herald, Oct. 8, 1861.][https://egwwritings.org/read?panels=p1685.5326]


\others{\textbf{الخطوة} \textbf{الأولى} للارتداد هي \textbf{وضع عقيدة}، تخبرنا بما يجب أن نؤمن به. \textbf{الثانية} هي \textbf{جعل تلك العقيدة اختبارًا للشركة}. \textbf{الثالثة} هي \textbf{اختبار الأعضاء بتلك العقيدة}. \textbf{الرابعة} هي \textbf{إدانة كهراطقة أولئك الذين لا يؤمنون بتلك العقيدة}. و\textbf{الخامسة}، \textbf{بدء الاضطهاد ضد هؤلاء}. أناشد بأننا لا نقلد الكنائس بأي معنى غير مبرر في الخطوة المقترحة.}[جون ن. لوبورو، ريفيو آند هيرالد، 8 أكتوبر 1861.][https://egwwritings.org/read?panels=p1685.5326]


These principles are important to have in mind, and we ought to ask ourselves if we, today, are patterning after the churches in any unwarrantable sense in the step proposed. What would happen to a Seventh-day Adventist who would reject the Trinity doctrine in favor of the \emcap{Fundamental Principles}? Do we have a creed set up in our church? Do we test our membership by it?


هذه المبادئ مهمة لنضعها في أذهاننا، وينبغي أن نسأل أنفسنا إذا كنا، اليوم، نقلد الكنائس بأي معنى غير مبرر في الخطوة المقترحة. ماذا سيحدث لأدفنتستي سبتي يرفض عقيدة الثالوث لصالح \emcap{المبادئ الأساسية}؟ هل لدينا عقيدة موضوعة في كنيستنا؟ هل نختبر عضويتنا بها؟


The \emcap{Fundamental Principles} had a different nature and role in the Seventh-day Adventist Church contrary to that of the pattern held by other churches. The \emcap{Fundamental Principles} were not designed as a creed. In the preface of the 1872 statement, we read about their nature:


كانت \emcap{المبادئ الأساسية} ذات طبيعة ودور مختلف في كنيسة الأدفنتست السبتيين على عكس النمط الذي تتبعه الكنائس الأخرى. لم تُصمم \emcap{المبادئ الأساسية} كعقيدة. في مقدمة بيان عام 1872، نقرأ عن طبيعتها:


\others{In presenting to the \textbf{public} this \textbf{synopsis of our faith}, we wish to have it distinctly understood that \textbf{\underline{we have no articles of faith, creed}, or discipline, \underline{aside from the Bible}}. We \textbf{do not} put forth this \textbf{\underline{as having any authority with our people}}, \textbf{nor is it designed to secure uniformity among them}, \textbf{as a system of faith}, \textbf{but is a brief statement of what is, and has been, with great unanimity, held by them}.}[A Declaration of the Fundamental Principles, Taught and Practiced by the Seventh-Day Adventists, 1872]


\others{في تقديم هذا \textbf{الموجز لإيماننا} \textbf{للجمهور}، نود أن يُفهم بوضوح أنه \textbf{\underline{ليس لدينا مواد إيمان أو عقيدة} أو نظام، \underline{غير الكتاب المقدس}}. نحن \textbf{لا} نقدم هذا \textbf{\underline{كشيء له أي سلطة على شعبنا}}، \textbf{ولا يهدف إلى ضمان التوحيد بينهم}، \textbf{كنظام إيمان}، \textbf{بل هو بيان موجز لما هو، وما كان، بإجماع كبير، يتمسكون به}.}[إعلان المبادئ الأساسية التي يعلمها ويمارسها الأدفنتست السبتيون، 1872]


In the preface of the 1889 statement, we read similar sentiments:


في مقدمة بيان عام 1889، نقرأ آراء مماثلة:


\others{As elsewhere stated, Seventh-day Adventists \textbf{have no creed but the Bible}; but they hold to \textbf{certain well-defined points of faith}, for which they \textbf{feel prepared to give a reason ‘to every man that asketh’ them}. The following propositions may be taken as a summary of \textbf{the principal features of their religious faith}, upon which there is, so far as we know, \textbf{entire unanimity throughout the body}.}[Seventh-day Adventist Year Book of statistics for 1889, pg. 147, The Fundamental Principles of Seventh-day Adventists]


\others{كما ذُكر في مكان آخر، الأدفنتست السبتيون \textbf{ليس لديهم عقيدة سوى الكتاب المقدس}؛ لكنهم يتمسكون \textbf{بنقاط إيمان محددة جيدًا}، والتي \textbf{يشعرون بأنهم مستعدون لتقديم سبب ‘لكل من يسألهم’}. يمكن اعتبار المقترحات التالية ملخصًا \textbf{للسمات الرئيسية لإيمانهم الديني}، والتي يوجد، على حد علمنا، \textbf{إجماع تام في جميع أنحاء الجسد}.}[كتاب إحصائيات الأدفنتست السبتيين السنوي لعام 1889، ص. 147، المبادئ الأساسية للأدفنتست السبتيين]


The \emcap{Fundamental Principles} were not designed to dictate someone’s faith. The believers, led by the Holy Spirit, freely rendered their consciences to the Word of God; under the influence of the Holy Spirit, they came to the same conclusions. There was entire unanimity throughout the body. All believers felt “\textit{prepared to give a reason to every man that asketh them}” regarding their faith.


لم تُصمم \emcap{المبادئ الأساسية} لتملي إيمان شخص ما. المؤمنون، بقيادة الروح القدس، قدموا ضمائرهم بحرية لكلمة الله؛ تحت تأثير الروح القدس، توصلوا إلى نفس الاستنتاجات. كان هناك إجماع تام في جميع أنحاء الجسد. شعر جميع المؤمنين بأنهم “\textit{مستعدون لتقديم سبب لكل من يسألهم}” بخصوص إيمانهم.


Today we see a striking difference in the principles and practice of Adventist beliefs compared to our pioneers. We are keeping the spirit of unity by disciplining our members for the denial of the Fundamental Beliefs. In our church manual, under the section “\textit{Reason for Disciplines}”, we read the first point which states the discipline for denial of faith in the Seventh-day Adventist Fundamental Beliefs.


اليوم نرى اختلافًا صارخًا في مبادئ وممارسة معتقدات الأدفنتست مقارنة بروادنا. نحن نحافظ على روح الوحدة من خلال تأديب أعضائنا لإنكار المعتقدات الأساسية. في دليل كنيستنا، تحت قسم “\textit{سبب التأديب}”، نقرأ النقطة الأولى التي تنص على التأديب لإنكار الإيمان بالمعتقدات الأساسية للأدفنتست السبتيين.


\others{Reasons for Discipline}


\others{أسباب التأديب}


\others{1. \textbf{Denial of faith} in the fundamentals of the gospel and \textbf{in the Fundamental Beliefs of the Church} or \textbf{teaching doctrines contrary to the same}.}[SDA Church Manual, 20th edition, Revised 2022, p. 67][https://www.adventist.org/wp-content/uploads/2023/07/2022-Seventh-day-Adventist-Church-Manual.pdf]


\others{1. \textbf{إنكار الإيمان} بأساسيات الإنجيل و\textbf{بالمعتقدات الأساسية للكنيسة} أو \textbf{تعليم عقائد مخالفة لها}.}[SDA Church Manual, 20th edition, Revised 2022, p. 67][https://www.adventist.org/wp-content/uploads/2023/07/2022-Seventh-day-Adventist-Church-Manual.pdf]


To discipline someone over their faith is nothing else than coercion of conscience. We are to render our conscience to the Bible alone—not to any man, councils or church creed(s). Disciplining members for their denial of the Fundamental Beliefs is clear evidence that we, indeed, have a creed besides the Bible. We cannot exercise the freedom of our conscience in subjection to the Word of God while confined to a set of beliefs that, if questioned with the authority of the Bible, will be disciplined. In our practice we have forgotten the foundation of protestantism and reformation. All reformers have had their conscience coerced to the extent of their lives. Martin Luther had famously put this principle in action in his defense before the Diet of Worms.


إن تأديب شخص ما بسبب إيمانه ليس سوى إكراه للضمير. علينا أن نخضع ضميرنا للكتاب المقدس وحده - وليس لأي إنسان أو مجالس أو عقيدة كنسية. إن تأديب الأعضاء لإنكارهم المعتقدات الأساسية هو دليل واضح على أننا، في الواقع، لدينا عقيدة إلى جانب الكتاب المقدس. لا يمكننا ممارسة حرية ضميرنا خاضعين لكلمة الله بينما نحن مقيدون بمجموعة من المعتقدات التي، إذا تم التشكيك فيها بسلطة الكتاب المقدس، سيتم التأديب عليها. لقد نسينا في ممارساتنا أساس البروتستانتية والإصلاح. جميع المصلحين تعرضوا لإكراه ضميرهم إلى حد تهديد حياتهم. وقد وضع مارتن لوثر هذا المبدأ موضع التنفيذ بشكل مشهور في دفاعه أمام مجلس ورمس.


\others{Unless I am \textbf{convicted by Scripture} and plain reason—I do not accept the authority of popes and councils, for they have contradicted each other—\textbf{\underline{my conscience is captive to the Word of God}}. I cannot and I will not recant anything, for \textbf{to go against conscience is neither right nor safe}. Here I stand, I cannot do otherwise. God help me. Amen.}[Bainton, 182]


\others{ما لم أكن \textbf{مقتنعًا بالكتاب المقدس} والعقل الواضح - فأنا لا أقبل سلطة الباباوات والمجالس، لأنهم ناقضوا بعضهم البعض - \textbf{\underline{ضميري أسير لكلمة الله}}. لا أستطيع ولن أتراجع عن أي شيء، لأن \textbf{مخالفة الضمير ليست صوابًا ولا آمنة}. هنا أقف، لا يمكنني أن أفعل غير ذلك. فليساعدني الله. آمين.}[Bainton, 182]


If one Seventh-day Adventist member has his conscience captive to the Word of God and is not in harmony with the Seventh-day Adventist Fundamental Beliefs, his conscience should not be coerced by church discipline. We know that in the end of time, the whole Seventh-day Adventist Church will be coerced over the issue of the Sabbath. We have been fighting for religious freedom, yet we’re allowing ourselves to coerce the conscience of those who are not in harmony with the Fundamental Beliefs. If today we discipline our members for not subjecting their consciences to men, councils and creeds, how shall we act tomorrow when the government will discipline their citizens for not subjecting their conscience to its power, when they will force obedience to legislation contrary to the Scriptures?


إذا كان أحد أعضاء الأدفنتست السبتيين ضميره أسيرًا لكلمة الله وليس متوافقًا مع المعتقدات الأساسية للأدفنتست السبتيين، فلا ينبغي إكراه ضميره من خلال التأديب الكنسي. نحن نعلم أنه في نهاية الزمان، ستتعرض كنيسة الأدفنتست السبتيين بأكملها للإكراه بشأن قضية السبت. لقد كنا نناضل من أجل الحرية الدينية، ومع ذلك نسمح لأنفسنا بإكراه ضمير أولئك الذين ليسوا متوافقين مع المعتقدات الأساسية. إذا كنا اليوم نؤدب أعضاءنا لعدم إخضاع ضميرهم للبشر والمجالس والعقائد، فكيف سنتصرف غدًا عندما تؤدب الحكومة مواطنيها لعدم إخضاع ضميرهم لسلطتها، عندما يفرضون الطاعة للتشريعات المخالفة للكتاب المقدس؟


Adventist pioneers were very much aware of the dangers of extorting church members’ consciences. The expression of their beliefs was not designed to form unity. They were ready to justify their faith, from the Bible, when asked. The Bible was their only creed and article of faith.


كان رواد الأدفنتست مدركين تمامًا لمخاطر انتزاع ضمائر أعضاء الكنيسة. لم يكن التعبير عن معتقداتهم مصممًا لتشكيل الوحدة. كانوا مستعدين لتبرير إيمانهم، من الكتاب المقدس، عندما يُسألون. كان الكتاب المقدس هو عقيدتهم الوحيدة ومادة إيمانهم.


In 1883, there was a suggestion to introduce the church manual into the Seventh-day Adventist Church. This proposal was rejected after close investigation of the committee appointed by the General Conference. In the article “\textit{No Church Manual}”, we read their reasons for not accepting the proposed church manual.


في عام 1883، كان هناك اقتراح لإدخال دليل الكنيسة في كنيسة الأدفنتست السبتيين. تم رفض هذا الاقتراح بعد فحص دقيق من قبل اللجنة التي عينها المؤتمر العام. في مقالة “\textit{لا دليل للكنيسة}”، نقرأ أسبابهم لعدم قبول دليل الكنيسة المقترح.


\others{\textbf{While brethren who have favored a manual have ever contended that such a work was not to be anything like a creed or a discipline, or to have authority to settle disputed points}, but was only to be considered as a book containing hints for the help of those of little experience, \textbf{yet it must be evident that such a work, issued under the auspices of the General Conference, would at once carry with it much weight of authority, and would be consulted by most of our younger ministers}. \textbf{\underline{It would gradually shape and mold the whole body}}; \textbf{and those who did not follow it would be considered out of harmony with established principles of church order}. \textbf{And, really, is this not the object of the manual?} And what would be the use of one if not to accomplish such a result? But would this result, on the whole, be a benefit? Would our ministers be broader, more original, more self-reliant men? Could they be better depended on in great emergencies? Would their spiritual experiences likely be deeper and their judgment more reliable? \textbf{We think the tendency all the other way}.}[No Church Manual, The Review and Herald, November 27, 1883, pg. 745][https://documents.adventistarchives.org/Periodicals/RH/RH18831127-V60-47.pdf]


\others{\textbf{بينما كان الإخوة الذين فضلوا وجود دليل يجادلون دائمًا بأن مثل هذا العمل لن يكون شبيهًا بعقيدة أو نظام، أو يكون له سلطة لتسوية النقاط المتنازع عليها}، بل كان يُعتبر فقط كتابًا يحتوي على تلميحات لمساعدة ذوي الخبرة القليلة، \textbf{إلا أنه يجب أن يكون واضحًا أن مثل هذا العمل، الصادر تحت رعاية المؤتمر العام، سيحمل على الفور الكثير من وزن السلطة، وسيتم الرجوع إليه من قبل معظم وزرائنا الأصغر سنًا}. \textbf{\underline{سيشكل ويصوغ تدريجيًا الجسد بأكمله}}؛ \textbf{وأولئك الذين لا يتبعونه سيعتبرون غير متوافقين مع المبادئ المعتمدة للنظام الكنسي}. \textbf{وفي الواقع، أليس هذا هو الهدف من الدليل؟} وما فائدة وجود دليل إذا لم يكن لتحقيق مثل هذه النتيجة؟ ولكن هل ستكون هذه النتيجة، بشكل عام، مفيدة؟ هل سيكون وزراؤنا أكثر اتساعًا، وأكثر أصالة، وأكثر اعتمادًا على أنفسهم؟ هل يمكن الاعتماد عليهم بشكل أفضل في حالات الطوارئ الكبرى؟ هل من المحتمل أن تكون تجاربهم الروحية أعمق وحكمهم أكثر موثوقية؟ \textbf{نعتقد أن الاتجاه كله في الاتجاه الآخر}.}[No Church Manual, The Review and Herald, November 27, 1883, pg. 745][https://documents.adventistarchives.org/Periodicals/RH/RH18831127-V60-47.pdf]


\others{\textbf{The Bible contains our creed and discipline. It \underline{thoroughly} furnishes the man of God unto all good works}. What it has not revealed relative to church organization and management, the duties of officers and ministers, and kindred subjects, should not be strictly defined and drawn out into minute specifications for the sake of uniformity, \textbf{but rather be left to individual judgment under the guidance of the Holy Spirit}. \textbf{Had it been best to have a book of directions of this sort, the Spirit would doubtless have gone further, and left one on record with the stamp of inspiration upon it}.}[Ibid.][https://documents.adventistarchives.org/Periodicals/RH/RH18831127-V60-47.pdf]


\others{\textbf{الكتاب المقدس يحتوي على عقيدتنا ونظامنا. إنه يجهز رجل الله \underline{تمامًا} لكل عمل صالح}. ما لم يكشف عنه فيما يتعلق بتنظيم الكنيسة وإدارتها، وواجبات المسؤولين والوزراء، والمواضيع المماثلة، لا ينبغي تحديده بدقة ووضعه في مواصفات دقيقة من أجل التوحيد، \textbf{بل ينبغي تركه للحكم الفردي تحت إرشاد الروح القدس}. \textbf{لو كان من الأفضل أن يكون هناك كتاب توجيهات من هذا النوع، لكان الروح بلا شك قد ذهب أبعد من ذلك، وترك واحدًا مسجلًا بختم الإلهام عليه}.}[Ibid.][https://documents.adventistarchives.org/Periodicals/RH/RH18831127-V60-47.pdf]


Since 1883, the Seventh-day Adventist Church had grown considerably; so, for the sake of convenience, in 1931, the General Conference Committee voted to publish a church manual.\footnote{Maratas, Prince. “Church Manual.” General Conference of Seventh-Day Adventists, 20 Aug. 2023, \href{https://gc.adventist.org/church-manual/}{gc.adventist.org/church-manual/}. Accessed 3 Feb. 2025.} The church, as an organized body, should exercise order and discipline, in the matters of organization and plans of the prosperity of the Church's mission. But no committee should exercise authority over someone’s conscience and someone’s belief. Only God holds the right to this authority. This is why the Bible is our only creed. We render our conscience to the Word of God, not a man, nor a group of men or committee. Contrary to this, many believe that God vested this authority to the general assembly of the General Conference. But such an idea is based on misrepresentation of one particular quotation. Let us read this quotation carefully.


منذ عام 1883، نمت كنيسة الأدفنتست السبتيين بشكل كبير؛ لذلك، من أجل الراحة، في عام 1931، صوتت لجنة المؤتمر العام على نشر دليل للكنيسة.\footnote{Maratas, Prince. “Church Manual.” General Conference of Seventh-Day Adventists, 20 Aug. 2023, \href{https://gc.adventist.org/church-manual/}{gc.adventist.org/church-manual/}. Accessed 3 Feb. 2025.} يجب على الكنيسة، كهيئة منظمة، أن تمارس النظام والانضباط، في مسائل التنظيم وخطط ازدهار رسالة الكنيسة. لكن لا ينبغي لأي لجنة أن تمارس سلطة على ضمير شخص ما ومعتقداته. الله وحده يملك الحق في هذه السلطة. لهذا السبب الكتاب المقدس هو عقيدتنا الوحيدة. نحن نخضع ضميرنا لكلمة الله، وليس لإنسان، ولا لمجموعة من الرجال أو لجنة. على عكس ذلك، يعتقد الكثيرون أن الله منح هذه السلطة للجمعية العامة للمؤتمر العام. لكن مثل هذه الفكرة تستند إلى سوء تمثيل لاقتباس معين. دعونا نقرأ هذا الاقتباس بعناية.


\egw{At times, when a small group of men entrusted with \textbf{the general management of the work} have, in the name of the General Conference, sought to carry out unwise plans and to restrict God’s work, I have said that I could no longer regard the voice of the General Conference, represented by these few men, as the voice of God. \textbf{But this is not saying that the decisions of a General Conference composed of an assembly of duly appointed, representative men from all parts of the field should not be respected}. \textbf{God has ordained that the representatives of His church from all parts of the earth, when assembled in a General Conference, \underline{shall have authority}}. The error that some are in danger of committing is in giving to the mind and judgment of one man, or of a small group of men, \textbf{the full measure of authority and influence that God has vested in His church in the judgment and voice of the General Conference assembled \underline{to plan for the prosperity and advancement of His work}}.}[9T 260.2; 1909][https://egwwritings.org/read?panels=p115.1474]


\egw{في بعض الأحيان، عندما سعت مجموعة صغيرة من الرجال المؤتمنين على \textbf{الإدارة العامة للعمل}، باسم المجمع العام، إلى تنفيذ خطط غير حكيمة وتقييد عمل الله، قلت إنني لم أعد أعتبر صوت المجمع العام، الذي يمثله هؤلاء الرجال القلائل، صوت الله. \textbf{لكن هذا لا يعني أن قرارات المجمع العام المكون من مجموعة من الممثلين المعينين حسب الأصول من جميع أنحاء الحقل لا ينبغي احترامها}. \textbf{لقد رتب الله أن ممثلي كنيسته من جميع أنحاء الأرض، عندما يجتمعون في مجمع عام، \underline{يكون لهم سلطان}}. الخطأ الذي قد يقع فيه البعض هو إعطاء عقل وحكم رجل واحد، أو مجموعة صغيرة من الرجال، \textbf{المقياس الكامل للسلطة والتأثير الذي أودعه الله في كنيسته في حكم وصوت المجمع العام المجتمع \underline{للتخطيط لازدهار عمله وتقدمه}}.}[9T 260.2; 1909][https://egwwritings.org/read?panels=p115.1474]


Sister White pointed out that the world wide assembly of the General Conference meeting does have authority as the voice of God, yet she is very particular over what matters it has this authority. The authority God vested in the assembly of the General Conference is \egwinline{to plan for the prosperity and advancement of His work}. It is about making mission plans, not about managing beliefs or the conscience. God’s church does have His voice regarding beliefs; the voice of God pertaining to the faith is the Bible. The Bible is fully sufficient for us and we are free to render our conscience to it. No synopsis of any denominational faith has authority to dictate someone's faith; neither do \emcap{Fundamental Principles}, or current Fundamental Beliefs.\footnote{Although the Fundamental Principles were not designed to have authority over the people, nor were they designed to secure uniformity among them, as a system of faith, there is some evidence to the contrary. In his article, “\textit{Seventh-day Adventists and the Doctrine of the Trinity}”, of the “\textit{Christian Workers Magazine}”, 1915, D.M. Canright gave evidence that a Conference president used the \emcap{Fundamental Principles} as a test of fellowship in 1911. Such practice is not constructive to the Truth, neither is it beneficial for believers.} Sister White was very clear about the Bible being the only rule of faith, and every doctrine should be questioned with Scripture. In the Great Controversy, we read the following:


أشارت الأخت وايت إلى أن اجتماع المجمع العام العالمي له سلطة كصوت الله، لكنها دقيقة جدًا فيما يتعلق بالأمور التي له فيها هذه السلطة. السلطة التي أودعها الله في مجمع المجمع العام هي \egwinline{للتخطيط لازدهار عمله وتقدمه}. الأمر يتعلق بوضع خطط للإرسالية، وليس بإدارة المعتقدات أو الضمير. كنيسة الله لديها صوته فيما يتعلق بالمعتقدات؛ صوت الله المتعلق بالإيمان هو الكتاب المقدس. الكتاب المقدس كافٍ تمامًا لنا ونحن أحرار في تقديم ضميرنا له. لا يوجد ملخص لأي إيمان طائفي له سلطة لإملاء إيمان شخص ما؛ ولا \emcap{المبادئ الأساسية}، أو المعتقدات الأساسية الحالية.\footnote{على الرغم من أن المبادئ الأساسية لم تُصمم لتكون لها سلطة على الناس، ولم تُصمم لتأمين التوحيد بينهم، كنظام إيمان، إلا أن هناك بعض الأدلة على العكس. في مقالته، “\textit{الأدفنتست السبتيون وعقيدة الثالوث}”، من “\textit{مجلة العمال المسيحيين}”، 1915، قدم د.م. كارايت دليلاً على أن رئيس مؤتمر استخدم \emcap{المبادئ الأساسية} كاختبار للشركة في عام 1911. مثل هذه الممارسة ليست بناءة للحق، ولا هي مفيدة للمؤمنين.} كانت الأخت وايت واضحة جدًا بشأن كون الكتاب المقدس هو القاعدة الوحيدة للإيمان، وينبغي مساءلة كل عقيدة بالكتاب المقدس. في كتاب الصراع العظيم، نقرأ ما يلي:


\egw{But God will have a people upon the earth \textbf{to maintain the Bible, and \underline{the Bible only}}, \textbf{as the standard of all doctrines and the basis of all reforms}. \textbf{The opinions of learned men, the deductions of science, \underline{the creeds or decisions of ecclesiastical councils}, as numerous and discordant as are the churches which they represent, the voice of the majority - not one nor all of these should be regarded as evidence for or against any point of religious faith.} \textbf{Before accepting any doctrine or precept, we should demand a plain ‘Thus saith the Lord’ in its support.}}[GC 595.1; 1888][https://egwwritings.org/read?panels=p132.2689]


\egw{لكن الله سيكون له شعب على الأرض \textbf{يتمسك بالكتاب المقدس، \underline{والكتاب المقدس وحده}}، \textbf{كمعيار لجميع العقائد وأساس جميع الإصلاحات}. \textbf{إن آراء الرجال المتعلمين، واستنتاجات العلم، \underline{وعقائد أو قرارات المجالس الكنسية}، المتعددة والمتضاربة كما هي الكنائس التي تمثلها، وصوت الأغلبية - لا ينبغي اعتبار أي من هذه أو كلها دليلاً مع أو ضد أي نقطة من نقاط الإيمان الديني.} \textbf{قبل قبول أي عقيدة أو وصية، يجب أن نطالب بـ ‘هكذا قال الرب’ واضحة لدعمها.}}[GC 595.1; 1888][https://egwwritings.org/read?panels=p132.2689]


The liberty of conscience is the basics of protestantism and reformation. We hope and believe that every Seventh-day Adventist can exercise freedom to render his conscience to the Bible without being coerced by discipline, or any other means. The issue of the church's creed and discipline becomes more relevant today, when we have the promise that God will re-establish the original foundation of our faith. We hope and pray that the evidence brought up here will bring light to the church leadership and encourage them to eradicate the false practices in our midst. As the religious leaders in Christ’s time were entrusted with the duty to preserve the Truth and to recognize the time of God’s visitation, so it is today with the leaders of the Seventh-day Adventist Church. In what follows, we will present the prophecies God specifically gave to the Seventh-day Adventist Church. In our time, the end-time, all the pillars of our faith that were held in the beginning will be re-established. May every member of the Seventh-day Adventist Church recognize the importance of the revival that God is about to establish.


حرية الضمير هي أساس البروتستانتية والإصلاح. نأمل ونؤمن أن كل أدفنتستي سبتي يمكنه ممارسة الحرية في تقديم ضميره للكتاب المقدس دون أن يُكره بالتأديب أو أي وسيلة أخرى. تصبح قضية عقيدة الكنيسة وتأديبها أكثر أهمية اليوم، عندما لدينا الوعد بأن الله سيعيد تأسيس الأساس الأصلي لإيماننا. نأمل ونصلي أن تجلب الأدلة المطروحة هنا النور لقيادة الكنيسة وتشجعهم على استئصال الممارسات الخاطئة في وسطنا. كما كان القادة الدينيون في زمن المسيح مؤتمنين على واجب الحفاظ على الحق والتعرف على وقت زيارة الله، فكذلك الأمر اليوم مع قادة كنيسة الأدفنتست السبتيين. فيما يلي، سنقدم النبوءات التي أعطاها الله خصيصًا لكنيسة الأدفنتست السبتيين. في وقتنا الحاضر، نهاية الزمان، ستُعاد جميع أعمدة إيماننا التي كانت موجودة في البداية. ليدرك كل عضو في كنيسة الأدفنتست السبتيين أهمية الإحياء الذي على وشك أن يؤسسه الله.


% Steps to Apostasy

\begin{titledpoem}
    \stanza{
        A creed established beyond God's Word, \\
        The voice of conscience no longer heard. \\
        Fellowship tested by human decree, \\
        From Bible authority we slowly flee.
    }

    \stanza{
        Those who dissent labeled heretics, lost, \\
        Their faith and conviction at terrible cost. \\
        Persecution follows for standing apart, \\
        When creeds replace Scripture within the heart. \\
    }

    \stanza{
        The Bible alone should guide our belief, \\
        All other authorities bringing grief. \\
        Our conscience surrenders to God's Word divine, \\
        Not to councils of men who draw the line.
    }

    \stanza{
        The pioneers knew this freedom well, \\
        Against human creeds they chose to rebel. \\
        For truth must flourish where conscience is free, \\
        As God intended His church to be.
    }
\end{titledpoem}