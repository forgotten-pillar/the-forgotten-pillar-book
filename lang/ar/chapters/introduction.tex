\qrchapterstar{https://forgottenpillar.com/rsc/en-fp-introduction}{Introduction}


\qrchapterstar{https://forgottenpillar.com/rsc/en-fp-introduction}{المقدمة}


\addcontentsline{toc}{chapter}{Introduction}


\addcontentsline{toc}{chapter}{المقدمة}


This book has three objectives to fulfill. The first one is to revive the old pillar of our faith called, “\textit{the personality of God}”. The second objective is to re-establish trust in the writings of Ellen White, and the third is to re-establish the original Adventist identity.


يهدف هذا الكتاب إلى تحقيق ثلاثة أهداف. الهدف الأول هو إحياء العمود القديم لإيماننا المسمى “\textit{شخصانية الله}”. والهدف الثاني هو إعادة بناء الثقة في كتابات إلين وايت، والهدف الثالث هو إعادة تأسيس الهوية الأدفنتستية الأصلية.


Prior to October 22, 1844, there was a great number of Adventists waiting for Christ to return on the clouds of heaven. It was a global movement of people awaiting His second coming. October 22 passed without Christ descending on the clouds and the great majority left the movement, scorning it, scorning the prophecies, the Bible, and God. Very few faithful, humble, men and women remained, who were unquestionably sure that God was leading this movement. They knew that God was shining the light of Truth and their hearts were eager to receive it. But in the eyes of the world, they were just demonstrated fanatics and dreamers. This great disappointment can be compared to the one Jesus’ disciples had after they saw their Lord being laid in the grave. They were unquestionably sure that Christ “\textit{was a prophet mighty in deed and word before God and all the people}”, but as He died on the cross, they were bitterly disappointed, because they “\textit{trusted that it had been He which should have redeemed Israel}.” Yet in their state of despair, in their state of self-disappointment, they were ready to receive the power to conquer the whole world with the Gospel. They met Christ and later received His Spirit. The same happened with the Adventist pioneers. They were a small group of people, bitterly disappointed; they sought the Lord with all their hearts and received Him in power and in Truth. The truths God revealed during this precious time of crisis constitute the foundation of Seventh-day Adventist faith. These truths were tested by all the seductive, deceptive theories of the world, by those scorning this small group, yet these grand truths prevailed. In the time of greatest need, Jesus gave His testimony by raising a little girl, the weakest of the weak, to approve all of His truths. Ellen White was not to be the source of the truths; rather, to support the brethren who were seeking the truth in the Bible. God used Ellen White to approve their studies and to point them to the Bible. The final result was the establishment of the foundation of faith based on the Bible, which standeth sure till the end of the world.


قبل 22 أكتوبر 1844، كان هناك عدد كبير من الأدفنتست ينتظرون عودة المسيح على سحاب السماء. كانت حركة عالمية من الناس ينتظرون مجيئه الثاني. مر يوم 22 أكتوبر دون أن ينزل المسيح على السحاب، وتركت الغالبية العظمى الحركة، مستهزئين بها، ومستهزئين بالنبوات، والكتاب المقدس، والله. بقي عدد قليل من الرجال والنساء المخلصين المتواضعين، الذين كانوا متأكدين بلا شك أن الله كان يقود هذه الحركة. عرفوا أن الله كان يشع نور الحق وكانت قلوبهم متلهفة لتلقيه. لكن في نظر العالم، كانوا مجرد متعصبين وحالمين. يمكن مقارنة هذه الخيبة العظيمة بتلك التي شعر بها تلاميذ يسوع بعد أن رأوا ربهم موضوعًا في القبر. كانوا متأكدين بلا شك أن المسيح “\textit{كان نبيًا مقتدرًا في الفعل والقول أمام الله وجميع الشعب}”، لكن عندما مات على الصليب، أصيبوا بخيبة أمل مريرة، لأنهم “\textit{كانوا يرجون أنه هو المزمع أن يفدي إسرائيل}.” ومع ذلك، في حالة اليأس، في حالة خيبة الأمل الذاتية، كانوا مستعدين لتلقي القوة لغزو العالم كله بالإنجيل. التقوا بالمسيح وتلقوا روحه لاحقًا. حدث الشيء نفسه مع رواد الأدفنتست. كانوا مجموعة صغيرة من الناس، خائبي الأمل بمرارة؛ طلبوا الرب بكل قلوبهم وتلقوه بالقوة والحق. الحقائق التي كشفها الله خلال هذا الوقت الثمين من الأزمة تشكل أساس إيمان الأدفنتست السبتيين. تم اختبار هذه الحقائق من قبل جميع النظريات المغرية والخادعة في العالم، من قبل أولئك الذين يستهزئون بهذه المجموعة الصغيرة، ومع ذلك سادت هذه الحقائق العظيمة. في وقت الحاجة الأكبر، أعطى يسوع شهادته برفع فتاة صغيرة، أضعف الضعفاء، للموافقة على كل حقائقه. لم تكن إلين وايت مصدر الحقائق؛ بل لدعم الإخوة الذين كانوا يبحثون عن الحق في الكتاب المقدس. استخدم الله إلين وايت للموافقة على دراساتهم وتوجيههم إلى الكتاب المقدس. كانت النتيجة النهائية هي تأسيس أساس الإيمان المبني على الكتاب المقدس، الذي يبقى ثابتًا حتى نهاية العالم.


Would you be surprised to know that the foundation of Seventh-day Adventist faith, which was laid at the beginning of our work, is in a fair degree different from what it is currently? Today, more than a century and a half later, we marvel in amazement over the accounts of the experiences of our pioneers; but since then, the Seventh-day Adventist Church has been subject to several new movements. Since then, the church has experienced many changes, including changes in our doctrine. Some argue that these changes are good and progressive; others argue that they are destructive and deceptive. Moving the spotlight to the original Seventh-day Adventism, it initiates the great controversy in the present days. We have personally been in this controversy for over 6 years now and we have seen that it will only get bigger and stronger, often with results of a sad record. Many people from both sides of this controversy are rejecting the Spirit of Prophecy in one way or another. Some have left the Seventh-day Adventist Church altogether. The Adventist identity is either lost or drastically changed from the initial one.


هل ستتفاجأ بمعرفة أن أساس إيمان الأدفنتست السبتيين، الذي تم وضعه في بداية عملنا، يختلف إلى حد كبير عما هو عليه حاليًا؟ اليوم، بعد أكثر من قرن ونصف، نندهش بإعجاب من روايات تجارب روادنا؛ ولكن منذ ذلك الحين، خضعت كنيسة الأدفنتست السبتيين لعدة حركات جديدة. منذ ذلك الحين، شهدت الكنيسة العديد من التغييرات، بما في ذلك تغييرات في عقيدتنا. يجادل البعض بأن هذه التغييرات جيدة وتقدمية؛ ويجادل آخرون بأنها مدمرة وخادعة. إن تسليط الضوء على الأدفنتستية الأصلية يبدأ الصراع العظيم في الأيام الحاضرة. لقد كنا شخصيًا في هذا الصراع لأكثر من 6 سنوات الآن ورأينا أنه سيزداد حجمًا وقوة فقط، غالبًا مع نتائج ذات سجل حزين. العديد من الناس من كلا جانبي هذا الصراع يرفضون روح النبوة بطريقة أو بأخرى. ترك البعض كنيسة الأدفنتست السبتيين تمامًا. الهوية الأدفنتستية إما ضائعة أو تغيرت بشكل جذري عن الهوية الأولية.


We are currently witnessing the shaking of the Seventh-day Adventist church, seeing her tossed through one wave of crisis after another. Many are losing their faith and their identity as Seventh-day Adventists. But we believe in a solution that the Lord, in His mercy, has already provided. The solution can be found in the history of the Seventh-day Adventist movement.


نحن نشهد حاليًا اهتزاز كنيسة الأدفنتست السبتيين، ونراها تتقاذفها موجة تلو الأخرى من الأزمات. كثيرون يفقدون إيمانهم وهويتهم كأدفنتست سبتيين. لكننا نؤمن بحل قدمه الرب، في رحمته، بالفعل. يمكن العثور على الحل في تاريخ حركة الأدفنتست السبتيين.


\egw{\textbf{In reviewing our past history}, having traveled over every step of advance to our present standing, I can say, Praise God! As I see what the Lord has wrought, I am filled with astonishment, and with confidence in Christ as leader. \textbf{We have nothing to fear for the future, \underline{except as we shall forget} the way the Lord has led us, and \underline{His teaching} in our past history}.}[LS 196.2; 1915][https://egwwritings.org/read?panels=p41.1083]


\egw{\textbf{عند مراجعة تاريخنا الماضي}، بعد أن سرت على كل خطوة من خطوات التقدم إلى موقفنا الحالي، يمكنني أن أقول، سبحوا الله! عندما أرى ما صنعه الرب، أمتلئ بالدهشة، وبالثقة في المسيح كقائد. \textbf{ليس لدينا ما نخشاه للمستقبل، \underline{إلا إذا نسينا} الطريقة التي قادنا بها الرب، و\underline{تعليمه} في تاريخنا الماضي}.}[LS 196.2; 1915][https://egwwritings.org/read?panels=p41.1083]


We shall not fear! This is a great reassurance and promise—though conditional. We must \textit{remember} how the Lord has led us, and \textit{His teaching in our past history}. When we look at what the Lord has taught us in our past history, we are surprised to see how things have changed. The change has taken several years and many crises. To judge these changes in doctrine, whether good and progressive or bad and destructive, evaluation should be based on past experiences, as the Lord clearly led His church.


لن نخاف! هذا تطمين ووعد عظيم - وإن كان مشروطًا. يجب أن \textit{نتذكر} كيف قادنا الرب، و\textit{تعليمه في تاريخنا الماضي}. عندما ننظر إلى ما علمنا الرب في تاريخنا الماضي، نتفاجأ برؤية كيف تغيرت الأمور. استغرق التغيير عدة سنوات والعديد من الأزمات. للحكم على هذه التغييرات في العقيدة، سواء كانت جيدة وتقدمية أو سيئة ومدمرة، يجب أن يستند التقييم إلى التجارب السابقة، حيث قاد الرب بوضوح كنيسته.


At this time, we put forth a bold claim—one that is supposed to make you hold this book until the end of its cover. Encouraged by the counsels of Ellen White to review our past history, we have concluded that we have forgotten one crucial pillar of our faith, which was the main subject of Kellogg’s controversy—the \emcap{personality of God}. One of the biggest crises that the SDA Church ever had in the time of the living prophet was the Kellogg crisis. It is out of this crisis that many other crises, today, find their roots. In this light, the subject of the \emcap{personality of God} is pivotal in our present time.


في هذا الوقت، نقدم ادعاءً جريئًا - من المفترض أن يجعلك تمسك بهذا الكتاب حتى نهاية غلافه. بتشجيع من نصائح إلين وايت لمراجعة تاريخنا الماضي، استنتجنا أننا نسينا ركنًا حاسمًا من أركان إيماننا، والذي كان الموضوع الرئيسي لجدل كيلوغ - \emcap{شخصانية الله}. كانت أزمة كيلوغ واحدة من أكبر الأزمات التي واجهتها كنيسة الأدفنتست السبتيين على الإطلاق في زمن النبي الحي. من هذه الأزمة تجد العديد من الأزمات الأخرى، اليوم، جذورها. في ضوء ذلك، فإن موضوع \emcap{شخصانية الله} محوري في وقتنا الحاضر.


Sister White wrote to Kellogg that the \emcap{personality of God} and the \emcap{personality of Christ} was a pillar of our faith in the same rank as is the sanctuary message:


كتبت الأخت وايت إلى كيلوغ أن \emcap{شخصانية الله} و\emcap{شخصانية المسيح} كانت ركنًا من أركان إيماننا في نفس مرتبة رسالة المقدس:


\egw{Those who seek to remove \textbf{the old landmarks} are not holding fast; they \textbf{are \underline{not remembering} how they have received and heard}. Those who try to \textbf{\underline{bring in} theories that would remove \underline{the pillars of our faith} concerning the sanctuary, \underline{or concerning the personality of God or of Christ}, are working as blind men}. They are seeking to bring in uncertainties and to set the people of God adrift, without an anchor.}[Ms62-1905.14][https://egwwritings.org/read?panels=p14070.10026020]


\egw{أولئك الذين يسعون لإزالة \textbf{المعالم القديمة} لا يتمسكون بقوة؛ إنهم \textbf{\underline{لا يتذكرون} كيف استلموا وسمعوا}. أولئك الذين يحاولون \textbf{\underline{إدخال} نظريات من شأنها أن تزيل \underline{أعمدة إيماننا} المتعلقة بالمقدس، \underline{أو المتعلقة بشخصانية الله أو المسيح}، يعملون كرجال عميان}. إنهم يسعون لإدخال الشكوك وإطلاق شعب الله بلا مرساة.}[Ms62-1905.14][https://egwwritings.org/read?panels=p14070.10026020]


The \emcap{personality of God} receives very little attention today as a subject, yet it is one of the crucial elements in dealing with other doctrines pertaining to Adventism, such as the doctrine of Trinity, the Sanctuary service, 1844 and any other doctrine dealing with the Heavenly reality.


إن \emcap{شخصانية الله} تحظى باهتمام قليل جدًا اليوم كموضوع، ومع ذلك فهي أحد العناصر الحاسمة في التعامل مع العقائد الأخرى المتعلقة بالأدفنتست، مثل عقيدة الثالوث، وخدمة المقدس، وعام 1844 وأي عقيدة أخرى تتعامل مع الواقع السماوي.


The \emcap{personality of God} was a pillar of our faith. Today, it is almost forgotten. We propose a reasonable explanation for that. It is due to the evolution of the English language. What is meant by the term, “\textit{the personality of God}”? The general understanding of the English word ‘\textit{personality}’ has changed over the years. Today, ‘\textit{personality}’ is generally viewed as, “\textit{the characteristic set of behaviors, cognitions, and emotional patterns}”\footnote{Wikipedia Contributors. “\textit{Personality.}” Wikipedia, Wikimedia Foundation, 19 Apr. 2019, \href{https://en.wikipedia.org/wiki/Personality}{en.wikipedia.org/wiki/Personality}.}, but in the nineteenth, and beginning of the twentieth century, it meant “\textit{the quality or state of \textbf{being a person}}”\footnote{\href{https://www.merriam-webster.com/dictionary/personality}{Merriam-Webster Dictionary}, - ‘personality’} \footnote{\href{https://babel.hathitrust.org/cgi/pt?id=mdp.39015050663213&view=1up&seq=780}{Hunter Robert, The American encyclopaedic dictionary}, ‘\textit{personality}’ - “\textit{the quality or state of being personal}”; Mentioned dictionary was in possession of Ellen White (see \href{https://repo.adventistdigitallibrary.org/PDFs/adl-22/adl-22251050.pdf?_ga=2.116010630.1065317374.1621993520-1506151612.1617862694&fbclid=IwAR3vwmp8jxtnpPEKv0KD9mCv8dJpmRGoyIXW0CkbQAjbU0h6YaBGqhgBzbk}{EGW Private and Office Libraries})}. We read this definition as the primary definition of the word ‘\textit{personality}’ from the Merriam-Webster Dictionary\footnote{\href{https://www.merriam-webster.com/dictionary/personality\#word-history}{Merriam-Webster Dictionary} marks that the first record of the definition “the quality or state of being a person” is recorded in the 15th century.}. When Sister White and our pioneers wrote about the \emcap{personality of God}, they referred to \textit{the quality or state of God being a person}. In other words, they dealt with the question, “\textit{is God a person}”, and, “\textit{what is it that makes Him a person}” or “\textit{what is the quality or state of God being a person}”? Try to remember the last time you had a Bible study on the question, “\textit{is God a person?}” Think about how you can prove to yourself, from the Bible, that God is a person. Think about it. It is an important question. Upon this question hangs your view of God and your relationship toward Him. The \emcap{personality of God} is fundamental to true spirituality; true spirituality is based on your personal relationship with God. No real relationship of any kind can be formed with anyone unless he/she is a person. Maybe you have never asked yourself this question because you never felt a need to question if God is a person, and what is it (the quality or state) that makes Him a person. Or, maybe you were refraining from this question because you felt it might be a mystery that God did not intend to reveal. Maybe it will surprise you to know that God has given a definite and affirmative answer in His Word to the question “\textit{what is the quality or state of God being a person}”. What was even more surprising for us, was that the Adventist pioneers, including Sister White, had definite light regarding this topic, and they held it as a pillar of our faith, as part of the foundation of Seventh-day Adventist faith. When the \emcap{personality of God} is rightly understood in light of our historical past, old quotations shine in a new light and new shreds of evidence are presented, which will deepen the understanding of our past history and the present crisis.


كانت \emcap{شخصانية الله} عمودًا من أعمدة إيماننا. واليوم، تم نسيانها تقريبًا. نقترح تفسيرًا معقولًا لذلك. يرجع ذلك إلى تطور اللغة الإنجليزية. ما المقصود بمصطلح “\textit{شخصانية الله}”؟ لقد تغير الفهم العام للكلمة الإنجليزية ‘\textit{personality}’ (شخصانية) على مر السنين. اليوم، يُنظر إلى ‘\textit{personality}’ (الشخصانية) عمومًا على أنها “\textit{مجموعة السلوكيات والإدراكات والأنماط العاطفية المميزة}”\footnote{Wikipedia Contributors. “\textit{Personality.}” Wikipedia, Wikimedia Foundation, 19 Apr. 2019, \href{https://en.wikipedia.org/wiki/Personality}{en.wikipedia.org/wiki/Personality}.}، ولكن في القرن التاسع عشر وبداية القرن العشرين، كانت تعني “\textit{الصفة أو الحالة التي يكون بها الكائن شخصًا}”\footnote{\href{https://www.merriam-webster.com/dictionary/personality}{Merriam-Webster Dictionary}, - ‘personality’} \footnote{\href{https://babel.hathitrust.org/cgi/pt?id=mdp.39015050663213&view=1up&seq=780}{Hunter Robert, The American encyclopaedic dictionary}, ‘\textit{personality}’ - “\textit{the quality or state of being personal}”; Mentioned dictionary was in possession of Ellen White (see \href{https://repo.adventistdigitallibrary.org/PDFs/adl-22/adl-22251050.pdf?_ga=2.116010630.1065317374.1621993520-1506151612.1617862694&fbclid=IwAR3vwmp8jxtnpPEKv0KD9mCv8dJpmRGoyIXW0CkbQAjbU0h6YaBGqhgBzbk}{EGW Private and Office Libraries})}. نقرأ هذا التعريف كتعريف أساسي لكلمة ‘\textit{personality}’ (شخصانية) من قاموس ميريام-وبستر\footnote{\href{https://www.merriam-webster.com/dictionary/personality\#word-history}{Merriam-Webster Dictionary} marks that the first record of the definition “the quality or state of being a person” is recorded in the 15th century.}. عندما كتبت الأخت وايت وروادنا عن \emcap{شخصانية الله}، كانوا يشيرون إلى \textit{الصفة أو الحالة التي يكون بها الله شخصًا}. بعبارة أخرى، تعاملوا مع السؤال، “\textit{هل الله شخص}”، و”\textit{ما الذي يجعله شخصًا}” أو “\textit{ما هي الصفة أو الحالة التي يكون بها الله شخصًا}”؟ حاول أن تتذكر آخر مرة كان لديك دراسة كتابية حول سؤال، “\textit{هل الله شخص؟}” فكر في كيف يمكنك أن تثبت لنفسك، من الكتاب المقدس، أن الله شخص. فكر في الأمر. إنه سؤال مهم. على هذا السؤال يتوقف نظرتك إلى الله وعلاقتك به. إن \emcap{شخصانية الله} أساسية للروحانية الحقيقية؛ الروحانية الحقيقية تستند على علاقتك الشخصية مع الله. لا يمكن تكوين أي علاقة حقيقية من أي نوع مع أي شخص ما لم يكن شخصًا. ربما لم تسأل نفسك هذا السؤال من قبل لأنك لم تشعر بالحاجة إلى التساؤل عما إذا كان الله شخصًا، وما هي (الصفة أو الحالة) التي تجعله شخصًا. أو ربما كنت تمتنع عن هذا السؤال لأنك شعرت أنه قد يكون سرًا لم ينوِ الله الكشف عنه. ربما سيفاجئك أن تعرف أن الله قد أعطى إجابة محددة وإيجابية في كلمته على سؤال “\textit{ما هي الصفة أو الحالة التي يكون بها الله شخصًا}”. ما كان أكثر إثارة للدهشة بالنسبة لنا، هو أن رواد الأدفنتست، بما في ذلك الأخت وايت، كان لديهم نور محدد بخصوص هذا الموضوع، واعتبروه عمودًا من أعمدة إيماننا، كجزء من أساس إيمان الأدفنتست السبتيين. عندما يتم فهم \emcap{شخصانية الله} بشكل صحيح في ضوء ماضينا التاريخي، تشرق الاقتباسات القديمة بنور جديد وتُقدم أدلة جديدة، مما سيعمق فهم تاريخنا الماضي والأزمة الحالية.


The root problem of the Kellogg crisis was about the \emcap{personality of God}. It is certainly important to evaluate Kellogg's crisis over the \emcap{personality of God} using the meaning intended at that time; that is, using the definition of ‘\textit{personality},’ as the quality or state of God being a person. With this definition in mind, the Kellogg crisis comes into a new light and new relevant evidence is brought forth for us today. In light of this evidence, we see how God has led us in the past; thus, we should not fear for the future. Knowing and understanding this, as well as its importance, helps us to not be shaken by any wave of deception in present controversies. When Sister White was drawing Kellogg’s attention to the importance of this subject, she was drawing our attention also, as it is everything to us as a people.


كانت المشكلة الأساسية في أزمة كيلوغ تتعلق بـ \emcap{شخصانية الله}. من المهم بالتأكيد تقييم أزمة كيلوغ حول \emcap{شخصانية الله} باستخدام المعنى المقصود في ذلك الوقت؛ أي باستخدام تعريف ‘\textit{personality}’ (شخصانية)، كصفة أو حالة كون الله شخصًا. مع هذا التعريف في الاعتبار، تظهر أزمة كيلوغ في ضوء جديد وتُقدم أدلة جديدة ذات صلة لنا اليوم. في ضوء هذه الأدلة، نرى كيف قادنا الله في الماضي؛ لذلك، يجب ألا نخاف من المستقبل. معرفة وفهم هذا، وكذلك أهميته، يساعدنا على عدم التزعزع بأي موجة من الخداع في الصراعات الحالية. عندما كانت الأخت وايت تلفت انتباه كيلوغ إلى أهمية هذا الموضوع، كانت تلفت انتباهنا أيضًا، لأنه كل شيء بالنسبة لنا كشعب.


[Writing to Kellogg] \egw{You are not definitely clear on \textbf{the personality of God}, which is \textbf{\underline{everything} to us as a people}.}[Lt300-1903.7][https://egwwritings.org/read?panels=p14068.7705013]


[كتابة إلى كيلوغ] \egw{أنت لست واضحًا تمامًا بشأن \textbf{شخصانية الله}، التي هي \textbf{\underline{كل شيء} بالنسبة لنا كشعب}.}[Lt300-1903.7][https://egwwritings.org/read?panels=p14068.7705013]


These studies on the \emcap{personality of God} will prompt a lot of new and hard questions. We do not promise to answer all of them, and perhaps you won’t be satisfied with the answers provided, but we pray, hope and believe that this book will fulfill the three objectives proposed in the beginning of this introduction. Through the reviving of the doctrine on the \emcap{personality of God}, we believe that your confidence in the Spirit of Prophecy will strengthen, and that you’ll find yourself rooted deeper in the Adventist message—where we find our identity as people—making you a more faithful Seventh-day Adventist. Most importantly, we want you to become more aware of God as your personal God. This will surely strengthen and deepen your relationship with Him.


ستثير هذه الدراسات حول \emcap{شخصانية الله} الكثير من الأسئلة الجديدة والصعبة. نحن لا نعد بالإجابة على جميعها، وربما لن تكون راضيًا عن الإجابات المقدمة، لكننا نصلي ونأمل ونؤمن بأن هذا الكتاب سيحقق الأهداف الثلاثة المقترحة في بداية هذه المقدمة. من خلال إحياء العقيدة حول \emcap{شخصانية الله}، نعتقد أن ثقتك في روح النبوة ستتقوى، وأنك ستجد نفسك متجذرًا بعمق أكبر في رسالة الأدفنتست - حيث نجد هويتنا كشعب - مما يجعلك أدفنتستيًا سبتيًا أكثر أمانة. الأهم من ذلك، نريدك أن تصبح أكثر وعيًا بالله كإلهك الشخصي. هذا سيقوي ويعمق علاقتك به بالتأكيد.


We find answers to the issue on the \emcap{personality of God} in examining the Kellogg crisis, where Sister White gave the most definite light on the \emcap{personality of God} and on the foundation of Seventh-day Adventist faith. The following is the complete tenth chapter from the book, \textit{Testimonies for the Church Containing Letters to Physicians and Ministers Instruction to Seventh-Day Adventists}. This chapter, \textit{The Foundation of our Faith}, contains deep insight into the history of Kellogg’s crisis. It gives a historical overview of the truths God gave as the foundation of our faith and in these truths we find our identity as Seventh-day Adventists— keeping the commandments of God and having the faith of Jesus.


نجد إجابات لقضية \emcap{شخصانية الله} في دراسة أزمة كيلوغ، حيث قدمت الأخت وايت النور الأكثر تحديدًا حول \emcap{شخصانية الله} وحول أساس إيمان الأدفنتست السبتيين. فيما يلي الفصل العاشر الكامل من كتاب \textit{تستومنيز فور ذا شرش كنتاينينغ لترز تو فيزيشنز أند منسترز انستركشن تو سفنث داي أدفنتستس}. يحتوي هذا الفصل، \textit{أساس إيماننا}، على نظرة عميقة في تاريخ أزمة كيلوغ. إنه يقدم نظرة عامة تاريخية عن الحقائق التي أعطاها الله كأساس لإيماننا وفي هذه الحقائق نجد هويتنا كأدفنتست سبتيين - حافظين وصايا الله ولهم إيمان يسوع.
