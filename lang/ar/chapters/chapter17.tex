\qrchapter{https://forgottenpillar.com/rsc/en-fp-chapter17}{Reply to Kellogg’s trinitarian sentiments}


\qrchapter{https://forgottenpillar.com/rsc/en-fp-chapter17}{الرد على آراء كيلوغ الثالوثية}


If we look at the Kellogg crisis through the perspective of the \emcap{personality of God} and the \emcap{Fundamental Principles}, Sister White’s quotations inevitably shine in a new light. In this light we see the conflict between the truth we have received in the beginning, on the \emcap{personality of God}, and the Trinity doctrine. In order to avoid discrepancy, in the interest of defending the Trinity doctrine, scholars always overemphasize the pantheistic side of the problem.


إذا نظرنا إلى أزمة كيلوغ من منظور \emcap{شخصانية الله} و\emcap{المبادئ الجوهرية}، فإن اقتباسات الأخت وايت تتألق حتماً في ضوء جديد. في هذا الضوء نرى الصراع بين الحق الذي تلقيناه في البداية، حول \emcap{شخصانية الله}، وعقيدة الثالوث. ولتجنب التناقض، وفي سبيل الدفاع عن عقيدة الثالوث، يبالغ العلماء دائماً في التركيز على الجانب الوحدوي من المشكلة.


We would like to challenge this tendency to overemphasize the pantheistic side of Kellogg’s controversy. Sister White generally wrote proactive truth; she approached the error by uplifting the truth. This is why she wrote so much about the \emcap{personality of God}. In most of her quotations on this subject, we see her dispelling the Trinitarian error, rather than pantheistic error. We read one such example where she establishes the truth on the \emcap{personality of God} referencing the seventeenth chapter of John.


نود أن نتحدى هذا الميل للمبالغة في التركيز على الجانب الوحدوي من صراع كيلوغ. كتبت الأخت وايت عموماً حقاً استباقياً؛ فقد تناولت الخطأ من خلال رفع الحق. لهذا السبب كتبت الكثير عن \emcap{شخصانية الله}. في معظم اقتباساتها حول هذا الموضوع، نراها تبدد الخطأ الثالوثي، بدلاً من الخطأ الوحدوي. نقرأ مثالاً على ذلك حيث تؤسس الحق حول \emcap{شخصانية الله} مشيرة إلى الفصل السابع عشر من إنجيل يوحنا.


\egw{\textbf{The personality of the Father and the Son, also the unity that exists between Them, are presented in the seventeenth chapter of John}, in the prayer of Christ for His disciples:}[MH 421.7; 1905][https://egwwritings.org/read?panels=p135.2173]


\egw{\textbf{إن شخصانية الآب والابن، وكذلك الوحدة القائمة بينهما، معروضتان في الفصل السابع عشر من إنجيل يوحنا}، في صلاة المسيح من أجل تلاميذه:}[MH 421.7; 1905][https://egwwritings.org/read?panels=p135.2173]


There are many cases where Sister White quotes John 17 in regard to Kellogg’s crisis. Those who assert that Kellogg’s crisis was solely about pantheism should inquire how John 17 addresses God in nature. And it is not only John 17, but also chapters 13-16. In her letter to Kellogg, she wrote:


هناك حالات كثيرة تقتبس فيها الأخت وايت يوحنا 17 فيما يتعلق بأزمة كيلوغ. على أولئك الذين يؤكدون أن أزمة كيلوغ كانت تتعلق فقط بالوحدوية أن يستفسروا كيف يتناول يوحنا 17 الله في الطبيعة. وليس فقط يوحنا 17، بل أيضاً الفصول 13-16. في رسالتها إلى كيلوغ، كتبت:


\egw{\textbf{\underline{…study the thirteenth, fourteenth, fifteenth, sixteenth, and seventeenth chapters of John}. The words of these chapters explain themselves. ‘This is life eternal,’ Christ declared, ‘that they might know \underline{Thee the only true God}, and Jesus Christ, whom Thou hast sent.’ \underline{In these words the personality of God and of His Son is clearly spoken of.} \underline{The personality of the one does not do away with the necessity for the personality of the other}.}}[Lt232-1903.48, 1903][https://egwwritings.org/read?panels=p10197.57]


\egw{\textbf{\underline{...ادرس الفصول الثالث عشر والرابع عشر والخامس عشر والسادس عشر والسابع عشر من إنجيل يوحنا}}. كلمات هذه الفصول تشرح نفسها. “وهذه هي الحياة الأبدية،“ أعلن المسيح، “أن يعرفوك \underline{أنت الإله الحقيقي وحدك}، ويسوع المسيح الذي أرسلته.” \underline{في هذه الكلمات يتم التحدث بوضوح عن شخصانية الله وابنه.} \underline{شخصانية أحدهما لا تلغي ضرورة شخصانية الآخر}.}}[Lt232-1903.48, 1903][https://egwwritings.org/read?panels=p10197.57]


In the aforementioned chapters of John, John did not reference anything pertaining to God in nature. The content of those chapters covers who is the only true God, how the Father and the Son are one, their true relation, and how Jesus can be everywhere present yet will ascend to the Father.


في الفصول المذكورة من إنجيل يوحنا، لم يشر يوحنا إلى أي شيء يتعلق بالله في الطبيعة. يغطي محتوى تلك الفصول من هو الإله الحقيقي الوحيد، وكيف أن الآب والابن واحد، وعلاقتهما الحقيقية، وكيف يمكن ليسوع أن يكون حاضراً في كل مكان مع أنه سيصعد إلى الآب.


\egw{Jesus said to the Jews: ‘My Father worketh hitherto, and I work.... The Son can do nothing of Himself, but what He seeth the Father do: for what things soever He doeth, these also doeth the Son likewise. For the Father loveth the Son, and showeth Him all things that Himself doeth.’ John 5:17-20.}[8T 268.4, 1904][https://egwwritings.org/read?panels=p112.1557]


\egw{قال يسوع لليهود: “أبي يعمل حتى الآن وأنا أعمل.... لا يقدر الابن أن يعمل من نفسه شيئاً إلا ما ينظر الآب يعمل. لأن مهما عمل ذاك فهذا يعمله الابن كذلك. لأن الآب يحب الابن ويريه جميع ما هو يعمله.” يوحنا 5: 17-20.}[8T 268.4, 1904][https://egwwritings.org/read?panels=p112.1557]


\egwnogap{\textbf{Here again is brought to view the \underline{personality of the Father and the Son}, showing the unity that exists between them}.}[8T 269.1; 1904][https://egwwritings.org/read?panels=p112.1560]


\egwnogap{\textbf{هنا مرة أخرى تُعرض \underline{شخصانية الآب والابن}، موضحة الوحدة القائمة بينهما}.}[8T 269.1; 1904][https://egwwritings.org/read?panels=p112.1560]


\egwnogap{\textbf{This unity is expressed also in \underline{the seventeenth chapter of John}}, in the prayer of Christ for His disciples:}[8T 269.2; 1904][https://egwwritings.org/read?panels=p112.1561]


\egwnogap{\textbf{هذه الوحدة معبر عنها أيضاً في \underline{الفصل السابع عشر من إنجيل يوحنا}}، في صلاة المسيح من أجل تلاميذه:}[8T 269.2; 1904][https://egwwritings.org/read?panels=p112.1561]


\egwnogap{‘Neither pray I for these alone, but for them also which shall believe on Me through their word; that they all may be one; \textbf{as Thou, Father, art in Me, and I in Thee, that they also may be one in Us}: that the world may believe that Thou hast sent Me. And \textbf{the glory which Thou gavest Me} I have given them; \textbf{that they may be one, even as We are one: I in them, and Thou in Me, that they may be made perfect in one}; and that the world may know that Thou hast sent Me, and hast loved them, as Thou hast loved Me.’ John 17:20-23.}[8T 269.3; 1904][https://egwwritings.org/read?panels=p112.1562]


\egwnogap{‘لا أسأل من أجل هؤلاء فقط، بل أيضًا من أجل الذين يؤمنون بي بواسطة كلامهم، ليكون الجميع واحدًا؛ \textbf{كما أنك أنت أيها الآب فيّ وأنا فيك، ليكونوا هم أيضًا واحدًا فينا}: لكي يؤمن العالم أنك أرسلتني. وأنا قد أعطيتهم \textbf{المجد الذي أعطيتني}؛ \textbf{ليكونوا واحدًا كما نحن واحد: أنا فيهم وأنت فيّ ليكونوا مكملين إلى واحد}؛ وليعلم العالم أنك أرسلتني، وأحببتهم كما أحببتني.’ يوحنا ١٧: ٢٠-٢٣.}[8T 269.3; 1904][https://egwwritings.org/read?panels=p112.1562]


\egwnogap{Wonderful statement! \textbf{The unity that exists between Christ and His disciples \underline{does not destroy the personality of either}. They are one in purpose, in mind, in character, but \underline{not in person}. It is thus that God and Christ are one}.}[8T 269.4; 1904][https://egwwritings.org/read?panels=p112.1563]


\egwnogap{بيان رائع! \textbf{الوحدة الموجودة بين المسيح وتلاميذه \underline{لا تدمر شخصانية أي منهما}. هم واحد في الهدف، في الفكر، في الصفة، ولكن \underline{ليس في الشخص}. هكذا الله والمسيح واحد}.}[8T 269.4; 1904][https://egwwritings.org/read?panels=p112.1563]


\egwnogap{\textbf{The relation between the Father and the Son, and the personality of both, are made plain in this scripture also}:}[8T 269.5; 1904][https://egwwritings.org/read?panels=p112.1564]


\egwnogap{\textbf{العلاقة بين الآب والابن، وشخصانية كليهما، موضحة أيضًا في هذا النص الكتابي}:}[8T 269.5; 1904][https://egwwritings.org/read?panels=p112.1564]


\egwnogap{Thus speaketh \textbf{Jehovah of hosts}, saying,} \\
\egw{Behold, \textbf{the man} whose name is\textbf{ the Branch}:} \\
\egw{And He shall grow up out of His place;} \\
\egw{\textbf{And He shall build the temple of Jehovah;... }} \\
\egw{\textbf{And He shall bear the glory,}} \\
\egw{\textbf{And shall sit and rule upon His throne;}} \\
\egw{\textbf{And He shall be a priest upon His throne;}} \\
\egw{\textbf{And \underline{the counsel of peace shall be between Them both}}.’}[8T 269.6; 1904][https://egwwritings.org/read?panels=p112.1565]


\egwnogap{هكذا يتكلم \textbf{يهوه الجنود} قائلاً،} \\
\egw{هوذا \textbf{الرجل} الذي اسمه \textbf{الغصن}:} \\
\egw{وهو ينبت من مكانه؛} \\
\egw{\textbf{وهو يبني هيكل يهوه؛... }} \\
\egw{\textbf{وهو يحمل الجلال،}} \\
\egw{\textbf{ويجلس ويتسلط على عرشه؛}} \\
\egw{\textbf{ويكون كاهنًا على عرشه؛}} \\
\egw{\textbf{و\underline{تكون مشورة السلام بينهما كليهما}}.’}[8T 269.6; 1904][https://egwwritings.org/read?panels=p112.1565]


The aforementioned chapters of the Gospel of John deal with the \emcap{personality of God}, which had been expressed in the first two points of the \emcap{Fundamental Principles}. What error did Sister White combat when she referenced verses on how the Father was the only true God, and how the Father and the Son are not one in person? Pantheism? Certainly not; but most probably the trinitarian sentiments, or belief in a one-in-three, or three-in-one God.


تتناول الفصول المذكورة أعلاه من إنجيل يوحنا \emcap{شخصانية الله}، التي تم التعبير عنها في النقطتين الأولى والثانية من \emcap{المبادئ الأساسية}. ما هو الخطأ الذي حاربته الأخت وايت عندما أشارت إلى آيات عن كيف أن الآب هو الإله الحقيقي الوحيد، وكيف أن الآب والابن ليسا واحدًا في الشخص؟ وحدة الوجود؟ بالتأكيد لا؛ ولكن على الأرجح الآراء الثالوثية، أو الإيمان بإله واحد في ثلاثة، أو ثلاثة في واحد.


Brother J. N. Loughborough, one of the first brethren who wrote on the \emcap{personality of God}, wrote the following comment on John chapter 17:


كتب الأخ ج. ن. لوبورو، أحد الإخوة الأوائل الذين كتبوا عن \emcap{شخصانية الله}، التعليق التالي على الفصل ١٧ من إنجيل يوحنا:


\others{\textbf{\underline{The seventeenth chapter of John is alone sufficient to refute the doctrine of the Trinity}}. \textbf{...\underline{Read the seventeenth chapter of John, and see if it does not completely upset the doctrine of the Trinity}}.}[John N. Loughborough, The Adventist Review, and Sabbath Herald, November 5, 1861, p. 184.10][https://egwwritings.org/read?panels=p1685.6615]


\others{\textbf{\underline{الفصل السابع عشر من يوحنا وحده كافٍ لدحض عقيدة الثالوث}}. \textbf{...\underline{اقرأ الفصل السابع عشر من يوحنا، وانظر إذا كان لا يقلب عقيدة الثالوث رأسًا على عقب تمامًا}}.}[John N. Loughborough, The Adventist Review, and Sabbath Herald, November 5, 1861, p. 184.10][https://egwwritings.org/read?panels=p1685.6615]


Sister White’s proactive writing in support of the truth on the \emcap{personality of God} and His presence is the same as other Adventist pioneers. If Adventist pioneers were debunking the Trinity doctrine by exalting the truth on the \emcap{personality of God} and God’s presence, what makes us think Ellen White was not doing the same, when the theological side of the question of the Trinity was raised? By stating this, we do not deny the pantheistic side of Kellogg’s controversy, but by overemphasizing it, it falls short of accurately describing its real issue. The correct understanding of the Kellogg controversy can only be accomplished by focusing primarily on the truth Sister White uplifted, rather than focusing on error, whether pantheism or Trinity. This truth that Sister White uplifted was the truth on the \emcap{personality of God} and where His presence is. This is expressed in the first point of the \emcap{Fundamental Principles}, which were the official synopsis and representation of Seventh-day Adventist beliefs in the time of Ellen White; the truth which we, as a church, \egwinline{have received and heard and advocated}[Ms124-1905.12; 1905][https://egwwritings.org/read?panels=p9099.18] in the beginning.


كتابات الأخت وايت الاستباقية في دعم الحق حول \emcap{شخصانية الله} وحضوره هي نفسها كتابات رواد الأدفنتست الآخرين. إذا كان رواد الأدفنتست يفندون عقيدة الثالوث من خلال تمجيد الحق حول \emcap{شخصانية الله} وحضور الله، فما الذي يجعلنا نعتقد أن إلين وايت لم تكن تفعل الشيء نفسه، عندما أثيرت المسألة اللاهوتية للثالوث؟ بقولنا هذا، لا ننكر الجانب الوحدوي من صراع كيلوغ، ولكن بالمبالغة في التأكيد عليه، فإنه يقصر عن وصف قضيته الحقيقية بدقة. لا يمكن تحقيق الفهم الصحيح لصراع كيلوغ إلا من خلال التركيز بشكل أساسي على الحق الذي رفعته الأخت وايت، بدلاً من التركيز على الخطأ، سواء كان وحدة الوجود أو الثالوث. هذا الحق الذي رفعته الأخت وايت كان الحق حول \emcap{شخصانية الله} وأين حضوره. وهذا معبر عنه في النقطة الأولى من \emcap{المبادئ الأساسية}، التي كانت الموجز والتمثيل الرسمي لمعتقدات الأدفنتست السبتيين في زمن إلين وايت؛ الحق الذي نحن، ككنيسة، \egwinline{استلمناه وسمعناه ودافعنا عنه}[Ms124-1905.12; 1905][https://egwwritings.org/read?panels=p9099.18] في البداية.


\egw{\textbf{I entreat every one to be clear and firm regarding the certain truths that we have received and heard and advocated. The statements of God’s Word are plain. Plant your feet firmly on \underline{the platform of eternal truth}. \underline{Reject every phase of error}, even \underline{though it be covered with a semblance of reality, which denies the personality of God or of Christ}}.}[Ms124-1905.12; 1905][https://egwwritings.org/read?panels=p9099.18]


\egw{\textbf{أتوسل إلى كل واحد أن يكون واضحًا وثابتًا بخصوص الحقائق المؤكدة التي استلمناها وسمعناها ودافعنا عنها. بيانات كلمة الله واضحة. ثبت قدميك بحزم على \underline{منصة الحق الأبدي}. \underline{ارفض كل مظهر من مظاهر الخطأ}، حتى \underline{وإن كان مغطى بمظهر من الواقعية، الذي ينكر شخصانية الله أو المسيح}}.}[Ms124-1905.12; 1905][https://egwwritings.org/read?panels=p9099.18]


The warning from the previous quotations did not lessen in the course of time. Today it is even more relevant. We should \egwinline{reject every phase of error, even though it be covered with a semblance of reality, which denies the personality of God or of Christ}. In the following chapter we want to point out to the specific phase of error that is covered with a semblance of reality, which denies the personality of God and of Christ—three living persons of \textit{one} God, as opposed to \egwinline{three living persons of the heavenly trio.}[Ms21-1906.11; 1906][https://egwwritings.org/read?panels=p9754.18]


لم يقل التحذير من الاقتباسات السابقة مع مرور الوقت. اليوم هو أكثر صلة. يجب علينا أن \egwinline{نرفض كل مظهر من مظاهر الخطأ، حتى وإن كان مغطى بمظهر من الواقعية، الذي ينكر شخصانية الله أو المسيح}. في الفصل التالي نريد أن نشير إلى المظهر المحدد من الخطأ الذي يغطى بمظهر من الواقعية، والذي ينكر شخصانية الله والمسيح—ثلاثة أشخاص أحياء من إله \textit{واحد}، على عكس \egwinline{ثلاثة أشخاص أحياء من الثلاثي السماوي.}[Ms21-1906.11; 1906][https://egwwritings.org/read?panels=p9754.18]


% Reply to Kellogg’s trinitarian sentiments

\begin{titledpoem}
    
    \stanza{
        The light of truth, so clear and bold, \\
        A crisis came, a story told. \\
        Not pantheism, dim and wide, \\
        But God’s persona, we confide.
    }

    \stanza{
        But God, through Ellen, did uphold \\
        God’s personality was told. \\
        Against the Trinity, she leaned, \\
        A unity, by John unseen.
    }

    \stanza{
        "The Father and the Son," she wrote, \\
        Are one in purpose was her quote. \\
        John seventeen, her chosen guide, \\
        Where God’s true nature cannot hide.
    }

    \stanza{
        The pioneers, with her agreed, \\
        Of God’s true person, they did plead. \\
        Loughborough echoed, his words clear, \\
        The Trinity dismissed, no fear.
    }

    \stanza{
        The Fundamental Points, so dear, \\
        They make it plain, we must revere. \\
        Not in the trinity’s wrong creed, \\
        But in His presence, faith is freed.
    }

    \stanza{
        So let us stand on truth so bright, \\
        Rejecting wrong, with all our might. \\
        God’s person, where we find our plea, \\
        Truth’s platform for eternity.
    }
    
\end{titledpoem}