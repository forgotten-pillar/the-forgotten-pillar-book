\qrchapter{https://forgottenpillar.com/rsc/en-fp-chapter25}{Setting up the wrong Fundamental Principles}


\qrchapter{https://forgottenpillar.com/rsc/en-fp-chapter25}{إقامة المبادئ الأساسية الخاطئة}


You might ask yourself: how could it be possible that we, as a church, have gone astray from the light God gave us in the beginning? The answer to this question is the same answer to the question why the Jews went astray from the light God gave them concerning His Son. Please, take a look at the driving force behind the church in Apostolic times and our time.


قد تسأل نفسك: كيف يمكن أن نكون، ككنيسة، قد انحرفنا عن النور الذي أعطانا الله إياه في البداية؟ الإجابة على هذا السؤال هي نفس الإجابة على سؤال لماذا انحرف اليهود عن النور الذي أعطاهم الله بخصوص ابنه. من فضلك، ألقِ نظرة على القوة الدافعة وراء الكنيسة في عصر الرسل وفي عصرنا.


\egw{‘The angel of the Lord by night opened the prison doors, and brought them forth, and said, Go, stand and speak in the temple to the people all the words of this life.’ [Acts 5:19, 20.] We see here that the men in authority are not always obeyed, even though they may profess to be teachers of Bible doctrines. \textbf{There are many today who feel indignant and aggrieved that any voice should be raised presenting ideas that differ from their own in regard to points of religious belief}. \textbf{Have they not long advocated their ideas as truth?} So the priests and rabbis reasoned in apostolic days. What mean these men who are unlearned, some of them mere fishermen, who are presenting ideas contrary to the doctrines which the learned priests and rulers are teaching the people? \textbf{They have no right to meddle with the fundamental principles of our faith}.}[Lt38-1896.23; 1896][https://egwwritings.org/read?panels=p5631.29]


\egw{‘فتح ملاك الرب أبواب السجن ليلاً، وأخرجهم وقال: اذهبوا قفوا وكلموا الشعب في الهيكل بجميع كلام هذه الحياة.’ [أعمال 5: 19، 20]. نرى هنا أن الرجال في السلطة لا يُطاعون دائمًا، حتى وإن كانوا يدّعون أنهم معلمو عقائد الكتاب المقدس. \textbf{هناك الكثيرون اليوم الذين يشعرون بالغضب والاستياء من أن أي صوت يُرفع يقدم أفكارًا تختلف عن أفكارهم فيما يتعلق بنقاط المعتقد الديني}. \textbf{ألم يدافعوا طويلاً عن أفكارهم باعتبارها الحق؟} هكذا فكر الكهنة والحاخامات في أيام الرسل. ماذا يعني هؤلاء الرجال غير المتعلمين، بعضهم مجرد صيادين، الذين يقدمون أفكارًا مخالفة للعقائد التي يعلمها الكهنة والحكام المتعلمون للشعب؟ \textbf{ليس لديهم الحق في التدخل في المبادئ الأساسية لإيماننا}.}[Lt38-1896.23; 1896][https://egwwritings.org/read?panels=p5631.29]


\egwnogap{“\textbf{But we see that the God of heaven sometimes commissions men to \underline{teach that which is regarded as contrary to the established doctrines}. Because those who were once the depositaries of truth \underline{became unfaithful to their sacred trust}, the Lord chose others who would receive the bright beams of the Sun of Righteousness, and would advocate truths that were not in accordance with the ideas of the religious leaders. And then these leaders, in the blindness of their minds, give full sway to what is supposed to be righteous indignation against the ones who have set aside cherished fables. They act like men that have lost their reason. They do not consider the possibility that they themselves have not rightly understood the Word. They will not open their eyes to discern the fact that they have misinterpreted and misapplied the Scriptures, and have built up false theories, \underline{calling them fundamental doctrines of the faith}}.“}[Lt38-1896.24; 1896][https://egwwritings.org/read?panels=p5631.30]


\egwnogap{“\textbf{لكننا نرى أن إله السماء يكلف أحيانًا رجالاً \underline{بتعليم ما يُعتبر مخالفًا للعقائد الراسخة}. لأن أولئك الذين كانوا في يوم من الأيام مستودعات للحق \underline{أصبحوا غير أمناء لأمانتهم المقدسة}، اختار الرب آخرين ليستقبلوا الأشعة المشرقة لشمس البر، وليدافعوا عن حقائق لم تكن متوافقة مع أفكار القادة الدينيين. وحينئذ هؤلاء القادة، في عمى أذهانهم، يطلقون العنان لما يُفترض أنه غضب بار ضد أولئك الذين وضعوا جانبًا الخرافات المحبوبة. إنهم يتصرفون مثل أناس فقدوا عقولهم. لا يفكرون في احتمال أنهم هم أنفسهم لم يفهموا الكلمة بشكل صحيح. لن يفتحوا أعينهم ليدركوا حقيقة أنهم قد أساءوا تفسير وتطبيق الكتب المقدسة، وبنوا نظريات خاطئة، \underline{يسمونها العقائد الأساسية للإيمان}}.”}[Lt38-1896.24; 1896][https://egwwritings.org/read?panels=p5631.30]


\egwnogap{\textbf{But the Holy Spirit will from time to time reveal the truth through its own chosen agencies; and no man, not even a priest or ruler, has a right to say, You shall not give publicity to your opinions, because I do not believe them. That wonderful ‘I’ may attempt to put down the Holy Spirit’s teaching. Men may, for a time, attempt to smother it and kill it; but that will not make error truth or truth error. The inventive minds of men have advanced speculative opinions in various lines, and when the Holy Spirit lets light shine into human minds, it does not respect every point of man’s application of the word. God impressed his servants to speak the truth irrespective of what men had taken for granted as truth}.}[Lt38-1896.25; 1896][https://egwwritings.org/read?panels=p5631.31]


\egwnogap{\textbf{لكن الروح القدس سيكشف من وقت لآخر الحق من خلال وكلائه المختارين؛ وليس لأي إنسان، ولا حتى كاهن أو حاكم، الحق في أن يقول: لا يجوز لك أن تنشر آراءك، لأنني لا أؤمن بها. ذلك ‘الأنا’ العجيب قد يحاول إخماد تعليم الروح القدس. قد يحاول الناس، لفترة من الوقت، خنقه وقتله؛ لكن ذلك لن يجعل الخطأ حقًا أو الحق خطأً. لقد طرحت العقول المبتكرة للبشر آراء تخمينية في مختلف المجالات، وعندما يسمح الروح القدس للنور أن يشرق في العقول البشرية، فإنه لا يحترم كل نقطة من تطبيق الإنسان للكلمة. لقد ألهم الله خدامه ليتكلموا بالحق بغض النظر عما اعتبره الناس مسلمًا به كحق}.}[Lt38-1896.25; 1896][https://egwwritings.org/read?panels=p5631.31]


\egwnogap{\textbf{\underline{Even Seventh-day Adventists are in danger of closing their eyes to truth as it is in Jesus}, because it contradicts something which they have taken for granted as truth, but which the Holy Spirit teaches is not truth. Let all be very modest, and seek most earnestly to put self out of the question, and to exalt Jesus.} \textbf{In most of the religious controversies, the foundation of the trouble is that self is striving for the supremacy}. About what? About matters which are not vital points at all, and which are regarded as such only because men have given importance to them. See Matthew 12:31-37; Mark 14:56; Luke 5:21; Matthew 9:3.}[Lt38-1896.26; 1896][https://egwwritings.org/read?panels=p5631.32]


\egwnogap{\textbf{\underline{حتى الأدفنتست السبتيون معرضون لخطر إغلاق أعينهم عن الحق كما هو في يسوع}، لأنه يتعارض مع شيء اعتبروه مسلمًا به كحق، لكن الروح القدس يعلّم أنه ليس حقًا. ليكن الجميع متواضعين جدًا، وليسعوا بكل جدية لإبعاد الذات عن المسألة، وتعظيم يسوع.} \textbf{في معظم الخلافات الدينية، أساس المشكلة هو أن الذات تسعى للسيادة}. حول ماذا؟ حول أمور ليست نقاطًا حيوية على الإطلاق، واعتُبرت كذلك فقط لأن الناس أعطوها أهمية. انظر متى 12: 31-37؛ مرقس 14: 56؛ لوقا 5: 21؛ متى 9: 3.}[Lt38-1896.26; 1896][https://egwwritings.org/read?panels=p5631.32]


The proud state of the heart resists the will of God and is the driving force behind apostasy; the humble heart is obedient to the will of God and is the driving force behind true reformation. The following quotations express future, concrete prophecies where the fanciful ideas of God will be brought in and \egwinline{many things of like character will in the future arise}[Ms137-1903.10; 1903][https://egwwritings.org/read?panels=p9939.17]. These ideas are of like character to the ideas contained in the Living Temple. They will do away with the \emcap{personality of God}. Ellen White gives warning after warning to adhere to the \emcap{Fundamental Principles}, and to be aware of the leaders who will tear down the old foundation.


حالة القلب المتكبرة تقاوم إرادة الله وهي القوة الدافعة وراء الارتداد؛ القلب المتواضع مطيع لإرادة الله وهو القوة الدافعة وراء الإصلاح الحقيقي. الاقتباسات التالية تعبر عن نبوءات مستقبلية ملموسة حيث ستُدخل الأفكار الخيالية عن الله و\egwinline{ستظهر في المستقبل أمور كثيرة من نفس الطابع}[Ms137-1903.10; 1903][https://egwwritings.org/read?panels=p9939.17]. هذه الأفكار هي من نفس طابع الأفكار الواردة في كتاب ذا ليفينغ تمبل. ستقضي على \emcap{شخصانية الله}. تقدم إلين وايت تحذيرًا تلو الآخر للالتزام بـ\emcap{المبادئ الأساسية}، وللانتباه من القادة الذين سيهدمون الأساس القديم.


\egw{In view of these Scriptures, who will dare to interpret God and place in the minds of others the sentiments regarding Him that are contained in Living Temple? \textbf{These theories are the theories of the great deceiver, and in the lives of \underline{those who receive them there will be sad chapters}}. \textbf{This is Satan’s device \underline{to unsettle the foundation of our faith}, to shake our confidence in the Lord’s guidance and in the experience that He has given us. \underline{Many things of like character will in the future arise}}. I entreat our medical missionary workers to be afraid to trust the suppositions and devising of any human being who entertains the thought that \textbf{the path over which the people of God have been led for the last fifty years is a wrong path}. \textbf{\underline{Beware of those who}, not having had any decided experience in the leading of the Lord’s Spirit, \underline{would suppose that this leading is all a fallacy}; that we have not the truth}; that we are not the people of the Lord, gathered by Him from all countries and nations. \textbf{\underline{Beware of those who would tear down the foundation, upon which we have been building for the last fifty years, to establish a new doctrine}}. \textbf{I know that these new theories are from the enemy}.}[Ms137-1903.10; 1903][https://egwwritings.org/read?panels=p9939.17]


\egw{في ضوء هذه النصوص الكتابية، من سيجرؤ على تفسير الله ووضع في أذهان الآخرين الآراء المتعلقة به والواردة في كتاب ذا ليفينغ تمبل؟ \textbf{هذه النظريات هي نظريات المخادع العظيم، وفي حياة \underline{أولئك الذين يقبلونها ستكون هناك فصول حزينة}}. \textbf{هذه هي خطة الشيطان \underline{لزعزعة أساس إيماننا}، لزعزعة ثقتنا في إرشاد الرب وفي الخبرة التي أعطانا إياها. \underline{ستظهر في المستقبل أمور كثيرة من نفس الطابع}}}. أناشد العاملين في الإرسالية الطبية أن يخافوا من الثقة في افتراضات وتدبيرات أي إنسان يحمل فكرة أن \textbf{المسار الذي قاد الله شعبه فيه خلال الخمسين سنة الماضية هو مسار خاطئ}. \textbf{\underline{احذروا من أولئك الذين}، لم يكن لديهم أي خبرة حاسمة في قيادة روح الرب، \underline{يفترضون أن هذه القيادة هي مجرد وهم}؛ أننا لا نملك الحق؛ أننا لسنا شعب الرب، الذي جمعه من جميع البلدان والأمم. \textbf{\underline{احذروا من أولئك الذين سيهدمون الأساس، الذي كنا نبني عليه خلال الخمسين سنة الماضية، لتأسيس عقيدة جديدة}}. \textbf{أعلم أن هذه النظريات الجديدة هي من العدو}.}[Ms137-1903.10; 1903][https://egwwritings.org/read?panels=p9939.17]


\egwnogap{\textbf{Let those who would \underline{bring in} fanciful ideas of God awake to a sense of their danger. This is too solemn a subject to be trifled with}.}[Ms137-1903.11; 1903][https://egwwritings.org/read?panels=p9939.18]


\egwnogap{\textbf{دع أولئك الذين \underline{يريدون إدخال} أفكار خيالية عن الله يستيقظون ليدركوا خطرهم. هذا موضوع جاد جدًا لا يمكن العبث به}.}[Ms137-1903.11; 1903][https://egwwritings.org/read?panels=p9939.18]


\egwnogap{The root of idolatry is an evil heart of unbelief in departing from the living God. It is because men have not faith in the presence and power of God \textbf{that they have been putting their trust in their own wisdom}. They have been devising and planning to exalt themselves and find salvation in their own works. \textbf{\underline{A deceptive influence from satanic agencies is coming in}, because leaders whom the Lord has warned and entreated and counseled are choosing their own wisdom in the place of the wisdom of God}. To such ones the warning comes, ‘Talk no more exceedingly proudly; let not arrogancy come out of your mouth; for the Lord is a God of knowledge, and by Him actions are weighed.’}[Ms137-1903.12; 1903][https://egwwritings.org/read?panels=p9939.19]


\egwnogap{جذر عبادة الأوثان هو قلب شرير من عدم الإيمان في الابتعاد عن الله الحي. إنه لأن الناس ليس لديهم إيمان في حضور وقوة الله \textbf{أنهم كانوا يضعون ثقتهم في حكمتهم الخاصة}. لقد كانوا يخططون ويدبرون لتمجيد أنفسهم وإيجاد الخلاص في أعمالهم الخاصة. \textbf{\underline{تأثير خادع من وكالات شيطانية آخذ في الدخول}، لأن القادة الذين حذرهم الرب وتوسل إليهم ونصحهم يختارون حكمتهم الخاصة بدلاً من حكمة الله}. إلى مثل هؤلاء يأتي التحذير: ‘لا تكثروا الكلام العالي المستعلي، ولا تخرج وقاحة من أفواهكم، لأن الرب إله عليم، وبه توزن الأعمال.’}[Ms137-1903.12; 1903][https://egwwritings.org/read?panels=p9939.19]


The difference between the old \emcap{Fundamental Principles} and the new Fundamental Beliefs is in our \egwinline{ideas of God.} The Trinitarian idea of God was not part of the foundation of our faith, which Sister White defended. How did this change take place? It was done through the leaders who chose \egwinline{their own wisdom in the place of the wisdom of God.} We should \egwinline{Beware of those who would tear down the foundation, upon which we have been building for the last fifty years, to establish a new doctrine.} In this observation, we recognize that this new Trinitarian idea of God was \egwinline{a deceptive influence from satanic agency} that came into our ranks.


الفرق بين \emcap{المبادئ الجوهرية} القديمة والمعتقدات الأساسية الجديدة يكمن في \egwinline{أفكارنا عن الله.} إن الفكرة الثالوثية عن الله لم تكن جزءًا من أساس إيماننا، الذي دافعت عنه الأخت وايت. كيف حدث هذا التغيير؟ تم ذلك من خلال القادة الذين اختاروا \egwinline{حكمتهم الخاصة بدلاً من حكمة الله.} يجب علينا \egwinline{الحذر من أولئك الذين يريدون هدم الأساس، الذي كنا نبني عليه خلال الخمسين سنة الماضية، لتأسيس عقيدة جديدة.} في هذه الملاحظة، ندرك أن هذه الفكرة الثالوثية الجديدة عن الله كانت \egwinline{تأثيرًا خادعًا من وكالة شيطانية} دخلت إلى صفوفنا.


% Setting up the wrong Fundamental Principles

\begin{titledpoem}
    
    \stanza{
        Sadly, within our own church walls \\
        From our own pulpits, error falls \\
        Members want smooth words for their ears \\
        Don’t step on toes, Allay their fears.
    }

    \stanza{
        Pastors and elders do preside \\
        While sins remain, untouched inside \\
        Laodicean comfort zone \\
        But they will reap what they have sown.
    }

    \stanza{
        Ask for the old paths, walk therein \\
        From the old truth, don’t move a pin. \\
        They spurned the truth which brightly shone, \\
        The Spirit’s pow’r, to them unknown.
    }

    \stanza{
        Do not the humble hearts perceive \\
        Whispers of truth they should believe? \\
        Meanwhile the stories ease concern. \\
        What God would tell them, they won’t learn.
    }

    \stanza{
        Beware of error, thinly veiled, \\
        God’s Word is true, but men have failed. \\
        Beware of shadows leaders cast. \\
        To the foundations true, hold fast,
    }

    \stanza{
        Let not man’s wisdom lead astray, \\
        Let God’s own Spirit show the way. \\
        For in the Scripture’s glowing light, \\
        We find the path of safety bright.
    }

    \stanza{
        Let us, in faith, each day commence, \\
        God’s Word our shield, not man’s pretense. \\
        For truth in Christ alone is found, \\
        And on this rock, our faith is sound.
    }
    
\end{titledpoem}