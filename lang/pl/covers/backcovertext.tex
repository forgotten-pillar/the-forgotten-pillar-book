Jeśli identyfikujesz się jako \textit{adwentysta dnia siódmego} i zależy Ci na swoim Kościele, prędzej czy później, jeśli jeszcze tego nie zrobiłeś, znajdziesz siebie i swój Kościół w kryzysie tożsamości. Istnieje wiele grup ludzi identyfikujących się jako \textit{adwentystami dnia siódmego}, ale mają oni odmienne poglądy na nasze najważniejsze doktryny, takie jak doktryna o Bogu, \textit{służbie świątynnej}, \textit{sądzie śledczym} itd. Celem tej książki jest zakorzenienie \textit{adwentystów dnia siódmego w ich pierwotnej tożsamości} — tej tożsamości, którą Bóg potężnie ustanowił, powołując \textit{Kościół Adwentystów Dnia Siódmego} na początku. Już wcześniej mieliśmy kryzysy tożsamości; wielki kryzys miał miejsce za czasów Ellen White. Był to \textit{kryzys Kellogga}. W odpowiedzi na ten kryzys Bóg dał rozwiązanie każdego kolejnego kryzysu, z jakim mielibyśmy się zmierzyć w przyszłości, poprzez pisma Ellen White. Rozwiązaniem jest powrót do prawd, które otrzymaliśmy na początku naszego ruchu. Te prawdy nazywane są \textit{filarami naszej wiary}. Jeśli zapomnimy którykolwiek filar, kryzys jest nieunikniony — a już w nim jesteśmy. Zapomnieliśmy o jednym kluczowym filarze. Ten filar nazywa się \textit{„osobowość Boga”}. Jest to odpowiedź na pytanie: Czy Bóg jest osobą, i jaka cecha charakteryzuje Boga jako osobę?

Dziś bardzo często odpowiadamy na to pytanie inaczej niż nasi pionierzy. To pytanie dotyczy różnych poglądów na Boga i oferuje różne odpowiedzi na orędzie \textit{pierwszego anioła} z czternastego rozdziału Objawienia. Ale nikt nie powinien pozostawać w ciemności w tej sprawie, ponieważ Bóg dał obfite światło na początku naszego ruchu i ponownie zatwierdził je jako odpowiedź na \textit{kryzys Kellogga}. Duch Proroctwa mówi nam, że Bóg ponownie zatwierdzi to przesłanie w naszych czasach. Odkryj te odpowiedzi i na nowo odkryj \textit{prawdziwą adwentystyczną tożsamość}.