\begin{titledpoem}

    \stanza{
        Podstawy wiary tak przejrzyste \\
        I słowa Ellen wyraziste \\
        John Harvey w książce swej przytaczał \\
        I ich znaczenie przeinaczał.
    }

    \stanza{
        Wróg — droga Ellen ostrzegała — \\
        Wypaczy prawdę, by ściemniała. \\
        Teorie groźne i przebrane \\
        W płaszczyku Biblii są podane.
    }

    \stanza{
        „Fałsz!” — przeciw fali tak krzyczała, \\
        Błędne stwierdzenia piętnowała. \\
        Wyraźna była wizja z nieba, \\
        Że fundamentów bronić trzeba.
    }

    \stanza{
        Niech joty, kreski nie zmieniają \\
        W filarach, co od dawna trwają. \\
        Niech zaprzestaną tej „reformy” \\
        Przez Pana wzniesionej platformy.
    }

    \stanza{
        Gdyż błędna była myśl Kellogga \\
        Co do wyobrażenia Boga. \\
        Osobie Boga trójca szkodzi, \\
        Choć Kellogg śmiało jej dowodzi.
    }

    \stanza{
        Lecz Ellen nic się nie ugięła, \\
        W obronę prawdę dawną wzięła. \\
        Mimo za Trójcą głosów mnóstwa \\
        Trzymała się wciąż prawdy Bóstwa.
    }

    \stanza{
        Więc dziś się też nie uginajmy, \\
        Przy dawnej wierze obstawajmy, \\
        Bo prawda dana z objawienia \\
        Na wieki wieków się nie zmienia.
    }

    \stanza{
        Nie zapomnijmy drogi Boga, \\
        Niech na niej stanie nasza noga, \\
        A wiatr niewiary nie odbierze, \\
        Czego jej pismo wiernie strzeże.
    }

\end{titledpoem}
