% Rzeczywistość nieba

\begin{titledpoem}
    \stanza{
        Bóg nie jest tajemnicą, lecz Istotą prawdziwą, \\
        Na tronie nieba siedzi z chwałą niewątpliwą. \\
        Choć w jednym miejscu przebywa jako Osoba, \\
        Przez Ducha jest obecny, gdzie tylko potrzeba.
    }

    \stanza{
        Duchowa Istota z ciałem i kształtem, \\
        Nie bezcielesny duch, lecz Byt z majestatem. \\
        Syn jest Jego obrazem, tę samą ma naturę, \\
        Obaj są namacalni, to prawda, nie złudzenie.
    }

    \stanza{
        Aniołowie, jak Bóg, są istotami realnymi, \\
        Choć dla oczu śmiertelnych pozostają niewidzialnymi. \\
        Mają ciała duchowe, lecz materialne zarazem, \\
        Takie, jakie my otrzymamy zmartwychwstania obrazem.
    }
\end{titledpoem}

\begin{titledpoem}
    \stanza{
        Niebiańska rzeczywistość nie jest mgłą tajemną, \\
        Lecz światem namacalnym z chwałą niepojemną. \\
        Bóg jako Osoba ma ciało i części, \\
        W jednym miejscu przebywa, lecz Duchem się mieści.
    }

    \stanza{
        Pionierzy adwentyzmu prawdę tę głosili, \\
        Że Bóg i Syn są Istotami, nie duchem bez siły. \\
        Duch Święty to przedstawiciel, nie trzecia Istota, \\
        To przez Niego Bóg działa, gdy pomoc jest potrzebna.
    }

    \stanza{
        Gdy zmartwychwstaniemy w ciałach duchowych, \\
        Będziemy jak aniołowie w kształtach gotowych. \\
        Materialni, namacalni, choć duchowej natury, \\
        Zamieszkamy na ziemi nowej, bez śmierci ponurej.
    }
\end{titledpoem}

\begin{titledpoem}
    \stanza{
        Bóg jest Istotą, nie abstrakcyjną mocą, \\
        Ma kształt i formę, nie jest bezcielesną nocą. \\
        Osobowość Jego to stan bycia Osobą, \\
        Ograniczoną miejscem, lecz z Ducha swobodą.
    }

    \stanza{
        Oczy Balaama ujrzały anioła dopiero, \\
        Gdy Pan je otworzył na prawdę i szczerość. \\
        Tak samo my ujrzymy niebiańskie istoty, \\
        Gdy Bóg nam pozwoli przeniknąć zasłony.
    }

    \stanza{
        Ciało duchowe to nie duch bez ciała, \\
        Lecz byt materialny, którego moc trwała. \\
        W takich ciałach będziemy na nowej ziemi żyli, \\
        Namacalnej, realnej, gdzie Boga będziemy czcili.
    }
\end{titledpoem}