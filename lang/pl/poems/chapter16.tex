\begin{titledpoem}

    \stanza{
        Panteizm to jest myśl zdradliwa, \\
        W przyrodzie Boga wciąż ukrywa. \\
        Czy z wiatru, ziemi, toni, drzewa \\
        Obecność Jego się wylewa.
    }

    \stanza{
        I choć się z częścią można zgodzić, \\
        Trzeba od prawdy fałsz odgrodzić. \\
        Od całej Trójcy — wierz przestrodze — \\
        Do panteizmu jest po drodze.
    }

    \stanza{
        Osobę Boga wychwalajmy \\
        I poza to nie wykraczajmy. \\
        Bóg bowiem większy od natury, \\
        Z jednego miejsca patrzy z góry.
    }

    \stanza{
        Zwieść ludzi chciał ten kurs doktora \\
        I trójcy go dręczyła zmora. \\
        Miał wypaczony obraz Boga, \\
        Co ściągał go na ścieżkę wroga.
    }

    \stanza{
        Choć siła w przyrodzie przebywa, \\
        To Boga natura nie skrywa. \\
        U Jego boku stoi zaś Syn, \\
        A Duch ożywia w nas każdy czyn.
    }

    \stanza{
        W przyrodzie widać palec Boży, \\
        Choć jest poza tym, co sam stworzy. \\
        Bóg to miłości, skarb bez miary, \\
        W Nim pokój wolny jest od kary.
    }

\end{titledpoem}
