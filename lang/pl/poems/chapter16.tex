% Dr. Kellogg i panteizm

\begin{titledpoem}

    \stanza{
        W naturze Bóg swą moc objawia, \\
        Życie każdej istocie nadawia. \\
        Kellogg tę prawdę dobrze znał, \\
        Lecz w błędnym kierunku podążał.
    }

    \stanza{
        Bóg nie jest naturą, choć w niej działa, \\
        Jego osobowość jest doskonała. \\
        Ma kształt i formę, na tronie zasiada, \\
        Przez Ducha Świętego wszędzie włada.
    }

    \stanza{
        Trynitarne poglądy Kellogga zwiodły, \\
        Do panteizmu go doprowadziły. \\
        Choć w naturze Boża moc się przejawia, \\
        Bóg osobowy w niebie przebywa.
    }

\end{titledpoem}

\begin{titledpoem}

    \stanza{
        Serce nie bije własnym ruchem, \\
        Bóg je podtrzymuje swoim duchem. \\
        Lecz nie znaczy to, że Bóg jest wszystkim, \\
        On jest bytem osobowym, nie mistycznym.
    }

    \stanza{
        W przyrodzie widzimy Bożą pieczęć, \\
        Jego mocy i mądrości potęgę. \\
        Jednak Bóg sam ma osobowość jasną, \\
        Nie jest energią bezkształtną, mglistą.
    }

    \stanza{
        Kellogg prawdę z błędem pomieszał, \\
        Gdy o Bogu w naturze pisał. \\
        Bóg jest Stwórcą, nie stworzeniem, \\
        Osobą, nie tylko natchnieniem.
    }

\end{titledpoem}

\begin{titledpoem}

    \stanza{
        Tajemnicze życie przenika naturę, \\
        Bóg podtrzymuje każdą strukturę. \\
        Lecz nie jest On z naturą tożsamy, \\
        Choć Jego obecność wszędzie spotykamy.
    }

    \stanza{
        Osobowość Boga to nie abstrakcja, \\
        To realna, konkretna manifestacja. \\
        Ma formę, kształt i miejsce w niebie, \\
        Przez Ducha działa, lecz nie jest w glebie.
    }

    \stanza{
        Panteizm z prawdą się mija, \\
        Gdy Boga w naturze ukrywa. \\
        Bóg jest ponad swoim stworzeniem, \\
        Osobowym, wiecznym istnieniem.
    }
    
\end{titledpoem}