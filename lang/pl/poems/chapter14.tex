% Pionierzy adwentyzmu i doktryna o Trójcy

\begin{titledpoem}
    \stanza{
        Pionierzy adwentyzmu prawdę głosili śmiało, \\
        O Bogu i Jego Synu świadectwo trwało. \\
        Odrzucili doktrynę, co Bożą osobowość niszczy, \\
        Trzymając się Pisma w wierze najczystszej.
    }

    \stanza{
        James White nazwał Trójcę "popularną baśnią" wprost, \\
        Która osobowość Boga i Syna usuwa w tłum. \\
        Ojciec i Syn to dwie odrębne istoty, \\
        Nie złączone w jedną, jak uczą trynitarne cnoty.
    }

    \stanza{
        Jedność ich w duchu, celu i działaniu trwa, \\
        Jak uczniów z Chrystusem, choć każdy siebie ma. \\
        Nie w osobie są jednym, lecz w zamiarze swym, \\
        Tak Bóg i Chrystus jednością są w dziele tym.
    }

    \stanza{
        Chrystus w pełni boski, bo z Ojca zrodzony, \\
        Nie przez stworzenie, lecz jako Syn uwielbiony. \\
        Na obraz Ojca dokładny ukształtowany, \\
        W boskiej naturze w pełni uczestniczy oddany.
    }
\end{titledpoem}

% Druga poezja o Trójcy i pionierach adwentyzmu

\begin{titledpoem}
    \stanza{
        Loughborough pytany o Trójcy zastrzeżenia, \\
        Trzy powody podał godne rozważenia. \\
        Sprzeczna ze zdrowym rozsądkiem i z Pismem całym, \\
        Z pogańskich źródeł pochodzi z blaskiem niemałym.
    }

    \stanza{
        Jak może Bóg być trzema i jednym zarazem? \\
        Jak może modlić się do siebie własnym głosem? \\
        Jan siedemnasty rozdział tę doktrynę obala, \\
        Gdy Chrystus o Ojcu jako odrębnym wspomina.
    }

    \stanza{
        Pionierzy widzieli jasno, co Pismo objawia, \\
        Że Ojciec jest Bogiem, a Chrystus Synem zostaje. \\
        Nie przez tajemnicę, lecz przez proste słowa, \\
        Prawda o Bogu w ich pismach się chowa.
    }

    \stanza{
        Ellen White z mężem w jedności trwali, \\
        Tę samą prawdę o Bogu wyznawali. \\
        Nie zmieniła poglądów, jak niektórzy twierdzą, \\
        Lecz wierną pozostała prawdzie z Bożą pieczęcią.
    }
\end{titledpoem}

% Trzecia poezja o osobowości Boga

\begin{titledpoem}
    \stanza{
        Osobowość Boga to filar wiary prawdziwej, \\
        Nie tajemnica w formule trynitarnej fałszywej. \\
        Bóg jest osobą realną, określoną jasno, \\
        Nie abstrakcją, co w trójcy ginie przedwcześnie.
    }

    \stanza{
        Ojciec i Syn w miłości zjednoczeni, \\
        Przez Ducha Świętego w sercach objawieni. \\
        Nie trzy osoby w jednej istocie złączone, \\
        Lecz Ojciec najwyższy i Syn z Niego zrodzony.
    }

    \stanza{
        Pionierzy tę prawdę z Pisma czerpali, \\
        Przed błędem trynitarnym wiernych ostrzegali. \\
        Ich świadectwo jasne jak światło poranka, \\
        Prowadzi do Boga, nie do ludzkiego zamku.
    }

    \stanza{
        Niech prawda o Bogu w sercach rozbrzmiewa, \\
        Jak u pionierów, co wiernie ją wyznawali. \\
        Bez wybiegów, domysłów i ludzkich tradycji, \\
        Lecz w prostocie wiary, co duszę oczyści.
    }
\end{titledpoem}