\begin{titledpoem}

    \stanza{
        Pionierzy byli bardzo szczerzy, \\
        Że żaden z nich w trójcę nie wierzy. \\
        Doktryna ta przebrana była, \\
        Aby się w oczy nie rzuciła.
    }

    \stanza{
        James Springer White wręcz bajką mienił \\
        Ten błąd, co się tak rozprzestrzenił. \\
        Nie można przecież wciąż nalegać, \\
        By wiary w trzech bogów przestrzegać.
    }

    \stanza{
        Osoby dwie — niech każdy słucha — \\
        Cel jeden Ich, jednego ducha. \\
        Tak, jak się ludzie jedno stają, \\
        Tak samo Syn i Ojciec trwają.
    }

    \stanza{
        Dwaj boscy, jednak oddzieleni, \\
        Trójcą też nie są określeni. \\
        Duch posłan od Boga w niebiesiech \\
        Na ziemię Bożą miłość niesie.
    }

    \stanza{
        Nicejskie credo napisane, \\
        A punkty w nim Pismu nieznane. \\
        Nie przyszło z Bożego natchnienia, \\
        Lecz z człowieczego urojenia.
    }

    \stanza{
        Jan w siedemnastym swym rozdziale \\
        Mówi o Bogu w pełnej chwale. \\
        Życie, co da zbawienie wielu, \\
        W Jednorodzonym Zbawicielu.
    }

    \stanza{
        Chrystus jest boski, to Syn Boga, \\
        Dajmy Mu cześć, nim przyjdzie trwoga. \\
        Odbicie Bożego oblicza, \\
        Po Ojcu pełnię odziedzicza.
    }

    \stanza{
        Pionierzy znali Boże Słowo \\
        I mieli odpowiedź gotową. \\
        Duch Boży zstąpił z wysokości, \\
        By przynieść Boga nam w bliskości.
    }

    \stanza{
        Prawda przez błędu noc widnieje, \\
        Przez siostrę White Bóg słowo sieje. \\
        Uszli pionierzy tajemnicy, \\
        Uczeni z Bożych Pism skarbnicy.
    }

\end{titledpoem}
