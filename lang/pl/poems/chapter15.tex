\begin{titledpoem}

    \stanza{
        Wyzwaniem było dla Kellogga \\
        Naturę zdefiniować Boga. \\
        I z czego Duch się Jego składa? \\
        To tajemnica jest nie lada.
    }

    \stanza{
        Było to głębsze, niż co widać, \\
        Więc trójca mu się może przydać. \\
        Czy Ojcem przestrzeń pełna cała? \\
        Bez twarzy lub nawet bez ciała?
    }

    \stanza{
        Jezus poinformował Ellen \\
        O swoim niczym Ojca ciele. \\
        „Obrazem jestem wiernym Jego, \\
        Więc swoją postać mam dlatego”.
    }

    \stanza{
        Przez wizję prawda objawiana, \\
        Natchnieniem przypieczętowana. \\
        I chociaż Syn to byt osobny, \\
        Do Ojca swego jest podobny.
    }

    \stanza{
        A co do Ducha właściwości, \\
        Jest w swojej też osobowości. \\
        Ma własne miejsce, w nas przebywa, \\
        Myśl Chrystusową nam odkrywa.
    }

    \stanza{
        Moc Boga tam się pokazuje, \\
        Gdzie swą obecność dziś kieruje. \\
        I tak już wszędzie się pojawia, \\
        Przez Duch swego się objawia.
    }

    \stanza{
        „Żywa Świątynia” wadę miała, \\
        Bo prawdę błędem zakrywała. \\
        I krnąbrne teorie głosiła, \\
        O których Ellen wprost mówiła.
    }

    \stanza{
        Opuścił Kellogg prawdy tory \\
        Do poszukiwań swoich skory. \\
        Gdy zostałby przy Bożym planie, \\
        Porzuciłby swe nauczanie.
    }

\end{titledpoem}
