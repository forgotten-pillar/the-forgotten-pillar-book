% Dr. Kellogg i doktryna o Trójcy

\begin{titledpoem}

    \stanza{
        W poszukiwaniach Kellogga, pytanie się rodzi, \\
        "Czy Duch Święty osobą jest, co w świecie chodzi?" \\
        Kontrowersja powstała, istota debaty, \\
        Tajemnica Trójcy, w spór bogaty.
    }

    \stanza{
        Kellogg kwestionował to, co namacalne, widzialne, \\
        Gdzie formy Ducha Świętego nie są dostrzegalne. \\
        Duchowa esencja, przestrzenią nieograniczona, \\
        Kwestionująca fizyczną postać Ojca, uświęcona.
    }

    \stanza{
        Duch Święty, osoba, lecz nie w formie cielesnej, \\
        W działaniach, emocjach, w normie boskiej, niebiańskiej. \\
        Świadczący, uczący, decyzje podejmujący, \\
        Obecność odczuwalna, choć niewidzialna, prowadząca.
    }

    \stanza{
        Butler kontrastował, w fizycznym sensie, \\
        Ojca i Chrystusa, ich obecność w immensie. \\
        Lecz Ducha osobowość, odrębna w swej roli, \\
        Duchowa manifestacja, dopełniająca w całości.
    }

\end{titledpoem}

% Osobowość Boga i Ducha Świętego

\begin{titledpoem}

    \stanza{
        Ellen pytała Jezusa, czy Ojciec formę jak On posiada, \\
        "Na obraz mego Ojca", odpowiedział, prawda to nie błaha. \\
        Lecz Duch, w istocie, światłem przewodnim jest, \\
        Niewidzialny, lecz odczuwalny, w wierzących sercach gest.
    }

    \stanza{
        Doktryna wyłania się, Trójcy rdzeń, \\
        Równi w osobowościach, lecz to więcej niż cień. \\
        Perspektywa Kellogga, niegdyś zbłąkana, \\
        Pyta teologicznie, prawda czy ułuda nadana?
    }

    \stanza{
        Pismo prowadzi, przez tajemnicy zasłonę, \\
        Objawia Bożą formę, gdzie ludzkie poglądy są stłumione. \\
        Ojciec, ucieleśniony, prawda, którą przyjmujemy, \\
        Podczas gdy Ducha obecność, bez formy, odczuwamy.
    }

    \stanza{
        W boskim objawieniu, odpowiedzi znalezione, \\
        Osobowość Boga, w Piśmie, głęboko zakorzenione. \\
        Ojciec, w formie; Duch, bez niej istnieje, \\
        W tym Biblia rozwiewa wszelkie wątpliwości i nadzieje.
    }

\end{titledpoem}

% Kontrowersja Kellogga

\begin{titledpoem}
    
    \stanza{
        Kellogg w swej książce, "The Living Temple" nazwanej, \\
        Prawdy z błędami splótł, w teorii zawiłej, nieznanej. \\
        O osobowości Boga, spekulacje prowadził, \\
        Od fundamentów wiary, krok po kroku odchodził.
    }

    \stanza{
        Duch Święty, trzecią osobą Bóstwa nazwany, \\
        W definicji słowa "osoba", spór był rozwijany. \\
        Butler twierdził, że Duch nie jest jak Ojciec i Syn, \\
        Nie chodzi pieszo, nie lata, to inny byt, inny czyn.
    }

    \stanza{
        Ellen White ostrzegała przed zwodniczymi teoriami, \\
        Które prawdę z fałszem mieszają, z błędnymi myślami. \\
        Osobowość Boga, miejsce Jego obecności, \\
        To fundamenty wiary, nie przedmiot wątpliwości.
    }

    \stanza{
        W boskim objawieniu, prawda jest objawiona, \\
        Bóg ma formę, osobowość, nie jest rozproszona. \\
        Duch Święty świadczy, bada tajemnice Boże, \\
        Lecz w inny sposób niż Ojciec i Syn, to zrozumieć może.
    }
    
\end{titledpoem}