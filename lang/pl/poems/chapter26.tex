\begin{titledpoem}

    \stanza{
        Dziś na filary padły cienie, \\
        A prawda idzie w zapomnienie. \\
        Przez kroków pięć w ciszy prowadzą, \\
        Aż rozłam wśród ludu wprowadzą.
    }

    \stanza{
        Fałsz kwitnie tam, gdzie prawda stała, \\
        Którą zrozumieć trzódka miała. \\
        A to, co fundamentem było, \\
        W wyznanie wiary się zmieniło.
    }

    \stanza{
        Prorocze słowa porzucone, \\
        Pionierzy też, pieśni zmienione. \\
        Świadectwa kiedyś tak dźwięczały, \\
        Lecz teraz respekt do nich mały.
    }

    \stanza{
        „Bóg jest osobą” odrzucono, \\
        Jego istotę znieważono. \\
        Prawdziwy filar zapomniany, \\
        A nowy, błędny jest nam dany.
    }

    \stanza{
        Uczeni przekręcają słowa, \\
        By zmieniła znaczenie mowa. \\
        Nikt nie chce widzieć Bożej twarzy, \\
        O Jego uścisku nie marzy.
    }

    \stanza{
        Kellogga kryzys przekręcany, \\
        Krok alfa nie jest zrozumiany. \\
        A że się prawdzie nie dowierza, \\
        To Kościół ku omedze zmierza.
    }

    \stanza{
        Wśród wiernych zamieszanie rządzi \\
        I wielu dziś w wierzeniach błądzi. \\
        Historię naszą przepisano, \\
        Miała być prawda, lecz skłamano.
    }

\end{titledpoem}
