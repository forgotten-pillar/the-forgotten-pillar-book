% The Historical Context

\begin{titledpoem}
    \stanza{
        W wizjach Ellen White prawdę ujrzała, \\
        Przed błędnymi poglądami ostrzegała. \\
        Że Bóg ma postać, jak Syn Jego, \\
        To fundament nauczania adwentowego.
    }

    \stanza{
        Spirytualiści chcieli zniszczyć wiarę, \\
        Twierdząc, że niebo to pojęcie stare. \\
        Lecz Bóg jest osobą, nie duchem wszędzie, \\
        Ta prawda w sercu wiary zawsze będzie.
    }

    \stanza{
        Na tronie w niebie Bóg zasiada, \\
        Jego obecność nie jest tylko metaforą blada. \\
        Dlaczego ta nauka dziś zanika? \\
        Czy z fundamentów naszych już umyka?
    }
\end{titledpoem}

\begin{titledpoem}
    \stanza{
        Młoda Ellen przez Maine wędrowała, \\
        Przeciw duchowym błędom występowała. \\
        Bóg ma formę, jak Chrystus objawił, \\
        Ten fakt w wizji Pan jej przedstawił.
    }

    \stanza{
        Osobowość Boga to nie abstrakcja pusta, \\
        To prawda, którą głosiły pionierów usta. \\
        W niebie świątynia prawdziwa istnieje, \\
        Gdzie Chrystus służbę kapłańską prowadzi z nadzieją.
    }

    \stanza{
        Alfa błędu w "Żywej Świątyni" się zjawiła, \\
        Omega wkrótce potem nastąpiła. \\
        Czy doktryna ta, niegdyś tak jasna, \\
        W naszych wierzeniach dziś już zgasła?
    }
\end{titledpoem}

\begin{titledpoem}
    \stanza{
        "Czy Ojciec Twój ma postać jak Ty?" \\
        Pytała Ellen w wizji pełnej czci. \\
        "Jestem obrazem Jego osoby" - \\
        Odpowiedział Jezus bez żadnej ozdoby.
    }

    \stanza{
        Bóg nie jest duchem wszechobecnym sam w sobie, \\
        Lecz przez Ducha Świętego działa w każdej dobie. \\
        Ta prawda broniła przed fałszem wierzących, \\
        Przed spirytualizmem umysły zwodzących.
    }

    \stanza{
        Fundamentalna zasada pierwsza głosiła, \\
        Że Bóg to osoba, nie siła czy bryła. \\
        Czy ta doktryna, tak ważna dla wiary, \\
        Zniknęła z naszych wyznań bez miary?
    }
\end{titledpoem}