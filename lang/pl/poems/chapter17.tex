%Odpowiedź na trynitarne poglądy Kellogga

% Wiersz pierwszy

\begin{titledpoem}

    \stanza{
        W kryzysie Kellogga prawda jaśnieje, \\
        Gdy o osobowość Boga chodzi, nadzieja tleje. \\
        Nie tylko panteizm był tam problemem, \\
        Lecz Trójcy doktryna z jej dylematem.
    }

    \stanza{
        Ellen White pisała o Bogu i Synu, \\
        O ich jedności, lecz nie w osoby czynie. \\
        W Ewangelii Jana, rozdział siedemnasty, \\
        Dowód na to jest jasny i wyrazisty.
    }

    \stanza{
        "Jedyny prawdziwy Bóg" - tak Jezus mówił, \\
        Osobowość Ojca i Syna tym słowem objawił. \\
        Pionierzy adwentyzmu tę prawdę głosili, \\
        Na fundamencie wiecznej prawdy się osadzili.
    }

\end{titledpoem}

% Wiersz drugi

\begin{titledpoem}

    \stanza{
        Osobowość Boga - prawda fundamentalna, \\
        W pismach Ellen White zawsze aktualna. \\
        Nie w Trójcy jedność, lecz w celu i myśli, \\
        Tak Ojciec i Syn są jedno, choć osobni.
    }

    \stanza{
        Loughborough pisał z przekonaniem szczerym, \\
        Że Jan siedemnasty Trójcę obala z impetem. \\
        Zasady Fundamentalne to potwierdzają, \\
        Prawdę o Bogu jasno przedstawiają.
    }

    \stanza{
        Odrzućmy błąd, choć pozorem prawdy okryty, \\
        Który osobowość Boga czyni rozmytą. \\
        Stańmy na platformie wiecznej prawdy mocno, \\
        By wiara nasza świeciła jasno i owocnie.
    }
\end{titledpoem}

% Wiersz trzeci

\begin{titledpoem}
    \stanza{
        W sporze z Kelloggiem nie tylko panteizm brzmiał, \\
        Lecz trynitarny błąd się tam również wkradał. \\
        Ellen White do Jana siedemnastego się odwołała, \\
        Gdzie prawda o Bogu wyraźnie została ukazana.
    }

    \stanza{
        "Ojciec i Syn są jedno" - lecz nie w osobie, \\
        W celu, charakterze, w miłości ku sobie. \\
        Jak uczniowie z Chrystusem mogą być złączeni, \\
        Tak Ojciec z Synem, choć nie są połączeni.
    }

    \stanza{
        Prawda, którą na początku otrzymaliśmy, \\
        O osobowości Boga, którą głosiliśmy. \\
        Niech będzie dla nas kotwicą wiary, \\
        By nie ulegać błędom nowej miary.
    }
    
\end{titledpoem}