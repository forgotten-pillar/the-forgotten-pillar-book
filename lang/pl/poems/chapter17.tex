\begin{titledpoem}

    \stanza{
        To światło prawdy, tak przejrzyste, \\
        W kryzysie już nieoczywiste. \\
        Panteizm w szeregi się wkrada, \\
        A prawda o Bogu upada.
    }

    \stanza{
        Lecz Ellen White w swojej twórczości \\
        Pisała o osobowości. \\
        O Trójcy nigdy nie mówiła \\
        I się z Kellogiem nie zgodziła.
    }

    \stanza{
        Trafiła za to w samo sedno: \\
        „Ojciec i Syn stanowią jedno”. \\
        O jedność celu tutaj chodzi, \\
        Jak w Ewangelii Jan dowodzi.
    }

    \stanza{
        Za Ellen White pionierzy stali, \\
        Osobę Boga wyznawali. \\
        Loughborough głosił przeciw triadzie \\
        W spisanym w „Review” swym wykładzie.
    }

    \stanza{
        Te Punkty Wiary, skarb nasz czysty, \\
        Wskazują w sposób tak oczywisty, \\
        Że błąd się kryje w Trójcy dogmacie, \\
        Lecz znając Boga, się wyzwalacie.
    }

    \stanza{
        Odrzućmy zatem błędy zawiłe, \\
        Wybierzmy Boga, a da nam siłę. \\
        Bóg jest osobą, ma kształt i formę — \\
        Na tym oprzyjmy naszą platformę.
    }

\end{titledpoem}
