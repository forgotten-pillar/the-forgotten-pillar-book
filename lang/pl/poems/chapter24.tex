\begin{titledpoem}

    \stanza{
        Proroctwa wczesny jeszcze głos \\
        Przewidział, jaki będzie los. \\
        „Diabeł ma spisek nam przeciwny, \\
        By wykraść prawdy w sposób dziwny”.
    }

    \stanza{
        Zasady, które niegdyś były, \\
        Które od Boga pochodziły, \\
        Na piasku teraz się ruszają, \\
        Bo ludzie Boga nie słuchają.
    }

    \stanza{
        Idee przebrane za światło, \\
        Usunąć filary tak łatwo. \\
        Książki też były przepisane, \\
        Nowe teorie nauczane.
    }

    \stanza{
        Czy szabat też mamy porzucić \\
        I tak od Boga się odwrócić? \\
        „Podstawy runą” — tak się zdaje, \\
        Gdy duma w miejsce prawdy staje.
    }

    \stanza{
        Wspomnijcie, jak żeśmy zaczęli, \\
        Gdy żeśmy prawdę w sercach mieli. \\
        Lecz poprzez podstępy szatana \\
        Teraz już prawda jest niechciana.
    }

    \stanza{
        Powróćmy zatem do przeszłości, \\
        Do wczesnej wiary i wierności, \\
        Bo miał zapewnić głos proroczy, \\
        Że z dobrej ścieżki nikt nie zboczy.
    }

    \stanza{
        Przetrwajmy burzę, która trwa, \\
        Gdyż warta tego prawda ta. \\
        Głosowi Ellen więc uwierzmy \\
        I prawdę drogą tak wybierzmy.
    }

    \stanza{
        A zanim przyjdzie na świat trwoga, \\
        Ustawmy się po stronie Boga. \\
        Ze starych ścieżek nie zbaczajmy \\
        I zepchnąć z nich się już nie dajmy.
    }

\end{titledpoem}
