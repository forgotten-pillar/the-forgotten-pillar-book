% Korekta „The Living Temple"

\begin{titledpoem}
    \stanza{
        Kellogga księga kryła w sobie błąd, \\
        O Bożej naturze fałszywy sąd. \\
        Osobowość Boga źle pojmowana, \\
        Przez White jako pułapka rozpoznana.
    }

    \stanza{
        Nie wystarczy zmienić kilka wyrażeń, \\
        By naprawić księgę pełną złych wrażeń. \\
        Trójca nie rozwiąże głównego problemu, \\
        Gdy Bożą osobowość pojmiesz po swojemu.
    }

    \stanza{
        Fundament naszej wiary musi trwać, \\
        Nie można go na nowo interpretować. \\
        Bóg przez White ostrzegał przed ostatnich dni sidłem, \\
        By prawda o Nim była zawsze naszym światłem.
    }
\end{titledpoem}

\begin{titledpoem}
    \stanza{
        W „Żyjącej Świątyni" wróg zastawił sieć, \\
        By dusze wiernych w błędy swoje wpleść. \\
        Osobowość Boga – oto sedno sprawy, \\
        Nie sposób jej naprawić przez trynitarne wprawy.
    }

    \stanza{
        White czytała listy, znała Kellogga plan, \\
        By książkę poprawić, zmienić kilka zdań. \\
        Lecz nazwała to bezowocnym działaniem, \\
        Bo błąd tkwił głębiej, nie w samym pisaniu.
    }

    \stanza{
        Adwentyści wtedy Trójcy nie głosili, \\
        Fundamentalne Zasady inaczej uczyli. \\
        Daniells widział sprzeczność z ewangelią jasną, \\
        Gdy Kellogg chciał naprawić doktrynę swą własną.
    }
\end{titledpoem}

\begin{titledpoem}
    \stanza{
        Świadectwa Specjalne musiały powstać, \\
        By prawdę o Bogu jasno ludziom podać. \\
        Nie wystarczy zmienić słowa czy wyrazy, \\
        Gdy pogląd o Bogu jest pełen skazy.
    }

    \stanza{
        Kellogg nie rozumiał własnych błędów wagi, \\
        Myślał, że wystarczą drobne poprawki, zmiany. \\
        Lecz White widziała głębiej, dostrzegała sedno, \\
        Że z Bożą osobowością igrać jest niegodnym.
    }

    \stanza{
        Bóg w opatrzności swojej kryzys zażegnał, \\
        Przez posłanniczkę swoją prawdę przekazał. \\
        Fundament naszej wiary musi być strzeżony, \\
        By Kościół Boży nie został zwiedziony.
    }
\end{titledpoem}