% Bóg szabatu a Bóg niedzieli - J. B. Frisbie

\begin{titledpoem}

    \stanza{
        Czy dnia siódmego, czy pierwszego, \\
        Tu prawda głębsza jest od tego. \\
        Nie tylko, kiedy się modlimy, \\
        Lecz którego Boga tym czcimy.
    }
    
    \stanza{
        Boga szabatu, konkretnego — \\
        Ma postać, miejsce — realnego. \\
        Na obraz Jego powstaliśmy \\
        I w Synu Go zobaczyliśmy.
    }
    
    \stanza{
        Gdyż Syn, Ojca jasne odbicie \\
        To ścieżka, którą płynie życie. \\
        „Widziałeś Mnie, więc też i Jego” — \\
        To słowa Świadka prawdziwego.
    }
    
    \stanza{
        Niedzieli bóg to bóg nicości, \\
        Trzej w jednym w przedziwnej jedności. \\
        Bez ciała czy też części ciała, \\
        Ludzka fantazja wybujała.
    }
    
    \stanza{
        Bóg z twarzą, w określonej formie, \\
        Odpoczął po stworzenia sztormie. \\
        Zaprzestał pracy dnia siódmego \\
        I mamy modlić się do Niego.
    }
    
    \stanza{
        A nie do oparów mistycznych, \\
        Do ducha bez kształtów fizycznych, \\
        Ale do Boga prawdziwego, \\
        Który ma Syna jedynego.
    }
    
    \stanza{
        Nie tylko więc, kiedy klękamy, \\
        Lecz z nich którego wybieramy, \\
        Ma zasadnicze tu znaczenie; \\
        W prawdziwym Bogu nasze tchnienie.
    }

\end{titledpoem}
