% Bóg szabatu a Bóg niedzieli - J. B. Frisbie

\begin{titledpoem}

    \stanza{
        Dwa obrazy Boga, dwie różne natury, \\
        Jeden prawdą jaśnieje, drugi w cieniu bury. \\
        Bóg szabatu osobą z ciałem i częściami, \\
        Bóg niedzieli bezkształtny, rozlany nad światami.
    }

    \stanza{
        Na obraz Stwórcy człowiek uformowany, \\
        Z prochu ziemi przez Boga mądrze ukształtowany. \\
        Syn jest obrazem Ojca, doskonałym odbiciem, \\
        Kto widział Syna, Ojca ujrzał z zachwytem.
    }

    \stanza{
        Bóg ma tron w niebie, miejsce przebywania, \\
        Przez Ducha swego wszędzie jest od zarania. \\
        Nie jest On wszędzie obecny w swej istocie, \\
        Lecz przez Ducha działa w każdej życia dobrocie.
    }

\end{titledpoem}

\begin{titledpoem}

    \stanza{
        Szabat czy niedziela – więcej niż dzień tygodnia, \\
        To wyznanie wiary, co duszę rozpogadnia. \\
        Szabat wskazuje na Boga osobowego, \\
        Niedziela na bóstwo z pogaństwa wyjętego.
    }

    \stanza{
        Bóg szabatu ma ciało, ma oblicze swoje, \\
        Syn Jego jest obrazem, nie jakąś częścią troje. \\
        Bóg niedzieli to "trzech w jednym" bez ciała, \\
        Doktryna, którą filozofia pogańska znała.
    }

    \stanza{
        Jeden Bóg prawdziwy na tronie zasiada, \\
        Drugi wszędzie i nigdzie, jak mgła się rozkłada. \\
        Wybór dnia świętego to wybór Boga twego, \\
        Kogo czcisz naprawdę? Osobę czy coś mglistego?
    }

\end{titledpoem}

\begin{titledpoem}

    \stanza{
        Frisbie odkrył prawdę, gdy szabat przyjmował, \\
        Od metodystów odszedł, Trójcę odrzutował. \\
        Poznał Boga Biblii, nie boga tradycji, \\
        Znalazł prawdę czystą, bez ludzkiej kompozycji.
    }

    \stanza{
        Bóg ma ciało duchowe, ma swoje mieszkanie, \\
        Przez Syna objawia swoje panowanie. \\
        Nie jest On substancją bez formy i kształtu, \\
        Lecz Osobą, co godna czci i zachwytu.
    }

    \stanza{
        Dzień siódmy czy pierwszy – to nie tylko data, \\
        To wyznanie, kogo dusza twa uważa za Brata. \\
        Boga, który stworzył i odpoczął potem, \\
        Czy bóstwo wymyślone ludzkim umysłem i słowem.
    }
    
\end{titledpoem}