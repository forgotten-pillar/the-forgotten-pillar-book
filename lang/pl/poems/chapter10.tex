% Czy Bóg jest osobą? - według Johna N. Loughborougha

\begin{titledpoem}

    \stanza{
        Bóg nie jest duchem bez postaci, \\
        Lecz osobą, co kształt swój ma w pamięci. \\
        Na Jego obraz człowiek stworzony, \\
        Z prochu ziemi, przez Boga ożywiony.
    }

    \stanza{    
        Przez Ducha Swego wszędzie obecny, \\
        Choć w niebie tron Jego jest odwieczny. \\
        Oblicza Jego śmiertelnik nie ujrzy, \\
        Zbyt jasna chwała dla oczu człowieczych.
    }

    \stanza{
        W Chrystusie widzimy Ojca odbicie, \\
        Wyraz istoty Jego w pełnym zachwycie. \\
        Nie jest to tylko moralne podobieństwo, \\
        Lecz kształt i forma - Boże pierwszeństwo.
    }

    \stanza{
        Gdy czyste serce kiedyś Go zobaczy, \\
        Ujrzy osobę, nie ducha bez znaczeń. \\
        Bo Bóg jest ciałem, choć niewidzialnym, \\
        Dla śmiertelników wciąż niedotykalnym.
    }

\end{titledpoem}

\begin{titledpoem}

    \stanza{
        Nie wszędzie równo Bóg przebywa, \\
        W niebie Jego chwała spoczywa. \\
        Stamtąd na ziemię spogląda z troską, \\
        Przez Ducha działa z mocą Boską.
    }

    \stanza{
        Mojżesz Go widział, lecz tylko od tyłu, \\
        Bo twarz Boża zbyt jasna dla wzroku. \\
        Dłonią Swą zakrył Mojżesza oblicze, \\
        By śmierć nie była spotkania zdobyczą.
    }

    \stanza{
        Nie jest to złuda czy mistyfikacja, \\
        Lecz prawda, którą Biblia oznacza. \\
        Bóg nie zwodzi nas pustym obrazem, \\
        Lecz jest osobą, z ciałem zarazem.
    }

\end{titledpoem}

\begin{titledpoem}

    \stanza{
        Chrystus w postaci Boga istnieje, \\
        I równość z Bogiem w sobie zawiera. \\
        Jak można mówić o Bożej postaci, \\
        Jeśli Bóg formy swej nie bogaci?
    }

    \stanza{
        Daniel opisał Przedwiecznego tron, \\
        Szatę jak śnieg i włosy jak wełnę. \\
        Głowę i ciało Bóg zatem posiada, \\
        Nie jest bezkształtny - Pismo tak powiada.
    }

    \stanza{
        Kto widzi Syna, Ojca też widzi, \\
        Nie że są jednym - to by nas zwiodło. \\
        Lecz że Syn Boży jest doskonałym, \\
        Obrazem Ojca w kształcie wspaniałym.
    }

    \stanza{
        Gdy zmartwychwstali Go zobaczymy, \\
        Osobę realną, nie ducha ujrzymy. \\
        Bo czystym sercem Boga oglądać będziemy, \\
        Gdy nieśmiertelne oczy otrzymamy.
    }
    
\end{titledpoem}