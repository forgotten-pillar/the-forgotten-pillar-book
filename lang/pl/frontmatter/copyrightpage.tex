\clearpage % Zapewnia, że następna treść zaczyna się na nowej stronie, ale unika dodatkowej pustej strony
%\null\par % Alternatywa, aby uniknąć pustej strony, jeśli to konieczne

{\small
\setlength{\parindent}{0em}\setlength{\parskip}{1em}

{\large \emcap{\booktitle}}

\emcap{Autorzy}: \textbf{\authorname} \\
\ifx\editor\undefined\else\if\relax\detokenize\expandafter{\editor}\relax\else{\emcap{Redaktor}: \textbf{\editor{}}} \\ \fi\fi
\ifx\translatedby\undefined\else\if\relax\detokenize\expandafter{\translatedby}\relax\else{\emcap{Tłumaczenie}: \textbf{\translatedby{}}} \\ \fi\fi
\ifx\publisher\undefined\else\if\relax\detokenize\expandafter{\publisher}\relax\else{\emcap{Wydawca}: \textbf{\publisher{}}} \\ \fi\fi
\ifx\publishingplace\undefined\else\if\relax\detokenize\expandafter{\publishingplace}\relax\else{\emcap{Miejsce i rok wydania}: \textbf{\publishingplace{}, \publishingyear{}}} \\ \fi\fi
\ifx\isbn\undefined\else\if\relax\detokenize\expandafter{\isbn}\relax\else{\emcap{ISBN}: \textbf{\isbn{}}} \\ \fi\fi
\emcap{Wersja}: \textbf{0.0.1} | \emcap{RD}: \textbf{20250401} \\
\emcap{Wersja oryginalna}: \textbf{3.0.1} | \emcap{RD}: \textbf{20250401} \\
\ifx\publishingplacefirstedition\undefined\else\if\relax\detokenize\expandafter
{\publishingplacefirstedition}\relax\else{\emcap{Pierwsze wydanie opublikowano}: \textbf{\publishingplacefirstedition{}, \publishingyearfirstedition{}}} \\ \fi\fi
\ifx\isbnfirstedition\undefined\else\if\relax\detokenize\expandafter{\isbnfirstedition}\relax\else{\emcap{ISBN (Pierwsze wydanie)}: \textbf{\isbnfirstedition{}}} \\ \fi\fi
\ifx\poems\undefined\else\if\relax\detokenize\expandafter{\poems}\relax\else{\emcap{Wiersze}: Dzięki uprzejmości \textbf{\poems{}}} \\ \fi\fi
\emcap{Źródło obrazów}: \textit{Dzięki uprzejmości \href{https://ellenwhite.org/}{Ellen G. White Estate, Inc}}

\vfill

% Tekst licencji
{\licensesize \licensetext}

\ifepub
    \includegraphics[width=\linewidth]{images/logo-black.png}

    Tę książkę można pobrać za darmo na stronie \href{https://forgottenpillar.com/book/the-forgotten-pillar}{www.forgottenpillar.com}
\else
    % Utworzenie wiersza z logo, tekstem i kodem QR
    \noindent
    \begin{minipage}{0.3\textwidth}
        \includegraphics[width=\linewidth]{images/logo-black.png}
    \end{minipage}%
    \hfill
    \begin{minipage}{\dimexpr\linewidth-0.3\textwidth-0.125\textwidth-1em\relax}
        \raggedleft\footnotesize
        Tę książkę można pobrać za darmo na stronie \href{https://forgottenpillar.com/book/the-forgotten-pillar?lang=\currentlang&type=\currentlayout}{www.forgottenpillar.com}
    \end{minipage}%
    \ifnum\pdfstrcmp{\currentlayout}{mobile}=0
    \else
        \hfill
        \begin{minipage}{0.125\textwidth}
            \centering
            \qrcode{https://forgottenpillar.com/book/the-forgotten-pillar?lang=\currentlang&type=\currentlayout}
        \end{minipage}
    \fi%
\fi
}

\pagebreak