\chapter{Kroki do Omegi}

W naszym dotychczasowym studium widzieliśmy dowody, na to że kontrowersja Kellogga była związana z doktryną o Trójcy i \emcap{osobowością Boga} wyrażoną w pierwszym punkcie \emcap{Fundamentalnych Zasad}. Niestety, dzisiaj nie stoimy na tym fundamencie dotyczącym \emcap{osobowości Boga}; zbudowaliśmy inny fundament, który zmienił prawdę o \emcap{osobowości Boga} w tajemniczego Trójjedynego Boga. Siostra White była wyraźnie przeciwna tej reorganizacji i prorokowała, że przy końcu Swojego dzieła Bóg powtórzy historię ruchu adwentowego i przywróci każdy filar naszej wiary, który był utrzymywany na początku.

\egw{\textbf{\underline{Pan oświadczył, że historia z przeszłości zostanie powtórzona, gdy wkroczymy w końcowe dzieło}. \underline{Każda prawda}, którą dał na te ostatnie dni, ma być ogłoszona światu. \underline{Każdy filar}, który ustanowił, \underline{ma być wzmocniony}. Nie możemy teraz zejść z fundamentu, który Bóg ustanowił. Nie możemy teraz wejść w żadną nową organizację; oznaczałoby to odstępstwo od prawdy}.}[Ms129-1905.6; 1905][https://egwwritings.org/?ref=en\_Ms129-1905.6&para=9797.13]

Porównując \emcap{Fundamentalne Zasady} z obecnymi Fundamentalnymi Wierzeniami Adwentystów Dnia Siódmego, widzimy, że weszliśmy w nową organizację. Boże ostrzeżenie, przekazane przez Siostrę White, aby przywrócić wszystkie filary naszej wiary w tych ostatnich dniach, staje się koniecznością. Śledząc doktrynę o Trójcy od kontrowersji Kellogga, natrafiliśmy na ostrzeżenia Ellen White przed odstępstwem alfa i omega, które wejdzie do naszego kościoła.

\egw{\textbf{‘Living Temple’ zawiera alfę tych teorii. Wiedziałam, że \underline{omega nastąpi wkrótce}; i drżałam o nasz lud}. Wiedziałam, że \textbf{muszę ostrzec naszych braci i siostry, aby nie wchodzili w spór \underline{dotyczący obecności i osobowości Boga}. Stwierdzenia zawarte w ‘Living Temple’ \underline{odnośnie tego punktu są niepoprawne}. }Pismo Święte użyte do poparcia doktryny tam przedstawionej jest błędnie zastosowane.}[SpTB02 53.2; 1904][https://egwwritings.org/?ref=en\_SpTB02.53.2&para=417.271]

W kontekście reorganizacji Adwentystów Dnia Siódmego, identyfikujemy kilka kroków, które były niezbędne do dokonania tej reorganizacji i są konieczne do jej podtrzymania.

\subsection*{Krok 1: Zaprzeczenie, że Fundamentalne Zasady są fundamentem naszej wiary oraz oficjalną i dokładną reprezentacją wierzeń Adwentystów Dnia Siódmego}

Pierwszym niezbędnym krokiem jest ukrycie oryginalnego fundamentu naszej wiary poprzez odłączenie go od \emcap{Fundamentalnych Zasad}.

\egw{\textbf{Jako lud mamy \underline{stać niewzruszenie  na platformie wiecznej prawdy}, która przetrwała próby i doświadczenia. Mamy \underline{trzymać się pewnych filarów naszej wiary}. \underline{Zasady prawdy}, które Bóg nam objawił, \underline{są naszym jedynym prawdziwym fundamentem}. To one uczyniły nas tym, czym jesteśmy. Upływ czasu nie zmniejszył ich wartości. \underline{Nieustannym wysiłkiem wroga jest usunięcie tych prawd z ich miejsca} i zastąpienie ich \underline{fałszywymi teoriami}. \underline{Wprowadzi} on wszystko, co tylko może, aby zrealizować swoje zwodnicze plany.}}[SpTB02 51.2; 1904][https://egwwritings.org/?ref=en\_SpTB02.51.2&para=417.261]

\egw{\textbf{Przesłania wszelkiego rodzaju i typu były narzucane Adwentystom Dnia Siódmego, aby zająć miejsce prawdy, która \underline{punkt po punkcie} została odkryta przez pełne modlitwy studium i potwierdzona przez cudotwórczą moc Pana}. \textbf{Lecz \underline{drogowskazy}, które \underline{uczyniły nas tym, czym jesteśmy}, \underline{mają być zachowane} i \underline{będą zachowane}, jak Bóg oznajmił przez swoje słowo i świadectwo swojego Ducha}. \textbf{Wzywa nas, abyśmy \underline{trzymali się mocno}, z uściskiem wiary, \underline{fundamentalnych zasad}, które \underline{są oparte na niepodważalnym autorytecie}}}[SpTB02 59.1; 1904][https://egwwritings.org/?ref=en\_SpTB02.59.1&para=417.299]

\emcap{Fundamentalne zasady} były prawdami, które Bóg objawił pionierom po upływie czasu w 1844 roku. Widzieliśmy świadectwa naszych pionierów, w tym Ellen White, dotyczące pierwszego punktu \emcap{Fundamentalnych zasad}. Wszyscy oni byli zgodni co do tych konkretnych punktów naszej wiary. W 1863 roku Adwentyści Dnia Siódmego zorganizowali się w kościół, jako zorganizowane ciało. Od tego czasu wielu błędnie przedstawiało stanowisko Kościoła Adwentystów Dnia Siódmego, a pionierzy uznali za konieczne odpowiadanie na zapytania, \others{a czasami korygowanie fałszywych stwierdzeń rozpowszechnianych przeciwko} wierzeniom i praktykom kościoła. W konsekwencji, w 1872 roku, pionierzy wydali dokument zatytułowany “\textit{Oświadczenie o fundamentalnych zasadach nauczanych i wyznawanych przez Adwentystów Dnia Siódmego}”\footnote{“A Declaration of the Fundamental Principles, Taught and Practiced by the Seventh-Day Adventists (1872) : MVT : Free Download, Borrow, and Streaming : Internet Archive.” Internet Archive, 2025, \href{https://archive.org/details/ADeclarationOfTheFundamentalPrinciplesTaughtAndPracticedByThe}{archive.org/details/ADeclarationOfTheFundamentalPrinciplesTaughtAndPracticedByThe}. Accessed 3 Feb. 2025.}. Ta deklaracja przedstawiła opinii publicznej \others{krótkie oświadczenie o tym, co jest i było, z wielką jednomyślnością, wyznawane przez}[Przedmowa do Fundamentalnych zasad z 1872 roku.] Adwentystów Dnia Siódmego.

W rozdziale “\hyperref[chap:authority]{Autorytet Fundamentalnych zasad}”, omówiliśmy, jak uczeni popierający Trójcę kompromitują autorytet \emcap{Fundamentalnych zasad}, zaprzeczając ich prawdziwej wartości w naszej adwentystycznej historii.

Uczeni popierający Trójcę twierdzą, że ta deklaracja nie była tym, za co się podaje- deklaracją \emcap{fundamentalnych zasad}, nauczanych i praktykowanych przez Adwentystów Dnia Siódmego. Ta deklaracja była podsumowaniem głównych cech wiary adwentystów, i żaden z jej punktów nie jest naprawdę tak problematyczny lub budzący zastrzeżenia jak  punkt pierwszy, dotyczący \emcap{osobowości Boga} i miejsca jego obecności. Jednak dowody przemawiające za  \emcap{Fundamentalnymi zasadami}, zwłaszcza w odniesieniu do pierwszego punktu, są przytłaczające.

Wszystkie te twierdzenia są łatwo obalone przez fakt, że \emcap{Fundamentalne zasady} były regularnie wydawane i przedrukowywane przez całe życie Siostry White, aż do 1914 roku. Gdyby były jedynie prywatnymi opiniami kilku osób, jak twierdzą uczeni\footnote{Ministry Magazine “Our Declaration of Fundamental Beliefs”: January 1958, Roy Anderson, J. Arthur Buckwalter, Louise Kleuser, Earl Cleveland and Walter Schubert}, czy byłyby konsekwentnie przedrukowywane przez 42 lata\footnote{Szczegółową listę publikacji w tych latach można znaleźć w Załączniku.}, publicznie twierdząc, że reprezentują streszczenie wiary Adwentystów Dnia Siódmego? Gdyby zostały wydane tylko raz, moglibyśmy uznać to za spisek kilku osób, które celowo błędnie przedstawiały wiarę Adwentystów Dnia Siódmego. Wręcz przeciwnie, \emcap{Fundamentalne zasady} były regularnie przedrukowywane i naprawdę reprezentowały oficjalną wiarę i praktykę Adwentystów Dnia Siódmego.

Innym argumentem jest to, że Siostra White zatwierdziła \emcap{Fundamentalne zasady} w swoich pismach, wyraźnie się do nich odnosząc, a także nauczając tych samych prawd, które są nauczane w \emcap{Fundamentalnych zasadach}. Prace naszych pionierów są również zgodne z oświadczeniami w tej Deklaracji \emcap{Fundamentalnych zasad}. Biorąc pod uwagę wszystkie te fakty, jest nieuniknione, że deklaracja ta była zgodna z prawdą w swoich twierdzeniach. Dokument ten był rzeczywiście deklaracją \emcap{fundamentalnych zasad}, nauczanych i praktykowanych przez Kościół Adwentystów Dnia Siódmego, stanowiący publiczne \others{streszczenie naszej wiary}, \others{krótkie oświadczenie o tym, co jest i było, z wielką jednomyślnością, wyznawane przez} Adwentystów Dnia Siódmego.\footnote{Przedmowa do Fundamentalnych zasad z 1872 roku.} Tym samym, dokładnie reprezentuje wierzenia i praktyki Adwentystów Dnia Siódmego oraz stanowi fundament wiary Adwentystów Dnia Siódmego w czasach Ellen White.

Dziś, w obronie doktryny o Trójcy, historycy adwentystyczni śmiało twierdzą, że kiedy nasi pionierzy studiowali adwentystyczne prawdy, takie jak świątynia, sąd śledczy, Sabat i inne doktryny, \others{nie studiowali tematu doktryny o Bogu}. Ci historycy adwentystyczni fałszywie twierdzą, że doktryna o Bogu \others{nie była kwestią, którą zajmowali się w tamtym czasie}[Denis Kaiser. “From Antitrinitarianism to Trinitarianism: The Adventist story” and Panelist. The God We Worship: A Godhead Symposium. Central California Conference, Dinuba, CA. March 23-24, 2018.]. Po tym fałszywym twierdzeniu przedstawiają dane historyczne o tym, jak doktryna adwentystyczna stopniowo zmierzała w kierunku trynitarnego zrozumienia. Prawda jest taka, że istnieją pewne wczesne przypadki\footnote{Najwcześniejsza wzmianka o doktrynie o Trójcy, w pozytywnym sensie, miała miejsce, gdy M.C. Wilcox przedrukował nieadwentystyczny artykuł Samuela Speara w Signs of the Times, 7 grudnia 1891 i 14 grudnia 1891}, kiedy doktryna o Trójcy jest wspominana w pozytywnym świetle w naszej literaturze. Ale gdy weźmie się pod uwagę fakt, że Kościół Adwentystów miał pozytywne stanowisko w kwestii doktryny o Bogu, wyrażone w \emcap{Fundamentalnych zasadach}, tych przypadków nie można interpretować jako postępu w zrozumieniu, ale raczej jako wtargnięcie doktryny o Trójcy do Kościoła Adwentystów Dnia Siódmego.

Łatwo jest obalić twierdzenie, że pionierzy adwentystyczni nie rozumieli doktryny o Bogu. Gdyby jej nie rozumieli, nie udałoby im się głosić poselstwa pierwszego anioła. Omówiliśmy ten punkt szczegółowo w rozdziale “\hyperref[chap:remembering-the-beginning]{Pamiętając początek}”. Ruch Adwentystów Dnia Siódmego nie był porażką, ale proroczym ruchem prowadzonym przez Boga.

\subsection*{Krok 2: Ignorowanie ostrzeżeń przed budowaniem nowego fundamentu}
Kiedy \emcap{Fundamentalne zasady} są usunięte z równania, wiele ostrzeżeń Ellen White nie świeci w swoim prawdziwym świetle, a ich prawdziwe znaczenie nie rezonuje z czytelnikiem.

Cytowaliśmy wiele wypowiedzi, w których Siostra White ostrzegała kościół, aby nie odchodził od \emcap{Fundamentalnych zasad}. Zajmowaliśmy się nimi w rozdziale “\hyperref[chap:apostasy]{Wielkie odstępstwo wkrótce się urzeczywistni}”, ale wspomnimy ponownie jeden z najbardziej znaczących cytatów.

\egw{\textbf{Wróg dusz starał się wprowadzić przypuszczenie, że wśród Adwentystów Dnia Siódmego miała nastąpić wielka reforma, i że ta reforma \underline{polegałaby na porzuceniu doktryn, które stoją jako filary naszej wiary} i zaangażowaniu się w proces reorganizacji}. Gdyby ta reforma miała miejsce, co by z tego wynikło? \textbf{Zasady prawdy, które Bóg w swojej mądrości dał Kościołowi ostatków, zostałyby odrzucone. Nasza religia zostałaby zmieniona. \underline{Fundamentalne zasady, które podtrzymywały dzieło przez ostatnie pięćdziesiąt lat, zostałyby uznane za błąd}}. \textbf{Zostałaby ustanowiona nowa organizacja. Zostałyby napisane książki nowego porządku. Zostałby wprowadzony system filozofii intelektualnej}...}[Lt242-1903.13; 1903][https://egwwritings.org/?ref=en\_Lt242-1903.13&para=7767.20]

\egwnogap{Kto ma upoważnienie do rozpoczęcia takiego ruchu? \textbf{Mamy nasze Biblie. Mamy nasze doświadczenie, potwierdzone cudownym działaniem Ducha Świętego}. \textbf{Mamy prawdę, która nie dopuszcza żadnego kompromisu.} \textbf{\underline{Czy nie powinniśmy odrzucić wszystkiego, co nie jest w harmonii z tą prawdą}?}}[Lt242-1903.14; 1903][https://egwwritings.org/?ref=en\_Lt242-1903.14&para=7767.21]

\subsection*{Krok 3: Zaprzeczenie, że osobowość Boga była filarem naszej wiary i częścią fundamentu naszej wiary}

Istnieje jedno stwierdzenie Ellen White, które pozornie popiera twierdzenie, że \emcap{osobowość Boga} nie była filarem naszej wiary. Innym wyrażeniem na “\textit{filary naszej wiary}” jest “\textit{znaki}”. W poniższych cytatach Siostra White wymienia kilka znaków: oczyszczenie świątyni, poselstwa trzech aniołów, świątynię Boga, Sabat i nieśmiertelność bezbożnych.

\egw{Upływ czasu w 1844 roku był okresem wielkich wydarzeń, otwierających naszym zdumionym oczom \textbf{oczyszczenie świątyni odbywające się w niebie} i mające zdecydowany związek z ludem Bożym na ziemi, [także] \textbf{poselstwa pierwszego i drugiego anioła oraz trzeciego}, rozwijające sztandar, na którym było napisane: ‘Przykazania Boże i wiara Jezusa.’ [Objawienie 14:12.] Jednym ze znaków pod tym poselstwem była \textbf{świątynia Boża}, widziana przez Jego miłujący prawdę lud w niebie, i arka zawierająca prawo Boże. Światło \textbf{Sabatu} z czwartego przykazania rzucało swoje silne promienie na ścieżkę przestępców prawa Bożego. \textbf{Nieśmiertelność bezbożnych} jest starym znakiem. \textbf{Nie mogę przypomnieć sobie niczego więcej, co mogłoby wchodzić pod nagłówek starych znaków}. Całe to wołanie o zmianę starych znaków jest całkowicie wyimaginowane.}[Ms13-1889.9; 1889][https://egwwritings.org/?ref=en\_Ms13-1889.9&para=4179.14]

Na końcu tej listy znaków, czyli filarów naszej wiary, stwierdza, że nie może przypomnieć sobie niczego innego, co wchodziłoby do kategorii starych znaków. Dla wielu ten cytat służy jako dowód, że \emcap{osobowość Boga} nie była ani starym znakiem, ani filarem. To prawda, że w tym cytacie Siostra White nie wspomniała wyraźnie o \emcap{osobowości Boga}, ale byłaby ona domyślnie zawarta w poselstwie pierwszego anioła, a także jako podstawowa doktryna poselstwa o Świątyni. Ponadto istnieją inne cytaty od Siostry White, które wyraźnie włączają \emcap{osobowość Boga} jako stary znak lub filar naszej wiary.

\egw{Ci, którzy starają się usunąć \textbf{stare znaki}, nie trzymają się mocno; \textbf{nie pamiętają, jak zostały przyjęte i usłyszane}. Ci, którzy próbują \textbf{\underline{wprowadzić} teorie, które usunęłyby \underline{filary naszej wiary}} \textbf{dotyczące świątyni}, \textbf{\underline{lub dotyczące osobowości Boga lub Chrystusa}, działają jak ślepi ludzie}. Starają się wprowadzić niepewność i sprawić by lud Boży \textbf{ dryfował}, bez kotwicy.}[Ms62-1905.14; 1905][https://egwwritings.org/?ref=en\_Ms62-1905.14&para=10026.20]

Siostra White uczy nas również, że filary naszej wiary stanowią fundament naszej wiary.

\egw{\textbf{Jaki wpływ prowadzi ludzi na tym etapie naszej historii do działania w podstępny, potężny sposób, \underline{aby zburzyć fundament naszej wiary},—fundament, który został położony na początku naszej pracy przez modlitewne studiowanie słowa i przez objawienie? Na \underline{tym fundamencie} budowaliśmy przez \underline{ostatnie pięćdziesiąt lat}. Czy dziwisz się, że kiedy widzę początek pracy, która \underline{usunęłaby niektóre z filarów naszej wiary}, mam coś do powiedzenia? Muszę być posłuszna rozkazowi: «Przeciwstaw się temu!»}}[SpTB02 58.1; 1904][https://egwwritings.org/?ref=en\_SpTB02.58.1&para=417.295]

Usunięcie niektórych filarów naszej wiary oznacza zburzenie fundamentu naszej wiary. W innym miejscu Siostra White powiedziała, że burzenie lub podważanie fundamentu naszej wiary odbywa się poprzez indoktrynację poglądów dotyczących \emcap{osobowości Boga}.

\egw{Uczelnia została usunięta z Battle Creek; jednak studenci nadal są tam wzywani i tam \textbf{zostają indoktrynowani poglądami dotyczącymi osobowości Boga i Chrystusa, które podważyłyby fundament naszej wiary}.}[Lt72-1906.5; 1906][https://egwwritings.org/?ref=en\_Lt72-1906.5&para=10013.11]

W świetle tych cytatów widzimy pozytywne świadectwo, że \emcap{osobowość Boga} była częścią fundamentu naszej wiary. Ponadto, w rozdziale 10 specjalnych świadectw, zatytułowanym “\textit{Fundament naszej wiary}”, Siostra White wspomniała o “\textit{Fundamentalnych Zasadach}” używając synonimów “\textit{filary naszej wiary}”, “\textit{znaki}” i “\textit{punkty orientacyjne}”, odnosząc się do fundamentu naszej wiary.

\subsection*{Krok 4: Zmiana znaczenia terminu “osobowość Boga”}

Termin ‘\textit{osobowość}’ ma dwa różne zastosowania, a najczęstsza definicja w codziennym użyciu dotyczy dziedziny psychologii. ‘\textit{Osobowość}’ jest definiowana jako “\textit{charakterystyczny zestaw zachowań, procesów poznawczych i wzorców emocjonalnych, które ewoluują z czynników biologicznych i środowiskowych}”\footnote{Wikipedia Contributors. “Personality.” Wikipedia, Wikimedia Foundation, 19 Apr. 2019, \href{https://en.wikipedia.org/wiki/Personality}{en.wikipedia.org/wiki/Personality}.}. Niezwykle ważne jest, aby zdać sobie sprawę, że gdy zajmujemy się filarem naszej wiary—“\textit{osobowością Boga}”—nie poruszamy się w dziedzinie psychologii. Dokładne zastosowanie słowa ‘\textit{osobowość}’ w doktrynie o \emcap{osobowości Boga} znajduje się w Słowniku Merriam-Webster: “\textit{właściwość lub stan jako osoby}”\footnote{\href{https://www.merriam-webster.com/dictionary/personality}{Merriam-Webster Dictionary} - ‘\textit{personality}’}. Według Słownika Merriam-Webster, ta definicja jest używana od XV wieku\footnote{Zobacz “\href{https://www.merriam-webster.com/dictionary/personality\#word-history}{First known use}” słowa ‘personality’ w Słowniku Merriam Webster}. W wydaniu Słownika Merriam Webster z 1828 roku czytamy definicję słowa ‘\textit{osobowość}’ jako: “\textit{to, co stanowi jednostkę odrębną osobą}”\footnote{\href{https://archive.org/details/americandictiona02websrich/page/272/mode/2up}{Merriam-Webster Dictionary, 1828 edition} - ‘\textit{personality}’} \footnote{\href{https://archive.org/details/websterscomplete00webs/page/974/mode/2up}{Wydanie Słownika Merriam-Webster z 1886 roku} definiuje słowo ‘\textit{personality}’ jako: “\textit{to, co stanowi lub odnosi się do osoby}”}. Obie definicje znajdują się w The Encyclopaedic Dictionary autorstwa Huntera Roberta\footnote{\href{https://babel.hathitrust.org/cgi/pt?id=mdp.39015050663213&view=1up&seq=780}{Hunter Robert, The Encyclopaedic Dictionary} - ‘\textit{personality}’}—słowniku należącym do Ellen White. Użycie tych definicji można zobaczyć w artykułach napisanych na temat \emcap{osobowości Boga}.

W 1903 roku, kiedy Siostra White napisała do dr. Kellogga, \egwinline{Zawsze \textbf{miałam }to samo świadectwo do złożenia, które teraz składam \textbf{odnośnie osobowości Boga}}[Lt253-1903.9; 1903][https://egwwritings.org/?ref=en\_Lt253-1903.9&para=9980.15], przypomniała sobie swoją wizję, w której widziała Ojca i Syna.

\egw{Często widziałam kochanego Jezusa, że \textbf{jest On osobą}. \textbf{Zapytałam Go, czy Jego Ojciec jest osobą} i \textbf{ma \underline{postać} jak On}. Jezus powiedział: «\textbf{Jestem doładnym obrazem osoby Mojego Ojca!}» [Hbr 1:3.]}[Lt253-1903.12; 1903][https://egwwritings.org/?ref=en\_Lt253-1903.12&para=9980.18]





Właściwość lub stan, który Siostra White definiuje, że Bóg jest osobą, to posiadanie \textit{postaci}—\textit{fizycznego wyglądu}. Dr Kellogg podąża za tym samym zastosowaniem słowa \textit{‘osobowość’},ale poprzez spekulację.

\others{Fakt, że Bóg jest tak wielki, że nie możemy stworzyć wyraźnego mentalnego obrazu \textbf{jego fizycznego wyglądu}, nie musi umniejszać w naszych umysłach rzeczywistości \textbf{Jego osobowości}...}[John H. Kellogg, The Living Temple, s. 31][https://archive.org/details/J.H.Kellogg.TheLivingTemple1903/page/n31/mode/2up]

Jak wcześniej widzieliśmy, nasi adwentystyczni pionierzy również wskazywali na fizyczny wygląd jako cechę, która czyni Boga osobą. James White napisał, \others{Ci, którzy zaprzeczają \textbf{osobowości Boga}, mówią, że ‘obraz’ tutaj nie oznacza \textbf{fizycznej formy}, ale obraz moralny...}[James S. White, PERGO 1.1; 1861][https://egwwritings.org/?ref=en\_PERGO.1.1&para=1471.3]. J. B. Frisbie napisał, \others{Niektórzy wydają się zakładać, że to przemawia przeciwko \textbf{osobowości Boga}, ponieważ jest On Duchem, i twierdzą, że jest On bez \textbf{ciała lub części}...}[\href{https://documents.adventistarchives.org/Periodicals/RH/RH18540307-V05-07.pdf}{Adventist Review and Sabbath Herald, 7 marca 1854}, J. B. Frisbie, “The Seventh-Day Sabbath Not Abolished”, s. 50]

W świetle faktów, rozpoznajemy zastosowanie słowa ‘\textit{osobowość}’. Kiedy temat \emcap{osobowości Boga} jest przedstawiany w związku z doktryną o Trójcy, często istnieje tendencja do zmiany znaczenia słowa ‘\textit{osobowość}’. Ważne jest również, aby wspomnieć, że temat \emcap{osobowości Boga} dotyczy osobowości Ojca. Widać to wyraźnie  z przedstawionych danych.

\subsection*{Krok 5: W badaniu kryzysu Kellogga, przesunięcie głównego punktu uwagi z osobowości Boga na panteizm}

Dane dotyczące kryzysu Kellogga, w związku z doktryną o Trójcy, są przytłaczające, jeśli w równaniu uwzględni się \emcap{osobowość Boga} . Jedynym sposobem, aby nie połączyć kropek, jest zignorowanie \emcap{osobowości Boga} i przesunięcie uwagi wyłącznie na panteizm. Nie zaprzeczamy panteistycznej naturze sporu Kellogga. Wierzymy, że panteistyczna natura sporu Kellogga nie może być właściwie zrozumiana, jeśli nie jest badana w prawdziwym świetle \emcap{osobowości Boga}. Niestety jednak, w badaniu kryzysu Kellogga, uwaga, jaką otrzymuje panteizm, przewyższa badanie prawdy o \emcap{osobowości Boga}.

Możesz przeszukać kompilacje Ellen White, aby zobaczyć, o ile więcej uwagi poświęcono panteizmowi niż \emcap{osobowości Boga}. Jeśli przeszukałbyś jej pisma pod kątem słów ‘panteizm’ lub ‘panteistyczny’, wykluczając kompilacje po jej śmierci, znalazłbyś 36 wystąpień. Wśród nich jest kilka powtarzających się cytatów, które Siostra White kopiowała z jednego listu do drugiego lub do specjalnych świadectw dla kościoła. Gdybyś policzył odrębne wystąpienia, znalazłbyś tylko 12 odrębnych cytatów zawierających słowa takie jak ‘\textit{panteizm}’ lub ‘\textit{panteistyczny}’\footnote{W pasku wyszukiwania na stronie \href{https://egwwritings.org/}{https://egwwritings.org/} wpisz słowo “\textit{pantheis*} “; obejmie to wszystkie słowa zaczynające się od ‘\textit{pantheis...}’, (w tym ‘\textit{panteizm}’ i ‘\textit{panteistyczny}’). Wyniki można porównać, dzieląc korpus pism Ellen White poprzez włączenie lub wykluczenie kompilacji po jej śmierci. Ta opcja jest dostępna w menu rozwijanym pod paskiem wyszukiwania.}. Jeśli przeprowadziłbyś to samo wyszukiwanie, ale tylko w kompilacjach wydanych po jej śmierci, znalazłbyś 140 wystąpień! Wszystkie one należą do jednego z dwunastu odrębnych przypadków, w których Siostra White pisała na temat panteizmu.

W wyszukiwaniu pism Ellen White na temat frazy “\textit{osobowość Boga}”, z wyłączeniem kompilacji po jej śmierci, znalazłbyś 58 wystąpień. Wśród nich również znajduje się kilka powtarzających się cytatów, które Siostra White kopiowała do kilku różnych listów i do świadectw dla kościoła. Jednak gdybyś przeszukał tę frazę w kompilacjach, które zostały wydane po jej śmierci, znalazłbyś tylko 52 wystąpienia.

Te proste statystyki pokazują, na czym skupiali się kompilatorzy po śmierci Siostry White. Taki nacisk na panteizm zmienił naszą opinię publiczną dotyczącą kryzysu Kellogga. Czterdzieści trzy z pięćdziesięciu ośmiu cytatów na temat frazy “\textit{osobowość Boga}” znajduje się w listach i rękopisach, dostępnych publicznie od 2015 roku. Oznacza to, że trzy czwarte (\textit{74 procent}) cytatów dotyczących \emcap{osobowości Boga}, przed 2015 rokiem, nie było dostępnych publicznie. Przed 2015 rokiem nie mieliśmy wielu dostępnych danych, aby badać kryzys Kellogga w świetle \emcap{osobowości Boga} i w jego kontekście.


%% Steps to Omega

\begin{titledpoem}
    
    \stanza{
        On pillars now, the shadows cast— \\
        A truth forsaken, from the past. \\
        In steps they chart the silent drift, \\
        Five marks of change, through sacred rift.
    }

    \stanza{
        Denial blooms when once truth stood, \\
        Foundations are not understood, \\
        The fundamentals, once held dear \\
        Obscured, as new creeds appear.
    }

    \stanza{
        Prophetic warnings have been dimmed, \\
        Pioneers are shunned, old hymns are trimmed. \\
        The testimonies once rang out \\
        But now they’re often tinged with doubt.
    }

    \stanza{
        “God is a person” cast aside, \\
        And now His essence they deride. \\
        Forgotten pillar once was strong \\
        Now a new pillar, which is wrong!
    }

    \stanza{
        Scholars now twist the sacred term, \\
        Words redefined, they now affirm. \\
        Gone is the quest to see God’s face, \\
        Dim the desire for His embrace.
    }

    \stanza{
        The Kellogg crisis point is missed, \\
        The alpha given untrue twist \\
        And thus, the lessons are not learned \\
        The church toward omega turned.
    }

    \stanza{
        Confusion reigns, we can’t perceive \\
        It is not clear what we believe \\
        Our history has been revised \\
        We wanted truth, but then they lied.
    }
    
\end{titledpoem}

% \begin{titledpoem}

    \stanza{
        Dziś na filary padły cienie, \\
        A prawda idzie w zapomnienie. \\
        Przez kroków pięć w ciszy prowadzą, \\
        Aż rozłam wśród ludu wprowadzą.
    }

    \stanza{
        Fałsz kwitnie tam, gdzie prawda stała, \\
        Którą zrozumieć trzódka miała. \\
        A to, co fundamentem było, \\
        W wyznanie wiary się zmieniło.
    }

    \stanza{
        Prorocze słowa porzucone, \\
        Pionierzy też, pieśni zmienione. \\
        Świadectwa kiedyś tak dźwięczały, \\
        Lecz teraz respekt do nich mały.
    }

    \stanza{
        „Bóg jest osobą” odrzucono, \\
        Jego istotę znieważono. \\
        Prawdziwy filar zapomniany, \\
        A nowy, błędny jest nam dany.
    }

    \stanza{
        Uczeni przekręcają słowa, \\
        By zmieniła znaczenie mowa. \\
        Nikt nie chce widzieć Bożej twarzy, \\
        O Jego uścisku nie marzy.
    }

    \stanza{
        Kellogga kryzys przekręcany, \\
        Krok alfa nie jest zrozumiany. \\
        A że się prawdzie nie dowierza, \\
        To Kościół ku omedze zmierza.
    }

    \stanza{
        Wśród wiernych zamieszanie rządzi \\
        I wielu dziś w wierzeniach błądzi. \\
        Historię naszą przepisano, \\
        Miała być prawda, lecz skłamano.
    }

\end{titledpoem}
