\qrchapter{https://forgottenpillar.com/rsc/pl-fp-chapter26}{Kroki do Omegi}

W naszym dotychczasowym studium widzieliśmy dowody, na to że spór wokół Kellogga był związany z doktryną o Trójcy i \emcap{osobowością Boga} wyrażoną w pierwszym punkcie \emcap{Fundamentalnych Zasad}. Niestety, dzisiaj nie stoimy na tym fundamencie dotyczącym \emcap{osobowości Boga}; zbudowaliśmy inny fundament, który zmienił prawdę o \emcap{osobowości Boga} w tajemniczego Trójjedynego Boga. Siostra White była wyraźnie przeciwna tej reorganizacji i prorokowała, że przy końcu swojego dzieła Bóg powtórzy historię ruchu adwentowego i ponownie ustanowi każdy filar naszej wiary, którego trzymano się na początku.

\egw{\textbf{\underline{Pan oświadczył, że historia z przeszłości powtórzy się, gdy będziemy wkraczać w końcowe dzieło}. \underline{Każda prawda}, którą dał na te ostatnie dni, ma być ogłoszona światu. \underline{Każdy filar}, który ustanowił, \underline{ma być wzmocniony}. Nie możemy teraz zejść z fundamentu, który Bóg ustanowił. Nie możemy teraz wejść w żadną nową organizację; oznaczałoby to odstępstwo od prawdy}}[Ms129-1905.6; 1905][https://egwwritings.org/read?panels=p9797.13]

Porównując \emcap{Fundamentalne Zasady} z obecnymi Fundamentalnymi Wierzeniami Adwentystów Dnia Siódmego, widzimy, że weszliśmy w nową organizację. Boże ostrzeżenie, przekazane przez siostrę White, aby ponownie ustanowić wszystkie filary naszej wiary w tych ostatnich dniach, staje się koniecznością. Śledząc doktrynę o Trójcy od sporu wokół Kellogga, natrafiliśmy na ostrzeżenia Ellen White przed odstępstwem alfa i omega, które wejdzie do naszego Kościoła.

\egw{\textbf{«The Living Temple» zawiera alfę tych teorii. Wiedziałam, że \underline{omega nastąpi wkrótce}; i drżałam o nasz lud}. Wiedziałam, że \textbf{muszę ostrzec naszych braci i siostry, aby nie wdawali się w spory \underline{dotyczące obecności i osobowości Boga}. Stwierdzenia zawarte w «The Living Temple» \underline{w tej kwestii są niepoprawne}.} Tylko przez błędne zastosowanie Pisma można poprzeć przedstawioną tam doktrynę}[SpTB02 53.2; 1904][https://egwwritings.org/read?panels=p417.271]

W kontekście reorganizacji Adwentystów Dnia Siódmego identyfikujemy kilka kroków, które były niezbędne do przeprowadzenia tej reorganizacji i są konieczne do jej podtrzymania.

\subsection*{Krok 1: Zaprzeczenie, że Fundamentalne Zasady są fundamentem naszej wiary oraz oficjalnym i dokładnym odzwierciedleniem wierzeń Adwentystów Dnia Siódmego}

Pierwszym niezbędnym krokiem jest ukrycie oryginalnego fundamentu naszej wiary poprzez odłączenie go od \emcap{Fundamentalnych Zasad}.

\egw{\textbf{Jako lud mamy \underline{stać niewzruszenie  na platformie wiecznej prawdy}, która przetrwała próby i doświadczenia. Mamy \underline{trzymać się pewnych filarów naszej wiary}. \underline{Zasady prawdy}, które Bóg nam objawił, \underline{są naszym jedynym prawdziwym fundamentem}. To one uczyniły nas tym, czym jesteśmy. Upływ czasu nie zmniejszył ich wartości. \underline{Nieustannym wysiłkiem wroga jest usunięcie tych prawd z ich miejsca} i zastąpienie ich \underline{fałszywymi teoriami}. \underline{Wprowadzi} on wszystko, co tylko może, aby zrealizować swoje zwodnicze plany}}[SpTB02 51.2; 1904][https://egwwritings.org/read?panels=p417.261]

\egw{\textbf{Przesłania wszelkiego rodzaju i typu były narzucane Adwentystom Dnia Siódmego, aby zająć miejsce prawdy, która \underline{punkt po punkcie} została odkryta przez pełne modlitwy studium i potwierdzona przez cudotwórczą moc Pana}. \textbf{Lecz \underline{drogowskazy}, które \underline{uczyniły nas tym, czym jesteśmy}, \underline{mają być zachowane} i \underline{będą zachowane}, jak Bóg oznajmił przez swoje słowo i świadectwo swojego Ducha}. \textbf{Wzywa nas, abyśmy \underline{trzymali się mocno}, z uściskiem wiary, \underline{fundamentalnych zasad}, które \underline{są oparte na niepodważalnym autorytecie}}}[SpTB02 59.1; 1904][https://egwwritings.org/read?panels=p417.299]

\emcap{Fundamentalne Zasady} były prawdami, które Bóg objawił pionierom po upływie czasu w 1844 roku. Widzieliśmy świadectwa naszych pionierów, w tym Ellen White, dotyczące pierwszego punktu \emcap{Fundamentalnych Zasad}. Wszyscy oni byli zgodni co do tych konkretnych punktów naszej wiary. W 1863 roku Adwentyści Dnia Siódmego zorganizowali się w Kościół, jako zorganizowane ciało. Od tego czasu wielu błędnie przedstawiało stanowisko Kościoła Adwentystów Dnia Siódmego, a pionierzy uznali za konieczne odpowiadanie na zapytania, \others{a czasami korygowanie fałszywych stwierdzeń rozpowszechnianych przeciwko} wierzeniom i praktykom kościoła. W konsekwencji, w 1872 roku, pionierzy wydali dokument zatytułowany „\textit{Deklaracja Fundamentalnych Zasadach, nauczanych i przestrzeganych przez Adwentystów Dnia Siódmego}”\footnote{\href{https://archive.org/details/ADeclarationOfTheFundamentalPrinciplesTaughtAndPracticedByThe}{\textit{A Declaration of the Fundamental Principles, Taught and Practiced by the Seventh-Day Adventists}, Steam Press of the Seventh-Day Adventist Publishing Association, Battle Creek, Mich. 1872.}}. Ta deklaracja przedstawiła opinii publicznej \others{krótkie oświadczenie o tym, co jest i było, z wielką jednomyślnością, wyznawane przez}[Przedmowa do Fundamentalnych Zasad, 1872.] Adwentystów Dnia Siódmego.

W rozdziale „\hyperref[chap:authority]{Autorytet Fundamentalnych Zasad}” omówiliśmy, w jaki sposób pro-trynitarni uczeni podważają autorytet \emcap{Fundamentalnych Zasad}, zaprzeczając ich prawdziwej wartości w naszej adwentystycznej historii.

Uczeni popierający Trójcę twierdzą, że ta deklaracja nie była tym, za co się podaje — deklaracją \emcap{Fundamentalnych Zasad}, nauczanych i przestrzeganych przez Adwentystów Dnia Siódmego. Ta deklaracja była podsumowaniem głównych aspektów wiary adwentystów, i żaden z jej punktów nie jest naprawdę tak problematyczny lub budzący zastrzeżenia jak  punkt pierwszy, dotyczący \emcap{osobowości Boga} i tego, gdzie jest Jego obecność. Jednak dowody przemawiające za  \emcap{Fundamentalnymi Zasadami}, zwłaszcza w odniesieniu do pierwszego punktu, są przytłaczające.

Wszystkie te twierdzenia łatwo obalić przez fakt, że \emcap{Fundamentalne Zasady} były regularnie wydawane i przedrukowywane przez całe życie siostry White, aż do 1914 roku. Gdyby były jedynie prywatnymi opiniami kilku osób, jak twierdzą uczeni\footnote{R. Anderson, J. A. Buckwalter, L. Kleuser, E. Cleveland, W. Schubert, \textit{Our Declaration of Fundamental Beliefs}, „Ministry Magazine”, styczeń 1958.}, czy byłyby konsekwentnie przedrukowywane przez 42 lata\footnote{Szczegółową listę publikacji w tych latach można znaleźć w Załączniku.}, publicznie twierdząc, że przedstawiają streszczenie wiary Adwentystów Dnia Siódmego? Gdyby zostały wydane tylko raz, moglibyśmy uznać to za spisek kilku osób, które celowo błędnie przedstawiały wiarę Adwentystów Dnia Siódmego. Wręcz przeciwnie, \emcap{Fundamentalne Zasady} były regularnie przedrukowywane i naprawdę odzwierciedlały oficjalną wiarę i praktykę Adwentystów Dnia Siódmego.

Innym argumentem jest to, że siostra White aprobowała \emcap{Fundamentalne Zasady} w swoich pismach, wyraźnie się do nich odnosząc, a także nauczając tych samych prawd, które są nauczane w \emcap{Fundamentalnych Zasadach}. Prace naszych pionierów są również zgodne z oświadczeniami w tej Deklaracji \emcap{Fundamentalnych Zasad}. Biorąc pod uwagę wszystkie te fakty, nie da się uniknąć wniosku, że deklaracja ta była zgodna z prawdą w swoich twierdzeniach. Dokument ten był rzeczywiście deklaracją \emcap{Fundamentalnych Zasad}, nauczanych i praktykowanych przez Kościół Adwentystów Dnia Siódmego, stanowiącą publiczne \othersnodot{streszczenie naszej wiary} i \othersnodot{krótkie oświadczenie o tym, co jest i było, z wielką jednomyślnością, wyznawane przez} Adwentystów Dnia Siódmego.\footnote{Przedmowa do Fundamentalnych Zasad, 1872.} Tym samym dokładnie odzwierciedla wierzenia i praktyki Adwentystów Dnia Siódmego oraz stanowi fundament wiary Adwentystów Dnia Siódmego w czasach Ellen White.

Dziś, w obronie doktryny o Trójcy, historycy adwentystyczni śmiało twierdzą, że kiedy nasi pionierzy badali adwentystyczne prawdy, takie jak świątynia, sąd śledczy, szabat i inne doktryny, \othersnodot{nie badali tematu doktryny o Bogu}. Ci historycy adwentystyczni fałszywie twierdzą, że doktryna o Bogu \othersnodot{nie była kwestią, którą zajmowali się w tamtym czasie}[D. Kaiser, „From Antitrinitarianism to Trinitarianism: The Adventist Story” (Od antytrynitaryzmu do trynitaryzmu: Historia Adwentystów) i panelista w: „The God We Worship: A Godhead Symposium” (Bóg, którego czcimy: Sympozjum o Bóstwie). Central California Conference, Dinuba, CA. 23--24 marca 2018.]. Po tym fałszywym twierdzeniu przedstawiają dane historyczne o tym, jak doktryna adwentystyczna stopniowo zmierzała w kierunku trynitarnego zrozumienia. Prawda jest taka, że istnieją pewne wczesne przypadki\footnote{Najwcześniejsza wzmianka o doktrynie o Trójcy w pozytywnym sensie miała miejsce, gdy M. C. Wilcox przedrukował nieadwentystyczny artykuł Samuela Speara w \textit{Signs of the Times} 7 i 14 grudnia 1891.}, kiedy doktryna o Trójcy jest wspominana w pozytywnym świetle w naszej literaturze. Jednak gdy weźmie się pod uwagę fakt, że Kościół Adwentystów miał konkretne stanowisko w kwestii doktryny o Bogu, wyrażone w \emcap{Fundamentalnych zasadach}, tych przypadków nie można interpretować jako postępu w zrozumieniu, ale raczej jako wtargnięcie doktryny o Trójcy do Kościoła Adwentystów Dnia Siódmego.

Łatwo jest obalić twierdzenie, że pionierzy adwentyzmu nie rozumieli doktryny o Bogu. Gdyby jej nie rozumieli, nie udałoby im się głosić poselstwa pierwszego anioła. Omówiliśmy ten punkt szczegółowo w rozdziale „\hyperref[chap:remembering-the-beginning]{Pamiętając początek}”. Ruch Adwentystów Dnia Siódmego nie był porażką, lecz proroczym ruchem prowadzonym przez Boga.

\subsection*{Krok 2: Ignorowanie ostrzeżeń przed budowaniem nowego fundamentu}

Kiedy \emcap{Fundamentalne Zasady} nie są brane pod uwagę, wiele ostrzeżeń Ellen White nie świeci w swoim prawdziwym świetle, a ich prawdziwe znaczenie nie trafia do czytelnika.

Cytowaliśmy wiele wypowiedzi, w których siostra White ostrzegała Kościół, aby nie odchodził od \emcap{Fundamentalnych Zasad}. Zajmowaliśmy się nimi w rozdziale „\hyperref[chap:apostasy]{Wielkie odstępstwo wkrótce się urzeczywistni}”, ale przytoczymy ponownie jeden z najbardziej znaczących cytatów.

\egw{\textbf{Wróg dusz starał się wprowadzić przypuszczenie, że wśród Adwentystów Dnia Siódmego miała nastąpić wielka reforma, i że ta reforma \underline{polegałaby na porzuceniu doktryn, które stoją jako filary naszej wiary}, i zaangażowaniu się w proces reorganizacji}. Gdyby doszło do tej reformy, co by z tego wynikło? \textbf{Zasady prawdy, które Bóg w swojej mądrości dał Kościołowi ostatków, zostałyby odrzucone. Nasza religia zostałaby zmieniona. \underline{Fundamentalne zasady, które podtrzymywały dzieło przez ostatnie pięćdziesiąt lat, uznano by za błąd}}. \textbf{Ustanowiono by nową organizację. Napisano by książki nowego porządku. Wprowadzono by system filozofii intelektualnej}}[Lt242-1903.13; 1903][https://egwwritings.org/read?panels=p7767.20]

\egwnogap{Kto ma upoważnienie do rozpoczęcia takiego ruchu? \textbf{Mamy nasze Biblie. Mamy nasze doświadczenie, potwierdzone cudownym działaniem Ducha Świętego}. \textbf{Mamy prawdę, która nie dopuszcza żadnego kompromisu.} \textbf{\underline{Czy nie powinniśmy odrzucić wszystkiego, co nie jest w harmonii z tą prawdą}?}}[Lt242-1903.14; 1903][https://egwwritings.org/read?panels=p7767.21]

\subsection*{Krok 3: Zaprzeczenie, że osobowość Boga była filarem naszej wiary i częścią fundamentu naszej wiary}

Istnieje jedna wypowiedź Ellen White, która pozornie popiera twierdzenie, że \emcap{osobowość Boga} nie była filarem naszej wiary. Innym wyrażeniem na określenie „\textit{filarów naszej wiary}” jest „\textit{znaki graniczne}”. W poniższych cytatach siostra White wymienia kilka znaków granicznych: oczyszczenie świątyni, poselstwa trzech aniołów, świątynię Boga, szabat oraz nie-nieśmiertelność bezbożnych.

\egw{Upływ czasu w 1844 roku był okresem wielkich wydarzeń otwierających nasze zdumione oczy na \textbf{oczyszczenie świątyni dokonujące się w niebie} i mające wyraźny związek z ludem Bożym na ziemi, [a także] \textbf{poselstwo pierwszego i drugiego anioła oraz trzeciego}, rozwijające sztandar, na którym było napisane: «Przykazania Boże i wiara Jezusa». [Obj 14:12]. Jednym ze znaków granicznych pod tym poselstwem była \textbf{świątynia Boża}, widziana przez Jego miłujący prawdę lud w niebie, i arka zawierająca prawo Boże. Światło \textbf{szabatu} z czwartego przykazania rzucało swoje silne promienie na ścieżkę przestępców prawa Bożego. \textbf{Nie-nieśmiertelność bezbożnych} jest starym znakiem granicznym. \textbf{Nie mogę przypomnieć sobie niczego więcej, co mogłoby należeć do kategorii starych znaków granicznych}. Cały ten krzyk o zmianie starych znaków granicznych jest całkowicie urojony}[Ms13-1889.9; 1889][https://egwwritings.org/read?panels=p4179.14]

Na końcu tej listy znaków granicznych, czyli filarów naszej wiary, stwierdza, że nie może przypomnieć sobie niczego innego, co wchodziłoby do kategorii starych znaków granicznych. Dla wielu ten cytat służy za dowód, że \emcap{osobowość Boga} nie była ani starym znakiem granicznym, ani filarem. To prawda, że w tym cytacie siostra White nie wspomniała wyraźnie o \emcap{osobowości Boga}, ale byłaby ona domyślnie zawarta w poselstwie pierwszego anioła, a także jako podstawowa doktryna poselstwa o świątyni. Ponadto istnieją inne cytaty z siostry White, które wyraźnie wymieniają \emcap{osobowość Boga} jako stary znak lub filar naszej wiary.

\egw{Ci, którzy starają się usunąć \textbf{stare znaki graniczne}, nie trzymają się mocno; \textbf{nie pamiętają, jak otrzymali i usłyszeli}. Ci, którzy próbują \textbf{\underline{wprowadzić} teorie, które usunęłyby \underline{filary naszej wiary}} \textbf{dotyczące świątyni}, \textbf{\underline{lub dotyczące osobowości Boga czy Chrystusa}, działają jak ślepcy}. Starają się wprowadzić niepewność i puścić lud Boży \textbf{bez steru}, niezakotwiczony}[Ms62-1905.14; 1905][https://egwwritings.org/read?panels=p14070.10026020]

Siostra White uczy nas również, że filary naszej wiary stanowią fundament naszej wiary.

\egw{\textbf{Jaki wpływ skłania ludzi na tym etapie naszej historii do działania w podstępny, potężny sposób, \underline{aby zburzyć fundament naszej wiary} — fundament, który został położony na początku naszej pracy poprzez pełne modlitwy studium Słowa i przez objawienie? Na \underline{tym fundamencie} budujemy przez \underline{ostatnie pięćdziesiąt lat}. Czy dziwicie się, że gdy widzę początek działalności, która \underline{dąży do usunięcia niektórych z filarów naszej wiary}, mam coś do powiedzenia? Muszę być posłuszna rozkazowi: «Przeciwstaw się temu!»}}[SpTB02 58.1; 1904][https://egwwritings.org/read?panels=p417.295]

Usunięcie niektórych filarów naszej wiary oznacza zburzenie fundamentu naszej wiary. W innym miejscu siostra White powiedziała, że burzenie lub podważanie fundamentu naszej wiary odbywa się poprzez indoktrynację poglądów dotyczących \emcap{osobowości Boga}.

\egw{Uczelnia została przeniesiona z Battle Creek; jednak studenci nadal są tam wzywani i tam \textbf{są indoktrynowani poglądami dotyczącymi osobowości Boga i Chrystusa, które podkopałyby fundament naszej wiary}}[Lt72-1906.5; 1906][https://egwwritings.org/read?panels=p10013.11]

W świetle tych cytatów widzimy niezbite świadectwo, że \emcap{osobowość Boga} była częścią fundamentu naszej wiary. Ponadto w rozdziale 10 \textit{Special Testimonies}, zatytułowanym „\textit{Fundament naszej wiary}”, siostra White wspomniała o „\textit{Fundamentalnych Zasadach}”, używając synonimów: „\textit{filary naszej wiary}”, „\textit{znaki graniczne}” i „\textit{drogowskazy}”, w odniesieniu do fundamentu naszej wiary.

\subsection*{Krok 4: Zmiana znaczenia terminu „osobowość Boga”}

Termin ‘\textit{osobowość}’ ma dwa różne znaczenia, a najczęstsza definicja w codziennym użyciu dotyczy dziedziny psychologii. ‘\textit{Osobowość}’ jest definiowana jako „\textit{charakterystyczny zestaw zachowań, procesów poznawczych i wzorców emocjonalnych, które ewoluują z czynników biologicznych i środowiskowych}”\footnote{„Personality”, „Wikipedia” [online], \href{https://en.wikipedia.org/wiki/Personality}{en.wikipedia.org/wiki/Personality} [dostęp: 19 maja 2025].}. Niezwykle ważne jest, aby zdać sobie sprawę, że gdy zajmujemy się filarem naszej wiary — „\textit{osobowością Boga}” — nie poruszamy się w dziedzinie psychologii. Dokładne znaczenie słowa ‘\textit{osobowość}’ w doktrynie o \emcap{osobowości Boga} znajduje się w słowniku Merriam-Webster: „\textit{właściwość lub stan jako osoby}”\footnote{„Personality”, „słownik Merriam-Webster” [online], \href{https://www.merriam-webster.com/dictionary/personality}{merriam-webster.com/dictionary/personality} [dostęp: 19 maja 2025].}. Według słownika Merriam-Webster ta definicja jest w użyciu od XV wieku\footnote{Patrz „\href{https://www.merriam-webster.com/dictionary/personality\#word-history}{First known use}” (pierwsze odnotowane użycie) słowa ‘\textit{personality}’ w Słowniku Merriam-Webster.}. W wydaniu Słownika Webstera z 1828 roku czytamy definicję słowa ‘\textit{osobowość}’ jako: „\textit{to, co stanowi jednostkę odrębną osobą}”\footnote{\href{https://archive.org/details/americandictiona02websrich/page/272/mode/2up}{N. Webster, „personality”, \textit{An American Dictionary of the English Language}, S. Converse, New York 1828, t. 2.}} \footnote{\href{https://archive.org/details/websterscomplete00webs/page/974/mode/2up}{Wydanie Słownika Webstera z 1886 roku} definiuje słowo ‘\textit{personality}’ jako: „\textit{to, co stanowi osobę lub jej dotyczy}”.}. Obie definicje znajdują się w słowniku encyklopedycznym autorstwa Roberta Huntera\footnote{\href{https://babel.hathitrust.org/cgi/pt?id=mdp.39015050663213&view=1up&seq=780}{R. Hunter (red.), „personality”, \textit{The American Encyclopædic Dictionary}, W. B. Conkey, Chicago 1895, t. 3, s. 3076.}} — słowniku posiadanym przez Ellen White. Użycie tych definicji można zobaczyć w artykułach napisanych na temat \emcap{osobowości Boga}.

W 1903 roku, kiedy siostra White napisała do dr. Kellogga: \egwinline{\normaltext{[...]} zawsze \textbf{miałam} do złożenia to samo świadectwo, które składam teraz \textbf{odnośnie do osobowości Boga}}[Lt253-1903.9; 1903][https://egwwritings.org/read?panels=p14068.9980015], wspomniała swoją wizję, w której widziała Ojca i Syna.

\egw{Często widziałam kochanego Jezusa, że \textbf{jest osobą}. \textbf{Zapytałam Go, czy Jego Ojciec jest osobą} i \textbf{ma \underline{postać} jak On}. Jezus odpowiedział: «\textbf{Jestem dokładnym obrazem osoby Mojego Ojca!}» [Hbr 1:3]}[Lt253-1903.12; 1903][https://egwwritings.org/read?panels=p14068.9980018]

Właściwość lub stan, który siostra White definiuje, że Bóg jest osobą, to posiadanie \textit{postaci} — \textit{fizycznego wyglądu}. Dr Kellogg podąża za tym samym znaczeniem słowa \textit{‘osobowość’}, choć poprzez spekulację.

\othersnodot{Fakt, że Bóg jest tak wielki, że nie możemy stworzyć wyraźnego mentalnego obrazu \textbf{jego fizycznego wyglądu}, nie musi umniejszać w naszych umysłach rzeczywistości \textbf{Jego osobowości} \normaltext{[...]}}[J. H. Kellogg, \textit{The Living Temple}, Good Health Publishing Company, Battle Creek, Mich. 1903, s. 31.][https://archive.org/details/J.H.Kellogg.TheLivingTemple1903/page/n31/mode/2up].

Jak już widzieliśmy, nasi adwentystyczni pionierzy również wskazywali na fizyczny wygląd jako właściwość, która czyni Boga osobą. James White napisał: \othersnodot{Ci, którzy zaprzeczają \textbf{osobowości Boga}, mówią, że ‘obraz’ nie oznacza tu \textbf{fizycznej postaci}, lecz obraz moralny \normaltext{[...]}}[J. S. White, \textit{Personality of God}, Seventh-day Adventist Publishing Association, Battle Creek, Michigan 1861, s. 1, akap. 1.][https://egwwritings.org/read?panels=p1471.3]. J. B. Frisbie napisał: \othersnodot{Niektórzy wydają się przypuszczać, że przeczy to \textbf{osobowości Boga}, ponieważ jest On Duchem, i mówią, że nie ma On \textbf{ciała ani części}}[\href{https://documents.adventistarchives.org/Periodicals/RH/RH18540307-V05-07.pdf}{J. B. Frisbie, \textit{The Seventh-Day Sabbath Not Abolished}, „The Advent Review, and Sabbath Herald”, 7 marca 1854, s. 50.}].

W świetle tych faktów rozpoznajemy znaczenie słowa ‘\textit{osobowość}’. Kiedy temat \emcap{osobowości Boga} jest przedstawiany w związku z doktryną o Trójcy, istnieje tendencja do zmiany znaczenia słowa ‘\textit{osobowość}’. Ważne jest również, aby wspomnieć, że temat \emcap{osobowości Boga} dotyczy osobowości Ojca. Widać to wyraźnie z przedstawionych informacji.

\subsection*{Krok 5: Przy badaniu kryzysu wokół Kellogga przesunięcie uwagi z osobowości Boga na panteizm}

Informacje na temat kryzysu wokół Kellogga, w związku z doktryną o Trójcy, są przytłaczające, jeżeli weźmie się pod uwagę \emcap{osobowość Boga}. Jedynym sposobem, aby nie połączyć kropek, jest zignorowanie \emcap{osobowości Boga} i przesunięcie uwagi wyłącznie na panteizm. Nie zaprzeczamy panteistycznej naturze sporu wokół Kellogga. Wierzymy, że panteistyczna natura sporu wokół Kellogga nie może być właściwie zrozumiana, jeśli nie bada jej się w prawdziwym świetle \emcap{osobowości Boga}. Niestety jednak przy badaniu kryzysu Kellogga uwaga, jaką poświęca się panteizmowi, przeważa nad badaniem prawdy o \emcap{osobowości Boga}.

Możesz przeszukać kompilacje Ellen White, aby zobaczyć, o ile więcej uwagi poświęcono panteizmowi niż \emcap{osobowości Boga}. Jeśli przeszukałbyś jej pisma pod kątem słów: ‘panteizm’ lub ‘panteistyczny’, wykluczając kompilacje po jej śmierci, znalazłbyś 36 wystąpień. Wśród nich jest kilka powtarzających się cytatów, które siostra White kopiowała z jednego listu do drugiego lub do specjalnych świadectw dla Kościoła. Gdybyś policzył odrębne wystąpienia, znalazłbyś tylko 12 odrębnych cytatów zawierających słowa takie, jak: ‘\textit{panteizm}’ lub ‘\textit{panteistyczny}’\footnote{W pasku wyszukiwania na stronie \href{https://egwwritings.org/}{egwwritings.org} wpisz słowo „\textit{pantheis*}”; obejmie to wszystkie słowa zaczynające się od ‘\textit{pantheis-}’, (w tym ‘\textit{pantheism}’ i ‘\textit{pantheistic}’). Wyniki można porównać, dzieląc korpus pism Ellen White poprzez włączenie lub wykluczenie kompilacji po jej śmierci. Ta opcja jest dostępna w menu rozwijanym pod paskiem wyszukiwania.}. Jeśli przeprowadziłbyś to samo wyszukiwanie, ale tylko w kompilacjach wydanych po jej śmierci, znalazłbyś 140 wystąpień! Wszystkie one należą do jednego z dwunastu odrębnych przypadków, w których siostra White pisała na temat panteizmu.

Przeszukując pisma Ellen White dla frazy „\textit{osobowość Boga}”, z wyłączeniem kompilacji po jej śmierci, znalazłbyś 58 wystąpień. Wśród nich również znajduje się kilka powtarzających się cytatów, które siostra White kopiowała do kilku różnych listów i do świadectw dla Kościoła. Jednak gdybyś poszukał tej frazy w kompilacjach, które zostały wydane po jej śmierci, znalazłbyś tylko 52 wystąpienia.

Te proste statystyki pokazują, na czym skupiali się kompilatorzy po śmierci siostry White. Taki nacisk na panteizm zmienił naszą opinię publiczną dotyczącą kryzysu wokół Kellogga. Czterdzieści trzy z pięćdziesięciu ośmiu cytatów dla frazy „\textit{osobowość Boga}” znajduje się w listach i rękopisach dostępnych publicznie od 2015 roku. Oznacza to, że trzy czwarte (\textit{74\%}) cytatów dotyczących \emcap{osobowości Boga} przed 2015 rokiem nie było dostępnych publicznie. Przed 2015 rokiem nie mieliśmy wielu dostępnych danych, aby badać kryzys wokół Kellogga w świetle \emcap{osobowości Boga} i w jego kontekście.

\begin{titledpoem}

    \stanza{
        Dziś na filary padły cienie, \\
        A prawda idzie w zapomnienie. \\
        Przez kroków pięć w ciszy prowadzą, \\
        Aż rozłam wśród ludu wprowadzą.
    }

    \stanza{
        Fałsz kwitnie tam, gdzie prawda stała, \\
        Którą zrozumieć trzódka miała. \\
        A to, co fundamentem było, \\
        W wyznanie wiary się zmieniło.
    }

    \stanza{
        Prorocze słowa porzucone, \\
        Pionierzy też, pieśni zmienione. \\
        Świadectwa kiedyś tak dźwięczały, \\
        Lecz teraz respekt do nich mały.
    }

    \stanza{
        „Bóg jest osobą” odrzucono, \\
        Jego istotę znieważono. \\
        Prawdziwy filar zapomniany, \\
        A nowy, błędny jest nam dany.
    }

    \stanza{
        Uczeni przekręcają słowa, \\
        By zmieniła znaczenie mowa. \\
        Nikt nie chce widzieć Bożej twarzy, \\
        O Jego uścisku nie marzy.
    }

    \stanza{
        Kellogga kryzys przekręcany, \\
        Krok alfa nie jest zrozumiany. \\
        A że się prawdzie nie dowierza, \\
        To Kościół ku omedze zmierza.
    }

    \stanza{
        Wśród wiernych zamieszanie rządzi \\
        I wielu dziś w wierzeniach błądzi. \\
        Historię naszą przepisano, \\
        Miała być prawda, lecz skłamano.
    }

\end{titledpoem}

