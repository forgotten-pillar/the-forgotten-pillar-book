\chapter{Konstruktywna krytyka}

Pierwszy punkt \emcap{Fundamentalnych Zasad} odpowiada na pytania: kim jest Bóg, jaka jest Jego osobowość i jak rozumiemy Jego obecność?

\others{I. Że jest \textbf{jeden Bóg}, \textbf{osobowa, duchowa }\textbf{\underline{istota}}, \textbf{stwórca wszystkich rzeczy}, wszechmocny, wszechwiedzący i wieczny; nieskończony w mądrości, świętości, sprawiedliwości, dobroci, prawdzie i miłosierdziu; niezmienny i \textbf{wszędzie obecny przez swojego przedstawiciela, Ducha Świętego}. Ps 139:7.}[FP1889 147.2; 1889][https://egwwritings.org/?ref=en\_FP1889.147.2&para=931.6]

Jeden Bóg, Stwórca, jest zidentyfikowany jako Ojciec, ponieważ drugi punkt \emcap{Fundamentalnych Zasad} stwierdza, że Jezus Chrystus, Syn Wiecznego Ojca, jest tym, przez którego Bóg stworzył wszystkie rzeczy\footnote{\href{https://egwwritings.org/?ref=en_FP1889.147.3&para=931.7}{FP1889 147.3; 1889}}. \emcap{Osobowość Boga} jest wyrażona w określeniu „\textit{osobowa duchowa istota}”. Wkrótce zobaczymy, że termin ten oznacza, że Ojciec ma materialne ciało i objawia się w sposób fizyczny. Zatem w swojej osobowości jest obecny tylko tam, gdzie przebywa fizycznie. Ale Jego obecność nie jest ograniczona do Jego osobowości, ponieważ jest \others{wszędzie obecny przez swojego przedstawiciela, Ducha Świętego}. W naszej przeszłości to zrozumienie i rozumowanie \emcap{osobowości Boga}, wyrażone w pierwszym punkcie \emcap{Fundamentalnych Zasad}, spotkało się z konstruktywną krytyką; przez „konstruktywną krytykę” rozumiemy krytykę popartą Biblią.

Przedstawiamy teraz następujące cytaty, pewną konstruktywną krytykę, od prominentnego trynitarnego brata w świecie Adwentystów Dnia Siódmego. Co ciekawe, uznawał on autorytet \emcap{Fundamentalnych Zasad}, a jednocześnie wierzył w doktrynę o Trójcy. Uważamy ten dokument za bardzo ważny element w zmianie naszych wierzeń z fundamentalnych zasad na obecne adwentystyczne trynitarne wierzenie.

Temu prominentnemu bratu zadano pytanie: „\textit{Czy nie wierzysz w osobowego, konkretnego Boga?}”.

\others{\textbf{Jak najbardziej wierzę. Nieskończona, boska, osobowa istota jest kwintesencją religii}. Uwielbienie wymaga kogoś, kogo można kochać, komu można być posłusznym, komu można ufać. \textbf{Wiara w osobowego Boga jest samym rdzeniem religii chrześcijańskiej}. Wyobrażenie Boga jako Wszech-Energii, nieskończonej Mocy, wszędzie sięgającej Obecności jest zbyt rozległe  do pojęcia dla ludzkiego umysłu; musi być coś bardziej \textbf{namacalnego}, bardziej \textbf{\underline{ograniczonego}}, na czym można skupić umysł w uwielbieniu. \textbf{Z tego powodu Chrystus przyszedł do nas na obraz \underline{osobowości} Boga, jako drugi Adam, aby pokazać nam przez swoje życie miłości i poświęcenia charakter i \underline{osobowość Boga}}. Możemy zbliżyć się do Boga tylko przez Chrystusa.}

\othersnogap{«Który, będąc blaskiem jego chwały i \textbf{wyrazem jego istoty} i podtrzymując wszystko słowem swojej mocy, dokonawszy oczyszczenia z naszych grzechów przez samego siebie, zasiadł po prawicy Majestatu na wysokościach».}

\othersnogap{‘Who being the effulgence of his glory, and the impress of his substance, and upholding all things by the word of his power.’}

\othersnogap{‘Który będąc blaskiem Jego chwały i odbiciem Jego istoty, i podtrzymując wszystko słowem swojej mocy.’}

\othersnogap{Apostoł mówi: «Lecz my wszyscy, którzy z odsłoniętą twarzą \textbf{patrzymy} na chwałę Pana, \textbf{jakby w zwierciadle}, zostajemy przemienieni w ten sam obraz, z chwały w chwałę, za sprawą Ducha Pana». 2Kor 3:18. Jakże trafna i piękna jest ta przenośnia! [...] Tak więc, \textbf{patrząc na Chrystusa} w Jego cudach, Jego pokusach, Jego napomnieniach, Jego życiu samozaparcia, Jego «chodzeniu i czynieniu dobrze», \textbf{możemy ujrzeć osobowość i moc Boga}. I jaką wielką nadzieję daje nam fakt, że \textbf{w Chrystusie znajdujemy cechy nie obce i nie cudze dla ludzkości}, ale pokrewne mentalne i moralne charakterystyki; tak że jesteśmy w stanie zobaczyć i uchwycić rzeczywistą, a nie tylko teologiczną czy abstrakcyjną lub figuratywną prawdę w deklaracji apostoła: «Teraz jesteśmy dziećmi Bożymi». 1J 3:2.}

\othersnogap{\textbf{Fakt, że Bóg jest tak wielki, że nie możemy stworzyć jasnego mentalnego obrazu Jego \underline{fizycznego wyglądu}, nie musi umniejszać w naszych umysłach rzeczywistości \underline{Jego osobowości}, ani ta koncepcja nie kłóci się ze szczególnym wyrazem Boga w jakiejś \underline{konkretnej formie lub miejscu}}. \textbf{\underline{Istnieją bowiem fragmenty Pisma, które przedstawiają Boga w tej określonej, można powiedzieć ograniczonej, formie jako siedzącego na tronie w niebie lub mieszkającego w świątyni w Jerozolimie}}. 1Krl 22:19; Ps 11:4; Mt 21:12, 13.}

\othersnogap{Ludzki umysł jest skończony i nie może pojąć nieskończoności. \textbf{Naturalnie pragniemy utworzyć określoną, jasno zdefiniowaną koncepcję istoty, którą czcimy}. \textbf{Biblia zaspokaja tę ludzką potrzebę, jak również wszystkie inne nasze duchowe wymagania, a \underline{w czterdziestym rozdziale Księgi Izajasza} prorok zajmuje się kwestią osobistego wyglądu Boga w cudowny sposób}. «Podnieś mocno swój głos, Jerozolimo, która opowiadasz dobre wieści; podnieś go, nie bój się, powiedz miastom Judy: \textbf{Oto wasz Bóg}. Jak pasterz będzie pasł swoją trzodę, swoim ramieniem zgromadzi baranki, na swoim łonie będzie je nosił».}

\othersnogap{«Kto zmierzył wody swoją \textbf{garścią}, a niebiosa piędzią wymierzył? Kto miarą odmierzył proch ziemi? Kto zważył góry na wadze, a pagórki na szalach? \textbf{Do kogo więc przyrównacie Boga?} \textbf{A jaką podobiznę z nim porównacie?} Czy nie wiecie? Czy nie słyszeliście? Czy wam nie opowiadano od początku? Czy nie zrozumieliście tego od założenia fundamentów ziemi? \textbf{To ten, który zasiada nad okręgiem ziemi} — jej mieszkańcy są jak szarańcza; \textbf{ten, który rozpostarł niebiosa jak zasłonę i rozciągnął je jak namiot mieszkalny}; \textbf{\underline{Do kogo więc mnie przyrównacie, abym był do niego podobny? — mówi Święty}}. Podnieście w górę swoje oczy i patrzcie: Kto stworzył te rzeczy? Ten, kto wyprowadza ich zastępy według liczby i to wszystko po imieniu nazywa, według ogromu jego siły i wielkiej potęgi, tak że ani jedna z nich nie zginie? Czy nie wiesz? Czy nie słyszałeś, że wieczny Bóg, PAN, Stwórca krańców ziemi, nie ustaje ani się nie męczy i że jego mądrość jest niezgłębiona? On dodaje siły spracowanemu i przymnaża mocy temu, który nie ma żadnej siły. Młodzież ustaje i mdleje, a młodzieńcy potykają się i padają; Ale ci, którzy oczekują PANA, nabiorą nowych sił; wzbiją się na skrzydłach jak orły, będą biec, a się nie zmęczą, będą chodzić, a nie ustaną.» Iz 40:9,11,12,18,21,22,25,26,28-31.}

\othersnogap{\textbf{Oto najwspanialszy opis Boga. Wspomniane są Jego dłoń, Jego ramię, Jego łono}. Jest opisany jako «zasiadający nad okręgiem ziemi», mierzy niebiosa piędzią, trzyma wody w swojej garści; \textbf{\underline{nie może więc być wątpliwości, że Bóg jest określoną, rzeczywistą, osobową istotą}}. \textbf{Zwykła abstrakcyjna zasada, prawo czy siła nie mogłyby mieć dłoni, ramienia. \underline{Bóg jest osobą}, choć zbyt wielką byśmy mogli Go pojąć, jak mówi Hiob}: «Oto Bóg jest wielki, a poznać go nie możemy». Hi 36:26...}

\othersnogap{\textbf{\underline{Ta wielka istota} jest przedstawiona jako zasiadająca nad okręgiem ziemi}. Orbita ziemi ma około trzystu milionów kilometrów średnicy. \textbf{Istota tak wielka, by zajmować miejsce o takich proporcjach, jest całkowicie \underline{poza naszym pojmowaniem co do jej formy}}. \textbf{Prorok zauważa to i \underline{odwraca naszą uwagę od spekulacji dotyczących dokładnego rozmiaru i postaci Boga}, pokazując nam absurdalność próby tworzenia nawet mentalnego obrazu, \underline{sugerując, że jest to bliskie bałwochwalstwu}. Zobacz wersety 18-21}. Następnie pokazuje nam, gdzie znaleźć prawdziwą koncepcję Boga, wskazując na rzeczy, które On stworzył: «Podnieście w górę swoje oczy i patrzcie: Kto stworzył te rzeczy?». Taka była też idea Pawła: «To bowiem, co niewidzialne, \textbf{to znaczy jego wieczna moc i \underline{bóstwo}}, są widzialne od stworzenia świata przez to, co stworzone, po to, aby oni byli bez wymówki». Rz 1:20.}

\othersnogap{\textbf{\underline{Dyskusje dotyczące postaci Boga są całkowicie bezużyteczne} i służą jedynie pomniejszaniu naszych wyobrażeń o Tym, który jest ponad wszystkim}, \textbf{i dlatego nie można Go porównywać pod względem postaci, rozmiaru, chwały czy majestatu z czymkolwiek, co człowiek kiedykolwiek widział lub co jest w stanie pojąć}. W obliczu takich pytań możemy jedynie uznać naszą głupotę i nieudolność, i pochylić głowy z bojaźnią i czcią \textbf{w obecności Osobowości, Inteligentnej Istoty}, o której istnieniu cała natura daje określone i niezbite świadectwo, \textbf{ale która jest tak daleko poza naszym pojmowaniem \underline{jak granice przestrzeni i czasu}}.}

Jak wspomniano wcześniej, ten brat uznaje \emcap{Fundamentalne Zasady}, a jednak wierzy w Trójcę. Oto krótkie podsumowanie jego konstruktywnej krytyki dotyczącej \emcap{osobowości Boga}: Bóg jest określoną, rzeczywistą, osobową istotą, posiadającą postać — \others{\textbf{Istnieją bowiem fragmenty Pisma, które przedstawiają Boga w tej \underline{określonej}, można powiedzieć \underline{ograniczonej}, formie jako siedzącego na tronie w niebie}}. Popiera to, ponieważ wierzy, że jest to konieczne dla nas, skończonych istot ludzkich, aby mieć określony przedmiot czci. Ale rozszerza ideę „\textit{ograniczonego}” Boga poprzez świadectwo z 40. rozdziału Księgi Izajasza, które dowodzi, że Bóg jest \others{\textbf{\underline{poza naszym pojmowaniem co do Jego postaci}}}. Każdy rodzaj konceptualizacji istoty Boga, w jakiejkolwiek formie, jest bliski bałwochwalstwu. \others{\textbf{\underline{Dyskusje dotyczące postaci Boga są całkowicie bezużyteczne}}}. Prawdziwa kwestia osobowości nieskończonego Boga jest poza naszym pojmowaniem. Prawdziwa osobowość Boga jest więcej niż tajemnicą dla naszych skończonych umysłów. Jest tak, ponieważ Bóg jest\others{\textbf{tak daleko poza naszym pojmowaniem \underline{jak granice przestrzeni i czasu}}}. Dla tego brata, rozumienie osobowości Boga jedynie jako określonej istoty jest w pewnym sensie prawdziwe, ale w innym fałszywe. To prawda, że Bóg przedstawił się w \others{\textbf{\underline{konkretnej postaci lub miejscu}}}, ponieważ \others{musi być coś bardziej \textbf{namacalnego}, bardziej \textbf{\underline{ograniczonego}}, na czym można skupić umysł w uwielbieniu}. Proste zrozumienie Boga jako określonej i namacalnej istoty jest ograniczające dla Boga. Podsumowaniem jego krytyki jest to, że powinniśmy kształtować nasze wyobrażenia o Bogu poza \others{\textbf{granicami przestrzeni i czasu}}.

Proszę, szczerze zbadaj powody stojące za wiarą tego brata. Zrozumienie toku myślenia stojącego za jego argumentami jest ważne, ponieważ odegrało istotną rolę w historii Kościoła Adwentystów Dnia Siódmego, jako śmiały krok w odchodzeniu od \emcap{Fundamentalnych Zasad}. Te argumenty nie są błahe; są bardzo przekonujące i zachęcamy do ich rozważenia. Być może się z nimi zgodzisz, ale pozwól nam zdemaskować to zwiedzenie. Te cytaty pochodzą z książki Dr. Kellogga \textit{The Living Temple}\footnote{\href{https://archive.org/details/J.H.Kellogg.TheLivingTemple1903}{Dr. J. H. Kellogg, The Living Temple, str. 29-33.}}. Z sekcji zatytułowanej „\textit{Nieskończona Inteligencja jako Osobowa Istota}”, strona 29 do 33, te fragmenty wyrażają stanowisko Kellogga na temat \emcap{osobowości Boga}, które było głównym problemem z jego książką.

To, co właśnie przeczytałeś, było dokładnie tym, do czego odnosiła się siostra White, gdy powiedziała: \egwinline{Mam coś do powiedzenia naszym nauczycielom \textbf{w odniesieniu do nowej książki «The Living Temple»}. \textbf{Bądźcie ostrożni w popieraniu poglądów tej książki \underline{dotyczących osobowości Boga}}. Według tego, jak Pan przedstawia mi te sprawy, \textbf{te poglądy nie mają poparcia Boga}. \textbf{Są one sidłem, które nieprzyjaciel przygotował na te ostatnie dni}...}[Lt211-1903.1; 1903][https://egwwritings.org/?ref=en\_Lt211-1903.1&para=9598.8]

W obecnym sporze w Kościele Adwentystów Dnia Siódmego dotyczącym doktryny o Trójcy, osobiście staraliśmy się przenieść spór z doktryny o Trójcy na \emcap{osobowość Boga}. Przedstawiliśmy stanowisko pierwszego punktu \emcap{Fundamentalnych Zasad} i napotkaliśmy argumenty, które w znacznym stopniu pokrywają się z poglądami Dr. Kellogga na temat \emcap{osobowości Boga}, przedstawionymi w \textit{The Living Temple}. Widzieliśmy to wielokrotnie. Gdy uwaga zostaje przeniesiona z kwestii Trójcy na \emcap{osobowość Boga}, poglądy Kellogga dotyczące \emcap{osobowości Boga} często odbijają się echem z ust zwolenników trynitarianizmu. Właściwość lub stan Boga jako osoby jest tajemnicą w doktrynie o Trójcy, a często poglądy Kellogga na temat \emcap{osobowości Boga} współbrzmią z trynitarnym rozumieniem osoby Boga.

Niektórzy ludzie uważają, że rozumienie osobowości Boga przez dr. Kellogga współgra z ich rozumieniem, choć są kuszeni, by sądzić, że to inne rzeczy są godne sprzeciwu w \textit{The Living Temple}. Poniższe dowody sugerują coś zupełnie przeciwnego. Istnieje list od dr. Kellogga do Williama C. White’a, w którym dr Kellogg proponuje \others{wycięcie kilku stron} z trzech tysięcy egzemplarzy \textit{The Living Temple} — właśnie tych stron zawierających \others{szczególnie kontrowersyjne treści, takie jak komentarz do 40. rozdziału Księgi Izajasza} i poglądy dotyczące \emcap{osobowości Boga} (strony, które przeczytaliśmy).

\others{Sanatorium posiada, jak się dowiedziałem, \textbf{dwa lub trzy tysiące książek, które zostały sprzedane}, ale które wróciły po tym, jak książka została potępiona. Pojawiło się pytanie, co z nimi zrobić. \textbf{Przyszło mi do głowy, że być może można by je uratować, \underline{wycinając kilka stron}, na których pojawiają się \underline{szczególnie kontrowersyjne treści}, takie jak \underline{komentarz do 40. rozdziału Księgi Izajasza}, który zapożyczyłem od A. T. Jonesa, oraz stronę, na której pojawia się niefortunny nagłówek «\underline{Osobowość Boga}», i wklejając strony zawierające jasne przedstawienie biblijnego poglądu na Boga jako osobę, przedstawionego w artykule starszego Haskella w «Review» kilka tygodni temu}. Te książki zostałyby sprzedane dawnym pacjentom, którzy bardzo domagają się tej książki na prezenty świąteczne...}[List od dr. J. H. Kellogga do W. C. White’a; 6 grudnia 1903, Chicago][https://174625.selcdn.ru/ellenwhite/EWhite/17226/17226.pdf]

Jaki jest prawdziwy problem z rozumowaniem w \textit{The Living Temple}? Zbadamy tę sprawę do samego sedna; na pierwszy rzut oka wyraźnie widzimy, że problemem jest odstąpienie od fundamentu naszej wiary — \emcap{Fundamentalnych Zasad} — w odniesieniu do \emcap{osobowości Boga} i tego, gdzie jest Jego obecność.

\egw{\textbf{Zostałam pouczona przez niebiańskiego posłańca}, że część rozumowania w książce «The Living Temple» jest niepoprawna i że \textbf{to rozumowanie sprowadziłoby na manowce} umysły tych, którzy nie są całkowicie utwierdzeni w \textbf{fundamentalnych zasadach} teraźniejszej prawdy. \textbf{Wprowadza to, co jest niczym innym jak spekulacją} w \textbf{odniesieniu do osobowości Boga i tego, gdzie jest Jego obecność}}[SpTB02 51.3; 1904][https://egwwritings.org/?ref=en\_SpTB02.51.3]

Dr Kellogg wprowadził myśl, która \egwinline{jest niczym innym jak spekulacją w odniesieniu do osobowości Boga}, przez co odstąpił od fundamentu naszej wiary — \emcap{Fundamentalnych Zasad}. Niezgodność między nauczaniem dr. Kellogga a \emcap{Fundamentalnymi Zasadami} znajduje się w pierwszym punkcie zasad, gdzie jesteśmy nauczani, że \others{Jest \textbf{jeden Bóg}, \textbf{osobowa, duchowa \underline{istota}}, \textbf{stwórca wszystkich rzeczy}, [...] i \textbf{wszędzie obecny przez swojego przedstawiciela, Ducha Świętego}. Ps 139:7.}

Siostra White bezpośrednio ostrzegała nas przed poglądami wyrażonymi w \textit{The Living Temple} dotyczącymi \emcap{osobowości Boga}. Nie są one zgodne z pierwszym punktem \emcap{Fundamentalnych Zasad}, które były częścią fundamentu naszej wiary.

\egw{\textbf{Musiałam już napisać wiele na temat dziwnych doktryn i teorii wyrażonych w «The Living Temple». \underline{Gdyby te teorie zostały przyjęte przez nasz lud, silne filary naszej wiary i prawdy, które uczyniły Adwentystów Dnia Siódmego tym, czym są, zostałyby zmiecione}. Musiałam pokazać błędność tych doktryn, przedstawiając je \underline{jako rodzaj herezji czasów końca}. Słowo Boże mówi nam, że właśnie takie nauczanie \underline{będzie wprowadzane w tym czasie}}}[Lt250-1903.2; 1903][https://egwwritings.org/?ref=en\_Lt250-1903.2&para=9337.8]

Dziś jesteśmy świadkami powszechnego przyjęcia teorii Kellogga dotyczących \emcap{osobowości Boga}. Fakt, że pierwszy punkt \emcap{Fundamentalnych Zasad} nie jest już obecny w naszych wierzeniach dowodzi, że teorie Kellogga dotyczące \emcap{osobowości Boga} miały wpływ na kształtowanie naszych wierzeń.

\egw{Jeden po drugim przychodzą do mnie, prosząc mnie o \textbf{wyjaśnienie stanowisk zajętych w “Living Temple.”} Odpowiadam: «Są one niewytłumaczalne». \textbf{Wyrażone poglądy nie dają prawdziwego poznania Boga.} \textbf{W całej książce znajdują się fragmenty Pisma Świętego}. \textbf{Te fragmenty Pisma są przedstawione w taki sposób, że \underline{błąd wydaje się prawdą}}. \textbf{Błędne teorie są przedstawione w tak ujmujący sposób, że jeśli nie zachowa się ostrożności, wielu zostanie wprowadzonych w błąd}}[SpTB02 52.1; 1904][https://egwwritings.org/?ref=en\_SpTB02.52.1&para=417.265]

Błąd jest przedstawiany jako prawda i wielu jest wprowadzanych w błąd.

Warto podkreślić, dla niektórych nieuważnych czytelników, że prawdziwy problem dr. Kellogga i jego książki \textit{The Living Temple} nie dotyczy Trójcy, lecz małego kroku, który wykonał, odchodząc od \emcap{fundamentalnych zasad}. Aby zrozumieć prawdziwy problem jego książki, błędem byłoby skupianie się na jej zbieżnych poglądach z doktryną o Trójcy. Zamiast tego powinniśmy skupić się na punkcie, który stanowił ten mały krok wykonany przez niego; a to wymaga głębokiego zrozumienia \emcap{fundamentalnych zasad}, tak jak mieli je nasi pionierzy. Kogo lepiej zapytać niż samych pionierów adwentyzmu?

% Constructive Criticism

\begin{titledpoem}
    
    \stanza{
        A person, God in heav’n, enthroned, \\
        In this our founding truths were zoned. \\
        All-Present by His Spirit’s might, \\
        These truths stood as our guiding light.
    }

    \stanza{
        False words that seemed so wise and deep, \\
        A subtle shift made faithful weep. \\
        "God’s form beyond all thought," they claimed, \\
        This mystery could not be named.
    }

    \stanza{
        "Discussions of God’s form," he said, \\
        "Are futile paths that lie ahead." \\
        Yet this deceit, so smoothly spun, \\
        Was Satan’s snare, and souls were won.
    }

    \stanza{
        The error dressed as truth so fair, \\
        And twisted in a clever snare. \\
        Just One small step from truths we held, \\
        By One giant leap our faith was felled.
    }

    \stanza{
        Beware the mind that seems too wise, \\
        To see deception in disguise. \\
        The truth is—God is personal \\
        This truth the Doctor would conceal.
    }
    
\end{titledpoem}

\begin{titledpoem}

    \stanza{
        Bóg osobowy na tronie zasiada, \\
        Tak nasza Zasada Fundamentalna powiada. \\
        Wszędzie obecny przez Ducha swojego, \\
        To fundament wiary ludu Bożego.
    }

    \stanza{
        Lecz przyszły słowa, co mądre się zdały, \\
        I w nieświadomy umysł głęboko się wlały. \\
        Subtelna zmiana Kelloga, tak gładko przedstawiona, \\
        Była sidłem, w którym dusza została uwięziona.
    }

    \stanza{
        Błąd ubrany w prawdę pięknie się prezentuje, \\
        Gdy Pismo wykręcone zręcznie argumentuje. \\
        Mały krok od Zasad, które wyznawaliśmy, \\
        Wielki skok, przez który wiarę utraciliśmy.
    }

    \stanza{
        Fundamenty wiary stoją niewzruszone, \\
        Choć teorie nowe są wciąż przedstawione. \\
        Bóg Ojciec, osobowa duchowa istota, \\
        To prawda, której broni nasza wiara złota.
    }
    
\end{titledpoem}