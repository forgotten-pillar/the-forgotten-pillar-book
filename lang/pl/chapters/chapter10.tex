\chapter{Is God a person? - by John N. Loughborough}


\chapter{Czy Bóg jest osobą? - autorstwa Johna N. Loughborougha}


One of the earliest articles on the \emcap{personality of God} is Loughborough’s article “\textit{Is God a person?}” where he discusses the \emcap{personality of God} and His presence. It is important to remember the definition of ‘personality’ according to the Merriam-Webster dictionary: “\textit{the quality or state of being a person}”\footnote{\href{https://www.merriam-webster.com/dictionary/personality}{Merriam-Webster Dictionary - ‘\textit{personality}’}}. We will look carefully at how Loughborough sees the quality or state of God being a person.


Jeden z najwcześniejszych artykułów na temat \emcap{osobowości Boga} to artykuł Loughborougha "\textit{Czy Bóg jest osobą?}", w którym omawia \emcap{osobowość Boga} i Jego obecność. Ważne jest, aby pamiętać definicję 'osobowości' według słownika Merriam-Webster: "\textit{właściwość lub stan jako osoby}"\footnote{\href{https://www.merriam-webster.com/dictionary/personality}{Słownik Merriam-Webster - '\textit{personality}'"}}. Przyjrzymy się uważnie, jak Loughborough postrzega właściwość lub stan Boga jako osoby.


\othersnogap{Whatever may be the truth in this matter, it certainly cannot be wrong for us to examine what the Word says respecting it. \textbf{Many there are that would refrain from the investigation of unpopular truths because the cry of heresy is raised against them}. We shall not consider ourselves subjects of the appellation, \textbf{neither are we prying into the secrets of the Almighty, as we pursue the investigation of this matter}. The Bible certainly contains testimony upon this point, and we again repeat, ‘\textbf{Things which are revealed belong to us}.’ We inquire then, What saith the Scripture?}


\othersnogap{Jakakolwiek może być prawda w tej sprawie, z pewnością nie może być błędem zbadanie tego, co mówi Słowo na ten temat. \textbf{Jest wielu takich, którzy powstrzymują się od badania niepopularnych prawd, ponieważ podnosi się przeciwko nim krzyk o herezję}. Nie będziemy uważać się za podmioty tego określenia, \textbf{ani nie wnikamy w tajemnice Wszechmogącego, gdy prowadzimy badanie tej sprawy}. Biblia z pewnością zawiera świadectwo w tej kwestii, i powtarzamy ponownie, '\textbf{Rzeczy, które są objawione, należą do nas}.' Pytamy więc, co mówi Pismo?}


\othersnogap{\textbf{The very testimony we have been examining in regard to man’s being formed of the dust in \underline{the image of God}, proves conclusively that \underline{God has a form}, although the sentiment is contrary to what we have been taught, while children, from the catechism}:}


\othersnogap{\textbf{To samo świadectwo, które badaliśmy odnośnie do stworzenia człowieka z prochu \underline{na obraz Boga}, dowodzi jednoznacznie, że \underline{Bóg ma formę}, chociaż ten pogląd jest sprzeczny z tym, czego uczono nas jako dzieci z katechizmu}:}


\othersnogap{Question. What is God?’}


\othersnogap{Pytanie. Czym jest Bóg?'}


\othersnogap{‘Answer. An infinite and eternal spirit; one that always was and always will be.’}


\othersnogap{'Odpowiedź. Nieskończonym i wiecznym duchem; tym, który zawsze był i zawsze będzie.'}


\othersnogap{‘Q. Where is God?’}


\othersnogap{'P. Gdzie jest Bóg?'}


\othersnogap{‘A. Everywhere.’}


\othersnogap{'O. Wszędzie.'}


\othersnogap{\textbf{But we inquire, \underline{Is not God in one place more than another}?} Oh no, say you: \textbf{the Bible says \underline{he is a spirit}, and if so he must be \underline{everywhere alike}}. Well, if when man dies his spirit goes to God, it must go everywhere. \textbf{But the Bible certainly represents God as located in heaven. ‘For he hath looked down from the height of his sanctuary: from heaven did the Lord behold the earth.’ Psalm 102:19}. \textbf{Then certainly heaven cannot be everywhere, for God is represented as looking down from it. ‘\underline{Elijah went up} by a whirlwind \underline{into heaven}.’ 2 Kings 2:11}. \textbf{But, says one, does not the Bible represent God \underline{as everywhere present}?} Psalm 139:8, 9, 10. ‘If I ascend up into heaven, \textbf{thou art there}: if I make my bed in hell, \textbf{behold, thou art there}; if I take the wings of the morning, and dwell in the uttermost parts of the sea,\textbf{ even there shall thy hand lead me}, and thy right hand shall hold me.’}


\othersnogap{\textbf{Ale pytamy, \underline{czy Bóg nie jest w jednym miejscu bardziej niż w innym}?} Och nie, mówicie: \textbf{Biblia mówi, że \underline{jest duchem}, a jeśli tak, to musi być \underline{wszędzie tak samo}}. Cóż, jeśli gdy człowiek umiera, jego duch idzie do Boga, to musi iść wszędzie. \textbf{Ale Biblia z pewnością przedstawia Boga jako umiejscowionego w niebie. 'Bo spojrzał z wysokości swojej świątyni: Pan popatrzył z nieba na ziemię.' Psalm 102:19}. \textbf{Zatem z pewnością niebo nie może być wszędzie, ponieważ Bóg jest przedstawiony jako spoglądający z niego w dół. '\underline{Eliasz wstąpił} w wichrze \underline{do nieba}.' 2 Królewska 2:11}. \textbf{Ale, ktoś powie, czy Biblia nie przedstawia Boga jako \underline{wszechobecnego}?} Psalm 139:8, 9, 10. 'Jeśli wstąpię do nieba, \textbf{tam jesteś}: jeślibym sobie posłał w piekle, \textbf{oto tam jesteś}; gdybym wziął skrzydła zorzy porannej i zamieszkał na krańcu morza, \textbf{nawet tam poprowadzi mnie twoja ręka} i pochwyci mnie twoja prawica.'}


\othersnogap{We reply, \textbf{the subject is introduced in verse 7, as follows}: ‘\textbf{\underline{Whither shall I go from thy Spirit}?} \textbf{or whither shall I flee from \underline{thy presence}?}’ \textbf{The Spirit is \underline{God’s representative}}. \textbf{His power is manifested wherever he listeth, through the agency of his Spirit}. Christ, when giving the commission to the disciples, says, ‘Go ye into all the world, and preach the gospel to every creature, and lo! \textbf{I am with you alway, even unto the end of the world}.’ Now, no one would contend that Christ had been on the earth personally ever since the disciples commenced to fulfill this commission. \textbf{But his Spirit has been on the earth; the Comforter that he promised to send.} \textbf{So in the same manner God manifests himself \underline{by his Spirit} which is also the power through which he works}. ‘But if \textbf{the Spirit of him} that raised up Jesus from the dead dwell in you, \textbf{he that raised up Christ} from the dead shall also quicken your mortal bodies \textbf{\underline{by his Spirit} that dwelleth in you}.’ Romans 8:11. \textbf{\underline{Here is a plain distinction made between the Spirit, and God that raises the dead by that Spirit}}. \textbf{If the living God is a Spirit in the strictest sense of the term, and at the same time is in possession of a Spirit, then we have at once the novel idea of the Spirit of a Spirit, something it will take at least a Spiritualist to explain}.}[The Adventist Review and Sabbath Herald, September 18, 1855][https://documents.adventistarchives.org/Periodicals/RH/RH18550918-V07-06.pdf]


\othersnogap{Odpowiadamy, \textbf{temat jest wprowadzony w wersecie 7, następująco}: '\textbf{\underline{Dokąd ujdę przed twoim Duchem}?} \textbf{Albo dokąd ucieknę przed \underline{twoją obecnością}?}' \textbf{Duch jest \underline{przedstawicielem Boga}}. \textbf{Jego moc objawia się wszędzie tam, gdzie zechce, poprzez działanie Jego Ducha}. Chrystus, dając uczniom zlecenie, mówi: 'Idźcie na cały świat i głoście ewangelię wszelkiemu stworzeniu, a oto \textbf{ja jestem z wami zawsze, aż do końca świata}.' Teraz, nikt nie będzie twierdził, że Chrystus był osobiście na ziemi przez cały czas od momentu, gdy uczniowie zaczęli wypełniać to zlecenie. \textbf{Ale Jego Duch był na ziemi; Pocieszyciel, którego obiecał posłać.} \textbf{W ten sam sposób Bóg objawia się \underline{przez swojego Ducha}, który jest również mocą, przez którą działa}. 'A jeśli \textbf{Duch tego}, który wskrzesił Jezusa z martwych, mieszka w was, \textbf{ten, który wskrzesił} Chrystusa z martwych, ożywi też wasze śmiertelne ciała \textbf{\underline{przez swojego Ducha}, który w was mieszka}.' Rzymian 8:11. \textbf{\underline{Tutaj jest wyraźne rozróżnienie między Duchem a Bogiem, który wskrzesza umarłych przez tego Ducha}}. \textbf{Jeśli żywy Bóg jest Duchem w najściślejszym znaczeniu tego słowa, a jednocześnie posiada Ducha, to mamy od razu nowatorską ideę Ducha Ducha, coś, co wyjaśnić potrafi co najwyżej spirytualista}.}


Allow us to make a short comment. We hope you recognize the specific topic being discussed here. The subject is the first point of the \emcap{Fundamental Principles} and the assertion is that God does have a form, for man is made in the image of God. Such understanding of God’s personality precludes the idea that God is everywhere present. Brother Loughborough gave the biblical reasons for God's omnipresence, together with the sentiment that “\textit{God is in one place more than another}”. God is everywhere present by His representative, the Holy Spirit, just as it is written in the first point of the \emcap{Fundamental Principles}. Further in this discussion, we will read that God is a spiritual being and possesses a tangible, material body, in contrast to the idea that He is purely a spirit.


Pozwólmy sobie na krótki komentarz. Mamy nadzieję, że rozpoznajecie konkretny temat, który jest tu omawiany. Tematem jest pierwszy punkt \emcap{Fundamentalnych Zasad} i twierdzenie, że Bóg ma formę, ponieważ człowiek został stworzony na obraz Boga. Takie zrozumienie osobowości Boga wyklucza ideę, że Bóg jest wszędzie obecny. Brat Loughborough podał biblijne powody wszechobecności Boga, wraz z poglądem, że "\textit{Bóg jest w jednym miejscu bardziej niż w innym}". Bóg jest wszędzie obecny przez swojego przedstawiciela, Ducha Świętego, tak jak jest napisane w pierwszym punkcie \emcap{Fundamentalnych Zasad}. W dalszej części tej dyskusji przeczytamy, że Bóg jest istotą duchową i posiada namacalne, materialne ciało, w przeciwieństwie do idei, że jest On wyłącznie duchem.


\others{There is at least one impassable difficulty in the way of \textbf{those who believe \underline{God is immaterial}, and heaven is not a literal, \underline{located place}: they are obliged to admit that \underline{Jesus is there bodily, a literal person}}; the same Jesus that was crucified, dead, and buried, was raised from the dead, \textbf{ascended up to heaven}, and is now \textbf{at the right hand of God}. \textbf{Jesus was possessed of flesh and bones after his resurrection}. Luke 24:39. ‘\textbf{Behold my hands and my feet, that it is I, myself; handle me, and see; \underline{for a spirit hath not flesh and bones as ye see me have}}.’ \textbf{If Jesus is there in heaven with a literal body of flesh and bones, may not heaven after all be a literal place, a habitation for a literal God, a literal Saviour, literal angels, and resurrected immortal saints?} \textbf{\underline{Oh no, says one, ‘God is a Spirit.’}} So Christ said to the woman of Samaria at the well. \textbf{It does not necessarily follow because God is a Spirit, \underline{that he has no body}}. In John 3:6, Christ says to Nicodemus, ‘\textbf{That which is born of the Spirit is spirit}.’ \textbf{If that which is born of the Spirit is spirit, then on the same principle, that which has a spiritual nature is spirit. God is \underline{a spirit being}, his nature is spirit, he is not of a mortal nature; }\textbf{\underline{but this does not exclude the idea of his having a body}}. David says, [Psalm 114:4,] ‘Who maketh \textbf{his angels spirits};’ yet \textbf{\underline{angels have bodies}}. Angels appeared to both Abraham and Lot, and ate with them. \textbf{We see the idea that angels are spirits, does not prove that they are not literal beings}.}


\others{Istnieje co najmniej jedna nieprzekraczalna trudność na drodze \textbf{tych, którzy wierzą, że \underline{Bóg jest niematerialny}, a niebo nie jest dosłownym, \underline{zlokalizowanym miejscem}: są zmuszeni przyznać, że \underline{Jezus jest tam cieleśnie, jako dosłowna osoba}}; ten sam Jezus, który został ukrzyżowany, umarł i został pogrzebany, został wskrzeszony z martwych, \textbf{wstąpił do nieba} i jest teraz \textbf{po prawicy Boga}. \textbf{Jezus posiadał ciało i kości po swoim zmartwychwstaniu}. Łukasz 24:39. '\textbf{Spójrzcie na moje ręce i nogi, że to Ja jestem; dotknijcie Mnie i zobaczcie, \underline{bo duch nie ma ciała ani kości, jak widzicie, że Ja mam}}.' \textbf{Jeśli Jezus jest tam w niebie z dosłownym ciałem z ciała i kości, czy niebo nie może być jednak dosłownym miejscem, mieszkaniem dla dosłownego Boga, dosłownego Zbawiciela, dosłownych aniołów i zmartwychwstałych nieśmiertelnych świętych?} \textbf{\underline{O nie, mówi ktoś, 'Bóg jest Duchem.'}} Tak Chrystus powiedział do Samarytanki przy studni. \textbf{Nie wynika z tego koniecznie, że skoro Bóg jest Duchem, \underline{to nie ma ciała}}. W Ew. Jana 3:6 Chrystus mówi do Nikodema: '\textbf{To, co się narodziło z Ducha, jest duchem}.' \textbf{Jeśli to, co narodziło się z Ducha, jest duchem, to na tej samej zasadzie, to co ma duchową naturę, jest duchem. Bóg jest \underline{istotą duchową}, Jego natura jest duchem, nie jest On śmiertelnej natury;} \textbf{\underline{ale to nie wyklucza idei, że ma On ciało}}. Dawid mówi [Psalm 114:4]: 'Który czyni \textbf{swoich aniołów duchami}'; jednak \textbf{\underline{aniołowie mają ciała}}. Aniołowie ukazali się zarówno Abrahamowi, jak i Lotowi, i jedli z nimi. \textbf{Widzimy, że idea, iż aniołowie są duchami, nie dowodzi, że nie są oni dosłownymi istotami}.}


\othersnogap{It is inferred because the Bible says that God is a Spirit, that he is not a person. An inference should not be made the basis for an argument. Great Scripture truths are plainly stated, and it will not do for us to found a doctrine on inferences, \textbf{contrary to positive statements in the word of God}. If the Scripture states in positive \textbf{terms that God is a person, it will not answer for us to draw an inference from the text which says ‘God is a Spirit,’ \underline{that he has no body}}.}


\othersnogap{Wnioskuje się, że ponieważ Biblia mówi, że Bóg jest Duchem, to nie jest On osobą. Wniosek nie powinien być podstawą argumentu. Wielkie prawdy Pisma są jasno określone i nie możemy opierać doktryny na wnioskach, \textbf{sprzecznych z pozytywnymi stwierdzeniami w Słowie Bożym}. Jeśli Pismo stwierdza w pozytywnych \textbf{słowach, że Bóg jest osobą, nie możemy wyciągać wniosku z tekstu, który mówi 'Bóg jest Duchem', \underline{że nie ma On ciała}}.}


\othersnogap{We will now present a few texts \textbf{which prove that God is a person}. Exodus 33:18, 23. ‘And he (Moses) said, I beseech thee shew me thy glory.’ Verse 20. ‘And he said, \textbf{Thou canst not see \underline{my face}, for there shall no man see me and live}.’ Verses 21-23. ‘And the Lord said, Behold there is a place by me, and thou shalt stand upon a rock: and it shall come to pass while my glory passeth by, that I will put thee in a cleft of the rock; and \textbf{will cover thee with \underline{my hand} while I pass by}; and I will take away \textbf{mine hand}, and thou shalt \textbf{see my \underline{back parts}}; but \textbf{\underline{my face} shall not be seen.’} \textbf{If God is \underline{an immaterial Spirit}, then Moses could not see him; for we are told a spirit cannot be seen by natural eyes}. \textbf{There would then be no propriety for God to say he would put his hand over Moses’ face while he passed by, (seemingly to prevent him from seeing his face,) for he could not see him}. Neither do we conceive how an immaterial hand could obstruct the rays of light from passing to Moses’ eyes. \textbf{But if the position be true \underline{that God is immaterial}, and cannot be seen by the natural eye, the text above is all superfluous}. \textbf{What sense is there in saying God put his hand over Moses’ face, to prevent him from seeing that which could not be seen}.}


\othersnogap{Przedstawimy teraz kilka tekstów, które \textbf{dowodzą, że Bóg jest osobą}. Księga Wyjścia 33:18, 23. 'I powiedział (Mojżesz): Proszę, pokaż mi swoją chwałę.' Werset 20. 'I powiedział: \textbf{Nie możesz oglądać \underline{mojego oblicza}, gdyż nie może człowiek oglądać mnie i pozostać przy życiu}.' Wersety 21-23. 'I Pan powiedział: Oto jest miejsce przy mnie i staniesz na skale. I stanie się, gdy będzie przechodzić moja chwała, że postawię cię w rozpadlinie skalnej i \textbf{osłonię cię \underline{moją ręką}, aż przejdę}; a gdy cofnę \textbf{moją rękę}, \textbf{ujrzysz moje \underline{plecy}}, ale \textbf{\underline{mojego oblicza} nie będzie można zobaczyć}.' \textbf{Jeśli Bóg jest \underline{niematerialnym Duchem}, to Mojżesz nie mógłby Go zobaczyć, ponieważ powiedziano nam, że duch nie może być widziany naturalnymi oczami}. \textbf{Nie byłoby wtedy sensu, aby Bóg mówił, że położy swoją rękę na twarzy Mojżesza, gdy będzie przechodził (pozornie, aby uniemożliwić mu zobaczenie Jego oblicza), ponieważ i tak nie mógłby Go zobaczyć}. Nie pojmujemy też, jak niematerialna ręka mogłaby powstrzymać promienie światła przed dotarciem do oczu Mojżesza. \textbf{Ale jeśli stanowisko, \underline{że Bóg jest niematerialny} i nie może być widziany naturalnym okiem, jest prawdziwe, powyższy tekst jest całkowicie zbędny}. \textbf{Jaki jest sens w mówieniu, że Bóg położył swoją rękę na twarzy Mojżesza, aby uniemożliwić mu zobaczenie tego, czego nie można zobaczyć}.}


\othersnogap{Says one, I see we cannot harmonize the matter any other way, than that there was a literal body seen by Moses; but that was not God’s own body, \textbf{it was a body he took that he might show himself to Moses}. \textbf{Moses could form no just conceptions of God unless he assumed a form.} \textbf{So God took a body}. This throws a worse coloring on the matter than the first position; \textbf{for it charges God with deception; telling Moses he should see him, when in fact Moses according to this testimony did not see God, but another body}. A person must be given to doubt almost beyond recovery, that would attempt thus to mystify, and do away the force of this testimony.}[Ibid.][https://documents.adventistarchives.org/Periodicals/RH/RH18550918-V07-06.pdf]


\othersnogap{Ktoś powie: Widzę, że nie możemy zharmonizować tej sprawy w inny sposób, niż że było to dosłowne ciało widziane przez Mojżesza; ale to nie było własne ciało Boga, \textbf{to było ciało, które przyjął, aby pokazać się Mojżeszowi}. \textbf{Mojżesz nie mógłby właściwie pojąć Boga, gdyby Ten nie przyjął formy}. \textbf{Więc Bóg przyjął ciało}. To rzuca gorsze światło na sprawę niż pierwsza pozycja; \textbf{ponieważ oskarża Boga o oszustwo; mówiąc Mojżeszowi, że zobaczy Jego, podczas gdy w rzeczywistości Mojżesz według tego świadectwa nie widział Boga, ale inne ciało}. Człowiek musiałby być prawie beznadziejnie pogrążony w wątpliwościach, aby próbować w ten sposób mistyfikować i niwelować siłę tego świadectwa.}


Do you recognize that Brother Loughborough is tackling the sentiment that Dr. Kellogg would present in the Living Temple 48 years later? Dr. Kellogg said that it is true that God presented Himself in a\others{\textbf{\underline{particular form or place}}}[Dr. John H. Kellogg, The Living Temple, p.31.][https://archive.org/details/J.H.Kellogg.TheLivingTemple1903/page/n31/] because \others{there must be something more \textbf{tangible}, more \textbf{\underline{restricted}}, upon which to center the mind in worship}[bid, p.30][https://archive.org/details/J.H.Kellogg.TheLivingTemple1903/page/n30/], but that He is, in reality,\others{\textbf{far beyond our comprehension \underline{as are the bounds of space and time}}}[Ibid, p.33][https://archive.org/details/J.H.Kellogg.TheLivingTemple1903/page/n33/]. Brother Loughborough reasonably objected to the idea that God is only manifesting Himself to man as a definite Being, but in reality, is not what He presents Himself to be. Such a claim\others{charges God with deception}. Brother Loughborough continues with the affirmative, Biblical testimony that God is a material being.


Czy rozpoznajecie, że Brat Loughborough zajmuje się poglądem, który Dr Kellogg przedstawi w The Living Temple 48 lat później? Dr Kellogg powiedział, że to prawda, że Bóg przedstawił się w\others{\textbf{\underline{konkretnej formie lub miejscu}}} ponieważ \others{musi być coś bardziej \textbf{namacalnego}, bardziej \textbf{\underline{ograniczonego}}, na czym można skupić umysł w uwielbieniu}, ale że w rzeczywistości jest On\others{\textbf{daleko poza naszym zrozumieniem \underline{jak granice przestrzeni i czasu}}}. Brat Loughborough słusznie sprzeciwił się idei, że Bóg tylko manifestuje się człowiekowi jako określona Istota, ale w rzeczywistości nie jest tym, za kogo się podaje. Takie twierdzenie\others{oskarża Boga o oszustwo}. Brat Loughborough kontynuuje z potwierdzającym, biblijnym świadectwem, że Bóg jest istotą materialną.


\others{Exodus 24:9. ‘Then went up Moses and Aaron, Nadab and Abihu, and seventy of the elders of Israel: \textbf{and they saw the God of Israel}: and there was under \textbf{his feet} as it were a paved work of a sapphire stone, and as it were the body of heaven in its clearness.’ They were permitted to \textbf{see his feet}, but no \textbf{man can see his face and live}. \textbf{No \underline{mortal eye} can bear the dazzling brightness of that glory of the face of God}. It far exceeds the light of the sun. For the prophet says, ‘The light of the moon shall be as the light of the sun, and the light of the sun shall be \textbf{seven fold}, as the light of seven days, in the day that the Lord bindeth up the breach of his people, and healeth the stroke of their wound.’ Isaiah 30:26. Notwithstanding this seven-fold light that is then to shine, the prophet speaking of the scene says, ‘Then the moon shall be confounded, and the sun ashamed, when the Lord of hosts shall reign in mount Zion, and in Jerusalem, and before his ancients gloriously.’ Isaiah 24:23. The testimony of John is, [Revelation 21:23,] ‘And the city had no need of the sun, neither of the moon, to shine in it: for \textbf{the glory of God did lighten it,} and the Lamb is the light thereof.’}


\others{Księga Wyjścia 24:9. 'Potem wstąpił Mojżesz i Aaron, Nadab i Abihu oraz siedemdziesięciu starszych Izraela: \textbf{i ujrzeli Boga Izraela}: a pod \textbf{Jego stopami} było jakby wybrukowanie z szafiru, niczym samo niebo w swojej czystości.' Pozwolono im \textbf{zobaczyć Jego stopy}, ale żaden \textbf{człowiek nie może zobaczyć Jego oblicza i żyć}. \textbf{Żadne \underline{śmiertelne oko} nie może znieść olśniewającej jasności tej chwały oblicza Boga}. Znacznie przewyższa ona światło słońca. Prorok bowiem mówi: 'Światło księżyca będzie jak światło słońca, a światło słońca będzie \textbf{siedmiokrotne}, jak światło siedmiu dni, w dniu, gdy Pan opatrzy złamanie swego ludu i uleczy zadaną mu ranę.' Izajasz 30:26. Mimo tego siedmiokrotnego światła, które wtedy będzie świecić, prorok mówiąc o tej scenie mówi: 'Wtedy księżyc się zawstydzi i słońce się zarumieni, gdy Pan Zastępów będzie królował na górze Syjon i w Jeruzalem, i przed swoimi starcami w chwale.' Izajasz 24:23. Świadectwo Jana brzmi [Objawienie 21:23]: 'A miasto nie potrzebuje słońca ani księżyca, aby świeciły w nim, gdyż \textbf{oświetla je chwała Boża}, a jego lampą jest Baranek.'}


\othersnogap{\textbf{Infidels claim that there is a contradiction in the testimony of Moses, because he said, he talked with God face to face}. \textbf{We reply, there was a cloud between them}, but God told Moses, ‘\textbf{No man shall see me and live}.’ The Testimony of the New Testament is in harmony with that of the Old upon this subject. ‘Follow peace with all men, and holiness without which \textbf{no man shall see the Lord}.’ Hebrews 12:14. \textbf{Who with \underline{mortal eyes} could behold a light that far outshines seven fold the brightness of the sun?} Surely none but the holy can behold him, \textbf{none but immortal eyes} could bear that radiant glory. Although the Word says we cannot see God now and live, the promise is, that the \textbf{pure in heart shall see him}. Matthew 5:3. ‘Blessed are the pure in heart, \textbf{for they shall see God}.’ Revelation 22:4. ‘And \textbf{they shall see his face}, and his name shall be in their foreheads.’}


\othersnogap{\textbf{Niewierzący twierdzą, że istnieje sprzeczność w świadectwie Mojżesza, ponieważ powiedział, że rozmawiał z Bogiem twarzą w twarz}. \textbf{Odpowiadamy, że była między nimi chmura}, ale Bóg powiedział Mojżeszowi: '\textbf{Żaden człowiek nie może Mnie zobaczyć i żyć}.' Świadectwo Nowego Testamentu jest zgodne ze Starym w tej kwestii. 'Dążcie do pokoju ze wszystkimi i do świętości, bez której \textbf{nikt nie ujrzy Pana}.' Hebrajczyków 12:14. \textbf{Kto ze \underline{śmiertelnymi oczami} mógłby patrzeć na światło, które siedmiokrotnie przewyższa jasność słońca?} Z pewnością tylko święci mogą Go oglądać, \textbf{tylko nieśmiertelne oczy} mogłyby znieść tę promienną chwałę. Chociaż Słowo mówi, że nie możemy teraz zobaczyć Boga i żyć, obietnica głosi, że \textbf{czystego serca ujrzą Go}. Mateusz 5:3. 'Błogosławieni czystego serca, \textbf{albowiem oni Boga oglądać będą}.' Objawienie 22:4. 'I \textbf{będą oglądać Jego oblicze}, a Jego imię będzie na ich czołach.'}


\othersnogap{Paul, [Colossians 1:15,] speaking of Christ, says, ‘Who is the image of \textbf{the invisible God}, the first born of every creature.’ Here Christ is said to be ‘\textbf{the image of the invisible God}.’ We have already shown, that\textbf{ Christ has a body composed of substance, flesh and bones; and he is said to be}, ‘\textbf{the image of the invisible God}.’ Well, says one, we admit his divine nature is in the image of God. If by his divine nature you mean the part that existed in glory with the Father before the world was, we reply, that which was in the beginning with God, (the Word,) \textbf{was made flesh, not came into flesh}, or as some state, \textbf{clothed upon with a human nature, but made flesh}. But says another, \textbf{God is said to be invisible}. \textbf{Because he is invisible now, it does not prove that he never will be seen}. The Word says, ‘The pure in heart \textbf{shall see him}’. Willing faith says, Amen.}


\othersnogap{Paweł [Kolosan 1:15] mówiąc o Chrystusie, mówi: 'On jest obrazem \textbf{niewidzialnego Boga}, pierworodnym wszelkiego stworzenia.' Tutaj Chrystus jest nazwany '\textbf{obrazem niewidzialnego Boga}.' Już pokazaliśmy, że \textbf{Chrystus ma ciało składające się z substancji, ciała i kości; i jest nazwany} '\textbf{obrazem niewidzialnego Boga}.' No cóż, powie ktoś, przyznajemy, że Jego boska natura jest na obraz Boga. Jeśli przez Jego boską naturę rozumiecie część, która istniała w chwale z Ojcem przed powstaniem świata, odpowiadamy, że to, co było na początku u Boga (Słowo), \textbf{stało się ciałem, nie weszło w ciało}, lub jak niektórzy twierdzą, \textbf{przyodziało się w ludzką naturę, ale stało się ciałem}. Ale mówi ktoś inny, \textbf{Bóg jest nazwany niewidzialnym}. \textbf{To, że jest On teraz niewidzialny, nie dowodzi, że nigdy nie będzie widziany}. Słowo mówi: 'Czystego serca \textbf{ujrzą Go}'. Chętna wiara mówi: Amen.}


\othersnogap{Paul’s testimony in Philippians 2:5, 6, shows plainly what may be understood by the statement, that Christ is the image of God. ‘Let this mind be in you which was in Christ Jesus: who \textbf{being in the form of God}, thought it not robbery to \textbf{be equal with God}.’ \textbf{How can Christ be said to be in the form of God, if God has no form?} Romans 8:3. ‘God sending his own Son in the likeness of sinful flesh.’ \textbf{Christ is in the form of God, and in the form of men. This at once reveals to us the form of God}.}


\othersnogap{Świadectwo Pawła w Liście do Filipian 2:5, 6 jasno pokazuje, co można rozumieć przez stwierdzenie, że Chrystus jest obrazem Boga. 'Niech będzie w was takie nastawienie umysłu, jakie było w Chrystusie Jezusie: który \textbf{będąc w postaci Boga}, nie uważał za grabież \textbf{być równym Bogu}.' \textbf{Jak można powiedzieć, że Chrystus jest w postaci Boga, jeśli Bóg nie ma postaci?} Rzymian 8:3. 'Bóg, posławszy własnego Syna w podobieństwie grzesznego ciała.' \textbf{Chrystus jest w postaci Boga i w postaci człowieka. To od razu objawia nam postać Boga}.}


\othersnogap{\textbf{\underline{Daniel speaking of God, calls him the Ancient of days}}. Daniel 7:9. ‘And the Ancient of days did sit, \textbf{whose garment was white as snow}, and \textbf{the hair of his head} like the pure wool.’ \textbf{This personage is said to have a head, and hair; this certainly could not be said of him} \textbf{\underline{if he was immaterial and had no form}}. \textbf{But Paul’s testimony in \underline{Hebrews 1:3}, ought to settle every candid mind in \underline{regard to the personality of God}}. Speaking of Christ, he says, ‘Who being the brightness of his glory, \textbf{and the express image of his (the \underline{Father’s person})}.’ \textbf{Here then it is plainly stated \underline{God has a person}. Christ is the express image of it.} Then we can understand Christ where he says, ‘\textbf{He that hath seen me, hath seen the Father}.’ John 14:19. \textbf{He could not have meant, that he was his own father; for when he prayed he addressed his Father as another person who had sent him into the world}. He styled himself \textbf{the Son of God}. \textbf{Then he could not be the Father of which he was the son}. When he says, ‘He that hath seen me hath seen the Father,’ he must mean, that as \textbf{he was the express image of the Father’s person, those who saw him saw the likeness of the Father in him}.}[The Adventist Review and Sabbath Herald, September 18, 1855][https://documents.adventistarchives.org/Periodicals/RH/RH18550918-V07-06.pdf]


\othersnogap{\textbf{\underline{Daniel mówiąc o Bogu, nazywa Go Przedwiecznym}}. Daniel 7:9. 'A Przedwieczny zasiadł, \textbf{którego szata była biała jak śnieg}, a \textbf{włosy na Jego głowie} jak czysta wełna.' \textbf{O tej osobie powiedziano, że ma głowę i włosy; z pewnością nie można by tego powiedzieć} \textbf{\underline{gdyby był niematerialny i nie miał formy}}. \textbf{Ale świadectwo Pawła w \underline{Hebrajczyków 1:3} powinno rozstrzygnąć w każdym szczerym umyśle \underline{kwestię osobowości Boga}}. Mówiąc o Chrystusie, mówi: 'Który będąc blaskiem Jego chwały \textbf{i wyrazem Jego (to jest \underline{Ojca}) istoty}.' \textbf{Tutaj jest wyraźnie stwierdzone, że \underline{Bóg ma osobowość}. Chrystus jest jej dokładnym obrazem.} Wtedy możemy zrozumieć Chrystusa, gdy mówi: '\textbf{Kto mnie widział, widział Ojca}.' Jan 14:19. \textbf{Nie mógł mieć na myśli, że był swoim własnym ojcem; bo gdy się modlił, zwracał się do Ojca jako do innej osoby, która posłała go na świat}. Nazywał siebie \textbf{Synem Bożym}. \textbf{Zatem nie mógł być Ojcem, którego był synem}. Kiedy mówi: 'Kto mnie widział, widział Ojca', musi mieć na myśli, że ponieważ \textbf{był dokładnym obrazem osoby Ojca, ci którzy go widzieli, widzieli podobieństwo Ojca w nim}.}[The Adventist Review and Sabbath Herald, September 18, 1855][https://documents.adventistarchives.org/Periodicals/RH/RH18550918-V07-06.pdf]


It is important to pay attention to the biblical evidence that brother Loughborough points out in the testimony that God has a body. Brother Loughborough reviews several Bible passages proving that God does have a material body, but it is invisible to our mortal eyes. Sister White wrote the same when she said\egwinline{\textbf{The Father is all the fulness of the Godhead \underline{bodily}} and \textbf{is invisible to mortal sight}}[Ms21-1906.9; 1906][https://egwwritings.org/?ref=en\_Ms21-1906.9&para=9754.16]. No mortal eye can see the Father, but that does not prove that God can never be seen. Jesus said: \bible{\textbf{He that hath seen me, hath seen the Father}}[John 14:19]. Jesus explained these words two chapters prior: \bible{Jesus cried and said, He that believeth on me, believeth not on me, \textbf{but on him that sent me}. And \textbf{he that seeth me seeth him that sent me}}[John 12:44-45]. Jesus did not send Himself, neither is Jesus the Father, one and the same person; but we see the Father in Christ because He is the \textit{express image of the Father's person}. (Hebrews 1:3). As Jesus is a person, possessing a body, so is the Father. Brother Loughborough continues to prove his point that God is a person, possessing form and shape, because man was created in the image of God.


Ważne jest, aby zwrócić uwagę na biblijne dowody, które brat Loughborough wskazuje w świadectwie, że Bóg ma ciało. Brat Loughborough analizuje kilka fragmentów Biblii dowodzących, że Bóg rzeczywiście ma materialne ciało, ale jest ono niewidzialne dla naszych śmiertelnych oczu. Siostra White napisała to samo, gdy powiedziała\egwinline{\textbf{Ojciec jest pełnią Bóstwa \underline{cieleśnie}} i \textbf{jest niewidzialny dla śmiertelnego wzroku}}[Ms21-1906.9; 1906][https://egwwritings.org/?ref=en\_Ms21-1906.9&para=9754.16]. Żadne śmiertelne oko nie może zobaczyć Ojca, ale to nie dowodzi, że Boga nigdy nie można zobaczyć. Jezus powiedział: \bible{\textbf{Kto mnie widział, widział Ojca}}[Jan 14:19]. Jezus wyjaśnił te słowa dwa rozdziały wcześniej: \bible{Jezus zawołał i powiedział: Kto wierzy we mnie, nie we mnie wierzy, \textbf{ale w tego, który mnie posłał}. A \textbf{kto mnie widzi, widzi tego, który mnie posłał}}[Jan 12:44-45]. Jezus nie posłał samego siebie, ani nie jest Jezus Ojcem, jedną i tą samą osobą; ale widzimy Ojca w Chrystusie, ponieważ On jest \textit{wyrazem istoty Ojca}. (Hebrajczyków 1:3). Jak Jezus jest osobą, posiadającą ciało, tak jest i Ojciec. Brat Loughborough kontynuuje dowodzenie swojego punktu, że Bóg jest osobą, posiadającą formę i kształt, ponieważ człowiek został stworzony na obraz Boga.


\others{But we will now return to the subject of The creation of man. \textbf{We have seen already that man’s being made in the image of God, could not refer to a moral image, for it would involve the absurdity that the lifeless clay of which man was formed, had a character like God}. \textbf{We now see the Scriptures clearly teach, that \underline{God is a person with a body and form}}. Then Genesis 1:26, may be understood to teach the fact, \textbf{that man was made in the form of God}. Other scriptures agree with this testimony. See Genesis 9:6. ‘Whoso sheddeth man’s blood, by man shall his blood be shed: \textbf{for in the image of God made he man}.’ \textbf{\underline{This testimony cannot apply to a spirit, or immaterial part of man: that which is in the image of God has blood}}. 1 Corinthians 11:7. ‘For a man indeed ought not to cover his head, \textbf{forasmuch as he is the image and glory of God}.’ James [Chap 3:9] speaking of the tongue says, ‘Therewith bless we God, even the Father; and therewith curse we men, \textbf{which are made after the similitude (likeness, resemblance – Webster) of God}.’ \textbf{The foregoing testimony settles the point, \underline{that the image of God does not refer to character but to form}}.}


\others{Ale teraz powróćmy do tematu stworzenia człowieka. \textbf{Widzieliśmy już, że stworzenie człowieka na obraz Boga nie mogło odnosić się do obrazu moralnego, ponieważ oznaczałoby to absurd, że martwa glina, z której uformowany został człowiek, miała charakter podobny do Boga}. \textbf{Teraz widzimy, że Pisma wyraźnie uczą, że \underline{Bóg jest osobą z ciałem i formą}}. Zatem Księga Rodzaju 1:26 może być rozumiana jako nauczająca faktu, \textbf{że człowiek został stworzony na podobieństwo Boga}. Inne pisma zgadzają się z tym świadectwem. Zobacz Księgę Rodzaju 9:6. 'Kto przeleje krew człowieka, przez człowieka będzie przelana jego krew: \textbf{bo na obraz Boży uczynił człowieka}.' \textbf{\underline{To świadectwo nie może odnosić się do ducha lub niematerialnej części człowieka: to, co jest na obraz Boga, ma krew}}. 1 Koryntian 11:7. 'Mężczyzna bowiem nie powinien nakrywać głowy, \textbf{gdyż jest obrazem i chwałą Boga}.' Jakub [Rozdz. 3:9] mówiąc o języku mówi: 'Nim błogosławimy Boga i Ojca i nim przeklinamy ludzi, \textbf{którzy są stworzeni na podobieństwo (podobieństwo, podobizna – Webster) Boga}.' \textbf{Powyższe świadectwo rozstrzyga kwestię, \underline{że obraz Boga nie odnosi się do charakteru, ale do formy}}.}


\othersnogap{Genesis 2:7. ‘\textbf{And the Lord God formed man of the dust of the ground, and breathed into his nostrils the breath of life; and man became a living soul}.’}[The Adventist Review and Sabbath Herald, September 18, 1855][https://documents.adventistarchives.org/Periodicals/RH/RH18550918-V07-06.pdf]


\othersnogap{Księga Rodzaju 2:7. '\textbf{I ukształtował Pan Bóg człowieka z prochu ziemi, i tchnął w jego nozdrza dech życia, i stał się człowiek duszą żyjącą}.'}


God formed man in His own image. God is a person, having a body, shape and form, and He formed man into His own image. From this reasoning we derive the obvious meaning of the Scriptures’ testimony about the \emcap{personality of God}. If we make false conceptions regarding God’s person, we are in danger of misunderstanding the other truths which are connected with man’s nature (mortality of the soul, the state of the dead, etc.). In his article, Brother Loughborough continues on to explain the connection between false doctrine on the immortality of the soul and wrong conceptions regarding the \emcap{personality of God}. His article in the Review and Herald from September 18, was taken from his book “\textit{An Examination of the Scripture Testimony}\footnote{\href{https://egwwritings.org/?ref=en_MPC.2&para=961.2}{John Norton Loughborough, An Examination of the Scripture Testimony, 1855}}.


Bóg ukształtował człowieka na swój własny obraz. Bóg jest osobą, mającą ciało, kształt i formę, i ukształtował człowieka na swój własny obraz. Z tego rozumowania wyprowadzamy oczywiste znaczenie świadectwa Pisma o \emcap{osobowości Boga}. Jeśli tworzymy fałszywe koncepcje dotyczące osoby Boga, jesteśmy w niebezpieczeństwie błędnego zrozumienia innych prawd, które są związane z naturą człowieka (śmiertelność duszy, stan umarłych, itd.). W swoim artykule Brat Loughborough kontynuuje wyjaśnianie związku między fałszywą doktryną o nieśmiertelności duszy a błędnymi koncepcjami dotyczącymi \emcap{osobowości Boga}. Jego artykuł w Review and Herald z 18 września został zaczerpnięty z jego książki "\textit{An Examination of the Scripture Testimony}\footnote{\href{https://egwwritings.org/?ref=en_MPC.2&para=961.2}{John Norton Loughborough, An Examination of the Scripture Testimony, 1855}}.
