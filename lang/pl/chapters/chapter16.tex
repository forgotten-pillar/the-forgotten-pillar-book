\qrchapter{https://forgottenpillar.com/rsc/pl-fp-chapter16}{Dr Kellogg i panteizm}

W swoim osobistym dzienniku 5 stycznia 1902 roku siostra White napisała, że \egwinline{nauka Kellogga o Bogu w przyrodzie jest \textbf{prawdziwa}}.

\egw{Przedstawiane są mi rzeczy, które niepokoją mój umysł. Dr Kellogg podąża tą samą drogą, którą szedł wkrótce po objęciu swoich obowiązków w Sanatorium. \textbf{Ludzka nauka jest kłamstwem w odniesieniu do tego, że Bóg nie ma osobowości}. Wiem, że to fałsz, a jednak jeśli możemy w jakikolwiek sposób pomóc doktorowi, musimy spróbować to zrobić. Co można powiedzieć? Jest tak wywyższany, że jest bliski upadku w przepaść. Co którekolwiek z nas może zrobić? Tylko Pan może uratować Dr. Kellogga. \textbf{\underline{Jego nauka o Bogu w przyrodzie jest prawdziwa}}, ale umieścił przyrodę tam, gdzie powinien być Bóg. Przyroda nie jest Bogiem, ale Bóg stworzył przyrodę. \textbf{\underline{Ta nauka o Bogu w przyrodzie jest w pewnym sensie poprawna}}. \textbf{Bóg daje przyrodzie jej życie, jej żywe właściwości, jej piękno}. [On] jest autorem całego piękna przyrody i podczas gdy daje nam ten dowód potężnej mocy, \textbf{jest osobowym Bogiem, a Chrystus jest osobowym Zbawicielem}}[Ms236-1902.1; 1902][https://egwwritings.org/?ref=en\_Ms236-1902.1&para=12779.6]

\egwnogap{\textbf{Przyjmujemy nie ludzkie fałsze, ale Słowo Boże, że człowiek został stworzony na obraz Boga i Chrystusa}, gdyż Słowo oznajmia: «Bóg, który wielokrotnie i na różne sposoby przemawiał niegdyś do ojców przez proroków, w tych ostatecznych dniach przemówił do nas przez swojego syna, którego ustanowił dziedzicem wszystkich rzeczy, \textbf{przez którego także stworzył światy; który, będąc blaskiem jego chwały i \underline{dokładnym obrazem Jego osoby}}, i podtrzymując wszystko słowem swojej mocy, gdy dokonał oczyszczenia z naszych grzechów przez samego siebie, \textbf{zasiadł po prawicy Majestatu nieba}». Hbr 1:1--3}[Ms236-1902.4; 1902][https://egwwritings.org/?ref=en\_Ms236-1902.4&para=12779.9]

Co ciekawe, siostra White również twierdziła, że Bóg jest w przyrodzie i że to On daje życie i żywe właściwości. Kellogg ma rację w tym punkcie i jego twierdzenie jest zdecydowanie poparte jej pismami. Na podstawie tego punktu Kellogg bronił się, mówiąc, że \textit{The Living Temple} jest w zgodzie z pismami siostry White. Napisał do brata G. I. Butlera, gdzie dokładnie siostra White głosiła te same poglądy, co on.

\others{Siostra White zajęła wyraźnie to samo stanowisko w tej sprawie, co ja. Znajdziesz to w jej małym dziele «\textbf{Wychowanie}» w rozdziałach «\textbf{Bóg w przyrodzie}» i «\textbf{Nauka a Biblia}». Znajdziesz to w całym «\textbf{Życiu Jezusa}» i w «\textbf{Patriarchach i Prorokach}».}[List dr. Kellogga do st. Butlera, 21 lutego 1904]

Przyjrzyjmy się rozdziałowi „\textit{Bóg w przyrodzie}” w książce \textit{Wychowanie}, gdzie możemy znaleźć te same poglądy dotyczące Boga w przyrodzie, które promował Kellogg.

\egw{\textbf{Na wszystkich stworzonych rzeczach widać pieczęć Boskości}. Natura świadczy o Bogu. Wrażliwy umysł, stykając się z cudem i tajemnicą wszechświata, nie może nie rozpoznać \textbf{działania nieskończonej mocy}. \textbf{\underline{Nie własną wrodzoną energią} ziemia wydaje swoje plony} i rok po roku kontynuuje swój ruch wokół słońca. \textbf{Niewidzialna ręka prowadzi planety w ich obiegu niebios}. \textbf{\underline{Tajemnicze życie przenika całą naturę — życie, które podtrzymuje niezliczone światy poprzez bezkres}}, \textbf{które żyje w atomie owada unoszącego się w letniej bryzie, które kieruje lotem jaskółki i karmi wołające młode kruki, które sprawia, że pąk zakwita, a kwiat wydaje owoc}}[Ed 99.1; 1903][https://egwwritings.org/?ref=en\_Ed.99.1&para=29.470]

\egwnogap{\textbf{Ta sama \underline{moc}, która podtrzymuje przyrodę, działa również w człowieku}. \textbf{Te same wielkie prawa, które kierują zarówno gwiazdą, jak i atomem, kontrolują ludzkie życie}. \textbf{Prawa, które rządzą działaniem serca, regulując przepływ życiodajnego prądu do ciała, są prawami potężnej Istoty duchowej, która ma władzę nad duszą}. \textbf{\underline{Od Niego pochodzi wszelkie życie}}. Tylko w harmonii z Nim można znaleźć jego prawdziwą sferę działania. Dla wszystkich przedmiotów Jego stworzenia warunek jest ten sam — \textbf{życie podtrzymywane przez przyjmowanie życia Boga}, życie prowadzone w harmonii z wolą Stwórcy...}[Ed 99.2; 1903][https://egwwritings.org/?ref=en\_Ed.99.2&para=29.471]

\egw{...Serce jeszcze niestwardniałe przez kontakt ze złem szybko \textbf{rozpoznaje \underline{Obecność}, która przenika wszystkie stworzone rzeczy}...}[Ed 100.2; 1903][https://egwwritings.org/?ref=en\_Ed.100.2&para=29.475]

W swojej obronie Kellogg odwoływał się również do książki \textit{Patriarchowie i Prorocy}. Czytamy tam, co następuje:

\egw{Wielu naucza, że materia posiada życiową moc — że pewne właściwości są nadane materii, a następnie pozostawiona jest ona, by działała poprzez swoją własną wewnętrzną energię; i że działania natury są prowadzone w harmonii z ustalonymi prawami, w które sam Bóg nie może ingerować. \textbf{To jest fałszywa nauka i nie jest potwierdzona przez słowo Boże}. Przyroda jest sługą swojego Stwórcy. Bóg nie unieważnia swoich praw ani nie działa wbrew nim; \textbf{ale nieustannie używa ich jako swoich narzędzi. Przyroda świadczy o inteligencji, \underline{obecności}, \underline{aktywnej energii}, która działa w jej prawach i poprzez nie. W przyrodzie jest nieustanne działanie \underline{Ojca i Syna}.} Chrystus mówi: «Moj Ojciec działa dotąd i ja działam». J 5:17.}[PP 114.4; 1980][https://egwwritings.org/?ref=en\_PP.114.4&para=84.445]

Te cytaty są w harmonii z cytatami z \textit{The Living Temple}.

\others{Przejawy życia są tak różnorodne jak różne poszczególne zwierzęta i rośliny oraz części istot ożywionych. Każdy liść, każde źdźbło trawy, każdy kwiat, każdy ptak, nawet każdy owad, jak również każde zwierzę czy każde drzewo, świadczą o nieskończonej wszechstronności i niewyczerpalnych zasobach \textbf{jednego wszystko-przenikającego, wszystko-stwarzającego, wszystko-podtrzymującego Życia}.}[John H. Kellogg, The Living Temple, str. 16][https://archive.org/details/J.H.Kellogg.TheLivingTemple1903/page/n15/]

\others{Inteligencja jest jedną z sił wszechświata, jednym z przejawów \textbf{\underline{wszechogarniającego życia, które} stworzyło i tworzy, \underline{ożywia i podtrzymuje}}.}[John H. Kellogg, The Living Temple, str. 396][https://archive.org/details/J.H.Kellogg.TheLivingTemple1903/page/n425/]

Jeśli rozumienie Boga przez Kellogga jako źródła, które podtrzymuje i ożywia przyrodę, jest prawidłowe, to gdzie jest jego błąd? Dlaczego jest nazywany panteistą? Czy sprawiedliwe jest nazywanie go panteistą? On zdecydowanie tak nie uważa. Spójrzmy, co napisał do starszego Butlera:

\others{\textbf{Brzydzę się panteizmem} tak samo, jak Ty. \textbf{W mojej książce starałem się po prostu nauczać faktu, że człowiek jest zależny od Boga we wszystkim i że bez boskiej mocy działającej w nim, Ducha Bożego działającego na pierwiastki, które tworzą jego ciało, byłby prochem}.}[List od dr. Kellogga do st. Butlera, 21 lutego 1904]

\others{Jestem gotów wyrzec się wszystkich strasznych doktryn, które Ty i inni mi przypisujecie. Jestem gotów wyznać, że \textbf{nie jestem panteistą} ani spirytualistą, i że nie wierzę w żadne doktryny nauczane przez tych ludzi ani \textbf{przez panteistyczne czy spirytualistyczne pisma}. Nigdy w życiu nie przeczytałem panteistycznej książki. Nigdy nie czytałem książki o «Nowej Myśli» ani niczego w tym rodzaju. Każdy, kto uważnie przeczyta «The Living Temple» od pierwszej strony prosto do ostatniej i rozważy tę sprawę uczciwie i konsekwentnie, powinien bardzo wyraźnie zobaczyć, że \textbf{nie mam żadnego związku z tymi panteistycznymi i spirytualistycznymi teoriami}.}[Tamże.]

To bardzo trudna zagadka do rozwiązania, dopóki nie napotkamy prawdy o \emcap{osobowości Boga}, którą omówiliśmy na początku tej książki. Tak, Bóg podtrzymuje życie w przyrodzie. W przyrodzie \egwinline{\textbf{rozpoznajemy \underline{Obecność}, która przenika wszystkie stworzone rzeczy}}[Ed 100.2; 1903][https://egwwritings.org/?ref=en\_Ed.100.2&para=29.475]. Ale Bóg \textit{sam} — w swojej osobowości — nie jest w przyrodzie, ani przyroda nie jest Bogiem. Bóg jest \textit{osobową istotą} i jest w swoim świętym przybytku, siedząc na swoim tronie. Bóg jest wszędzie obecny przez swojego \textit{przedstawiciela}, Ducha Świętego.

Kiedy siostra White powiedziała: \egwinline{Ludzka nauka jest kłamstwem w odniesieniu do tego, że Bóg \textbf{nie ma osobowości}}[Ms236-1902; 1902][https://egwwritings.org/?ref=en\_Ms236-1902.1&para=12779.6], odnosiła się szczególnie do tego, że Bóg ma fizyczną postać osoby, co można było zobaczyć w kontekście tego cytatu. Ale kiedy dr. Kellogg odnosił się do ‘\textit{osobowości}’, nie mówił o postaci czy kształcie osoby. W 1936 roku w swoim wykładzie wyraził te same poglądy, które miał w \textit{The Living Temple}, tylko bardziej wyraziście:

\others{Widzicie zatem, że niemożliwe jest pojęcie rzeczy nieskończonych. Są one poza naszym zasięgiem. Są \textbf{poza zrozumieniem} i to samo dotyczy \textbf{\underline{nieskończonej osobowości}}. \textbf{Nie możemy stworzyć żadnego wyobrażenia o jej kształcie, rozmiarze ani żadnych ograniczeniach, ponieważ jest nieskończona}. Być może ta idea jest dla Was trudna do przyjęcia, a \textbf{trudność w zaakceptowaniu tej idei polega na tym, że \underline{nie mamy jasnego pojęcia osobowości}}. \textbf{Myślimy o osobowości \underline{jako związanej z postacią}}.}

\others{...\textbf{Dało mi to nowe pojęcie osobowości}. \textbf{\underline{Osobowość nie oznacza osoby, mężczyzny czy kobiety}}. Wcale nie oznacza tego rodzaju rzeczy. \textbf{Oznacza posiadanie mocy, aby chcieć i czynić, myśleć i planować}.}[\href{https://forgotten-pillar.s3.us-east-2.amazonaws.com/Sanitarium+Lecture+1936.pdf}{Dr. Kellogg, Sanitarium Lectures, 1936}; Transkrypcja na \href{https://notefp.link/1938-kellogg-lecture}{https://notefp.link/1938-kellogg-lecture}]

Takie spojrzenie na osobowość zastosowane do Boga doprowadziło dr. Kellogga do panteizmu. Doktryna o \emcap{osobowości Boga} dotyczy właściwego postrzegania Boga. Postrzeganie Boga przez dr. Kellogga było postrzeganiem trynitarnym.

\others{W «The Living Temple» chciałem wyjaśnić jedynie to, że ta praca, która odbywa się w człowieku, \textbf{nie dzieje się sama z siebie \underline{jak w nakręconym zegarze}; ale jest to moc Boża i \underline{Duch Boży, który ją podtrzymuje}}. \textbf{Zatem myślałem, że całkowicie usunąłem teologiczną stronę kwestii \underline{trójcy i temu podobnych rzeczy}}. \textbf{Nie zamierzałem tego w ogóle umieszczać} i zadbałem o to, aby stwierdzić to w przedmowie. Nigdy nie śniło mi się, że \textbf{coś takiego} jak jakiekolwiek kwestie teologiczne będą \textbf{do tego wprowadzone}. Chciałem tylko pokazać, że \textbf{\underline{serce nie bije samo z siebie}, ale że to \underline{moc Boża utrzymuje je w ruchu}}.}[Interview, J. H. Kellogg, G. W. Amadon and A. C. Bourdeau, October 7th 1907 held at Kellogg's residence][https://archive.org/details/KelloggVs.TheBrethrenHisLastInterviewAsAnAdventistoct71907/page/n37]

Serce nie bije samo z siebie; to moc Boża utrzymuje je w ruchu. W tym Kellogg miał całkowitą rację.

\egw{\textbf{Fizyczny organizm człowieka jest pod nadzorem Boga, ale \underline{nie jest jak zegar, który został wprawiony w ruch i musi działać sam z siebie}}. \textbf{Serce bije, uderzenie następuje po uderzeniu, oddech po oddechu, ale pamiętajmy, że istnienie jest pod nadzorem Boga}. Jesteście Bożą rolą, jesteście Bożą budowlą. \textbf{W Bogu żyjemy i poruszamy się, i mamy nasze istnienie}. \textbf{Każde uderzenie serca, każdy oddech jest tchnieniem tego Boga, który tchnął w nozdrza Adama tchnienie życia}, tchnieniem zawsze obecnego Boga, wielkiego JA JESTEM}[13LtMs, Ms 92, 1898, akap. 7][https://egwwritings.org/read?panels=p14063.7342012&index=0]

Nauka dr. Kellogga o \egwinline{Bogu w przyrodzie jest prawdziwa.}[Ms236-1902; 1902][https://egwwritings.org/?ref=en\_Ms236-1902.1&para=12779.6] Pisma wyraźnie tego nauczają: \bible{Gdyby \normaltext{[Bóg]} zwrócił swe serce ku człowiekowi, \textbf{gdyby zabrał do siebie \underline{jego ducha} i jego tchnienie}, \textbf{\underline{wszelkie ciało by razem zginęło}, a człowiek obróciłby się w proch}.}[Hi 34:14--15] \bible{...twoje sądy są jak wielka głębia: \textbf{O Panie, \underline{zachowujesz} ludzi i zwierzęta}... \textbf{Albowiem u ciebie jest źródło życia}; w twojej światłości będziemy oglądać światłość.}[Ps 36:6b,9]

Te dowody świadczą, że nauka dr. Kellogga o Bogu w przyrodzie jest prawdziwa, ale jego problemem były błędne poglądy na temat osobowości Boga, które były poglądami trynitarnymi. Nawet gdy wyjaśnił, że \others{Bóg Ojciec zasiada na swoim tronie w niebie, gdzie jest także Syn Boży; podczas gdy życie Boże, czyli duch lub obecność jest wszechogarniającą mocą, która wypełnia wolę Bożą w całym wszechświecie}[List: Dr. Kellogg do W. W. Prescotta, 25 października 1903][https://forgotten-pillar.s3.us-east-2.amazonaws.com/1903-10-25-JHKellogg-to-W.W.Prescott.pdf], nadal miał błędne poglądy na temat osobowości Boga — Boga w \others{całościowym sensie} jako \others{Bóstwo... Bóg Ojciec, Bóg Syn i Bóg Duch Święty}[Ibid.][https://forgotten-pillar.s3.us-east-2.amazonaws.com/1903-10-25-JHKellogg-to-W.W.Prescott.pdf]. Jego trynitarny pogląd \textit{nie mógł} \others{wyjaśnić sprawy w zadowalający sposób.}[List: A. G. Daniells do W. C. White’a, 29 października 1903][https://forgotten-pillar.s3.us-east-2.amazonaws.com/Letter-A-G-Daniells-to-W-C-White-October-29-1903.pdf]

Wniosek jest przerażający. Jeśli wierzysz, że serce nie bije samo z siebie, ale że to moc Boża utrzymuje je w ruchu, i łączysz to z przekonaniem, że sam Bóg nie jest namacalną istotą, ale duchem obecnym wszędzie, to w oczach Ducha Proroctwa jesteś panteistą. Postrzeganie właściwości lub stanu Boga jako osoby stanowi różnicę między prawdziwym wierzącym a panteistą.

% Dr. Kellogg i panteizm

\begin{titledpoem}

    \stanza{
        W naturze Bóg swą moc objawia, \\
        Życie każdej istocie nadawia. \\
        Kellogg tę prawdę dobrze znał, \\
        Lecz w błędnym kierunku podążał.
    }

    \stanza{
        Bóg nie jest naturą, choć w niej działa, \\
        Jego osobowość jest doskonała. \\
        Ma kształt i formę, na tronie zasiada, \\
        Przez Ducha Świętego wszędzie włada.
    }

    \stanza{
        Trynitarne poglądy Kellogga zwiodły, \\
        Do panteizmu go doprowadziły. \\
        Choć w naturze Boża moc się przejawia, \\
        Bóg osobowy w niebie przebywa.
    }

\end{titledpoem}

\begin{titledpoem}

    \stanza{
        Serce nie bije własnym ruchem, \\
        Bóg je podtrzymuje swoim duchem. \\
        Lecz nie znaczy to, że Bóg jest wszystkim, \\
        On jest bytem osobowym, nie mistycznym.
    }

    \stanza{
        W przyrodzie widzimy Bożą pieczęć, \\
        Jego mocy i mądrości potęgę. \\
        Jednak Bóg sam ma osobowość jasną, \\
        Nie jest energią bezkształtną, mglistą.
    }

    \stanza{
        Kellogg prawdę z błędem pomieszał, \\
        Gdy o Bogu w naturze pisał. \\
        Bóg jest Stwórcą, nie stworzeniem, \\
        Osobą, nie tylko natchnieniem.
    }

\end{titledpoem}

\begin{titledpoem}

    \stanza{
        Tajemnicze życie przenika naturę, \\
        Bóg podtrzymuje każdą strukturę. \\
        Lecz nie jest On z naturą tożsamy, \\
        Choć Jego obecność wszędzie spotykamy.
    }

    \stanza{
        Osobowość Boga to nie abstrakcja, \\
        To realna, konkretna manifestacja. \\
        Ma formę, kształt i miejsce w niebie, \\
        Przez Ducha działa, lecz nie jest w glebie.
    }

    \stanza{
        Panteizm z prawdą się mija, \\
        Gdy Boga w naturze ukrywa. \\
        Bóg jest ponad swoim stworzeniem, \\
        Osobowym, wiecznym istnieniem.
    }
    
\end{titledpoem}
