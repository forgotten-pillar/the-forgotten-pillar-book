\chapter{Łatane teorie — Lt253-1903}

\egw{Drogi Bracie,\\ \textbf{muszę Ci powiedzieć, że Twoje poglądy w odniesieniu do niektórych kwestii \underline{były i są całkowicie błędne}.} Chciałabym, żebyś mógł dostrzec swoje błędy. \textbf{Książka «The Living Temple» \underline{nie powinna być łatana}, a po wprowadzeniu kilku zmian polecana i wychwalana jako wartościowa publikacja}. Lepiej byłoby przedstawić części dotyczące fizjologii w innej książce pod innym tytułem. \textbf{Kiedy pisałeś tę książkę}, \textbf{nie byłeś pod natchnieniem Bożym}. Był przy Tobie ten, który skłonił Adama do patrzenia na Boga w fałszywym świetle. Twoje całe serce potrzebuje przemiany oraz gruntownego i pełnego oczyszczenia}[Lt253-1903.1; 1903][https://egwwritings.org/?ref=en\_Lt253-1903.1&para=9980.7]

\egwnogap{\textbf{Mój Bracie, nie pozwól się oddzielić od braci w służbie, którzy ostrzegają Cię przed niebezpieczeństwem. Ci, którzy wiernie i szczerze wytykają Ci Twoje błędy, są Twoimi najlepszymi przyjaciółmi.} Jest mi bardzo bardzo przykro z powodu Twoich współpracowników medycznych. Byli oni niewierni Bogu i nieszczerzy wobec Ciebie, nie mówiąc Ci z uprzejmością, ale i  stanowczością, gdzie postępowałeś niesłusznie}[Lt253-1903.2; 1903][https://egwwritings.org/?ref=en\_Lt253-1903.2&para=9980.8]

\egwnogap{Jest wiele rzeczy, które musisz przezwyciężyć, zanim będziesz mógł być zbawiony. W sercu, które nie jest prowadzone przez Boga, jest coś, co prowadzi do pragnienia podążania niewłaściwym kursem. Ludzi, którzy wiernie mówią Ci prawdę, wskazując Twoje błędy, uważasz za swoich wrogów. Ale często są oni Twoimi najlepszymi przyjaciółmi i mówiąc Ci, gdzie błądziłeś, wypełniali bardzo nieprzyjemny obowiązek. Słudzy Pańscy nie mają schlebiać Twojej dumie; nie mają milczeć i bać się powiedzieć: «Dlaczego tak czynisz?». Mają wiernie ostrzegać Cię przed niebezpieczeństwem}[Lt253-1903.3; 1903][https://egwwritings.org/?ref=en\_Lt253-1903.3&para=9980.9]

\egwnogap{\textbf{Mój mąż, starszy Joseph Bates, ojciec Pierce, starszy Edson i wielu innych sumiennych, szlachetnych i prawdziwych, byli wśród tych, którzy po upływie czasu w 1844 roku szukali prawdy}. \textbf{Na naszych ważnych spotkaniach ci mężczyźni zbierali się razem i szukali prawdy jak ukrytego skarbu}. Spotykałam się z nimi i studiowaliśmy, i modliliśmy się żarliwie; czuliśmy bowiem, że musimy poznać Bożą prawdę. Często pozostawaliśmy razem do późnej nocy, a czasem przez całą noc, modląc się o światło i studiując Słowo. Gdy pościliśmy i modliliśmy się, zstępowała na nas wielka moc. Nie mogłam jednak pojąć rozumowania braci. Mój umysł był jakby zamknięty i nie mogłam zrozumieć tego, co studiowaliśmy. Wtedy Duch Boży zstępował na mnie, byłam zabierana w widzeniu i otrzymywałam jasne wyjaśnienie fragmentów, które studiowaliśmy, wraz z pouczeniem co do stanowiska, jakie powinniśmy zająć odnośnie prawdy i obowiązku. Działo się tak wielokrotnie. \textbf{Linia prawdy rozciągająca się od tamtego czasu do momentu, gdy wejdziemy do miasta Bożego, została wyraźnie przede mną nakreślona} i przekazałam braciom i siostrom pouczenie, które dał mi Pan. Wiedzieli oni, że gdy nie byłam w widzeniu, nie mogłam zrozumieć tych spraw, i przyjmowali dane mi objawienia jako światło bezpośrednio z nieba. \textbf{W ten sposób zostały pewnie ustanowione główne punkty naszej wiary, które wyznajemy dzisiaj}. \textbf{\underline{Punkt po punkcie} został jasno zdefiniowany i wszyscy bracia doszli do zgodności}}[Lt253-1903.4; 1903][https://egwwritings.org/?ref=en\_Lt253-1903.4]

\egwnogap{\textbf{Cała społeczność wierzących była zjednoczona w prawdzie}. \textbf{Byli tacy, którzy przychodzili z dziwnymi doktrynami, ale nigdy nie baliśmy się z nimi spotkać. Nasze doświadczenie zostało cudownie utwierdzone przez objawienia Ducha Świętego}.}[Lt253-1903.5; 1903][https://egwwritings.org/?ref=en\_Lt253-1903.5&para=9980.11]

\egwnogap{Przez dwa lub trzy lata mój umysł pozostawał zamknięty na Pisma. W 1846 roku wyszłam za mąż za starszego Jamesa White’a. Po jakimś czasie od narodzin mojego drugiego syna byliśmy w wielkiej rozterce odnośnie do pewnych punktów doktryny. Modliłam się do Pana, aby otworzył mój umysł, bym mogła zrozumieć Jego Słowo. Nagle wydało się, że zostałam otoczona jasnym, pięknym światłem i od tamtej pory \textbf{Pisma są dla mnie otwartą księgą}}[Lt253-1903.6; 1903][https://egwwritings.org/?ref=en\_Lt253-1903.6]

\egwnogap{Byłam wtedy w Paris, w stanie Maine. Stary ojciec Andrews był bardzo chory. Od pewnego czasu bardzo cierpiał z powodu reumatycznego zapalenia stawów. Nie mógł się poruszać bez intensywnego bólu. Modliliśmy się za niego. Położyłam ręce na jego głowie i powiedziałam: «Ojcze Andrews, Pan Jezus Cię uzdrawia». Został uzdrowiony natychmiast. Wstał i chodził po pokoju, chwaląc Boga i mówiąc: «Nigdy wcześniej nie widziałem czegoś takiego. Aniołowie Boży są w tym pokoju». Chwała Boża została objawiona. \textbf{Światło zdawało się świecić w całym domu, a ręka anioła została położona na mojej głowie. Od tamtego czasu do teraz jestem w stanie rozumieć Słowo Boże}}[Lt253-1903.7; 1903][https://egwwritings.org/?ref=en\_Lt253-1903.7&para=9980.13]

\egwnogap{\textbf{Po upływie czasu byliśmy prześladowani i okrutnie oczerniani. Mężczyźni i kobiety, którzy popadli w fanatyzm, zarzucali nam błędne teorie}. Kazano mi udać się w miejsca, gdzie ci ludzie głosili te błędne teorie, a gdy tam szłam, cudownie ukazywała się moc Ducha w napominaniu błędów, które się wkradały. \textbf{\underline{Sam szatan, w osobie człowieka}, działał, aby obrócić wniwecz moje świadectwo dotyczące stanowiska, które teraz, jak wiemy, jest potwierdzone przez Pismo}}[Lt253-1903.8; 1903][https://egwwritings.org/?ref=en\_Lt253-1903.8&para=9980.14]

\egwnogap{\textbf{Dokładnie takie teorie, jakie przedstawiłeś w «The Living Temple», były przedstawiane wtedy}. \textbf{Te subtelne, zwodnicze sofizmaty wielokrotnie próbowały znaleźć miejsce wśród nas. \underline{Lecz zawsze miałam do złożenia to samo świadectwo, które składam teraz odnośnie do osobowości Boga}}}[Lt253-1903.9; 1903][https://egwwritings.org/?ref=en\_Lt253-1903.9&para=9980.15]

\egwnogap{W («Early Writings», str. 60, 66, 67)\footnote{Wydaje się, że podane strony są nieprawidłowe. Wspomniane akapity można znaleźć w \textit{Early Writings} na stronach \href{https://egwwritings.org/read?panels=p28.462&index=0}{70.2}, \href{https://egwwritings.org/read?panels=p28.490&index=0}{77} i \href{https://egwwritings.org/read?panels=p28.390&index=0}{54.2}.} znajdują się następujące stwierdzenia:}[Lt253-1903.10; 1903][https://egwwritings.org/?ref=en\_Lt253-1903.10&para=9980.16]

\egwnogap{14 maja 1851 roku ujrzałam piękno i śliczność Jezusa. Gdy patrzyłam na Jego chwałę, nie przyszło mi do głowy, że kiedykolwiek mogłabym zostać oddzielona od Jego obecności. \textbf{Zobaczyłam światło wychodzące z chwały, która otaczała Ojca}, a gdy zbliżyło się do mnie, moje ciało drżało i trzęsło się jak liść. Myślałam, że jeśli podejdzie bliżej, przestanę istnieć; ale światło mnie minęło. \textbf{Wtedy mogłam w pewnym stopniu pojąć, z jak wielkim i strasznym \underline{Bogiem} mamy do czynienia}}[Lt253-1903.11; 1903][https://egwwritings.org/?ref=en\_Lt253-1903.11&para=9980.17]

\egwnogap{Często widziałam \textbf{ukochanego Jezusa, że jest osobą}. \textbf{Zapytałam Go, czy Jego Ojciec jest osobą i czy ma \underline{postać} podobną do Niego}. Jezus odpowiedział: «\textbf{Jestem wyrazem istoty Mojego Ojca!}» [Hbr 1:3]}[Lt253-1903.12; 1903][https://egwwritings.org/?ref=en\_Lt253-1903.12&para=9980.18]

\egwnogap{\textbf{Często widziałam, że duchowe wyobrażenia odbierały całą chwałę niebu, a w umysłach wielu ludzi tron Dawida i umiłowana osoba Jezusa zostały spalone w ogniu spirytualizmu}. Widziałam, że niektórzy, którzy zostali zwiedzeni i wprowadzeni w ten błąd, zostaną wyprowadzeni na światło prawdy, \textbf{ale będzie dla nich prawie niemożliwe, aby całkowicie uwolnić się od zwodniczej mocy spirytualizmu. Tacy powinni dokładnie wyznać swoje błędy i na zawsze je porzucić}}[Lt253-1903.13; 1903][https://egwwritings.org/?ref=en\_Lt253-1903.13&para=9980.19]

\egwnogap{\textbf{Wśród naszego ludu pojawia się nurt spirytualizmu, który \underline{wkrada się} i \underline{podkopie wiarę} tych, którzy mu ulegną, prowadząc ich do zwracania uwagi na zwodnicze duchy i nauki demonów}. Błędy będą przedstawiane w przyjemny i pochlebny sposób. Wróg pragnie odwrócić umysły naszych braci i sióstr od dzieła przygotowania ludu na te ostatnie dni}[Lt253-1903.14; 1903][https://egwwritings.org/?ref=en\_Lt253-1903.14&para=9980.21]

\egwnogap{Zostałam pouczona, aby ostrzec naszych braci i siostry, \textbf{by nie dyskutowali o naturze naszego Boga}. Wielu ciekawskich, którzy próbowali otworzyć arkę przymierza, aby zobaczyć, co jest w środku, zostało ukaranych za swoją zuchwałość. \textbf{Nie możemy mówić, że Pan Bóg niebios jest w liściu lub w drzewie, ponieważ Go tam nie ma. \underline{On zasiada na swoim tronie w niebie}}}[Lt253-1903.15; 1903][https://egwwritings.org/?ref=en\_Lt253-1903.15&para=9980.22]

\egwnogap{Dzieło Stwórcy widziane w naturze objawia Jego moc. Lecz natura nie jest ponad Bogiem ani Bóg nie jest w naturze, jak niektórzy Go przedstawiają. Bóg stworzył świat, ale świat nie jest Bogiem; jest tylko dziełem Jego rąk. \textbf{Natura objawia dzieło prawdziwego, \underline{osobowego Boga}, pokazując, że Bóg jest i że nagradza tych, którzy go szukają}}[Lt253-1903.16, 1903][https://egwwritings.org/?ref=en\_Lt253-1903.16&para=9980.23]

\egwnogap{Mogłabym wiele powiedzieć o świątyni; o arce zawierającej prawo Boże; o wieku arki, które jest przebłagalnią; o aniołach po obu stronach arki; i o innych rzeczach związanych z niebiańską świątynią i wielkim dniem pojednania. Mogłabym wiele powiedzieć o tajemnicach nieba; ale moje usta są zamknięte. Nie mam ochoty próbować ich opisywać}[Lt253-1903.17; 1903][https://egwwritings.org/?ref=en\_Lt253-1903.17&para=9980.25]

\egwnogap{\textbf{Nie ośmieliłabym się mówić o Bogu tak, jak Ty o Nim mówiłeś}. On jest wysoki i wywyższony, a Jego chwała wypełnia niebiosa. «Głos Pana jest potężny; wstrząsa cedrami Libanu. \textbf{Pan jest w swoim świętym przybytku}. Niech cała ziemia zamilknie przed Nim». [Zobacz Ps 29:5, Ha 2:20]}[Lt253-1903.18; 1903][https://egwwritings.org/?ref=en\_Lt253-1903.18&para=9980.26]

\egwnogap{\textbf{Mój Bracie, gdy jesteś kuszony, aby mówić o Bogu, \underline{gdzie On jest lub czym jest}, pamiętaj, że w tej kwestii milczenie jest krasomówstwem}. Zdejmij buty z nóg; bo ziemia, na której stawiasz swoje niedbałe, nieuświęcone stopy, jest święta}[Lt253-1903.19; 1903][https://egwwritings.org/?ref=en\_Lt253-1903.19]

\egwnogap{\textbf{Jestem pouczona, aby powiedzieć, że w Słowie Bożym nie ma niczego, co potwierdzałoby Twoje spirytualistyczne teorie. Czy nie zechcesz od razu wyrzec się tych teorii? Twój umysł zajmował się nimi przez długi czas, ale nie miały one uświęcającego, oczyszczającego ani uszlachetniającego wpływu na Twoje życie. Pan nie ma zastosowania dla tych teorii i nie chce, aby Jego lud ich bronił lub je propagował}}[Lt253-1903.20; 1903][https://egwwritings.org/?ref=en\_Lt253-1903.20&para=9980.28]

\egwnogap{\textbf{Ojciec, Wszechwiedzący, stworzył świat przez Chrystusa Jezusa}. Chrystus jest światłością świata, drogą do życia wiecznego. Jego, Pomazańca, Bóg dał, aby dokonał przebłagania za grzechy świata. Musisz zrozumieć, że jeśli nie uwierzysz w to przebłaganie i nie będziesz wiedział, że zostałeś kupiony za cenę krwi jednorodzonego Syna Bożego, z pewnością zostaniesz związany ze złym. \textbf{Jeśli nadal będziesz pielęgnować teorie, które dotąd pielęgnowałeś, zostaniesz wydany na pastwę pokus szatana}. On toczy grę o Twoją duszę. Pozostań jeszcze trochę związany z nim, a możesz być pewien, że stracisz swoją duszę}[Lt253-1903.21; 1903][https://egwwritings.org/?ref=en\_Lt253-1903.21&para=9980.29]

\egwnogap{Twierdząc, że nasze instytucje są bezdenominacyjne, postawiłeś naszych ludzi i naszą pracę w fałszywej pozycji. Zostałeś poprowadzony straszną ścieżką, której niebezpieczeństw nie znałeś, ale możesz je kiedyś zobaczyć. Nie jest jeszcze za późno, aby naprawić błędy. Jest dla Ciebie nadzieja. \textbf{Podążałeś za wrogiem krok po kroku, starając się zgłębić tajemnice zbyt wysokie i święte dla Twojego pojmowania}. \textbf{Potem w Twoim nauczaniu Święty został sprowadzony do poziomu ludzkich \underline{naukowych, spirytualistycznych idei}}. Chodziłeś krętymi ścieżkami. Utraciłeś moralny obraz Boga. Jednak jest dla Ciebie nadzieja. Wciąż możesz skierować swoje stopy na właściwą ścieżkę. Czy teraz nie wyprostujesz ścieżek dla swoich stóp, aby to, co chrome, nie zeszło z drogi? Czy teraz odmówisz zasiania choćby jednego więcej ziarna sceptycyzmu i sofistyki w umysłach innych? Czy teraz przyjdziesz do Chrystusa, aby zostać uzdrowiony?}[Lt253-1903.22; 1903][https://egwwritings.org/?ref=en\_Lt253-1903.22]

\egwnogap{\textbf{Wahałam się i zwlekałam z wysłaniem tego, do czego napisania pobudził mnie Duch Pański}. Nie chciałam być zmuszona do przedstawienia szatańskiego wpływu tych sofizmatów. Lecz jeżeli nie nastąpi zdecydowana zmiana w Tobie i Twoich współpracownikach, będę musiała to zrobić, aby uchronić innych przed podążaniem ścieżką, którą Ty podążałeś. Będę musiała być posłuszna rozkazowi danemu mi przez Boga: «\textbf{Przeciwstaw się temu}». To jedyna rzecz, którą mogę zrobić}[Lt253-1903.23; 1903][https://egwwritings.org/?ref=en\_Lt253-1903.23&para=9980.31]

\egwnogap{Przedstawiam Ci rzeczy, które Pan mi przedstawił. Jest wielka praca do wykonania. Musimy podjąć się tej pracy ze zrozumieniem, modląc się, wierząc i przyjmując Ducha Świętego. Tylko w ten sposób możemy wykonać powierzoną nam pracę. \textbf{Bóg wymaga ode mnie, abym świadczyła przeciwko książce «The Living Temple»}. Bez względu na to, co Twoi współpracownicy powiedzą o tej książce, \textbf{zajmuję stanowisko teraz i na zawsze, że jest ona sidłem}. \textbf{Nasz lud jako całość nie zjednoczy się wokół \underline{teorii}, które zacząłeś przedstawiać w tej książce}. \textbf{Możesz uznać to za ostatecznie przesądzone}. \textbf{Jako lud będziemy stać niewzruszenie \underline{na platformie, która wytrzymała próby i doświadczenia}. Będziemy trzymać się \underline{pewnych filarów naszej wiary}. \underline{Zasady prawdy}, które Bóg nam objawił, są naszym jedynym fundamentem. One uczyniły nas tym, kim jesteśmy. Te nowe, fantazyjne teorie są czarujące i zwodnicze. Zagrażają wiecznemu dobru duszy. Pisma ich nie potwierdzają}. Przyodziani w chrześcijańską zbroję, obuci w gotowość ewangelii pokoju, będziemy stać \textbf{niewzruszenie przeciwko tym zwodniczym teoriom}. Możesz przekręcać i wykrzywiać Słowo Boże ku swemu własnemu zatraceniu, ale błagam Cię, nie rób tego}[Lt253-1903.24; 1903][https://egwwritings.org/?ref=en\_Lt253-1903.24&para=9980.32]

\egwnogap{\textbf{Niebo nie jest parą. Jest miejscem}. \textbf{Chrystus poszedł przygotować mieszkania dla tych, którzy Go miłują}, tych, którzy w posłuszeństwie Jego przykazaniom wychodzą ze świata i są oddzieleni. Zasady nieba muszą stać się naszym doświadczeniem, aby można było nas odróżnić od świata. \textbf{Musi być wyraźny kontrast między nami a światem; jesteśmy bowiem ludem nazwanym przez Boga}}[Lt253-1903.25; 1903][https://egwwritings.org/?ref=en\_Lt253-1903.25&para=9980.33]

\egwnogap{Pan dał ci możliwość naprawienia sytuacji. \textbf{Cieszę się, że zacząłeś to robić. Nie myśl, że nie mamy prawa próbować naprawić Twoich błędów i ich skutków. Dopóki Bóg daje mi oddech i zleca mi używanie pióra i głosu do odpierania tego zła, które pojawiło się wśród nas, będę pełnić swoją rolę w tej walce. Od kiedy miałam siedemnaście lat, musiałam toczyć tę bitwę przeciwko fałszywym teoriom w obronie prawdy}. \textbf{Historia naszego przeszłego doświadczenia jest trwale zapisana w moim umyśle i jestem przeświadczona, że \underline{żadne teorie w rodzaju tych, które przyjmowałeś}, nie wejdą w nasze szeregi}. Jeżeli nie będziesz chciał się zmienić i będziesz starał się prowadzić swoich współpracowników za sobą, a oni odważą się podążać za Twoim przewodnictwem, odpowiedzialność spoczywa na Tobie i na nich, nie na mojej duszy}[Lt253-1903.26, 1903][https://egwwritings.org/?ref=en\_Lt253-1903.26&para=9980.34]

\egwnogap{\textbf{Mówię stanowczo, abyś wiedział, że dopóki nie nastąpi w Tobie zdecydowana zmiana, nie może być nadziei na zjednoczenie między Tobą a tymi, którzy trzymają się swoich początkowych wierzeń mocno aż do końca}. To Ty utworzyłeś podział. \textbf{\underline{Musimy stać niewzruszenie przy prawdach, które Pan dał nam jako filary naszej wiary}}}[Lt253-1903.27; 1903][https://egwwritings.org/?ref=en\_Lt253-1903.27&para=9980.35]

\egwnogap{Błagam Cię, zwróć się do Pana z pełnym postanowieniem serca, zanim będzie na zawsze za późno. Oddziel się od wpływów, które oddzieliły Cię od twoich braci zaangażowanych w służbę ewangelii i od ludu, który Bóg prowadzi. \textbf{\underline{Łatane teorie} nie mogą być przyjęte przez lojalnych wobec wiary i \underline{zasad}, które wytrzymały wszelki sprzeciw szatańskich wpływów}}[Lt253-1903.28; 1903][https://egwwritings.org/?ref=en\_Lt253-1903.28&para=9980.36]

\egwnogap{Jeśli opróżnisz się ze wszystkiego, co oddzieliło Cię od Chrystusa, i przyjmiesz Zbawiciela do swojego serca, Twój charakter zostanie przemieniony. Odłóż na jakiś czas obowiązki i udaj się gdzieś z kilkoma swoimi braćmi, i razem z nimi badaj Pisma. Ukórz swoje serce przed Panem i dokonaj gruntownej pracy ku pokucie. \textbf{Religia Chrystusa jest duchowym zaczynem, który ma być wprowadzony do serca. To zmienia życie i charakter}. Ta religia jest niebiańską zasadą, widoczną w życiu i rozmowach chrześcijanina. Objawia się w chrześcijańskiej czystości. Miłość Chrystusa jest widoczna w czułości i łasce uświęconego człowieczeństwa. To przez Słowo, które stało się ciałem, jesteśmy zbawieni. Nasze odkupienie zostało dokonane \textbf{nie przez to, że Syn Boży pozostał w niebie, ale przez to, że Syn Boży stał się wcielony — przyjął na siebie człowieczeństwo i przyszedł na ten świat}. W ten sposób przyniesiono nam życie wieczne. To, czego nie mogły dokonać autorytet, przykazania i obietnice, Bóg uczynił, przychodząc na ten świat w podobieństwie grzesznego ciała}[Lt253-1903.29; 1903][https://egwwritings.org/?ref=en\_Lt253-1903.29&para=9980.37]

\egwnogap{Chrystus przyszedł na ziemię, aby żyć jako człowiek wśród ludzi, nie po to, by zostać zepsutym przez ludzką słabość, ale by umieścić w umysłach ludzi zasady prawdy, które nigdy nie mogą zostać wymazane, ponieważ są wiecznie prawdziwe. Przyszedł, aby przynieść nowe życie upadłym istotom ludzkim — doskonałość, która nie mogła być splamiona ani zniszczona przez grzech}[Lt253-1903.30; 1903][https://egwwritings.org/?ref=en\_Lt253-1903.30&para=9980.38]

\egwnogap{\textbf{Mój Bracie, muszę Ci powiedzieć, że nie zdajesz sobie sprawy, dokąd zmierzają Twoje kroki}. Związałeś się z tymi, którzy należą do armii wielkiego odstępcy. \textbf{Twój umysł stał się ciemny jak Egipt}. \textbf{Jeśli upadniesz na Skałę i zostaniesz skruszony}, Chrystus cię przyjmie.  Stałeś jednak na terenie wroga, wykonując jego pracę. \textbf{Religijny świat szybko podąża tą samą drogą, którą Ty idziesz. Jeśli będziesz dalej nią podążał, będziesz miał mnóstwo towarzystwa. Lecz jaki będzie koniec?}}[Lt253-1903.31; 1903][https://egwwritings.org/?ref=en\_Lt253-1903.31]

\egwnogap{Tak długo chodziłeś w ciemności, tak długo podążałeś własną drogą, że możesz być silnie kuszony, by oprzeć się temu apelowi, który wystosowuję. Gdyby nie to, że chodzi o Twoje \textbf{wieczne dobro}, nie mówiłabym Ci o tej sprawie. Mogłoby się wydawać, że napisałam wystarczająco dużo, że nie ma potrzeby dalszego narzucani Ci tego tematu. \textbf{Lecz mówię Ci zgodnie z prawdą, że doskonale rozumiem, co robię}. Otrzymałeś wystarczające światło. Ale od kilku lat nie zważasz na to światło. Gdybyś chciał wiedzieć, co Pan powiedział, mógłbyś się dowiedzieć; \textbf{masz bowiem książki, które zostały napisane pod przewodnictwem Jego Ducha}. Otrzymałeś wszystkie wskazówki, o jakie można było prosić, aby znaleźć właściwą drogę. Posłano Ci bezpośrednie światło. Lecz uznałeś to za mniej ważne niż swoje własne plany i zamysły. Gdybyś zważał na posłane Ci świadectwa, książka «The Living Temple» nigdy nie zostałaby napisana}[Lt253-1903.32; 1903][https://egwwritings.org/?ref=en\_Lt253-1903.32&para=9980.40]

\egwnogap{Czy nie podejmiesz gruntownego, zdecydowanego, chrystusowego wysiłku, aby przełamać czar, który rzucił na ciebie szatan? Miał dotąd wielką władzę nad Twoim umysłem i kierował Cię na złe drogi. Myśli, że może Cię teraz zatrzymać. Czy nie pokonasz go i nie rozczarujesz?}[Lt253-1903.33; 1903][https://egwwritings.org/?ref=en\_Lt253-1903.33&para=9980.41]

\egwnogap{Piszę do Ciebie jak do syna. Oderwij się od wroga — oskarżyciela braci. Powiedz mu: «Idź precz ode mnie, szatanie. Popełniłem ciężki grzech, słuchając twoich sugestii. Nie będę już ich więcej słuchał». Błagam Cię, dla dobra Twojej duszy, oprzyj się kusicielowi, aby uciekł od Ciebie. Zbliż się do Boga, a On przybliży się do Ciebie. \textbf{Stracisz niebo, jeśli nie upadniesz na Skałę i nie zostaniesz skruszony}}[Lt253-1903.34; 1903][https://egwwritings.org/?ref=en\_Lt253-1903.34&para=9980.42]

Wiele spraw w tym liście do dr. Kellogga pozostaje niewypowiedzianych, ale stają się jasne, gdy zrozumiemy kontekst. Ellen White przeczytała list od brata Daniellsa wyrażający, jak dr Kellogg chciał poprawić \textit{The Living Temple}, ponieważ \others{przemyślał sprawę i zaczął dostrzegać, że popełnił drobny błąd w \textbf{wyrażeniu} swoich poglądów}, i \others{że wkrótce \textbf{zaczął wierzyć w trójcę} i teraz mógł dość jasno dostrzec, gdzie był problem, i wierzył, że może w zadowalający sposób wyjaśnić tę sprawę}. Kellogg wyznał, \others{że teraz wierzy \textbf{w Boga Ojca, Boga Syna i Boga Ducha Świętego}}. W odpowiedzi siostra White osobiście napisała do niego: \egwinline{Książka «The Living Temple» \textbf{nie powinna być łatana}, a po wprowadzeniu kilku zmian polecana i wychwalana jako wartościowa publikacja}. Jak Kellogg chciał załatać swoją książkę? Według świadectwa A. G. Daniellsa myślał o zmianie kilku wyrażeń poprzez wyraźne stwierdzenie swojego trynitarnego poglądu. Ale wyrażenie poglądów nie było prawdziwym problemem — były nim same poglądy. Siostra White nie szczędziła mu nagany za jego poglądy o Bogu, które były poglądami \textit{trynitarnymi}. Powiedziała mu, że jest \egwinline{\textbf{przeświadczona, że }\textbf{\underline{żadne teorie w rodzaju tych, które on przyjmuje}, nie wejdą w nasze szeregi}}. To bardzo mocne stwierdzenie. Czy możliwe, że skoro Kellogg wyznał, iż przyjmuje doktrynę o Trójcy, siostra White włączyła ją również w swoje oświadczenie? Wydaje się to nie do pomyślenia, gdyż ta doktryna jest dziś w naszych szeregach. Ale jej oświadczenie faktycznie wskazuje na Trójcę, gdy powiedziała: \egwinline{\textbf{Łatane teorie} nie mogą być przyjęte przez lojalnych wobec \textbf{wiary i zasad}, które wytrzymały wszelki sprzeciw szatańskich wpływów}. Kellogg chciał załatać \textit{The Living Temple} przez wyraźne nadmienienie doktryny o Trójcy. Dlaczego siostra White była zdecydowana, by trzymać tę doktrynę z dala od naszych szeregów, choć jednak jest ona dziś w naszych szeregach? Uczciwie należy zauważyć, że Trójca nie była częścią wiary Adwentystów Dnia Siódmego w jej czasach i weszła do naszych szeregów później. Dziś wielu twierdzi, że to dzięki jej pracom Trójca jest częścią naszych wierzeń, ale reakcja Ellen White i jej odpowiedź na wiarę Kellogga w nią pokazuje, jak ona traktowała taką doktrynę. Czego możemy się z tego nauczyć?

Rozważany w swoim kontekście, ten list rzuca nowe światło na kontrowersję Kellogga i pokazuje, jak powinniśmy podchodzić do doktryny o Trójcy. Pierwszą rzeczą, którą kwestionujemy, jest to, dlaczego siostra White nigdy nie użyła słowa „Trójca” w swoich pismach, nawet gdy bezpośrednio odnosiła się do tej doktryny? W innym miejscu odpowiada:

\egw{Zostałam przestrzeżona, aby nie wchodzić w spory \textbf{dotyczące kwestii}, które \textbf{\underline{pojawią się}} w związku z \textbf{tymi sprawami, ponieważ spory \underline{mogłyby doprowadzić ludzi do uciekania się do matactwa, a ich umysły zostałyby odwiedzione od prawdy Słowa Bożego do przypuszczeń i domysłów}}. \textbf{Im więcej dyskutuje się o wymyślnych teoriach, tym \underline{mniej ludzie będą wiedzieć o Bogu i o prawdzie, która uświęca duszę}}}[Lt232-1903.41; 1903][https://egwwritings.org/?ref=en\_Lt323-1903.41]

To bardzo ważna lekcja i zasada, której uczy nas tutaj siostra White. Kiedy powstała kontrowersja wokół teorii Kellogga, nie zagłębiała się w same teorie, ponieważ to prowadziłoby umysły ludzi od prawdy Słowa Bożego do przypuszczeń i domysłów. Zamiast tego, kierowała umysły ludzi ku prawdzie, która uświęca duszę. Prowadziła przykładem, co widać tutaj w jej liście do dr. Kellogga. Ta prawda, do której kierowała umysły ludzi, była prawdą o \emcap{osobowości Boga}. Zganiła Kellogga za jego teorie, ale, co bardzo ważne, właściwie identyfikujemy te teorie poprzez ich kontekst i jej domyślne ich wyrażenie.

Widzimy, że pokazała kontrast między Trójcą a \emcap{osobowością Boga}. Pokazała kontrast między starymi zasadami naszej wiary a nowymi teoriami. Najpierw skierowała nasze umysły z powrotem do początku naszego duchowego dziedzictwa, \egwinline{po upływie czasu w 1844 roku}, gdy jej mąż James White, Joseph Bates, ojciec Pierce, starszy Edson i wielu innych, którzy byli sumienni, szlachetni i prawdziwi, szukali prawdy. Wskazała na wspaniałe i potężne doświadczenia, jak główne punkty naszej wiary, wyznawane w 1903 roku, zostały pewnie ustanowione.\egwinline{\textbf{W ten sposób }\textbf{\underline{główne punkty naszej wiary}}, jakie wyznajemy dzisiaj, zostały mocno ustanowione.} \egwinline{\textbf{\underline{Punkt po punkcie}}\textbf{ został jasno zdefiniowany i wszyscy bracia doszli do zgodności}.} \egwinline{\textbf{Cała społeczność wierzących była zjednoczona w prawdzie}}. Oczywiście z kontekstu 10. rozdziału \textit{Special Testimonies} wiemy, że te doświadczenia wyjaśniają, \egwinline{\textbf{jak mocno został położony fundament naszej wiary}}[SpTB02 56.4; 1904][https://egwwritings.org/?ref=en\_SpTB02.56.4\&para=417.288]. Ten fundament jest wyrażony w \emcap{Fundamentalnych Zasadach}\footnote{\href{https://static1.squarespace.com/static/554c4998e4b04e89ea0c4073/t/59d17e24c027d84167e17617/1506901547915/SDA-YB1905+\%28P.+188-192\%29.pdf}{Yearbook Of Seventh-day Adventist denomination 1905, str. 188-192}}. Ten fundament jest prawdą, która \egwinline{\textbf{\underline{punkt po punkcie} została odkryta przez pełne modlitwy studium i potwierdzona przez cudowną moc Pana}}. Bóg \egwinline{\textbf{wzywa nas, abyśmy \underline{mocno trzymali się}, z uściskiem wiary, \underline{fundamentalnych zasad}, które są \underline{oparte na niepodważalnym autorytecie}}}[SpTB02 59.1; 1904][https://egwwritings.org/?ref=en\_SpTB02.59.1]. W świetle tych doświadczeń i prawdy wyrażonej w \emcap{fundamentalnych zasadach} \egwinline{\textbf{\underline{łatane teorie} nie mogą być przyjęte przez lojalnych wobec  \underline{wiary} i \underline{zasad}, które wytrzymały wszelki sprzeciw szatańskich wpływów}}[Lt253-1903.28; 1903][https://egwwritings.org/?ref=en\_Lt253-1903.28]. Z historycznego zapisu tych braci, którzy byli sumienni, szlachetni i prawdziwi, mamy dowody, że oni również przeciwstawiali doktrynę o Trójcy prawdzie o \emcap{osobowości Boga}. James White w artykule w \textit{Review and Herald} wymienił \others{niektóre z popularnych bajek epoki}, mówiąc: \others{Tutaj możemy nadmienić \textbf{Trójcę, która \underline{pozbywa się osobowości Boga i Jego Syna Jezusa Chrystusa}}}[James White, Review \& Herald, 11 grudnia 1855, str. 85.15][http://documents.adventistarchives.org/Periodicals/RH/RH18551211-V07-11.pdf]. J. N. Andrews powiedział: \others{\textbf{Doktryna Trójcy, która została ustanowiona w Kościele przez sobór nicejski w 325 r. n.e. Ta doktryna \underline{niszczy osobowość Boga i Jego Syna Jezusa Chrystusa, naszego Pana}}...}. J. B. Frisbie w swoim artykule „\textit{Szabat dnia siódmego nie zniesiony}” porównuje Boga szabatu z bogiem niedzieli; opisuje Boga szabatu w świetle \emcap{osobowości Boga} wyrażonej w pierwszym punkcie \emcap{Fundamentalnych Zasad}. Bóg niedzieli jest opisany jako \others{jedność tego Bóstwa, są trzy osoby jednej substancji, mocy i wieczności; Ojciec, Syn i Duch Święty}[J. B. Frisbie, Review \& Herald, 7 marca 1854, str. 50][http://documents.adventistarchives.org/Periodicals/RH/RH18540307-V05-07.pdf]. Wyjaśnił, jak doktryna o \emcap{osobowości Boga} stoi w konflikcie z doktryną o Trójcy, w ten sam sposób jak Święty Szabat stoi w konflikcie z pogańskim kultem niedzieli. Również brat J. N. Loughborough spisał zastrzeżenia do doktryny o Trójcy w \textit{Adventist Review and Sabbath Herald}\footnote{\href{https://adventistdigitallibrary.org/adl-349160/advent-review-and-sabbath-herald-november-5-1861}{J. N. Loughborough, 5 października 1861, Review \& Herald, tom 18, str. 184, akap. 1-11}}. W innym wydaniu \textit{Review and Herald} opublikował artykuł „\textit{Czy Bóg jest osobą?}”, wyjaśniając stanowisko wiary Adwentystów Dnia Siódmego w sprawie \emcap{osobowości Boga}, wyrażone w pierwszym punkcie \emcap{Fundamentalnych Zasad}\footnote{\href{http://documents.adventistarchives.org/Periodicals/RH/RH18550918-V07-06.pdf}{J. N. Loughborough, 18 września 1855, Review \& Herald, tom 7, str. 6.}}. James White również wyjaśniał to samo stanowisko w swojej wielokrotnie drukowanej broszurze „\textit{Osobowość Boga}”\footnote{\href{https://egwwritings.org/?ref=en_PERGO.1.1&para=1471.3}{J. White, The Personality of God, 18 czerwca 1861.}}. To tylko kilka przykładów, gdzie pionierzy adwentyzmu wyjaśniali stanowisko w sprawie \emcap{osobowości Boga} wyrażone w pierwszym punkcie \emcap{fundamentalnych zasad}.

Siostra White zganiła Kellogga: \egwinline{\textbf{Lecz mówię Ci zgodnie z prawdą, że doskonale rozumiem, co robię}. \textbf{Otrzymałeś wystarczające światło}. Ale od kilku lat nie zważasz na to światło. Gdybyś chciał wiedzieć, co Pan powiedział, mógłbyś się dowiedzieć; \textbf{\underline{masz bowiem książki}, które zostały napisane pod przewodnictwem Jego Ducha}. Otrzymałeś wszystkie wskazówki, o jakie można było prosić, aby znaleźć właściwą drogę. Posłano Ci bezpośrednie światło. Lecz uznałeś to za mniej ważne niż swoje własne plany i zamysły. Gdybyś zważał na posłane Ci świadectwa, książka «The Living Temple» nigdy nie zostałaby napisana.}[Lt253-1903.32; 1903][https://egwwritings.org/?ref=en\_Lt253-1903.32] Centralnym problemem kontrowersji dr. Kellogga była \egwinline{osobowość Boga i to, gdzie jest jego obecność}[SpTB02 51.3; 1904][https://egwwritings.org/?ref=en\_SpTB02.51.3&para=417.262]. Dr. Kellogg miał dostęp do pism pionierów, książek oraz kościelnych \emcap{Fundamentalnych Zasad}, które były potwierdzone cudotwórczą mocą Ducha Świętego.

Siostra White przypomniała doświadczenia, jak \textit{główne punkty naszej wiary}, które były wyznawane w dawnych czasach, zostały pewnie ustanowione.\egwinline{\textbf{\underline{Punkt po punkcie}} \textbf{został jasno zdefiniowany i wszyscy bracia doszli do zgodności}}[Lt253-1903.4; 1903][https://egwwritings.org/?ref=en\_Lt253-1903.4]. Te główne punkty były \emcap{Fundamentalnymi Zasadami}, a \emcap{osobowość Boga} była jedną z nich. Ten punkt i świadectwo siostry White o nim pozostały takie same przez całe jej życie. Powiedziała\egwinline{\textbf{\underline{Zawsze miałam to samo świadectwo do złożenia, które teraz składam odnośnie osobowości Boga}}}[Lt253-1903.9; 1903][https://egwwritings.org/?ref=en\_Lt253-1903.9]. Z \textit{Early Writings} zacytowała swoje wizje niebiańskiej rzeczywistości. Przypomniała, jak miała przywilej być w obecności Boga, jak Bóg, otoczony światłem Swojej chwały, przeszedł obok niej. Nie widziała Boga z powodu światła, którym był otoczony; bała się Go, myśląc, że gdyby się do niej zbliżył, \egwinline{zostałaby unicestwiona}. Wtedy zobaczyła\egwinline{\textbf{ukochanego Jezusa, że On jest osobą}. \textbf{Zapytałam Go, czy Jego Ojciec jest osobą i ma \underline{postać taką jak} On}. Jezus powiedział: «\textbf{Jestem wyrazem istoty Mojego Ojca!}»}[Lt253-1903.12; 1903][https://egwwritings.org/?ref=en\_Lt253-1903.12]. Pytanie, które miała, brzmiało: \textit{Czy Bóg jest osobą, mającą postać jak Jezus}? Odpowiedź była twierdząca — z silną biblijną podstawą. Jej wizje nie były źródłem prawdy o \emcap{osobowości Boga}; raczej potwierdzały prawdę, którą pionierzy odkryli poprzez pilne studiowanie Bożego słowa.

Dlatego ich ostateczny wniosek dotyczący \emcap{osobowości Boga} brzmiał: \others{Że jest \textbf{jeden Bóg}, \textbf{osobowa, duchowa }\textbf{\underline{istota}}, \textbf{stwórca wszystkich rzeczy}, wszechmocny, wszechwiedzący i wieczny, nieskończony w mądrości, świętości, sprawiedliwości, dobroci, prawdzie i miłosierdziu; niezmienny i \textbf{wszędzie obecny przez swojego przedstawiciela, Ducha Świętego}. Ps 139:7; Że jest jeden Pan Jezus Chrystus, \textbf{Syn Wiecznego Ojca, ten, przez którego stworzył wszystkie rzeczy i przez którego one istnieją}...} \others{...i w końcowej części Jego dzieła jako kapłana, zanim obejmie swój tron jako król, dokona \textbf{wielkiego pojednania} za grzechy wszystkich takich, a ich grzechy zostaną wtedy wymazane (Dz 3:19) i wyniesione ze świątyni, jak pokazano w służbie kapłaństwa lewickiego, które zapowiadało i przedstawiało służbę naszego Pana w niebie. Patrz m.in. Kpł 16; Hbr 8:4, 5; 9:6, 7}[Pierwszy i część drugiego punktu Fundamentalnych Zasad, 1905.]

Ellen White przypomniała dr. Kelloggowi o tym punkcie \emcap{fundamentalnych zasad}, stwierdzając: \egwinline{\textbf{Ojciec, Wszechwiedzący, stworzył świat przez Chrystusa Jezusa}. Chrystus jest światłością świata, drogą do życia wiecznego. \textbf{Jego, Pomazańca, Bóg dał, aby dokonał pojednania za grzechy świata}...}[Lt253-1903.21; 1903][https://egwwritings.org/?ref=en\_Lt253-1903.21]

Kwestia \emcap{osobowości Boga} dotyczy właściwości lub stanu Boga jako osoby. Pionierzy adwentyzmu dali na to odpowiedź, a Bóg zatwierdził ją przez pisma Ellen White: Bóg jest \textit{osobową duchową Istotą} i jest naszym niebiańskim Ojcem. Gdzie jest Jego obecność? \egwinline{\textbf{Nie mamy mówić, że Pan Bóg niebios jest w liściu lub w drzewie; bo Go tam nie ma. }\textbf{\underline{On zasiada na swoim tronie w niebiosach}}}[Lt253-1903.15; 1903][https://egwwritings.org/?ref=en\_Lt253-1903.15]. \\
Jego obecność jest na tronie w niebie. \\
\egwinline{\textbf{Niebo nie jest parą. Jest miejscem}. \textbf{Chrystus poszedł przygotować mieszkania dla tych, którzy Go miłują}, tych, którzy w posłuszeństwie Jego przykazaniom, wychodzą ze świata i są oddzieleni...}[EGW, Lt253-1903.25; 1903][https://egwwritings.org/?ref=en\_Lt253-1903.25]. \\
\egwinline{... «Głos Pana jest potężny; wstrząsa cedrami Libanu. \textbf{Pan jest w swoim świętym przybytku}. Niech cała ziemia zamilknie przed Nim». [Zobacz Ps 29:5; Ha 2:20]}[Lt253-1903.18; 1903][https://egwwritings.org/?ref=en\_Lt253-1903.18].

Według pionierów adwentyzmu i siostry White nasz niebiański Ojciec jest jednym Bogiem. Jest On osobową Duchową Istotą, obecną w niebie, na swoim tronie. Tron w niebie jest rzeczywistym, fizycznym tronem, na którym zasiada rzeczywista Osoba (Istota, mająca postać, tak jak Jezus) — nasz niebiański Ojciec. To miejsce jest rzeczywistym miejscem; nie jest parą ani żadnym innym duchowym wyobrażeniem.

\egwinline{\textbf{Często widziałam, że duchowe wyobrażenia odbierały całą chwałę niebu, a w umysłach wielu ludzi tron Dawida i umiłowana osoba Jezusa zostały spalone w ogniu spirytualizmu}. Widziałam, że niektórzy ze zwiedzonych i wprowadzonych w ten błąd zostaną wyprowadzeni na światło prawdy, \textbf{ale będzie prawie niemożliwe, aby całkowicie uwolnili się od zwodniczej mocy spirytualizmu. Tacy powinni dokonać gruntownego dzieła w wyznawaniu swoich błędów i porzuceniu ich na zawsze}.}[Lt253-1903.13; 1903][https://egwwritings.org/?ref=en\_Lt253-1903.13]

Duchowe wyobrażenie osoby Boga jest błędnym poglądem. W Biblii mamy świadectwa o niebie, niebiańskim tronie i Bogu, który na nim zasiada. Jeśli przyjmiemy te świadectwa w ich oczywistym znaczeniu, wtedy doktryna o Trójcy nie może się ostać. Biblia i Duch Proroctwa przedstawiają jednego Boga w niebie, jako osobową istotę, mającą ciało i postać tak jak Jezus. Ten pogląd nie jest w harmonii z doktryną o Trójjedynym Bogu, ponieważ ta wymaga, aby Duch Święty był Istotą\footnote{W \hyperref[appendix:unauthenticated-reports]{Dodatku} można znaleźć więcej cytatów, które wykluczają, aby Duch Święty był istotą posiadającą fizyczne ciało i formę.}, mającą ciało i formę — ta idea zagroziłaby temu, że Duch Święty jest środkiem Ojca i Syna, przez który Oni są wszędzie obecni. Aby podtrzymać doktrynę o Trójcy, świadectwa dotyczące tronu Bożego i osoby Boga muszą być rozumiane w jakimś duchowym sensie. Tutaj widzieliśmy, że siostra White przeciwstawiła prawdę o \emcap{osobowości Boga} doktrynie o Trójcy. Przeciwstawiła doktrynę o Trójcy pierwszym dwóm punktom \emcap{Fundamentalnych Zasad}, które były wynikiem studiowania Słowa Bożego przez naszych pionierów. Odnosząc się do pionierów i \emcap{Fundamentalnych Zasad}, powiedziała: \egwinline{\textbf{\underline{Łatane teorie}}\textbf{ nie mogą być przyjęte przez \underline{lojalnych wobec wiary i zasad}}\textbf{, które wytrzymały wszelki sprzeciw szatańskich wpływów}}[Lt253-1903.28; 1903][https://egwwritings.org/?ref=en\_Lt253-1903.28]

Wniosek jest prosty i jasny. Ci, którzy są lojalni wobec wiary i zasad otrzymanych na początku dzieła, nie mogą przyjąć łatanych teorii. Rozpatrywana w kontekście, łatana teoria, którą jest doktryna o Trójcy, nie może być przyjęta przez tych, którzy trzymają się mocno\egwinline{\textbf{ }\textbf{\underline{fundamentalnych zasad}\underline{opartych na niepodważalnym autorytecie}}}[SpTB02 59.1; 1904][https://egwwritings.org/?ref=en\_SpTB02.59.1]. Ten wniosek wiedzie nas z powrotem do naszego pierwszego proponowanego testu fundamentu naszej wiary.

% Poem 1: Łatane Teorie
\begin{titledpoem}{Łatane Teorie}
    \stanza{
        Prawda ustalona przez modlitwy żar, \\
        Punkt po punkcie, jak bezcenny dar. \\
        Zasady wiary, co przetrwały czas, \\
        Fundamentem są dla wszystkich nas.
    }

    \stanza{
        Łatane teorie chcą nas zwieść, \\
        Od starej ścieżki prawdy odwieść. \\
        Nie przyjmiemy poprawek do prawd tych, \\
        Wierność zasadom to nasz święty plig.
    }

    \stanza{
        Osobowość Boga to objawienie święte, \\
        Nie ludzkim rozumem pojęte. \\
        Wierne serca stoją na pewnym gruncie, \\
        Gdzie fundamenty trwają w każdym punkcie.
    }
\end{titledpoem}

% Poem 2: Osobowy Bóg
\begin{titledpoem}{Osobowy Bóg}
    \stanza{
        Bóg nie jest parą, lecz Osobą prawdziwą, \\
        Na tronie w niebie siedzi z mocą żywą. \\
        Nie w liściu, nie w drzewie Go szukaj, \\
        Lecz w świętym przybytku, gdzie światłość nieustająca.
    }

    \stanza{
        Chrystus, Syn Ojca Wiecznego, \\
        Jest wyrazem istoty Boga Wszechmocnego. \\
        Przez Niego świat został stworzony, \\
        Przez Niego grzesznik jest odkupiony.
    }

    \stanza{
        Spirytualizm pali tron Dawida w ogniu, \\
        Odbiera chwałę niebu w swoim pragnieniu. \\
        Lecz my trzymamy się prawdy objawionej, \\
        Nie teorii przez człowieka wymyślonej.
    }
\end{titledpoem}

% Poem 3: Wierne Świadectwo
\begin{titledpoem}{Wierne Świadectwo}
    \stanza{
        To samo świadectwo zawsze składam, \\
        O osobowości Boga prawdę wam przekazuję. \\
        Nie można łatać błędnych teorii, \\
        Gdy prawda jasno w Słowie się znajduje.
    }

    \stanza{
        Niebo jest miejscem rzeczywistym, \\
        Gdzie Chrystus mieszkania przygotowuje. \\
        Bóg nie jest wszędzie w panteistycznym sensie, \\
        Lecz przez Ducha Swego świat przenikuje.
    }

    \stanza{
        Lojalni wobec wiary i zasad, \\
        Nie przyjmą teorii, co prawdę zniekształcają. \\
        Fundamenty nasze są pewne i trwałe, \\
        Bo na niepodważalnym autorytecie stają.
    }
\end{titledpoem}