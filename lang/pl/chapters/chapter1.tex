\chapter{Fundament naszej wiary}


\egw{\textbf{Pan tchnie nową, życiodajną siłę w Swoje dzieło}, gdy ludzcy przedstawiciele będą posłuszni rozkazowi, by iść naprzód i głosić prawdę. \textbf{Ten, który oświadczył, że Jego prawda będzie świecić na wieki, będzie głosił tę prawdę przez wiernych posłańców, którzy sprawią, że trąba wyda wyraźny dźwięk}. \textbf{Prawda będzie krytykowana, wyśmiewana i wyszydzana; ale im \underline{dokładniej} będzie badana i testowana, tym \underline{jaśniej będzie świecić}}.}[SpTB02 51.1; 1904][https://egwwritings.org/?ref=en\_SpTB02.51.1&para=417.260]



\egwnogap{\textbf{Jako lud, mamy \underline{stać niewzruszenie na platformie wiecznej prawdy}, która przetrwała próby i doświadczenia. Mamy \underline{trzymać się pewnych filarów naszej wiary}. \underline{Zasady prawdy}, które Bóg nam objawił, \underline{są naszym jedynym prawdziwym fundamentem}. To one uczyniły nas tym, czym jesteśmy. Upływ czasu nie zmniejszył ich wartości. \underline{Nieustannym wysiłkiem wroga jest usunięcie tych prawd z ich miejsca} i zastąpienie ich \underline{fałszywymi teoriami}. \underline{Wprowadzi on} wszystko, co tylko możliwe, aby zrealizować swoje zwodnicze plany. Ale Pan wzbudzi ludzi o bystrym umyśle, którzy nadadzą tym prawdom właściwe miejsce w Bożym planie.}}[SpTB02 51.2; 1904][https://egwwritings.org/?ref=en\_SpTB02.51.2&para=417.261]



\egwnogap{\textbf{Zostałam pouczona przez niebiańskiego posłańca, że niektóre rozumowanie w książce ‘Living Temple’ jest błędne i że \underline{to rozumowanie wprowadziłoby w błąd} umysły tych, którzy nie są gruntownie utwierdzeni w \underline{podstawowych zasadach} obecnej prawdy. Wprowadza ona to, co jest jedynie spekulacją w \underline{kwestii osobowości Boga i miejsca Jego obecności}}. Nikt na tej ziemi nie ma prawa spekulować na ten temat. \textbf{Im więcej fantazyjnych teorii jest dyskutowanych, tym mniej ludzie będą wiedzieć o Bogu i prawdzie, która uświęca duszę}.}[SpTB02 51.3; 1904][https://egwwritings.org/?ref=en\_SpTB02.51.3&para=417.262]



\egwnogap{Jeden po drugim przychodzą do mnie  z prośbą \textbf{o wyjaśnienie stanowisk zajętych w ‘Living Temple’.} Odpowiadam: ‘\textbf{Są one niewytłumaczalne}.’ \textbf{Wyrażone poglądy nie dają prawdziwego poznania Boga}. W całej książce znajdują się fragmenty Pisma Świętego. Te fragmenty Pisma są przedstawione w taki sposób, że błąd wydaje się być prawdą. \textbf{Błędne teorie są przedstawione w tak ujmujący sposób, że jeśli nie zachowa się ostrożności, wielu zostanie wprowadzonych w błąd}.}[SpTB02 52.1; 1904][https://egwwritings.org/?ref=en\_SpTB02.52.1&para=417.265]



\egwnogap{\textbf{Nie potrzebujemy mistycyzmu, który jest w tej książce}. Ci, którzy przyjmują te sofizmaty, wkrótce znajdą się w pozycji, gdzie wróg może z nimi rozmawiać i odwieść ich od Boga. Zostało mi przedstawione, że autor tej książki jest na fałszywej ścieżce. \textbf{Stracił z oczu wyróżniające prawdy \underline{na ten czas}}. Nie wie, dokąd prowadzą jego kroki. \textbf{\underline{Ścieżka prawdy biegnie blisko ścieżki błędu}, i obie ścieżki mogą wydawać się być jedną dla tych umysłów, które nie są prowadzone przez Ducha Świętego i dlatego nie są w stanie szybko rozpoznać różnicy między prawdą a błędem}.}[SpTB02 52.2; 1904][https://egwwritings.org/?ref=en\_SpTB02.52.2&para=417.266]


\egwnogap{\textbf{Mniej więcej w czasie, gdy opublikowano ‘Living Temple’, pojawiły się przede mną w nocnej porze \underline{znaki wskazujące, że zbliża się niebezpieczeństwo}, i że muszę się na nie przygotować poprzez \underline{spisanie rzeczy}, które Bóg mi objawił \underline{odnośnie podstawowych zasad naszej wiary}}.}[SpTB02 52.3; 1904][https://egwwritings.org/?ref=en\_SpTB02.52.3&para=417.267]


\egwnogap{Przysłano mi egzemplarz ‘Living Temple’, ale pozostał w mojej bibliotece nieprzeczytany. Ze światła danego mi przez Pana \textbf{wiedziałam, że niektóre poglądy propagowane w książce nie miały poparcia Boga} i że \textbf{były one \underline{pułapką, którą wróg przygotował na ostatnie dni}}. Myślałam, że zostanie to z pewnością dostrzeżone i że nie będzie konieczne, abym cokolwiek o tym mówiła.}[SpTB02 52.4; 1904][https://egwwritings.org/?ref=en\_SpTB02.52.4&para=417.268]


\egwnogap{W sporze, który powstał wśród naszych braci \textbf{odnośnie nauk tej książki}, ci, którzy opowiadali się za jej szerokim rozpowszechnianiem, oświadczyli: ‘\textbf{Zawiera ona dokładnie te same poglądy, których naucza Siostra White}.’ To stwierdzenie ugodziło prosto w moje serce. Byłam załamana, ponieważ \textbf{wiedziałam, że to przedstawienie sprawy \underline{nie było prawdziwe}}.}[SpTB02 53.1; 1904][https://egwwritings.org/?ref=en\_SpTB02.53.1&para=417.270]


\egwnogap{W końcu mój syn powiedział do mnie: ‘Matko, powinnaś przeczytać przynajmniej niektóre części książki, żebyś zobaczyła, czy są one w harmonii ze światłem, które Bóg ci dał.’ Usiadł obok mnie i razem \textbf{przeczytaliśmy przedmowę i większość pierwszego rozdziału, a także akapity z innych rozdziałów}. Podczas czytania rozpoznałam te same poglądy, przeciwko którym zostałam wezwana do ostrzeżenia \textbf{podczas \underline{wczesnych dni} mojej publicznej służby}. Kiedy po raz pierwszy opuściłam stan Maine, było to po to, by udać się do Vermont i Massachusetts, aby nieść świadectwo przeciwko tym poglądom. \textbf{‘Living Temple’ zawiera alfę tych teorii. Wiedziałam, że \underline{omega nastąpi wkrótce}; i drżałam o nasz lud}. \textbf{Wiedziałam, że muszę ostrzec naszych braci i siostry, aby nie wdawali się w spory \underline{dotyczące obecności i osobowości Boga}}. \textbf{Stwierdzenia zawarte w ‘Living Temple’ \underline{w tej kwestii są nieprawidłowe}}. Pismo Święte użyte do poparcia przedstawionej tam doktryny jest błędnie zastosowane.}[SpTB02 53.2; 1904][https://egwwritings.org/?ref=en\_SpTB02.53.2&para=417.271]

\egwnogap{\textbf{Jestem zmuszona wypowiedzieć się przeciwko twierdzeniu, że nauki ‘Living Temple’ mogą być poparte stwierdzeniami z moich pism}. \textbf{Mogą być w tej książce wyrażenia i poglądy, które są w harmonii z moimi pismami}. \textbf{I mogą być w moich pismach liczne stwierdzenia, które wyrwane z kontekstu i interpretowane zgodnie z umysłem autora ‘Living Temple’, wydawałyby się być w harmonii z naukami tej książki}. Może to dawać pozorne poparcie twierdzeniu, że poglądy w ‘Living Temple’ są w harmonii z moimi pismami. \textbf{Ale nie daj Boże, aby ten pogląd przeważył}.}[SpTB02 53.3; 1904][https://egwwritings.org/?ref=en\_SpTB02.53.3&para=417.272]


\egwnogap{\textbf{Niewielu potrafi dostrzec rezultat przyjmowania sofizmatów głoszonych przez niektórych w obecnym czasie}. \textbf{Ale Pan podniósł zasłonę i \underline{pokazał mi rezultat, który by nastąpił}}. \textbf{Spirytualistyczne teorie \underline{dotyczące osobowości Boga}, doprowadzone do logicznej konkluzji, usuwają całą chrześcijańską ekonomię}. \textbf{Za nic mają światło, które Chrystus przyszedł z nieba dać Janowi, aby przekazał je Swojemu ludowi. Nauczają, że sceny, które są tuż przed nami, nie są wystarczająco ważne, by poświęcać im szczególną uwagę. Czynią bezskuteczną prawdę niebiańskiego pochodzenia, \underline{pozbawiają lud Boży ich przeszłego doświadczenia}, dając im w zamian fałszywą naukę}.}[SpTB02 54.1; 1904][https://egwwritings.org/?ref=en\_SpTB02.54.1&para=417.275]


\egwnogap{\textbf{W nocnej wizji} zostało mi wyraźnie pokazane, że \textbf{te poglądy} były postrzegane przez niektórych jako \textbf{wspaniałe prawdy}, \textbf{które mają być \underline{wprowadzone}} i wyeksponowane w obecnym czasie. \textbf{Pokazano mi \underline{platformę}, wspartą \underline{solidnymi belkami} — prawdami Słowa Bożego}. \textbf{Ktoś wysoko postawiony w dziele medycznym kierował jednym jak i drugim człowiekiem, by poluzowali belki podtrzymujące tę platformę}. Wtedy usłyszałam głos mówiący: ‘Gdzie są strażnicy, którzy powinni stać na murach Syjonu? Czy śpią? \textbf{\underline{Ten fundament został zbudowany przez Mistrza} i \underline{przetrwa} burzę i nawałnicę. Czy pozwolą temu człowiekowi \underline{przedstawiać doktryny}, które \underline{zaprzeczają przeszłemu doświadczeniu} ludu Bożego? Nadszedł czas aby podjąć zdecydowane działania}.’}[SpTB02 54.2; 1904][https://egwwritings.org/?ref=en\_SpTB02.54.2&para=417.276]


\egwnogap{\textbf{Nieprzyjaciel dusz starał się \underline{wprowadzić} przypuszczenie, że \underline{wielka reformacja} miała nastąpić wśród Adwentystów Dnia Siódmego, i że \underline{ta reformacja} miałaby \underline{polegać na porzuceniu doktryn, które stoją jako filary naszej wiary,} i zaangażowaniu się w proces reorganizacji}. \textbf{Gdyby ta reforma miała miejsce, \underline{co by z tego wynikło}?} \textbf{\underline{Zasady prawdy}, które Bóg w Swojej mądrości dał Kościołowi Ostatków, \underline{zostałyby odrzucone}}. \textbf{Nasza religia zostałaby zmieniona}. \textbf{\underline{Fundamentalne zasady}, które podtrzymywały dzieło przez ostatnie pięćdziesiąt lat, \underline{zostałyby uznane za błąd}}. \textbf{Zostałaby ustanowiona nowa organizacja}. \textbf{Zostałyby napisane książki nowego porządku}. \textbf{Zostałby wprowadzony system filozofii intelektualnej}. Założyciele tego systemu poszliby do miast i dokonali wspaniałego dzieła. Szabat, oczywiście, byłby lekceważony, \textbf{jak również Bóg, który go stworzył}. Nic nie mogłoby stanąć na drodze nowego ruchu. \textbf{Przywódcy nauczaliby, że cnota jest lepsza od występku, ale usunąwszy Boga, pokładaliby ufność w ludzkiej mocy, która bez Boga jest bezwartościowa}. \textbf{Ich fundament byłby zbudowany na piasku, a burza i nawałnica zmiotłyby tę konstrukcję}.}[SpTB02 54.3; 1904][https://egwwritings.org/?ref=en\_SpTB02.54.3&para=417.277]


\egwnogap{Kto ma autorytet, by rozpocząć taki ruch? \textbf{Mamy nasze Biblie}. \textbf{Mamy nasze doświadczenie, potwierdzone cudownym działaniem Ducha Świętego}. \textbf{Mamy prawdę, która nie dopuszcza żadnego kompromisu}. \textbf{\underline{Czy nie powinniśmy odrzucić wszystkiego, co nie jest w harmonii z tą prawdą}}?}[SpTB02 55.1; 1904][https://egwwritings.org/?ref=en\_SpTB02.55.1&para=417.280]
]


\egwnogap{Wahałam się i zwlekałam z wysłaniem tego, co Duch Pański przynaglał mnie do napisania. \textbf{Nie chciałam być zmuszona do przedstawiania tych zwodniczych wpływów sofizmatów. Ale w opatrzności Bożej, błędy, które się pojawiły, muszą zostać usunięte}.}[SpTB02 55.2; 1904][https://egwwritings.org/?ref=en\_SpTB02.55.2&para=417.281]


\egwnogap{Krótko przed tym, jak \textbf{wysłałam świadectwa dotyczące \underline{wysiłków wroga zmierzających do podkopania fundamentu naszej wiary} poprzez rozpowszechnianie \underline{zwodniczych teorii}}, przeczytałam historię o statku we mgle napotykającym górę lodową. Przez kilka nocy spałam niewiele. Czułam się przytłoczona jak wóz pod snopami. Pewnej nocy została mi wyraźnie przedstawiona scena. Statek płynął po wodach w gęstej mgle. Nagle strażnik krzyknął: ‘Góra lodowa prosto przed nami!’ Tam, wznosząca się wysoko  nad statkiem, była gigantyczna góra lodowa. Autorytatywny głos zawołał: ‘Staranuj ją!’ Nie było ani chwili wahania. Był to czas na natychmiastowe działanie. Mechanik włączył pełną parę, a sternik skierował statek prosto na górę lodową. Z trzaskiem uderzył w lód. Nastąpiło straszliwe wstrząśnięcie, a góra lodowa rozpadła się na wiele kawałków, spadających z hukiem jak grzmot na pokład. Pasażerowie zostali gwałtownie wstrząśnięci siłą zderzenia, ale nikt nie stracił życia. Statek został uszkodzony, ale nie bez możliwości naprawy. Odbił się od zderzenia, drżąc od dziobu do rufy jak żywa istota. Następnie ruszył naprzód swoją drogą.}[SpTB02 55.3; 1904][https://egwwritings.org/?ref=en\_SpTB02.55.3&para=417.282]


\egwnogap{“Dobrze znałam znaczenie tego przedstawienia. \textbf{Miałam swoje rozkazy}. Usłyszałam słowa, jak głos od naszego Kapitana, ‘\textbf{Staranuj ją}!’ Wiedziałam, jaki jest mój obowiązek i że nie ma ani chwili do stracenia. \textbf{Musiałam bez zwłoki wykonać rozkaz: ‘Staranuj ją!’“}}[SpTB02 56.1; 1904][https://egwwritings.org/?ref=en\_SpTB02.56.1&para=417.285]



\egwnogap{Tej nocy wstałam o pierwszej, pisząc tak szybko, jak tylko moja ręka mogła przesuwać się po papierze. Przez następne dni pracowałam od rana do wieczora, \textbf{przygotowując dla naszego ludu instrukcje, które otrzymałam \underline{odnośnie błędów}, które \underline{pojawiały się} wśród nas}.}[SpTB02 56.2; 1904][https://egwwritings.org/?ref=en\_SpTB02.56.2&para=417.286]


\egwnogap{\textbf{Miałam nadzieję, że nastąpi gruntowna reformacja i że \underline{zasady}, o które walczyliśmy \underline{we wczesnych dniach}, i które zostały wydobyte w mocy Ducha Świętego, \underline{zostaną zachowane}}.}[SpTB02 56.3; 1904][https://egwwritings.org/?ref=en\_SpTB02.56.3&para=417.287]



\egwnogap{\textbf{Wielu z naszych ludzi nie zdaje sobie sprawy, \underline{jak solidne} fundamenty naszej wiary zostały położone}. \textbf{Mój mąż, Starszy Joseph Bates, Ojciec Pierce, Starszy Edson i inni, którzy byli gorliwi, szlachetni i prawi, byli wśród tych po upływie czasu w 1844 roku, szukali prawdy jak ukrytego skarbu}. Spotykałam się z nimi i studiowaliśmy, i modliliśmy się żarliwie. Często pozostawaliśmy razem do późnej nocy, a czasami przez całą noc, modląc się o światło i studiując słowo. Raz za razem ci bracia zbierali się wspólnie, aby studiować Biblię, żeby mogli poznać jej znaczenie i być przygotowani do nauczania jej z mocą. Kiedy dochodzili do punktu w swoim studium, gdzie mówili: ‘Nie możemy zrobić nic więcej’, Duch Pański zstępował na mnie, byłam zabierana w widzeniu i otrzymywałam jasne wyjaśnienie fragmentów, które studiowaliśmy, wraz z instrukcjami, jak mamy skutecznie pracować i nauczać. W ten sposób zostało dane światło, które pomogło nam zrozumieć Pisma odnośnie Chrystusa, Jego misji i Jego kapłaństwa. \textbf{Linia prawdy rozciągająca się od tego czasu aż do czasu, gdy wejdziemy do miasta Bożego, została mi wyraźnie ukazana, i przekazałam innym instrukcje, które Pan mi dał}.}[SpTB02 56.4; 1904][https://egwwritings.org/?ref=en\_SpTB02.56.4&para=417.288]

\egwnogap{Przez cały ten czas nie mogłam zrozumieć rozumowania braci. Mój umysł był jakby zamknięty i nie mogłam pojąć znaczenia studiowanych przez nas fragmentów Pisma Świętego. Był to jeden z największych smutków mojego życia. \textbf{Pozostawałam w tym stanie umysłu, dopóki wszystkie \underline{główne punkty naszej wiary} nie zostały wyjaśnione w naszych umysłach, w harmonii ze Słowem Bożym}. Bracia wiedzieli, że gdy nie byłam w widzeniu, nie mogłam zrozumieć tych spraw i przyjmowali objawienia jako światło bezpośrednio z nieba.}[SpTB02 57.1; 1904][https://egwwritings.org/?ref=en\_SpTB02.57.1&para=417.291]


\egwnogap{Przez dwa lub trzy lata mój umysł pozostawał zamknięty na zrozumienie Pisma Świętego. W trakcie naszej pracy, mój mąż i ja odwiedziliśmy ojca Andrewsa, który cierpiał na silny reumatyzm zapalny. Modliliśmy się za niego. Położyłam ręce na jego głowie i powiedziałam: “Ojcze Andrews, Pan Jezus cię uzdrawia.” Został uzdrowiony natychmiast. Wstał i chodził po pokoju, chwaląc Boga i mówiąc: “Nigdy wcześniej czegoś takiego nie widziałem. Aniołowie Boży są w tym pokoju.” Chwała Pańska została objawiona. Światło zdawało się świecić w całym domu, a ręka anioła została położona na mojej głowie. Od tego czasu aż do teraz jestem w stanie rozumieć Słowo Boże.}[SpTB02 57.2; 1904][https://egwwritings.org/?ref=en\_SpTB02.57.2&para=417.292]


\egwnogap{\textbf{Jaki to wpływ prowadzi ludzi na tym etapie naszej historii do działania w ukryty, potężny sposób, aby \underline{zburzyć fundament naszej wiary} - fundament, który został położony na początku naszej pracy poprzez pobożne studiowanie słowa i przez objawienie? Na \underline{tym fundamencie} budujemy przez \underline{ostatnie pięćdziesiąt lat}. Czy dziwicie się, że gdy widzę początek dzieła, które \underline{usuwa niektóre z filarów naszej wiary}, mam coś do powiedzenia? Muszę być posłuszna rozkazowi: “Przeciwstaw się temu!”}}[SpTB02 58.1; 1904][https://egwwritings.org/?ref=en\_SpTB02.58.1&para=417.295]


\egwnogap{Żywię najczulsze uczucia wobec dr. Kellogga. Przez wiele lat starałam się trzymać go mocno. Słowo Boże zawsze mówiło mi: “Możesz mu pomóc.” Czasami budzę się w nocy i chodząc po pokoju, modlę się: “O Panie, trzymaj dr. Kellogga mocno. Nie pozwól mu odejść. Zachowaj go niezłomnym. Namaść jego oczy niebiańską maścią, aby mógł widzieć wszystko wyraźnie.” Noc po nocy leżałam bezsennie, zastanawiając się, jak mogę mu pomóc. Gorliwie i często modliłam się, aby Pan nie pozwolił mu odwrócić się od uświęcającej prawdy. To jest ciężar, który mnie przygniata - pragnienie, aby nie popełniał błędów, które zraniłyby jego duszę i \textbf{zaszkodziły sprawie obecnej prawdy}. Ale od pewnego czasu jego działania ujawniają, że kontroluje go dziwny duch. Pan weźmie tę sprawę w swoje ręce. Muszę nieść poselstwa ostrzeżenia, które Bóg mi daje, a następnie pozostawić Panu rezultaty. \textbf{Muszę teraz przedstawić tę sprawę we wszystkich jej aspektach, ponieważ lud Boży nie może zostać ograbiony}.}[SpTB02 58.2; 1904][https://egwwritings.org/?ref=en\_SpTB02.58.2&para=417.296]


\egwnogap{\textbf{Jesteśmy ludem zachowującym przykazania Boże. Przez ostatnie pięćdziesiąt lat każda forma herezji była kierowana przeciwko nam, aby zaciemnić nasze umysły odnośnie nauczania Słowa},\textbf{—szczególnie dotyczącego służby Chrystusa w niebiańskiej świątyni i poselstwa niebios na te ostatnie dni, danego przez aniołów z czternastego rozdziału Objawienia}. \textbf{Poselstwa wszelkiego rodzaju i typu były narzucane Adwentystom Dnia Siódmego, aby zastąpić prawdę, która \underline{punkt po punkcie} została odkryta przez pobożne studiowanie i potwierdzona przez cudowną moc Pana}. \textbf{Ale \underline{znaki}, które \underline{uczyniły nas tym, czym jesteśmy}, \underline{mają być zachowane}, i \underline{będą zachowane}, jak Bóg wskazał przez swoje słowo i świadectwo swojego Ducha}. \textbf{Wzywa nas, abyśmy \underline{trzymali się mocno}, z uściskiem wiary, \underline{fundamentalnych zasad}, które są \underline{oparte na niepodważalnym autorytecie}}.}[SpTB02 59.1; 1904][https://egwwritings.org/?ref=en\_SpTB02.59.1&para=417.299]


Istniała konieczność ostrzeżenia kościoła przed działaniem wroga zmierzającym do wykorzenienia fundamentu naszej wiary. Istniała konieczność przypomnienia kościołowi, co stanowi prawdziwy fundament wiary Adwentystów Dnia Siódmego. Wydaje się, że Adwentyści Dnia Siódmego w tamtym czasie zapominali o tym, jak \egwinline{Pan nas prowadził i Jego nauczaniu w naszej przeszłej historii.}[LS 196.2; 1915][https://egwwritings.org/?ref=en\_LS.196.2&para=41.1083]


\egw{Jaki to wpływ prowadzi ludzi na tym etapie naszej historii do działania w ukryty, potężny sposób, aby \textbf{zburzyć fundament naszej wiary} - fundament, który został położony \textbf{na początku naszego dzieła} poprzez pobożne studiowanie słowa i przez objawienie? Na \textbf{tym fundamencie} budujemy przez \textbf{ostatnie pięćdziesiąt lat}. Czy dziwicie się, że gdy widzę początek dzieła, które \textbf{usuwa niektóre z filarów naszej wiary}, mam coś do powiedzenia? Muszę być posłuszna rozkazowi: ‘\textbf{Przeciwstaw się temu}!’}[SpTB02 58.1; 1904][https://egwwritings.org/?ref=en\_SpTB02.58.1&para=417.295]


Czemu Siostra White otrzymała polecenie, aby się przeciwstawić?


\egwinline{Mniej więcej w czasie, gdy opublikowano «Living Temple»} w porze nocnej  otrzymała ona\egwinline{znaki wskazujące, że zbliża się niebezpieczeństwo,} i że musi się na nie \egwinline{przygotować poprzez spisanie rzeczy,które Bóg jej objawił} \egwinline{\textbf{odnośnie podstawowych zasad naszej wiary}.}

Została \egwinline{pouczona przez niebiańskiego posłańca, że niektóre rozumowanie w książce ‘Living Temple’ jest błędne i że \textbf{to rozumowanie wprowadzi w błąd} umysły tych, którzy nie są gruntownie utwierdzeni w \textbf{podstawowych zasadach} obecnej prawdy.}


Więc jaki był faktyczny problem z książką “Living Temple”?


Jeśli jesteś uczonym, historykiem adwentystycznym, teologiem lub po prostu studentem teologii, zanim udzielisz prostej odpowiedzi i powiesz, że problemem był panteizm, chcielibyśmy zwrócić twoją uwagę z powrotem na tekst. Siostra White jasno określiła istotę problemu, stwierdzając, że “Living Temple” \egwinline{wprowadza to, co jest jedynie spekulacją w \textbf{kwestii osobowości Boga i miejsca Jego obecności}.}


Nie zaprzeczamy panteistycznemu problemowi książki, ale chcemy przekierować uwagę z błędu Kellogga na światło, które dał Bóg. Istnieją dwa sposoby podejścia do kryzysu Kellogga. Jeden polega na zajęciu się panteizmem, a drugi na zajęciu się \egwinline{\textbf{osobowością Boga} i \textbf{miejscem Jego obecności}}. Jeden sposób to studiowanie błędu, a drugi to studiowanie Prawdy. Jeden sposób to analizowanie ciemności, a drugi to picie ze źródła Prawdy. Wybieramy to drugie i z tego powodu ta książka różni się od setek innych książek napisanych o kryzysie Kellogga. Tematem tej książki nie jest panteizm ani żaden inny błąd, ale prawda i to, co Bóg objawił o swojej osobowości i miejscu swojej obecności. To był prawdziwy problem publikacji Kellogga.

Wierzymy, że studiowanie i analizowanie błędu jest bardzo niebezpieczne, ponieważ błąd prowadzi do zwiedzenia. Problem ze zwiedzeniem polega na tym, że możemy być zwiedzeni, oczywiście nie wiedząc, że jesteśmy zwiedzeni! Mocno wierzymy, że Ellen White była prorokiem Boga i że otrzymywała Światło od Boga, \bible{który jest światłością, a nie ma w nim żadnej ciemności}[1 J 1:5]. Dlatego nie oczekujemy, że Siostra White wyjaśni błędy w książce “Living Temple”. Wielu przychodziło do niej, prosząc ją \egwinline{o wyjaśnienie stanowisk zajętych w ‘Living Temple’.} Odpowiedziała, \egwinline{Są one niewytłumaczalne}. Jej celem nie było analizowanie błędu, ale rzucenie światła Prawdy na \emcap{osobowość Boga} i miejsce Jego obecności. W ten sposób wskazywała na prawdy, na których Bóg założył Kościół Adwentystów Dnia Siódmego. Te prawdy stanowiły fundament naszej wiary. Te prawdy zostały nam dane w naszych wczesnych dniach. Przekierowując naszą uwagę z osobowości Boga na panteizm, tracimy okazję, by pamiętać \egwinline{\textbf{sposób, w jaki Pan nas prowadził}, i \textbf{Jego nauczanie} w naszej \textbf{przeszłej historii}}. W tym świetle wyrażamy nasze zaniepokojenie kryzysem Kellogga i jego panteistycznym podejściem, ponieważ \egwinline{ścieżka prawdy biegnie blisko ścieżki błędu i obie ścieżki mogą wydawać się jedną}; rozwiązaniem jest być \egwinline{gruntownie utwierdzonym w \textbf{podstawowych zasadach} obecnej prawdy}. W innych miejscach Siostra White mocno ustanowiła tę zasadę.


\egw{Szatan wcale nie śpi; jest on w pełni rozbudzony i rozgrywa grę o życie dusz ludu Bożego. Przyjdzie do nich z wszelkiego rodzaju pochlebstwami, mając nadzieję doprowadzić ich do zboczenia z drogi wierności. \textbf{Pragnie odwrócić ich uwagę od prawdziwych kwestii ku fałszywym teoriom}.}[Ms132-1903.42; 1904][https://egwwritings.org/?ref=en\_Ms132-1903.42&para=9056.56]


Skupmy więc naszą uwagę na prawdziwym problemie zamiast na fałszywych teoriach.

% The Foundation of our Faith

\begin{titledpoem}
    \stanza{
        Pillars of truth, laid with care, \\
        By pioneers who sought in prayer. \\
        Principles firm, the Lord's design, \\
        A platform strong, for all time.
    }

    \stanza{
        Beware the subtle shifts that call, \\
        To change what should not change at all. \\
        Our identity, in these we find, \\
        God's revelations to mankind.
    }

    \stanza{
        Stand fast upon this solid ground, \\
        Where wisdom and God's light abound. \\
        Defend these truths with all your might, \\
        For in them shines eternal light.
    }
\end{titledpoem}

% Fundament naszej wiary

\begin{titledpoem}
    \stanza{
        Prawda wieczna mocno stoi, \\
        Żaden wróg jej nie rozbroi. \\
        Pan swą siłą dzieło tchnie, \\
        Wiernych posłów wspiera swe.
    }

    \stanza{
        Na platformie prawdy trwajmy, \\
        Filarów wiary się trzymajmy. \\
        To fundament dany nam, \\
        Który przetrwa czasu plan.
    }

    \stanza{
        Wróg chce usunąć prawdy te, \\
        Fałszywe teorie szerząc swe. \\
        Lecz Bóg wzbudzi wiernych swych, \\
        By strzec zasad odwiecznych tych.
    }

    \stanza{
        O Bożej osobowości nie spekulujmy, \\
        Mistycyzmu się wystrzegajmy. \\
        Ścieżka prawdy blisko błędu biegnie, \\
        Tylko Duch Święty nas bezpiecznie wiedzie.
    }
\end{titledpoem}

% Staranuj górę lodową

\begin{titledpoem}
    \stanza{
        "Staranuj ją!" - rozkaz brzmi, \\
        Gdy góra lodowa przed nami tkwi. \\
        Bez wahania naprzód płyń, \\
        By rozbić zwodniczy cień.
    }

    \stanza{
        Fundament wiary solidny jest, \\
        Przez pięćdziesiąt lat przetrwał test. \\
        Przez studiowanie i modlitwy żar, \\
        Bóg dał nam prawdy wiecznej dar.
    }

    \stanza{
        Zasady, które Bóg nam dał, \\
        By każdy wiernie przy nich stał. \\
        Nie pozwól, by je zabrał wróg, \\
        Gdy burzy to, co zbudował Bóg.
    }

    \stanza{
        Trzymajmy się więc mocno dziś, \\
        Tego, co dane nam przez Pana jest. \\
        Niech prawda jasno świeci nam, \\
        Aż wejdziemy do niebios bram.
    }
\end{titledpoem}