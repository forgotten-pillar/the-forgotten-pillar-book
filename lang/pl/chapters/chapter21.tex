\chapter{Remembering the beginning} \label{chap:remembering-the-beginning}


\chapter{Pamiętając początek} \label{chap:remembering-the-beginning}


\egw{\textbf{We cannot for a moment have any \underline{misrepresentation} upon these solemn and important subjects of truth which have been the faith of our people since 1844.}}[Lt300-1903.9; 1903][https://egwwritings.org/?ref=en\_Lt300-1903.9&para=7705.15]


\egw{\textbf{Nie możemy ani na chwilę dopuścić do jakiegokolwiek \underline{fałszywego przedstawienia} tych uroczystych i ważnych tematów prawdy, które były wiarą naszego ludu od 1844 roku.}}[Lt300-1903.9; 1903][https://egwwritings.org/?ref=en\_Lt300-1903.9&para=7705.15]


The true meaning of the \emcap{Fundamental Principles} is a broader view of the three angels’ messages.


Prawdziwe znaczenie \emcap{Fundamentalnych Zasad} to szersze spojrzenie na poselstwa trzech aniołów.


\egw{\textbf{We are God’s commandment-keeping people}. For the past fifty years every phase of heresy has been brought to bear upon us, to \textbf{becloud our minds regarding the teaching of the word,—\underline{especially concerning the ministration of Christ in the heavenly sanctuary}, and the message of heaven for these last days, as \underline{given by the angels of the fourteenth chapter of Revelation}}. Messages of every order and kind have been urged upon Seventh-day Adventists, to \textbf{take the place of the truth which}, \textbf{point by point}, has been sought out by prayerful study, and testified to by the miracle-working power of the Lord. \textbf{But the way-marks which have made us what we are, are to be preserved, and they will be preserved}, as God has signified through His word and the testimony of His Spirit. \textbf{He calls upon us to hold firmly, with the grip of faith, to \underline{the fundamental principles} that are based upon unquestionable authority}.}[SpTB02 59.1; 1904][https://egwwritings.org/?ref=en\_SpTB02.59.1]


\egw{\textbf{Jesteśmy ludem Bożym zachowującym przykazania}. Przez ostatnie pięćdziesiąt lat każda forma herezji była kierowana przeciwko nam, aby \textbf{zaciemnić nasze umysły odnośnie nauczania Słowa — \underline{szczególnie dotyczącego służby Chrystusa w niebiańskiej świątyni} i poselstwa niebios na te ostatnie dni, \underline{danego przez aniołów z czternastego rozdziału Apokalipsy}}. Poselstwa wszelkiego rodzaju i typu były narzucane Adwentystom Dnia Siódmego, aby \textbf{zająć miejsce prawdy, która}, \textbf{punkt po punkcie}, została odkryta przez modlitewne studium i potwierdzona przez cudowną moc Pana. \textbf{Ale znaki, które uczyniły nas tym, kim jesteśmy, mają być zachowane i będą zachowane}, jak Bóg oznajmił przez swoje słowo i świadectwo swojego Ducha. \textbf{Wzywa nas, abyśmy mocno trzymali się, z uściskiem wiary, \underline{fundamentalnych zasad}, które są oparte na niepodważalnym autorytecie}.}[SpTB02 59.1; 1904][https://egwwritings.org/?ref=en\_SpTB02.59.1]


Here we see how Ellen White described the message of the \emcap{Fundamental Principles} as the messages of the three angels’, from the fourteenth chapter of Revelation, and as a message concerning the ministration of Christ in the heavenly sanctuary. The first point of the \emcap{Fundamental Principles}, which is widely discussed here, answers the important question given by the first angel in the fourteenth chapter of Revelation: \textit{who is the God we ought to worship}?


Tutaj widzimy, jak Ellen White opisała poselstwo \emcap{Fundamentalnych Zasad} jako poselstwa trzech aniołów z czternastego rozdziału Apokalipsy i jako poselstwo dotyczące służby Chrystusa w niebiańskiej świątyni. Pierwszy punkt \emcap{Fundamentalnych Zasad}, który jest szeroko omawiany tutaj, odpowiada na ważne pytanie zadane przez pierwszego anioła w czternastym rozdziale Apokalipsy: \textit{kim jest Bóg, którego powinniśmy czcić}?


\bible{Fear \textbf{God}, and \textbf{give glory \underline{to him}}; for \textbf{the hour of \underline{his} judgment is come}: and \textbf{worship \underline{him}} that made heaven, and earth, and the sea, and the fountains of waters.}[Revelation 14:7]


\bible{Bójcie się \textbf{Boga} i \textbf{oddajcie \underline{mu} chwałę}, gdyż \textbf{nadeszła godzina \underline{jego} sądu}. \textbf{Oddajcie pokłon \underline{temu}}, który stworzył niebo i ziemię, morze i źródła wód.}[Objawienie 14:7]


Who is the God we ought to worship, declared by the first angel? In the spectrum of time we find different answers to this question. Today the answer is the Triune God, or Trinity God, as presented in the Fundamental Beliefs of Seventh-day Adventists. But, we raise the question: who was the God that the Adventist pioneers worshipped? The first angel’s message is tied to prophetic time, which was fulfilled in the times of our pioneers. The entire purpose behind their labor was the proclamation of the three angels’ messages. In 1844, the hour of God’s judgment had come. If the Trinity God was the God whose hour had come, and our pioneers did not worship the Trinity, wouldn't they have failed in their purpose of creating this movement?


Kim jest Bóg, którego powinniśmy czcić, ogłoszony przez pierwszego anioła? W spektrum czasu znajdujemy różne odpowiedzi na to pytanie. Dziś odpowiedzią jest Bóg Trójjedyny, czyli Trójca, jak przedstawiono w Fundamentalnych Wierzeniach Adwentystów Dnia Siódmego. Ale stawiamy pytanie: kim był Bóg, którego czcili pionierzy adwentyzmu? Poselstwo pierwszego anioła jest związane z czasem proroczym, który wypełnił się w czasach naszych pionierów. Całym celem ich pracy było głoszenie poselstw trzech aniołów. W 1844 roku nadeszła godzina Bożego sądu. Jeśli Bóg Trójca był Bogiem, którego godzina nadeszła, a nasi pionierzy nie czcili Trójcy, czy nie zawiedliby w swoim celu stworzenia tego ruchu?


Let us examine the history of our prophetic movement with this question: did our pioneers worship the true God in proclaiming the message of the first angel? We read the explanation of the events in the passing of 1844.


Zbadajmy historię naszego proroczego ruchu z tym pytaniem: czy nasi pionierzy czcili prawdziwego Boga głosząc poselstwo pierwszego anioła? Czytamy wyjaśnienie wydarzeń po przejściu roku 1844.


\egw{\textbf{Like the first disciples, William Miller and his associates did not, themselves, fully comprehend the import of the message which they bore}. Errors that had been long established in the church prevented them from arriving at a correct interpretation of an important point in the prophecy. Therefore, though they proclaimed the message which God had committed to them to be given to the world, yet through a misapprehension of its meaning they suffered disappointment.}[GC 351.2; 1888][https://egwwritings.org/?ref=en\_GC.351.2&para=132.1604]


\egw{\textbf{Podobnie jak pierwsi uczniowie, William Miller i jego współpracownicy sami nie pojmowali w pełni znaczenia poselstwa, które głosili}. Błędy, które od dawna były utrwalone w kościele, uniemożliwiały im dojście do prawidłowej interpretacji ważnego punktu proroctwa. Dlatego, chociaż głosili poselstwo, które Bóg im powierzył, aby przekazać je światu, to jednak z powodu błędnego zrozumienia jego znaczenia doznali rozczarowania.}[GC 351.2; 1888][https://egwwritings.org/?ref=en\_GC.351.2&para=132.1604]


\egwnogap{In explaining Daniel 8:14, ‘Unto \textbf{two thousand and three hundred days; then shall \underline{the sanctuary be cleansed}},’ Miller, as has been stated, adopted the generally received view that the earth is the sanctuary, and he believed that the cleansing of the sanctuary represented the purification of the earth by fire at the coming of the Lord. When, therefore, he found that the close of the 2300 days was definitely foretold, he concluded that this revealed the time of the second advent. His error resulted from accepting the popular view as to what constitutes the sanctuary.}[GC 352.1; 1888][https://egwwritings.org/?ref=en\_GC.352.1&para=132.1607]


\egwnogap{Wyjaśniając Daniela 8:14, ‘Aż do \textbf{dwóch tysięcy trzystu wieczorów i poranków. Wtedy \underline{świątynia zostanie oczyszczona}}’, Miller, jak już wspomniano, przyjął powszechnie uznawany pogląd, że ziemia jest świątynią, i wierzył, że oczyszczenie świątyni oznacza oczyszczenie ziemi przez ogień przy przyjściu Pana. Gdy zatem odkrył, że koniec 2300 dni był wyraźnie przepowiedziany, doszedł do wniosku, że to ujawniało czas drugiego przyjścia. Jego błąd wynikał z przyjęcia popularnego poglądu na temat tego, co stanowi świątynię.}[GC 352.1; 1888][https://egwwritings.org/?ref=en\_GC.352.1&para=132.1607]


\egwnogap{In the typical system, which was a shadow of the sacrifice and \textbf{priesthood of Christ}, \textbf{the cleansing of the sanctuary was the last service performed by the high priest }in the yearly round of ministration.\textbf{ It was the closing work of the atonement—a removal or putting away of sin from Israel}. \textbf{It prefigured the closing work in the ministration of our High Priest in heaven, in the removal or blotting out of the sins of His people, which are registered in the heavenly records}. \textbf{This service involves a work of \underline{investigation, a work of judgment}; and it immediately precedes the coming of Christ} in the clouds of heaven with power and great glory; for when He comes, every case has been decided. Says Jesus: ‘My reward is with Me, to give every man according as his work shall be.’ Revelation 22:12. \textbf{It is this work of judgment, immediately preceding the second advent, that is \underline{announced in the first angel’s message of Revelation 14:7}: ‘Fear \underline{God}, and give glory to Him; \underline{for the hour of His judgment is come}.}’}[GC 352.2; 1888][https://egwwritings.org/?ref=en\_GC.352.2&para=132.1608]


\egwnogap{W systemie typicznym, który był cieniem ofiary i \textbf{kapłaństwa Chrystusa}, \textbf{oczyszczenie świątyni było ostatnią służbą wykonywaną przez arcykapłana }w corocznym cyklu posługi.\textbf{ Było to końcowe dzieło pojednania—usunięcie lub oddalenie grzechu od Izraela}. \textbf{Zapowiadało to końcowe dzieło w posłudze naszego Arcykapłana w niebie, w usunięciu lub wymazaniu grzechów Jego ludu, które są zapisane w niebiańskich księgach}. \textbf{Ta służba obejmuje dzieło \underline{badania, dzieło sądu}; i bezpośrednio poprzedza przyjście Chrystusa} w obłokach nieba z mocą i wielką chwałą; bo gdy On przyjdzie, każda sprawa została już rozstrzygnięta. Mówi Jezus: ‘Moja nagroda jest ze Mną, aby dać każdemu człowiekowi według jego uczynków.’ Objawienie 22:12. \textbf{To właśnie to dzieło sądu, bezpośrednio poprzedzające drugie przyjście, jest \underline{ogłoszone w pierwszym poselstwie anielskim z Objawienia 14:7}: ‘Bójcie się \underline{Boga} i oddajcie Mu chwałę, \underline{gdyż nadeszła godzina Jego sądu}.}’}[GC 352.2; 1888][https://egwwritings.org/?ref=en\_GC.352.2&para=132.1608]


\egwnogap{\textbf{Those who proclaimed this warning gave the right message at the right time}. But as the early disciples declared, ‘The time is fulfilled, and the kingdom of God is at hand,’ based on the prophecy of Daniel 9, while they failed to perceive that the death of the Messiah was foretold in the same scripture, \textbf{so Miller and his associates preached the message based on \underline{Daniel 8:14 and Revelation 14:7}, and failed to see that there were still other messages brought to view in Revelation 14}, which were also to be given before the advent of the Lord. As the disciples were mistaken in regard to the kingdom to be set up at the end of the seventy weeks, so Adventists were mistaken in regard to the event to take place at the expiration of the 2300 days. In both cases there was an acceptance of, or rather an adherence to, popular errors that blinded the mind to the truth. Both classes fulfilled the will of God in delivering the message which He desired to be given, and both, through their own misapprehension of their message, suffered disappointment.}[GC 352.3; 1888][https://egwwritings.org/?ref=en\_GC.352.3&para=132.1609]


\egwnogap{\textbf{Ci, którzy głosili to ostrzeżenie, przekazali właściwe poselstwo we właściwym czasie}. Ale tak jak pierwsi uczniowie głosili: ‘Wypełnił się czas i przybliżyło się królestwo Boże’, opierając się na proroctwie Daniela 9, podczas gdy nie dostrzegli, że śmierć Mesjasza była przepowiedziana w tym samym fragmencie Pisma, \textbf{tak Miller i jego współpracownicy głosili poselstwo oparte na \underline{Daniela 8:14 i Objawienia 14:7}, i nie dostrzegli, że były jeszcze inne poselstwa ukazane w Objawieniu 14}, które również miały być głoszone przed przyjściem Pana. Tak jak uczniowie mylili się co do królestwa, które miało być ustanowione na końcu siedemdziesięciu tygodni, tak adwentyści mylili się co do wydarzenia, które miało nastąpić po upływie 2300 dni. W obu przypadkach nastąpiło przyjęcie, a raczej przylgnięcie do popularnych błędów, które zaślepiły umysł na prawdę. Obie grupy wypełniły wolę Bożą, przekazując poselstwo, które On pragnął, aby zostało przekazane, i obie, z powodu własnego niezrozumienia swojego poselstwa, doznały rozczarowania.}[GC 352.3; 1888][https://egwwritings.org/?ref=en\_GC.352.3&para=132.1609]


In reading the explanation of the great disappointment, did you see the answer to the question, “\textit{who is God whose judgment has come}?” The first angel’s message from Revelation 14:7 aligns exactly with the prophetic time declared in Daniel 8:14. The judgment that has come was the investigative judgment, which started in 1844. The Bible clearly describes whose hour of judgment has come in the first angel’s message. Let us read it in the Bible and see Ellen White’s comment.


Czytając wyjaśnienie wielkiego rozczarowania, czy dostrzegłeś odpowiedź na pytanie: “\textit{kim jest Bóg, którego sąd nadszedł}?” Pierwsze poselstwo anielskie z Objawienia 14:7 dokładnie pokrywa się z czasem proroczym ogłoszonym w Daniela 8:14. Sąd, który nadszedł, był sądem śledczym, który rozpoczął się w 1844 roku. Biblia wyraźnie opisuje, czyja godzina sądu nadeszła w pierwszym poselstwie anielskim. Przeczytajmy to w Biblii i zobaczmy komentarz Ellen White.


\egw{‘I beheld,’ says the prophet Daniel, \textbf{‘till thrones were placed, and One that was \underline{Ancient of Days} \underline{did sit}}: \textbf{His raiment} was white as snow, and \textbf{the hair of His head} like pure wool; \textbf{His throne was fiery flames}, and the wheels thereof burning fire. A fiery stream issued and came forth from before Him: thousand thousands ministered unto Him, and ten thousand times ten thousand stood before Him: \textbf{\underline{the judgment was set, and the books were opened}}.’ Daniel 7:9, 10, R.V.}[GC 479.1; 1888][https://egwwritings.org/?ref=en\_GC.479.1&para=132.2169]


\egw{‘Patrzyłem,’ mówi prorok Daniel, \textbf{‘aż postawiono trony i \underline{Przedwieczny} \underline{zasiadł}}: \textbf{Jego szata} była biała jak śnieg, a \textbf{włosy Jego głowy} jak czysta wełna; \textbf{Jego tron był jak płomienie ognia}, a koła jego jak płonący ogień. Strumień ognia wypływał i wychodził sprzed Niego: tysiąc tysięcy służyło Mu, a dziesięć tysięcy razy dziesięć tysięcy stało przed Nim: \textbf{\underline{sąd zasiadł i księgi zostały otwarte}}.’ Daniela 7:9, 10, R.V.}[GC 479.1; 1888][https://egwwritings.org/?ref=en\_GC.479.1&para=132.2169]


\egwnogap{\textbf{Thus was presented to the prophet’s vision the great and solemn day when the characters and the lives of men should pass in review before the Judge of all the earth, and to every man should be rendered ‘according to his works.’ \underline{The Ancient of Days is God the Father}.} Says the psalmist: \textbf{‘Before }the mountains were brought forth, or ever Thou hadst formed the earth and the world, even \textbf{from everlasting to everlasting}, \textbf{Thou art God}.’ Psalm 90:2. \textbf{\underline{It is He, the source of all being, and the fountain of all law, that is to preside in the judgment}}. And holy angels as ministers and witnesses, in number ‘ten thousand times ten thousand, and thousands of thousands,’ attend this great tribunal.}[GC 479.2; 1888][https://egwwritings.org/?ref=en\_GC.479.2&para=132.2170]


\egwnogap{\textbf{Tak oto przedstawiono prorokowi w wizji wielki i uroczysty dzień, w którym charaktery i życie ludzi miały zostać poddane przeglądowi przed Sędzią całej ziemi, a każdemu człowiekowi miało być oddane ‘według jego uczynków’. \underline{Przedwieczny to Bóg Ojciec}.} Mówi psalmista: \textbf{‘Zanim }góry powstały, zanim ukształtowałeś ziemię i świat, \textbf{od wieków na wieki}, \textbf{Ty jesteś Bogiem}.’ Psalm 90:2. \textbf{\underline{To On, źródło wszelkiego bytu i źródło wszelkiego prawa, będzie przewodniczył w sądzie}}. A święci aniołowie jako słudzy i świadkowie, w liczbie ‘dziesięć tysięcy razy dziesięć tysięcy i tysiące tysięcy’, uczestniczą w tym wielkim trybunale.}[GC 479.2; 1888][https://egwwritings.org/?ref=en\_GC.479.2&para=132.2170]


\egwnogap{\textbf{‘And, behold, one like \underline{the Son of man} came with the clouds of heaven, and came to \underline{the Ancient of Days}, and they \underline{brought Him near before Him}}. And there was given Him dominion, and glory, and a kingdom, that all people, nations, and languages, should serve Him: His dominion is an everlasting dominion, which shall not pass away.’ Daniel 7:13, 14. \textbf{The coming of Christ here described is not His second coming to the earth}. \textbf{\underline{He comes to the Ancient of Days in heaven} to receive dominion and glory and a kingdom}, \textbf{which will be given Him at the close of His work as a mediator}. \textbf{\underline{It is this coming, and not His second advent to the earth, that was foretold in prophecy to take place at the termination of the 2300 days in 1844}}. \textbf{Attended by heavenly angels, our great High Priest enters the holy of holies and there appears in \underline{the presence of God}} to engage in the last acts of His ministration in behalf of man—\textbf{to perform the work of investigative judgment} and to \textbf{make an atonement} for all who are shown to be entitled to its benefits.}[GC 479.3; 1888][https://egwwritings.org/?ref=en\_GC.479.3&para=132.2171]


\egwnogap{\textbf{‘I oto, ktoś podobny do \underline{Syna Człowieczego} przyszedł z obłokami nieba i przystąpił do \underline{Przedwiecznego}, i \underline{przyprowadzili Go przed Niego}}. I dano Mu władzę i chwałę, i królestwo, aby wszystkie ludy, narody i języki służyły Mu: Jego panowanie jest panowaniem wiecznym, które nie przeminie.’ Daniela 7:13, 14. \textbf{Przyjście Chrystusa opisane tutaj nie jest Jego drugim przyjściem na ziemię}. \textbf{\underline{On przychodzi do Przedwiecznego w niebie}, aby otrzymać władzę, chwałę i królestwo}, \textbf{które zostaną Mu dane na zakończenie Jego dzieła jako pośrednika}. \textbf{\underline{To właśnie to przyjście, a nie Jego drugie przyjście na ziemię, było przepowiedziane w proroctwie, że nastąpi po zakończeniu 2300 dni w 1844 roku}}. \textbf{W towarzystwie niebiańskich aniołów nasz wielki Arcykapłan wchodzi do miejsca najświętszego i tam staje \underline{przed obliczem Boga}}, aby zaangażować się w ostatnie akty Jego posługi w imieniu człowieka—\textbf{aby wykonać dzieło sądu śledczego} i \textbf{dokonać pojednania} dla wszystkich, którzy okażą się uprawnieni do jego korzyści.}[GC 479.3; 1888][https://egwwritings.org/?ref=en\_GC.479.3&para=132.2171]


The answer is simple and straightforward: The God of our pioneers was the Ancient of Days. \egwinline{The Ancient of Days is God the Father}. He is \textit{a personal}, \textit{spiritual being}. We see this in His description: \bible{Whose garment was white as snow, and the hair of his head like the pure wool: his throne was like the fiery flame, and his wheels as burning fire.}[Daniel 7:9]. In the termination of the 2300 days prophecy, in 1844, \bible{The hour of His judgment has come}[Revelation 14:7], \bible{the Ancient of days did sit} and \bible{the judgment was set, and the books were opened.}[Daniel 7:9,10]. The God from the first angel’s message is the Ancient of Days. Our pioneers were not ignorant regarding the truth about God. They believed \others{That there is \textbf{one God}, \textbf{\underline{a personal, spiritual being}}, \textbf{the creator of all things}, omnipotent, omniscient, and eternal, infinite in wisdom, holiness, justice, goodness, truth, and mercy; unchangeable, and \textbf{\underline{everywhere present by his representative, the Holy Spirit}}. Ps. 139:7.}[First point of the Fundamental Principles.] This one God is the Father, the Ancient of Days, \others{the creator of all things}, and we are to \bible{worship Him that made heaven, and earth, and the sea, and the fountains of waters}[Revelation 14:7]. He \bible{created all things by Jesus Christ}[Ephesians 3:9].


Odpowiedź jest prosta i bezpośrednia: Bogiem naszych pionierów był Przedwieczny. \egwinline{Przedwieczny to Bóg Ojciec}. Jest On \textit{osobową}, \textit{duchową istotą}. Widzimy to w Jego opisie: \bible{Jego szata była biała jak śnieg, a włosy jego głowy jak czysta wełna; jego tron jak płomienie ognia, a jego koła jak płonący ogień.}[Daniela 7:9]. W zakończeniu proroctwa 2300 dni, w 1844 roku, \bible{Nadeszła godzina Jego sądu}[Objawienie 14:7], \bible{Przedwieczny zasiadł} i \bible{sąd zasiadł i księgi zostały otwarte.}[Daniela 7:9,10]. Bogiem z pierwszego poselstwa anielskiego jest Przedwieczny. Nasi pionierzy nie byli ignorantami w kwestii prawdy o Bogu. Wierzyli \others{Że jest \textbf{jeden Bóg}, \textbf{\underline{osobowa, duchowa istota}}, \textbf{stwórca wszystkich rzeczy}, wszechmocny, wszechwiedzący i wieczny, nieskończony w mądrości, świętości, sprawiedliwości, dobroci, prawdzie i miłosierdziu; niezmienny i \textbf{\underline{wszędzie obecny przez swojego przedstawiciela, Ducha Świętego}}. Ps. 139:7.}[Pierwszy punkt Fundamentalnych Zasad.] Ten jeden Bóg to Ojciec, Przedwieczny, \others{stwórca wszystkich rzeczy}, i mamy \bible{oddawać cześć Temu, który stworzył niebo i ziemię, morze i źródła wód}[Objawienie 14:7]. On \bible{stworzył wszystko przez Jezusa Chrystusa}[Efezjan 3:9].


Today, the first angel’s message has not lost any of its importance. The messages of the second and third angel’s depend on the first message and only the first message requires action on our part. We are to worship God. More specifically, we are to worship the right God. In the last and final conflict, there will be two kinds of worshippers, as we have been told in Revelation 13 and 14.


Dziś pierwsze poselstwo anielskie nie straciło nic ze swojego znaczenia. Poselstwa drugiego i trzeciego anioła zależą od pierwszego poselstwa i tylko pierwsze poselstwo wymaga działania z naszej strony. Mamy oddawać cześć Bogu. Dokładniej, mamy oddawać cześć właściwemu Bogu. W ostatnim i końcowym konflikcie będą dwa rodzaje czcicieli, jak zostaliśmy poinformowani w Objawieniu 13 i 14.


\bible{And all that dwell upon the earth shall \textbf{worship him} \normaltext{[the beast]}, \textbf{whose names are not written in the book of life of the Lamb} slain from the foundation of the world.}[Revelation 13:8]


\bible{I wszyscy mieszkańcy ziemi będą \textbf{oddawać mu pokłon} \normaltext{[bestii]}, \textbf{ci, których imiona nie są zapisane w księdze życia Baranka} zabitego od założenia świata.}[Objawienie 13:8]


The group that worships the beast will receive the mark of the beast. The whole world will be compelled to worship the beast and his image with the threat of death.


Grupa, która oddaje cześć bestii, otrzyma znamię bestii. Cały świat będzie zmuszony do oddawania czci bestii i jej obrazowi pod groźbą śmierci.


\bible{And he \normaltext{[the beast]} had power to give life unto \textbf{the image of the beast}, that the image of the beast should both speak, and cause that \textbf{as many as would not worship the image of the beast should be killed}.}[Revelation 13:15]


\bible{I dał mu \normaltext{[bestii]} moc, aby obraz bestii ożywił, tak żeby \textbf{obraz bestii} przemówił i sprawił, aby \textbf{ci, którzy nie chcieli oddać pokłonu obrazowi bestii, zostali zabici}.}[Objawienie 13:15]


We should not participate in this worship. Let us learn and have faith just like Daniel’s three friends who refused to worship the image of King Nebuchadnezzar. The beast represented in Revelation 13, that extorts the consciences of men by the peril of their lives, is the papacy. Dear friend, don't be fooled. The papal God is a Trinity God. Do not overlook that.


Nie powinniśmy uczestniczyć w tym kulcie. Uczmy się i miejmy wiarę jak trzej przyjaciele Daniela, którzy odmówili oddania pokłonu posągowi króla Nabuchodonozora. Bestia przedstawiona w Objawieniu 13, która wymusza sumienia ludzi pod groźbą śmierci, to papiestwo. Drogi przyjacielu, nie daj się zwieść. Papieski Bóg to Bóg Trójcy. Nie przeocz tego.


We should worship the Ancient of Days as it is proclaimed in the first angel’s message. This is God the Creator who created everything through His Son, Jesus Christ. This is God from the first point of the \emcap{Fundamental Principles}. Our pioneers got this right.


Powinniśmy oddawać cześć Przedwiecznemu, jak głosi poselstwo pierwszego anioła. To Bóg Stwórca, który stworzył wszystko przez swojego Syna, Jezusa Chrystusa. To Bóg z pierwszego punktu \emcap{Fundamentalnych Zasad}. Nasi pionierzy dobrze to zrozumieli.


True understanding of the mission and purpose of the Seventh-day Adventist movement should be conclusive evidence that the Trinity doctrine is a foreign doctrine to us. We’ve ended up where we are today because we have forgotten \egwinline{\textbf{the way the Lord has led us, and \underline{His teaching} in our past history.}}[LS 196.2; 1915][https://egwwritings.org/?ref=en\_LS.196.2] It is very sad to see how our Adventist scholars claim that our pioneers did not correctly understand the doctrine of God. If that were true, our pioneers would have failed to proclaim the first angel's message. They did not fail. We have failed.


Prawdziwe zrozumienie misji i celu ruchu Adwentystów Dnia Siódmego powinno być przekonującym dowodem, że doktryna o Trójcy jest dla nas obcą doktryną. Znaleźliśmy się tam, gdzie jesteśmy dzisiaj, ponieważ zapomnieliśmy \egwinline{\textbf{drogę, którą Pan nas prowadził, i \underline{Jego nauczanie} w naszej przeszłej historii.}}[LS 196.2; 1915][https://egwwritings.org/?ref=en\_LS.196.2] Bardzo smutno jest widzieć, jak nasi adwentystyczni uczeni twierdzą, że nasi pionierzy nie rozumieli poprawnie doktryny Boga. Gdyby to była prawda, nasi pionierzy nie zdołaliby głosić poselstwa pierwszego anioła. Oni nie zawiedli. To my zawiedliśmy.


\others{\textbf{Most of the founders of Seventh-day Adventism would not be able to join the church today if they had to subscribe to the denomination's Fundamental Beliefs}.}\others{\textbf{More specifically, most would not be able to agree to belief number 2, which deals with the doctrine of the Trinity.} For Joseph Bates the Trinity was an unscriptural doctrine, for James White it was that “old Trinitarian absurdity,” and for M. E. Cornell it was a fruit of the great apostasy, along with such false doctrines as Sunday-keeping and the immortality of the soul.}[George Night, Ministry Magazine, October 1993][https://www.ministrymagazine.org/archive/1993/10/adventists-and-change]


\others{\textbf{Większość założycieli Adwentyzmu Dnia Siódmego nie mogłaby dziś przystąpić do kościoła, gdyby musieli zaakceptować Fundamentalne Wierzenia denominacji}.}\others{\textbf{Dokładniej mówiąc, większość nie byłaby w stanie zgodzić się z wierzeniem numer 2, które dotyczy doktryny o Trójcy.} Dla Josepha Batesa doktryna o Trójcy była niebiblijną doktryną, dla Jamesa White'a była to “stara trynitarna niedorzeczność”, a dla M. E. Cornella była owocem wielkiego odstępstwa, wraz z takimi fałszywymi doktrynami jak zachowywanie niedzieli i nieśmiertelność duszy.}[George Night, Ministry Magazine, październik 1993][https://www.ministrymagazine.org/archive/1993/10/adventists-and-change]


The doctrine of Trinity is the doctrine that undermines the foundation of our faith, the foundation that was laid at the beginning of our work. Yet, similarly to Dr. Kellogg, many claim that Sister White promoted the Trinity doctrine and this is why we accepted it. The quotations used to support this claim are mostly taken out of their context. In what follows, we will look at one of the most prominent quotations that supposedly promoted the Trinity doctrine—The Heavenly Trio quotation.


Doktryna o Trójcy jest doktryną, która podważa fundament naszej wiary, fundament, który został położony na początku naszej pracy. Jednak podobnie jak dr Kellogg, wielu twierdzi, że Siostra White promowała doktrynę o Trójcy i dlatego ją przyjęliśmy. Cytaty używane do poparcia tego twierdzenia są w większości wyrwane z kontekstu. W dalszej części przyjrzymy się jednemu z najbardziej znanych cytatów, który rzekomo promował doktrynę o Trójcy - cytatowi o Niebiańskim Trio.


\begin{titledpoem}
\stanza{
    In faith's first light, they sought His face, \\
    A vision pure, of divine grace. \\
    The pioneers, with vision clear, \\
    In 1844, held God so dear.
}

\stanza{    
    "The hour of judgment has come," they cried, \\
    To a world, both far and wide. \\
    The Ancient of Days, they did proclaim, \\
    Not a trinity, but a singular name.
}

\stanza{    
    Ellen White, with pen in hand, \\
    Spoke of a sanctuary, not of earthly land. \\
    A message of heaven, pure and bright, \\
    Guiding the faithful through the night.
}

\stanza{    
    The first angel’s message, a call to revere, \\
    God the Father, whom we should fear. \\
    "Who is the God we are to adore?" \\
    Not a trinity—this, they implored.
}

\stanza{    
    The Trinity, a concept unembraced, \\
    By pioneers, who in God's word traced. \\
    The Father, the Ancient, they did declare, \\
    His judgment and mercy, beyond compare.
}

\stanza{
    Yet, whispers now, through time have spread, \\ 
    A trinity's shadow, causing dread. \\
    If this be the God they were to declare, \\
    Their mission failed, caught in despair.
}

\stanza{
    But this is a falsehood, bold and cold, \\
    A narrative modern, but wrongly told. \\
    The Father they worshiped, with fervent zeal, \\
    Was the true God, their mission real.
}

\stanza{
    In unity, may we seek His face, \\
    Embracing truth, with grace and grace. \\
    The pioneers' vision, let us not lose, \\
    For in their footsteps, we must choose.
}

\stanza{
    To worship the God, of days of old, \\
    The Ancient of Days, as was foretold. \\
    In Revelation’s message, clear and bright, \\
    Guiding us still, through darkest night.
}
\end{titledpoem}

% \begin{titledpoem}

    \stanza{
        Gdy wiary światło zobaczyli, \\
        O Jego łaskę się modlili. \\
        Pionierzy wizje roztaczali, \\
        Boga już wtedy miłowali.
    }

    \stanza{
        Że sąd już nadszedł, wraz krzyczeli, \\
        Ogłosić bowiem światu mieli, \\
        Że to Odwieczny w nim zasiada, \\
        A nie pogańska bogów triada.
    }

    \stanza{
        Zaś Ellen White im opisała \\
        Widzenia nieba, które miała. \\
        Świątynia będzie oczyszczona, \\
        Miłość ma wielką Jezus do nas.
    }

    \stanza{
        Wraz z pierwszym aniołem wołali, \\
        Byśmy się Boga Stwórcy bali. \\
        „Którego Boga czcić to mamy?” — \\
        Nie trójcę, której nie poznamy.
    }

    \stanza{
        Pionierzy w Trójcę nie wierzyli, \\
        Bo w Słowo Boże się wpatrzyli. \\
        Odwieczny Ojciec, to wiedzieli, \\
        Innego Boga mieć nie chcieli.
    }

    \stanza{
        Dziś błędu szept się rozprzestrzenia, \\
        I z Trójcy trudno jest wyjść cienia. \\
        Jeśli ten bóg stoi zwycięsko, \\
        Stała się nasza misja klęską.
    }

    \stanza{
        Lecz to jest fałsz i rażący błąd, \\
        Którego trzeba się pozbyć stąd, \\
        By Bóg, co niegdyś był przez nich czczony \\
        Jako prawdziwy, był przywrócony.
    }

    \stanza{
        Wspólnie szukajmy oblicza Jego, \\
        By zyskać prawdę Boga naszego. \\
        W pionierskiej wizji musimy trwać, \\
        Na ich podstawach do końca stać.
    }

    \stanza{
        Czcijmy więc Boga, Sędziwego, \\
        I prawda niech na temat Jego \\
        Z trzecim aniołem w świat popłynie, \\
        I w mroku nocy nie zaginie.
    }

\end{titledpoem}
