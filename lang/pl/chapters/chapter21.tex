
\chapter{Pamiętając początek} \label{chap:remembering-the-beginning}

\egw{\textbf{Nie możemy ani na chwilę dopuścić do jakiegokolwiek \underline{fałszywego przedstawienia} tych uroczystych i ważnych tematów prawdy, które były wiarą naszego ludu od 1844 roku.}}[Lt300-1903.9; 1903][https://egwwritings.org/?ref=en\_Lt300-1903.9&para=7705.15]

Prawdziwe znaczenie \emcap{Fundamentalnych Zasad} to szersze spojrzenie na poselstwo trzech aniołów.

\egw{\textbf{Jesteśmy zachowującym przykazania ludem Bożym}. Przez ostatnie pięćdziesiąt lat była kierowana przeciwko nam każda forma herezji, aby \textbf{zaciemnić nasze umysły w odniesieniu do nauczania Słowa — \underline{szczególnie odnośnie do służby Chrystusa w niebiańskiej świątyni} i poselstwa niebios na te ostatnie dni, \underline{danego przez aniołów z czternastego rozdziału Księgi Objawienia}}. Poselstwa wszelkiego rodzaju i typu były narzucane Adwentystom Dnia Siódmego, aby \textbf{zastąpić prawdę, która punkt po punkcie} została odkryta przez pełne modlitwy studium i potwierdzona przez cudotwórczą moc Pana. \textbf{Lecz  drogowskazy, które uczyniły nas tym, czym jesteśmy, mają być zachowane i będą zachowane}, jak Bóg oznajmił przez swoje słowo i świadectwo swojego Ducha. \textbf{Wzywa nas, abyśmy trzymali się mocno, z uściskiem wiary, \underline{fundamentalnych zasad}, które są oparte
na niepodważalnym autorytecie}}[SpTB02 59.1; 1904][https://egwwritings.org/?ref=en\_SpTB02.59.1]

Tutaj widzimy, jak Ellen White opisała poselstwo \emcap{Fundamentalnych Zasad} jako poselstwa trzech aniołów z czternastego rozdziału Apokalipsy i jako poselstwo dotyczące służby Chrystusa w niebiańskiej świątyni. Pierwszy punkt \emcap{Fundamentalnych Zasad}, który jest szeroko omawiany tutaj, odpowiada na ważne pytanie zadane przez pierwszego anioła w czternastym rozdziale Apokalipsy: \textit{kim jest Bóg, którego powinniśmy czcić}?

\bible{Bójcie się \textbf{Boga} i \textbf{oddajcie \underline{mu} chwałę}, gdyż \textbf{nadeszła godzina \underline{jego} sądu}. \textbf{Oddajcie pokłon \underline{temu}}, który stworzył niebo i ziemię, morze i źródła wód.}[Objawienie 14:7]

Kim jest Bóg, którego powinniśmy czcić, ogłoszony przez pierwszego anioła? W zakresie czasu znajdujemy różne odpowiedzi na to pytanie. Dziś odpowiedzią jest Bóg Trójjedyny, czyli Trójca, jak przedstawiono w Fundamentalnych Wierzeniach Adwentystów Dnia Siódmego. Ale stawiamy pytanie: kim był Bóg, którego czcili pionierzy adwentyzmu? Poselstwo pierwszego anioła jest związane z czasem proroczym, który wypełnił się w czasach naszych pionierów. Całym celem ich pracy było głoszenie poselstwa trzech aniołów. W 1844 roku nadeszła godzina Bożego sądu. Jeśli Bóg Trójca był Bogiem, którego godzina nadeszła, a nasi pionierzy nie czcili Trójcy, to czy nie ponieśliby porażki w celu założenia tego ruchu?


Let us examine the history of our prophetic movement with this question: did our pioneers worship the true God in proclaiming the message of the first angel? We read the explanation of the events in the passing of 1844.


Zbadajmy historię naszego proroczego ruchu z tym pytaniem: czy nasi pionierzy czcili prawdziwego Boga głosząc poselstwo pierwszego anioła? Czytamy wyjaśnienie wydarzeń po mijającym 1844 roku.

\egw{\textbf{Podobnie jak pierwsi uczniowie, William Miller i jego współpracownicy sami nie pojmowali w pełni znaczenia poselstwa, które głosili}. Błędy, które od dawna zakorzeniły się w kościele, uniemożliwiały im dojście do prawidłowej interpretacji ważnego punktu proroctwa. Dlatego, chociaż głosili poselstwo, które Bóg im powierzył, aby przekazać je światu, to jednak z powodu błędnego zrozumienia jego znaczenia doznali rozczarowania.}[GC 351.2; 1888][https://egwwritings.org/?ref=en\_GC.351.2&para=132.1604]

\egwnogap{Wyjaśniając Daniela 8:14, ‘Aż do \textbf{dwóch tysięcy trzystu wieczorów i poranków. Wtedy \underline{świątynia zostanie oczyszczona}}’, Miller, jak już wspomniano, przyjął powszechnie uznawany pogląd, że ziemia jest świątynią, i wierzył, że oczyszczenie świątyni oznacza oczyszczenie ziemi przez ogień przy przyjściu Pana. Gdy zatem odkrył, że koniec 2300 dni był wyraźnie przepowiedziany, doszedł do wniosku, że to ujawniało czas drugiego przyjścia. Jego błąd wynikał z przyjęcia popularnego poglądu na temat tego, co stanowi świątynię.}[GC 352.1; 1888][https://egwwritings.org/?ref=en\_GC.352.1&para=132.1607]

\egwnogap{W systemie typicznym, który był cieniem ofiary i \textbf{kapłaństwa Chrystusa}, \textbf{oczyszczenie świątyni było ostatnią służbą wykonywaną przez arcykapłana }w corocznym cyklu posługi.\textbf{ Było to końcowe dzieło pojednania—usunięcie lub oddalenie grzechu od Izraela}. \textbf{Zapowiadało to końcowe dzieło w posłudze naszego Arcykapłana w niebie, w usunięciu lub wymazaniu grzechów Jego ludu, które są zapisane w niebiańskich księgach}. \textbf{Ta służba obejmuje dzieło \underline{badania, dzieło sądu}; i bezpośrednio poprzedza przyjście Chrystusa} w obłokach nieba z mocą i wielką chwałą; bo gdy On przyjdzie, każda sprawa została już rozstrzygnięta. Mówi Jezus: ‘Moja nagroda jest ze Mną, aby dać każdemu człowiekowi według jego uczynków.’ Objawienie 22:12. \textbf{To właśnie to dzieło sądu, bezpośrednio poprzedzające drugie przyjście, jest \underline{ogłoszone w  poselstwie pierwszego anioła z Objawienia 14:7}: ‘Bójcie się \underline{Boga} i oddajcie Mu chwałę, \underline{gdyż nadeszła godzina Jego sądu}.}’}[GC 352.2; 1888][https://egwwritings.org/?ref=en\_GC.352.2&para=132.1608]

\egwnogap{\textbf{Ci, którzy głosili to ostrzeżenie, przekazali właściwe poselstwo we właściwym czasie}. Ale tak jak pierwsi uczniowie głosili: ‘Wypełnił się czas i przybliżyło się królestwo Boże’, opierając się na proroctwie Daniela 9, podczas gdy nie dostrzegli, że śmierć Mesjasza była przepowiedziana w tym samym fragmencie Pisma, \textbf{tak Miller i jego współpracownicy głosili poselstwo oparte na \underline{Daniela 8:14 i Objawienia 14:7}, i nie dostrzegli, że były jeszcze inne poselstwa ukazane w Objawieniu 14}, które również miały być głoszone przed przyjściem Pana. Tak jak uczniowie mylili się co do królestwa, które miało być ustanowione na końcu siedemdziesięciu tygodni, tak adwentyści mylili się co do wydarzenia, które miało nastąpić po upływie 2300 dni. W obu przypadkach nastąpiło przyjęcie, a raczej przylgnięcie do popularnych błędów, które zaślepiły umysł na prawdę. Obie grupy wypełniły wolę Bożą, przekazując poselstwo, które On pragnął, aby zostało przekazane, i obie, z powodu własnego niezrozumienia swojego poselstwa, doznały rozczarowania.}[GC 352.3; 1888][https://egwwritings.org/?ref=en\_GC.352.3&para=132.1609]



Czytając wyjaśnienie wielkiego rozczarowania, czy dostrzegłeś odpowiedź na pytanie: “\textit{kim jest Bóg, którego sąd nadszedł}?” Pierwsze poselstwo anielskie z Objawienia 14:7 dokładnie pokrywa się z czasem proroczym ogłoszonym w Daniela 8:14. Sąd, który nadszedł, był sądem śledczym, który rozpoczął się w 1844 roku. Biblia wyraźnie opisuje, czyja godzina sądu nadeszła w pierwszym poselstwie anielskim. Przeczytajmy to w Biblii i zobaczmy komentarz Ellen White.

\egw{‘Patrzyłem,’ mówi prorok Daniel, \textbf{‘aż postawiono trony i \underline{Przedwieczny} \underline{zasiadł}}: \textbf{Jego szata} była biała jak śnieg, a \textbf{włosy Jego głowy} jak czysta wełna; \textbf{Jego tron był jak płomienie ognia}, a koła jego jak płonący ogień. Strumień ognia wypływał i wychodził sprzed Niego: tysiąc tysięcy służyło Mu, a dziesięć tysięcy razy dziesięć tysięcy stało przed Nim: \textbf{\underline{ wyrok został wydany i księgi zostały otwarte}}.’ Daniela 7:9, 10, R.V.}[GC 479.1; 1888][https://egwwritings.org/?ref=en\_GC.479.1&para=132.2169]

\egwnogap{\textbf{Tak oto przedstawiono prorokowi w wizji wielki i uroczysty dzień, w którym charaktery i życie ludzi przejdą ponowny  przegląd  przed Sędzią całej ziemi, a każdemu człowiekowi zostanie oddane ‘według jego uczynków’. \underline{Przedwieczny to Bóg Ojciec}.} Mówi psalmista: \textbf{‘Zanim }góry powstały, zanim ukształtowałeś ziemię i świat, \textbf{od wieków na wieki}, \textbf{Ty jesteś Bogiem}.’ Psalm 90:2. \textbf{\underline{To On, źródło wszelkiego bytu i źródło wszelkiego prawa, będzie przewodniczył w sądzie}}. A święci aniołowie jako słudzy i świadkowie, w liczbie ‘dziesięć tysięcy razy dziesięć tysięcy i tysiące tysięcy’, uczestniczą w tym wielkim trybunale.}[GC 479.2; 1888][https://egwwritings.org/?ref=en\_GC.479.2&para=132.2170]

\egwnogap{\textbf{‘I oto, ktoś podobny do \underline{Syna Człowieczego} przybył z obłokami nieba i przystąpił do \underline{Przedwiecznego}, i \underline{przyprowadzili Go przed Niego}}. I dano Mu panowanie i chwałę, i królestwo, aby wszystkie ludy, narody i języki służyły Mu: Jego panowanie jest panowaniem wiecznym, które nie przeminie.’ Daniela 7:13, 14. \textbf{Przyjście Chrystusa opisane tutaj nie jest Jego drugim przyjściem na ziemię}. \textbf{\underline{On przychodzi do Przedwiecznego w niebie}, aby otrzymać władzę, chwałę i królestwo}, \textbf{które zostaną Mu dane na zakończenie Jego dzieła jako pośrednika}. \textbf{\underline{To właśnie to przyjście, a nie Jego drugie przyjście na ziemię, było przepowiedziane w proroctwie, że nastąpi po zakończeniu 2300 dni w 1844 roku}}. \textbf{W towarzystwie niebiańskich aniołów nasz wielki Arcykapłan wchodzi do miejsca najświętszego i tam staje \underline{przed obliczem Boga}}, aby zaangażować się w ostatnie akty Jego posługi w imieniu człowieka—\textbf{aby wykonać dzieło sądu śledczego} i \textbf{dokonać pojednania} za wszystkich, którzy okażą się uprawnieni do korzystania z jego dobrodziejstw.}[GC 479.3; 1888][https://egwwritings.org/?ref=en\_GC.479.3&para=132.2171]

Odpowiedź jest prosta i bezpośrednia: Bogiem naszych pionierów był Przedwieczny. \egwinline{Przedwieczny to Bóg Ojciec}. Jest On \textit{osobową}, \textit{duchową istotą}. Widzimy to w Jego opisie: \bible{Jego szata była biała jak śnieg, a włosy jego głowy jak czysta wełna; jego tron jak płomienie ognia, a jego koła jak płonący ogień.}[Daniela 7:9]. W zakończeniu proroctwa 2300 dni, w 1844 roku, \bible{Nadeszła godzina Jego sądu}[Objawienie 14:7], \bible{Przedwieczny zasiadł} i \bible{wyrok został ogłoszony i księgi zostały otwarte.}[Daniela 7:9,10]. Bóg z poselstwa pierwszego anioła jest Przedwieczny. Nasi pionierzy nie byli ignorantami w kwestii prawdy o Bogu. Wierzyli \others{Że jest \textbf{jeden Bóg}, \textbf{\underline{osobowa, duchowa istota}}, \textbf{stwórca wszystkich rzeczy}, wszechmocny, wszechwiedzący i wieczny, nieskończony w mądrości, świętości, sprawiedliwości, dobroci, prawdzie i miłosierdziu; niezmienny i \textbf{\underline{wszędzie obecny przez swojego przedstawiciela, Ducha Świętego}}. Ps. 139:7.}[Pierwszy punkt Fundamentalnych Zasad.] Ten jeden Bóg to Ojciec, Przedwieczny, \others{stwórca wszystkich rzeczy}, i mamy \bible{oddawać cześć Temu, który stworzył niebo i ziemię, morze i źródła wód}[Objawienie 14:7]. On \bible{stworzył wszystko przez Jezusa Chrystusa}[Efezjan 3:9].

Dziś poselstwo pierwszego anioła nie straciło nic na swoim znaczeniu. Poselstwa drugiego i trzeciego anioła zależą od pierwszego poselstwa i tylko pierwsze poselstwo wymaga działania z naszej strony. Mamy oddawać cześć Bogu. Dokładniej, mamy oddawać cześć właściwemu Bogu. W ostatnim i końcowym konflikcie będą dwa rodzaje czcicieli, jak zostaliśmy poinformowani w Objawieniu 13 i 14.

\bible{I wszyscy mieszkańcy ziemi będą \textbf{oddawać mu pokłon} \normaltext{[bestii]}, \textbf{ci, których imiona nie są zapisane w księdze życia Baranka} zabitego od założenia świata.}[Objawienie 13:8]

Grupa, która oddaje cześć bestii, otrzyma znamię bestii. Cały świat będzie zmuszony do oddawania czci bestii i jej obrazowi pod groźbą śmierci.

\bible{I dał mu \normaltext{[bestii]} moc, aby obraz bestii ożywił, tak żeby \textbf{obraz bestii} przemówił i sprawił, aby \textbf{ci, którzy nie chcieli oddać pokłonu obrazowi bestii, zostali zabici}.}[Objawienie 13:15]

Nie powinniśmy uczestniczyć w tym kulcie. Uczmy się i miejmy wiarę jak trzej przyjaciele Daniela, którzy odmówili oddania pokłonu posągowi króla Nabuchodonozora. Bestia przedstawiona w Objawieniu 13, która wymusza na ludzkich sumieniach pod groźbą śmierci, to papiestwo. Drogi przyjacielu, nie daj się zwieść. Papieski Bóg to Bóg Trójcy. Nie przeocz tego.

Powinniśmy oddawać cześć Przedwiecznemu, jak głosi poselstwo pierwszego anioła. To Bóg Stwórca, który stworzył wszystko przez swojego Syna, Jezusa Chrystusa. To Bóg z pierwszego punktu \emcap{Fundamentalnych Zasad}. Nasi pionierzy dobrze to rozumieli.

Prawdziwe zrozumienie misji i celu ruchu Adwentystów Dnia Siódmego powinno być przekonującym dowodem, że doktryna o Trójcy jest dla nas obcą doktryną. Znaleźliśmy się tam, gdzie jesteśmy dzisiaj, ponieważ zapomnieliśmy \egwinline{\textbf{drogę, którą Pan nas prowadził, i \underline{Jego nauczanie} w naszej przeszłej historii.}}[LS 196.2; 1915][https://egwwritings.org/?ref=en\_LS.196.2] Bardzo smutno jest widzieć, jak nasi adwentystyczni uczeni twierdzą, że nasi pionierzy nie rozumieli poprawnie doktryny Boga. Gdyby to była prawda, nasi pionierzy nie zdołaliby głosić poselstwa pierwszego anioła. Oni nie zawiedli. To my zawiedliśmy.

\others{\textbf{Większość założycieli Adwentyzmu Dnia Siódmego nie mogłaby dziś przystąpić do kościoła, gdyby musieli zaakceptować Fundamentalne Wierzenia denominacji}.}\others{\textbf{Dokładniej mówiąc, większość nie byłaby w stanie zgodzić się z wierzeniem numer 2, które dotyczy doktryny o Trójcy.} Dla Josepha Batesa doktryna o Trójcy była niebiblijną doktryną, dla Jamesa White'a była to “stara trynitarna niedorzeczność”, a dla M. E. Cornella była owocem wielkiego odstępstwa, wraz z takimi fałszywymi doktrynami jak zachowywanie niedzieli i nieśmiertelność duszy.}[George Night, Ministry Magazine, październik 1993][https://www.ministrymagazine.org/archive/1993/10/adventists-and-change]

Doktryna o Trójcy jest doktryną, która podważa fundament naszej wiary, fundament, który został położony na początku naszej pracy.Rozróżnienie między prawdą a błędem leży w hermeneutyce - metodzie interpretacji Biblii. Przeanalizujmy dokładnie tę kwestię.


% \begin{titledpoem}
\stanza{
    In faith's first light, they sought His face, \\
    A vision pure, of divine grace. \\
    The pioneers, with vision clear, \\
    In 1844, held God so dear.
}

\stanza{    
    "The hour of judgment has come," they cried, \\
    To a world, both far and wide. \\
    The Ancient of Days, they did proclaim, \\
    Not a trinity, but a singular name.
}

\stanza{    
    Ellen White, with pen in hand, \\
    Spoke of a sanctuary, not of earthly land. \\
    A message of heaven, pure and bright, \\
    Guiding the faithful through the night.
}

\stanza{    
    The first angel’s message, a call to revere, \\
    God the Father, whom we should fear. \\
    "Who is the God we are to adore?" \\
    Not a trinity—this, they implored.
}

\stanza{    
    The Trinity, a concept unembraced, \\
    By pioneers, who in God's word traced. \\
    The Father, the Ancient, they did declare, \\
    His judgment and mercy, beyond compare.
}

\stanza{
    Yet, whispers now, through time have spread, \\ 
    A trinity's shadow, causing dread. \\
    If this be the God they were to declare, \\
    Their mission failed, caught in despair.
}

\stanza{
    But this is a falsehood, bold and cold, \\
    A narrative modern, but wrongly told. \\
    The Father they worshiped, with fervent zeal, \\
    Was the true God, their mission real.
}

\stanza{
    In unity, may we seek His face, \\
    Embracing truth, with grace and grace. \\
    The pioneers' vision, let us not lose, \\
    For in their footsteps, we must choose.
}

\stanza{
    To worship the God, of days of old, \\
    The Ancient of Days, as was foretold. \\
    In Revelation’s message, clear and bright, \\
    Guiding us still, through darkest night.
}
\end{titledpoem}

% \begin{titledpoem}

    \stanza{
        Gdy wiary światło zobaczyli, \\
        O Jego łaskę się modlili. \\
        Pionierzy wizje roztaczali, \\
        Boga już wtedy miłowali.
    }

    \stanza{
        Że sąd już nadszedł, wraz krzyczeli, \\
        Ogłosić bowiem światu mieli, \\
        Że to Odwieczny w nim zasiada, \\
        A nie pogańska bogów triada.
    }

    \stanza{
        Zaś Ellen White im opisała \\
        Widzenia nieba, które miała. \\
        Świątynia będzie oczyszczona, \\
        Miłość ma wielką Jezus do nas.
    }

    \stanza{
        Wraz z pierwszym aniołem wołali, \\
        Byśmy się Boga Stwórcy bali. \\
        „Którego Boga czcić to mamy?” — \\
        Nie trójcę, której nie poznamy.
    }

    \stanza{
        Pionierzy w Trójcę nie wierzyli, \\
        Bo w Słowo Boże się wpatrzyli. \\
        Odwieczny Ojciec, to wiedzieli, \\
        Innego Boga mieć nie chcieli.
    }

    \stanza{
        Dziś błędu szept się rozprzestrzenia, \\
        I z Trójcy trudno jest wyjść cienia. \\
        Jeśli ten bóg stoi zwycięsko, \\
        Stała się nasza misja klęską.
    }

    \stanza{
        Lecz to jest fałsz i rażący błąd, \\
        Którego trzeba się pozbyć stąd, \\
        By Bóg, co niegdyś był przez nich czczony \\
        Jako prawdziwy, był przywrócony.
    }

    \stanza{
        Wspólnie szukajmy oblicza Jego, \\
        By zyskać prawdę Boga naszego. \\
        W pionierskiej wizji musimy trwać, \\
        Na ich podstawach do końca stać.
    }

    \stanza{
        Czcijmy więc Boga, Sędziwego, \\
        I prawda niech na temat Jego \\
        Z trzecim aniołem w świat popłynie, \\
        I w mroku nocy nie zaginie.
    }

\end{titledpoem}
