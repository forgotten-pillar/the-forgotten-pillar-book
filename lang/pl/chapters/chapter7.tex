\qrchapter{https://forgottenpillar.com/rsc/pl-fp-chapter7}{Autorytet Fundamentalnych Zasad} \label{chap:authority}

W 10. rozdziale \textit{Special Testimonies} czytamy, jak Bóg ustanowił fundament naszej wiary. Siostra White używała kilku różnych wyrażeń na określenie fundamentu naszej wiary. Odnosiła się do nich jako: \textit{„platformę wiecznej prawdy”}, \textit{„filary naszej wiary”}, \textit{„zasady prawdy”}, \textit{„główne punkty”}, \textit{„drogowskazy”} oraz \textit{„podstawowe zasady”} — wszystkie one odnoszą się do \emcap{Fundamentalnych Zasad}. Na końcu rozdziału potwierdziła wolę Boga, że \egwinline{wzywa nas, abyśmy trzymali się mocno, z uściskiem wiary, \textbf{fundamentalnych zasad}, które są \textbf{oparte na niepodważalnym \underline{autorytecie}}}[SpTB02 59.1; 1904][https://egwwritings.org/read?panels=p417.299].

Autorytet, na którym zostały ustanowione \emcap{Fundamentalne Zasady}, jest niepodważalny. Były one wynikiem głębokiego, żarliwego studium w czasie wielkiego rozczarowania, gdy prawda \egwinline{\textbf{\underline{punkt po punkcie} została odkryta poprzez pełne modlitwy studium i potwierdzona przez \underline{cudotwórczą moc Pana}}}\footnote{Tamże.}. \egwinline{\textbf{W ten sposób zostały pewnie ustanowione \underline{główne punkty naszej wiary}, które wyznajemy dzisiaj}. \textbf{\underline{Punkt po punkcie} został jasno zdefiniowany i wszyscy bracia doszli do zgodności}}[Lt253-1903.4; 1903][https://egwwritings.org/read?panels=p14068.9980010].

Były one wynikiem gorliwego badania Biblii przez naszych pionierów po upływie czasu w 1844 roku. Wraz z postępem ruchu Adwentystów Dnia Siódmego pojawiła się potrzeba ustanowienia organizacji, co zostało zrealizowane w 1863 roku. W 1872 roku Kościół Adwentystów Dnia Siódmego wydał dokument zatytułowany „Deklaracja Fundamentalnych Zasad, nauczanych i przestrzeganych przez Adwentystów Dnia Siódmego”. Był to pierwszy pisemny dokument ogłaszający \emcap{Fundamentalne Zasady} jako publiczne oświadczenie wiary Adwentystów Dnia Siódmego. Dokument ten był publicznym streszczeniem wiary adwentystycznej i określał, \othersnodot{co jest i było, z wielką jednomyślnością, wyznawane przez} lud Adwentystów Dnia Siódmego. Został napisany \othersnodot{aby odpowiadać na pytania} o to, w co wierzą Adwentyści Dnia Siódmego, \othersnodot{aby korygować fałszywe stwierdzenia} i \othersnodot{usuwać błędne wrażenia}[FP1872 3.1; 1872][https://egwwritings.org/read?panels=p928.8].

Do dziś trwa debata nad tym, kto był autorem tego streszczenia, ponieważ pierwotnie w 1872 roku pozostało ono anonimowe. W 1874 roku James White opublikował je w „Signs of the Times”\footnote{\href{https://adventistdigitallibrary.org/adl-364148/signs-times-june-4-1874}{\textit{Fundamental Principles}, „The Signs of the Times”, 4 czerwca 1874, s. 3.}}, a Uriah Smith w „Review and Herald”\footnote{\href{http://documents.adventistarchives.org/Periodicals/RH/RH18741124-V44-22.pdf}{U. Smith, \textit{The Seventh-Day Adventists: A Brief Sketch of Their Origin, Progress, and Principles}, „The Advent Review and Herald of the Sabbath”, 24 października 1874, s. 171.}} — obaj podpisując się własnym nazwiskiem. W 1889 roku Uriah Smith poprawił je, dodając trzy punkty; co zostało wydane w Adventist Yearbook  z jego podpisem. Uriah Smith zmarł w 1903 roku i wszystkie kolejne wydania \emcap{Fundamentalnych Zasad} były drukowane pod jego nazwiskiem. Były one drukowane w rocznikach — co roku od 1905 do 1914\footnote{Aby zobaczyć bardziej szczegółową oś czasu Fundamentalnych Zasad, patrz \hyperref[appendix:timeline]{Dodatek: Fundamentalne Zasady — oś czasu}.}. Siostra White zmarła w 1915 roku i przez następne 17 lat \emcap{Fundamentalne Zasady} nie były drukowane. Pojawiły się ponownie w roczniku z 1931 roku, kiedy to wprowadzono w nich znaczące zmiany.

W 1971 roku LeRoy Froom napisał o oświadczeniu z 1872 roku: \othersnodot{Chociaż pojawiło się anonimowo, w rzeczywistości zostało ułożone przez Smitha}[L. E. Froom, \textit{Movement of Destiny}, Review and Herald Publishing Association, Washington, D.C. 1971, s. 160.]. Niestety nie przedstawił żadnych danych potwierdzających to twierdzenie. Przykro widzieć, jak uczeni pro-trynitarni uważają \emcap{Fundamentalne Zasady} za mało istotne. Ich prawdziwa wartość jest znacznie umniejszana poprzez przypisywanie tych wierzeń małej grupie ludzi, głównie osobistym przekonaniom Jamesa White'a lub Uriaha Smitha, zamiast wierzeniom, które były, \othersnodot{z wielką jednomyślnością, wyznawane przez}[Przedmowa do Fundamentalnych Zasad, 1872.] lud Adwentystów Dnia Siódmego. W 1958 roku czasopismo Ministry Magazine opisało \emcap{Fundamentalne Zasady} w następujący sposób:

\othersnodot{To prawda, że w 1872 roku wydrukowano «Deklarację Fundamentalnych Zasad, nauczanych i przestrzeganych przez Adwentystów Dnia Siódmego», \textbf{lecz nigdy nie zostało ono przyjęte przez denominację i dlatego nie może być uznane za oficjalne}. Najwyraźniej niewielka grupa, \textbf{być może nawet jedna lub dwie osoby, starała się ująć w słowa to, co ich zdaniem było poglądem całego Kościoła}}[R. Anderson, J. A. Buckwalter, L. Kleuser, E. Cleveland, W. Schubert, \textit{Our Declaration of Fundamental Beliefs}, „Ministry Magazine”, styczeń 1958.].

Problematyczne jest to, że nie ma dowodów potwierdzających poparcie stwierdzenia, że \emcap{Fundamentalne Zasady} nie były odzwierciedleniem wiary całego ciała. Wiemy na pewno, że siostra White je popierała i już sam jej wpływ świadczy o tym, że te poglądy były rzeczywiście przyjęte przez denominację — nie licząc nawet faktu, że były one wielokrotnie drukowane przez 42 lata za życia Ellen White.

Ale nie powinno być żadnego sporu o autorstwo \emcap{Fundamentalnych Zasad}. Mamy cytat z wypowiedzi siostry White o tym, kto je napisał. Mówiąc o Uriahu Smisie, siostra White napisała:

\egw{\textbf{Brat Smith był z nami przy powstaniu tego dzieła. On rozumie, jak \underline{my — mój mąż i ja} — prowadziliśmy tę pracę naprzód i w górę krok po kroku i znosiliśmy trudy, ubóstwo i brak środków}. Z nami byli ci pierwsi pracownicy. Przede wszystkim starszy Smith był wraz z moim mężem w jego wczesnej młodości}[Ms54-1890.6; 1890][https://egwwritings.org/read?panels=p7213.15]

\egwnogap{\textbf{\underline{Staliśmy ramię w ramię ze starszym Smithem w tym dziele, podczas gdy Pan ustanawiał fundamentalne zasady}}. \textbf{Musieliśmy nieustannie pracować przeciwko ludziom o jednostronnych poglądach}, którzy uważali, że właściwe relacje biznesowe w odniesieniu do pracy, którą należało wykonać, były dowodem światowości, oraz przeciwko ekscentrykom, którzy przedstawiali się jako zdolni do ponoszenia odpowiedzialności, ale nie można było im zaufać w związku z pracą, aby nie skierowali jej na złe tory. \textbf{Musiał być podejmowany krok po kroku, \underline{nie według ludzkiej mądrości}, lecz według mądrości i wskazówek Tego, który jest zbyt mądry, by się mylić, i zbyt dobry, by wyrządzić nam krzywdę}. \textbf{Było tak wiele elementów, które musiały zostać sprawdzone i wypróbowane. Dziękuję Panu, że starszy Smith, Amadon i Batchellor wciąż żyją. To oni byli częścią naszej rodziny w najtrudniejszych momentach naszej historii}}[Ms54-1890.7; 1890][https://egwwritings.org/read?panels=p7213.16]

Według tego cytatu, kto ustanowił fundamentalne zasady?

\egwinline{\textbf{\underline{Staliśmy ramię w ramię ze starszym Smithem w tym dziele, podczas gdy Pan układał fundamentalne zasady}}}. \textbf{To był Pan!} Ale kto zapisał je jako deklarację naszej wiary? Był to starszy Smith wraz z Jamesem White'em i siostrą White; widzimy to, gdy siostra White mówi: \egwinline{\textbf{staliśmy} ramię w ramię ze starszym Smithem}. To \textit{«my»} jest wyjaśnione w poprzednim akapicie: \egwinline{On \normaltext{[starszy Smith]} rozumie, jak \textbf{my — mój mąż i ja} — prowadziliśmy tę pracę naprzód}. Ten cytat wskazuje na wyraźne zaangażowanie siostry White, gdy Pan ustanawiał \emcap{Fundamentalne Zasady}.

To prawda, że Deklaracja \emcap{Fundamentalnych Zasadach} została napisana przez małą grupę ludzi, a mianowicie starszego Smitha, Jamesa White'a i Ellen White, ale starali się oni wyrazić słowami to, co było prawdziwym poglądem całego ciała Kościoła. Dokładnie przedstawili \emcap{Fundamentalne Zasady} — prawdy otrzymane na początku naszej pracy. Gdyby tak nie było, wówczas ta deklaracja byłaby całkowitym przeciwieństwem tego, za co się podaje. Zostały one napisane, \othersnodot{aby odpowiedzieć na pytania} związane z tym, w co wierzyli Adwentyści Dnia Siódmego, \othersnodot{aby korygować fałszywe stwierdzenia} i \othersnodot{usuwać błędne wrażenia}[FP1872 3.1; 1872][https://egwwritings.org/read?panels=p928.8]. Jeśli ten dokument błędnie przedstawiał stanowisko Adwentystów, dlaczego pozwolono na jego ciągłe przedrukowywanie przez 42 lata? Był przedrukowywany aż do śmierci Ellen White. Gdyby ten dokument błędnie przedstawiał stanowisko Kościoła, czy Ellen White nie podniosłaby swojego głosu przeciwko niemu? Zawsze podnosiła swój głos przeciwko błędnemu przedstawianiu stanowiska Adwentystów Dnia Siódmego, jak to zrobiła w przypadku D. M. Canrighta i Dr. Kellogga. Jeśli \emcap{Fundamentalne Zasady} błędnie przedstawiały stanowisko Adwentystów Dnia Siódmego, to wszystkie kolejne przedruki należałoby przypisać teorii spiskowej. Byłaby to największa teoria spiskowa w historii Kościoła Adwentystów Dnia Siódmego. Harmonia między pismami Ellen White, pionierami adwentyzmu i twierdzeniami zawartymi w Deklaracji \emcap{Fundamentalnych Zasad} świadczy o tym, że ta deklaracja jest dokładnym \othersnodot{podsumowaniem głównych aspektów} adwentystycznej \othersnodot{wiary, co do których istnieje, o ile nam wiadomo, całkowita jednomyślność w całym ciele}[Przedmowa do Fundamentalnych Zasad, 1889.].

Wraz ze śmiercią siostry White w 1915 roku zaprzestano drukowania \emcap{Fundamentalnych Zasad}. Od 1915 roku rocznik nie drukował żadnego oświadczenia wiary aż do 1931 roku. W tym czasie do \emcap{Fundamentalnych Zasad} wprowadzono istotne zmiany. Po raz pierwszy wprowadzono do nich Trójcę. W punkcie 2 i 3 czytamy:

\others{2. \textbf{Że Bóstwo, czyli Trójca, składa się z Wiecznego Ojca, \underline{osobowej, duchowej Istoty}}, wszechmocnej, \textbf{\underline{wszechobecnej}}, wszechwiedzącej, nieskończonej w mądrości i miłości; \textbf{Pana Jezusa Chrystusa, Syna Wiecznego Ojca}, \textbf{przez którego wszystkie rzeczy zostały stworzone} i przez którego zbawienie odkupionych zastępów zostanie dokonane; \textbf{Ducha Świętego, trzeciej osoby Bóstwa}, wielkiej odradzającej mocy w dziele odkupienia. Mt 28:19}

\othersnodot{3. \textbf{Że Jezus Chrystus jest prawdziwym Bogiem, będącym tej samej natury i istoty co Wieczny Ojciec}}[Rocznik denominacji Adwentystów Dnia Siódmego, 1931, s. 377.][https://static1.squarespace.com/static/554c4998e4b04e89ea0c4073/t/59d17eec12abd9c6194cd26d/1506901758727/SDA-YB1931-22+\%28P.+377-380\%29.pdf].

Ta zmiana na rzecz Trójcy pojawiła się szesnaście lat po śmierci siostry White. Porównanie tego oświadczenia z oryginalnymi \emcap{Fundamentalnymi Zasadami} przedstawia kilka uderzających różnic. Ojciec nadal jest osobową, duchową Istotą, stwórcą wszystkich rzeczy, ale nie jest już określany jako „\textit{jeden Bóg}”. Jezus Chrystus nadal jest Synem Wiecznego Ojca, przez którego Ojciec stworzył wszystkie rzeczy; Jezus jest również tej samej natury i istoty co Ojciec. Chociaż były to te same terminy, których użyto do opisania doktryny o \emcap{osobowości Boga} w oryginalnych \emcap{Fundamentalnych Zasadach}, pytamy o znaczenie terminu „\textit{osobowa, duchowa istota}” w odniesieniu do Ojca, jeśli jest On, według nowego oświadczenia, sam przez siebie wszechobecny? Duch Święty nie jest już instrumentem ani środkiem wszechobecności Ojca. Chociaż to oświadczenie używa podobnej retoryki co oryginalne \emcap{Fundamentalne Zasady}, odchodzi od oryginalnej doktryny o obecności i \emcap{osobowości Boga}.

Według LeRoya Frooma to oświadczenie zostało napisane w całości przez Francisa Wilcoxa, za zgodą trzech innych braci (C. H. Watsona, M. E. Kerna i E. R. Palmera)\footnote{L. E. Froom, \textit{Movement of Destiny}, Review and Herald Publishing Association, Washington, D.C. 1971, s. 411, 413--414.}. W niepublikowanym dokumencie \textit{The Seventh-day Adventist Church in Mission: 1919--1979} czytamy, jak starszy Wilcox stworzył to oświadczenie wbrew przekonaniom ciała Kościoła i opublikował je bez ich zgody.

\othersnodot{\textbf{Zdając sobie sprawę, że Komitet Generalnej Konferencji ani żadne inne ciało kościelne nigdy nie zaakceptowałoby dokumentu w formie, w jakiej został napisany}, starszy Wilcox, za pełną świadomością grupy \normaltext{[C. H. Watson, M. E. Kern i E. R. Palmer]}, przekazał to  Oświadczenie bezpośrednio do Edsona Rogersa, statystyka Generalnej Konferencji, który opublikował je w wydaniu rocznika z 1931 roku, gdzie pojawia się do dziś. Było to zatem bez oficjalnej akceptacji Komitetu Generalnej Konferencji i bez żadnego formalnego przyjęcia przez denominację, gdy oświadczenie starszego Wilcoxa stało się przyjętą deklaracją naszej wiary}[B. Dwyer, \textit{A New Statement of Fundamental Beliefs (1980)}, „Spectrum Magazine” [online], 7 czerwca 2009, \href{https://spectrummagazine.org/news/new-statement-fundamental-beliefs-1980/}{spectrummagazine.org/news/new-statement-fundamental-beliefs-1980} [dostęp: 30 stycznia 2025].].

W 1980 roku dokonano ostatecznej zmiany w publicznym streszczeniu wiary Kościoła Adwentystów Dnia Siódmego. Generalna Konferencja przegłosowała przyjęcie dzisiejszego oficjalnego oświadczenia:

\othersnodot{\textbf{Jest jeden Bóg: Ojciec, Syn i Duch Święty, jedność trzech odwiecznie współistniejących Osób}. Bóg jest nieśmiertelny, wszechmogący, wszechwiedzący, ponad wszystkim i \textbf{wszechobecny}. Jest On nieskończony i niepojęty dla ludzkiego rozumu, a jednak poznany dzięki temu, że sam siebie objawił. Godzien jest wiecznej czci, chwały i służby całego stworzenia}[\textit{Seventh-day Adventists Believe: A Biblical Exposition of 27 Fundamental Doctrines}, Ministerial Association of the General Conference of Seventh-day Adventists, Washington, DC, 1988, s. 16.].

W tym krótkim przeglądzie historycznym widzimy, że oświadczenie z 1931 roku stanowi „pośredni krok” między oryginalną wiarą adwentystyczną a pełnym trynitarnym wierzeniem.

Zmiana w naszych przekonaniach następowała z biegiem czasu poprzez wiele dyskusji. Nasza adwentystyczna historia pozostawiła ślad tych zmian. Jeśli jesteśmy szczerymi poszukiwaczami prawdy, powinniśmy szczegółowo zbadać tę kwestię. Czy możemy dostrzec w naszej adwentystycznej historii, dlaczego porzuciliśmy pierwszy punkt \emcap{Fundamentalnych Zasad} na rzecz doktryny o Trójcy? Z całą pewnością! W kolejnych rozważaniach przyjrzymy się pewnym historycznym dokumentom, które pokazują, dlaczego przeszliśmy od pierwszego punktu \emcap{Fundamentalnych Zasad}, utrzymywanych we wczesnych latach, do przyjęcia doktryny o Trójcy. Podczas tych studiów zachęcamy do rozważenia w  modlitwie i oceny tych zmian w kontekście własnych przekonań.

% The authority of the Fundamental Principles

\begin{titledpoem}
\stanza{
    Fundamentalne Zasady, Boży dar, \\
    Ustanowione mocą niebios czar. \\
    Punkt po punkcie, w modlitwie odkryte, \\
    Przez Pana samego potwierdzone i obfite.
}

\stanza{
    Stali ramię w ramię, Smith, White i żona, \\
    Gdy prawda przez Boga była objawiona. \\
    Nie ludzką mądrością, lecz Bożą mocą, \\
    Zasady te świecą w ciemności jak gwiazdy nocą.
}

\stanza{
    Po śmierci Ellen White zmiany nastały, \\
    Nowe doktryny się w wierze pojawiały. \\
    Trójca wkroczyła do zasad wiary, \\
    Choć pionierzy mieli inne poglądy i miary.
}
\end{titledpoem}

\begin{titledpoem}
\stanza{
    Autorytet zasad niepodważalny trwa, \\
    W nich Boża mądrość i prawda się tka. \\
    Filary wiary, drogowskazy jasne, \\
    Przez pionierów odkryte, przez Boga dane własne.
}

\stanza{
    Wilcox zmienił to, co było święte, \\
    Bez zgody kościoła, w cichości przyjęte. \\
    Szesnaście lat po White, nowa doktryna, \\
    Czy to była Boża czy ludzka przyczyna?
}

\stanza{
    Wróćmy do źródeł, do prawdy pierwotnej, \\
    Do zasad przez Boga ustanowionych, gotowej. \\
    Badajmy historię, szukajmy prawdy szczerze, \\
    By nasza wiara na pewnym stała fundamencie.
}
\end{titledpoem}

\begin{titledpoem}
\stanza{
    Pan sam układał fundamentalne zasady, \\
    Gdy pionierzy szukali prawdy bez zdrady. \\
    Ellen White stała z mężem i Smithem ramię w ramię, \\
    Gdy Bóg rozpalał w ich sercach prawdy płomień.
}

\stanza{
    Czterdzieści dwa lata zasady te trwały, \\
    Aż do śmierci prorokini niezmienne zostały. \\
    Potem przyszły zmiany, nowe definicje, \\
    Trójca zastąpiła dawne Boże pozycje.
}

\stanza{
    Czy to, co zmieniono, było wolą Pana? \\
    Czy ludzka mądrość została tu dodana? \\
    Módlmy się o światło, badajmy z uwagą, \\
    By prawda, nie tradycja, była naszą przewagą.
}
\end{titledpoem}
