\chapter{The bottom of the issue}


\chapter{Sedno sprawy}


Today, when we compare our current Fundamental Beliefs with the previous \emcap{Fundamental Principles} we see the change in the foundation of Seventh-day Adventist faith. This change has occurred in the understanding of God’s person, or the \emcap{personality of God}. Particular to the \emcap{personality of God}, Sister White wrote that the track of truth lies close beside the track of error:


Dziś, gdy porównujemy nasze obecne Fundamentalne Wierzenia z wcześniejszymi \emcap{Fundamentalnymi Zasadami}, widzimy zmianę w fundamencie wiary Adwentystów Dnia Siódmego. Ta zmiana nastąpiła w zrozumieniu osoby Boga, czyli \emcap{osobowości Boga}. W odniesieniu do \emcap{osobowości Boga}, Siostra White napisała, że ścieżka prawdy leży blisko ścieżki błędu:


\egw{\textbf{The track of truth lies close beside the track of error}, and both tracks may \textbf{seem }to be one to minds which are not worked by the Holy Spirit, and which, therefore, \textbf{are not quick to discern the difference between truth and error}.}[SpTB02 52.2; 1904][https://egwwritings.org/?ref=en\_SpTB02.52.2]


\egw{\textbf{Ścieżka prawdy leży blisko ścieżki błędu}, i obie ścieżki mogą \textbf{wydawać się} jedną dla umysłów, które nie są prowadzone przez Ducha Świętego, i które dlatego \textbf{nie są zdolne szybko rozpoznać różnicy między prawdą a błędem}.}[SpTB02 52.2; 1904][https://egwwritings.org/?ref=en\_SpTB02.52.2]


We ask ourselves, how can we draw a clear line between these two tracks? In order to do that we need to get to the bottom of the issue. We need to find a distinguishing principle that separates these two tracks.


Zadajemy sobie pytanie, jak możemy wytyczyć wyraźną linię między tymi dwiema ścieżkami? Aby to zrobić, musimy dotrzeć do sedna sprawy. Musimy znaleźć wyróżniającą zasadę, która oddziela te dwie ścieżki.


By studying our current Trinitarian belief and the works of our pioneers regarding the \emcap{personality of God}, we have found one characterizing principle that distinguishes the truth on the \emcap{personality of God}, as held by our pioneers, from our current Trinitarian belief. Both sides claim the Bible to be their ultimate authority, yet differences can be discerned by the interpretation of the Bible. In the following, we are talking about understanding and interpreting Scripture concerning God’s person. Understanding God’s person can be presented in two distinct, mutually exclusive understandings, which clearly draw a line between the two opposing camps.


Studiując nasze obecne trynitarne wierzenie i prace naszych pionierów dotyczące \emcap{osobowości Boga}, znaleźliśmy jedną charakterystyczną zasadę, która odróżnia prawdę o \emcap{osobowości Boga}, wyznawaną przez naszych pionierów, od naszego obecnego trynitarnego wierzenia. Obie strony twierdzą, że Biblia jest ich ostatecznym autorytetem, jednak różnice można dostrzec w interpretacji Biblii. W dalszej części mówimy o zrozumieniu i interpretacji Pisma Świętego dotyczącego osoby Boga. Zrozumienie osoby Boga można przedstawić w dwóch odrębnych, wzajemnie wykluczających się ujęciach, które wyraźnie wyznaczają linię między dwoma przeciwnymi obozami.


One, more popular, view is that God presented Himself in a language that is familiar to us in order to explain only the concepts of salvation. So, God presented Himself in words such as ‘\textit{father}’, ‘\textit{son}’, and ‘\textit{spirit}’, to describe the relationships between these concepts. This makes none of these words interpretable by their obvious meaning; rather, they hold symbolic or metaphoric value. The principle behind this reasoning is: \textbf{God adjusted Himself to man}.


Jeden, bardziej popularny pogląd jest taki, że Bóg przedstawił siebie w języku, który jest nam znany, aby wyjaśnić tylko koncepcje zbawienia. Więc Bóg przedstawił siebie w słowach takich jak ‘\textit{ojciec}’, ‘\textit{syn}’ i ‘\textit{duch}’, aby opisać relacje między tymi koncepcjami. To sprawia, że żadne z tych słów nie jest interpretowane w ich oczywistym znaczeniu; raczej mają one wartość symboliczną lub metaforyczną. Zasada stojąca za tym rozumowaniem to: \textbf{Bóg dostosował siebie do człowieka}.


The other, opposing, view is that \textbf{God adjusted man to Himself}; \textit{He created man in His own image}. Therefore, words like ‘\textit{father}’, ‘\textit{son}’, and ‘\textit{spirit}’, as they address God, imply their obvious meaning. This is the fundamental difference.


Drugi, przeciwny pogląd jest taki, że \textbf{Bóg dostosował człowieka do siebie}; \textit{stworzył człowieka na swój własny obraz}. Dlatego słowa takie jak ‘\textit{ojciec}’, ‘\textit{syn}’ i ‘\textit{duch}’, gdy odnoszą się do Boga, implikują ich oczywiste znaczenie. To jest fundamentalna różnica.


When we come to understand biblical terms like ‘\textit{person}’, ‘\textit{father}’, ‘\textit{son}’ and ‘\textit{spirit}’, we must choose which view we support and apply it accordingly. Either these terms are understood in their obvious meaning, or symbolically or metaphorically. There is no middle ground between these two; we must choose one. The following quotation should settle any dilemma.


Kiedy dochodzimy do zrozumienia biblijnych terminów takich jak ‘\textit{osoba}’, ‘\textit{ojciec}’, ‘\textit{syn}’ i ‘\textit{duch}’, musimy wybrać, który pogląd popieramy i zastosować go odpowiednio. Albo te terminy są rozumiane w ich oczywistym znaczeniu, albo symbolicznie czy metaforycznie. Nie ma tu środkowego gruntu; musimy wybrać jedno. Poniższy cytat powinien rozstrzygnąć wszelkie dylematy.


\egw{\textbf{\underline{The language of the Bible should be explained according to its obvious meaning, unless a symbol or figure is employed}}.}[GC 598.3; 1888][https://egwwritings.org/?ref=en\_GC88.598.3&para=133.2717]


\egw{\textbf{\underline{Język Biblii powinien być wyjaśniany zgodnie z jego oczywistym znaczeniem, chyba że użyty jest symbol lub figura}}.}[GC 598.3; 1888][https://egwwritings.org/?ref=en\_GC88.598.3&para=133.2717]


We believe that it is impossible for the Bible to be its own interpreter and not explain its own symbols. If the Bible applies the word ‘father’ to God, but never explains this term, then it should be accepted in its obvious meaning. The same applies to the words ‘son’ and ‘spirit’. Man is created in the image of God. God adjusted man to Himself. The obvious meaning is derived from the experience of man. We understand the obvious meaning of the word ‘father’ through regular, human fathership. But our fathership is the image of our God Who is the Father to His Son. Paul testified:


Wierzymy, że niemożliwe jest, aby Biblia była swoim własnym interpretatorem i nie wyjaśniała swoich własnych symboli. Jeśli Biblia stosuje słowo ‘ojciec’ do Boga, ale nigdy nie wyjaśnia tego terminu, to powinno być ono przyjęte w jego oczywistym znaczeniu. To samo dotyczy słów ‘syn’ i ‘duch’. Człowiek jest stworzony na obraz Boga. Bóg dostosował człowieka do siebie. Oczywiste znaczenie wywodzi się z doświadczenia człowieka. Rozumiemy oczywiste znaczenie słowa ‘ojciec’ poprzez zwykłe, ludzkie ojcostwo. Ale nasze ojcostwo jest obrazem naszego Boga, który jest Ojcem dla swojego Syna. Paweł zaświadczył:


\bible{For this cause I bow my knees unto \textbf{the Father of our Lord Jesus Christ, Of whom the whole \underline{family} in heaven and earth is \underline{named}}}[Ephesians 3:14-15].


\bible{Dlatego zginam kolana przed \textbf{Ojcem naszego Pana Jezusa Chrystusa, Od którego cała \underline{rodzina} w niebie i na ziemi bierze swoje \underline{imię}}}[Efezjan 3:14-15].


In Greek, the word ‘\textit{family}’ is the word ‘\textit{patria}’, derived from the word ‘\textit{pater}’, which means ‘\textit{father}’. Some translations even render this verse with \bible{Of whom all \textbf{paternity} in heaven and earth is named} (DRB), which is a more literal translation. The Father of our Lord Jesus Christ is truly the father to His Son, just as truly as we are fathers to our children on Earth. Our paternity on Earth is named according to Paternity in Heaven, where God is the Father of our Lord Jesus Christ. Our earthly paternity is an image of Heavenly Paternity, where God is the Father to His Son. This supports the obvious meaning that Jesus is truly the Son of our God.


W języku greckim słowo ‘\textit{rodzina}’ to słowo ‘\textit{patria}’, pochodzące od słowa ‘\textit{pater}’, które oznacza ‘\textit{ojciec}’. Niektóre tłumaczenia oddają ten werset jako \bible{Od którego wszelkie \textbf{ojcostwo} w niebie i na ziemi bierze swoje imię} (DRB), co jest bardziej dosłownym tłumaczeniem. Ojciec naszego Pana Jezusa Chrystusa jest prawdziwie ojcem dla swojego Syna, tak samo prawdziwie jak my jesteśmy ojcami dla naszych dzieci na Ziemi. Nasze ojcostwo na Ziemi jest nazwane według Ojcostwa w Niebie, gdzie Bóg jest Ojcem naszego Pana Jezusa Chrystusa. Nasze ziemskie ojcostwo jest obrazem Niebiańskiego Ojcostwa, gdzie Bóg jest Ojcem dla swojego Syna. To potwierdza oczywiste znaczenie, że Jezus jest prawdziwie Synem naszego Boga.


The same underlying principle applies to the understanding behind the word ‘\textit{spirit}’ and the word ‘\textit{being}’. God adjusted man to Himself; He created man in His own image. Man is a being, possessing body and spirit, just like God—and in saying this, we are not saying that man and God possess the same nature. God formed man from the dust of the ground. His physical nature is confined to the elements found on the earth. We do not pry into the nature of God. That will forever remain a mystery to us; it is not revealed unto us. But what is revealed to us is that He has a form, and the form of a man is an image of the form of God. The Bible plainly approves this understanding when describing God sitting upon His throne:


Ta sama podstawowa zasada odnosi się do zrozumienia słowa ‘\textit{duch}’ i słowa ‘\textit{istota}’. Bóg dostosował człowieka do Siebie; stworzył człowieka na swój własny obraz. Człowiek jest istotą, posiadającą ciało i ducha, tak jak Bóg—i mówiąc to, nie twierdzimy, że człowiek i Bóg posiadają tę samą naturę. Bóg ukształtował człowieka z prochu ziemi. Jego fizyczna natura jest ograniczona do pierwiastków znajdujących się na ziemi. Nie wnikamy w naturę Boga. To na zawsze pozostanie dla nas tajemnicą; nie jest nam to objawione. Ale to, co jest nam objawione, to fakt, że On ma formę, a forma człowieka jest obrazem formy Boga. Biblia wyraźnie potwierdza to zrozumienie, opisując Boga siedzącego na swoim tronie:


\bible{\textbf{upon the likeness of the throne was the likeness as \underline{the appearance of a man}} above upon it}[Ezekiel 1:26].


\bible{\textbf{a na podobieństwie tronu było podobieństwo \underline{z wyglądu jak człowiek}} na nim z wierzchu}[Ezechiela 1:26].


The obvious meaning of the word ‘\textit{spirit}’, applied to the Spirit of God, is derived from the understanding of “\textit{the spirit of man}”. God adjusted man to Himself; He created man in His own image. Just as man possesses a spirit, God possesses a Spirit. The spirit of man has the nature of man, and the spirit of God has the nature of God. With respect to their nature, they are not the same, but respective of their relation to their inner being, they are the same; the Bible puts them on the same level. \bible{\textbf{The \underline{Spirit} itself beareth witness with \underline{our spirit}}, that we are the children of God:}[Romans 8:16]; \bible{For what \textbf{man knoweth the things of a man}, save the \textbf{\underline{spirit of man} which is in him}? \textbf{\underline{even so}} the things of \textbf{God knoweth} no man, \textbf{but \underline{the Spirit of God}}.}[1 Corinthians 2:11].


Oczywiste znaczenie słowa ‘\textit{duch}’, zastosowane do Ducha Bożego, wywodzi się ze zrozumienia “\textit{ducha człowieka}”. Bóg dostosował człowieka do Siebie; stworzył człowieka na swój własny obraz. Tak jak człowiek posiada ducha, Bóg posiada Ducha. Duch człowieka ma naturę człowieka, a duch Boga ma naturę Boga. W odniesieniu do ich natury, nie są one takie same, ale w odniesieniu do ich relacji do ich wewnętrznej istoty, są takie same; Biblia stawia je na tym samym poziomie. \bible{\textbf{Ten \underline{Duch} świadczy wraz z \underline{naszym duchem}}, że jesteśmy dziećmi Bożymi:}[Rzymian 8:16]; \bible{Bo kto z \textbf{ludzi wie, co jest w człowieku}, oprócz \textbf{\underline{ducha ludzkiego}}, który w nim jest? \textbf{\underline{Tak samo}} i tego, co jest w \textbf{Bogu, nikt nie zna}, \textbf{oprócz \underline{Ducha Bożego}}.}[1 Koryntian 2:11].


In terms of family relationships and the quality or state of being a person, man and God are alike, because God created man in His own image. God adjusted man unto Himself. But in their nature, God and man are not alike. God is divine and man is earthly.


Pod względem relacji rodzinnych i właściwości lub stanu jako osoby, człowiek i Bóg są podobni, ponieważ Bóg stworzył człowieka na swój własny obraz. Bóg dostosował człowieka do Siebie. Ale w swojej naturze, Bóg i człowiek nie są podobni. Bóg jest boski, a człowiek jest ziemski.


The Trinity doctrine adheres to the understanding that God adjusted Himself to man, and that God merely used the terms ‘\textit{father}’, ‘\textit{son}’ and ‘\textit{spirit}’ so that we might understand Him better. This idea underpins and drives the trinitarian paradigm. In what follows, we will not extensively examine our Trinitarian literature, but will support our claim by a few official statements from the Seventh-day Adventist Church.


Doktryna o Trójcy opiera się na zrozumieniu, że Bóg dostosował Siebie do człowieka i że Bóg po prostu użył terminów ‘\textit{ojciec}’, ‘\textit{syn}’ i ‘\textit{duch}’, abyśmy mogli Go lepiej zrozumieć. Ta idea stanowi podstawę i napędza trynitarny paradygmat. W dalszej części nie będziemy szczegółowo analizować naszej trynitarnej literatury, ale potwierdzimy nasze twierdzenie kilkoma oficjalnymi oświadczeniami Kościoła Adwentystów Dnia Siódmego.


The first statement comes from the Biblical Research Institute, the official institution of the General Conference, which promotes the teachings and doctrines of the Seventh-day Adventist Church. They openly negate the parental relationship between the Father and His Son, in favour of a metaphorical understanding.


Pierwsze oświadczenie pochodzi z Instytutu Badań Biblijnych, oficjalnej instytucji Generalnej Konferencji, która promuje nauki i doktryny Kościoła Adwentystów Dnia Siódmego. Otwarcie negują oni rodzicielską relację między Ojcem a Jego Synem, na rzecz metaforycznego zrozumienia.


\others{The father-son image \textbf{cannot be literally applied to the divine Father-Son relationship} within the Godhead. \textbf{The Son is not the natural, literal Son of the Father} ... \textbf{The term ‘Son’ is used metaphorically} when applied to the Godhead.}[Adventist Biblical Research Institute; also published in the official ‘Adventist World’ magazine][https://www.adventistbiblicalresearch.org/materials/a-question-of-sonship/]


\others{Obraz ojca-syna \textbf{nie może być dosłownie zastosowany do boskiej relacji Ojciec-Syn} w Bóstwie. \textbf{Syn nie jest naturalnym, dosłownym Synem Ojca} ... \textbf{Termin ‘Syn’ jest używany metaforycznie}, gdy odnosi się do Bóstwa.}[Adwentystyczny Instytut Badań Biblijnych; opublikowane również w oficjalnym czasopiśmie ‘Adventist World’][https://www.adventistbiblicalresearch.org/materials/a-question-of-sonship/]


Regarding the \emcap{personality of God}, in the context of the trinitarian paradigm, the Seventh-day Adventist church issued the following statements in a Sabbath school lesson:


Odnośnie \emcap{osobowości Boga}, w kontekście trynitarnego paradygmatu, Kościół Adwentystów Dnia Siódmego wydał następujące oświadczenia w lekcji szkoły sobotniej:


\others{\textbf{The \underline{word persons} used in the title of today's lesson \underline{must be understood in a theological sense}}. \textbf{If we equate human personality with God, we would say that three persons means three individuals. But then we would have three Gods, or tritheism}. \textbf{But \underline{historic Christianity} has given to the word person, when used of God, \underline{a special meaning}}: a personal self-distinction, which gives distinctiveness in the Persons of the Godhead without destroying the concept of oneness. \textbf{This idea is not easy to grasp or to explain! \underline{It is part of the mystery of the Godhead}}.}[“Lesson 3.” Ssnet.org, 2025, \href{http://www.ssnet.org/qrtrly/eng/98d/less03.html}{www.ssnet.org/qrtrly/eng/98d/less03.html}. Accessed 3 Feb. 2025.]


\others{\textbf{Słowo \underline{osoby} użyte w tytule dzisiejszej lekcji \underline{musi być rozumiane w sensie teologicznym}}. \textbf{Jeśli zrównamy ludzką osobowość z Bogiem, powiedzielibyśmy, że trzy osoby oznaczają trzech jednostek. Ale wtedy mielibyśmy trzech Bogów, czyli triteizm}. \textbf{Ale \underline{historyczne chrześcijaństwo} nadało słowu osoba, gdy jest używane w odniesieniu do Boga, \underline{specjalne znaczenie}}: osobiste samo-rozróżnienie, które nadaje odrębność Osobom Bóstwa bez niszczenia koncepcji jedności. \textbf{Ta idea nie jest łatwa do uchwycenia ani wyjaśnienia! \underline{Jest to część tajemnicy Bóstwa}}.}[“Lesson 3.” Ssnet.org, 2025, \href{http://www.ssnet.org/qrtrly/eng/98d/less03.html}{www.ssnet.org/qrtrly/eng/98d/less03.html}. Accessed 3 Feb. 2025.]


\others{These texts and others lead us to believe that \textbf{our wonderful God is \underline{three Persons in one},} a mind-boggling \textbf{mystery }but a truth we accept by faith because Scripture reveals it.}[Ibid.]


\others{Te teksty i inne prowadzą nas do przekonania, że \textbf{nasz wspaniały Bóg jest \underline{trzema Osobami w jednej},} oszałamiającą \textbf{tajemnicą}, ale prawdą, którą przyjmujemy przez wiarę, ponieważ Pismo Święte ją objawia.}[Ibid.]


According to official statements presented in the Sabbath School Lesson, the word \textit{‘persons’},\textit{ }in regard to God, should not be equated with human personality, but should be applied in the theological sense. This is in sharp contrast to the vision Sister White had regarding the \emcap{personality of God}. \egwinline{‘I have often seen the lovely Jesus, that \textbf{He is a person}. I asked Him if \textbf{His Father was a person}, and \textbf{had \underline{a form} like Himself}. Said Jesus, ‘\textbf{I am the express image of My Father’s person!}’ [Hebrews 1:3.]}[Lt253-1903.12; 1903][https://egwwritings.org/?ref=en\_Lt253-1903.12&para=9980.18] Her understanding of the quality or state of God being a person is that God is a person in an obvious way—He possesses a form. In the same way she recognized Jesus to be a person, Jesus testified that God is a person, having a form just as He has. Contrary to the obvious and literal view is a spiritual view. She continues to address the error of the spiritual view. \egwinline{\textbf{I have often seen that \underline{the spiritual view} took away all the glory of heaven, and that in many minds the throne of David and the lovely person of Jesus have been burned up in the fire of spiritualism}. I have seen that some who have been deceived and led into this error, will be brought out into the light of truth, \textbf{but it will be almost impossible for them to get entirely rid of the deceptive power of spiritualism. Such should make thorough work in confessing their errors, and leaving them forever}.}[Lt253-1903.13; 1903][https://egwwritings.org/?ref=en\_Lt253-1903.13&para=9980.19] According to the Sabbath School Lesson, the obvious understanding of the term \textit{‘person’ }is incorrect because this would \others{\textbf{equate human personality with God}}, meaning that \others{\textbf{three persons means three individuals}}. Opposite to the obvious view is the theological view. For Sister White, the opposite is the spiritual view. This view takes \egwinline{away all the glory of heaven, and that in many minds the throne of David and the lovely person of Jesus have been burned up in the fire of spiritualism}. In the writings of our pioneers, previously examined, we recognize the truthfulness of her claim. The presented theological view of God’s person does away with the truth on the \emcap{personality of God} that Sister White received in a vision. The theological view is explained as one God, Who is a person, yet three persons, made up of three distinct Gods—God the Father, God the Son, and God the Holy Ghost. The Bible never explains God with such a quality or state of being a person. It is simply presumed by trinitarian believers and, because it is never explained, is deemed a mystery of God, but in fact—it is an error.


Zgodnie z oficjalnymi oświadczeniami przedstawionymi w lekcji szkoły sobotniej, słowo \textit{‘osoby’},\textit{ }w odniesieniu do Boga, nie powinno być utożsamiane z ludzką osobowością, ale powinno być stosowane w sensie teologicznym. Jest to w ostrym kontraście do wizji, jaką Siostra White miała odnośnie \emcap{osobowości Boga}. \egwinline{‘Często widziałam umiłowanego Jezusa, że \textbf{On jest osobą}. Zapytałam Go, czy \textbf{Jego Ojciec jest osobą} i \textbf{czy \underline{ma formę} podobną do Niego}. Jezus powiedział: ‘\textbf{Jestem wyrazem istoty Mojego Ojca!}’ [Hebrajczyków 1:3.]}[Lt253-1903.12; 1903][https://egwwritings.org/?ref=en\_Lt253-1903.12&para=9980.18] Jej zrozumienie właściwości lub stanu Boga jako osoby jest takie, że Bóg jest osobą w oczywisty sposób—On posiada formę. W taki sam sposób, w jaki rozpoznała Jezusa jako osobę, Jezus zaświadczył, że Bóg jest osobą, mającą formę tak jak On. Przeciwieństwem oczywistego i dosłownego poglądu jest duchowy pogląd. Ona kontynuuje, odnosząc się do błędu duchowego poglądu. \egwinline{\textbf{Często widziałam, że \underline{duchowy pogląd} odebrał całą chwałę nieba, i że w wielu umysłach tron Dawida i umiłowana osoba Jezusa zostały spalone w ogniu spirytualizmu}. Widziałam, że niektórzy, którzy zostali zwiedzeni i doprowadzeni do tego błędu, zostaną wyprowadzeni na światło prawdy, \textbf{ale będzie prawie niemożliwe, aby całkowicie pozbyć się zwodniczej mocy spirytualizmu. Tacy powinni dokładnie wyznać swoje błędy i porzucić je na zawsze}.}[Lt253-1903.13; 1903][https://egwwritings.org/?ref=en\_Lt253-1903.13&para=9980.19] Według lekcji szkoły sobotniej, oczywiste zrozumienie terminu \textit{‘osoba’ }jest niepoprawne, ponieważ to \others{\textbf{zrównałoby ludzką osobowość z Bogiem}}, co oznacza, że \others{\textbf{trzy osoby oznaczają trzech jednostek}}. Przeciwieństwem oczywistego poglądu jest pogląd teologiczny. Dla Siostry White przeciwieństwem jest duchowy pogląd. Ten pogląd \egwinline{odbiera całą chwałę nieba, i w wielu umysłach tron Dawida i umiłowana osoba Jezusa zostały spalone w ogniu spirytualizmu}. W pismach naszych pionierów, wcześniej zbadanych, rozpoznajemy prawdziwość jej twierdzenia. Przedstawiony teologiczny pogląd na osobę Boga odrzuca prawdę o \emcap{osobowości Boga}, którą Siostra White otrzymała w wizji. Pogląd teologiczny jest wyjaśniany jako jeden Bóg, który jest osobą, a jednak trzema osobami, składającymi się z trzech odrębnych Bogów—Boga Ojca, Boga Syna i Boga Ducha Świętego. Biblia nigdy nie wyjaśnia Boga z taką właściwością lub stanem bycia osobą. Jest to po prostu zakładane przez wierzących w Trójcę i, ponieważ nigdy nie jest wyjaśnione, jest uważane za tajemnicę Boga, ale w rzeczywistości—jest to błąd.


When we draw the line between truth and error, we also need to draw the line between the things that are mystery and those that are revealed. Regarding the nature of God, silence is eloquence. Unfortunately, many who are advocating the Trinity doctrine fail to draw this line in the proper place. We protest that the \emcap{personality of God}, that is the quality or state of God being a person, is a mystery. Our pioneers understood it and they clearly explained it from the Bible. If they did not read and accept the Bible in its plain and simple language, they wouldn’t be able to explain the \emcap{personality of God}.


Kiedy wyznaczamy granicę między prawdą a błędem, musimy również wyznaczyć granicę między rzeczami, które są tajemnicą, a tymi, które są objawione. Jeśli chodzi o naturę Boga, milczenie jest wymowne. Niestety, wielu, którzy opowiadają się za doktryną o Trójcy, nie wyznacza tej granicy we właściwym miejscu. Protestujemy przeciwko temu, że \emcap{osobowość Boga}, czyli właściwość lub stan Boga jako osoby, jest tajemnicą. Nasi pionierzy to rozumieli i jasno wyjaśnili to z Biblii. Gdyby nie czytali i nie akceptowali Biblii w jej prostym i jasnym języku, nie byliby w stanie wyjaśnić \emcap{osobowości Boga}.


There are brethren who completely agree with the \emcap{personality of God} laid out in the \emcap{Fundamental Principles}. They agree that the terms ‘\textit{father}’, ‘\textit{son}’ and ‘\textit{spirit}’ should be interpreted by their obvious meaning, yet they continue to advocate the Trinity doctrine because they fail to correctly draw the line between what is being revealed by God and what is not. The argument goes something like this: yes, God is a personal, spiritual being; He does have a body of some sort, Christ is His only begotten Son, and the Holy Spirit is Their representative, but that all applies to our physical universe, which is cumbered by space and time; beyond space and time, God is Trinity.


Są bracia, którzy całkowicie zgadzają się z \emcap{osobowością Boga} przedstawioną w \emcap{fundamentalnych zasadach}. Zgadzają się, że terminy ‘\textit{ojciec}’, ‘\textit{syn}’ i ‘\textit{duch}’ powinny być interpretowane w ich oczywistym znaczeniu, jednak nadal opowiadają się za doktryną o Trójcy, ponieważ nie potrafią poprawnie wyznaczyć granicy między tym, co jest objawione przez Boga, a tym, co nie jest. Argument brzmi mniej więcej tak: tak, Bóg jest osobistą, duchową istotą; On ma jakiegoś rodzaju ciało, Chrystus jest Jego jednorodzonym Synem, a Duch Święty jest Ich przedstawicielem, ale to wszystko odnosi się do naszego fizycznego wszechświata, który jest obciążony przestrzenią i czasem; poza przestrzenią i czasem Bóg jest Trójcą.


Such a view fails to draw the line between what is revealed and what is a mystery. One consequence of such a conception of God is that it casts doubt on the things which are revealed unto us. To recognize that takes honesty because it is very enticing to conceptualize God beyond space and time, but it is, ultimately, unjustifiable because we are finite and bound to space and time. In his book, the Living Temple, Dr. Kellogg conceptualized God beyond \others{the bounds of space and time}. Dr. Kellogg objected to the conception of God depicted by the \emcap{Fundamental Principles}, because God, in His personality, was bound to His body and thus “\textit{circumscribed}” to one locality, say the temple, or the throne in Heaven\footnote{\href{https://archive.org/details/J.H.Kellogg.TheLivingTemple1903/page/n31/mode/2up}{John H. Kellogg, The Living Temple, p. 31}}. This was unprofitable for Dr. Kellogg, and he advocated that God is far beyond our comprehension as are the bounds of space and time.


Taki pogląd nie wyznacza granicy między tym, co jest objawione, a tym, co jest tajemnicą. Jedną z konsekwencji takiej koncepcji Boga jest to, że rzuca ona cień wątpliwości na rzeczy, które są nam objawione. Aby to rozpoznać, potrzeba uczciwości, ponieważ bardzo kuszące jest konceptualizowanie Boga poza przestrzenią i czasem, ale ostatecznie jest to nieuzasadnione, ponieważ jesteśmy skończeni i związani przestrzenią i czasem. W swojej książce, The Living Temple, dr Kellogg konceptualizował Boga poza \others{granicami przestrzeni i czasu}. Dr Kellogg sprzeciwiał się koncepcji Boga przedstawionej przez \emcap{fundamentalne zasady}, ponieważ Bóg, w Swojej osobowości, był związany ze Swoim ciałem i tym samym “\textit{ograniczony}” do jednej lokalizacji, powiedzmy świątyni lub tronu w Niebie\footnote{\href{https://archive.org/details/J.H.Kellogg.TheLivingTemple1903/page/n31/mode/2up}{John H. Kellogg, The Living Temple, p. 31}}. To było nieopłacalne dla dr. Kellogga, i opowiadał się za tym, że Bóg jest daleko poza naszym zrozumieniem, tak jak granice przestrzeni i czasu.


\others{\textbf{\underline{Discussions respecting the form of God are utterly unprofitable}, and serve only to belittle our conceptions of him who is above all things}, \textbf{and hence not to be compared in form or size or glory or majesty with anything which man has ever seen or which it is within his power to conceive}. In the presence of questions like these, we have only to acknowledge our foolishness and incapacity, and bow our heads with awe and reverence \textbf{in the presence of a Personality, an Intelligent Being} to the existence of which all nature bears definite and positive testimony, \textbf{but which is as far beyond our comprehension \underline{as are the bounds of space and time}}.}[The Living Temple, pg. 33][https://archive.org/details/J.H.Kellogg.TheLivingTemple1903/page/n33/mode/2up]


\others{\textbf{\underline{Dyskusje dotyczące formy Boga są całkowicie bezużyteczne} i służą jedynie pomniejszeniu naszych wyobrażeń o tym, który jest ponad wszystkim}, \textbf{a zatem nie można go porównywać pod względem formy, rozmiaru, chwały czy majestatu z czymkolwiek, co człowiek kiedykolwiek widział lub co jest w jego mocy pojąć}. W obliczu takich pytań musimy jedynie uznać naszą głupotę i niezdolność, i skłonić nasze głowy z czcią i szacunkiem \textbf{w obecności Osobowości, Inteligentnej Istoty}, o której istnieniu cała natura daje określone i pozytywne świadectwo, \textbf{ale która jest tak daleko poza naszym zrozumieniem \underline{jak granice przestrzeni i czasu}}.}[The Living Temple, pg. 33][https://archive.org/details/J.H.Kellogg.TheLivingTemple1903/page/n33/mode/2up]


Dr. Kellogg was reproved for his conceptions of God. His conception of God was God beyond the bounds of space and time. This conception is problematic because it is beyond the bounds of the Scriptures; it is pure conjecture, which casts doubt on the revelation of the Scripture. If the Scriptures testify that God is a definite, tangible being, being present in one place more than another, then any discussions regarding God being beyond space are utterly unprofitable. Such discussions tend to lead toward skepticism on the very conceptions of God that the Scriptures plainly testify of. As we can recall, this was the main problem with Dr. Kellogg, and Sister White gave us many warnings regarding this issue.


Dr Kellogg został zganiony za swoje koncepcje Boga. Jego koncepcja Boga to Bóg poza granicami przestrzeni i czasu. Ta koncepcja jest problematyczna, ponieważ wykracza poza granice Pisma Świętego; jest to czysta domysł, który rzuca wątpliwość na objawienie Pisma Świętego. Jeśli Pismo Święte świadczy, że Bóg jest określoną, namacalną istotą, będącą obecną w jednym miejscu bardziej niż w innym, to wszelkie dyskusje dotyczące Boga będącego poza przestrzenią są całkowicie bezużyteczne. Takie dyskusje mają tendencję do prowadzenia w kierunku sceptycyzmu wobec samych koncepcji Boga, o których Pismo Święte wyraźnie świadczy. Jak pamiętamy, to był główny problem z dr. Kelloggiem, a Siostra White dała nam wiele ostrzeżeń dotyczących tej kwestii.


\egw{‘The secret things belong unto the Lord our God: but those things which are revealed belong unto us and to our children forever.’ Deuteronomy 29:29. \textbf{The revelation of Himself that God has given in His word is for our study}. \textbf{This we may seek to understand}. \textbf{\underline{But beyond this we are not to penetrate}}. \textbf{The highest intellect may tax itself until it is wearied out in \underline{conjectures}\footnote{\href{https://www.merriam-webster.com/dictionary/conjectures}{Merriam Webster Dictionary} - ‘\textit{conjecture}’ - “\textit{a: inference formed without proof or sufficient evidence; b: a conclusion deduced by surmise or guesswork}”} \underline{regarding the nature of God}, but the effort will be fruitless}. \textbf{This problem has not been given us to solve. No human mind can comprehend God.} \textbf{None are to indulge in speculation regarding His nature. Here silence is eloquence. The Omniscient One is above discussion}.}[MH 429.3; 1905][https://egwwritings.org/?ref=en\_MH.429.3&para=135.2227]


\egw{‘Rzeczy tajemne należą do Pana, Boga naszego, a objawione do nas i do naszych synów na wieki.’ Powtórzonego Prawa 29:29. \textbf{Objawienie samego siebie, które Bóg dał w swoim słowie, jest dla naszego studiowania}. \textbf{Tego możemy starać się zrozumieć}. \textbf{\underline{Ale poza to nie powinniśmy wnikać}}. \textbf{Najwyższy intelekt może się męczyć aż do wyczerpania w \underline{domysłach}\footnote{\href{https://www.merriam-webster.com/dictionary/conjectures}{Merriam Webster Dictionary} - ‘\textit{conjecture}’ - “\textit{a: wnioskowanie formowane bez dowodu lub wystarczających dowodów; b: wniosek wyciągnięty przez przypuszczenie lub zgadywanie}”} \underline{dotyczących natury Boga}, ale wysiłek będzie bezowocny}. \textbf{Ten problem nie został nam dany do rozwiązania. Żaden ludzki umysł nie może pojąć Boga.} \textbf{Nikt nie powinien oddawać się spekulacjom dotyczącym Jego natury. Tutaj milczenie jest wymowne. Wszechwiedzący jest ponad dyskusją}.}[MH 429.3; 1905][https://egwwritings.org/?ref=en\_MH.429.3&para=135.2227]


\egw{I say, and have ever said, \textbf{that I will not engage in controversy with any one in regard to \underline{the nature} and personality of God}. \textbf{Let those who try to describe God know that on such a subject silence is eloquence}. \textbf{\underline{Let the Scriptures be read in simple faith, and let each one form his conceptions of God from His inspired Word}}.}[Lt214-1903.9; 1903][https://egwwritings.org/?ref=en\_Lt214-1903.9&para=10700.15]


\egw{Mówię i zawsze mówiłam, \textbf{że nie będę angażować się w kontrowersje z nikim w odniesieniu do \underline{natury} i osobowości Boga}. \textbf{Niech ci, którzy próbują opisać Boga, wiedzą, że w takim temacie milczenie jest wymowne}. \textbf{\underline{Niech Pisma będą czytane w prostej wierze i niech każdy formuje swoje koncepcje Boga z Jego natchnionego Słowa}}.}[Lt214-1903.9; 1903][https://egwwritings.org/?ref=en\_Lt214-1903.9&para=10700.15]


\egw{No human mind can comprehend God. No man hath seen Him at any time. We are as ignorant of God as little children. But as little children we may love and obey Him. \textbf{Had this been understood, such sentiments as are in this book would never have been expressed}.}[Lt214-1903.10; 1903][https://egwwritings.org/?ref=en\_Lt214-1903.10&para=10700.16]


\egw{Żaden ludzki umysł nie może pojąć Boga. Nikt nigdy Go nie widział. Jesteśmy tak samo nieświadomi Boga jak małe dzieci. Ale jak małe dzieci możemy Go kochać i być Mu posłuszni. \textbf{Gdyby to było zrozumiane, takie poglądy, jakie znajdują się w tej książce, nigdy nie zostałyby wyrażone}.}[Lt214-1903.10; 1903][https://egwwritings.org/?ref=en\_Lt214-1903.10&para=10700.16]


You might wonder why Sister White said that she will not engage in controversy with anyone concerning the nature and \emcap{personality of God}, while she was heavily engaged in the controversy over the \emcap{personality of God}, and wrote many different testimonies regarding it. Discussions regarding the \emcap{personality of God}, to some degree, touch the nature of God; yet, those regarding the nature of God, in connection to the \emcap{personality of God}, Sister White did not engage in. She knew where to draw the line. She pointed out that the Bible should draw this line for us. \egw{\textbf{\underline{Let the Scriptures be read in simple faith, and let each one form his conceptions of God from His inspired Word.}}} The \emcap{Fundamental Principles} obey this rule. Sister White told us that we must not try to explain in regard to the \emcap{personality of God} any further than the Bible has done.


Możesz się zastanawiać, dlaczego Siostra White powiedziała, że nie będzie angażować się w spór z nikim dotyczący natury i \emcap{osobowości Boga}, podczas gdy była mocno zaangażowana w spór o \emcap{osobowość Boga} i napisała wiele różnych świadectw na ten temat. Dyskusje dotyczące \emcap{osobowości Boga} w pewnym stopniu dotykają natury Boga; jednak w te dotyczące natury Boga, w połączeniu z \emcap{osobowością Boga}, Siostra White się nie angażowała. Wiedziała, gdzie postawić granicę. Wskazała, że Biblia powinna wyznaczać tę granicę dla nas. \egw{\textbf{\underline{Niech Pisma będą czytane w prostej wierze i niech każdy kształtuje swoje pojęcia o Bogu z Jego natchnionego Słowa.}}} \emcap{Fundamentalne Zasady} przestrzegają tej reguły. Siostra White powiedziała nam, że nie wolno nam próbować wyjaśniać \emcap{osobowości Boga} dalej, niż uczyniła to Biblia.


\egw{Keep your eyes fixed on the Lord Jesus Christ, and by beholding Him you will be changed into His likeness. \textbf{Talk not of these spiritualistic theories. Let them find no place in your mind.} Let our papers be kept free from everything of the kind. Publish the truth; do not publish error. \textbf{Do not try to explain in regard to the personality of God. \underline{You cannot give any further explanation than the Bible has given}}. \textbf{Human theories regarding Him are good for nothing}. Do not soil your minds by studying the misleading theories of the enemy. Labor to draw minds away from everything of this character. It will be better to keep these subjects out of our papers. Let the doctrines of present truth be put into our papers, but give no room to a repeating of erroneous theories.}[Lt179-1904.4; 1904][https://egwwritings.org/?ref=en\_Lt179-1904.4&para=7751.11]


\egw{Trzymaj swoje oczy skupione na Panu Jezusie Chrystusie, a patrząc na Niego, zostaniesz przemieniony na Jego podobieństwo. \textbf{Nie mów o tych spirytualistycznych teoriach. Niech nie znajdą one miejsca w twoim umyśle.} Niech nasze czasopisma będą wolne od wszystkiego tego rodzaju. Publikuj prawdę; nie publikuj błędu. \textbf{Nie próbuj wyjaśniać osobowości Boga. \underline{Nie możesz dać żadnego dalszego wyjaśnienia niż to, które dała Biblia}}. \textbf{Ludzkie teorie na Jego temat są bezwartościowe}. Nie kalaj swoich umysłów, studiując wprowadzające w błąd teorie wroga. Staraj się odciągnąć umysły od wszystkiego, co ma taki charakter. Lepiej będzie trzymać te tematy z dala od naszych czasopism. Niech doktryny obecnej prawdy będą umieszczane w naszych czasopismach, ale nie dawaj miejsca na powtarzanie błędnych teorii.}[Lt179-1904.4; 1904][https://egwwritings.org/?ref=en\_Lt179-1904.4&para=7751.11]


Let the Bible form our conceptions of God. We cannot give any further explanation of the \emcap{personality of God} than the Bible has given. If the Bible speaks of God that, in His person, He is bound to one locality, like His temple, the sanctuary, and His throne, we should accept that regardless of whether it sounds limiting to God. God is limited in space, in His body, but His presence is not limited, for He is everywhere present by His representative, the Holy Spirit.


Niech Biblia kształtuje nasze pojęcia o Bogu. Nie możemy dać żadnego dalszego wyjaśnienia \emcap{osobowości Boga} niż to, które dała Biblia. Jeśli Biblia mówi o Bogu, że w Swojej osobie jest On związany z jednym miejscem, jak Jego świątynia, sanktuarium i Jego tron, powinniśmy to zaakceptować, niezależnie od tego, czy brzmi to ograniczająco dla Boga. Bóg jest ograniczony w przestrzeni, w Swoim ciele, ale Jego obecność nie jest ograniczona, ponieważ jest On wszędzie obecny przez Swojego przedstawiciela, Ducha Świętego.


The revelation of God does express some limitations of His, and some of them are of a salvational matter. For instance, the Bible clearly says that God is omnipotent (Revelation 19:6), He can do all, yet we find that He could save men by no other means than giving His only begotten Son for us. In the garden of Gethsemane, when God handed the cup of His wrath to His Son, Christ prayed for the possibility that this cup could pass from Him, but ultimately for God's will to be done. Here we see all of the available options the Father had in order to save men. It was not possible to save fallen men, other than for God’s Son to die in their stead. Many protest the idea that something was impossible for God. But if it was possible for God to save men, without His Son drinking the cup of His wrath, surely God would have done it. Some protest this idea of God being limited to only one option of saving men, while He might have infinite options—He is omnipotent, after all. With this thinking, God’s salvation of lost men by the sacrifice of His own begotten Son is enshrouded with doubt, and essentially rejected, even scorned, depicting God as a child murderer. But the revelation is clear in the face of these skeptics. It is not God who is heinous for giving His Son for us; it is sin that is heinous. Sin had demanded this infinite sacrifice to be laid, and there was no other way. That was not roleplay\footnote{The Week of Prayer issue by the Adventist Review, October 31, 1996}, but a reality, that caused infinite grief and suffering to our heavenly Father in giving His own begotten\footnote{Read about God’s gift of His \egwinline{own begotten Son} in \href{https://egwwritings.org/?ref=en_Lt13-1894.18&para=5486.24}{{EGW, Lt13-1894.18; 1894}}}, obedient Son to die in our stead.


Objawienie Boga wyraża pewne Jego ograniczenia, a niektóre z nich mają charakter zbawczy. Na przykład, Biblia wyraźnie mówi, że Bóg jest wszechmocny (Objawienie 19:6), może wszystko, a jednak odkrywamy, że nie mógł zbawić ludzi w żaden inny sposób, jak tylko oddając Swojego jednorodzonego Syna za nas. W ogrodzie Getsemane, gdy Bóg podał kielich Swojego gniewu Swojemu Synowi, Chrystus modlił się o możliwość, aby ten kielich mógł Go ominąć, ale ostatecznie, aby wola Boża się wypełniła. Tutaj widzimy wszystkie dostępne opcje, jakie miał Ojciec, aby zbawić ludzi. Nie było możliwe zbawienie upadłych ludzi inaczej, niż przez śmierć Syna Bożego w ich zastępstwie. Wielu protestuje przeciwko idei, że coś było niemożliwe dla Boga. Ale gdyby było możliwe dla Boga zbawienie ludzi bez tego, aby Jego Syn wypił kielich Jego gniewu, z pewnością Bóg by to uczynił. Niektórzy protestują przeciwko tej idei, że Bóg był ograniczony do tylko jednej opcji zbawienia ludzi, podczas gdy mógł mieć nieskończoną liczbę opcji - jest przecież wszechmocny. Z takim myśleniem, zbawienie zgubionych ludzi przez ofiarę Jego własnego zrodzonego Syna jest spowite wątpliwościami i zasadniczo odrzucone, a nawet wzgardzone, przedstawiając Boga jako mordercę dziecka. Ale objawienie jest jasne w obliczu tych sceptyków. To nie Bóg jest okrutny, dając Swojego Syna za nas; to grzech jest okrutny. Grzech wymagał złożenia tej nieskończonej ofiary i nie było innego sposobu. To nie była gra ról\footnote{The Week of Prayer issue by the Adventist Review, October 31, 1996}, ale rzeczywistość, która spowodowała nieskończony smutek i cierpienie naszemu niebiańskiemu Ojcu, gdy oddawał Swojego własnego zrodzonego\footnote{Przeczytaj o darze Boga, Jego \egwinline{własnego zrodzonego Syna} w \href{https://egwwritings.org/?ref=en_Lt13-1894.18&para=5486.24}{{EGW, Lt13-1894.18; 1894}}}, posłusznego Syna, aby umarł w naszym zastępstwie.


Let our conceptions of who God is, what God is, and of what character He is, be molded by plain Scripture, and let us not doubt it.


Niech nasze pojęcia o tym, kim jest Bóg, czym jest Bóg i jakiego charakteru jest, będą kształtowane przez proste Pismo Święte, i nie wątpmy w to.


% The bottom of the issue

\begin{titledpoem}
    
    \stanza{
        Beside truth’s track, the false does tread, \\
        A line so fine, we must be led. \\
        Without the Spirit’s help to guide \\
        Satan will cause the truth to hide.
    }

    \stanza{
        Two views diverge upon this script, \\
        One view symbolic roles depict. \\
        The other literal, true and real, \\
        Both views are held with ardent zeal.
    }

    \stanza{
        Father and Son, in actual hues? \\
        Or metaphors, just finding clues? \\
        Opposite paths, so close beside, \\
        Without God’s help, we can’t decide.
    }

    \stanza{
        Within the Bible, truth unfolds, \\
        Made in God’s image, we are told. \\
        Not just in symbols, but our frame, \\
        Reflecting Him, likeness the same.
    }

    \stanza{
        Fathers on earth reveal God’s ways, \\
        Our spirit too, like Him portrays. \\
        In speculation, myst’ries thrive \\
        Truth from God’s Word, it is alive.
    }

    \stanza{
        So take God’s Word just as it reads \\
        Reject men’s theories and their creeds \\
        Take what God says, no more, no less, \\
        With humble hearts, the Lord will bless.
    }
    
\end{titledpoem}

% \begin{titledpoem}

    \stanza{
        Szlak fałszu obok prawdy biegnie \\
        I kto nań wkroczy, wnet polegnie. \\
        Bez Ducha Boga prowadzenia \\
        Trudno się ustrzec jest zwiedzenia.
    }

    \stanza{
        To nie są tylko jakieś role, \\
        Przybrany Ojciec, Syn, symbole. \\
        Drugie jest takie stanowisko, \\
        Że tutaj jest dosłowne wszystko.
    }

    \stanza{
        Czy Syn jest synem biologicznym, \\
        Czy tylko czymś metaforycznym? \\
        Dwie sprzeczne drogi blisko siebie, \\
        Potrzebna pomoc Boga w niebie.
    }

    \stanza{
        Na kartach Pisma prawda droga — \\
        Jesteśmy tu na obraz Boga. \\
        Nie jako symbol — nasze ciała \\
        To podobizna Boga cała.
    }

    \stanza{
        Ojcowie ziemscy Boże ślady \\
        Przez swe odsłaniają przykłady, \\
        A nasze ciało ducha mieści \\
        Według Słowa Bożego treści.
    }

    \stanza{
        Prawda przez Biblii studiowanie \\
        Jest lepsza niż wiary wyznanie. \\
        Bóg mówi coś, a Ty mu wierz, \\
        Jeśli błogosławieństwa chcesz.
    }

\end{titledpoem}
