\qrchapter{https://forgottenpillar.com/rsc/pl-fp-chapter22}{Sedno sprawy}

Dziś, gdy porównujemy nasze obecne Fundamentalne Wierzenia z wcześniejszymi \emcap{Fundamentalnymi Zasadami}, widzimy zmianę w fundamencie wiary Adwentystów Dnia Siódmego. Ta zmiana nastąpiła w zrozumieniu osoby Boga, czy też \emcap{osobowości Boga}. Szczególnie w kwestii \emcap{osobowości Boga} siostra White napisała, że ścieżka prawdy leży blisko ścieżki błędu:

\egw{\textbf{Ścieżka prawdy biegnie blisko ścieżki błędu} i obie ścieżki mogą \textbf{wydawać się} być jedną dla umysłów, które nie są prowadzone przez Ducha Świętego, i dlatego \textbf{nie są w stanie szybko rozpoznać różnicy między prawdą a błędem}}[SpTB02 52.2; 1904][https://egwwritings.org/read?panels=p417.266]

Zadajemy sobie pytanie, jak możemy wytyczyć wyraźną linię między tymi dwiema ścieżkami? Aby to zrobić, musimy dotrzeć do sedna sprawy. Musimy znaleźć wyróżniającą zasadę, która oddziela te dwie ścieżki.

Badając nasze obecne trynitarne wierzenie i dzieła naszych pionierów dotyczące \emcap{osobowości Boga}, znaleźliśmy jedną charakterystyczną zasadę, która odróżnia prawdę o \emcap{osobowości Boga}, wyznawaną przez naszych pionierów, od naszego obecnego trynitarnego wierzenia. Obie strony twierdzą, że Biblia jest ich ostatecznym autorytetem, jednak różnice można dostrzec w interpretacji Biblii. W dalszej części mówimy o zrozumieniu i interpretacji Pisma Świętego dotyczącego osoby Boga. Zrozumienie osoby Boga można przedstawić w dwóch odrębnych, wzajemnie wykluczających się pojęciach, które wyraźnie wyznaczają linię między dwoma przeciwstawnymi obozami.

Jednym z popularniejszych poglądów jest ten, że Bóg przedstawił siebie w języku, który jest nam znany, aby wyjaśnić tylko koncepcje zbawienia. Zatem Bóg przedstawił siebie w słowach takich, jak: ‘\textit{ojciec}’, ‘\textit{syn}’ i ‘\textit{duch}’, aby opisać relacje między tymi koncepcjami. To sprawia, że żadne z tych słów nie jest interpretowane w jego oczywistym znaczeniu; przeciwnie, mają one wartość symboliczną lub metaforyczną. Zasada stojąca za tym rozumowaniem to: \textbf{Bóg dostosował siebie do człowieka}.

Drugi, przeciwny pogląd jest taki, że \textbf{Bóg dostosował człowieka do siebie}; \textit{stworzył człowieka na swój własny obraz}. Dlatego słowa takie, jak: ‘\textit{ojciec}’, ‘\textit{syn}’ i ‘\textit{duch}’, gdy odnoszą się do Boga, sugerują swoje oczywiste znaczenie. To jest fundamentalna różnica.

Kiedy dochodzimy do zrozumienia biblijnych terminów takich, jak: ‘\textit{osoba}’, ‘\textit{ojciec}’, ‘\textit{syn}’ i ‘\textit{duch}’, musimy opowiedzieć się za którymś poglądem i odpowiednio go zastosować. Te terminy należy rozumieć albo w ich oczywistym znaczeniu, albo symbolicznie lub metaforycznie. Nie ma tu kompromisu pomiędzy tymi dwoma stanowiskami; musimy wybrać jedno z nich. Poniższy cytat powinien rozstrzygnąć wszelkie dylematy.

\egw{\textbf{\underline{Język Biblii powinien być wyjaśniany zgodnie ze swoim oczywistym znaczeniem, chyba że użyto symbolu lub przenośni}}}[GC 598.3; 1888][https://egwwritings.org/read?panels=p133.2717]

Wierzymy, że niemożliwe jest, aby Biblia była swoją własną wykładnią i nie wyjaśniała swoich własnych symboli. Jeżeli Biblia stosuje słowo ‘ojciec’ w odniesieniu do Boga, ale nigdy nie wyjaśnia tego terminu, to powinno być ono przyjęte w swoim oczywistym znaczeniu. To samo dotyczy słowa ‘syn’ i ‘duch’. Człowiek jest stworzony na obraz Boga. Bóg dostosował człowieka do siebie. Oczywiste znaczenie pochodzi z doświadczenia człowieka. Rozumiemy oczywiste znaczenie słowa ‘ojciec’ poprzez zwykłe, ludzkie ojcostwo. Ale nasze ojcostwo jest obrazem naszego Boga, który jest Ojcem dla swojego Syna. Paweł zaświadczył:

\bible{Dlatego zginam kolana przed \textbf{Ojcem naszego Pana Jezusa Chrystusa, od którego cała \underline{rodzina} w niebie i na ziemi bierze swoje \underline{imię}}}[Ef 3:14--15].

W języku greckim słowo ‘\textit{rodzina}’ to słowo ‘\textit{patria}’, pochodzące od słowa ‘\textit{pater}’, które oznacza ‘\textit{ojciec}’. Niektóre tłumaczenia oddają ten werset jako: \bible{Od którego wszelkie \textbf{ojcostwo} w niebie i na ziemi bierze swoje imię} (DRB), co jest bardziej dosłownym tłumaczeniem. Ojciec naszego Pana Jezusa Chrystusa jest prawdziwie ojcem dla swojego Syna, tak samo prawdziwie, jak my jesteśmy ojcami dla naszych dzieci na Ziemi. Nasze ojcostwo na Ziemi jest nazwane według Ojcostwa w Niebie, gdzie Bóg jest Ojcem naszego Pana Jezusa Chrystusa. Nasze ziemskie ojcostwo jest obrazem Niebiańskiego Ojcostwa, gdzie Bóg jest Ojcem dla swojego Syna. To potwierdza oczywiste znaczenie, że Jezus jest prawdziwie Synem naszego Boga.

Ta sama podstawowa zasada odnosi się do zrozumienia słowa ‘\textit{duch}’ i słowa ‘\textit{istota}’. Bóg dostosował człowieka do Siebie; stworzył człowieka na swój własny obraz. Człowiek jest istotą, posiadającą ciało i ducha, tak jak Bóg — i mówiąc to, nie twierdzimy, że człowiek i Bóg posiadają tę samą naturę. Bóg ukształtował człowieka z prochu ziemi. Jego fizyczna natura jest ograniczona do pierwiastków znajdujących się na ziemi. Nie wnikamy w naturę Boga. To na zawsze pozostanie dla nas tajemnicą; nie jest nam to objawione. Ale to, co jest nam objawione, to fakt, że On ma postać, a postać człowieka jest obrazem postaci Boga. Biblia wyraźnie potwierdza to zrozumienie, opisując Boga siedzącego na swoim tronie:

\bible{\textbf{Na podobieństwie tronu było podobieństwo jak \underline{wyglądu człowieka powyżej na nim}}}[Ez 1:26].

Oczywiste znaczenie słowa ‘\textit{duch}’, w odniesieniu do Ducha Bożego, wywodzi się ze zrozumienia „\textit{ducha człowieka}”. Bóg dostosował człowieka do siebie; stworzył człowieka na swój własny obraz. Tak, jak człowiek posiada ducha, Bóg posiada Ducha. Duch człowieka ma naturę człowieka, a duch Boga ma naturę Boga. Jeśli chodzi o ich naturę, nie są one takie same, ale jeśli chodzi o ich związek z ich wewnętrzną istotą, są takie same; Biblia stawia je na tym samym poziomie. \bible{\textbf{\underline{Duch} sam świadczy z \underline{naszym duchem}}, że jesteśmy dziećmi Bożymi}[Rz 8:16]; \bible{Bo któż z \textbf{ludzi wie, co jest w człowieku}, oprócz \textbf{\underline{ducha ludzkiego}}, który w nim jest? \textbf{\underline{Tak samo}} i tego, co jest w \textbf{Bogu, nikt nie wie}, \textbf{oprócz \underline{Ducha Bożego}}}[1Kor 2:11].

Pod względem relacji rodzinnych i właściwości lub stanu jako osoby człowiek i Bóg są podobni, ponieważ Bóg stworzył człowieka na swój własny obraz. Bóg dostosował człowieka do siebie. Ale w swojej naturze Bóg i człowiek nie są podobni. Bóg jest boski, a człowiek jest ziemski.

Doktryna o Trójcy opiera się na zrozumieniu, że Bóg dostosował siebie do człowieka i że Bóg po prostu użył terminów: ‘\textit{ojciec}’s, ‘\textit{syn}’ i ‘\textit{duch}’, abyśmy mogli Go lepiej zrozumieć. Ta idea stanowi podstawę trynitarnego paradygmatu i go napędza. W dalszej części nie będziemy szczegółowo analizować naszej trynitarnej literatury, ale podeprzemy nasze twierdzenie kilkoma oficjalnymi oświadczeniami Kościoła Adwentystów Dnia Siódmego.

Pierwsze oświadczenie pochodzi z Instytutu Badań Biblijnych, oficjalnej instytucji Generalnej Konferencji, która promuje nauki i doktryny Kościoła Adwentystów Dnia Siódmego. Otwarcie negują oni rodzicielską relację między Ojcem a Jego Synem na rzecz metaforycznego zrozumienia.

\othersnodot{Obraz ojciec-syn \textbf{nie może być dosłownie zastosowany do boskiej relacji Ojciec-Syn} w obrębie Bóstwia. \textbf{Syn nie jest naturalnym, dosłownym Synem Ojca}. \normaltext{[...]} \textbf{Termin ‘Syn’ jest używany metaforycznie}, kiedy odnosi się do Bóstwa}[Adwentystyczny Instytut Badań Biblijnych; opublikowane również w oficjalnym czasopiśmie „Adventist World”.][https://www.adventistbiblicalresearch.org/materials/a-question-of-sonship/].

Odnośnie do \emcap{osobowości Boga}, w kontekście trynitarnego paradygmatu, Kościół Adwentystów Dnia Siódmego wydał następujące oświadczenia w lekcji szkoły sobotniej:

\othersnodot{\textbf{Słowo \underline{osoby} użyte w tytule dzisiejszej lekcji \underline{musi być rozumiane w sensie teologicznym}}. \textbf{Jeśli zrównamy ludzką osobowość z Bogiem, powiedzielibyśmy, że trzy osoby oznaczają trzy jednostki. Lecz wtedy mielibyśmy trzech Bogów, czyli triteizm}. \textbf{Jednak \underline{historyczne chrześcijaństwo} nadało słowu osoba, gdy jest używane w odniesieniu do Boga, \underline{specjalne znaczenie}}: osobiste samo-rozróżnienie, które nadaje odrębność Osobom Bóstwa bez niszczenia koncepcji jedności. \textbf{Ta idea nie jest łatwa do uchwycenia ani wyjaśnienia! \underline{Jest to część tajemnicy Bóstwa}}}[\textit{Lekcja 3}. „Sabbath School Net”, 2025, \href{https://www.ssnet.org/qrtrly/eng/98d/less03.html}{ssnet.org/qrtrly/eng/98d/less03.html} [dostęp: 3 lutego 2025].].

\othersnodot{Te teksty i inne prowadzą nas do przekonania, że \textbf{nasz wspaniały Bóg jest \underline{trzema Osobami w jednej}}, co jest niepojętą \textbf{tajemnicą}, ale jednocześnie prawdą, którą przyjmujemy przez wiarę, gdyż Pismo Święte ją objawia}[Tamże.].

Zgodnie z oficjalnymi oświadczeniami przedstawionymi w lekcji szkoły sobotniej, słowo \textit{‘osoby’}, w odniesieniu do Boga, nie powinno być utożsamiane z ludzką osobowością, ale powinno być stosowane w sensie teologicznym. Jest to w ostrym kontraście do wizji, jaką Siostra White miała odnośnie \emcap{osobowości Boga}. \egwinline{Często widziałam kochanego Jezusa, że \textbf{jest osobą}. Zapytałam Go, czy \textbf{Jego Ojciec jest osobą} i \textbf{\underline{ma postać} jak On}. Jezus powiedział: «\textbf{Jestem dokładnym obrazem osoby Mojego Ojca!}» [Hbr 1:3]}[Lt253-1903.12; 1903][https://egwwritings.org/read?panels=p14068.9980018]. Jej zrozumienie właściwości lub stanu Boga jako osoby jest takie, że Bóg jest osobą w oczywisty sposób — posiada On postać. W taki sam sposób, w jaki rozpoznała Jezusa jako osobę, Jezus zaświadczył, że Bóg jest osobą, mającą postać jak On. Przeciwieństwem oczywistego i dosłownego poglądu jest duchowy pogląd. Kontynuuje, odnosząc się do błędu duchowego poglądu. \egwinline{\textbf{Często widziałam, że \underline{spirytualistyczny pogląd} odebrał całą chwałę nieba, a w wielu umysłach tron Dawida i urocza osoba Jezusa zostały spalone w ogniu spirytualizmu}. Widziałam, że pewni ludzie, którzy zostali zwiedzeni i wprowadzeni w ten błąd, zostaną wyprowadzeni na światło prawdy, \textbf{ale będzie prawie niemożliwe, aby całkowicie uwolnili się od zwodniczej mocy spirytualizmu. Tacy powinni skrupulatnie wyznać swoje błędy i na zawsze je porzucić}}[Lt253-1903.13; 1903][https://egwwritings.org/read?panels=p14068.9980019]. Według lekcji szkoły sobotniej oczywiste zrozumienie terminu \textit{‘osoba’} jest niepoprawne, ponieważ to \othersnodot{\textbf{zrównałoby ludzką osobowość z Bogiem}}, co oznacza, że \othersnodot{\textbf{trzy osoby oznaczają trzy jednostki}}. Przeciwieństwem oczywistego poglądu jest pogląd teologiczny. Dla siostry White przeciwieństwem jest duchowy pogląd. Ten pogląd odbiera \egwinline{całą chwałę niebu, a w wielu umysłach tron Dawida i urocza osoba Jezusa zostały spalone w ogniu spirytualizmu}. W pismach naszych pionierów, przeanalizowanych wcześniej, rozpoznajemy prawdziwość jej twierdzenia. Przedstawiony teologiczny pogląd na osobę Boga odrzuca prawdę o \emcap{osobowości Boga}, którą siostra White otrzymała w widzeniu. Pogląd teologiczny jest wyjaśniany jako jeden Bóg, który jest osobą, a jednak trzema osobami, składający się z trzech odrębnych Bogów — Boga Ojca, Boga Syna i Boga Ducha Świętego. Biblia nigdy nie wyjaśnia Boga za pomocą takiej właściwości lub stanu bycia osobą. Jest to po prostu zakładane przez wierzących w Trójcę, a ponieważ nigdy nie jest wyjaśnione, jest uważane za tajemnicę Boga, ale w rzeczywistości — jest to błąd.

Kiedy wyznaczamy granicę między prawdą a błędem, musimy również wyznaczyć granicę między rzeczami, które są tajemnicą, a tymi, które są objawione. Jeśli chodzi o naturę Boga, milczenie jest elokwencją. Niestety wielu, którzy opowiadają się za doktryną o Trójcy, nie wyznacza tej granicy we właściwym miejscu. Protestujemy przeciwko temu, że \emcap{osobowość Boga}, czyli właściwość lub stan Boga jako osoby, jest tajemnicą. Nasi pionierzy to rozumieli i jasno wyjaśnili to na podstawie Biblii. Gdyby nie czytali i nie akceptowali Biblii w jej prostym i jasnym języku, nie byliby w stanie wyjaśnić \emcap{osobowości Boga}.

Są bracia, którzy całkowicie zgadzają się z \emcap{osobowością Boga} przedstawioną w \emcap{Fundamentalnych Zasadach}. Zgadzają się, że terminy: ‘\textit{ojciec}’, ‘\textit{syn}’ i ‘\textit{duch}’, powinny być interpretowane w ich oczywistym znaczeniu, jednak nadal opowiadają się za doktryną o Trójcy, ponieważ nie potrafią poprawnie wyznaczyć granicy między tym, co jest objawione przez Boga, a tym, co nie jest. Argument brzmi mniej więcej tak: Tak, Bóg jest osobową, duchową istotą; Ma pewnego rodzaju ciało, Chrystus jest Jego jednorodzonym Synem, a Duch Święty jest Ich przedstawicielem, ale to wszystko odnosi się do naszego fizycznego wszechświata, który jest ograniczony przestrzenią i czasem; poza przestrzenią i czasem Bóg jest Trójcą.

Taki pogląd nie wyznacza granicy między tym, co jest objawione, a tym, co jest tajemnicą. Jedną z konsekwencji takiej koncepcji Boga jest to, że rzuca ona cień wątpliwości na rzeczy, które są nam objawione. Aby to rozpoznać, potrzeba uczciwości, ponieważ bardzo kuszące jest konceptualizowanie Boga poza przestrzenią i czasem, ale ostatecznie jest to nieuzasadnione, ponieważ jesteśmy ograniczeni przestrzenią i czasem. W swojej książce, \textit{The Living Temple}, dr Kellogg konceptualizował Boga poza \othersnodot{granicami przestrzeni i czasu}. Dr Kellogg sprzeciwiał się koncepcji Boga przedstawionej przez \emcap{Fundamentalne Zasady}, ponieważ Bóg, w swojej osobowości, był związany ze swoim ciałem i tym samym „\textit{ograniczony}” do jednej lokalizacji, na przykład świątyni lub tronu w Niebie\footnote{\href{https://archive.org/details/J.H.Kellogg.TheLivingTemple1903/page/n31/mode/2up}{J. H. Kellogg, \textit{The Living Temple}, Good Health Publishing Company, Battle Creek, Mich. 1903, s. 31.}}. To było niekorzystne dla dr. Kellogga, gdyż opowiadał się za tym, że Bóg jest daleko poza naszym zrozumieniem, tak jak granice przestrzeni i czasu.

\othersnodot{\textbf{\underline{Dyskusje dotyczące postaci Boga są całkowicie bezużyteczne} i służą jedynie pomniejszeniu naszych wyobrażeń o Tym, który jest ponad wszystkim}, \textbf{i dlatego nie można Go porównywać pod względem postaci, rozmiaru, chwały czy majestatu z czymkolwiek, co człowiek kiedykolwiek widział lub co jest w stanie pojąć}. W obliczu takich pytań możemy jedynie uznać naszą głupotę i nieudolność, i pochylić głowy z bojaźnią i czcią \textbf{w obecności Osobowości, Inteligentnej Istoty}, o której istnieniu cała przyroda daje konkretne i niezbite świadectwo, \textbf{ale która jest tak daleko poza naszym pojmowaniem \underline{jak granice przestrzeni i czasu}}}[Tamże, s. 33.][https://archive.org/details/J.H.Kellogg.TheLivingTemple1903/page/n33/mode/2up].

Dr Kellogg został zganiony za swoje koncepcje Boga. Jego koncepcja Boga to Bóg poza granicami przestrzeni i czasu. Ta koncepcja jest problematyczna, ponieważ wykracza poza granice Pisma Świętego; jest to czysty domysł, który rzuca wątpliwość na objawienie Pisma Świętego. Jeśli Pismo Święte świadczy, że Bóg jest określoną, namacalną istotą, będącą obecną w jednym miejscu bardziej niż w innych, to wszelkie dyskusje dotyczące Boga znajdującego się poza przestrzenią są całkowicie bezużyteczne. Takie dyskusje mają tendencję do prowadzenia w kierunku sceptycyzmu wobec samych koncepcji Boga, o których Pismo Święte wyraźnie świadczy. Jak pamiętamy, to był główny problem z dr. Kelloggiem, a siostra White dała nam wiele ostrzeżeń dotyczących tej kwestii.

\egw{«Rzeczy tajemne należą do Pana, naszego Boga, ale te rzeczy, które są objawione, należą do nas i do naszych synów na wieki». Pwt 29:29. \textbf{Objawienie samego siebie, które Bóg dał w swoim Słowie, jest dla naszego studiowania}. \textbf{To możemy starać się zrozumieć}. \textbf{\underline{Ale poza to nie powinniśmy wnikać}}. \textbf{Najwyższy intelekt może się męczyć aż do wyczerpania w \underline{domysłach}\footnote{\href{https://www.merriam-webster.com/dictionary/conjecture}{Słownik Merriam-Webster: ‘\textit{conjecture}’} — „\textit{a: wnioskowanie formowane bez dowodu lub wystarczających dowodów; b: wniosek wyciągnięty przez przypuszczenie lub zgadywanie}”.} \underline{dotyczących natury Boga}, ale wysiłek będzie bezowocny}. \textbf{Ten problem nie został nam dany do rozwiązania. Żaden ludzki umysł nie jest w stanie pojąć Boga.} \textbf{Nikt nie powinien oddawać się spekulacjom dotyczącym Jego natury. Tutaj milczenie jest elokwencją. Wszechwiedzący jest ponad dyskusją}}[MH 429.3; 1905][https://egwwritings.org/read?panels=p135.2227]

\egw{Mówię i zawsze mówiłam, \textbf{że nie będę wdawać się z nikim w spory o \underline{naturę} i osobowość Boga}. \textbf{Niech ci, którzy próbują opisać Boga, wiedzą, że w takim temacie milczenie jest elokwencją}. \textbf{\underline{Niech Pisma będą czytane w prostej wierze i niech każdy kształtuje swoje koncepcje Boga na bazie Jego natchnionego Słowa}}}[Lt214-1903.9; 1903][https://egwwritings.org/read?panels=p10700.15]

\egw{Żaden ludzki umysł nie może pojąć Boga. Nikt nigdy Go nie widział. Jesteśmy tak samo nieświadomi Boga jak małe dzieci. Ale jak małe dzieci możemy Go kochać i być Mu posłuszni. \textbf{Gdyby to zrozumiano, takie poglądy, jakie znajdują się w tej książce, nigdy nie zostałyby wyrażone}}[Lt214-1903.10; 1903][https://egwwritings.org/read?panels=p10700.16]

Możesz się zastanawiać, dlaczego siostra White powiedziała, że nie będzie wdawać się z nikim  w spór o naturę i \emcap{osobowość Boga}, podczas gdy była mocno zaangażowana w spór o \emcap{osobowość Boga} i napisała wiele różnych świadectw na ten temat. Dyskusje dotyczące \emcap{osobowości Boga} w pewnym stopniu dotykają natury Boga; jednak w te dotyczące natury Boga, w połączeniu z \emcap{osobowością Boga}, siostra White się nie angażowała. Wiedziała, gdzie postawić granicę. Wskazała, że Biblia powinna wyznaczać tę granicę dla nas. \egw{\textbf{\underline{Niech Pisma będą czytane w prostej wierze i niech każdy kształtuje swoje koncepcje Boga na bazie Jego natchnionego Słowa}}} \emcap{Fundamentalne Zasady} przestrzegają tej reguły. Siostra White powiedziała nam, że nie wolno nam próbować wyjaśniać \emcap{osobowości Boga} bardziej, niż uczyniła to Biblia.

\egw{Miejcie swoje oczy skupione na Panu Jezusie Chrystusie, a patrząc na Niego, zostaniecie przemienieni na Jego podobieństwo. \textbf{Nie mówcie o tych spirytualistycznych teoriach. Niech nie znajdą one miejsca w Waszym umyśle.} Niech nasze czasopisma będą wolne od wszystkiego tego rodzaju. Publikujcie prawdę; nie publikujcie błędu. \textbf{Nie próbujcie wyjaśniać osobowości Boga. \underline{Nie możecie dać żadnego głębszego wyjaśnienia niż to, które dała Biblia}}. \textbf{Ludzkie teorie na Jego temat są bezwartościowe}. Nie kalajcie swoich umysłów, badając wprowadzające w błąd teorie wroga. Starajcie się odciągnąć umysły od wszystkiego, co ma taki charakter. Lepiej będzie trzymać te tematy z dala od naszych czasopism. Niech doktryny teraźniejszej prawdy będą umieszczane w naszych czasopismach, ale niech nie będzie miejsca na powtarzanie błędnych teorii}[Lt179-1904.4; 1904][https://egwwritings.org/read?panels=p7751.11]

Niech Biblia kształtuje nasze koncepcje Boga. Nie możemy dać żadnego głębszego wyjaśnienia \emcap{osobowości Boga} niż to, które dała Biblia. Jeśli Biblia mówi o Bogu, że w swojej osobie jest On związany z jednym miejscem, jak Jego świątynia, przybytek i Jego tron, powinniśmy to zaakceptować, niezależnie od tego, czy brzmi to ograniczająco dla Boga. Bóg jest ograniczony w przestrzeni, w Swoim ciele, ale Jego obecność nie jest ograniczona, ponieważ jest On wszędzie obecny przez Swojego przedstawiciela, Ducha Świętego.

Objawienie Boga wyraża pewne Jego ograniczenia, a niektóre z nich mają charakter zbawczy. Na przykład Biblia wyraźnie mówi, że Bóg jest wszechmocny (Obj 19:6), może wszystko, a jednak odkrywamy, że nie mógł zbawić ludzi w żaden inny sposób, jak tylko oddając za nas swojego jednorodzonego Syna. W ogrodzie Getsemani, gdy Bóg podał kielich swojego gniewu swemu Synowi, Chrystus modlił się o możliwość, aby ten kielich mógł Go ominąć, ale ostatecznie, aby wypełniła się wola Boża. Tutaj widzimy wszystkie dostępne opcje, jakie miał Ojciec, aby zbawić ludzi. Nie było możliwe zbawienie upadłych ludzi inaczej niż przez śmierć Syna Bożego w ich zastępstwie. Wielu protestuje przeciwko idei, że coś było niemożliwe dla Boga. Lecz gdyby było możliwe dla Boga zbawienie ludzi bez tego, aby Jego Syn wypił kielich Jego gniewu, z pewnością Bóg by to uczynił. Niektórzy protestują przeciwko tej idei, że Bóg był ograniczony do tylko jednej opcji zbawienia ludzi, podczas gdy mógł mieć nieskończoną liczbę opcji — jest przecież wszechmocny. Z takim myśleniem, zbawienie zgubionych ludzi przez ofiarę Jego własnego zrodzonego Syna jest spowite wątpliwościami i zasadniczo odrzucone, a nawet wzgardzone, przedstawiając Boga jako mordercę dziecka. Ale objawienie jest jasne w obliczu tych sceptyków. To nie Bóg jest okrutny, dając Swojego Syna za nas; to grzech jest okrutny. Grzech wymagał złożenia tej nieskończonej ofiary i nie było innego sposobu. To nie była gra ról\footnote{Wydanie Tygodnia Modlitwy „Adventist Review", 31 października 1996.}, ale rzeczywistość, która spowodowała nieskończony smutek i cierpienie naszemu niebiańskiemu Ojcu, gdy oddawał swojego własnego zrodzonego\footnote{Przeczytaj o darze Boga Jego \egwinline{własnego zrodzonego Syna} w \href{https://egwwritings.org/?ref=en_Lt13-1894.18&para=5486.24}{{EGW, Lt13-1894.18; 1894}}}, posłusznego Syna, aby umarł za nas.

Niech nasze koncepcje tego, kim jest Bóg, czym jest Bóg i jakiego charakteru jest, będą kształtowane przez samo Pismo Święte, i nie wątpmy w to.

\begin{titledpoem}

    \stanza{
        Szlak fałszu obok prawdy biegnie \\
        I kto nań wkroczy, wnet polegnie. \\
        Bez Ducha Boga prowadzenia \\
        Trudno się ustrzec jest zwiedzenia.
    }

    \stanza{
        To nie są tylko jakieś role, \\
        Przybrany Ojciec, Syn, symbole. \\
        Drugie jest takie stanowisko, \\
        Że tutaj jest dosłowne wszystko.
    }

    \stanza{
        Czy Syn jest synem biologicznym, \\
        Czy tylko czymś metaforycznym? \\
        Dwie sprzeczne drogi blisko siebie, \\
        Potrzebna pomoc Boga w niebie.
    }

    \stanza{
        Na kartach Pisma prawda droga — \\
        Jesteśmy tu na obraz Boga. \\
        Nie jako symbol — nasze ciała \\
        To podobizna Boga cała.
    }

    \stanza{
        Ojcowie ziemscy Boże ślady \\
        Przez swe odsłaniają przykłady, \\
        A nasze ciało ducha mieści \\
        Według Słowa Bożego treści.
    }

    \stanza{
        Prawda przez Biblii studiowanie \\
        Jest lepsza niż wiary wyznanie. \\
        Bóg mówi coś, a Ty mu wierz, \\
        Jeśli błogosławieństwa chcesz.
    }

\end{titledpoem}

