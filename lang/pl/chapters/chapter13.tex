\chapter{The Sabbath God vs. Sunday God - J. B. Frisbie}


\chapter{Bóg Sabatu vs. Bóg Niedzieli - J. B. Frisbie}


There are other articles written on the \emcap{personality of God} by our pioneers and it would be too much to include everything here, but we would like to add one more testimony from brother J. B. Frisbie’s article where he compares the Sabbath God with the Sunday God. He compares the truth on the \emcap{personality of God} expressed in the first point of the \emcap{Fundamental Principles} with the Trinity doctrine. Let us take a look at a portion of his article, “\textit{The Seventh Day-Sabbath Not Abolished}” from the Review and Herald, March 7, 1854.


Istnieją inne artykuły napisane na temat \emcap{osobowości Boga} przez naszych pionierów i byłoby zbyt wiele, aby zawrzeć tu wszystko, ale chcielibyśmy dodać jeszcze jedno świadectwo z artykułu brata J. B. Frisbie, w którym porównuje on Boga Sabatu z Bogiem Niedzieli. Porównuje on prawdę o \emcap{osobowości Boga} wyrażoną w pierwszym punkcie \emcap{Fundamentalnych Zasad} z doktryną o Trójcy. Przyjrzyjmy się fragmentowi jego artykułu “\textit{The Seventh Day-Sabbath Not Abolished}” z Review and Herald z 7 marca 1854 roku.


\section*{The Sabbath God}


\section*{Bóg Sabatu}


\others{After we know and remember God, by keeping his holy Sabbath, \textbf{then the Bible will teach of his personality and dwelling place}. \textbf{Man is in the image and likeness of God}. Genesis 1:26. ‘And God said, Let us (speaking to his son) make man in our image, after our likeness’. Chap 2:7. ‘And the Lord God formed man of the dust of the ground, and breathed into his nostrils the breath of life: and man became a living soul’. Genesis 9:6; 1 Corinthians 11:7; James 3:9. \textbf{That which was made in \underline{the image and likeness of God} was made of the dust of the ground called man}.}


\others{Gdy poznamy i będziemy pamiętać o Bogu, przestrzegając Jego świętego Sabatu, \textbf{wtedy Biblia nauczy nas o Jego osobowości i miejscu zamieszkania}. \textbf{Człowiek jest na obraz i podobieństwo Boga}. Księga Rodzaju 1:26. ‘I rzekł Bóg: Uczyńmy (mówiąc do swojego syna) człowieka na nasz obraz, według naszego podobieństwa’. Rozdział 2:7. ‘I ukształtował PAN Bóg człowieka z prochu ziemi, i tchnął w jego nozdrza tchnienie życia. I stał się człowiek duszą żywą’. Księga Rodzaju 9:6; 1 List do Koryntian 11:7; List Jakuba 3:9. \textbf{To, co zostało stworzone na \underline{obraz i podobieństwo Boga} zostało uczynione z prochu ziemi i nazwane człowiekiem}.}


\othersnogap{This is known to be the true sense from other testimonies that may be given from the Bible. \textbf{Jesus was in the form of a man and the express image of his Father’s person}.}


\othersnogap{To jest uznawane za prawdziwe znaczenie na podstawie innych świadectw, które można znaleźć w Biblii. \textbf{Jezus był w postaci człowieka i dokładnym obrazem osoby swojego Ojca}.}


\othersnogap{Philippians 2:6-8. \textbf{Christ Jesus}: ‘Who, being in \textbf{the form of God}, thought it not robbery to be \textbf{equal with God}. But made himself of no reputation, and took upon him \textbf{the form of a servant}, and was \textbf{made in the likeness of men’}. 2 Corinthians 4:4. \textbf{‘And being formed in fashion as a man’}, etc. Colossians 1:15. ‘\textbf{Who is the image of the invisible God}’.}


\othersnogap{List do Filipian 2:6-8. \textbf{Chrystus Jezus}: ‘Który, będąc w \textbf{postaci Boga}, nie uważał za grabież być \textbf{równym Bogu}. Lecz ogołocił samego siebie, przyjmując \textbf{postać sługi} i \textbf{stając się podobny ludziom}’. 2 List do Koryntian 4:4. \textbf{‘A z postawy będąc znaleziony jako człowiek’}, itd. List do Kolosan 1:15. ‘\textbf{On jest obrazem niewidzialnego Boga}’.}


\othersnogap{Hebrews 1:3. \textbf{The Son; ‘Who being the brightness of his glory, and the express image of his person’}. In this sense could Jesus say to Philip in truth, ‘He that hath seen me hath seen the Father.’ John 14:9. Some seem to suppose it argues \textbf{against the personality of God, \underline{because he is a Spirit, and say that he is without body, or parts}}. John 4:24. ‘\textbf{God is a Spirit}’. Hebrews 1:7. ‘\textbf{Who maketh his angels spirits}’. \textbf{Who would pretend to say that angels have no bodies or parts because they are spirits}. \textbf{\underline{None the less is God a spiritual being having body and parts as we may learn by his having a dwelling place and because he has and may be seen}}. Exodus 33:23. ‘And I will take away mine hand, and thou shalt\textbf{ see my back parts}, but my \textbf{face shall not be seen}’. Matthew 5:8. ‘Blessed are the pure in heart, for \textbf{they shall see God}’. Hebrews 12:14. ‘Follow peace with all men, and holiness, without which \textbf{no man shall see the Lord}’. Matthew 18:10. ‘That in heaven their angels do \textbf{always behold the face of my Father which is in heaven}’. Matthew 6:9. ‘After this manner therefore pray ye, \textbf{Our Father which art in heaven}’, etc. John 6:38. ‘For I \textbf{came down from heaven} not to do mine own will, but the will of him that sent me’. Chap 16:28. ‘\textbf{I came forth from the Father, and am come into the world}: again I \textbf{leave the world, and go to the Father}’.}


\othersnogap{List do Hebrajczyków 1:3. \textbf{Syn; ‘Który, będąc jasnością jego chwały i wyrazem jego istoty’}. W tym sensie Jezus mógł prawdziwie powiedzieć do Filipa: ‘Kto mnie widział, widział i Ojca’. Ew. Jana 14:9. Niektórzy wydają się sądzić, że \textbf{przeczy to osobowości Boga, \underline{ponieważ jest On Duchem, i mówią, że nie ma On ciała ani części}}. Ew. Jana 4:24. ‘\textbf{Bóg jest duchem}’. List do Hebrajczyków 1:7. ‘\textbf{Który czyni swoich aniołów duchami}’. \textbf{Kto ośmieliłby się twierdzić, że aniołowie nie mają ciał ani części, ponieważ są duchami}. \textbf{\underline{Nie mniej Bóg jest istotą duchową posiadającą ciało i części, co możemy poznać przez to, że ma miejsce zamieszkania i ponieważ może być widziany}}. Księga Wyjścia 33:23. ‘I odejmę moją rękę, i \textbf{ujrzysz moje plecy}, ale \textbf{mojego oblicza nie będzie można zobaczyć}’. Ew. Mateusza 5:8. ‘Błogosławieni czystego serca, albowiem \textbf{oni Boga oglądać będą}’. List do Hebrajczyków 12:14. ‘Dążcie do pokoju ze wszystkimi i do świętości, bez której \textbf{nikt nie ujrzy Pana}’. Ew. Mateusza 18:10. ‘Że w niebie ich aniołowie \textbf{zawsze patrzą na oblicze mojego Ojca, który jest w niebie}’. Ew. Mateusza 6:9. ‘Wy więc tak się módlcie: \textbf{Ojcze nasz, który jesteś w niebie}’, itd. Ew. Jana 6:38. ‘Bo \textbf{zstąpiłem z nieba} nie po to, żeby czynić swoją wolę, ale wolę tego, który mnie posłał’. Rozdział 16:28. ‘\textbf{Wyszedłem od Ojca i przyszedłem na świat}; znowu \textbf{opuszczam świat i idę do Ojca}’.}


\othersnogap{\textbf{Does not God say he fills immensity of space? \underline{We answer, No}}. Psalm 139:7, 8. ‘Whither shall I go \textbf{from thy Spirit}? or whither shall I flee \textbf{from thy presence}? If I ascend up into heaven, thou art there’, etc. \textbf{\underline{God by his Spirit may fill heaven and earth}}, etc. \textbf{Some confound God with his Spirit, which makes confusion}. Psalm 11:4. ‘\textbf{The Lord is in his holy temple, the Lord’s throne is in heaven}: his eyes behold’, etc. Habakkuk 2:20; Psalm 102:19. ‘For he hath looked \textbf{down from the height of his Sanctuary}; \textbf{\underline{from heaven} did the Lord behold the earth’}. 1 Peter 3:12. ‘For the eyes of the Lord are over the righteous, and his ears are open unto their prayers’, etc. Psalm 80:1. ‘Give ear, O Shepherd of Israel, thou that leadest Joseph like a flock; thou \textbf{that dwellest between the cherubims}, shine forth’. Psalm 99:1; Isaiah 37:16.}


\othersnogap{\textbf{Czy Bóg nie mówi, że wypełnia nieskończoność przestrzeni? \underline{Odpowiadamy, Nie}}. Psalm 139:7, 8. ‘Dokąd ujdę przed \textbf{twoim duchem}? I dokąd przed \textbf{twoim obliczem} ucieknę? Jeśli wstąpię do nieba, ty tam jesteś’, itd. \textbf{\underline{Bóg przez swojego Ducha może wypełniać niebo i ziemię}}, itd. \textbf{Niektórzy mylą Boga z Jego Duchem, co prowadzi do zamieszania}. Psalm 11:4. ‘\textbf{PAN jest w swoim świętym przybytku, tron PANA jest w niebie}: jego oczy patrzą’, itd. Habakuk 2:20; Psalm 102:19. ‘Bo spojrzał \textbf{ze swojej wysokiej świątyni}; \textbf{\underline{z nieba} PAN popatrzył na ziemię’}. 1 List Piotra 3:12. ‘Oczy Pana są nad sprawiedliwymi, a jego uszy otwarte są na ich modlitwy’, itd. Psalm 80:1. ‘Słuchaj, Pasterzu Izraela, ty, który prowadzisz Józefa jak stado; ty, \textbf{który zasiadasz między cherubinami}, zajaśniej’. Psalm 99:1; Księga Izajasza 37:16.}


\othersnogap{John 14:2. ‘In my Father’s house are many mansions. I go to prepare a place for you’. Revelation 21:2-5; Hebrews 11:6. ‘For he that cometh to God must believe that he is’, etc. \textbf{This testimony we deem highly important at this time, to know that there is a God. We have no doubt that if our eyes could be opened in vision, or see as angels see, we should see God in heaven sitting on his throne, and is present to all that exists, however distant from him in his creation}.}[\href{https://documents.adventistarchives.org/Periodicals/RH/RH18540307-V05-07.pdf}{Adventist Review and Sabbath Herald, March 7, 1854}, J. B. Frisbie, “The Seventh-Day Sabbath Not Abolished”, p. 50]


\othersnogap{Ew. Jana 14:2. ‘W domu mego Ojca jest wiele mieszkań. Idę przygotować wam miejsce’. Objawienie 21:2-5; List do Hebrajczyków 11:6. ‘Bo kto przychodzi do Boga, musi wierzyć, że on jest’, itd. \textbf{To świadectwo uważamy za niezwykle ważne w tym czasie, aby wiedzieć, że jest Bóg. Nie mamy wątpliwości, że gdyby nasze oczy mogły zostać otwarte w wizji lub widzieć tak jak widzą aniołowie, zobaczylibyśmy Boga w niebie siedzącego na swoim tronie, i jest On obecny we wszystkim, co istnieje, bez względu na to, jak daleko od Niego w Jego stworzeniu}.}[\href{https://documents.adventistarchives.org/Periodicals/RH/RH18540307-V05-07.pdf}{Adventist Review and Sabbath Herald, 7 marca 1854}, J. B. Frisbie, “The Seventh-Day Sabbath Not Abolished”, str. 50]


Here we see the same argument and reasoning, that God is a personal spiritual Being. This God is the Sabbath God. Brother Frisbie compares this God with the Sunday God, who is a trinitarian God.


Tutaj widzimy ten sam argument i rozumowanie, że Bóg jest osobową istotą duchową. Ten Bóg jest Bogiem Sabatu. Brat Frisbie porównuje tego Boga z Bogiem Niedzieli, który jest Bogiem trynitarnym.


\section*{The Sunday God}


\section*{Bóg Niedzieli}


\others{We will make a few extracts, that the reader may \textbf{see the broad contrast between \underline{the God of the Bible} brought to light through Sabbath-keeping, and the god in the dark through Sunday-keeping}. Catholic Catechism Abridged by the Rt. Rev. John Dubois, Bishop of New York. Page 5. ‘\textbf{Ques. Where is God? Ans. God is everywhere}. Q. Does God see and know all things? A. Yes, he does know and see all things. \textbf{Q. Has God any body? A. \underline{No; God has no body, he is a pure Spirit}}. \textbf{Q. Are there more Gods than one? A. No; there is but one God. Q. Are there more persons than one in God? A. \underline{Yes; in God there are three persons}. Q. Which are they? A. God the Father, God the Son and God the Holy Ghost. Q. Are there not three Gods? A. No; the Father, the Son and the Holy Ghost, are all but one and the same God}’.}


\others{Przedstawimy kilka fragmentów, aby czytelnik mógł \textbf{zobaczyć wyraźny kontrast między \underline{Bogiem Biblii} ujawnionym przez zachowywanie Sabatu, a bogiem w ciemności przez zachowywanie niedzieli}. Katechizm Katolicki skrócony przez Wielebnego Johna Dubois, Biskupa Nowego Jorku. Strona 5. ‘\textbf{Pyt. Gdzie jest Bóg? Odp. Bóg jest wszędzie}. P. Czy Bóg widzi i wie wszystko? O. Tak, On wie i widzi wszystko. \textbf{P. Czy Bóg ma ciało? O. \underline{Nie; Bóg nie ma ciała, jest czystym Duchem}}. \textbf{P. Czy jest więcej niż jeden Bóg? O. Nie; jest tylko jeden Bóg. P. Czy jest więcej niż jedna osoba w Bogu? O. \underline{Tak; w Bogu są trzy osoby}. P. Które to są? O. Bóg Ojciec, Bóg Syn i Bóg Duch Święty. P. Czy nie ma trzech Bogów? O. Nie; Ojciec, Syn i Duch Święty, to wszystko jeden i ten sam Bóg}’.}


\othersnogap{The first article of the Methodist Religion, p. 8. \textbf{‘There is but one living and true God}, everlasting, \textbf{without body or parts}, of infinite power, wisdom and goodness: the maker and preserver of all things, visible and invisible. \textbf{And in unity of this God-head, there are three persons of one substance, power and eternity; the Father, the Son, and the Holy Ghost}.’}


\othersnogap{Pierwszy artykuł Religii Metodystycznej, str. 8. \textbf{‘Jest tylko jeden żywy i prawdziwy Bóg}, wieczny, \textbf{bez ciała i części}, o nieskończonej mocy, mądrości i dobroci: stwórca i zachowawca wszystkich rzeczy, widzialnych i niewidzialnych. \textbf{A w jedności tej Boskości są trzy osoby jednej substancji, mocy i wieczności; Ojciec, Syn i Duch Święty}.’}


\othersnogap{In this article like the Catholic doctrine, \textbf{we are taught that there are three persons of one substance,} power and eternity making\textbf{ in all one living and true God}, everlasting \textbf{without body or parts}. But in all this we are not told \textbf{what became of the body of Jesus who had a body when he ascended, who went to God who ‘is everywhere’ or nowhere}. Doxology.}


\othersnogap{W tym artykule, podobnie jak w doktrynie katolickiej, \textbf{uczy się nas, że są trzy osoby jednej substancji,} mocy i wieczności tworzące \textbf{w sumie jednego żywego i prawdziwego Boga}, wiecznego \textbf{bez ciała i części}. Ale w tym wszystkim nie powiedziano nam \textbf{co stało się z ciałem Jezusa, który miał ciało gdy wstąpił do nieba, który poszedł do Boga, który ‘jest wszędzie’ albo nigdzie}. Doksologia.}


\othersnogap{‘\textbf{To God the Father, God the Son,}} \\
\others{\textbf{God the Spirit, three in one.}’} \\
\others{Again} \\
\others{‘Warms in the sun, refreshes in the breeze,} \\
\others{Glows in the stars, and blossoms in the trees.} \\
\others{\textbf{Lives through all life, extends through all extent},} \\
\others{Spreads undivided and operates unspent.’ - Pope.}


\othersnogap{‘\textbf{Bogu Ojcu, Bogu Synowi,}} \\
\others{\textbf{Bogu Duchowi, trzem w jednym.}’} \\
\others{Ponownie} \\
\others{‘Grzeje w słońcu, orzeźwia w powiewie,} \\
\others{Świeci w gwiazdach i kwitnie w drzewie.} \\
\others{\textbf{Żyje poprzez całe życie, rozciąga się przez cały zasięg},} \\
\others{Rozprzestrzenia się niepodzielnie i działa niewyczerpanie.’ - Pope.}


\othersnogap{These ideas well accord with those heathen philosophers. One says, ‘That water was the principle of all things, and that God is that intelligence, by whom all things are formed out of water.’ Another, ‘That air is God, that it is produced, that it is immense and infinite,’ etc. A third, ‘That God is a soul diffused throughout all beings of nature,’ etc. \textbf{Some, who had the idea of \underline{a pure Spirit}}. Last of all, ‘That God is an eternal substance.’}


\othersnogap{Te idee dobrze współgrają z poglądami filozofów pogańskich. Jeden mówi: ‘Że woda była zasadą wszystkich rzeczy, a Bóg jest tą inteligencją, przez którą wszystkie rzeczy są formowane z wody.’ Inny, ‘Że powietrze jest Bogiem, że jest wytwarzane, że jest niezmierzone i nieskończone,’ itd. Trzeci, ‘Że Bóg jest duszą rozprzestrzenioną we wszystkich bytach natury,’ itd. \textbf{Niektórzy, którzy mieli ideę \underline{czystego Ducha}}. Na końcu, ‘Że Bóg jest wieczną substancją.’}


\othersnogap{These extracts are taken from Rollin’s History, Vol. II, pp. 597-8, published by Harpers. \textbf{We should rather mistrust that the Sunday god came from the same source that Sunday-keeping did}. ‘Sunday was a name given by the heathens to the first day of the week, because it was the day on which they worshiped the sun.’ - Union Bible Dictionary. \textbf{Afterward modified by the Roman Catholic Church, in the form we now find it taught through the land}.}


\othersnogap{Te fragmenty są wzięte z Historii Rollina, Tom II, str. 597-8, wydanej przez Harpers. \textbf{Powinniśmy raczej podejrzewać, że niedzielny bóg pochodzi z tego samego źródła co zachowywanie niedzieli}. ‘Niedziela była nazwą nadaną przez pogan pierwszemu dniowi tygodnia, ponieważ był to dzień, w którym czcili słońce.’ - Słownik Biblijny Union. \textbf{Później zmodyfikowana przez Kościół Rzymskokatolicki w formie, w jakiej obecnie jest nauczana w kraju}.}


\othersnogap{It is very natural to suppose when \textbf{the Pope set himself up to be God in the temple of God}, [2 Thessalonians 2:4] that he should have a day sanctified to his worship. This he has done. - Douay Catechism, p. 59. ‘Q. What is the best means to sanctify Sunday? A. By hearing mass, etc. This saying mass is for the priest to gabble over Latin, drink some wine, and give the people a wafer to eat.’}”“\others{But God sanctified his day because he had rested on it. Another day for a very different purpose. Genesis 2:3.}


\othersnogap{Jest bardzo naturalne przypuszczać, że gdy \textbf{Papież ustanowił się Bogiem w świątyni Bożej}, [2 Tesaloniczan 2:4] powinien mieć dzień uświęcony dla swojego kultu. To właśnie uczynił. - Katechizm Douay, str. 59. ‘P. Jaki jest najlepszy sposób na uświęcenie niedzieli? O. Przez słuchanie mszy, itd. To odprawianie mszy polega na tym, że ksiądz mamrocze po łacinie, pije wino i daje ludziom opłatek do zjedzenia.’}”“\others{Ale Bóg uświęcił swój dzień, ponieważ odpoczął w nim. Inny dzień w zupełnie innym celu. Księga Rodzaju 2:3.}


\othersnogap{In days before the moral fall of Babylon God directed the minds of his honest children right in their prayers, whatever they might think at other times, but now since the apostasy the mind reaches to no god but to the people only, there are many prayers to men we know by their effect and eloquence. \textbf{We are truly thankful to our heavenly Father that \underline{he has led our minds from such folly}, to know, and remember \underline{his holy name} by keeping his holy day that we might love, serve and worthily \underline{glorify him through our great High Priest in the heavenly Sanctuary in this day of atonement}}.}[Ibid.][https://documents.adventistarchives.org/Periodicals/RH/RH18540307-V05-07.pdf]


\othersnogap{W dniach przed moralnym upadkiem Babilonu Bóg kierował umysły swoich szczerych dzieci właściwie w ich modlitwach, cokolwiek mogliby myśleć w innych chwilach, ale teraz od czasu odstępstwa umysł nie sięga do żadnego boga, tylko do ludzi, jest wiele modlitw do ludzi, które znamy po ich efekcie i elokwencji. \textbf{Jesteśmy prawdziwie wdzięczni naszemu niebiańskiemu Ojcu, że \underline{wyprowadził nasze umysły z takiego szaleństwa}, aby poznać i pamiętać \underline{Jego święte imię} przez zachowywanie Jego świętego dnia, abyśmy mogli kochać, służyć i godnie \underline{chwalić Go przez naszego wielkiego Arcykapłana w niebiańskiej Świątyni w tym dniu pojednania}}.}[Ibid.][https://documents.adventistarchives.org/Periodicals/RH/RH18540307-V05-07.pdf]


Before becoming a Seventh-day Adventist, Frisbie was a Methodist preacher and a bitter opponent of Adventist beliefs. In 1853, after a debate on the Sabbath with Joseph Bates, he reversed his position and began to keep the Sabbath and preach the Seventh-day Adventist doctrine. He renounced Sunday, the Trinity, and accepted the Seventh-day Sabbath and the truth about God, that the Seventh-day Adventist’s taught in the first point of the \emcap{Fundamental Principles}.


Zanim został Adwentystą Dnia Siódmego, Frisbie był kaznodzieją metodystycznym i zagorzałym przeciwnikiem wierzeń adwentystycznych. W 1853 roku, po debacie o Sabacie z Josephem Batesem, zmienił swoje stanowisko i zaczął zachowywać Sabat oraz głosić doktrynę Adwentystów Dnia Siódmego. Wyrzekł się niedzieli, Trójcy i przyjął prawdę o Sabacie dnia siódmego oraz prawdę o Bogu, której Adwentyści Dnia Siódmego nauczali w pierwszym punkcie \emcap{Fundamentalnych Zasad}.


Do other Adventist pioneers see discordance between the Trinity doctrine and the \emcap{personality of God} expressed in the first point of the \emcap{Fundamental Principles}?


Czy inni pionierzy adwentyzmu dostrzegają niezgodność między doktryną o Trójcy a \emcap{osobowością Boga} wyrażoną w pierwszym punkcie \emcap{fundamentalnych zasad}?
