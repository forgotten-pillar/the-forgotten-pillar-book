\chapter{Ustanawianie błędnych Fundamentalnych Zasad}

Możesz zadać sobie pytanie: jak to możliwe, że my, jako kościół, odeszliśmy od światła, które Bóg dał nam na początku? Odpowiedź na to pytanie jest taka sama jak odpowiedź na pytanie, dlaczego Żydzi odeszli od światła, które Bóg dał im odnośnie Jego Syna. Proszę, spójrz na siłę napędową kościoła w czasach apostolskich i w naszych czasach.

\egw{‘Anioł Pański otworzył w nocy drzwi więzienia i wyprowadziwszy ich, rzekł: Idźcie, stańcie i mówcie w świątyni do ludzi wszystkie słowa tego życia.’ [Dz 5:19, 20.] Widzimy tutaj, że ludzie sprawujący władzę nie zawsze są posłuszni, nawet jeśli mogą twierdzić, że są nauczycielami doktryn biblijnych. \textbf{Jest dziś wielu, którzy czują się oburzeni i pokrzywdzeni, że jakikolwiek głos jest podnoszony prezentując idee, które różnią się od ich własnych w odniesieniu do punktów wiary religijnej}. \textbf{Czy nie głosili od dawna swoich idei jako prawdy?} Tak rozumowali kapłani i rabini w czasach apostolskich. Co znaczą ci niewykształceni ludzie, niektórzy z nich zwykli rybacy, którzy przedstawiają idee sprzeczne z doktrynami, których uczeni kapłani i przywódcy nauczają lud? \textbf{Nie mają prawa ingerować w fundamentalne zasady naszej wiary}.}[Lt38-1896.23; 1896][https://egwwritings.org/?ref=en\_Lt38-1896.23&para=5631.29]

\egwnogap{“\textbf{Ale widzimy, że Bóg niebios czasami powołuje ludzi, aby \underline{nauczali tego, co jest uważane za sprzeczne z ustalonymi doktrynami}. Ponieważ ci, którzy kiedyś byli depozytariuszami prawdy, \underline{stali się niewierni swojemu świętemu zadaniu}, Pan wybrał innych, którzy przyjęliby jasne promienie Słońca Sprawiedliwości i głosiliby prawdy, które nie były zgodne z ideami przywódców religijnych. A wtedy ci przywódcy, w zaślepieniu swoich umysłów, dają pełną swobodę temu, co uważają za sprawiedliwe oburzenie przeciwko tym, którzy odrzucili hołubione bajki. Zachowują się jak ludzie, którzy stracili rozum. Nie biorą pod uwagę możliwości, że sami mogli niewłaściwie zrozumieć Słowo. Nie otworzą oczu, aby dostrzec fakt, że błędnie interpretowali i niewłaściwie stosowali Pisma, i zbudowali fałszywe teorie, \underline{nazywając je fundamentalnymi doktrynami wiary}}.”}[Lt38-1896.24; 1896][https://egwwritings.org/?ref=en\_Lt38-1896.24&para=5631.30]

\egwnogap{\textbf{Ale Duch Święty będzie od czasu do czasu objawiał prawdę przez swoich wybranych działaczy; i żaden człowiek, nawet duchowny czy władca, nie ma prawa powiedzieć: Nie wolno ci upubliczniać swoich opinii, ponieważ ja w nie nie wierzę. To cudowne ‘Ja’ może próbować stłumić nauczanie Ducha Świętego. Ludzie mogą przez pewien czas usiłować ją stłumić i zabić; ale to nie sprawi, że błąd stanie się prawdą lub prawda błędem. Pomysłowe umysły ludzkie rozwinęły spekulatywne opinie w różnych kierunkach, a kiedy Duch Święty pozwala, aby światło zaświeciło w ludzkich umysłach, nie uwzględnia każdego punktu zastosowania słowa przez człowieka. Bóg nakazał swoim sługom, aby mówili prawdę niezależnie od tego, co ludzie uznają za  prawdę}.}[Lt38-1896.25; 1896][https://egwwritings.org/?ref=en\_Lt38-1896.25&para=5631.31]

\egwnogap{\textbf{\underline{Nawet Adwentyści Dnia Siódmego są w niebezpieczeństwie zamknięcia oczu na prawdę, jaka jest w Jezusie}, ponieważ przeczy ona czemuś, co uznali za pewnik jako prawdę, ale czego Duch Święty uczy, że nie jest prawdą. Niech wszyscy będą bardzo skromni i niech gorliwie starają się odrzucić siebie i wywyższyć Jezusa.} \textbf{W większości kontrowersji religijnych podstawą problemu jest to, że własne ja walczy o supremację}. O co? O sprawy, które wcale nie są istotne, a są uważane za takie tylko dlatego, że ludzie nadali im znaczenie. Zobacz Mateusza 12:31-37; Marka 14:56; Łukasza 5:21; Mateusza 9:3.}[Lt38-1896.26; 1896][https://egwwritings.org/?ref=en\_Lt38-1896.26&para=5631.32]

Dumny stan serca sprzeciwia się woli Bożej i jest siłą napędową odstępstwa; pokorne serce jest posłuszne woli Bożej i jest siłą napędową prawdziwej reformacji. Poniższe cytaty wyrażają przyszłe, konkretne proroctwa, gdzie fantazyjne idee o Bogu zostaną wprowadzone i \egwinline{wiele rzeczy o podobnym charakterze pojawi się w przyszłości}[Ms137-1903.10; 1903][https://egwwritings.org/?ref=en\_Ms137-1903.10&para=9939.17]. Te idee mają podobny charakter do idei zawartych w The Living Temple. Zniosą one \emcap{osobowość Boga}. Ellen White daje ostrzeżenie za ostrzeżeniem, aby trzymać się \emcap{Fundamentalnych Zasad} i być świadomym przywódców, którzy będą burzyć stary fundament.

\egw{W świetle tych fragmentów Pisma, kto odważy się interpretować Boga i umieszczać w umysłach innych poglądy na Jego temat, które są zawarte w Living Temple? \textbf{Te teorie są teoriami wielkiego zwodziciela, a w życiu \underline{tych, którzy je przyjmują, będą smutne rozdziały}}. \textbf{To jest narzędzie Szatana, \underline{aby zachwiać fundamentem naszej wiary}, zachwiać naszym zaufaniem do prowadzenia Pana i do doświadczenia, które nam dał. \underline{Wiele rzeczy o podobnym charakterze pojawi się w przyszłości}}. Błagam naszych pracowników misji medycznej, aby bali się ufać przypuszczeniom i pomysłom jakiejkolwiek istoty ludzkiej, która żywi myśl, że \textbf{ścieżka, po której lud Boży był prowadzony przez ostatnie pięćdziesiąt lat, jest niewłaściwą ścieżką}. \textbf{\underline{Strzeżcie się tych, którzy}, nie mając żadnego zdecydowanego doświadczenia w prowadzeniu Ducha Pańskiego, \underline{przypuszczaliby, że to prowadzenie jest całkowitą pomyłką}; że nie mamy prawdy}; że nie jesteśmy ludem Pana, zgromadzonym przez Niego ze wszystkich krajów i narodów. \textbf{\underline{Strzeżcie się tych, którzy chcieliby zburzyć fundament, na którym budowaliśmy przez ostatnie pięćdziesiąt lat, aby ustanowić nową doktrynę}}. \textbf{Wiem, że te nowe teorie pochodzą od wroga}.}[Ms137-1903.10; 1903][https://egwwritings.org/?ref=en\_Ms137-1903.10&para=9939.17]

\egwnogap{\textbf{Niech ci, którzy chcieliby \underline{wprowadzić} fantazyjne idee o Bogu, obudzą się i zrozumieją swoje niebezpieczeństwo. To zbyt poważny temat, aby się nim bawić}.}[Ms137-1903.11; 1903][https://egwwritings.org/?ref=en\_Ms137-1903.11&para=9939.18]

\egwnogap{Korzeniem bałwochwalstwa jest złe serce niewiary, odchodzące od żywego Boga. To dlatego, że ludzie nie mają wiary w obecność i moc Boga, \textbf{pokładali zaufanie w swojej własnej mądrości}. Wymyślali i planowali, aby wywyższyć samych siebie i znaleźć zbawienie w swoich własnych uczynkach. \textbf{\underline{Zwodniczy wpływ z szatańskich agencji nadchodzi}, ponieważ przywódcy, których Pan ostrzegał, błagał i radził, wybierają swoją własną mądrość zamiast mądrości Bożej}. Do takich przychodzi ostrzeżenie: ‘Nie mówcie więcej tak bardzo dumnie; niech nie wychodzi z ust waszych arogancja; gdyż Pan jest Bogiem wiedzy i przez Niego uczynki są ważone.’}[Ms137-1903.12; 1903][https://egwwritings.org/?ref=en\_Ms137-1903.12&para=9939.19]

Porównując stare \emcap{Fundamentalne Zasady} z nowymi trynitarnymi Fundamentalnymi Wierzeniami, widzimy różnicę. Widzimy, że odeszliśmy od fundamentów naszej wiary i zbudowaliśmy nowy. To był proces. Dziś, tak jak za czasów Siostry White, szatańskie siły wprowadziły zwodniczy wpływ do Kościoła Adwentystów Dnia Siódmego, \egwinline{ponieważ przywódcy, których Pan ostrzegał, błagał i którym doradzał, wybierają własną mądrość zamiast mądrości Boga}. Powinniśmy \egwinline{\textbf{Wystrzegać się tych, którzy chcieliby zburzyć fundament, na którym budowaliśmy przez ostatnie pięćdziesiąt lat, aby ustanowić nową doktrynę}.}


%% Setting up the wrong Fundamental Principles

\begin{titledpoem}
    
    \stanza{
        Sadly, within our own church walls \\
        From our own pulpits, error falls \\
        Members want smooth words for their ears \\
        Don’t step on toes, Allay their fears.
    }

    \stanza{
        Pastors and elders do preside \\
        While sins remain, untouched inside \\
        Laodicean comfort zone \\
        But they will reap what they have sown.
    }

    \stanza{
        Ask for the old paths, walk therein \\
        From the old truth, don’t move a pin. \\
        They spurned the truth which brightly shone, \\
        The Spirit’s pow’r, to them unknown.
    }

    \stanza{
        Do not the humble hearts perceive \\
        Whispers of truth they should believe? \\
        Meanwhile the stories ease concern. \\
        What God would tell them, they won’t learn.
    }

    \stanza{
        Beware of error, thinly veiled, \\
        God’s Word is true, but men have failed. \\
        Beware of shadows leaders cast. \\
        To the foundations true, hold fast,
    }

    \stanza{
        Let not man’s wisdom lead astray, \\
        Let God’s own Spirit show the way. \\
        For in the Scripture’s glowing light, \\
        We find the path of safety bright.
    }

    \stanza{
        Let us, in faith, each day commence, \\
        God’s Word our shield, not man’s pretense. \\
        For truth in Christ alone is found, \\
        And on this rock, our faith is sound.
    }
    
\end{titledpoem}

% \begin{titledpoem}

    \stanza{
        Niestety w naszych własnych ścianach \\
        Z kazalnic prawda jest ściągana. \\
        Ludzie wolą pochlebne baśnie, \\
        Aż cała czujność w każdym zaśnie.
    }

    \stanza{
        Starsi wraz z pastorami rządzą, \\
        Podczas gdy ludzie w grzechu błądzą. \\
        Laodycea wciąż się chwali, \\
        Lecz wkrótce zbiorą to, co siali.
    }

    \stanza{
        Do starych ścieżek powracajcie \\
        I ani sworznia nie ruszajcie. \\
        Prawdę stopniowo odrzucali \\
        I mocy Ducha nie poznali.
    }

    \stanza{
        Czy serce nie odczuwa tego, \\
        Że prawda zbliża się do niego? \\
        Tymczasem bajki uszy tuczą, \\
        Od Boga się niechętnie uczą.
    }

    \stanza{
        Strzeżcie się błędu zakrytego \\
        Z ręki człowieka ułomnego. \\
        Strzeżcie się swych przywódców cieni, \\
        Bo błędem będziecie zwiedzeni.
    }

    \stanza{
        Sprowadza z drogi mądrość człeka, \\
        Więc Bóg niech prowadzi człowieka. \\
        A w jasnym świetle prosto z Pisma \\
        Jest nasza ścieżka, nasza przystań.
    }

    \stanza{
        Niech każdy dzień się zacznie w wierze \\
        I Słowo będzie Twym puklerzem, \\
        Gdyż prawda jest w Chrystusie stała \\
        A wiara w Niego — nasza skała.
    }

\end{titledpoem}
