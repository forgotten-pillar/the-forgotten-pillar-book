\qrchapter{https://forgottenpillar.com/rsc/pl-fp-chapter25}{Ustanowienie błędnych Fundamentalnych Zasad}

Możesz zadać sobie pytanie: Jak to możliwe, że my, jako Kościół, odeszliśmy od światła, które Bóg dał nam na początku? Odpowiedź na to pytanie jest taka sama jak odpowiedź na pytanie, dlaczego Żydzi odeszli od światła, które Bóg im dał odnośnie do swego Syna. Proszę, spójrz na siłę napędową Kościoła w czasach apostolskich i w naszych czasach.

\egw{«Anioł Pański otworzył w nocy drzwi więzienia, wyprowadził ich i rzekł: Idźcie, stańcie i mówcie w świątyni do ludzi wszystkie słowa tego życia» [Dz 5:19--20]. Widzimy tutaj, że ludzie sprawujący władzę nie zawsze są posłuszni, nawet jeśli twierdzą, że są nauczycielami doktryn biblijnych. \textbf{Jest dziś wielu, którzy czują się oburzeni i dotknięci, gdy pojawia się głos przedstawiający poglądy odmienne od ich własnych w odniesieniu do kwestii wiary religijnej}. \textbf{Czyż przez długi czas nie głosili swoich idei jako prawdy?} Tak rozumowali kapłani i rabini w czasach apostolskich. O co chodzi tym niewykształconym ludziom, pośród których są zwykli rybacy, że przedstawiają idee sprzeczne z doktrynami, których uczeni kapłani i przywódcy nauczają lud? \textbf{Nie mają prawa ruszać fundamentalnych zasad naszej wiary}}[Lt38-1896.23; 1896][https://egwwritings.org/read?panels=p5631.29]

\egwnogap{\textbf{Widzimy jednak, że Bóg niebios czasami powołuje ludzi, aby \underline{nauczali tego, co jest uważane za sprzeczne z ustalonymi doktrynami}. Ponieważ ci, którzy kiedyś byli szafarzami prawdy, \underline{stali się niewierni swojemu świętemu zadaniu}, Pan wybrał innych, którzy przyjęli jasne promienie Słońca Sprawiedliwości i głosili prawdy niezgodne z ideami przywódców religijnych. A wtedy ci przywódcy, w zaślepieniu swoich umysłów, dają pełen upust temu, co uważają za sprawiedliwe oburzenie wobec tych, którzy odrzucili hołubione bajki. Zachowują się jak ludzie, którzy postradali rozum. Nie biorą pod uwagę możliwości, że sami mogli niewłaściwie zrozumieć Słowo. Nie otworzą oczu, aby dostrzec fakt, że błędnie interpretowali i niewłaściwie stosowali Pisma, budując fałszywe teorie i \underline{nazywając je fundamentalnymi doktrynami wiary}}}[Lt38-1896.24; 1896][https://egwwritings.org/read?panels=p5631.30]

\egwnogap{\textbf{Lecz Duch Święty będzie od czasu do czasu objawiał prawdę przez swoje wybrane jednostki; i żaden człowiek, nawet duchowny czy władca, nie ma prawa powiedzieć: Nie wolno ci upubliczniać swoich opinii, ponieważ ja w nie nie wierzę. To cudowne «Ja» może próbować stłumić nauczanie Ducha Świętego. Ludzie mogą przez pewien czas usiłować je stłumić i zabić; ale to nie sprawi, że błąd stanie się prawdą lub prawda błędem. Pomysłowe umysły ludzkie wysuwały spekulatywne opinie w różnych kierunkach, a kiedy Duch Święty pozwala, aby światło zaświeciło w ludzkich umysłach, nie respektuje każdego punktu ludzkiej interpretacji Słowa. Bóg nakazał swoim sługom, aby mówili prawdę niezależnie od tego, co ludzie uznali za bezwzględną prawdę}}[Lt38-1896.25; 1896][https://egwwritings.org/read?panels=p5631.31]

\egwnogap{\textbf{\underline{Nawet adwentystom dnia siódmego grozi niebezpieczeństwo zamknięcia oczu na prawdę, jaka jest w Jezusie}, ponieważ przeczy ona czemuś, co uznali za bezwzględną prawdę, ale co Duch Święty przedstawia jako nieprawdę. Niech wszyscy będą bardzo pokorni i niech gorliwie starają się odsunąć na bok siebie i wywyższyć Jezusa.} \textbf{W większości sporów religijnych podstawą problemu jest to, że własne ja walczy o dominację}. W czym? W sprawach, które wcale nie są istotne, a są uważane za takie tylko dlatego, że ludzie nadali im ważność. Patrz Mt 12:31--37; Mk 14:56; Łk 5:21; Mt 9:3}[Lt38-1896.26; 1896][https://egwwritings.org/read?panels=p5631.32]

Dumny stan serca sprzeciwia się woli Bożej i jest siłą napędową odstępstwa; pokorne serce jest posłuszne woli Bożej i jest siłą napędową prawdziwej przemiany. Poniższe cytaty zawierają przyszłe, konkretne proroctwa, gdzie dziwaczne idee o Bogu zostaną wprowadzone i \egwinline{wiele podobnych rzeczy pojawi się w przyszłości}[Ms137-1903.10; 1903][https://egwwritings.org/read?panels=p9939.17]. Te idee są podobne do idei zawartych w \textit{The Living Temple}. Wyeliminują one \emcap{osobowość Boga}. Ellen White daje ostrzeżenie za ostrzeżeniem, aby trzymać się \emcap{Fundamentalnych Zasad} i być świadomym przywódców, którzy będą burzyć stary fundament.

\egw{W świetle tych fragmentów Pisma, kto odważy się interpretować Boga i umieszczać w umysłach innych poglądy na Jego temat, które są zawarte w «The Living Temple»? \textbf{Te teorie są teoriami wielkiego zwodziciela, a w życiu \underline{tych, którzy je przyjmują, będą smutne rozdziały}. To jest sztuczka szatana, \underline{aby zachwiać fundamentem naszej wiary}, zachwiać naszym zaufaniem do prowadzenia Pana i do doświadczenia, które nam dał. \underline{Wiele podobnych rzeczy pojawi się w przyszłości}}. Błagam naszych pracowników misji medycznej, aby bali się ufać przypuszczeniom i pomysłom jakiejkolwiek istoty ludzkiej, która żywi myśl, że \textbf{ścieżka, po której lud Boży był prowadzony przez ostatnie pięćdziesiąt lat, jest niewłaściwą ścieżką}. \textbf{\underline{Strzeżcie się tych, którzy}, nie mając żadnego zdecydowanego doświadczenia z prowadzeniem Ducha Pańskiego, \underline{przypuszczaliby, że to prowadzenie jest całkowitą pomyłką}; że nie mamy prawdy}; że nie jesteśmy ludem Pana, zgromadzonym przez Niego ze wszystkich krajów i narodów. \textbf{\underline{Strzeżcie się tych, którzy chcieliby zburzyć fundament, na którym budowaliśmy przez ostatnie pięćdziesiąt lat, aby ustanowić nową doktrynę}}. \textbf{Wiem, że te nowe teorie pochodzą od wroga}}[Ms137-1903.10; 1903][https://egwwritings.org/read?panels=p9939.17]

\egwnogap{\textbf{Niech ci, którzy chcieliby \underline{wprowadzać} dziwaczne idee o Bogu, obudzą się i poczują grożące im niebezpieczeństwo. To zbyt poważny temat, aby z nim igrać}}[Ms137-1903.11; 1903][https://egwwritings.org/read?panels=p9939.18]

\egwnogap{Korzeniem bałwochwalstwa jest złe serce niewiary, odchodzące od żywego Boga. To dlatego, że ludzie nie mają wiary w obecność i moc Boga, \textbf{pokładali zaufanie w swojej własnej mądrości}. Zamierzali i planowali wywyższyć samych siebie i znaleźć zbawienie w swoich własnych uczynkach. \textbf{\underline{Zwodniczy wpływ sił szatańskich nadchodzi}, ponieważ przywódcy, których Pan ostrzegał, błagał i którym radził, wybierają swoją własną mądrość zamiast mądrości Bożej}. Do takich przychodzi ostrzeżenie: «Nie mówcie więcej tak z taką dumą; niech nie wychodzi arogancja z waszych ust; gdyż Pan jest Bogiem wiedzy i przez Niego uczynki są ważone»}[Ms137-1903.12; 1903][https://egwwritings.org/read?panels=p9939.19]

% Porównując stare \emcap{Fundamentalne Zasady} z nowymi trynitarnymi Fundamentalnymi Wierzeniami, widzimy różnicę. Widzimy, że odeszliśmy od fundamentów naszej wiary i zbudowaliśmy nowy. To był proces. Dziś, tak jak za czasów Siostry White, szatańskie siły wprowadziły zwodniczy wpływ do Kościoła Adwentystów Dnia Siódmego, \egwinline{ponieważ przywódcy, których Pan ostrzegał, błagał i którym doradzał, wybierają własną mądrość zamiast mądrości Boga}. Powinniśmy \egwinline{\textbf{Wystrzegać się tych, którzy chcieliby zburzyć fundament, na którym budowaliśmy przez ostatnie pięćdziesiąt lat, aby ustanowić nową doktrynę}}.

Różnica między starymi Fundamentalnymi Zasadami a nowymi \emcap{Fundamentalnymi Zasadami} tkwi w naszych \egwinline{ideach o Bogu}. Trynitarna idea Boga nie była częścią fundamentu naszej wiary, którego broniła siostra White. Jak doszło do tej zmiany? Stało się to za sprawą przywódców, którzy wybrali \egwinline{własną mądrość zamiast mądrości Bożej}. Powinniśmy \egwinline{strzec się tych, którzy chcieliby zburzyć fundament, na którym budowaliśmy przez ostatnie pięćdziesiąt lat, aby ustanowić nową doktrynę}. W tym spostrzeżeniu rozpoznajemy, że ta nowa trynitarna idea Boga była  \egwinline{zwodniczym wpływem sił szatańskich}, który wszedł w nasze szeregi.

\begin{titledpoem}

    \stanza{
        Niestety w naszych własnych ścianach \\
        Z kazalnic prawda jest ściągana. \\
        Ludzie wolą pochlebne baśnie, \\
        Aż cała czujność w każdym zaśnie.
    }

    \stanza{
        Starsi wraz z pastorami rządzą, \\
        Podczas gdy ludzie w grzechu błądzą. \\
        Laodycea wciąż się chwali, \\
        Lecz wkrótce zbiorą to, co siali.
    }

    \stanza{
        Do starych ścieżek powracajcie \\
        I ani sworznia nie ruszajcie. \\
        Prawdę stopniowo odrzucali \\
        I mocy Ducha nie poznali.
    }

    \stanza{
        Czy serce nie odczuwa tego, \\
        Że prawda zbliża się do niego? \\
        Tymczasem bajki uszy tuczą, \\
        Od Boga się niechętnie uczą.
    }

    \stanza{
        Strzeżcie się błędu zakrytego \\
        Z ręki człowieka ułomnego. \\
        Strzeżcie się swych przywódców cieni, \\
        Bo błędem będziecie zwiedzeni.
    }

    \stanza{
        Sprowadza z drogi mądrość człeka, \\
        Więc Bóg niech prowadzi człowieka. \\
        A w jasnym świetle prosto z Pisma \\
        Jest nasza ścieżka, nasza przystań.
    }

    \stanza{
        Niech każdy dzień się zacznie w wierze \\
        I Słowo będzie Twym puklerzem, \\
        Gdyż prawda jest w Chrystusie stała \\
        A wiara w Niego — nasza skała.
    }

\end{titledpoem}

