\chapter{Steps to apostasy}


\chapter{Kroki do apostazji}


In the following quotation, brother J. N. Loughborough, who was one of the pioneers of the Seventh-day Adventist Church, warned us about the five steps to apostasy.


W poniższym cytacie brat J. N. Loughborough, który był jednym z pionierów Kościoła Adwentystów Dnia Siódmego, ostrzegał nas przed pięcioma krokami do apostazji.


\others{\textbf{The} \textbf{first step} of apostasy is to \textbf{get up a creed}, telling us what we shall believe. \textbf{The second} is to \textbf{make that creed a test of fellowship}. \textbf{The third} is to \textbf{try members by that creed}. \textbf{The fourth} is to \textbf{denounce as heretics those who do not believe that creed}. And \textbf{fifth}, to \textbf{commence persecution against such}. I plead that we are not patterning after the churches in any unwarrantable sense in the step proposed.}[John N. Loughborough, Review and Herald, Oct. 8, 1861.][https://egwwritings.org/?ref=en\_ARSH.October.8.1861.p.149.7&para=1685.5326]


\others{\textbf{Pierwszym krokiem} apostazji jest \textbf{stworzenie wyznania wiary}, mówiącego nam, w co mamy wierzyć. \textbf{Drugim} jest \textbf{uczynienie tego wyznania wiary testem przynależności}. \textbf{Trzecim} jest \textbf{ocenianie członków według tego wyznania wiary}. \textbf{Czwartym} jest \textbf{potępianie jako heretyków tych, którzy nie wierzą w to wyznanie wiary}. A \textbf{piątym}, \textbf{rozpoczęcie prześladowań przeciwko takim osobom}. Proszę, abyśmy nie naśladowali kościołów w żadnym nieuzasadnionym sensie w proponowanym kroku.}[John N. Loughborough, Review and Herald, 8 października 1861 r.][https://egwwritings.org/?ref=en\_ARSH.October.8.1861.p.149.7&para=1685.5326]


These principles are important to have in mind, and we ought to ask ourselves if we, today, are patterning after the churches in any unwarrantable sense in the step proposed. What would happen to a Seventh-day Adventist who would reject the Trinity doctrine in favor of the \emcap{Fundamental Principles}? Do we have a creed set up in our church? Do we test our membership by it?


Te zasady są ważne, aby mieć je na uwadze, i powinniśmy zadać sobie pytanie, czy my, dzisiaj, naśladujemy kościoły w jakimkolwiek nieuzasadnionym sensie w proponowanym kroku. Co by się stało z adwentystą dnia siódmego, który odrzuciłby doktrynę o Trójcy na rzecz \emcap{Fundamentalnych Zasad}? Czy mamy ustanowione wyznanie wiary w naszym kościele? Czy sprawdzamy nasze członkostwo według niego?


The \emcap{Fundamental Principles} had a different nature and role in the Seventh-day Adventist Church contrary to that of the pattern held by other churches. The \emcap{Fundamental Principles} were not designed as a creed. In the preface of the 1872 statement, we read about their nature:


\emcap{Fundamentalne Zasady} miały inny charakter i rolę w Kościele Adwentystów Dnia Siódmego, w przeciwieństwie do wzorca przyjętego przez inne kościoły. \emcap{Fundamentalne Zasady} nie zostały zaprojektowane jako wyznanie wiary. W przedmowie do oświadczenia z 1872 roku czytamy o ich charakterze:


\others{In presenting to the \textbf{public} this \textbf{synopsis of our faith}, we wish to have it distinctly understood that \textbf{\underline{we have no articles of faith, creed}, or discipline, \underline{aside from the Bible}}. We \textbf{do not} put forth this \textbf{\underline{as having any authority with our people}}, \textbf{nor is it designed to secure uniformity among them}, \textbf{as a system of faith}, \textbf{but is a brief statement of what is, and has been, with great unanimity, held by them}.}[A Declaration of the Fundamental Principles, Taught and Practiced by the Seventh-Day Adventists, 1872]


\others{Przedstawiając \textbf{publicznie} to \textbf{streszczenie naszej wiary}, pragniemy, aby było wyraźnie zrozumiane, że \textbf{\underline{nie mamy artykułów wiary, wyznania wiary} lub dyscypliny \underline{poza Biblią}}. \textbf{Nie} przedstawiamy tego \textbf{\underline{jako mającego jakikolwiek autorytet wśród naszego ludu}}, \textbf{ani nie jest to zaprojektowane, aby zapewnić jednolitość wśród nich}, \textbf{jako system wiary}, \textbf{ale jest to krótkie oświadczenie o tym, co jest i było, z wielką jednomyślnością, przez nich wyznawane}.}[Oświadczenie o fundamentalnych zasadach nauczanych i wyznawanych przez Adwentystów Dnia Siódmego, 1872]


In the preface of the 1889 statement, we read similar sentiments:


W przedmowie do oświadczenia z 1889 roku czytamy podobne poglądy:


\others{As elsewhere stated, Seventh-day Adventists \textbf{have no creed but the Bible}; but they hold to \textbf{certain well-defined points of faith}, for which they \textbf{feel prepared to give a reason ‘to every man that asketh’ them}. The following propositions may be taken as a summary of \textbf{the principal features of their religious faith}, upon which there is, so far as we know, \textbf{entire unanimity throughout the body}.}[Seventh-day Adventist Year Book of statistics for 1889, pg. 147, The Fundamental Principles of Seventh-day Adventists]


\others{Jak stwierdzono gdzie indziej, Adwentyści Dnia Siódmego \textbf{nie mają innego wyznania wiary poza Biblią}; ale trzymają się \textbf{pewnych dobrze zdefiniowanych punktów wiary}, dla których \textbf{czują się przygotowani, aby dać powód ‘każdemu człowiekowi, który ich pyta’}. Następujące propozycje można uznać za podsumowanie \textbf{głównych cech ich religijnej wiary}, co do których, o ile nam wiadomo, \textbf{panuje całkowita jednomyślność w całym ciele}.}[Rocznik statystyczny Adwentystów Dnia Siódmego za 1889 r., str. 147, Fundamentalne Zasady Adwentystów Dnia Siódmego]


The \emcap{Fundamental Principles} were not designed to dictate someone’s faith. The believers, led by the Holy Spirit, freely rendered their consciences to the Word of God; under the influence of the Holy Spirit, they came to the same conclusions. There was entire unanimity throughout the body. All believers felt “\textit{prepared to give a reason to every man that asketh them}” regarding their faith.


\emcap{Fundamentalne Zasady} nie zostały zaprojektowane, aby dyktować czyjąś wiarę. Wierzący, prowadzeni przez Ducha Świętego, dobrowolnie poddawali swoje sumienia Słowu Bożemu; pod wpływem Ducha Świętego dochodzili do tych samych wniosków. Panowała całkowita jednomyślność w całym ciele. Wszyscy wierzący czuli się “\textit{przygotowani, aby dać powód każdemu człowiekowi, który ich pyta}” odnośnie ich wiary.


Today we see a striking difference in the principles and practice of Adventist beliefs compared to our pioneers. We are keeping the spirit of unity by disciplining our members for the denial of the Fundamental Beliefs. In our church manual, under the section “\textit{Reason for Disciplines}”, we read the first point which states the discipline for denial of faith in the Seventh-day Adventist Fundamental Beliefs.


Dzisiaj widzimy uderzającą różnicę w zasadach i praktyce adwentystycznych wierzeń w porównaniu do naszych pionierów. Utrzymujemy ducha jedności poprzez dyscyplinowanie naszych członków za zaprzeczanie Fundamentalnym Wierzeniom. W naszym podręczniku kościelnym, w sekcji “\textit{Powody dyscypliny}”, czytamy pierwszy punkt, który określa dyscyplinę za zaprzeczanie wiary w Fundamentalne Wierzenia Adwentystów Dnia Siódmego.


\others{Reasons for Discipline}


\others{Powody dyscypliny}


\others{1. \textbf{Denial of faith} in the fundamentals of the gospel and \textbf{in the Fundamental Beliefs of the Church} or \textbf{teaching doctrines contrary to the same}.}[SDA Church Manual, 20th edition, Revised 2022, p. 67][https://www.adventist.org/wp-content/uploads/2023/07/2022-Seventh-day-Adventist-Church-Manual.pdf]


\others{1. \textbf{Zaprzeczanie wiary} w fundamenty ewangelii i \textbf{w Fundamentalne Wierzenia Kościoła} lub \textbf{nauczanie doktryn sprzecznych z nimi}.}[SDA Church Manual, 20th edition, Revised 2022, p. 67][https://www.adventist.org/wp-content/uploads/2023/07/2022-Seventh-day-Adventist-Church-Manual.pdf]


To discipline someone over their faith is nothing else than coercion of conscience. We are to render our conscience to the Bible alone—not to any man, councils or church creed(s). Disciplining members for their denial of the Fundamental Beliefs is clear evidence that we, indeed, have a creed besides the Bible. We cannot exercise the freedom of our conscience in subjection to the Word of God while confined to a set of beliefs that, if questioned with the authority of the Bible, will be disciplined. In our practice we have forgotten the foundation of protestantism and reformation. All reformers have had their conscience coerced to the extent of their lives. Martin Luther had famously put this principle in action in his defense before the Diet of Worms.


Dyscyplinowanie kogoś z powodu jego wiary nie jest niczym innym jak przymusem sumienia. Nasze sumienie powinniśmy poddawać wyłącznie Biblii - nie człowiekowi, radom czy kościelnym wyznaniom wiary. Dyscyplinowanie członków za zaprzeczanie Fundamentalnym Wierzeniom jest wyraźnym dowodem na to, że rzeczywiście mamy wyznanie wiary poza Biblią. Nie możemy korzystać z wolności sumienia w poddaniu się Słowu Bożemu, będąc jednocześnie ograniczeni zestawem wierzeń, które, jeśli zostaną zakwestionowane na podstawie autorytetu Biblii, będą skutkować dyscypliną. W naszej praktyce zapomnieliśmy o fundamencie protestantyzmu i reformacji. Wszyscy reformatorzy doświadczyli przymusu sumienia, nawet za cenę życia. Marcin Luter słynnie zastosował tę zasadę w swojej obronie przed Sejmem w Wormacji.


\others{Unless I am \textbf{convicted by Scripture} and plain reason—I do not accept the authority of popes and councils, for they have contradicted each other—\textbf{\underline{my conscience is captive to the Word of God}}. I cannot and I will not recant anything, for \textbf{to go against conscience is neither right nor safe}. Here I stand, I cannot do otherwise. God help me. Amen.}[Bainton, 182]


\others{Jeśli nie zostanę \textbf{przekonany przez Pismo Święte} i prosty rozum - nie uznaję autorytetu papieży i soborów, ponieważ zaprzeczały sobie nawzajem - \textbf{\underline{moje sumienie jest zniewolone Słowem Bożym}}. Nie mogę i nie odwołam niczego, ponieważ \textbf{działanie wbrew sumieniu nie jest ani słuszne, ani bezpieczne}. Tu stoję, nie mogę inaczej. Boże, pomóż mi. Amen.}[Bainton, 182]


If one Seventh-day Adventist member has his conscience captive to the Word of God and is not in harmony with the Seventh-day Adventist Fundamental Beliefs, his conscience should not be coerced by church discipline. We know that in the end of time, the whole Seventh-day Adventist Church will be coerced over the issue of the Sabbath. We have been fighting for religious freedom, yet we’re allowing ourselves to coerce the conscience of those who are not in harmony with the Fundamental Beliefs. If today we discipline our members for not subjecting their consciences to men, councils and creeds, how shall we act tomorrow when the government will discipline their citizens for not subjecting their conscience to its power, when they will force obedience to legislation contrary to the Scriptures?


Jeśli członek Kościoła Adwentystów Dnia Siódmego ma sumienie zniewolone Słowem Bożym i nie jest w zgodzie z Fundamentalnymi Wierzeniami Adwentystów Dnia Siódmego, jego sumienie nie powinno być przymuszane przez dyscyplinę kościelną. Wiemy, że w czasach końca cały Kościół Adwentystów Dnia Siódmego będzie przymuszany w kwestii Sabatu. Walczyliśmy o wolność religijną, a jednak pozwalamy sobie na przymuszanie sumienia tych, którzy nie są w zgodzie z Fundamentalnymi Wierzeniami. Jeśli dziś dyscyplinujemy naszych członków za to, że nie poddają swojego sumienia ludziom, radom i wyznaniom wiary, jak będziemy postępować jutro, gdy rząd będzie dyscyplinował swoich obywateli za niepoddawanie sumienia jego władzy, gdy będzie wymuszał posłuszeństwo prawu sprzecznemu z Pismem Świętym?


Adventist pioneers were very much aware of the dangers of extorting church members’ consciences. The expression of their beliefs was not designed to form unity. They were ready to justify their faith, from the Bible, when asked. The Bible was their only creed and article of faith.


Pionierzy adwentyzmu byli bardzo świadomi niebezpieczeństw wymuszania na członkach kościoła posłuszeństwa sumienia. Wyrażanie ich wierzeń nie miało na celu tworzenia jedności. Byli gotowi uzasadniać swoją wiarę z Biblii, gdy ich o to pytano. Biblia była ich jedynym wyznaniem wiary i artykułem wiary.


In 1883, there was a suggestion to introduce the church manual into the Seventh-day Adventist Church. This proposal was rejected after close investigation of the committee appointed by the General Conference. In the article “\textit{No Church Manual}”, we read their reasons for not accepting the proposed church manual.


W 1883 roku pojawiła się propozycja wprowadzenia podręcznika kościelnego do Kościoła Adwentystów Dnia Siódmego. Ta propozycja została odrzucona po dokładnym zbadaniu przez komitet powołany przez Generalną Konferencję. W artykule “\textit{No Church Manual}” (Brak podręcznika kościelnego), czytamy ich powody odrzucenia proponowanego podręcznika kościelnego.


\others{\textbf{While brethren who have favored a manual have ever contended that such a work was not to be anything like a creed or a discipline, or to have authority to settle disputed points}, but was only to be considered as a book containing hints for the help of those of little experience, \textbf{yet it must be evident that such a work, issued under the auspices of the General Conference, would at once carry with it much weight of authority, and would be consulted by most of our younger ministers}. \textbf{\underline{It would gradually shape and mold the whole body}}; \textbf{and those who did not follow it would be considered out of harmony with established principles of church order}. \textbf{And, really, is this not the object of the manual?} And what would be the use of one if not to accomplish such a result? But would this result, on the whole, be a benefit? Would our ministers be broader, more original, more self-reliant men? Could they be better depended on in great emergencies? Would their spiritual experiences likely be deeper and their judgment more reliable? \textbf{We think the tendency all the other way}.}[No Church Manual, The Review and Herald, November 27, 1883, pg. 745][https://documents.adventistarchives.org/Periodicals/RH/RH18831127-V60-47.pdf]


\others{\textbf{Podczas gdy bracia, którzy popierali podręcznik, zawsze twierdzili, że takie dzieło nie miało być w żaden sposób podobne do wyznania wiary czy dyscypliny, ani mieć autorytetu do rozstrzygania spornych kwestii}, ale miało być jedynie uważane za książkę zawierającą wskazówki pomocne dla osób o małym doświadczeniu, \textbf{to jednak musi być oczywiste, że takie dzieło, wydane pod auspicjami Generalnej Konferencji, natychmiast niosłoby ze sobą dużą wagę autorytetu i byłoby konsultowane przez większość naszych młodszych duchownych}. \textbf{\underline{Stopniowo kształtowałoby i formowałoby całe ciało}}; \textbf{a ci, którzy by się do niego nie stosowali, byliby uważani za niezgodnych z ustalonymi zasadami porządku kościelnego}. \textbf{I czy nie jest to właśnie celem podręcznika?} I jaki byłby z niego pożytek, gdyby nie miał osiągnąć takiego rezultatu? Ale czy ten rezultat, ogólnie rzecz biorąc, byłby korzystny? Czy nasi duchowni byliby bardziej otwarci, bardziej oryginalni, bardziej samodzielni? Czy można by na nich bardziej polegać w wielkich sytuacjach kryzysowych? Czy ich duchowe doświadczenia byłyby prawdopodobnie głębsze, a ich osąd bardziej wiarygodny? \textbf{Uważamy, że tendencja jest zupełnie odwrotna}.}[No Church Manual, The Review and Herald, November 27, 1883, pg. 745][https://documents.adventistarchives.org/Periodicals/RH/RH18831127-V60-47.pdf]


\others{\textbf{The Bible contains our creed and discipline. It \underline{thoroughly} furnishes the man of God unto all good works}. What it has not revealed relative to church organization and management, the duties of officers and ministers, and kindred subjects, should not be strictly defined and drawn out into minute specifications for the sake of uniformity, \textbf{but rather be left to individual judgment under the guidance of the Holy Spirit}. \textbf{Had it been best to have a book of directions of this sort, the Spirit would doubtless have gone further, and left one on record with the stamp of inspiration upon it}.}[Ibid.][https://documents.adventistarchives.org/Periodicals/RH/RH18831127-V60-47.pdf]


\others{\textbf{Biblia zawiera nasze wyznanie wiary i dyscyplinę. \underline{Dokładnie} wyposaża człowieka Bożego do wszelkiego dobrego dzieła}. To, czego nie ujawniła w odniesieniu do organizacji i zarządzania kościołem, obowiązków urzędników i duchownych oraz pokrewnych tematów, nie powinno być ściśle definiowane i rozpisywane na szczegółowe specyfikacje dla zachowania jednolitości, \textbf{ale raczej pozostawione indywidualnemu osądowi pod kierownictwem Ducha Świętego}. \textbf{Gdyby najlepiej było mieć księgę wskazówek tego rodzaju, Duch niewątpliwie poszedłby dalej i pozostawił taką z pieczęcią natchnienia na niej}.}[Ibid.][https://documents.adventistarchives.org/Periodicals/RH/RH18831127-V60-47.pdf]


Since 1883, the Seventh-day Adventist Church had grown considerably; so, for the sake of convenience, in 1931, the General Conference Committee voted to publish a church manual.\footnote{Maratas, Prince. “Church Manual.” General Conference of Seventh-Day Adventists, 20 Aug. 2023, \href{https://gc.adventist.org/church-manual/}{gc.adventist.org/church-manual/}. Accessed 3 Feb. 2025.} The church, as an organized body, should exercise order and discipline, in the matters of organization and plans of the prosperity of the Church's mission. But no committee should exercise authority over someone’s conscience and someone’s belief. Only God holds the right to this authority. This is why the Bible is our only creed. We render our conscience to the Word of God, not a man, nor a group of men or committee. Contrary to this, many believe that God vested this authority to the general assembly of the General Conference. But such an idea is based on misrepresentation of one particular quotation. Let us read this quotation carefully.


Od 1883 roku Kościół Adwentystów Dnia Siódmego znacznie się rozrósł; więc dla wygody, w 1931 roku, Komitet Generalnej Konferencji zagłosował za opublikowaniem podręcznika kościelnego.\footnote{Maratas, Prince. “Church Manual.” General Conference of Seventh-Day Adventists, 20 Aug. 2023, \href{https://gc.adventist.org/church-manual/}{gc.adventist.org/church-manual/}. Accessed 3 Feb. 2025.} Kościół, jako zorganizowane ciało, powinien stosować porządek i dyscyplinę w sprawach organizacji i planów dla powodzenia misji Kościoła. Ale żaden komitet nie powinien sprawować władzy nad czyimś sumieniem i czyimiś przekonaniami. Tylko Bóg ma prawo do tej władzy. Dlatego Biblia jest naszym jedynym wyznaniem wiary. Poddajemy nasze sumienie Słowu Bożemu, nie człowiekowi, ani grupie ludzi czy komitetowi. Wbrew temu, wielu wierzy, że Bóg powierzył tę władzę ogólnemu zgromadzeniu Generalnej Konferencji. Ale taka idea opiera się na błędnej interpretacji jednego konkretnego cytatu. Przeczytajmy ten cytat uważnie.


\egw{At times, when a small group of men entrusted with \textbf{the general management of the work} have, in the name of the General Conference, sought to carry out unwise plans and to restrict God’s work, I have said that I could no longer regard the voice of the General Conference, represented by these few men, as the voice of God. \textbf{But this is not saying that the decisions of a General Conference composed of an assembly of duly appointed, representative men from all parts of the field should not be respected}. \textbf{God has ordained that the representatives of His church from all parts of the earth, when assembled in a General Conference, \underline{shall have authority}}. The error that some are in danger of committing is in giving to the mind and judgment of one man, or of a small group of men, \textbf{the full measure of authority and influence that God has vested in His church in the judgment and voice of the General Conference assembled \underline{to plan for the prosperity and advancement of His work}}.}[9T 260.2; 1909][https://egwwritings.org/?ref=en\_9T.260.2&para=115.1474]


\egw{Czasami, gdy mała grupa mężczyzn, którym powierzono \textbf{ogólne zarządzanie dziełem}, w imieniu Generalnej Konferencji, starała się realizować niemądre plany i ograniczać Boże dzieło, mówiłam, że nie mogę już uważać głosu Generalnej Konferencji, reprezentowanego przez tych kilku mężczyzn, za głos Boga. \textbf{Ale nie oznacza to, że decyzje Generalnej Konferencji złożonej z należycie powołanego zgromadzenia przedstawicieli ze wszystkich części pola nie powinny być szanowane}. \textbf{Bóg postanowił, że przedstawiciele Jego kościoła ze wszystkich części ziemi, gdy zgromadzą się na Generalnej Konferencji, \underline{będą mieli autorytet}}. Błędem, który niektórzy są w niebezpieczeństwie popełnienia, jest przyznawanie umysłowi i osądowi jednego człowieka lub małej grupy ludzi \textbf{pełnej miary autorytetu i wpływu, który Bóg powierzył swojemu kościołowi w osądzie i głosie zgromadzonej Generalnej Konferencji \underline{aby planować dla pomyślności i rozwoju Jego dzieła}}.}[9T 260.2; 1909][https://egwwritings.org/?ref=en\_9T.260.2&para=115.1474]


Sister White pointed out that the world wide assembly of the General Conference meeting does have authority as the voice of God, yet she is very particular over what matters it has this authority. The authority God vested in the assembly of the General Conference is \egwinline{to plan for the prosperity and advancement of His work}. It is about making mission plans, not about managing beliefs or the conscience. God’s church does have His voice regarding beliefs; the voice of God pertaining to the faith is the Bible. The Bible is fully sufficient for us and we are free to render our conscience to it. No synopsis of any denominational faith has authority to dictate someone's faith; neither do \emcap{Fundamental Principles}, or current Fundamental Beliefs.\footnote{Although the Fundamental Principles were not designed to have authority over the people, nor were they designed to secure uniformity among them, as a system of faith, there is some evidence to the contrary. In his article, “\textit{Seventh-day Adventists and the Doctrine of the Trinity}”, of the “\textit{Christian Workers Magazine}”, 1915, D.M. Caright gave evidence that a Conference president used the \emcap{Fundamental Principles} as a test of fellowship in 1911. Such practice is not constructive to the Truth, neither is it beneficial for believers.} Sister White was very clear about the Bible being the only rule of faith, and every doctrine should be questioned with Scripture. In the Great Controversy, we read the following:


Siostra White wskazała, że ogólnoświatowe zgromadzenie Generalnej Konferencji ma autorytet jako głos Boga, jednak jest bardzo precyzyjna co do spraw, w których ma ten autorytet. Autorytet, który Bóg powierzył zgromadzeniu Generalnej Konferencji, jest \egwinline{aby planować dla pomyślności i rozwoju Jego dzieła}. Chodzi o tworzenie planów misyjnych, a nie o zarządzanie wierzeniami czy sumieniem. Boży kościół ma Jego głos dotyczący wierzeń; głosem Boga odnoszącym się do wiary jest Biblia. Biblia jest w pełni wystarczająca dla nas i jesteśmy wolni, aby poddać jej nasze sumienie. Żadne streszczenie jakiejkolwiek denominacyjnej wiary nie ma autorytetu, aby dyktować czyjąś wiarę; ani \emcap{Fundamentalne Zasady}, ani obecne Fundamentalne Wierzenia.\footnote{Chociaż Fundamentalne Zasady nie zostały zaprojektowane, aby mieć autorytet nad ludźmi, ani nie zostały zaprojektowane, aby zapewnić jednolitość wśród nich, jako system wiary, istnieją pewne dowody na przeciwne. W swoim artykule “\textit{Seventh-day Adventists and the Doctrine of the Trinity}” w “\textit{Christian Workers Magazine}” z 1915 roku, D.M. Caright przedstawił dowody, że przewodniczący Konferencji używał \emcap{Fundamentalnych Zasad} jako testu członkostwa w 1911 roku. Taka praktyka nie jest konstruktywna dla Prawdy, ani nie jest korzystna dla wierzących.} Siostra White była bardzo jasna co do tego, że Biblia jest jedyną regułą wiary, a każda doktryna powinna być kwestionowana za pomocą Pisma Świętego. W Wielkim Boju czytamy:


\egw{But God will have a people upon the earth \textbf{to maintain the Bible, and \underline{the Bible only}}, \textbf{as the standard of all doctrines and the basis of all reforms}. \textbf{The opinions of learned men, the deductions of science, \underline{the creeds or decisions of ecclesiastical councils}, as numerous and discordant as are the churches which they represent, the voice of the majority - not one nor all of these should be regarded as evidence for or against any point of religious faith.} \textbf{Before accepting any doctrine or precept, we should demand a plain ‘Thus saith the Lord’ in its support.}}[GC 595.1; 1888][https://egwwritings.org/?ref=en\_GC.595.1&para=132.2689]


\egw{Ale Bóg będzie miał lud na ziemi \textbf{który będzie stał przy Biblii, i \underline{tylko Biblii}}, \textbf{jako standardzie wszystkich doktryn i podstawie wszystkich reform}. \textbf{Opinie uczonych ludzi, wnioski nauki, \underline{wyznania wiary lub decyzje rad kościelnych}, tak liczne i niezgodne jak kościoły, które reprezentują, głos większości - żadne z nich ani wszystkie razem nie powinny być uważane za dowód za lub przeciw jakiemukolwiek punktowi wiary religijnej.} \textbf{Przed przyjęciem jakiejkolwiek doktryny lub nakazu, powinniśmy domagać się wyraźnego ‘Tak mówi Pan’ na jej poparcie.}}[GC 595.1; 1888][https://egwwritings.org/?ref=en\_GC.595.1&para=132.2689]


The liberty of conscience is the basics of protestantism and reformation. We hope and believe that every Seventh-day Adventist can exercise freedom to render his conscience to the Bible without being coerced by discipline, or any other means. The issue of the church's creed and discipline becomes more relevant today, when we have the promise that God will re-establish the original foundation of our faith. We hope and pray that the evidence brought up here will bring light to the church leadership and encourage them to eradicate the false practices in our midst. As the religious leaders in Christ’s time were entrusted with the duty to preserve the Truth and to recognize the time of God’s visitation, so it is today with the leaders of the Seventh-day Adventist Church. In what follows, we will present the prophecies God specifically gave to the Seventh-day Adventist Church. In our time, the end-time, all the pillars of our faith that were held in the beginning will be re-established. May every member of the Seventh-day Adventist Church recognize the importance of the revival that God is about to establish.


Wolność sumienia jest podstawą protestantyzmu i reformacji. Mamy nadzieję i wierzymy, że każdy Adwentysta Dnia Siódmego może korzystać z wolności poddania swojego sumienia Biblii bez bycia przymuszanym przez dyscyplinę lub jakiekolwiek inne środki. Kwestia wyznania wiary kościoła i dyscypliny staje się bardziej istotna dzisiaj, gdy mamy obietnicę, że Bóg przywróci pierwotny fundament naszej wiary. Mamy nadzieję i modlimy się, aby dowody przedstawione tutaj przyniosły światło przywódcom kościoła i zachęciły ich do wykorzenienia fałszywych praktyk w naszym środowisku. Tak jak przywódcy religijni w czasach Chrystusa byli obdarzeni obowiązkiem zachowania Prawdy i rozpoznania czasu Bożego nawiedzenia, tak jest dzisiaj z przywódcami Kościoła Adwentystów Dnia Siódmego. W dalszej części przedstawimy proroctwa, które Bóg dał specjalnie Kościołowi Adwentystów Dnia Siódmego. W naszych czasach, czasach końca, wszystkie filary naszej wiary, które były utrzymywane na początku, zostaną przywrócone. Niech każdy członek Kościoła Adwentystów Dnia Siódmego rozpozna znaczenie odrodzenia, które Bóg zamierza ustanowić.


% Steps to Apostasy

\begin{titledpoem}
    \stanza{
        A creed established beyond God's Word, \\
        The voice of conscience no longer heard. \\
        Fellowship tested by human decree, \\
        From Bible authority we slowly flee.
    }

    \stanza{
        Those who dissent labeled heretics, lost, \\
        Their faith and conviction at terrible cost. \\
        Persecution follows for standing apart, \\
        When creeds replace Scripture within the heart. \\
    }

    \stanza{
        The Bible alone should guide our belief, \\
        All other authorities bringing grief. \\
        Our conscience surrenders to God's Word divine, \\
        Not to councils of men who draw the line.
    }

    \stanza{
        The pioneers knew this freedom well, \\
        Against human creeds they chose to rebel. \\
        For truth must flourish where conscience is free, \\
        As God intended His church to be.
    }
\end{titledpoem}

% \begin{titledpoem}

    \stanza{
        Wyznanie wiary to stwierdzenie, \\
        Co nasze zniewala sumienie. \\
        Ludzkim dekretem członkostwa test — \\
        Po co nam Biblia, skoro tak jest?
    }

    \stanza{
        Kto się wyłamie, wszystko straci \\
        I cenę wiary swej zapłaci. \\
        Na zewnątrz będzie wyrzucony \\
        I heretykiem określony.
    }

    \stanza{
        Lecz Słowo Boże nam przewodzi \\
        I Jezus przy nas blisko chodzi. \\
        Nie cofaj się od przekonania — \\
        To prawdy lekcja do zbadania.
    }

    \stanza{
        Pionierzy dobrze to wiedzieli, \\
        Więc wyznania wiary nie chcieli. \\
        A prawda wstęp ma do sumienia, \\
        Które nie czuje zniewolenia.
    }

\end{titledpoem}
