\qrchapter{https://forgottenpillar.com/rsc/pl-fp-chapter27}{Kroki do apostazji}

W poniższym cytacie brat J. N. Loughborough, który był jednym z pionierów Kościoła Adwentystów Dnia Siódmego, ostrzegał nas przed pięcioma krokami do apostazji.

\others{\textbf{Pierwszym krokiem} do apostazji jest \textbf{utworzenie wyznania wiary}, mówiącego nam, w co mamy wierzyć. \textbf{Drugim} jest \textbf{uczynienie tego wyznania wiary testem przynależności}. \textbf{Trzecim} jest \textbf{wypróbowanie członków według tego wyznania}. \textbf{Czwartym} jest \textbf{potępienie jako heretyków tych, którzy nie wierzą w to wyznanie}. A \textbf{piątym}, \textbf{rozpoczęcie prześladowań przeciwko takim osobom}. Proszę, abyśmy nie naśladowali Kościołów w jakimkolwiek nieuzasadnionym sensie w odniesieniu do wymienionych kroków.}[John N. Loughborough, Review and Herald, 8 października 1861 r.][https://egwwritings.org/?ref=en\_ARSH.October.8.1861.p.149.7&para=1685.5326]

Te zasady są ważne, aby mieć je na uwadze, i powinniśmy zadać sobie pytanie, czy my, dzisiaj, naśladujemy kościoły w jakimkolwiek nieuzasadnionym sensie w proponowanym kroku. Co by się stało z adwentystą dnia siódmego, który odrzuciłby doktrynę o Trójcy na rzecz \emcap{Fundamentalnych Zasad}? Czy mamy jakieś ustanowione wyznanie wiary w naszym kościele? Czy na jego podstawie sprawdzamy nasze członkostwo ?


\emcap{Fundamentalne Zasady} miały inny charakter i inną rolę w Kościele Adwentystów Dnia Siódmego, w przeciwieństwie do wzorca przyjętego przez inne kościoły. \emcap{Fundamentalne Zasady} nie zostały zaprojektowane jako wyznanie wiary. W przedmowie do oświadczenia z 1872 roku czytamy o ich charakterze:

\others{Przedstawiając \textbf{publicznie} to \textbf{streszczenie naszej wiary}, chcemy, aby było wyraźnie zrozumiane, że \textbf{\underline{nie mamy artykułów wiary, wyznania wiary} lub dyscypliny \underline{poza Biblią}}. \textbf{Nie} przedstawiamy tego \textbf{\underline{jako mającego jakikolwiek autorytet wśród naszego ludu}}, \textbf{ani nie jest to zaprojektowane, aby zapewnić jednolitość wśród nich}, \textbf{jako system wiary}, \textbf{ale jest to krótkie oświadczenie o tym, co jest i było, z wielką jednomyślnością, przez nich wyznawane}.}[Oświadczenie o fundamentalnych zasadach nauczanych i wyznawanych przez Adwentystów Dnia Siódmego, 1872]

W przedmowie do oświadczenia z 1889 roku czytamy podobne poglądy:

\others{Jak stwierdzono gdzie indziej, Adwentyści Dnia Siódmego \textbf{nie mają innego wyznania wiary poza Biblią}; ale trzymają się \textbf{pewnych dobrze zdefiniowanych punktów wiary}, dla których \textbf{czują się przygotowani, aby dać powód ‘każdemu człowiekowi, który ich pyta’}. Następujące propozycje można uznać za podsumowanie \textbf{głównych cech ich religijnej wiary}, co do których, o ile nam wiadomo, \textbf{panuje całkowita jednomyślność w całym ciele}.}[Rocznik statystyczny Adwentystów Dnia Siódmego za 1889 r., str. 147, Fundamentalne Zasady Adwentystów Dnia Siódmego]

\emcap{Fundamentalne Zasady} nie zostały zaprojektowane, aby narzucać komuś wiarę. Wierzący, prowadzeni przez Ducha Świętego, dobrowolnie poddawali swoje sumienia Słowu Bożemu; pod wpływem Ducha Świętego dochodzili do tych samych wniosków. Panowała całkowita jednomyślność w całym ciele. Wszyscy wierzący czuli się „\textit{przygotowani, do podania wyjaśnień każdemu, kto ich zapyta}” w odniesieniu do ich wiary.

Dzisiaj widzimy uderzającą różnicę w zasadach i praktyce adwentystycznych wierzeń w porównaniu z naszymi pionierami. Utrzymujemy ducha jedności poprzez dyscyplinowanie naszych członków za zaprzeczanie Fundamentalnym Wierzeniom. W naszym podręczniku kościelnym, w sekcji „\textit{Powody do dyscypliny}”, czytamy pierwszy punkt, który określa sposób postępowania za zaprzeczanie wiary w Fundamentalne Wierzenia Adwentystów Dnia Siódmego.

\others{Powody do dyscypliny}

\others{1. \textbf{Zaprzeczanie wiary} w podstawy ewangelii i \textbf{w Fundamentalne Wierzenia Kościoła} lub \textbf{nauczanie doktryn sprzecznych z nimi}.}[SDA Church Manual, 20th edition, Revised 2022, str. 67][https://www.adventist.org/wp-content/uploads/2023/07/2022-Seventh-day-Adventist-Church-Manual.pdf]

Dyscyplinowanie kogoś z powodu jego wiary nie jest niczym innym jak przymuszaniem sumienia. Nasze sumienie powinniśmy poddawać wyłącznie Biblii — nie człowiekowi, radom czy kościelnym wyznaniom wiary. Dyscyplinowanie członków za zaprzeczanie Fundamentalnym Wierzeniom jest wyraźnym dowodem na to, że rzeczywiście mamy wyznanie wiary poza Biblią. Nie możemy korzystać z wolności sumienia w poddaniu się Słowu Bożemu, będąc jednocześnie ograniczeni zestawem wierzeń, które, jeśli zostaną zakwestionowane na podstawie autorytetu Biblii, będą skutkować dyscypliną. W naszej praktyce zapomnieliśmy o fundamencie protestantyzmu i reformacji. Wszyscy reformatorzy doświadczyli przymusu sumienia, nawet za cenę życia. Marcin Luter słynnie zastosował tę zasadę w swojej obronie przed Sejmem w Wormacji.

\others{Jeśli nie zostanę \textbf{przekonany przez Pismo Święte} i prosty rozum — nie uznaję autorytetu papieży i soborów, ponieważ zaprzeczały sobie nawzajem — \textbf{\underline{moje sumienie jest niewolnikiem Słowa Bożego}}. Nie mogę odwołać i nie odwołam niczego, ponieważ \textbf{działanie wbrew sumieniu nie jest ani słuszne, ani bezpieczne}. Tu stoję, nie mogę inaczej. Boże, pomóż mi. Amen.}[Bainton, 182]

Jeśli członek Kościoła Adwentystów Dnia Siódmego ma sumienie będące niewolnikiem Słowa Bożego i nie jest w zgodzie z Fundamentalnymi Wierzeniami Adwentystów Dnia Siódmego, jego sumienie nie powinno być przymuszane przez dyscyplinę kościelną. Wiemy, że w czasach końca cały Kościół Adwentystów Dnia Siódmego będzie przymuszany w kwestii szabatu. Walczyliśmy o wolność religijną, a jednak pozwalamy sobie na przymuszanie sumienia tych, którzy nie są w zgodzie z Fundamentalnymi Wierzeniami. Jeśli dziś dyscyplinujemy naszych członków za to, że nie poddają swojego sumienia ludziom, radom i wyznaniom wiary, jak będziemy postępować jutro, gdy rząd będzie dyscyplinował swoich obywateli za niepoddawanie sumienia jego władzy, gdy będzie wymuszał posłuszeństwo prawu sprzecznemu z Pismem Świętym?

Pionierzy adwentyzmu byli bardzo świadomi niebezpieczeństw wymuszania na członkach Kościoła posłuszeństwa sumienia. Wyrażanie ich wierzeń nie miało na celu tworzenia jedności. Byli gotowi uzasadniać swoją wiarę z Biblii, gdy ich o to proszono. Biblia była ich jedynym kredo i artykułem wiary.

W 1883 roku pojawiła się propozycja wprowadzenia podręcznika kościelnego do Kościoła Adwentystów Dnia Siódmego. Ta propozycja została odrzucona po dokładnym zbadaniu przez komitet powołany przez Generalną Konferencję. W artykule „\textit{No Church Manual}” (Brak podręcznika kościelnego) czytamy ich powody odrzucenia proponowanego podręcznika kościelnego.

\others{\textbf{Podczas gdy bracia, którzy popierali podręcznik, zawsze twierdzili, że takie dzieło nie miało być w żaden sposób podobne do wyznania wiary czy dyscypliny, ani mieć autorytetu do rozstrzygania spornych kwestii}, ale miało być jedynie uważane za książkę zawierającą wskazówki pomocne dla osób o małym doświadczeniu, \textbf{to jednak musi być oczywiste, że takie dzieło, wydane pod auspicjami Generalnej Konferencji, natychmiast niosłoby ze sobą dużą wagę autorytetu i byłoby konsultowane przez większość naszych młodszych duchownych}. \textbf{\underline{Stopniowo kształtowałoby i formowałoby całe ciało}}; \textbf{a ci, którzy by się do niego nie stosowali, byliby uważani za niezgodnych z ustalonymi zasadami porządku kościelnego}. \textbf{I czy nie jest to właśnie celem podręcznika?} I jaki byłby z niego pożytek, gdyby nie miał osiągnąć takiego rezultatu? Ale czy ten rezultat, ogólnie rzecz biorąc, byłby korzystny? Czy nasi duchowni byliby bardziej otwarci, bardziej oryginalni, bardziej samodzielni? Czy można by na nich bardziej polegać w wielkich sytuacjach kryzysowych? Czy jest prawdopodobne, że ich duchowe doświadczenia byłyby głębsze, a ich osąd bardziej wiarygodny? \textbf{Uważamy, że tendencja jest zupełnie odwrotna}.}[No Church Manual, The Review and Herald, 27 października 1883, str. 745][https://documents.adventistarchives.org/Periodicals/RH/RH18831127-V60-47.pdf]

\others{\textbf{Biblia zawiera nasze wyznanie wiary i dyscyplinę. \underline{Całkowicie} wyposaża człowieka Bożego do wszelkiego dobrego dzieła}. To, czego nie ujawniła w odniesieniu do organizacji i zarządzania kościołem, obowiązków urzędników i duchownych oraz pokrewnych tematów, nie powinno być ściśle definiowane i rozpisywane na szczegółowe specyfikacje dla zachowania jednolitości, \textbf{ale raczej pozostawione indywidualnemu osądowi pod kierownictwem Ducha Świętego}. \textbf{Gdyby najlepiej było mieć księgę wskazówek tego rodzaju, Duch niewątpliwie poszedłby dalej i pozostawił taką z pieczęcią natchnienia na niej}.}[Tamże.][https://documents.adventistarchives.org/Periodicals/RH/RH18831127-V60-47.pdf]

Od 1883 roku Kościół Adwentystów Dnia Siódmego znacznie się rozrósł; więc dla wygody, w 1931 roku, Komitet Generalnej Konferencji zagłosował za opublikowaniem podręcznika kościelnego.\footnote{Maratas, Prince. „Church Manual.” General Conference of Seventh-Day Adventists, 20 sierpnia 2023, \href{https://gc.adventist.org/church-manual/}{gc.adventist.org/church-manual/}. Dostęp 3 lutego 2025.} Kościół, jako zorganizowane ciało, powinien stosować porządek i dyscyplinę w sprawach organizacji i planów dla powodzenia misji Kościoła. Ale żaden komitet nie powinien sprawować władzy nad czyimś sumieniem i czyimiś przekonaniami. Tylko Bóg ma prawo do tej władzy. Dlatego Biblia jest naszym jedynym wyznaniem wiary. Poddajemy nasze sumienie Słowu Bożemu, nie człowiekowi ani grupie ludzi czy komitetowi. Wbrew temu wielu wierzy, że Bóg powierzył tę władzę ogólnemu zgromadzeniu Generalnej Konferencji. Lecz taka idea opiera się na błędnej interpretacji jednego konkretnego cytatu. Przeczytajmy ten cytat uważnie.

\egw{Czasami, gdy mała grupa ludzi, którym powierzono \textbf{ogólne zarządzanie dziełem}, w imieniu Generalnej Konferencji, starała się realizować niemądre plany i ograniczać Boże dzieło, mówiłam, że nie mogę już uważać głosu Generalnej Konferencji, reprezentowanego przez tych kilku ludzi, za głos Boga. \textbf{Ale nie oznacza to, że decyzje Generalnej Konferencji złożonej z należycie powołanego zgromadzenia przedstawicieli ze wszystkich części terenu nie powinny być szanowane}. \textbf{Bóg postanowił, że przedstawiciele Jego Kościoła ze wszystkich części ziemi, gdy zgromadzą się na Generalnej Konferencji, \underline{będą mieli autorytet}}. Błędem, którego popełnienie grozi niektórym, jest przyznawanie umysłowi i osądowi jednego człowieka lub małej grupy ludzi \textbf{pełnej miary autorytetu i wpływu, który Bóg powierzył swojemu Kościołowi w osądzie i głosie zgromadzonej Generalnej Konferencji \underline{w planowaniu dla pomyślności i rozwoju Jego dzieła}}}[9T 260.2; 1909][https://egwwritings.org/?ref=en\_9T.260.2&para=115.1474]

Siostra White wskazała, że ogólnoświatowe zgromadzenie Generalnej Konferencji ma autorytet jako głos Boga, jednak jest bardzo precyzyjna co do spraw, w których ma ten autorytet. Autorytet, który Bóg powierzył zgromadzeniu Generalnej Konferencji, jest \egwinline{w planowaniu dla pomyślności i rozwoju Jego dzieła}. Chodzi o tworzenie planów misyjnych, a nie o zarządzanie wierzeniami czy sumieniem. Boży Kościół ma Jego głos dotyczący wierzeń; głosem Boga odnoszącym się do wiary jest Biblia. Biblia jest w pełni wystarczająca dla nas i jesteśmy wolni, aby poddać jej nasze sumienie. Żadne streszczenie jakiejkolwiek denominacyjnej wiary nie ma autorytetu, aby dyktować czyjąś wiarę; ani \emcap{Fundamentalne Zasady}, ani obecne Fundamentalne Wierzenia.\footnote{Chociaż Fundamentalne Zasady nie zostały ułożone po to, aby mieć autorytet nad ludźmi, ani nie zostały ułożone, aby zapewnić jednolitość wśród nich, jako system wiary, istnieją pewne dowody, które temy zaprzeczają. W swoim artykule „\textit{Seventh-day Adventists and the Doctrine of the Trinity}” w „\textit{Christian Workers Magazine}” z 1915 roku D. M. Caright przedstawił dowody, że przewodniczący Konferencji używał \emcap{Fundamentalnych Zasad} jako testu członkostwa w 1911 roku. Taka praktyka nie jest konstruktywna dla Prawdy ani nie jest korzystna dla wierzących.} Siostra White była bardzo jasna co do tego, że Biblia jest jedyną regułą wiary, a każda doktryna powinna być kwestionowana za pomocą Pisma Świętego. W Wielkim Boju czytamy:

\egw{Ale Bóg będzie miał lud na ziemi \textbf{który będzie stał przy Biblii, i \underline{tylko Biblii}}, \textbf{jako standardzie wszystkich doktryn i podstawie wszystkich reform}. \textbf{Opinie uczonych ludzi, wnioski nauki, \underline{wyznania wiary lub decyzje rad kościelnych}, tak liczne i niezgodne jak Kościoły, które reprezentują, głos większości — żadne z nich ani wszystkie razem nie powinny być uważane za dowód za lub przeciw jakiemukolwiek punktowi wiary religijnej.} \textbf{Przed przyjęciem jakiejkolwiek doktryny lub nakazu powinniśmy domagać się wyraźnego «Tak mówi Pan» na jej poparcie}}[GC 595.1; 1888][https://egwwritings.org/?ref=en\_GC.595.1&para=132.2689]

Wolność sumienia jest podstawą protestantyzmu i reformacji. Mamy nadzieję i wierzymy, że każdy adwentysta dnia siódmego może korzystać z wolności poddania swojego sumienia Biblii bez bycia przymuszanym przez dyscyplinę lub jakiekolwiek inne środki. Kwestia wyznania wiary Kościoła i dyscypliny staje się bardziej istotna dzisiaj, gdy mamy obietnicę, że Bóg przywróci pierwotny fundament naszej wiary. Mamy nadzieję i modlimy się, aby dowody przedstawione tutaj przyniosły światło przywódcom Kościoła i zachęciły ich do wykorzenienia fałszywych praktyk w naszym środowisku. Tak jak przywódcy religijni w czasach Chrystusa byli obdarzeni obowiązkiem zachowania Prawdy i rozpoznania czasu Bożego nawiedzenia, tak jest i dzisiaj z przywódcami Kościoła Adwentystów Dnia Siódmego. W dalszej części przedstawimy proroctwa, które Bóg dał specjalnie Kościołowi Adwentystów Dnia Siódmego. W naszych czasach, czasach końca, wszystkie filary naszej wiary, które były utrzymywane na początku, zostaną przywrócone. Niech każdy członek Kościoła Adwentystów Dnia Siódmego rozpozna znaczenie odrodzenia, które Bóg zamierza zaprowadzić.

\begin{titledpoem}

    \stanza{
        Wyznanie wiary to stwierdzenie, \\
        Co nasze zniewala sumienie. \\
        Ludzkim dekretem członkostwa test — \\
        Po co nam Biblia, skoro tak jest?
    }

    \stanza{
        Kto się wyłamie, wszystko straci \\
        I cenę wiary swej zapłaci. \\
        Na zewnątrz będzie wyrzucony \\
        I heretykiem określony.
    }

    \stanza{
        Lecz Słowo Boże nam przewodzi \\
        I Jezus przy nas blisko chodzi. \\
        Nie cofaj się od przekonania — \\
        To prawdy lekcja do zbadania.
    }

    \stanza{
        Pionierzy dobrze to wiedzieli, \\
        Więc wyznania wiary nie chcieli. \\
        A prawda wstęp ma do sumienia, \\
        Które nie czuje zniewolenia.
    }

\end{titledpoem}
