\chapter{Heaven's reality}


\chapter{Rzeczywistość nieba}


The \emcap{personality of God} deals with the quality or state of God being a person. Whenever we look at the pioneer's work on the \emcap{personality of God}, we see that they were all in harmony with the view that God is a tangible \textit{being}, possessing both body and parts. We always see the same underlying reasoning, which differentiates the term ‘\textit{spirit}’ and the term ‘\textit{being}’. By differentiating these terms, they explain the quality or state of God being a person\footnote{\href{https://www.merriam-webster.com/dictionary/personality}{Merriam-Webster Dictionary} defines the word ‘\textit{personality}’ as “\textit{quality or state of being a person}”.}—a \emcap{personality of God}. All their conclusions are summed up in the first point of the \emcap{Fundamental Principles}. \others{There is \textbf{one God}, a \textbf{personal}, \textbf{spiritual being}, the Creator of all things, omnipotent, omniscient, … and \textbf{every-where present by his representative, the Holy Spirit}. Psalm 139:7.}[FPSDA 1.2][https://egwwritings.org/?ref=en\_FPSDA.1.2&para=1299.6]


\emcap{Osobowość Boga} dotyczy właściwości lub stanu Boga jako osoby. Ilekroć patrzymy na pracę pionierów dotyczącą \emcap{osobowości Boga}, widzimy, że wszyscy byli zgodni co do poglądu, że Bóg jest namacalną \textit{istotą}, posiadającą zarówno ciało, jak i części. Zawsze widzimy to samo podstawowe rozumowanie, które rozróżnia termin ‘\textit{duch}’ i termin ‘\textit{istota}’. Rozróżniając te terminy, wyjaśniają właściwość lub stan Boga jako osoby\footnote{\href{https://www.merriam-webster.com/dictionary/personality}{Słownik Merriam-Webster} definiuje słowo ‘\textit{osobowość}’ jako “\textit{właściwość lub stan jako osoby}”.}—\emcap{osobowość Boga}. Wszystkie ich wnioski są podsumowane w pierwszym punkcie \emcap{Fundamentalnych Zasad}. \others{Jest \textbf{jeden Bóg}, \textbf{osobowa}, \textbf{duchowa istota}, Stwórca wszystkich rzeczy, wszechmogący, wszechwiedzący, … i \textbf{wszędzie obecny przez swojego przedstawiciela, Ducha Świętego}. Psalm 139:7.}[FPSDA 1.2][https://egwwritings.org/?ref=en\_FPSDA.1.2&para=1299.6]


So far, in the pioneers’ work, we have seen that the \emcap{personality of God} is tightly connected to the reality of God’s presence. God is a personal spiritual being, having a body and shape; as such, His presence is cumbered to one locality—as the Bible says, in His temple, at His throne where He is surrounded with unapproachable glory. But He is everywhere present by His representative, the Holy Spirit. Obviously, the Holy Spirit is a spirit, and not a being, \bible{for a spirit hath not flesh and bones as ye see me have}, said Jesus (Luke 24:39). Christ is also a Being, like His Father. He is an express image of the Father’s person; therefore, He bears the same personality, or quality or state of being a person, as His Father does.


Do tej pory w pracach pionierów widzieliśmy, że \emcap{osobowość Boga} jest ściśle związana z rzeczywistością Bożej obecności. Bóg jest osobową duchową istotą, mającą ciało i kształt; jako taki, Jego obecność jest ograniczona do jednego miejsca—jak mówi Biblia, w Jego świątyni, przy Jego tronie, gdzie otacza Go niedostępna chwała. Ale jest wszędzie obecny przez swojego przedstawiciela, Ducha Świętego. Oczywiście, Duch Święty jest duchem, a nie istotą, \bible{bo duch nie ma ciała ani kości, jak widzicie, że ja mam}, powiedział Jezus (Łukasz 24:39). Chrystus jest również Istotą, jak Jego Ojciec. Jest wyrazem istoty Ojca; dlatego nosi tę samą osobowość, czyli właściwość lub stan jako osoby, co Jego Ojciec.


In our experience, when we present the original Seventh-day Adventist beliefs on the \emcap{personality of God} to our trinitarian brothers, as expressed in the first two points of the \emcap{Fundamental Principles}, they often claim that the statements in the \emcap{Fundamental Principles} are correct in some way, but the understanding attributed to the terms “\textit{personal spiritual being}” are false. They usually attempt to harmonize the \emcap{Fundamental Principles} with the Trinity doctrine by twisting the words “\textit{spiritual being}”, as if the word ‘\textit{spiritual}’ means something mysterious, suitable to equalize the \emcap{personality of God} and of Christ with the personality of the Holy Ghost\footnote{The quality or state of the Holy Spirit being a person is bearing witness, not having the form of a person. \egw{\textbf{The Holy Spirit has a personality}, \textbf{\underline{else} He could not \underline{bear witness} to our spirits} and with our spirits that we are the children of God. \textbf{He must also be a divine person}, \textbf{\underline{else} He could not \underline{search out} the secrets which lie hidden in the mind of God}. ‘For what man knoweth the things of a man save the spirit of man, which is in him; even so the things of God knoweth no man, but the Spirit of God.’ [1 Corinthians 2:11.]}[21LtMs, Ms 20, 1906, par. 32][https://egwwritings.org/read?panels=p14071.10296041&index=0]. It is crystal clear that the Holy Spirit is a person, yet not in the same way as the Father and the Son, as the Holy Spirit does not possess the quality of an outward physical personage like the Father and the Son do.}. The underlying problem comes down to the understanding of the heavenly realities. The Bible is not silent about heaven and its realities, and our pioneers understood it well. Below we read about the explanation of the terms “\textit{spiritual being}” from James White and Uriah Smith in their book, “\textit{The Biblical Institute}”. The Bible explains these terms using the example of angels, which are “\textit{spiritual beings}”.


W naszym doświadczeniu, gdy przedstawiamy pierwotne wierzenia Adwentystów Dnia Siódmego dotyczące \emcap{osobowości Boga} naszym trynitarnym braciom, wyrażone w pierwszych dwóch punktach \emcap{Fundamentalnych Zasad}, często twierdzą oni, że stwierdzenia w \emcap{Fundamentalnych Zasadach} są w pewien sposób poprawne, ale przypisane rozumienie terminów “\textit{osobowa duchowa istota}” jest fałszywe. Zwykle próbują zharmonizować \emcap{Fundamentalne Zasady} z doktryną o Trójcy poprzez przekręcanie słów “\textit{duchowa istota}”, jakby słowo ‘\textit{duchowa}’ oznaczało coś tajemniczego, odpowiedniego do zrównania \emcap{osobowości Boga} i Chrystusa z osobowością Ducha Świętego\footnote{Właściwość lub stan Ducha Świętego jako osoby polega na świadczeniu, a nie na posiadaniu formy osoby. \egw{\textbf{Duch Święty ma osobowość}, \textbf{\underline{inaczej} nie mógłby \underline{świadczyć} naszemu duchowi} i z naszym duchem, że jesteśmy dziećmi Bożymi. \textbf{Musi On również być boską osobą}, \textbf{\underline{inaczej} nie mógłby \underline{badać} tajemnic, które są ukryte w umyśle Boga}. ‘Bo któż z ludzi wie, co jest w człowieku, oprócz ducha ludzkiego, który w nim jest? Tak samo i tego, co jest w Bogu, nikt nie zna, oprócz Ducha Bożego.’ [1 Koryntian 2:11.]}[21LtMs, Ms 20, 1906, par. 32][https://egwwritings.org/read?panels=p14071.10296041&index=0]. Jest krystalicznie jasne, że Duch Święty jest osobą, jednak nie w ten sam sposób co Ojciec i Syn, ponieważ Duch Święty nie posiada cechy zewnętrznej fizycznej osobowości jak Ojciec i Syn.}. Podstawowy problem sprowadza się do zrozumienia niebiańskich rzeczywistości. Biblia nie milczy o niebie i jego rzeczywistościach, a nasi pionierzy dobrze to rozumieli. Poniżej czytamy wyjaśnienie terminów “\textit{duchowa istota}” od Jamesa White'a i Uriaha Smitha w ich książce “\textit{The Biblical Institute}”. Biblia wyjaśnia te terminy na przykładzie aniołów, którzy są “\textit{duchowymi istotami}”.


\others{\textbf{Angels are real beings}. They are described in the Bible as \textbf{possessing face, feet, wings} \&x. Ezekiel says of the cherubim, ‘\textbf{Their whole \underline{body} and their backs and their hands and their wings},’ \&c. Eze. 10:12. Angels \textbf{appeared }unto Abraham. Gen. 18:1-8. They talked and ate with him. They went on to Sodom and communed with Lot, who, entering into his house baked unleavened bread for them and they did eat. \textbf{These person were called angels}. David speaks of the manna as the corn of Heaven and angel’s food. Ps. 78:23-25.}


\others{\textbf{Aniołowie są rzeczywistymi istotami}. Są opisani w Biblii jako \textbf{posiadający twarz, stopy, skrzydła} \&x. Ezechiel mówi o cherubinach: ‘\textbf{Całe ich \underline{ciało}, ich plecy, ich ręce i ich skrzydła},’ \&c. Ez. 10:12. Aniołowie \textbf{ukazali się} Abrahamowi. Rdz. 18:1-8. Rozmawiali i jedli z nim. Poszli do Sodomy i rozmawiali z Lotem, który wszedłszy do swego domu, upiekł dla nich przaśny chleb, a oni jedli. \textbf{Te osoby były nazywane aniołami}. Dawid mówi o mannie jako o zbożu z Nieba i pokarmie aniołów. Ps. 78:23-25.}


\othersnogap{The case of Balaam, Num. 22:22-31, is an interesting incident. The angel \textbf{appeared }to Balaam with a sword \textbf{drawn in his hand}. The question is sometimes asked \textbf{how angels can be \underline{material beings since we cannot see them}. This case illustrates it}. The record says the \textbf{Lord opened the eyes of Balaam and he saw the angel}. \textbf{The angel did not create a body for that occasion}.\textbf{ He was just the same as he was before Balaam saw him; \underline{but the change took place in Balaam}. His eyes were opened, then he beheld the angel}. It was the same with the servant of Elisha when he and his master were brought into a straight place, surrounded by the army of the king of Syria. 2 Kings 6:17. Elisha prayed that \textbf{the eyes of his servant might be opened}; and he immediately saw the whole mountain full of horses and chariots round about Elisha.}


\othersnogap{Przypadek Balaama, Lb. 22:22-31, jest interesującym zdarzeniem. Anioł \textbf{ukazał się} Balaamowi z \textbf{wyciągniętym mieczem w ręku}. Czasami zadawane jest pytanie, \textbf{jak aniołowie mogą być \underline{materialnymi istotami, skoro nie możemy ich widzieć}. Ten przypadek to ilustruje}. Zapis mówi, że \textbf{Pan otworzył oczy Balaama i zobaczył on anioła}. \textbf{Anioł nie stworzył ciała na tę okazję}. \textbf{Był dokładnie taki sam jak przed tym, gdy Balaam go zobaczył; \underline{ale zmiana zaszła w Balaamie}}. Jego oczy zostały otwarte, wtedy ujrzał anioła. Tak samo było ze sługą Elizeusza, gdy on i jego pan znaleźli się w trudnym położeniu, otoczeni przez armię króla Syrii. 2 Krl 6:17. Elizeusz modlił się, aby \textbf{oczy jego sługi zostały otwarte}; i natychmiast zobaczył on całą górę pełną koni i rydwanów wokół Elizeusza.}


\othersnogap{\textbf{This may be further illustrated referring to things which we know are material and yet which we cannot see}. Air is material, light is material, even thought itself is only the result of material organizations — matter acting upon matter — and yet we can see none of these things. \textbf{Just so with the angels}.}


\othersnogap{\textbf{Można to dalej zilustrować odnosząc się do rzeczy, które wiemy, że są materialne, a jednak których nie możemy zobaczyć}. Powietrze jest materialne, światło jest materialne, nawet myśl sama w sobie jest tylko wynikiem materialnych organizacji — materia działająca na materię — a jednak nie możemy zobaczyć żadnej z tych rzeczy. \textbf{Tak samo jest z aniołami}.}


\othersnogap{\textbf{It is further objected to the materiality of the angels that they are called spirits. }Heb. 1:13, 14.\textbf{\underline{But this is no objection to their being literal beings}}. \textbf{They are simply spiritual beings organized differently from these earthly bodies which we possess}. Paul says, 1 Cor. 15:44, ‘\textbf{There is a natural body and there is \underline{a spiritual body}}.’ \textbf{The natural body we now have; the spiritual body we shall have in the resurrection}. ‘\textbf{It is raised a spiritual body}.’ Verse 44. \textbf{But then we are equal unto the angels}, Luke 20:36; \textbf{then we have bodies like unto Christ’s most glorious body}. Phil. 3:4 \textbf{and Christ is no less a spirit than the angels}. \textbf{We read that God is a spirit, that is, simply \underline{a spiritual being}}.}[James White and Uriah Smith, The Biblical Institute (Kindle Locations 2537-2553). Kindle Edition.]


\othersnogap{\textbf{Dalszym zarzutem wobec materialności aniołów jest to, że są nazywani duchami}. Hbr. 1:13, 14. \textbf{\underline{Ale to nie jest zarzut przeciwko temu, że są dosłownymi istotami}}. \textbf{Są po prostu duchowymi istotami zorganizowanymi inaczej niż te ziemskie ciała, które posiadamy}. Paweł mówi, 1 Kor. 15:44, ‘\textbf{Jest ciało cielesne i jest \underline{ciało duchowe}}’. \textbf{Ciało cielesne mamy teraz; ciało duchowe będziemy mieli przy zmartwychwstaniu}. ‘\textbf{Powstanie jako ciało duchowe}’. Werset 44. \textbf{Wtedy będziemy równi aniołom}, Łk 20:36; \textbf{wtedy będziemy mieli ciała podobne do najchwalebniejszego ciała Chrystusa}. Flp. 3:4 \textbf{a Chrystus nie jest mniej duchem niż aniołowie}. \textbf{Czytamy, że Bóg jest duchem, to znaczy po prostu \underline{duchową istotą}}.}[James White and Uriah Smith, The Biblical Institute (Kindle Locations 2537-2553). Kindle Edition.]


The Bible gives us the insight that angels are spiritual beings that possess material bodies, but are still unseen to us, unless the Lord opens our eyes to see them. When the righteous will rise up in their new glorified bodies, they will rise in a spiritual body, an incorruptible one. This body will be tangible and material just as the new Earth will be tangible and material. And with our spiritual bodies we will possess the renewed Earth, we will replenish it \bible{and subdue it: and have dominion over the fish of the sea, and over the fowl of the air, and over every living thing that moveth upon the earth}[Genesis 1:28].


Biblia daje nam wgląd w to, że aniołowie są duchowymi istotami, które posiadają materialne ciała, ale są dla nas niewidzialni, chyba że Pan otworzy nasze oczy, abyśmy ich zobaczyli. Kiedy sprawiedliwi powstaną w swoich nowych uwielbionych ciałach, powstaną w ciele duchowym, niezniszczalnym. To ciało będzie namacalne i materialne, tak jak nowa Ziemia będzie namacalna i materialna. A z naszymi duchowymi ciałami będziemy posiadać odnowioną Ziemię, będziemy ją \bible{napełniać i czynić ją sobie poddaną; i panować nad rybami morskimi i nad ptactwem niebieskim, i nad wszelkim zwierzęciem, które porusza się po ziemi}[Księga Rodzaju 1:28].
