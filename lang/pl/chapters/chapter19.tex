\qrchapter{https://forgottenpillar.com/rsc/pl-fp-chapter19}{Ellen White i Ewangelia Mateusza 28:19}

Wielu twierdzi, że Ellen White promowała doktrynę o Trójcy i że to ona jest odpowiedzialna za przyjęcie jej do naszych szeregów. Twierdzenia te nie uwzględniają faktu, że broniła ona \emcap{osobowości Boga} przedstawionej w pierwszym punkcie \emcap{Fundamentalnych Zasad}. Na poparcie twierdzeń, że Ellen White była trynitarna, przedstawiane są cytaty z jej komentarza do Ew. Mateusza 28:19:

\bible{Idźcie więc i nauczajcie wszystkie narody, \textbf{chrzcząc je w imię \underline{Ojca} i \underline{Syna}, i \underline{Ducha Świętego}}}[Mt 28:19].

Ten werset jest najbardziej znanym w poparciu doktryny o Trójcy. Doktryna o Trójcy zawiera stwierdzenia o \emcap{osobowości Boga}, których ten tekst w żaden sposób nie popiera. Sam ten werset nie naucza, że Ojciec, Syn i Duch Święty tworzą \textit{jednego} Boga, Boga Biblii. Istnieją inne wyraźne wersety w Biblii, które wykluczają taką interpretację tekstu, tj. 1Kor 8:4--6; J 17:3; Ef 4:4--6; 1Tm 2:5.

Niestety, te same nieuzasadnione założenia dotyczące Mt 28:19 są czynione względem cytatów siostry White odnoszących się do tego wersetu. Na przykład siostra White używa terminów takich, jak: \egwinline{trzy najwyższe moce w niebie}[Lt253a-1903.18; 1903][https://egwwritings.org/read?panels=p10143.25], \egwinline{trzy wielkie moce nieba}[8T 254.1; 1904][https://egwwritings.org/read?panels=p112.1450], \egwinline{trzej święci dostojnicy nieba}[Ms92-1901.26: 1901][https://egwwritings.org/read?panels=p10732.32], i podobnych zwrotów — jednak żaden z tych cytatów nie uzasadnia założenia, że ci trzej (Ojciec, Syn i Duch Święty) tworzą \textit{jednego} Boga. Wręcz przeciwnie, jak omówiono w poprzednim rozdziale, gdy osadzi się poglądy Williama Boardmana  i \egwinline{niebiańskie trio} w kontekście, poglądy typu „\textit{trzy-w-jednym}” \egwinline{nie są godne zaufania} [Ms21-1906.8; 1906][https://egwwritings.org/read?panels=p9754.15].

Niebiańskie trio (grupa trzech: Ojciec, Syn i Duch Święty) są również obecne w innych wersetach biblijnych oprócz Mt 28:19. Istnieje kilka innych przykładów w Nowym Testamencie, gdzie wymienieni są Ojciec, Syn i Duch Święty, i wersety te powinny być używane do interpretacji znaczenia niebiańskiego trio. Żaden z wersetów o niebiańskim trio nie dowodzi istnienia Boga trzy-w-jednym; przeciwnie, wszystkie odnoszą się do Ojca jako jedynego Boga. W następujących wersetach niebiańskie trio jest pogrubione, aby lepiej rozróżnić Ojca, Syna i Ducha Świętego.

\bible{Jest jedno ciało i \textbf{jeden Duch}, jak też zostaliście powołani w jednej nadziei waszego powołania; \textbf{jeden Pan}, jedna wiara, jeden chrzest; \textbf{jeden Bóg i Ojciec} wszystkich, który jest ponad wszystkimi, przez wszystkich i w was wszystkich}[Ef 4:4--6].

\bible{A różne są dary, lecz \textbf{ten sam Duch}. Różne też są posługi, ale \textbf{ten sam Pan}. I różne są działania, lecz \textbf{ten sam jest Bóg}, który sprawia wszystko we wszystkich}[1Kor 12:4--6].`

\bible{Łaska \textbf{Pana Jezusa Chrystusa} i miłość \textbf{Boga}, i społeczność \textbf{Ducha Świętego} niech będą z wami wszystkimi. Amen}[2Kor 13:14].

\bible{Albowiem przez \textbf{niego} \normaltext{[Chrystusa]} my wszyscy w \textbf{jednym Duchu} mamy dostęp do \textbf{Ojca}}[Ef 2:18].

\bible{Lecz my powinniśmy zawsze dziękować \textbf{Bogu} za was, bracia umiłowani przez \textbf{Pana}, że \textbf{Bóg} od początku wybrał was do zbawienia przez uświęcenie \textbf{Ducha} i wiarę w prawdę}[2Tes 2:13].

\bible{To o ile bardziej krew \textbf{Chrystusa}, który przez wiecznego \textbf{Ducha} ofiarował samego siebie bez skazy \textbf{Bogu}, oczyści wasze sumienie z martwych uczynków, by służyć \textbf{żyjącemu Bogu}?}[Hbr 9:14].

\bible{Wybranym według uprzedniej wiedzy \textbf{Boga Ojca}, przez uświęcenie \textbf{Ducha} dla posłuszeństwa i pokropienia krwią \textbf{Jezusa Chrystusa}. Łaska wam i pokój niech się pomnożą}[1P 1:2].

Wszystkie powyższe wersety mówią o niebiańskim trio (Ojcu, Synu i Duchu Świętym) i wszystkie one konsekwentnie świadczą, że Ojciec jest tym, który jest określany jako Bóg. To samo rozumowanie ma zastosowanie w interpretacji Mt 28:19 przez Ellen White.

\egw{Chrystus dał swoim naśladowcom wyraźną obietnicę, że po swoim wniebowstąpieniu pośle im swojego Ducha. «Idźcie więc», powiedział, «i nauczajcie wszystkie narody, chrzcząc je w imię \textbf{Ojca (osobowego Boga)} i \textbf{Syna (osobowego Księcia i Zbawiciela)}, i \textbf{Ducha Świętego (posłanego z nieba, aby reprezentować Chrystusa)}; ucząc je przestrzegać wszystkiego, co wam przykazałem, a oto ja jestem z wami zawsze, aż do końca świata». Mt 28:19--20}[RH, 26 października 1897, akap. 9.][https://egwwritings.org/read?panels=p821.16317]

Nawiasy w tym cytacie znajdują się w oryginalnym rękopisie autorstwa Ellen White. Tutaj podaje ona swoją własną interpretację Mt 28:19. Ojciec jest osobowym Bogiem, Syn jest osobowym Księciem i Zbawicielem, a Duch Święty jest przedstawicielem Chrystusa. Ta interpretacja jest zgodna z \emcap{osobowością Boga} wyrażoną w pierwszym punkcie \emcap{Fundamentalnych Zasad}. Mt 28:19 jest kwestią interpretacji. Interpretacja, która czyni Niebiańskie Trio jednym Bogiem, nie jest natchniona. Nie to wskazuje tekst. Zamiast tego, przeczytajmy Mateusza 28:19 w natchnionym zestawieniu: „\textit{Idźcie więc i nauczajcie wszystkie narody, chrzcząc je w imię osobowego Boga, osobowego Księcia i Zbawiciela, i Ducha Świętego}”. Gdyby ktoś przeczytał tekst w ten sposób, nikt nigdy nie założyłby, że jeden Bóg jest jednością trzech osób. Dlatego trzymajmy się natchnienia, a nie wybiegów\footnote{\href{https://egwwritings.org/?ref=en\_Lt232-1903.41&para=10197.50}{{EGW, Lt232-1903.41; 1903}}}.

\egw{Niech będą wdzięczni Bogu za Jego obfite miłosierdzie i niech będą uprzejmi wobec siebie nawzajem. \textbf{Mają \underline{jednego Boga} i \underline{jednego Zbawiciela}; i \underline{jednego Ducha} — \underline{Ducha Chrystusa} — który ma wprowadzić jedność w ich szeregi}}[9T 189.3; 1909][https://egwwritings.org/read?panels=p115.1057]

W świetle przedstawionych dowodów widzimy, że samo wymienienie Ojca, Syna i Ducha Świętego nie dowodzi założenia \textit{trzy-w-jednym}, ani nie jest w konflikcie z \emcap{osobowością Boga} wyrażoną w \emcap{Fundamentalnych Zasadach}. Nie ma zaprzeczenia trzech osób Bóstwa, ale jedynie zaprzeczenie założenia, że te Trzy Wielkie Osobistości tworzą jednego Boga.

Mt 28:19 jest cennym wersetem i otwiera nowe pole badań w Biblii i Duchu Proroctwa. W kontekście książki \textit{The Living Temple}, jak i odnosząc się do jej poglądów, siostra White napisała, że ten werset powinien być badany z największą gorliwością, ponieważ nie jest on nawet w połowie zrozumiany.

\egw{Tuż przed swoim wniebowstąpieniem Chrystus dał swoim uczniom wspaniałą instrukcję, \textbf{jak zapisano w dwudziestym ósmym rozdziale Ewangelii Mateusza}. \textbf{Rozdział ten zawiera pouczenie}, które nasi duchowni, nasi \textbf{lekarze}, nasza młodzież i wszyscy członkowie naszego Kościoła muszą \textbf{badać \underline{z największą gorliwością}}. \textbf{Ci, którzy zapoznają się z tymi instrukcjami tak, jak powinni, \underline{nie odważą się propagować teorii, które nie mają podstaw w Słowie Bożym}}. Moi bracia i siostry, uczyńcie Pismo Święte, które zawiera alfę i omegę wiedzy, przedmiotem waszego studium. \textbf{W całym Starym i Nowym Testamencie są rzeczy, które \underline{nie są nawet w połowie zrozumiane}}. «A Jezus podszedł i przemówił do nich, mówiąc: Dana mi jest wszelka władza w niebie i na ziemi. Idźcie więc i nauczajcie wszystkie narody, \textbf{chrzcząc je w imię Ojca i Syna, i Ducha Świętego}; ucząc je przestrzegać wszystkiego, co wam przykazałem. A oto ja jestem z wami zawsze, aż do końca świata» [Wersety 18--20]}[Lt214-1906.10; 1906][https://egwwritings.org/read?panels=p10171.16]

Istnieje powód, dla którego Ellen White wskazała Mt 28:19 jako fragment Pisma Świętego, który nie jest \egwinline{nawet w połowie zrozumiany}. To stwierdzenie zostało wypowiedziane w kontekście roku 1906, kiedy wielu pastorów i lekarzy propagowało doktrynę o Trójcy. Jak widzieliśmy, rozumienie Boga jako Trójcy nie było czymś, co Ellen White popierała, i z tego powodu sama nie odważyła się \egwinline{głosić teorii, które nie mają podstaw w Słowie Bożym}.

\egw{Wielki Nauczyciel trzymał w swojej ręce \textbf{całą mapę prawdy. W \underline{prostym} języku \underline{wyjaśnił} swoim uczniom} drogę do nieba i \textbf{niekończące się tematy boskiej mocy}. \textbf{Kwestia \underline{istoty Boga} była tematem, w którym zachował mądrą powściągliwość}, ponieważ ich zawiłości i specyfikacje wprowadziłyby naukę, która nie mogłaby być rozważana przez nieuświęcone umysły bez zamieszania. \textbf{W odniesieniu do Boga i w odniesieniu do Jego osobowości Pan Jezus powiedział}: «Tak długo jestem z wami, a nie poznałeś mnie, Filipie? Kto mnie widział, widział Ojca» [J 14:9]. \textbf{Chrystus był dokładnym obrazem osoby swojego Ojca}}[19LtMs, Ms 45, 1904, par. 15][https://egwwritings.org/read?panels=p14069.9381023&index=0]

\egwnogap{Otwarta ścieżka, bezpieczna ścieżka chodzenia drogą Jego przykazań, jest ścieżką, od której nie ma bezpiecznego odejścia. \textbf{A kiedy ludzie podążają za swoimi własnymi ludzkimi teoriami ubranymi w miękkie, czarujące wyobrażenia, tworzą sidło, w którą łapią dusze}. \textbf{\underline{Zamiast poświęcać swoje moce teoretyzowaniu}}, Chrystus dał wam pracę do wykonania. Jego zlecenie brzmi: Idźcie <na cały świat> i czyńcie uczniami wszystkie narody, \textbf{chrzcząc je w imię Ojca i Syna, i Ducha Świętego}. Zanim uczniowie przekroczą próg misji, musi być odcisk \textbf{świętego imienia, chrzczącego wierzących w \underline{imię trzech mocy} w niebiańskim świecie}. Umysł ludzki zostaje poruszony w trakcie tej ceremonii, początku chrześcijańskiego życia. To znaczy bardzo wiele. Dzieło zbawienia nie jest sprawą błahą, lecz tak doniosłą, że \textbf{najwyższe władze} zostają zaangażowane przez wyznanie wiary ludzkiej istoty. \textbf{Ojciec, Syn i Duch Święty — \underline{wieczne Bóstwo} — biorą udział w tym akcie, by dać człowiekowi pewność, że \underline{całe niebo} współdziała, by wspomóc ludzkie zdolności w osiągnięciu i objęciu pełni \underline{trzech mocy}, aby zjednoczyć się w wielkim wyznaczonym dziele, sprzymierzając niebiańskie moce z ludzkimi, aby ludzie mogli, z pomocą nieba, stać się uczestnikami boskiej natury i współpracownikami Chrystusa}}[19LtMs, Ms 45, 1904, akap. 16][https://egwwritings.org/read?panels=p14069.9381024&index=0]

Ten cytat jest kolejnym często błędnie interpretowanym stwierdzeniem. Często był używany, aby argumentować, że Ellen White opowiadała się za Trójcą, odnosząc się do Ojca, Syna i Ducha Świętego terminem \egwinline{wieczne Bóstwo}. Musimy jednak oddzielić kolejne warstwy jego kontekstu. Ellen White wyjaśniała znaczenie Mt 28:19. Stwierdziła: \egwinline{Zamiast poświęcać swoje moce teoretyzowaniu}, wypełnij zlecenie dane przez Chrystusa. Teoretyzowanie o czym? Teoretyzowanie o \egwinline{istocie Boga}. To kolejny „niezbity dowód” na doktrynę o Trójcy, zwłaszcza że odniosła się do \emcap{osobowości Boga}, stwierdzając: \egwinline{\textbf{W odniesieniu do Boga i w odniesieniu do Jego osobowości} Pan Jezus powiedział \normaltext{[...]} [J 14:9]. Chrystus był dokładnym obrazem osoby swojego \textbf{Ojca}}. J 14:9 nie oznacza tego, że widzenie Ojca w Chrystusie dowodzi, że są jedną i tą samą osobą, częścią jednego Boga. Przeciwnie — potwierdza, że Chrystus jest wyraźnym obrazem istoty Ojca. „Bóg”, do którego się odnosiła, to Ojciec. Istotnie, Jezus nauczał prawdy o tym, kim i czym jest Bóg. To właśnie \egwinline{wyjaśnił} \egwinline{w prostym języku}. Twierdzenie, że przez termin \egwinline{wieczne Bóstwo} Ellen White popierała Trójcę, byłoby sprzeczne z samą jej ostrożnością, którą wyraziła w kontekście tego fragmentu.

Niestety, desperackie pragnienie trynitarian, aby przedstawić Ellen White jako zwolenniczkę Trójcy, przyćmiło prawdziwe, natchnione znaczenie Mt 28:19. Jej przesłanie brzmiało: \egwinline{Zamiast poświęcać swoje moce teoretyzowaniu} o \egwinline{istocie Boga}, Chrystus dał nam zlecenie w Mt 28:19. I wyjaśniła znaczenie Mt 28:19. Chodziło jej o to: Ojciec, Syn i Duch Święty jednoczą wszystkie zasoby nieba z ludzkim wysiłkiem, aby poprzez boską moc ludzie mogli uczestniczyć w naturze Boga i pracować u boku Chrystusa. To jest znaczenie tego \egwinline{potrójnego imienia}. Kontynuowała wyjaśnianie:

\egw{\textbf{Możliwości człowieka mogą się zwielokrotnić poprzez połączenie ludzkich sił z boskimi}. \textbf{W połączeniu z niebiańskimi mocami} ludzkie zdolności wzrastają zgodnie z tą wiarą, która działa przez miłość i oczyszcza, uświęca i uszlachetnia całego człowieka. \textbf{\underline{Niebiańskie moce} \underline{zobowiązały się} do służenia ludzkim istotom, aby uczynić imię Boga i Chrystusa, i Ducha Świętego ich żyjącą skutecznością, działającą i zasilającą uświęconego człowieka, aby uczynić to imię ponad wszelkie inne imię}. \textbf{Wszystkie skarby niebios są zobowiązane uczynić dla człowieka} nieskończenie więcej, niż istoty ludzkie mogą pojąć, pomnażając trzykrotnie to, co ludzkie, z niebiańskimi siłami}[19LtMs, Ms 45, 1904, par. 17][https://egwwritings.org/read?panels=p14069.9381026&index=0]

\egwnogap{\textbf{\underline{Trzy wielkie i chwalebne niebiańskie postacie} są obecne przy uroczystości chrztu. Wszystkie ludzkie zdolności mają być odtąd poświęconymi siłami do pełnienia służby dla Boga w reprezentowaniu Ojca, Syna i Ducha Świętego, od których są zależne. \underline{Całe niebo jest reprezentowane przez tych trzech} w relacji przymierza z odnowionym życiem}. «Jeśli więc razem z Chrystusem powstaliście z martwych, szukajcie tego, co w górze, gdzie Chrystus zasiada \textbf{po prawicy Boga}» [Kol 3:1]}[19LtMs, Ms 45, 1904, par. 18][https://egwwritings.org/read?panels=p14069.9381027&index=0]

Wielu twierdzi, że Mt 28:19 nie jest natchniony, ponieważ został wstawiony przez Kościół katolicki\footnote{Uwaga: 1J 5:7, \bible{Trzej bowiem dają świadectwo w niebie: Ojciec, Słowo i Duch Święty, a ci trzej są jedno}, jest interpolacją znaną jako „\textit{Comma Johanneum}” (\textit{klauzula Janowa}). Ellen White nigdy nie używała tego wersetu. Nie było tak w przypadku Mt 28:19.}. Jednak tutaj mamy boskie natchnienie ujawniające jego prawdziwe znaczenie — znaczenie chrztu w potrójnym imieniu jako zobowiązanie złożone przez te \egwinline{trzy wielkie i chwalebne niebiańskie postacie}. Ich zobowiązanie polega na tym, że \egwinline{\textbf{wszystkie skarby niebios są zobowiązane uczynić dla człowieka} nieskończenie więcej, niż istoty ludzkie mogą pojąć, pomnażając trzykrotnie to, co ludzkie, z niebiańskimi siłami}.

Ellen White często cytowała Mt 28:19, wyjaśniając zobowiązanie Ojca, Syna i Ducha Świętego. To zobowiązanie służy jako wspaniała zachęta i obietnica podtrzymywana przez Niebo. Szczegółowe studium tego zobowiązania wykracza poza zakres tej książki, ponieważ nie dotyczy ono bezpośrednio obecności i \emcap{osobowości Boga}. Jednak zachęcamy do samodzielnego zgłębienia tego tematu. Kiedy zagłębisz się w jego znaczenie, zrozumiesz rzeczywistość służby aniołów.

Siostra White stwierdziła, że \egwinline{całe niebo jest reprezentowane przez tych trzech w relacji przymierza z nowym życiem}. Tymi trzema są Ojciec, Syn i Duch Święty. W innym przypadku powiedziała:

\egw{\textbf{Całe niebo interesuje się Twoim domem}. \textbf{Bóg i Chrystus, i \underline{niebiańscy aniołowie}} pragną gorąco, abyś tak wychowywał swoje dzieci, by były przygotowane do wejścia do rodziny odkupionych}[17LtMs, Ms 161, 1902, akap. 11][https://egwwritings.org/read?panels=p14067.9877018&index=0]

To nie jest sprzeczność. Całe niebo jest reprezentowane przez Ojca, Syna i Ducha Świętego, a w tym cytacie konkretnie wspomniała \egwinline{Boga i Chrystusa, i \textbf{niebiańskich aniołów}}. Istnieje ścisły związek między działaniem Ducha Świętego a służbą aniołów. Natchnienie świadczy:

\egw{Każdemu człowiekowi dana jest miara \textbf{Ducha} dla wspólnego pożytku. \textbf{Poprzez służbę aniołów \underline{Duch Święty może działać} na umysł i serce ludzkiej istoty} i pociągnąć ją do Chrystusa, który zapłacił okup za jej duszę, aby grzesznik mógł zostać uratowany od niewoli grzechu i szatana}[8LtMs, Lt 71, 1893, akap. 10][https://egwwritings.org/read?panels=p14058.6086016&index=0]

Ta anielska służba jest jednym z elementów chrzcielnego zobowiązania z Mt 28:19. Kiedy Ellen White powiedziała: \egwinline{\textbf{Niebiańskie moce} \textbf{zobowiązały się} do służenia ludzkim istotom \normaltext{[...]}}, odnosiła się do świętych aniołów. Związek między Duchem Świętym a świętymi aniołami wykracza poza zakres tej książki, ale możesz zgłębić ten temat w jej kontynuacji, \textit{Rediscovering the Pillar}\footnote{Pobierz za darmo: \href{https://forgottenpillar.com/book/rediscovering-the-pillar}{https://forgottenpillar.com/book/rediscovering-the-pillar}.}, w sekcji o Duchu Świętym\footnote{Zobacz również studium o aniołach \href{https://notefp.link/angels}{https://notefp.link/angels}.}.

\begin{titledpoem}

    \stanza{
        W potrójnym mamy chrzcić imieniu, \\
        Nie w boga trójcy tym złudzeniu. \\
        Ojciec, Syn, Duch są zjednoczeni, \\
        Nie w boga „trzech-w-jednym” złączeni.
    }

    \stanza{
        Tłumaczyć słów Ellen nie trzeba, \\
        Zjednuje chrzest przychylność nieba. \\
        Trzy moce będą nas prowadzić \\
        I wiernym w ich wędrówce radzić.
    }

    \stanza{
        Nie, w jednym trzech to nie dowodzi, \\
        Lecz że się obietnica rodzi. \\
        Ten gest to z niebem jest przymierze, \\
        Które człowiek zawiera w wierze.
    }

    \stanza{
        Ojciec — to Bóg w żywej osobie, \\
        A Syna — Księcia — ma po sobie, \\
        Duch — status ma przedstawiciela, \\
        Przez Niego Syn łaski udziela.
    }

\end{titledpoem}

