\chapter{Ellen White and Matthew 28:19}


\chapter{Ellen White i Mateusz 28:19}


Many assert that Ellen White promoted the Trinity doctrine, and that she is the one responsible for accepting it into our ranks. These claims do not consider that she defended the \emcap{personality of God} expressed in the first point of the \emcap{Fundamental Principles}. To support the claims that Ellen White was trinitarian, quotations are presented to her comment on Matthew 28:19:


Wielu twierdzi, że Ellen White promowała doktrynę o Trójcy i że to ona jest odpowiedzialna za przyjęcie jej w nasze szeregi. Twierdzenia te nie biorą pod uwagę, że broniła ona \emcap{osobowości Boga} wyrażonej w pierwszym punkcie \emcap{Fundamentalnych Zasad}. Na poparcie twierdzeń, że Ellen White była trynitariańska, przedstawiane są cytaty z jej komentarza do Mateusza 28:19:


\bible{Go ye therefore, and teach all nations, \textbf{baptizing them in the name of \underline{the Father}, and of \underline{the Son}, and of \underline{the Holy Ghost}}.}[Matthew 28:19]


\bible{Idźcie więc i nauczajcie wszystkie narody, \textbf{chrzcząc je w imię \underline{Ojca} i \underline{Syna}, i \underline{Ducha Świętego}}.}[Mateusz 28:19]


This verse has been most prominent in support of the Trinity doctrine. The Trinity doctrine has propositions about the \emcap{personality of God} of which this text says nothing to support. This verse itself does not teach that the Father, the Son, and the Holy Ghost, comprise \textit{one} God, the God of the Bible. There are other explicit verses in the Bible that exclude such interpretation of the text, i.e. 1 Corinthians 8:4-6; John 17:3; Ephesians 4:4-6; 1 Timothy 2:5.


Ten werset był najbardziej prominentny w poparciu doktryny o Trójcy. Doktryna o Trójcy zawiera twierdzenia o \emcap{osobowości Boga}, o których ten tekst nic nie mówi. Sam ten werset nie naucza, że Ojciec, Syn i Duch Święty tworzą \textit{jednego} Boga, Boga Biblii. Istnieją inne wyraźne wersety w Biblii, które wykluczają taką interpretację tekstu, tj. 1 Koryntian 8:4-6; Jana 17:3; Efezjan 4:4-6; 1 Tymoteusza 2:5.


Unfortunately, the same unsupported assumptions made about Matthew 28:19 are made about Sister White’s quotations dealing with this verse. For example, Sister White uses terms like \egwinline{three highest powers in heaven}[Lt253a-1903.18; 1903][https://egwwritings.org/?ref=en\_Lt253a-1903.18&para=10143.25], \egwinline{three great powers of heaven}[8T 254.1; 1904][https://egwwritings.org/?ref=en\_8T.254.1&para=112.1450], \egwinline{the three holy dignitaries of heaven}[Ms92-1901.26: 1901][https://egwwritings.org/?ref=en\_Ms92-1901.26&para=10732.32] and similar expressions—none of these quotations justify the assumption that these three (the Father, the Son, and the Holy Spirit) make \textit{one} God. On the contrary, as discussed in the previous chapter, keeping William Boardman’s sentiments and \egwinline{the heavenly trio} in context, “\textit{three-in-one}” sentiments \egwinline{should not be trusted}[Ms21-1906.8; 1906][https://egwwritings.org/?ref=en\_Ms21-1906.8&para=9754.15].


Niestety, te same nieuzasadnione założenia dotyczące Mateusza 28:19 są czynione w odniesieniu do cytatów Siostry White dotyczących tego wersetu. Na przykład, Siostra White używa terminów takich jak \egwinline{trzy najwyższe moce w niebie}[Lt253a-1903.18; 1903][https://egwwritings.org/?ref=en\_Lt253a-1903.18&para=10143.25], \egwinline{trzy wielkie moce nieba}[8T 254.1; 1904][https://egwwritings.org/?ref=en\_8T.254.1&para=112.1450], \egwinline{trzej święci dostojnicy nieba}[Ms92-1901.26: 1901][https://egwwritings.org/?ref=en\_Ms92-1901.26&para=10732.32] i podobne wyrażenia—żaden z tych cytatów nie uzasadnia założenia, że te trzy (Ojciec, Syn i Duch Święty) tworzą \textit{jednego} Boga. Przeciwnie, jak omówiono w poprzednim rozdziale, trzymając poglądy Williama Boardmana i \egwinline{niebiańskie trio} w kontekście, “\textit{trzech-w-jednym}” poglądy \egwinline{nie powinny być zaufane}[Ms21-1906.8; 1906][https://egwwritings.org/?ref=en\_Ms21-1906.8&para=9754.15].


The heavenly trio (the group of three: the Father, the Son and the Holy Spirit) are also present in other Bible verses, in addition to Matthew 28:19. There are several other instances in the New Testament where the Father, the Son and the Holy Spirit are mentioned, and these verses should be used to interpret the meaning behind the heavenly trio. None of the verses on the heavenly trio prove a three-in-one God; rather, all of them refer to the Father as one God. In the following verses, the heavenly trio is bolded in order to better distinguish the Father, the Son and the Holy Spirit.


Niebiańskie trio (grupa trzech: Ojciec, Syn i Duch Święty) są również obecne w innych wersetach biblijnych, oprócz Mateusza 28:19. Istnieje kilka innych przypadków w Nowym Testamencie, gdzie Ojciec, Syn i Duch Święty są wymienieni, i te wersety powinny być używane do interpretacji znaczenia niebiańskiego trio. Żaden z wersetów o niebiańskim trio nie dowodzi istnienia Boga trzech-w-jednym; raczej wszystkie odnoszą się do Ojca jako jedynego Boga. W następujących wersetach niebiańskie trio jest pogrubione, aby lepiej wyróżnić Ojca, Syna i Ducha Świętego.


\bible{There is one body, and \textbf{one Spirit}, even as ye are called in one hope of your calling; \textbf{One Lord}, one faith, one baptism, \textbf{One God and Father} of all, who is above all, and through all, and in you all.}[Ephesians 4:4-6]


\bible{Jedno jest ciało i \textbf{jeden Duch}, jak też zostaliście powołani w jednej nadziei waszego powołania; \textbf{Jeden Pan}, jedna wiara, jeden chrzest; \textbf{Jeden Bóg i Ojciec} wszystkich, który jest ponad wszystkimi, przez wszystkich i w was wszystkich.}[Efezjan 4:4-6]


\bible{Now there are diversities of gifts, but the \textbf{same Spirit}. And there are differences of administrations, but the \textbf{same Lord}. And there are diversities of operations, but it is \textbf{the same God} which worketh all in all.}[1 Corinthians 12:4-6]


\bible{A różne są dary, lecz \textbf{ten sam Duch}. Różne też są posługi, ale \textbf{ten sam Pan}. I różne są działania, lecz \textbf{ten sam jest Bóg}, który sprawia wszystko we wszystkich.}[1 Koryntian 12:4-6]


\bible{The grace of \textbf{the Lord Jesus Christ}, and the love of \textbf{God}, and the communion of \textbf{the Holy Ghost}, be with you all. Amen.}[2 Corinthians 13:14]


\bible{Łaska \textbf{Pana Jezusa Chrystusa} i miłość \textbf{Boga}, i społeczność \textbf{Ducha Świętego} niech będzie z wami wszystkimi. Amen.}[2 Koryntian 13:14]


\bible{For through \textbf{him} \normaltext{[Christ]} we both have access by one \textbf{Spirit} unto the \textbf{Father}.}[Ephesians 2:18]


\bible{Przez \textbf{niego} \normaltext{[Chrystusa]} bowiem my wszyscy w \textbf{jednym Duchu} mamy przystęp do \textbf{Ojca}.}[Efezjan 2:18]


\bible{But we are bound to give thanks alway to \textbf{God} for you, brethren beloved of \textbf{the Lord}, because \textbf{God} hath from the beginning chosen you to salvation through sanctification of \textbf{the Spirit} and belief of the truth.}[2 Thessalonians 2:13]


\bible{Ale my powinniśmy zawsze dziękować \textbf{Bogu} za was, bracia umiłowani przez \textbf{Pana}, że \textbf{Bóg} od początku wybrał was do zbawienia przez uświęcenie \textbf{Ducha} i wiarę w prawdę.}[2 Tesaloniczan 2:13]


\bible{How much more shall the blood of \textbf{Christ}, who through the eternal \textbf{Spirit} offered himself without spot to \textbf{God}, purge your conscience from dead works to serve \textbf{the living God}?}[Hebrews 9:14]


\bible{To o ile bardziej krew \textbf{Chrystusa}, który przez wiecznego \textbf{Ducha} ofiarował samego siebie bez skazy \textbf{Bogu}, oczyści wasze sumienie z martwych uczynków, by służyć \textbf{żywemu Bogu}?}[Hebrajczyków 9:14]


\bible{Elect according to the foreknowledge of \textbf{God the Father}, through sanctification of \textbf{the Spirit}, unto obedience and sprinkling of the blood of \textbf{Jesus Christ}: Grace unto you, and peace, be multiplied.}[1 Peter 1:2]


\bible{Wybranym według uprzedniej wiedzy \textbf{Boga Ojca}, przez uświęcenie \textbf{Ducha} dla posłuszeństwa i pokropienia krwią \textbf{Jezusa Chrystusa}. Łaska wam i pokój niech się pomnożą.}[1 Piotra 1:2]


All of the above verses talk about the heavenly trio (the Father, the Son and the Holy Spirit), and all of them consistently testify that the Father is the one referred to as God.
The same reasoning holds ground for Ellen White’s interpretation of Matthew 28:19.


Wszystkie powyższe wersety mówią o niebiańskiej trójcy (Ojcu, Synu i Duchu Świętym) i wszystkie one konsekwentnie świadczą, że Ojciec jest tym, który jest określany jako Bóg.
To samo rozumowanie ma zastosowanie w interpretacji Mateusza 28:19 przez Ellen White.


\egw{Christ gave His followers a positive promise that after His ascension He would send them His Spirit. ‘Go ye therefore,’ He said, ‘and teach all nations, baptizing them in the name of \textbf{the Father (a personal God),} and of \textbf{the Son (a personal Prince and Saviour),} and of \textbf{the Holy Ghost (sent from heaven to represent Christ);} teaching them to observe all things whatsoever I have commanded you, and, lo, I am with you alway, even unto the end of the world.’ Matthew 28:19, 20.}[RH October 26, 1897, par. 9; 1897][https://egwwritings.org/?ref=en\_RH.October.26.1897.par.9&para=821.16317]


\egw{Chrystus dał swoim naśladowcom stanowczą obietnicę, że po swoim wniebowstąpieniu pośle im swojego Ducha. „Idźcie więc”, powiedział, „i nauczajcie wszystkie narody, chrzcząc je w imię \textbf{Ojca (osobowego Boga),} i \textbf{Syna (osobowego Księcia i Zbawiciela),} i \textbf{Ducha Świętego (posłanego z nieba, aby reprezentować Chrystusa);} ucząc je przestrzegać wszystkiego, co wam przykazałem, a oto ja jestem z wami zawsze, aż do końca świata”. Mateusza 28:19, 20.}[RH October 26, 1897, par. 9; 1897][https://egwwritings.org/?ref=en\_RH.October.26.1897.par.9&para=821.16317]


The brackets in this quotation are in the original manuscript written by Ellen White. Here, she gives her own interpretation of Matthew 28:19. The Father is a personal God, the Son is a personal Prince and Saviour, and the Holy Spirit is Christ’s representative. This interpretation is in harmony with the \emcap{personality of God} expressed in the first point of the \emcap{Fundamental Principles}. Matthew 28:19 is a matter of interpretation. The interpretation which makes the Heavenly Trio one God is not inspired. This is not what the text indicates. Rather, let's read Matthew 28:19 within inspired compound: “\textit{Go ye therefore, and teach all nations, baptizing them in the name of a personal God, a personal Prince and Savior, and of the Holy Ghost}.” If one would read the text as such, no one would ever assume that one God is a unity of three persons. Therefore, let's stick to the inspiration, rather than subterfuge\footnote{\href{https://egwwritings.org/?ref=en\_Lt232-1903.41&para=10197.50}{{EGW, Lt232-1903.41; 1903}}}.


Nawiasy w tym cytacie znajdują się w oryginalnym rękopisie napisanym przez Ellen White. Tutaj podaje ona swoją własną interpretację Mateusza 28:19. Ojciec jest osobowym Bogiem, Syn jest osobowym Księciem i Zbawicielem, a Duch Święty jest przedstawicielem Chrystusa. Ta interpretacja jest zgodna z \emcap{osobowością Boga} wyrażoną w pierwszym punkcie \emcap{Fundamentalnych Zasad}. Mateusz 28:19 jest kwestią interpretacji. Interpretacja, która czyni Niebiańską Trójcę jednym Bogiem, nie jest natchniona. Nie to wskazuje tekst. Raczej przeczytajmy Mateusza 28:19 w natchnionym zestawieniu: “\textit{Idźcie więc i nauczajcie wszystkie narody, chrzcząc je w imię osobowego Boga, osobowego Księcia i Zbawiciela, i Ducha Świętego}.” Gdyby ktoś przeczytał tekst w ten sposób, nikt nigdy nie założyłby, że jeden Bóg jest jednością trzech osób. Dlatego trzymajmy się natchnienia, a nie wybiegów\footnote{\href{https://egwwritings.org/?ref=en\_Lt232-1903.41&para=10197.50}{{EGW, Lt232-1903.41; 1903}}}.


\egw{Let them be thankful to God for His manifold mercies and be kind to one another. \textbf{They have \underline{one God} and \underline{one Saviour}; and \underline{one Spirit}—\underline{the Spirit of Christ}—is to bring unity into their ranks}.}[9T 189.3; 1909][https://egwwritings.org/?ref=en\_9T.189.3&para=115.1057]


\egw{Niech będą wdzięczni Bogu za Jego rozliczne miłosierdzie i niech będą uprzejmi wobec siebie nawzajem. \textbf{Mają \underline{jednego Boga} i \underline{jednego Zbawiciela}; i \underline{jednego Ducha}—\underline{Ducha Chrystusa}—który ma wprowadzić jedność w ich szeregi}.}[9T 189.3; 1909][https://egwwritings.org/?ref=en\_9T.189.3&para=115.1057]


In light of the presented evidence, we see that simply numbering the Father, the Son and the Holy Spirit, does not prove the \textit{three-in-one} assumption, nor is it in conflict with the \emcap{personality of God} expressed in the \emcap{Fundamental Principles}. There is no denial of three persons of the Godhead, but only a denial of the assumption that these Three Great Worthies make one God.


W świetle przedstawionych dowodów widzimy, że samo wyliczenie Ojca, Syna i Ducha Świętego nie dowodzi założenia \textit{trzech-w-jednym}, ani nie jest w konflikcie z \emcap{osobowością Boga} wyrażoną w \emcap{Fundamentalnych Zasadach}. Nie ma zaprzeczenia trzech osób Bóstwa, ale tylko zaprzeczenie założenia, że te Trzy Wielkie Osobistości tworzą jednego Boga.


Matthew 28:19 is a valuable verse and it opens a new field of study within the Bible and the Spirit of Prophecy. In the context of the Living Temple, and referring to its sentiments, Sister White wrote that this verse should be studied most earnestly because it is not half understood.


Mateusz 28:19 jest cennym wersetem i otwiera nowe pole badań w Biblii i Duchu Proroctwa. W kontekście The Living Temple i odnosząc się do jego poglądów, Siostra White napisała, że ten werset powinien być studiowany najgorliwiej, ponieważ nie jest w połowie zrozumiany.


\egw{Just before His ascension, Christ gave His disciples a wonderful presentation, \textbf{as recorded in the twenty-eighth chapter of Matthew}. \textbf{This chapter contains instruction} that our ministers, our \textbf{physicians}, our youth, and all our church members need to \textbf{study most \underline{earnestly}}. \textbf{Those who study this instruction as they should will \underline{not dare to advocate theories that have no foundation in the Word of God}}. My brethren and sisters, make the Scriptures, which contain the alpha and omega of knowledge, your study. \textbf{All through the Old Testament and the New, there are things \underline{that are not half understood}}. ‘And Jesus came and spake unto them, saying, All power is given unto Me in heaven and in earth. Go ye therefore, and teach all nations, \textbf{baptizing them in the name of the Father, and of the Son, and of the Holy Ghost}; teaching them to observe all things whatsoever I have commanded you; and, lo, I am with you alway, even unto the end of the world.’ [Verses 18-20.]}[Lt214-1906.10; 1906][https://egwwritings.org/?ref=en\_Lt214-1906.10&para=10171.16]


\egw{Tuż przed swoim wniebowstąpieniem Chrystus dał swoim uczniom wspaniałą prezentację, \textbf{jak zapisano w dwudziestym ósmym rozdziale Ewangelii Mateusza}. \textbf{Ten rozdział zawiera instrukcje}, które nasi duchowni, nasi \textbf{lekarze}, nasza młodzież i wszyscy członkowie naszego kościoła muszą \textbf{studiować \underline{najgorliwiej}}. \textbf{Ci, którzy studiują te instrukcje tak, jak powinni, \underline{nie odważą się propagować teorii, które nie mają podstaw w Słowie Bożym}}. Moi bracia i siostry, uczyńcie Pismo Święte, które zawiera alfę i omegę wiedzy, przedmiotem waszych studiów. \textbf{W całym Starym i Nowym Testamencie są rzeczy, które \underline{nie są w połowie zrozumiane}}. „A Jezus podszedł i przemówił do nich, mówiąc: Dana mi jest wszelka władza w niebie i na ziemi. Idźcie więc i nauczajcie wszystkie narody, \textbf{chrzcząc je w imię Ojca i Syna, i Ducha Świętego}; ucząc je przestrzegać wszystkiego, co wam przykazałem. A oto ja jestem z wami zawsze, aż do końca świata”. [Wersety 18-20].}[Lt214-1906.10; 1906][https://egwwritings.org/?ref=en\_Lt214-1906.10&para=10171.16]


There is a reason why Ellen White pipointed to Matthew 28:19 as a Scripture which is \egwinline{not half understood.} This statement is made in the context of 1906, where many ministers, and physicians were advocating the trinity doctrine. As we have seen, the understanding of God as a trinity, was not something Ellen White supported, and for this reason, herself, she dared not \egwinline{to advocate theories that have no foundation in the Word of God.}


Istnieje powód, dla którego Ellen White wskazała na Mateusza 28:19 jako na fragment Pisma Świętego, który \egwinline{nie jest w połowie zrozumiany.} To stwierdzenie zostało wypowiedziane w kontekście roku 1906, kiedy wielu pastorów i lekarzy propagowało doktrynę o Trójcy. Jak widzieliśmy, rozumienie Boga jako Trójcy nie było czymś, co Ellen White popierała, i z tego powodu sama nie odważyła się \egwinline{głosić teorii, które nie mają podstaw w Słowie Bożym.}


\egw{The great Teacher held in His hand \textbf{the entire map of truth. In \underline{simple} language He \underline{made plain} to His disciples} the way to heaven and \textbf{the endless subjects of divine power}. \textbf{The question of \underline{the essence of God} was a subject on which He maintained a wise reserve}, for their entanglements and specifications would bring in science which could not be dwelt upon by unsanctified minds without confusion. \textbf{In regard to God and in regard to His personality, the Lord Jesus said}, ‘Have I been so long time with you, and yet hast thou not known Me, Philip? He that hath seen Me hath seen the Father.’ [John 14:9.] \textbf{Christ was the express image of His Father’s person}.}[19LtMs, Ms 45, 1904, par. 15][https://egwwritings.org/read?panels=p14069.9381023&index=0]


\egw{Wielki Nauczyciel trzymał w swojej ręce \textbf{całą mapę prawdy. W \underline{prostym} języku \underline{wyjaśnił} swoim uczniom} drogę do nieba i \textbf{nieskończone tematy boskiej mocy}. \textbf{Kwestia \underline{istoty Boga} była tematem, w którym zachował mądrą powściągliwość}, ponieważ ich uwikłania i specyfikacje wprowadziłyby naukę, nad którą nieuświęcone umysły nie mogłyby się zastanawiać bez zamieszania. \textbf{W odniesieniu do Boga i w odniesieniu do Jego osobowości, Pan Jezus powiedział}: ‘Tak długo jestem z wami, a nie poznałeś mnie, Filipie? Kto mnie widział, widział Ojca.’ [Jan 14:9.] \textbf{Chrystus był wyrazem istoty swojego Ojca}.}[19LtMs, Ms 45, 1904, par. 15][https://egwwritings.org/read?panels=p14069.9381023&index=0]


\egwnogap{The open path, the safe path of walking in the way of His commandments, is a path from which there is no safe departing. \textbf{And when men follow their own human theories dressed up in soft, fascinating representations, they make a snare in which to catch souls}. \textbf{\underline{In the place of devoting your powers to theorizing}}, Christ has given you a work to do. His commission is, Go <throughout the world> and make disciples of all nations, \textbf{baptizing them in the name of the Father, and of the Son, and of the Holy Ghost}. Before the disciples shall compass the threshold, there is to be the imprint of \textbf{the sacred name, baptizing the believers in \underline{the name of the threefold powers} in the heavenly world}. The human mind is impressed in this ceremony, the beginning of the Christian life. It means very much. The work of salvation is not a small matter, but so vast that \textbf{the highest authorities} are taken hold of by the expressed faith of the human agency. \textbf{The Father, the Son, and the Holy Ghost, \underline{the eternal Godhead} is involved in the action required to make assurance to the human agent to unite \underline{all heaven} to contribute to the exercise of human faculties to reach and embrace the fulness of \underline{the threefold powers} to unite in the great work appointed, confederating the heavenly powers with the human, that men may become, through heavenly efficiency, partakers of the divine nature and workers together with Christ}.}[19LtMs, Ms 45, 1904, par. 16][https://egwwritings.org/read?panels=p14069.9381024&index=0]


\egwnogap{Otwarta ścieżka, bezpieczna ścieżka chodzenia drogą Jego przykazań, jest ścieżką, od której nie ma bezpiecznego odejścia. \textbf{A kiedy ludzie podążają za swoimi własnymi ludzkimi teoriami ubranymi w miękkie, fascynujące przedstawienia, tworzą pułapkę, w którą łapią dusze}. \textbf{\underline{Zamiast poświęcać swoje moce teoretyzowaniu}}, Chrystus dał ci pracę do wykonania. Jego zlecenie brzmi: Idźcie <po całym świecie> i czyńcie uczniami wszystkie narody, \textbf{chrzcząc je w imię Ojca i Syna, i Ducha Świętego}. Zanim uczniowie przekroczą próg, ma być odciśnięte \textbf{święte imię, chrzcząc wierzących w \underline{imię trzech mocy} w niebiańskim świecie}. Ludzki umysł jest pod wrażeniem tej ceremonii, początku chrześcijańskiego życia. To znaczy bardzo wiele. Dzieło zbawienia nie jest małą sprawą, ale tak ogromną, że \textbf{najwyższe autorytety} są zaangażowane przez wyrażoną wiarę ludzkiego pośrednika. \textbf{Ojciec, Syn i Duch Święty, \underline{wieczna Boskość} jest zaangażowana w działanie wymagane, aby dać pewność ludzkiemu pośrednikowi, aby zjednoczyć \underline{całe niebo} w przyczynianiu się do ćwiczenia ludzkich zdolności, aby dosięgnąć i objąć pełnię \underline{trzech mocy}, aby zjednoczyć się w wielkiej wyznaczonej pracy, konfederując niebiańskie moce z ludzkimi, aby ludzie mogli stać się, poprzez niebiańską skuteczność, uczestnikami boskiej natury i współpracownikami z Chrystusem}.}[19LtMs, Ms 45, 1904, par. 16][https://egwwritings.org/read?panels=p14069.9381024&index=0]


This quotation is yet another often misrepresented statement. It has been often used to argue that Ellen White advocated for the Trinity by referencing the Father, the Son and the Holy Spirit by term \egwinline{eternal Godhead.} However, we must peel back the layers of its context. Ellen White was explaining the meaning behind Matthew 28:19. She stated: \egwinline{In the place of devoting your powers to theorizing,} fulfill the commission given by Christ. Theorizing about what? Theorizing about \egwinline{the essence of God.} This is another “smoking gun” for the Trinity doctrine, especially when she referenced the \emcap{personality of God} by stating: \egwinline{\textbf{In regard to God and in regard to His personality}, the Lord Jesus said…[John 14:9.] Christ was the express image of His \textbf{Father’s person}.} John 14:9 does not mean that seeing the Father in Christ implies they are one and the same person, all part of one God. Rather, it affirms that Christ is the express image of the Father’s person. The “God” she referred to was the Father. Indeed, Jesus taught the truth about who and what God is. This is what He \egwinline{made plain} \egwinline{in the simple language.} To claim that by the term \egwinline{eternal Godhead} Ellen White was endorsing the Trinity would contradict the very caution she expressed in the context of this passage.


Ten cytat jest kolejnym często błędnie interpretowanym stwierdzeniem. Często był używany, aby argumentować, że Ellen White opowiadała się za Trójcą, odnosząc się do Ojca, Syna i Ducha Świętego terminem \egwinline{wieczna Boskość.} Jednak musimy odkryć warstwy jego kontekstu. Ellen White wyjaśniała znaczenie Mateusza 28:19. Stwierdziła: \egwinline{Zamiast poświęcać swoje moce teoretyzowaniu,} wypełnij zlecenie dane przez Chrystusa. Teoretyzowanie o czym? Teoretyzowanie o \egwinline{istocie Boga.} To kolejny “dowód” na doktrynę o Trójcy, zwłaszcza gdy odniosła się do \emcap{osobowości Boga}, stwierdzając: \egwinline{\textbf{W odniesieniu do Boga i w odniesieniu do Jego osobowości}, Pan Jezus powiedział...[Jan 14:9.] Chrystus był wyrazem istoty swojego \textbf{Ojca}.} Jan 14:9 nie oznacza, że widzenie Ojca w Chrystusie implikuje, że są jedną i tą samą osobą, wszystko częścią jednego Boga. Raczej potwierdza, że Chrystus jest wyrazem istoty Ojca. “Bóg”, do którego się odnosiła, to Ojciec. Istotnie, Jezus nauczał prawdy o tym, kim i czym jest Bóg. To właśnie \egwinline{wyjaśnił} \egwinline{w prostym języku.} Twierdzenie, że przez termin \egwinline{wieczna Boskość} Ellen White popierała Trójcę, byłoby sprzeczne z samą ostrożnością, którą wyraziła w kontekście tego fragmentu.


Unfortunately, the desperate desire of Trinitarians to paint Ellen White as a Trinitarian advocate has overshadowed the true, inspired meaning of Matthew 28:19. Her message was: \egwinline{In the place of devoting your powers to theorizing} about \egwinline{the essence of God,} Christ has given us the commission in Matthew 28:19. And she explained the meaning of Matthew 28:19. Her point was: The Father, Son, and Holy Spirit unite all of heaven’s resources with human effort so that, through divine power, people may share in God’s nature and work alongside Christ. That is the meaning of this \egwinline{threefold name.} She continued explaining:


Niestety, desperackie pragnienie Trynitarzy, aby przedstawić Ellen White jako zwolenniczkę Trójcy, przyćmiło prawdziwe, natchnione znaczenie Mateusza 28:19. Jej przesłanie brzmiało: \egwinline{Zamiast poświęcać swoje moce teoretyzowaniu} o \egwinline{istocie Boga,} Chrystus dał nam zlecenie w Mateuszu 28:19. I wyjaśniła znaczenie Mateusza 28:19. Jej punkt był taki: Ojciec, Syn i Duch Święty jednoczą wszystkie zasoby nieba z ludzkim wysiłkiem, aby poprzez boską moc ludzie mogli uczestniczyć w naturze Boga i pracować u boku Chrystusa. To jest znaczenie tego \egwinline{potrójnego imienia.} Kontynuowała wyjaśnianie:


\egw{\textbf{Man’s capabilities can multiply through the connection of human agencies with divine agencies}. \textbf{United with the heavenly powers}, the human capabilities increase according to that faith that works by love and purifies, sanctifies, and ennobles the whole man. \textbf{\underline{The heavenly powers} have \underline{pledged themselves} to minister to human agents to make the name of God and of Christ and of the Holy Spirit their living efficiency, working and energizing the sanctified man, to make this name above every other name}. \textbf{All the treasures of heaven are under obligation to do for man} infinitely more than human beings can comprehend by multiplying threefold the human with the heavenly agencies.}[19LtMs, Ms 45, 1904, par. 17][https://egwwritings.org/read?panels=p14069.9381026&index=0]


\egw{\textbf{Ludzkie zdolności mogą się mnożyć poprzez połączenie ludzkich pośredników z boskimi pośrednikami}. \textbf{Zjednoczone z niebiańskimi mocami}, ludzkie zdolności wzrastają zgodnie z tą wiarą, która działa przez miłość i oczyszcza, uświęca i uszlachetnia całego człowieka. \textbf{\underline{Niebiańskie moce} \underline{zobowiązały się} do służenia ludzkim pośrednikom, aby uczynić imię Boga i Chrystusa, i Ducha Świętego ich żywą skutecznością, działającą i energetyzującą uświęconego człowieka, aby uczynić to imię ponad wszelkie inne imię}. \textbf{Wszystkie skarby nieba są zobowiązane do zrobienia dla człowieka} nieskończenie więcej, niż ludzie mogą pojąć, mnożąc trzykrotnie ludzkie z niebiańskimi pośrednikami.}[19LtMs, Ms 45, 1904, par. 17][https://egwwritings.org/read?panels=p14069.9381026&index=0]


\egwnogap{\textbf{\underline{The three great and glorious heavenly characters} are present on the occasion of baptism. All the human capabilities are to be henceforth consecrated powers to do service for God in representing the Father, the Son, and the Holy Ghost upon whom they depend. \underline{All heaven is represented by these three} in covenant relation with the new life}. ‘If ye then be risen with Christ, seek those things that are above, where Christ sitteth at \textbf{the right hand of God}.’ [Colossians 3:1.]}[19LtMs, Ms 45, 1904, par. 18][https://egwwritings.org/read?panels=p14069.9381027&index=0]


\egwnogap{\textbf{\underline{Trzy wielkie i chwalebne niebiańskie postacie} są obecne przy okazji chrztu. Wszystkie ludzkie zdolności mają być odtąd uświęconymi mocami do służby Bogu w reprezentowaniu Ojca, Syna i Ducha Świętego, od których są zależne. \underline{Całe niebo jest reprezentowane przez tych trzech} w przymierzu z nowym życiem}. ‘Jeśli więc razem z Chrystusem powstaliście z martwych, szukajcie tego, co w górze, gdzie Chrystus zasiada \textbf{po prawicy Boga}.’ [Kolosan 3:1.]}[19LtMs, Ms 45, 1904, par. 18][https://egwwritings.org/read?panels=p14069.9381027&index=0]


Many claim that Matthew 28:19 is uninspired because it was inserted by the Catholic Church\footnote{Note, 1 John 5:7 \bible{For there are three that bear record in heaven, the Father, the Word, and the Holy Ghost: and these three are one.} is an interpolation known as “\textit{Johannine Comma}”. Ellen White never used that verse. This was not the case with Matthew 28:19.}. Yet, here we have divine inspiration revealing its true meaning—the significance of baptism in the threefold name as a pledge made by these \egwinline{three great and glorious heavenly characters.} Their pledge is that \egwinline{\textbf{all the treasures of heaven are under obligation to do for man} infinitely more than human beings can comprehend by multiplying threefold the human with the heavenly agencies.}


Wielu twierdzi, że Mateusz 28:19 nie jest natchniony, ponieważ został wstawiony przez Kościół Katolicki\footnote{Uwaga, 1 Jana 5:7 \bible{Trzej bowiem świadczą w niebie: Ojciec, Słowo i Duch Święty, a ci trzej jedno są.} jest interpolacją znaną jako “\textit{Comma Johanneum}”. Ellen White nigdy nie używała tego wersetu. Nie był to przypadek z Mateuszem 28:19.}. Jednak tutaj mamy boskie natchnienie ujawniające jego prawdziwe znaczenie - znaczenie chrztu w potrójnym imieniu jako zobowiązanie złożone przez te \egwinline{trzy wielkie i chwalebne niebiańskie postacie.} Ich zobowiązanie polega na tym, że \egwinline{\textbf{wszystkie skarby nieba są zobowiązane do zrobienia dla człowieka} nieskończenie więcej, niż ludzie mogą pojąć, mnożąc trzykrotnie ludzkie z niebiańskimi pośrednikami.}


Ellen White frequently quoted Matthew 28:19, explaining the pledge of the Father, the Son, and the Holy Spirit. This pledge serves as a wonderful encouragement and a promise upheld by Heaven. A detailed study of this pledge is beyond the scope of this book, as it does not directly address the presence and \emcap{personality of God}. However, we encourage you to explore this topic for yourself. When you delve deeper into its meaning, you will come to understand the reality of the ministry of angels.


Ellen White często cytowała Mateusza 28:19, wyjaśniając zobowiązanie Ojca, Syna i Ducha Świętego. To zobowiązanie służy jako wspaniała zachęta i obietnica podtrzymywana przez Niebo. Szczegółowe studium tego zobowiązania wykracza poza zakres tej książki, ponieważ nie odnosi się bezpośrednio do obecności i \emcap{osobowości Boga}. Jednak zachęcamy do samodzielnego zbadania tego tematu. Kiedy zagłębisz się w jego znaczenie, zrozumiesz rzeczywistość służby aniołów.


Sister White stated that \egwinline{all heaven is represented by these three in covenant relation with the new life.} These three are the Father, the Son, and the Holy Spirit. In another instance, she said:


Siostra White stwierdziła, że \egwinline{całe niebo jest reprezentowane przez tych trzech w przymierzu z nowym życiem.} Tymi trzema są Ojciec, Syn i Duch Święty. W innym przypadku powiedziała:


\egw{\textbf{All heaven is interested in your home}. \textbf{God and Christ and \underline{the heavenly angels}} are intensely desirous that you shall so train your children that they will be prepared to enter the family of the redeemed.}[17LtMs, Ms 161, 1902, par. 11][https://egwwritings.org/read?panels=p14067.9877018&index=0]


\egw{\textbf{Całe niebo interesuje się twoim domem}. \textbf{Bóg i Chrystus i \underline{niebiańscy aniołowie}} pragną gorąco, abyś tak wychowywał swoje dzieci, by były przygotowane do wejścia do rodziny odkupionych.}[17LtMs, Ms 161, 1902, par. 11][https://egwwritings.org/read?panels=p14067.9877018&index=0]


This is not a contradiction. All of heaven is represented by the Father, the Son, and the Holy Spirit, and in this quote, she specifically mentioned \egwinline{God and Christ and \textbf{the heavenly angels}.} There is a close connection between the workings of the Holy Spirit and the ministry of angels. The Inspiration testifies:


To nie jest sprzeczność. Całe niebo jest reprezentowane przez Ojca, Syna i Ducha Świętego, a w tym cytacie konkretnie wspomniała \egwinline{Bóg i Chrystus i \textbf{niebiańscy aniołowie}.} Istnieje ścisły związek między działaniem Ducha Świętego a służbą aniołów. Natchnienie świadczy:


\egw{A measure of \textbf{the Spirit} is given to every man to profit withal. \textbf{Through the ministry of the angels \underline{the Holy Spirit is enabled} to work upon the mind and heart of the human agent}, and draw him to Christ who has paid the ransom money for his soul, that the sinner may be rescued from the slavery of sin and Satan.}[8LtMs, Lt 71, 1893, par. 10][https://egwwritings.org/read?panels=p14058.6086016&index=0]


\egw{Każdemu człowiekowi dana jest miara \textbf{Ducha} dla wspólnego pożytku. \textbf{Poprzez służbę aniołów \underline{Duch Święty może działać} na umysł i serce ludzkiego pośrednika}, i pociągnąć go do Chrystusa, który zapłacił okup za jego duszę, aby grzesznik mógł zostać uwolniony z niewoli grzechu i Szatana.}[8LtMs, Lt 71, 1893, par. 10][https://egwwritings.org/read?panels=p14058.6086016&index=0]


This angelic ministry is one of the elements in the baptismal pledge of Matthew 28:19. When Ellen White said, \egwinline{\textbf{The heavenly powers} have \textbf{pledged themselves} to minister to human agents…,} she was referring to the holy angels. The connection between the Holy Spirit and the holy angels is beyond the scope of this book, but you can explore this topic further in the sequel, \textit{Rediscovering the Pillar}\footnote{Download for free: \href{https://forgottenpillar.com/book/rediscovering-the-pillar}{https://forgottenpillar.com/book/rediscovering-the-pillar}}, in the section on the Holy Spirit\footnote{Also, see the study on the angels \href{https://notefp.link/angels}{https://notefp.link/angels}}.


Ta anielska służba jest jednym z elementów przyrzeczenia chrzcielnego z Mateusza 28:19. Kiedy Ellen White powiedziała, \egwinline{\textbf{Niebiańskie moce} \textbf{zobowiązały się} do służenia ludzkim pośrednikom...}, odnosiła się do świętych aniołów. Związek między Duchem Świętym a świętymi aniołami wykracza poza zakres tej książki, ale możesz zgłębić ten temat w kontynuacji, \textit{Rediscovering the Pillar}\footnote{Pobierz za darmo: \href{https://forgottenpillar.com/book/rediscovering-the-pillar}{https://forgottenpillar.com/book/rediscovering-the-pillar}}, w sekcji o Duchu Świętym\footnote{Zobacz również studium o aniołach \href{https://notefp.link/angels}{https://notefp.link/angels}}.