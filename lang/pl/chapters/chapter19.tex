\chapter{Ellen White i Ewangelia Mateusza 28:19}

Wielu twierdzi, że Ellen White promowała doktrynę o Trójcy i że to ona jest odpowiedzialna za przyjęcie jej do naszych szeregów. Twierdzenia te nie uwzględniają faktu, że broniła ona \emcap{osobowości Boga} przedstawionej w pierwszym punkcie \emcap{Fundamentalnych Zasad}. Na poparcie twierdzeń, że Ellen White była trynitarna, przedstawiane są cytaty z jej komentarza do Mateusza 28:19:

\bible{Idźcie więc i nauczajcie wszystkie narody, \textbf{chrzcząc je w imię \underline{Ojca} i \underline{Syna}, i \underline{Ducha Świętego}}.}[Mateusz 28:19]

Ten werset był najbardziej znanym w poparciu doktryny o Trójcy. Doktryna o Trójcy zawiera stwierdzenia o \emcap{osobowości Boga}, których ten tekst w żaden sposób nie popiera. Sam ten werset nie naucza, że Ojciec, Syn i Duch Święty tworzą \textit{jednego} Boga, Boga Biblii. Istnieją inne wyraźne wersety w Biblii, które wykluczają taką interpretację tekstu, tj. 1 Koryntian 8:4-6; Jana 17:3; Efezjan 4:4-6; 1 Tymoteusza 2:5.

Niestety, te same nieuzasadnione założenia dotyczące Mateusza 28:19 są czynione w odniesieniu do cytatów Siostry White dotyczących tego wersetu. Na przykład, Siostra White używa terminów takich jak \egwinline{trzy najwyższe moce w niebie}[Lt253a-1903.18; 1903][https://egwwritings.org/?ref=en\_Lt253a-1903.18&para=10143.25], \egwinline{trzy wielkie moce nieba}[8T 254.1; 1904][https://egwwritings.org/?ref=en\_8T.254.1&para=112.1450], \egwinline{trzej święci dostojnicy nieba}[Ms92-1901.26: 1901][https://egwwritings.org/?ref=en\_Ms92-1901.26&para=10732.32] i podobne zwroty—jednak żaden z tych cytatów nie uzasadnia założenia, że te trzy (Ojciec, Syn i Duch Święty) tworzą \textit{jednego} Boga. Wręcz przeciwnie, jak omówiono w poprzednim rozdziale, zachowując poglądy Williama Boardmana  \egwinline{niebiańskie trio} w kontekście typu “\textit{trzy-w-jednym}” poglądy te \egwinline{nie powinny być godne zaufania} [Ms21-1906.8; 1906][https://egwwritings.org/?ref=en\_Ms21-1906.8&para=9754.15].

Niebiańskie trio (grupa trzech: Ojciec, Syn i Duch Święty) są również obecne w innych wersetach biblijnych, oprócz Mateusza 28:19. Istnieje kilka innych przykładów w Nowym Testamencie, gdzie Ojciec, Syn i Duch Święty są wymienieni, i  wersety te powinny być używane do interpretacji ukrytych w nim treści odnośnie niebiańskiego trio. Żaden z wersetów o niebiańskim trio nie dowodzi istnienia Boga trzech-w-jednym; raczej wszystkie odnoszą się do Ojca jako jedynego Boga. W następujących wersetach niebiańskie trio jest pogrubione, aby lepiej wyróżnić Ojca, Syna i Ducha Świętego.

\bible{Jedno jest ciało i \textbf{jeden Duch}, jak też zostaliście powołani w jednej nadziei waszego powołania; \textbf{Jeden Pan}, jedna wiara, jeden chrzest; \textbf{Jeden Bóg i Ojciec} wszystkich, który jest ponad wszystkimi, przez wszystkich i w was wszystkich.}[Efezjan 4:4-6]

\bible{A różne są dary, lecz \textbf{ten sam Duch}. Różne też są posługi, ale \textbf{ten sam Pan}. I różne są działania, lecz \textbf{ten sam jest Bóg}, który sprawia wszystko we wszystkich.}[1 Koryntian 12:4-6]

\bible{Łaska \textbf{Pana Jezusa Chrystusa} i miłość \textbf{Boga}, i społeczność \textbf{Ducha Świętego} niech będzie z wami wszystkimi. Amen.}[2 Koryntian 13:14]

\bible{Przez \textbf{niego} \normaltext{[Chrystusa]} bowiem my wszyscy w \textbf{jednym Duchu} mamy przystęp do \textbf{Ojca}.}[Efezjan 2:18]

\bible{Ale my powinniśmy zawsze dziękować \textbf{Bogu} za was, bracia umiłowani przez \textbf{Pana}, że \textbf{Bóg} od początku wybrał was do zbawienia przez uświęcenie \textbf{Ducha} i wiarę w prawdę.}[2 Tesaloniczan 2:13]

\bible{To o ile bardziej krew \textbf{Chrystusa}, który przez wiecznego \textbf{Ducha} ofiarował samego siebie bez skazy \textbf{Bogu}, oczyści wasze sumienie z martwych uczynków, by służyć \textbf{żywemu Bogu}?}[Hebrajczyków 9:14]

\bible{Wybranym według uprzedniej wiedzy \textbf{Boga Ojca}, przez uświęcenie \textbf{Ducha} dla posłuszeństwa i pokropienia krwią \textbf{Jezusa Chrystusa}. Łaska wam i pokój niech się pomnożą.}[1 Piotra 1:2]

Wszystkie powyższe wersety mówią o niebiańskim trio (Ojcu, Synu i Duchu Świętym) i wszystkie one konsekwentnie świadczą, że Ojciec jest tym, który jest określany jako Bóg.
To samo rozumowanie ma zastosowanie w interpretacji Mateusza 28:19 przez Ellen White.

\egw{Chrystus dał swoim naśladowcom wyraźną obietnicę, że po swoim wniebowstąpieniu pośle im swojego Ducha. „Idźcie więc”, powiedział, „i nauczajcie wszystkie narody, chrzcząc je w imię \textbf{Ojca (osobowego Boga),} i \textbf{Syna (osobowego Księcia i Zbawiciela),} i \textbf{Ducha Świętego (posłanego z nieba, aby reprezentować Chrystusa);} ucząc je przestrzegać wszystkiego, co wam przykazałem, a oto ja jestem z wami zawsze, aż do końca świata”. Mateusza 28:19, 20.}[RH October 26, 1897, par. 9; 1897][https://egwwritings.org/?ref=en\_RH.October.26.1897.par.9&para=821.16317]

Nawiasy w tym cytacie znajdują się w oryginalnym rękopisie napisanym przez Ellen White. Tutaj podaje ona swoją własną interpretację Mateusza 28:19. Ojciec jest osobowym Bogiem, Syn jest osobowym Księciem i Zbawicielem, a Duch Święty jest przedstawicielem Chrystusa. Ta interpretacja jest zgodna z \emcap{osobowością Boga} wyrażoną w pierwszym punkcie \emcap{Fundamentalnych Zasad}. Mateusz 28:19 jest kwestią interpretacji. Interpretacja, która czyni Niebiańskie Trio jednym Bogiem, nie jest natchniona. Nie to wskazuje tekst. Raczej przeczytajmy Mateusza 28:19 w natchnionym zestawieniu: “\textit{Idźcie więc i nauczajcie wszystkie narody, chrzcząc je w imię osobowego Boga, osobowego Księcia i Zbawiciela, i Ducha Świętego}.” Gdyby ktoś przeczytał tekst w ten sposób, nikt nigdy nie założyłby, że jeden Bóg jest jednością trzech osób. Dlatego trzymajmy się natchnienia, a nie matactw\footnote{\href{https://egwwritings.org/?ref=en\_Lt232-1903.41&para=10197.50}{{EGW, Lt232-1903.41; 1903}}}.

\egw{Niech będą wdzięczni Bogu za Jego obfite miłosierdzie i niech będą uprzejmi wobec siebie nawzajem. \textbf{Mają \underline{jednego Boga} i \underline{jednego Zbawiciela}; i \underline{jednego Ducha}—\underline{Ducha Chrystusa}—który ma wprowadzić jedność w ich szeregi}.}[9T 189.3; 1909][https://egwwritings.org/?ref=en\_9T.189.3&para=115.1057]

W świetle przedstawionych dowodów widzimy, że samo wyliczenie Ojca, Syna i Ducha Świętego nie dowodzi założenia \textit{trzech-w-jednym}, ani nie jest w konflikcie z \emcap{osobowością Boga} wyrażoną w \emcap{Fundamentalnych Zasadach}. Nie ma zaprzeczenia trzech osób Bóstwa, ale jedynie zaprzeczenie założeniu, że te Trzy Wielkie Osobistości tworzą jednego Boga.

Mateusz 28:19 jest cennym wersetem i otwiera nowe pole badań w Biblii i Duchu Proroctwa. W kontekście The Living Temple jak i odnosząc się do jego poglądów, Siostra White napisała, że ten werset powinien być studiowany z największą gorliwością, ponieważ nie jest on choć w  połowie zrozumiany.

\egw{Tuż przed swoim wniebowstąpieniem Chrystus dał swoim uczniom wspaniałą instrukcję, \textbf{jak zapisano w dwudziestym ósmym rozdziale Ewangelii Mateusza}. \textbf{Rozdział ten zawiera pouczenie}, które nasi duchowni, nasi \textbf{lekarze}, nasza młodzież i wszyscy członkowie naszego kościoła muszą \textbf{studiować \underline{pilnie z gorliwością}}. \textbf{Ci, którzy zapoznają sie z tymi  instrukcjami tak, jak powinni, \underline{nie odważą się propagować teorii, które nie mają podstaw w Słowie Bożym}}. Moi bracia i siostry, uczyńcie Pismo Święte, które zawiera alfę i omegę wiedzy, przedmiotem waszych studiów. \textbf{W całym Starym i Nowym Testamencie są rzeczy, które \underline{nie są w połowie zrozumiane}}. „A Jezus podszedł i przemówił do nich, mówiąc: Dana mi jest wszelka władza w niebie i na ziemi. Idźcie więc i nauczajcie wszystkie narody, \textbf{chrzcząc je w imię Ojca i Syna, i Ducha Świętego}; ucząc je przestrzegać wszystkiego, co wam przykazałem. A oto ja jestem z wami zawsze, aż do końca świata”. [Wersety 18-20].}[Lt214-1906.10; 1906][https://egwwritings.org/?ref=en\_Lt214-1906.10&para=10171.16]

Istnieje powód, dla którego Ellen White wskazała na Mateusza 28:19 jako na fragment Pisma Świętego, który \egwinline{nie jest w połowie zrozumiany.} To stwierdzenie zostało wypowiedziane w kontekście roku 1906, kiedy wielu pastorów i lekarzy propagowało doktrynę o Trójcy. Jak widzieliśmy, rozumienie Boga jako Trójcy nie było czymś, co Ellen White popierała, i z tego powodu sama nie odważyła się \egwinline{głosić teorii, które nie mają podstaw w Słowie Bożym.}

\egw{Wielki Nauczyciel trzymał w swojej ręce \textbf{całą mapę prawdy. W \underline{prostym} języku \underline{wyjaśnił} swoim uczniom} drogę do nieba i \textbf{niekończące się tematy boskiej mocy}. \textbf{Kwestia \underline{istoty Boga} była tematem, w którym zachował mądrą powściągliwość}, ponieważ ich zawiłości i specyfikacje wprowadziłyby naukę, która nie mogłaby być rozważana przez nieuświęcone umysły  bez zamieszania. \textbf{W odniesieniu do Boga i w odniesieniu do Jego osobowości, Pan Jezus powiedział}: ‘Tak długo jestem z wami, a nie poznałeś mnie, Filipie? Kto mnie widział, widział Ojca.’ [Jan 14:9.] \textbf{Chrystus był wyrazem istoty swojego Ojca}.}[19LtMs, Ms 45, 1904, par. 15][https://egwwritings.org/read?panels=p14069.9381023&index=0]


\egwnogap{The open path, the safe path of walking in the way of His commandments, is a path from which there is no safe departing. \textbf{And when men follow their own human theories dressed up in soft, fascinating representations, they make a snare in which to catch souls}. \textbf{\underline{In the place of devoting your powers to theorizing}}, Christ has given you a work to do. His commission is, Go <throughout the world> and make disciples of all nations, \textbf{baptizing them in the name of the Father, and of the Son, and of the Holy Ghost}. Before the disciples shall compass the threshold, there is to be the imprint of \textbf{the sacred name, baptizing the believers in \underline{the name of the threefold powers} in the heavenly world}. The human mind is impressed in this ceremony, the beginning of the Christian life. It means very much. The work of salvation is not a small matter, but so vast that \textbf{the highest authorities} are taken hold of by the expressed faith of the human agency. \textbf{The Father, the Son, and the Holy Ghost, \underline{the eternal Godhead} is involved in the action required to make assurance to the human agent to unite \underline{all heaven} to contribute to the exercise of human faculties to reach and embrace the fulness of \underline{the threefold powers} to unite in the great work appointed, confederating the heavenly powers with the human, that men may become, through heavenly efficiency, partakers of the divine nature and workers together with Christ}.}[19LtMs, Ms 45, 1904, par. 16][https://egwwritings.org/read?panels=p14069.9381024&index=0]


\egwnogap{Otwarta ścieżka, bezpieczna ścieżka chodzenia drogą Jego przykazań, jest ścieżką, od której nie ma bezpiecznego odejścia. \textbf{A kiedy ludzie podążają za swoimi własnymi ludzkimi teoriami ubranymi w miękkie, fascynujące przedstawienia, tworzą pułapkę, w którą łapią dusze}. \textbf{\underline{Zamiast poświęcać swoje moce teoretyzowaniu}}, Chrystus dał ci pracę do wykonania. Jego zlecenie brzmi: Idźcie <na cały świat> i czyńcie uczniami wszystkie narody, \textbf{chrzcząc je w imię Ojca i Syna, i Ducha Świętego}. Zanim uczniowie przekroczą próg, ma być odciśnięte \textbf{święte imię, chrzcząc wierzących w \underline{imię trzech mocy} w niebiańskim świecie}. Ludzki umysł jest pod wrażeniem tej ceremonii, początku chrześcijańskiego życia. To znaczy bardzo wiele. Dzieło zbawienia nie jest małą sprawą, ale tak ogromną, że \textbf{najwyższe autorytety} są zaangażowane przez wyrażoną wiarę ludzkiego pośrednika. \textbf{Ojciec, Syn i Duch Święty, \underline{wieczna Boskość} jest zaangażowana w działanie wymagane, aby dać pewność ludzkiemu pośrednikowi, aby zjednoczyć \underline{całe niebo} w przyczynianiu się do ćwiczenia ludzkich zdolności, aby dosięgnąć i objąć pełnię \underline{trzech mocy}, aby zjednoczyć się w wielkiej wyznaczonej pracy, konfederując niebiańskie moce z ludzkimi, aby ludzie mogli stać się, poprzez niebiańską skuteczność, uczestnikami boskiej natury i współpracownikami z Chrystusem}.}[19LtMs, Ms 45, 1904, par. 16][https://egwwritings.org/read?panels=p14069.9381024&index=0]

Ten cytat jest kolejnym często błędnie interpretowanym stwierdzeniem. Często był używany, aby argumentować, że Ellen White opowiadała się za Trójcą, odnosząc się do Ojca, Syna i Ducha Świętego terminem \egwinline{wieczna Boskość.} Musimy jednak oddzielić kolejne warstwy jego kontekstu. Ellen White wyjaśniała znaczenie Mateusza 28:19. Stwierdziła: \egwinline{Zamiast poświęcać swoje siły na teoretyzowaniu,} wypełnij zlecenie dane przez Chrystusa. Teoretyzowanie o czym? Teoretyzowanie o \egwinline{istocie Boga.} To kolejny “dowód” na doktrynę o Trójcy, zwłaszcza gdy odniosła się do \emcap{osobowości Boga}, stwierdzając: \egwinline{\textbf{W odniesieniu do Boga i w odniesieniu do Jego osobowości}, Pan Jezus powiedział...[Jan 14:9.] Chrystus był wyrazem istoty swojego \textbf{Ojca}.} Jan 14:9 nie oznacza to, że widzenie Ojca w Chrystusie sugeruje, że są jedną i tą samą osobą, częścią jednego Boga. Raczej potwierdza, że Chrystus jest wyraźnym obrazem istoty Ojca. “Bóg”, do którego się odnosiła, to Ojciec. Istotnie, Jezus nauczał prawdy o tym, kim i czym jest Bóg. To właśnie \egwinline{wyjaśnił} \egwinline{w prostym języku.} Twierdzenie, że przez termin \egwinline{wieczna Boskość} Ellen White popierała Trójcę, byłoby sprzeczne z samą jej ostrożnością, którą wyraziła w kontekście tego fragmentu.

Niestety, desperackie pragnienie Trynitarzy, aby przedstawić Ellen White jako zwolenniczkę Trójcy, przyćmiło prawdziwe, natchnione znaczenie Mateusza 28:19. Jej przesłanie brzmiało: \egwinline{Zamiast poświęcać swoje siły teoretyzowaniu} o \egwinline{istocie Boga,} Chrystus dał nam zlecenie w Mateuszu 28:19. I wyjaśniła znaczenie Mateusza 28:19. Jej punkt był taki: Ojciec, Syn i Duch Święty jednoczą wszystkie zasoby nieba z ludzkim wysiłkiem, aby poprzez boską moc ludzie mogli uczestniczyć w naturze Boga i pracować u boku Chrystusa. To jest znaczenie tego \egwinline{potrójnego imienia.} Kontynuowała wyjaśnianie:

\egw{\textbf{Możliwości człowieka mogą się zwielokrotnić poprzez połączenie ludzkich pośredników z boskimi pośrednikami}. \textbf{W połączeniu z niebiańskimi mocami}, ludzkie zdolności wzrastają zgodnie z tą wiarą, która działa przez miłość i oczyszcza, uświęca i uszlachetnia całego człowieka. \textbf{\underline{Niebiańskie moce} \underline{zobowiązały się} do służenia ludzkim pośrednikom, aby uczynić imię Boga i Chrystusa, i Ducha Świętego ich żywą skutecznością, działającą i zasilającą uświęconego człowieka, aby uczynić to imię ponad wszelkie inne imię}. \textbf{Wszystkie skarby niebios są zobowiązane uczynić dla człowieka} nieskończenie więcej, niż istoty ludzkie mogą pojąć, pomnażając trzykrotnie to co ludzkie z niebiańskimi pośrednikami.}[19LtMs, Ms 45, 1904, par. 17][https://egwwritings.org/read?panels=p14069.9381026&index=0]

\egwnogap{\textbf{\underline{Trzy wielkie i chwalebne niebiańskie postacie} są obecne przy uroczystości chrztu. Wszystkie ludzkie zdolności mają być odtąd poświęconymi siłami do pełnienia służby dla Boga  reprezentując Ojca, Syna i Ducha Świętego, od których są zależne. \underline{Całe niebo jest reprezentowane przez tych trzech} w relacji przymierza z odnowionym życiem}. ‘Jeśli więc razem z Chrystusem powstaliście z martwych, szukajcie tego, co w górze, gdzie Chrystus zasiada \textbf{po prawicy Boga}.’ [Kolosan 3:1.]}[19LtMs, Ms 45, 1904, par. 18][https://egwwritings.org/read?panels=p14069.9381027&index=0]

Wielu twierdzi, że Mateusz 28:19 nie jest natchniony, ponieważ został wstawiony przez Kościół Katolicki\footnote{Uwaga, 1 Jana 5:7 \bible{Trzej bowiem świadczą w niebie: Ojciec, Słowo i Duch Święty, a ci trzej jedno są.} jest interpolacją znaną jako “\textit{Comma Johanneum}”. Ellen White nigdy nie używała tego wersetu. Nie był to przypadek z Mateuszem 28:19.}. Jednak tutaj mamy boskie natchnienie ujawniające jego prawdziwe znaczenie - znaczenie chrztu w potrójnym imieniu jako zobowiązanie złożone przez te \egwinline{trzy wielkie i chwalebne niebiańskie postacie.} Ich zobowiązanie polega na tym, że \egwinline{\textbf{wszystkie skarby niebios są zobowiązane uczynić dla człowieka} nieskończenie więcej, niż istoty ludzkie mogą pojąć, przez  trzykrotne pomnożenie tego co ludzkie z niebiańskimi pośrednikami.}

Ellen White często cytowała Mateusza 28:19, wyjaśniając przyrzeczenie Ojca, Syna i Ducha Świętego. To przyrzeczenie służy jako wspaniała zachęta i obietnica podtrzymywana przez Niebo. Szczegółowe studium tego przyrzeczenia wykracza poza zakres tej książki, ponieważ nie dotyczy się ono bezpośrednio do obecności i \emcap{osobowości Boga}. Jednak zachęcamy do samodzielnego zgłębienia tego tematu. Kiedy zagłębisz się w jego znaczenie, zrozumiesz rzeczywistość służby aniołów.

Siostra White stwierdziła, że \egwinline{całe niebo jest reprezentowane przez tych trzech w przymierzu z nowym życiem.} Tymi trzema są Ojciec, Syn i Duch Święty. W innym przypadku powiedziała:

\egw{\textbf{Całe niebo interesuje się twoim domem}. \textbf{Bóg i Chrystus i \underline{niebiańscy aniołowie}} pragną gorąco, abyś tak wychowywał swoje dzieci, by były przygotowane do wejścia do rodziny odkupionych.}[17LtMs, Ms 161, 1902, par. 11][https://egwwritings.org/read?panels=p14067.9877018&index=0]

To nie jest sprzeczność. Całe niebo jest reprezentowane przez Ojca, Syna i Ducha Świętego, a w tym cytacie konkretnie wspomniała \egwinline{Bóg i Chrystus i \textbf{niebiańscy aniołowie}.} Istnieje ścisły związek między działaniem Ducha Świętego a służbą aniołów. Natchnienie świadczy:

\egw{Każdemu człowiekowi dana jest miara \textbf{Ducha} dla wspólnego pożytku. \textbf{Poprzez służbę aniołów \underline{Duch Święty może działać} na umysł i serce ludzkiego pośrednika}, i pociągnąć go do Chrystusa, który zapłacił okup za jego duszę, aby grzesznik mógł zostać uwolniony z niewoli grzechu i Szatana.}[8LtMs, Lt 71, 1893, par. 10][https://egwwritings.org/read?panels=p14058.6086016&index=0]

Ta anielska służba jest jednym z elementów przyrzeczenia chrztu świętego z Mateusza 28:19. Kiedy Ellen White powiedziała, \egwinline{\textbf{Niebiańskie moce} \textbf{zobowiązały się} do służenia ludzkim pośrednikom...}, odnosiła się do świętych aniołów. Związek między Duchem Świętym a świętymi aniołami wykracza poza zakres tej książki, ale możesz zgłębić ten temat w kontynuacji, \textit{Rediscovering the Pillar}\footnote{Pobierz za darmo: \href{https://forgottenpillar.com/book/rediscovering-the-pillar}{https://forgottenpillar.com/book/rediscovering-the-pillar}}, w sekcji o Duchu Świętym\footnote{Zobacz również studium o aniołach \href{https://notefp.link/angels}{https://notefp.link/angels}}.