\chapter{The personality of God - by James S. White}


\chapter{Osobowość Boga - autorstwa Jamesa S. White'a}


In what follows, we will examine James White’s pamphlet titled “\textit{The Personality of God}”. When we read this article, we will see that James White continues where Brother Loughborough left off, and that he expands and deepens the understanding behind the first point of the \emcap{Fundamental Principles}.


W dalszej części przyjrzymy się broszurze Jamesa White'a zatytułowanej "\textit{Osobowość Boga}". Czytając ten artykuł, zobaczymy, że James White kontynuuje tam, gdzie Brat Loughborough zakończył, oraz że rozszerza i pogłębia zrozumienie stojące za pierwszym punktem \emcap{Fundamentalnych Zasad}.


James White’s tract was printed multiple times, advertised 54 times, and reprinted twice in the Review and Herald publication. His view on the \emcap{personality of God} was well known and spread throughout Adventism. In this pamphlet, we will see clear criticism toward the ideas that Kellogg advocated in the Living Temple.


Traktat Jamesa White'a był drukowany wielokrotnie, reklamowany 54 razy i przedrukowany dwukrotnie w publikacji Review and Herald. Jego pogląd na temat \emcap{osobowości Boga} był dobrze znany i rozpowszechniony w adwentyzmie. W tej broszurze zobaczymy wyraźną krytykę idei, które Kellogg propagował w The Living Temple.


\othersQuote{\textbf{MAN was made in the image of God}. ‘And God said, Let us make man in our image, after our likeness.’ ‘So God created man in his own image, in the image of God created he him.’ Genesis 1:26, 27. See also chap. 9:6; 1 Corinthians 11:7. \textbf{Those who deny the personality of God, say that ‘image’ here does not mean \underline{physical form}, but moral image, and they make this the grand starting point to prove the immortality of all men}. The argument stands thus: First, man was made in God’s moral image. Second, God is an immortal being. Third, therefore all men are immortal. But this mode of reasoning would also prove man omnipotent, omniscient, and omnipresent, and thus clothe mortal man with all the attributes of the deity. Let us try it: First, man was made in God’s moral image. Second, God is omnipotent, omniscient, and omnipresent. Third, therefore, man is omnipotent, omniscient, and omnipresent. That which proves too much, proves nothing to the point, therefore the position that the image of God means his moral image, cannot be sustained. \textbf{As proof that God is a person, read his own words to Moses}: ‘And the Lord said, Behold there is a place by me, and thou shalt stand upon a rock; and it shall come to pass, while my glory passeth by, that I will put thee in a cleft of the rock, and will cover thee \textbf{with my hand} while \textbf{I pass by}. And I will take away \textbf{mine hand} and thou shalt \textbf{see my back parts}; \textbf{but my face shall not be seen}.’ Exodus 33:21-23. See also chap. 24:9-11. \textbf{Here God tells Moses that he shall \underline{see his form}}. \textbf{To say that God made it appear to Moses that he saw his form, when he has no form, is charging God with adding to falsehood a sort of juggling deception upon his servant Moses}.}[James S. White, PERGO 1.1; 1861][https://egwwritings.org/?ref=en\_PERGO.1.1&para=1471.3]


\othersQuote{\textbf{CZŁOWIEK został stworzony na obraz Boga}. 'I rzekł Bóg: Uczyńmy człowieka na nasz obraz, według naszego podobieństwa.' 'I stworzył Bóg człowieka na swój obraz, na obraz Boga go stworzył.' Księga Rodzaju 1:26, 27. Zobacz także rozdz. 9:6; 1 List do Koryntian 11:7. \textbf{Ci, którzy zaprzeczają osobowości Boga, mówią, że 'obraz' nie oznacza tu \underline{fizycznej formy}, lecz obraz moralny, i czynią to głównym punktem wyjścia do udowodnienia nieśmiertelności wszystkich ludzi}. Argument przedstawia się następująco: Po pierwsze, człowiek został stworzony na moralny obraz Boga. Po drugie, Bóg jest istotą nieśmiertelną. Po trzecie, zatem wszyscy ludzie są nieśmiertelni. Ale ten sposób rozumowania dowodziłby również, że człowiek jest wszechmocny, wszechwiedzący i wszechobecny, przyodziewając tym samym śmiertelnego człowieka we wszystkie atrybuty bóstwa. Spróbujmy: Po pierwsze, człowiek został stworzony na moralny obraz Boga. Po drugie, Bóg jest wszechmocny, wszechwiedzący i wszechobecny. Po trzecie, zatem człowiek jest wszechmocny, wszechwiedzący i wszechobecny. To, co dowodzi zbyt wiele, nie dowodzi niczego, dlatego stanowisko, że obraz Boga oznacza Jego moralny obraz, nie może być utrzymane. \textbf{Jako dowód na to, że Bóg jest osobą, przeczytaj Jego własne słowa do Mojżesza}: 'I rzekł Pan: Oto jest miejsce przy mnie i staniesz na skale. A gdy będzie przechodzić moja chwała, postawię cię w rozpadlinie skalnej i \textbf{zakryję cię swoją ręką}, aż \textbf{przejdę}. A gdy \textbf{cofnę rękę}, \textbf{ujrzysz mnie z tyłu}, ale \textbf{mojego oblicza nie będzie można zobaczyć}.' Księga Wyjścia 33:21-23. Zobacz także rozdz. 24:9-11. \textbf{Tutaj Bóg mówi Mojżeszowi, że \underline{zobaczy Jego postać}}. \textbf{Twierdzenie, że Bóg sprawił, iż Mojżeszowi wydawało się, że widzi Jego postać, podczas gdy On nie ma postaci, jest oskarżaniem Boga o dodanie do fałszu rodzaju kuglarskiego oszustwa wobec swojego sługi Mojżesza}.}[James S. White, PERGO 1.1; 1861][https://egwwritings.org/?ref=en\_PERGO.1.1&para=1471.3]


\othersQuoteNoGap{But the skeptic thinks he sees a contradiction between verse 11, which says that the Lord spake unto Moses face to face, and verse 20, which states that Moses could not see his face. But let Numbers 12:5-8 remove the difficulty. \textbf{‘And the Lord came down in the pillar of the cloud}, and stood in the door of the tabernacle, and called Aaron and Miriam, and they both came forth. And he said, Hear now my words. If there be a prophet among you, I, the Lord, will make myself known unto him in a vision, and will speak unto him in a dream. My servant Moses is not so, who is faithful in all mine house. \textbf{With him will I speak mouth to mouth, even \underline{apparently}}.’}[James S. White, PERGO 2.1; 1861][https://egwwritings.org/?ref=en\_PERGO.2.1&para=1471.6]


\othersQuoteNoGap{Ale sceptyk uważa, że widzi sprzeczność między wierszem 11, który mówi, że Pan rozmawiał z Mojżeszem twarzą w twarz, a wierszem 20, który stwierdza, że Mojżesz nie mógł zobaczyć Jego oblicza. Niech Księga Liczb 12:5-8 usunie tę trudność. \textbf{'I zstąpił Pan w słupie obłoku}, i stanął u wejścia do namiotu, i zawołał Aarona i Miriam, i wyszli oboje. I rzekł: Słuchajcie teraz moich słów: Jeśli jest między wami prorok, objawię mu się w widzeniu, będę mówił z nim we śnie. Lecz nie tak jest z moim sługą Mojżeszem, który jest wierny w całym moim domu. \textbf{Z nim będę rozmawiał usta w usta, \underline{jawnie}}.'}[James S. White, PERGO 2.1; 1861][https://egwwritings.org/?ref=en\_PERGO.2.1&para=1471.6]


\othersQuoteNoGap{The great and dreadful God came down, wrapped in a cloud of glory. \textbf{This cloud could be seen, but not the face which possesses more dazzling brightness than a thousand suns}. Under these circumstances Moses was permitted to draw near and \textbf{converse with God face to face, or mouth to mouth, even \underline{apparently}}.}[James S. White, PERGO 2.2; 1861][https://egwwritings.org/?ref=en\_PERGO.2.2&para=1471.7]


\othersQuoteNoGap{Wielki i straszny Bóg zstąpił, otoczony obłokiem chwały. \textbf{Ten obłok można było zobaczyć, ale nie oblicze, które posiada jaśniejszy blask niż tysiąc słońc}. W tych okolicznościach Mojżeszowi pozwolono zbliżyć się i \textbf{rozmawiać z Bogiem twarzą w twarz, czyli usta w usta, \underline{jawnie}}.}[James S. White, PERGO 2.2; 1861][https://egwwritings.org/?ref=en\_PERGO.2.2&para=1471.7]


\othersQuoteNoGap{Says the prophet Daniel, ‘I beheld till the thrones were cast down, and \textbf{the Ancient of days did sit}, whose garment was white as snow, \textbf{and the hairs of his head like the pure wool}; \textbf{his throne was like the fiery flame, and his wheels as burning fire}.’ Chap. 7:9. ‘I saw in the night visions, and, behold, one like the Son of man came with the clouds of heaven, and \textbf{came to the Ancient of days}, and they brought \textbf{him near before him}, and there was given him dominion and glory and a kingdom.’ Verses 13, 14.}[James S. White, PERGO 2.3; 1861][https://egwwritings.org/?ref=en\_PERGO.2.3&para=1471.8]


\othersQuoteNoGap{Mówi prorok Daniel: 'Patrzyłem, aż postawione zostały trony i \textbf{zasiadł Przedwieczny}; jego szata była biała jak śnieg, a \textbf{włosy na jego głowie jak czysta wełna}; \textbf{jego tron jak płomienie ognia, a jego koła jak płonący ogień}.' Rozdz. 7:9. 'Widziałem w nocnych widzeniach: Oto na obłokach nieba przyszedł ktoś podobny do Syna Człowieczego, \textbf{zbliżył się do Przedwiecznego} i \textbf{przyprowadzono go przed Niego}. I dano mu panowanie, chwałę i królestwo.' Wersety 13, 14.}[James S. White, PERGO 2.3; 1861][https://egwwritings.org/?ref=en\_PERGO.2.3&para=1471.8]


\othersQuoteNoGap{Here is a sublime description of the action of \textbf{two personages}; viz, \textbf{God the Father, and his Son Jesus Christ}. \textbf{Deny their personality, and there is not a distinct idea in these quotations from Daniel}. In connection with this quotation read the apostle’s declaration that \textbf{the Son was in the express image of his Father’s person}. ‘God, who at sundry times, and in divers manners, spake in time past unto the fathers by the prophets, hath in these last days spoken unto us by his Son, whom he hath appointed heir of all things, by whom also he made the worlds; \textbf{who being the brightness of his glory, and the express image of his person}.’ Hebrews 1:1-3.}[James S. White, PERGO 3.1; 1861][https://egwwritings.org/?ref=en\_PERGO.3.1&para=1471.11]


\othersQuoteNoGap{Oto wzniosły opis działania \textbf{dwóch osób}; a mianowicie \textbf{Boga Ojca i Jego Syna Jezusa Chrystusa}. \textbf{Zaprzecz ich osobowości, a w tych cytatach z Daniela nie ma żadnej wyraźnej myśli}. W związku z tym cytatem przeczytaj deklarację apostoła, że \textbf{Syn był wyrazem istoty swojego Ojca}. 'Bóg, który wielokrotnie i na różne sposoby przemawiał niegdyś do ojców przez proroków, w tych ostatnich dniach przemówił do nas przez Syna, którego ustanowił dziedzicem wszystkiego, przez którego także stworzył światy; \textbf{który, będąc blaskiem jego chwały i wyrazem jego istoty}.' List do Hebrajczyków 1:1-3.}[James S. White, PERGO 3.1; 1861][https://egwwritings.org/?ref=en\_PERGO.3.1&para=1471.11]


\othersQuoteNoGap{We here add the testimony of Christ. ‘And the Father himself which hath sent me, hath borne witness of me. Ye have neither heard his voice at any time, \textbf{nor seen his shape}.’ John 5:37. See also Philippians 2:6. \textbf{To say that the Father has not a personal shape, seems the most pointed contradiction of plain scripture terms}. \\
OBJECTION. - ‘\textbf{\underline{God is a Spirit}}.’ John 4:24.}[James S. White, PERGO 3.2; 1861][https://egwwritings.org/?ref=en\_PERGO.3.2&para=1471.12]


\othersQuoteNoGap{Dodajemy tu świadectwo Chrystusa. 'A Ojciec, który mnie posłał, on wydał świadectwo o mnie. Nigdy nie słyszeliście jego głosu \textbf{ani nie widzieliście jego postaci}.' Ewangelia Jana 5:37. Zobacz także List do Filipian 2:6. \textbf{Twierdzenie, że Ojciec nie ma osobowej postaci, wydaje się najbardziej wyraźnym zaprzeczeniem jasnych słów Pisma}. \\
ZARZUT. - '\textbf{\underline{Bóg jest duchem}}.' Ewangelia Jana 4:24.}[James S. White, PERGO 3.2; 1861][https://egwwritings.org/?ref=en\_PERGO.3.2&para=1471.12]


\othersQuoteNoGap{ANSWER. - \textbf{Angels are also spirits} [Psalm 104:4], yet those that visited Abram and Lot, lay down, ate, and took hold of Lot’s hand. \textbf{They were spirit beings. So is God a Spirit being}.}[James S. White, PERGO 3.3; 1861][https://egwwritings.org/?ref=en\_PERGO.3.3&para=1471.13]


\othersQuoteNoGap{ODPOWIEDŹ. - \textbf{Aniołowie też są duchami} [Psalm 104:4], jednak ci, którzy odwiedzili Abrahama i Lota, położyli się, jedli i chwycili Lota za rękę. \textbf{Byli istotami duchowymi. Tak samo Bóg jest istotą duchową}.}[James S. White, PERGO 3.3; 1861][https://egwwritings.org/?ref=en\_PERGO.3.3&para=1471.13]


\othersQuoteNoGap{OBJ. - \textbf{God is everywhere}. Proof. Psalm 139:1-8. \textbf{He is as much in every place as in any one place}.}[James S. White, PERGO 3.4; 1861][https://egwwritings.org/?ref=en\_PERGO.3.4&para=1471.14]


\othersQuoteNoGap{ZARZUT - \textbf{Bóg jest wszędzie}. Dowód. Psalm 139:1-8. \textbf{Jest On tak samo obecny w każdym miejscu, jak w jakimkolwiek pojedynczym miejscu}.}[James S. White, PERGO 3.4; 1861][https://egwwritings.org/?ref=en\_PERGO.3.4&para=1471.14]


\othersQuoteNoGap{ANS. - 1. \textbf{God is everywhere by virtue of his omniscience}, as will be seen by the very words of David referred to above. Verses 1-6. ‘O Lord, \textbf{thou hast searched me, and known me}. \textbf{Thou knowest} my down-sitting and mine uprising; \textbf{thou understandest} my thought afar off. Thou compassest my path and my lying down, and art \textbf{acquainted }with all my ways. For there is not a word in my tongue, but, lo, O Lord, \textbf{thou knowest it altogether}. Thou hast beset me behind and before, and laid thy hand upon me. \textbf{Such knowledge} is too wonderful for me. It is high; I cannot attain unto it.’}[James S. White, PERGO 3.5; 1861][https://egwwritings.org/?ref=en\_PERGO.3.5&para=1471.15]


\othersQuoteNoGap{ODP. - 1. \textbf{Bóg jest wszędzie poprzez swoją wszechwiedzę}, jak to widać w przytoczonych powyżej słowach Dawida. Wersety 1-6. 'O Panie, \textbf{zbadałeś mnie i poznałeś}. \textbf{Ty wiesz}, kiedy siadam i wstaję, \textbf{rozumiesz} moje myśli z daleka. Ty wyznaczasz moją ścieżkę i spoczynek, i jesteś \textbf{obeznany} ze wszystkimi moimi drogami. Zanim słowo pojawi się na moim języku, ty, PANIE, już je \textbf{znasz całkowicie}. Otaczasz mnie z tyłu i z przodu i położyłeś na mnie swoją rękę. \textbf{Taka wiedza} jest zbyt cudowna dla mnie, zbyt wzniosła, nie mogę jej pojąć.'}[James S. White, PERGO 3.5; 1861][https://egwwritings.org/?ref=en\_PERGO.3.5&para=1471.15]


\othersQuoteNoGap{2. \textbf{God is \underline{everywhere by virtue of his Spirit}, \underline{which is his representative}, and is manifested wherever he pleases}, as will be seen by the very words the objector claims, referred to above. Verses 7-10. ‘\textbf{Whither shall I go from \underline{thy Spirit}}? \textbf{or whither shall I flee from \underline{thy presence}}? If I ascend up into heaven, thou art there; if I make my bed in hell, behold, thou art there. If I take the wings of the morning, and dwell in the uttermost parts of the sea, even there shall thy hand lead me, and thy right hand shall hold me.’}[James S. White, PERGO 4.1; 1861][https://egwwritings.org/?ref=en\_PERGO.4.1&para=1471.18]


\othersQuoteNoGap{2. \textbf{Bóg jest \underline{wszędzie poprzez swojego Ducha}, \underline{który jest Jego przedstawicielem} i objawia się gdziekolwiek On zechce}, jak to widać w tych samych słowach, na które powołuje się oponent. Wersety 7-10. '\textbf{Dokąd ujdę przed \underline{twoim Duchem}}? \textbf{Dokąd ucieknę przed \underline{twoim obliczem}}? Jeśli wstąpię do nieba, ty tam jesteś, jeśli przygotuję sobie posłanie w piekle, i tam jesteś. Gdybym wziął skrzydła jutrzenki, aby zamieszkać na krańcu morza, nawet tam twoja ręka będzie mnie prowadzić i twoja prawica mnie podtrzyma.'}[James S. White, PERGO 4.1; 1861][https://egwwritings.org/?ref=en\_PERGO.4.1&para=1471.18]


\othersQuoteNoGap{\textbf{God is in heaven.} This we are taught in the Lord’s prayer. ‘\textbf{Our Father which art in heaven}.’ Matthew 6:9; Luke 11:2. \textbf{But if God is as much in every place as he is in any one place, then heaven is also as much in every place as it is in any one place, and the idea of going to heaven is all a mistake}. We are all in heaven; and the Lord’s prayer, according to this foggy theology simply means, Our Father \textbf{which art everywhere,} hallowed be thy name. Thy kingdom come, thy will be done, on earth, \textbf{as it is everywhere}.}[James S. White, PERGO 4.2; 1861][https://egwwritings.org/?ref=en\_PERGO.4.2&para=1471.19]


\othersQuoteNoGap{\textbf{Bóg jest w niebie}. Tego uczy nas Modlitwa Pańska. '\textbf{Ojcze nasz, który jesteś w niebie}.' Mateusz 6:9; Łukasz 11:2. \textbf{Ale jeśli Bóg jest tak samo obecny w każdym miejscu, jak w jakimkolwiek pojedynczym miejscu, to niebo również jest tak samo obecne w każdym miejscu, jak w jakimkolwiek pojedynczym miejscu, a idea pójścia do nieba jest całkowitym nieporozumieniem}. Wszyscy jesteśmy w niebie; a Modlitwa Pańska, według tej mglistej teologii, oznacza po prostu: Ojcze nasz, \textbf{który jesteś wszędzie}, święć się imię twoje. Przyjdź królestwo twoje, bądź wola twoja, jak na ziemi, \textbf{tak i wszędzie}.}[James S. White, PERGO 4.2; 1861][https://egwwritings.org/?ref=en\_PERGO.4.2&para=1471.19]


\othersQuoteNoGap{Again, Bible readers have believed that Enoch and Elijah were really taken up \textbf{to God in heaven}. \textbf{But if God and heaven be as much in every place as in any one place, this is all a mistake}. They were not translated. And all that is said about the chariot of fire, and horses of fire, and the attending whirlwind to take Elijah up into heaven, was a useless parade. They only evaporated, and a misty vapor passed through the entire universe. This is all of Enoch and Elijah that the mind can possibly grasp, \textbf{admitting that God and heaven are no more in any one place than in every place}. But it is said of Elijah that he ‘\textbf{went up} by a whirlwind \textbf{into heaven}.’ 2 Kings 2:11. And of Enoch it is said that he ‘walked with God, and was not, for God took him.’ Genesis 5:24.}[James S. White, PERGO 4.3; 1861][https://egwwritings.org/?ref=en\_PERGO.4.3&para=1471.20]


\othersQuoteNoGap{Ponadto, czytelnicy Biblii wierzyli, że Enoch i Eliasz zostali rzeczywiście zabrani \textbf{do Boga w niebie}. \textbf{Ale jeśli Bóg i niebo są tak samo obecne w każdym miejscu jak w jakimkolwiek pojedynczym miejscu, to jest to całkowite nieporozumienie}. Nie zostali oni przemienieni. A cała ta historia o ognistym rydwanie, ognistych koniach i towarzyszącej wichurze, która zabrała Eliasza do nieba, była niepotrzebną paradą. Oni po prostu wyparowali, a mglista para przeszła przez cały wszechświat. To wszystko, co umysł może pojąć o Enochu i Eliaszu, \textbf{przyjmując, że Bóg i niebo nie są bardziej obecni w jakimkolwiek pojedynczym miejscu niż wszędzie}. Ale o Eliaszu jest napisane, że '\textbf{wstąpił} w wichurze \textbf{do nieba}.' 2 Królewska 2:11. A o Enochu jest napisane, że 'chodził z Bogiem i nie było go, bo zabrał go Bóg.' Księga Rodzaju 5:24.}[James S. White, PERGO 4.3; 1861][https://egwwritings.org/?ref=en\_PERGO.4.3&para=1471.20]


\othersQuoteNoGap{\textbf{Jesus is said to be on the right hand of the Majesty on high}. Hebrews 1:3. ‘So, then, after the Lord had spoken unto them \textbf{he was received \underline{up into heaven}}, \textbf{and sat on the right hand of God}.’ Mark 16:19. \textbf{But if heaven be everywhere, and God everywhere, then Christ’s ascension up to heaven, at the Father’s right hand, simply means that he went everywhere}! He was only taken up where the cloud hid him from the gaze of his disciples, and then evaporated and went everywhere! So that instead of the lovely Jesus, so beautifully described in both Testaments, we have only a sort of essence dispersed through the entire universe. And in harmony with this rarified theology, Christ’s second advent, or his return, would be the condensation of this essence to some locality, say the mount of Olivet! \textbf{Christ arose from the dead with a physical form}. ‘He is not here,’ said the angel, ‘for he is risen as he said.’ Matthew 28:6.}[James S. White, PERGO 5.1; 1861][https://egwwritings.org/?ref=en\_PERGO.5.1&para=1471.23]


\othersQuoteNoGap{\textbf{Jest powiedziane, że Jezus jest po prawicy Majestatu na wysokości}. Hebrajczyków 1:3. 'A Pan, gdy to do nich powiedział, \textbf{został \underline{wzięty w górę do nieba}} \textbf{i zasiadł po prawicy Boga}.' Marek 16:19. \textbf{Ale jeśli niebo jest wszędzie i Bóg jest wszędzie, to wniebowstąpienie Chrystusa, do prawicy Ojca, oznacza po prostu, że udał się wszędzie}! Został uniesiony tylko tam, gdzie obłok ukrył go przed wzrokiem uczniów, a potem wyparował i udał się wszędzie! Tak więc zamiast ukochanego Jezusa, tak pięknie opisanego w obu Testamentach, mamy tylko rodzaj esencji rozproszonej po całym wszechświecie. A zgodnie z tą rozrzedzoną teologią, powtórne przyjście Chrystusa, czyli Jego powrót, byłby kondensacją tej esencji w jakimś miejscu, powiedzmy na Górze Oliwnej! \textbf{Chrystus powstał z martwych w fizycznej postaci}. 'Nie ma go tu,' powiedział anioł, 'bo zmartwychwstał, jak powiedział.' Mateusz 28:6.}[James S. White, PERGO 5.1; 1861][https://egwwritings.org/?ref=en\_PERGO.5.1&para=1471.23]


\othersQuoteNoGap{‘And as they went to tell his disciples, behold, Jesus met them, saying, All hail! And they came and \textbf{held him by the feet}, and they worshiped him.’ Verse 9.}[James S. White, PERGO 5.2; 1861][https://egwwritings.org/?ref=en\_PERGO.5.2&para=1471.24]


\othersQuoteNoGap{'A gdy szły, aby oznajmić to jego uczniom, oto Jezus wyszedł im naprzeciw, mówiąc: Witajcie! A one podeszły i \textbf{objęły go za nogi}, i oddały mu pokłon.'}[James S. White, PERGO 5.2; 1861][https://egwwritings.org/?ref=en\_PERGO.5.2&para=1471.24]


\othersQuoteNoGap{‘\textbf{Behold my hands and my feet},’ said Jesus to those who stood in doubt of his resurrection, ‘that it is I myself. \textbf{Handle me and see, \underline{for a spirit hath not flesh and bones} as ye see me have}. And when he had thus spoken, he \textbf{showed them his hands and his feet}. And while they yet believed not for joy, and wondered, he said unto them, Have ye here any meat? And they gave him a piece of broiled fish, and of an honey-comb, and he took it and did eat before them.’ Luke 24:39-43.}[James S. White, PERGO 5.3; 1861][https://egwwritings.org/?ref=en\_PERGO.5.3&para=1471.25]


\othersQuoteNoGap{'\textbf{Spójrzcie na moje ręce i nogi},' powiedział Jezus do tych, którzy wątpili w jego zmartwychwstanie, 'że to ja jestem. \textbf{Dotknijcie mnie i zobaczcie, \underline{bo duch nie ma ciała ani kości}, jak widzicie, że ja mam}. A gdy to powiedział, \textbf{pokazał im ręce i nogi}. A gdy oni z radości jeszcze nie wierzyli i dziwili się, zapytał ich: Macie tu coś do jedzenia? I podali mu kawałek pieczonej ryby i plaster miodu. A on wziął i jadł przy nich.'}[James S. White, PERGO 5.3; 1861][https://egwwritings.org/?ref=en\_PERGO.5.3&para=1471.25]


\othersQuoteNoGap{After Jesus addressed his disciples on the mount of Olivet, he \textbf{was taken up from them}, and a cloud received him out of their sight. ‘And while they looked steadfastly \textbf{toward heaven as he went up,} behold two men stood by them in white apparel, which also said, Ye men of Galilee, why stand ye gazing up into heaven? This same Jesus which is \textbf{taken up from you into heaven}, shall so come in like manner as ye have seen him \textbf{go into heaven}.’ Acts 1:9-11. J. W.}[James S. White, PERGO 6.1; 1861][https://egwwritings.org/?ref=en\_PERGO.6.1&para=1471.27]


\othersQuoteNoGap{Po tym, jak Jezus przemówił do swoich uczniów na Górze Oliwnej, \textbf{został uniesiony od nich}, a obłok zabrał go sprzed ich oczu. 'A gdy patrzyli uważnie \textbf{w niebo, jak wstępował}, oto stanęli przy nich dwaj mężowie w białych szatach, którzy powiedzieli: Mężowie z Galilei, dlaczego stoicie, patrząc w niebo? Ten Jezus, który \textbf{został od was wzięty w górę do nieba}, przyjdzie tak samo, jak widzieliście go \textbf{wstępującego do nieba}.' Dzieje Apostolskie 1:9-11. J. W.}[James S. White, PERGO 6.1; 1861][https://egwwritings.org/?ref=en\_PERGO.6.1&para=1471.27]


James White fights the idea that God is just a spirit, and as such, is present \others{as much in every place as in any one place}. He gives plain and positive testimony from Scripture that God is a personal being; we see the very same sentiments in Ellen White’s writings.


James White zwalcza ideę, że Bóg jest tylko duchem i jako taki jest obecny \others{tak samo w każdym miejscu, jak w jakimkolwiek pojedynczym miejscu}. Przedstawia jasne i jednoznaczne świadectwo z Pisma Świętego, że Bóg jest osobową istotą; te same poglądy widzimy w pismach Ellen White.


\egw{The mighty power that works through all nature and sustains all things is not, as some men of science claim, \textbf{merely an all-pervading principle}, an actuating energy. \textbf{\underline{God is a spirit; yet He is a personal being}}, \textbf{for man was made in His image}. \textbf{As \underline{a personal being}}, God has revealed Himself in His Son. Jesus, the outshining of the Father’s glory, “and \textbf{the express \underline{image of His person}}” (Hebrews 1:3), was on earth found in fashion as a man. As \textbf{a personal Saviour} He came to the world. As \textbf{a personal Saviour He ascended \underline{on high}}. As \textbf{a personal Saviour He intercedes \underline{in the heavenly courts}}. \textbf{Before the throne of God} in our behalf ministers “One like the Son of man.” Daniel 7:13.}[Ed 131.5; 1903][https://egwwritings.org/?ref=en\_Ed.131.5&para=29.632]


\egw{Potężna moc, która działa w całej naturze i podtrzymuje wszystkie rzeczy, nie jest, jak twierdzą niektórzy naukowcy, \textbf{jedynie wszechobecną zasadą}, energią napędową. \textbf{\underline{Bóg jest duchem, ale jest osobową istotą}}, \textbf{gdyż człowiek został stworzony na Jego obraz}. \textbf{Jako \underline{osobowa istota}}, Bóg objawił się w swoim Synu. Jezus, blask chwały Ojca, "i \textbf{\underline{wyraz Jego istoty}}" (Hebrajczyków 1:3), był na ziemi znaleziony w postaci człowieka. Jako \textbf{osobowy Zbawiciel} przyszedł na świat. Jako \textbf{osobowy Zbawiciel wstąpił \underline{na wysokość}}. Jako \textbf{osobowy Zbawiciel wstawia się \underline{w niebiańskich sądach}}. \textbf{Przed tronem Bożym} w naszym imieniu służy "Jakby Syn Człowieczy." Daniela 7:13.}


Ellen White and the Adventist pioneers made a distinction between the terms ‘\textit{spirit}’ and ‘\textit{being}’. God is a personal being, not just a spirit. He is not\others{as much in every place as in any one place}, but He is\others{in one place more than another}[John. N. Loughborough, “Is God a Person?” The Adventist Review and Sabbath Herald, September 18, 1855][https://documents.adventistarchives.org/Periodicals/RH/RH18550918-V07-06.pdf]. He is in heaven, in His temple, sitting on His throne—in person—and He is everywhere present by His representative, the Holy Spirit.


Ellen White i pionierzy adwentyzmu rozróżniali między terminami '\textit{duch}' a '\textit{istota}'. Bóg jest osobową istotą, nie tylko duchem. On nie jest\others{tak samo obecny w każdym miejscu jak w jakimkolwiek jednym miejscu}, ale jest\others{w jednym miejscu bardziej niż w innym}. Jest w niebie, w swojej świątyni, siedząc na swoim tronie—osobiście—i jest wszędzie obecny przez swojego przedstawiciela, Ducha Świętego.


Here are some other quotations from Sister White that are in harmony with the pioneers’ views on the \emcap{personality of God}:


Oto kilka innych cytatów od Siostry White, które są w harmonii z poglądami pionierów na temat \emcap{osobowości Boga}:


\egw{He \normaltext{[Jesus]} taught that God was a rewarder of the righteous, and a punisher of the transgressor. \textbf{He was not an intangible spirit}, but a living ruler of the universe. \textbf{This gracious Father} was constantly working for the good of man, and mindful of all that concerns him...}[3SP 47.1; 1878][https://egwwritings.org/?ref=en\_3SP.47.1&para=142.195]


\egw{On \normaltext{[Jezus]} nauczał, że Bóg był nagradzającym sprawiedliwych i karzącym przestępców. \textbf{Nie był On nieuchwytnym duchem}, ale żywym władcą wszechświata. \textbf{Ten łaskawy Ojciec} nieustannie działał dla dobra człowieka i troszczył się o wszystko, co go dotyczy...}


\egw{\textbf{The Bible shows us \underline{God in His high and holy place}}, not in a state of inactivity, not in silence and solitude, but surrounded by ten thousand times ten thousand and thousands of thousands of holy beings, all waiting to do His will. \textbf{Through these messengers He is in active communication with every part of His dominion}. \textbf{\underline{By His Spirit He is everywhere present}}. \textbf{Through the agency of His Spirit and His angels} He ministers to the children of men.}[MH 417.2; 1905][https://egwwritings.org/?ref=en\_MH.417.2&para=135.2136]


\egw{\textbf{Biblia ukazuje nam \underline{Boga w Jego wysokim i świętym miejscu}}, nie w stanie bezczynności, nie w ciszy i samotności, ale otoczonego dziesięć tysięcy razy dziesięcioma tysiącami i tysiącami tysięcy świętych istot, wszystkich gotowych wypełnić Jego wolę. \textbf{Poprzez tych posłańców utrzymuje On aktywną komunikację z każdą częścią swojego władztwa}. \textbf{\underline{Przez swojego Ducha jest wszędzie obecny}}. \textbf{Poprzez działanie swojego Ducha i swoich aniołów} służy dzieciom ludzkim.}


\egw{The greatness of God is to us incomprehensible. ‘\textbf{The Lord’s throne is in heaven}’ (Psalm 11:4); \textbf{\underline{yet by His Spirit He is everywhere present}}. \textbf{He has an intimate knowledge} of, and a personal interest in, all the works of His hand.}[Ed 132.2; 1903][https://egwwritings.org/?ref=en\_Ed.132.2&para=29.636]


\egw{Wielkość Boga jest dla nas niepojęta. '\textbf{Tron Pana jest w niebie}' (Psalm 11:4); \textbf{\underline{jednak przez swojego Ducha jest On wszędzie obecny}}. \textbf{Ma On dokładną wiedzę} o wszystkich dziełach swoich rąk i osobiste zainteresowanie nimi.}


\egw{Through Jesus Christ, \textbf{God—not a perfume, \underline{not something intangible}, \underline{but a personal God}}—created man and endowed him with intelligence and power.}[Ms117-1898.10; 1898][https://egwwritings.org/?ref=en\_Ms117-1898.10&para=7182.15]


\egw{Przez Jezusa Chrystusa, \textbf{Bóg—nie perfumy, \underline{nie coś nieuchwytnego}, \underline{ale osobowy Bóg}}—stworzył człowieka i obdarzył go inteligencją i mocą.}


Continuing in James White’s pamphlet, we read his sharp criticism on the notion of an immaterial God. Before that, let’s briefly recall Dr. Kellogg’s argument that\others{\textbf{\underline{Discussions respecting the form of God are utterly unprofitable}}}[Dr. John H. Kellogg, The Living Temple, p.33.][https://archive.org/details/J.H.Kellogg.TheLivingTemple1903/page/n33/] because God is\others{\textbf{far beyond our comprehension }\textbf{\underline{as are the bounds of space and time}}}. He believed that God’s person is not constrained to one locality because He is in\others{as much in every place as in any one place}[James S. White, PERGO 4.3; 1861][https://egwwritings.org/?ref=en\_PERGO.4.3&para=1471.20] \footnote{In the Living Temple, Dr. Kellogg objected that God cannot be everywhere presente at once: “\textit{Says one}, ‘God may be present by his Spirit, or by his power, but certainly God himself \textit{cannot be present everywhere at once}.’ We answer: How can power be separated from the source of power? Where God’s Spirit is at work, where God’s power is manifested, God \textit{himself is actually and truly present}…” \href{https://archive.org/details/J.H.Kellogg.TheLivingTemple1903/page/n29/}{John H. Kellogg, The Living Temple, p.28}.}. If God in His personality were truly a definite being, having a tangible body, then He would not be able to be present\others{as much in every place as in any one place} and, thus, sustain life. James White continues against the reasoning that God is immaterial in His person.


Kontynuując w broszurze Jamesa White'a, czytamy jego ostrą krytykę pojęcia niematerialnego Boga. Wcześniej przypomnijmy krótko argument dr. Kellogga, że\others{\textbf{\underline{Dyskusje dotyczące formy Boga są całkowicie bezużyteczne}}} ponieważ Bóg jest\others{\textbf{daleko poza naszym pojmowaniem }\textbf{\underline{tak jak granice przestrzeni i czasu}}}. Wierzył on, że osoba Boga nie jest ograniczona do jednego miejsca, ponieważ jest On\others{tak samo obecny w każdym miejscu jak w jakimkolwiek jednym miejscu} \footnote{W The Living Temple, dr Kellogg sprzeciwiał się temu, że Bóg nie może być wszędzie obecny naraz: "\textit{Ktoś mówi}, 'Bóg może być obecny przez swojego Ducha lub przez swoją moc, ale z pewnością sam Bóg \textit{nie może być obecny wszędzie naraz}.' Odpowiadamy: Jak można oddzielić moc od źródła mocy? Gdzie Duch Boży działa, gdzie moc Boża się objawia, tam sam Bóg jest \textit{rzeczywiście i prawdziwie obecny}…"}.


\othersQuote{IMMATERIALITY}


\othersQuote{NIEMATERIALNOŚĆ}


\othersQuoteNoGap{\textbf{THIS is but another name for nonentity}. \textbf{It is the negative of all} \textbf{things and} \textbf{\underline{beings} }- of all existence. There is not one particle of proof to be advanced to establish its existence. It has no way to manifest itself to any intelligence in heaven or on earth. \textbf{Neither God, angels, nor men could possibly conceive of such a substance, being, or thing}. \textbf{It possesses no property or power by which \underline{to make itself manifest to any intelligent being} in the universe}. Reason and analogy never scan it, or even conceive of it. \textbf{Revelation never reveals it, nor do any of our senses witness its existence}. \textbf{It cannot be seen, felt, heard, tasted, or smelled, even by the strongest organs, or the most acute sensibilities}. It is neither liquid nor solid, soft nor hard - it can neither extend nor contract. In short, it can exert no influence whatever - it can neither act nor be acted upon. And even if it does exist, it can be of no possible use. It possesses no one, desirable property, faculty, or use, yet, strange to say, \textbf{immateriality is the modern Christian’s God}, \textbf{his anticipated heaven}, \textbf{his immortal self} - \textbf{his all}!}[James S. White, PERGO 6.2; 1861][https://egwwritings.org/?ref=en\_PERGO.6.2&para=1471.29]


\othersQuoteNoGap{\textbf{TO jest tylko inne określenie nicości}. \textbf{Jest to zaprzeczenie wszystkich} \textbf{rzeczy i} \textbf{\underline{istot}} - całego istnienia. Nie ma ani jednego dowodu potwierdzającego jej istnienie. Nie ma sposobu, by mogła się objawić jakiejkolwiek inteligencji w niebie czy na ziemi. \textbf{Ani Bóg, ani aniołowie, ani ludzie nie mogliby pojąć takiej substancji, istoty czy rzeczy}. \textbf{Nie posiada ona żadnej właściwości ani mocy, przez którą \underline{mogłaby się objawić jakiejkolwiek inteligentnej istocie} we wszechświecie}. Rozum i analogia nigdy jej nie badają, ani nawet nie pojmują. \textbf{Objawienie nigdy jej nie ujawnia, ani żaden z naszych zmysłów nie świadczy o jej istnieniu}. \textbf{Nie można jej zobaczyć, poczuć, usłyszeć, posmakować ani powąchać, nawet przy najsilniejszych organach czy najbardziej wyczulonych zmysłach}. Nie jest ani płynna, ani stała, miękka ani twarda - nie może się ani rozszerzać, ani kurczyć. Krótko mówiąc, nie może wywierać żadnego wpływu - nie może ani działać, ani być przedmiotem działania. A nawet jeśli istnieje, nie może być w żaden sposób użyteczna. Nie posiada ani jednej pożądanej właściwości, zdolności czy zastosowania, jednak, co dziwne, \textbf{niematerialność jest Bogiem współczesnego chrześcijanina}, \textbf{jego oczekiwanym niebem}, \textbf{jego nieśmiertelnym ja} - \textbf{jego wszystkim}!}


\othersQuoteNoGap{\textbf{O sectarianism! O atheism!! O annihilation!!!} \textbf{who can perceive the nice shades of difference between the one and the other?} They seem alike, all but in name. \textbf{The atheist has no God. \underline{The sectarian has a God without body or parts}.} Who can define the difference? For our part we do not perceive a difference of a single hair; \textbf{they both claim to be the negative of all things which exist} - and both are equally powerless and unknown.}[James S. White, PERGO 6.3; 1861][https://egwwritings.org/?ref=en\_PERGO.6.3&para=1471.30]


\othersQuoteNoGap{\textbf{O sekciarstwo! O ateizm!! O unicestwienie!!!} \textbf{Kto może dostrzec subtelne różnice między jednym a drugim?} Wydają się podobne, różnią się tylko nazwą. \textbf{Ateista nie ma Boga. \underline{Sekciarze mają Boga bez ciała i części}.} Kto może zdefiniować różnicę? Z naszej strony nie dostrzegamy różnicy nawet o włos; \textbf{obaj twierdzą, że są negacją wszystkich rzeczy, które istnieją} - i obaj są równie bezsilni i nieznani.}[James S. White, PERGO 6.3; 1861][https://egwwritings.org/?ref=en\_PERGO.6.3&para=1471.30]


\othersQuoteNoGap{\textbf{The atheist has no after life, or conscious existence beyond the grave. The sectarian has one, \underline{but it is immaterial, like his God; and without body or parts}. Here again both are negative, and both arrive at the same point}. Their faith and hope amount to the same; only it is expressed by different terms.}[James S. White, PERGO 7.1; 1861][https://egwwritings.org/?ref=en\_PERGO.7.1&para=1471.33]


\othersQuoteNoGap{\textbf{Ateista nie ma życia po śmierci ani świadomego istnienia po grobie. Sekciarz ma jedno, \underline{ale jest ono niematerialne, jak jego Bóg; i bez ciała czy części}. Tutaj znowu obaj są negatywni i obaj dochodzą do tego samego punktu}. Ich wiara i nadzieja sprowadzają się do tego samego; tylko wyrażone są innymi terminami.}[James S. White, PERGO 7.1; 1861][https://egwwritings.org/?ref=en\_PERGO.7.1&para=1471.33]


\othersQuoteNoGap{Again, \textbf{the atheist has no heaven in eternity}. \textbf{The sectarian has one, but it is \underline{immaterial in all its properties}, and is therefore the negative of all riches and substances}. Here again they are equal, and arrive at the same point.}[James S. White, PERGO 7.2; 1861][https://egwwritings.org/?ref=en\_PERGO.7.2&para=1471.34]


\othersQuoteNoGap{Ponownie, \textbf{ateista nie ma nieba w wieczności}. \textbf{Sekciarz ma jedno, ale jest ono \underline{niematerialne we wszystkich swoich właściwościach}, i dlatego jest negacją wszelkich bogactw i substancji}. Tutaj znowu są równi i dochodzą do tego samego punktu.}[James S. White, PERGO 7.2; 1861][https://egwwritings.org/?ref=en\_PERGO.7.2&para=1471.34]


\othersQuoteNoGap{As we do not envy them the possession of all they claim, we will now leave them in the quiet and undisturbed enjoyment of the same, and proceed to examine the portion still left for the despised materialist to enjoy.}[James S. White, PERGO 7.3; 1861][https://egwwritings.org/?ref=en\_PERGO.7.3&para=1471.35]


\othersQuoteNoGap{Ponieważ nie zazdrościmy im posiadania wszystkiego, do czego roszczą sobie prawo, pozostawimy ich teraz w spokojnym i niezakłóconym korzystaniu z tego, i przejdziemy do zbadania części, która pozostała do wykorzystania przez pogardzanego materialistę.}[James S. White, PERGO 7.3; 1861][https://egwwritings.org/?ref=en\_PERGO.7.3&para=1471.35]


\othersQuoteNoGap{\textbf{What is God? He is material, organized intelligence, \underline{possessing both body and parts}. Man is in his image.}}[James S. White, PERGO 7.4; 1861][https://egwwritings.org/?ref=en\_PERGO.7.4&para=1471.36]


\othersQuoteNoGap{\textbf{Czym jest Bóg? Jest On materialną, zorganizowaną inteligencją, \underline{posiadającą zarówno ciało jak i części}. Człowiek jest na Jego obraz.}}[James S. White, PERGO 7.4; 1861][https://egwwritings.org/?ref=en\_PERGO.7.4&para=1471.36]


\othersQuoteNoGap{\textbf{What is Jesus Christ? He is the Son of God, and is \underline{like his Father}, being ‘the brightness of his Father’s glory, and the express image of his person.’ \underline{He is a material intelligence, with body, parts}, and passions; possessing immortal flesh and immortal bones}.}[James S. White, PERGO 7.5; 1861][https://egwwritings.org/?ref=en\_PERGO.7.5&para=1471.37]


\othersQuoteNoGap{\textbf{Kim jest Jezus Chrystus? Jest Synem Bożym i jest \underline{podobny do swojego Ojca}, będąc 'jasnością chwały Ojca i wyrazem Jego istoty.' \underline{Jest On materialną inteligencją, z ciałem, częściami} i uczuciami; posiadającą nieśmiertelne ciało i nieśmiertelne kości}.}[James S. White, PERGO 7.5; 1861][https://egwwritings.org/?ref=en\_PERGO.7.5&para=1471.37]


\othersQuoteNoGap{\textbf{What are men?} They are the offspring of Adam. \textbf{They are capable of receiving intelligence and exaltation to such a degree as to be \underline{raised from the dead with a body like that of Jesus Christ}, \underline{and to possess immortal flesh and bones}}. Thus perfected, they will possess \textbf{the material universe}, that is, the earth, as their ‘everlasting inheritance.’ With these hopes and prospects before us, we say to the Christian world who hold to immateriality, that they are welcome to their God - their life - their heaven, and their all. They claim nothing but that which we throw away; and we claim nothing but that which they throw away. \textbf{Therefore, there is no ground for quarrel or contention between us}.}[James S. White, PERGO 7.6; 1861][https://egwwritings.org/?ref=en\_PERGO.7.6&para=1471.38]


\othersQuoteNoGap{\textbf{Kim są ludzie?} Są potomstwem Adama. \textbf{Są zdolni do otrzymywania inteligencji i wywyższenia do takiego stopnia, aby być \underline{wskrzeszonymi z martwych z ciałem podobnym do ciała Jezusa Chrystusa}, \underline{i posiadać nieśmiertelne ciało i kości}}. Tak udoskonaleni, będą posiadać \textbf{materialny wszechświat}, to jest ziemię, jako ich 'wieczne dziedzictwo.' Mając przed sobą te nadzieje i perspektywy, mówimy chrześcijańskiemu światu, który trzyma się niematerialności, że mogą zachować swojego Boga - swoje życie - swoje niebo i wszystko inne. Roszczą sobie prawo tylko do tego, co my odrzucamy; a my roszczmy sobie prawo tylko do tego, co oni odrzucają. \textbf{Dlatego nie ma podstaw do kłótni czy sporu między nami}.}[James S. White, PERGO 7.6; 1861][https://egwwritings.org/?ref=en\_PERGO.7.6&para=1471.38]


\othersQuoteNoGap{We choose all substance - what remains \\
The mystical sectarian gains; \\
All that each claims, each shall possess, \\
Nor grudge each other’s happiness. \\
An immaterial God they choose, \\
For such a God we have no use; \\
\textbf{An immaterial heaven and hell,} \\
\textbf{In such a heaven we cannot dwell.} \\
\textbf{We claim the earth, the air, and sky,} \\
\textbf{And all the starry worlds on high;} \\
\textbf{Gold, silver, ore, and precious stones,} \\
\textbf{And bodies made of flesh and bones.} \\
\textbf{Such is our hope, our heaven, our all,} \\
\textbf{When once redeemed from Adam’s fall;} \\
\textbf{All things are ours, and we shall be,} \\
\textbf{The Lord’s to all eternity}.}[James S. White, PERGO 8.1; 1861][https://egwwritings.org/?ref=en\_PERGO.8.1&para=1471.41]


\othersQuoteNoGap{Wybieramy całą substancję - co zostaje \\
Mistyczny sekciarz zyskuje; \\
Wszystko, co każdy żąda, każdy posiądzie, \\
Nie zazdrościmy sobie szczęścia. \\
Niematerialnego Boga wybierają, \\
Dla takiego Boga nie mamy zastosowania; \\
\textbf{Niematerialne niebo i piekło,} \\
\textbf{W takim niebie nie możemy mieszkać.} \\
\textbf{Żądamy ziemi, powietrza i nieba,} \\
\textbf{I wszystkich gwiezdnych światów w górze;} \\
\textbf{Złota, srebra, rudy i drogich kamieni,} \\
\textbf{I ciał uczynionych z ciała i kości.} \\
\textbf{Taka jest nasza nadzieja, nasze niebo, nasze wszystko,} \\
\textbf{Gdy raz zostaniemy odkupieni z upadku Adama;} \\
\textbf{Wszystkie rzeczy są nasze, i będziemy,} \\
\textbf{Pańscy na całą wieczność}.}[James S. White, PERGO 8.1; 1861][https://egwwritings.org/?ref=en\_PERGO.8.1&para=1471.41]


James White compared the sentiments on the immaterial God with sectarianism, atheism, and annihilation. “\textit{Immaterial God}" is another expression for the nonentity of God. James White never received any reproof from Sister White for these views; rather, they were supported by her writings. Many assert that Sister White changed her views over time and, later, accepted the Trinity doctrine, but this is not backed up by detailed historical records. In 1905, Sister White recalls the occasion with Dr. Kellogg when, twenty years prior, he came to her with the very sentiments regarding the \emcap{personality of God} that James White and other pioneers were refuting:


James White porównał poglądy o niematerialnym Bogu z sekciarstwem, ateizmem i unicestwieniem. "\textit{Niematerialny Bóg}" jest innym wyrażeniem na nieistnienie Boga. James White nigdy nie otrzymał żadnej nagany od Siostry White za te poglądy; przeciwnie, były one wspierane przez jej pisma. Wielu twierdzi, że Siostra White zmieniła swoje poglądy z czasem i później przyjęła doktrynę o Trójcy, ale nie jest to poparte szczegółowymi zapisami historycznymi. W 1905 roku Siostra White przypomina sobie spotkanie z Dr. Kelloggiem, gdy dwadzieścia lat wcześniej przyszedł do niej z tymi samymi poglądami dotyczącymi \emcap{osobowości Boga}, które James White i inni pionierzy odpierali:


\egw{Now this subject has been kept before me for more than twenty years. My husband has been dead twenty years, and before he died, things came in. Dr. Kellogg came into my room; I was occupying one of the large rooms at the office as my home. I had two or three rooms there, and \textbf{he got a great light}; and he sat down and told what his light was: \textbf{it is just the same theories or errors, the same sophistries, that he is presenting, and did present in ‘Living Temple.’} I said, ‘Dr. Kellogg, \textbf{I have met that.}’ I met it when I first started out to travel. I met it in the North; I met it in New Hampshire. I saw the curse of its influence in Massachusetts, and \textbf{the testimonies that were given to me were right to the point that we were not to have anything of this kind to be taught in our churches}. And I talked with him. \textbf{I gave the history}—I have not time to give it to you here.\textbf{ I gave him the history of how that was treated by the Spirit of God, and how we as a people must escape the sophistries and delusions}. And it was ministers that were deceiving the people with these sophistries. \textbf{I will not tell you what they led to}—\textbf{it may have to come}; but I will not tell you now what they led to; \textbf{but I will tell you what this sophistry leads to:} \textbf{It leads to \underline{the nonentity of Christ, to the nonentity of God}, \underline{his personality}, and brings in,—what shall I call it?—a sort of \underline{manufactured theory of God and Christ}}.}[Ms70a-1905.11; 1905][https://egwwritings.org/?ref=en\_Ms70a-1905.11&para=12696.17]


\egw{Teraz ten temat był przede mną przez ponad dwadzieścia lat. Mój mąż nie żyje od dwudziestu lat, a przed jego śmiercią pojawiły się pewne rzeczy. Dr. Kellogg przyszedł do mojego pokoju; zajmowałam jeden z dużych pokoi w biurze jako mój dom. Miałam tam dwa lub trzy pokoje, i \textbf{otrzymał wielkie światło}; i usiadł i powiedział, jakie było jego światło: \textbf{to są dokładnie te same teorie czy błędy, te same sofizmaty, które przedstawia i przedstawiał w 'Living Temple.'} Powiedziałam: 'Dr. Kellogg, \textbf{już się z tym spotkałam.}' Spotkałam się z tym, gdy pierwszy raz zaczęłam podróżować. Spotkałam się z tym na Północy; spotkałam się z tym w New Hampshire. Widziałam przekleństwo jego wpływu w Massachusetts, i \textbf{świadectwa, które mi dano, mówiły wprost, że nie możemy mieć niczego takiego nauczanego w naszych kościołach}. I rozmawiałam z nim. \textbf{Przedstawiłam historię}—nie mam czasu, żeby przedstawić ją tutaj. \textbf{Przedstawiłam mu historię o tym, jak Duch Boży to traktował i jak my jako lud musimy uniknąć tych sofizmatów i złudzeń}. I to byli kaznodzieje, którzy zwodzili ludzi tymi sofizmatami. \textbf{Nie powiem wam, do czego to prowadziło}—\textbf{może to musi nadejść}; ale nie powiem wam teraz, do czego to prowadziło; \textbf{ale powiem wam, do czego prowadzą te sofizmaty:} \textbf{Prowadzą do \underline{nieistnienia Chrystusa, do nieistnienia Boga}, \underline{Jego osobowości}, i wprowadzają,—jak to nazwać?—rodzaj \underline{sfabrykowanej teorii o Bogu i Chrystusie}}.}[Ms70a-1905.11; 1905][https://egwwritings.org/?ref=en\_Ms70a-1905.11&para=12696.17]


Kellogg’s sentiment in the Living Temple regarding the \emcap{personality of God} leads to the nonentity of Christ and the nonentity of God. Why? Because his views of God claim an immaterial God. The church was faced with such sentiments in the beginning of their work. James White wrote about them in his pamphlet “\textit{The Personality of God}”, and Sister White recalled these early experiences when she and her husband combatted the error that God is an immaterial, all-prevailing spirit.


Poglądy Kellogga w The Living Temple dotyczące \emcap{osobowości Boga} prowadzą do nieistnienia Chrystusa i nieistnienia Boga. Dlaczego? Ponieważ jego poglądy o Bogu zakładają niematerialnego Boga. Kościół mierzył się z takimi poglądami na początku swojej działalności. James White pisał o nich w swojej broszurze "\textit{The Personality of God}", a Siostra White przypominała te wczesne doświadczenia, gdy wraz z mężem zwalczali błąd, że Bóg jest niematerialnym, wszechobecnym duchem.
