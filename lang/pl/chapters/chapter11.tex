\chapter{Osobowość Boga — James S. White}

W dalszej części przyjrzymy się broszurze Jamesa White’a zatytułowanej „\textit{Osobowość Boga}”. Czytając ten artykuł, zobaczymy, że James White kontynuuje tam, gdzie brat Loughborough zakończył, oraz że rozszerza i pogłębia zrozumienie stojące za pierwszym punktem \emcap{Fundamentalnych Zasad}.

Traktat Jamesa White’a był drukowany wielokrotnie, reklamowany 54 razy i przedrukowany dwukrotnie w magazynie \textit{Review and Herald}. Jego pogląd na temat \emcap{osobowości Boga} był dobrze znany i rozpowszechniony w adwentyzmie. W tej broszurze zobaczymy wyraźną krytykę idei, które Kellogg propagował w \textit{The Living Temple}.

\othersQuote{\textbf{CZŁOWIEK został stworzony na obraz Boga}. «Potem Bóg powiedział: Uczyńmy człowieka na nasz obraz według naszego podobieństwa». «Stworzył więc Bóg człowieka na swój obraz, na obraz Boga go stworzył». Rdz 1:26, 27. Zobacz także rozdz. 9:6; 1Kor 11:7. \textbf{Ci, którzy zaprzeczają osobowości Boga, mówią, że ‘obraz' nie oznacza tu \underline{fizycznej postaci}, lecz obraz moralny, i czynią to głównym punktem wyjścia do udowodnienia nieśmiertelności wszystkich ludzi}. Argument przedstawia się następująco: Po pierwsze, człowiek został stworzony na moralny obraz Boga. Po drugie, Bóg jest istotą nieśmiertelną. Po trzecie, zatem wszyscy ludzie są nieśmiertelni. Ale ten sposób rozumowania dowodziłby również, że człowiek jest wszechmocny, wszechwiedzący i wszechobecny, przyodziewając tym samym śmiertelnego człowieka we wszystkie atrybuty boskości. Spróbujmy: Po pierwsze, człowiek został stworzony na moralny obraz Boga. Po drugie, Bóg jest wszechmocny, wszechwiedzący i wszechobecny. Po trzecie, zatem człowiek jest wszechmocny, wszechwiedzący i wszechobecny. To, co dowodzi zbyt wiele, nie dowodzi niczego, dlatego stanowiska, że obraz Boga oznacza Jego moralny obraz, nie da się obronić. \textbf{Jako dowód na to, że Bóg jest osobą, przeczytajmy Jego własne słowa do Mojżesza}: «PAN mówił dalej: Oto miejsce przy mnie, staniesz na skale. A gdy będzie przechodzić moja chwała, postawię cię w rozpadlinie skalnej i \textbf{zakryję cię swoją dłonią}, aż \textbf{przejdę}. Potem \textbf{odejmę dłoń} i \textbf{ujrzysz mnie od tyłu}, ale \textbf{moje oblicze nie będzie widziane}». Wj 33:21--23. Zobacz także rozdz. 24:9--11. \textbf{Tutaj Bóg mówi Mojżeszowi, że \underline{zobaczy Jego postać}}. \textbf{Twierdzenie, że Bóg sprawił, iż Mojżeszowi wydawało się, że widzi Jego postać, podczas gdy On nie ma postaci, jest oskarżaniem Boga o dodanie do fałszu pewnego rodzaju kuglarskiego oszustwa wobec swojego sługi Mojżesza}.}[James S. White, PERGO 1.1; 1861][https://egwwritings.org/?ref=en\_PERGO.1.1&para=1471.3]

\othersQuoteNoGap{Lecz sceptyk uważa, że widzi sprzeczność między wierszem 11, który mówi, że Pan rozmawiał z Mojżeszem twarzą w twarz, a wierszem 20, który stwierdza, że Mojżesz nie mógł zobaczyć Jego oblicza. Niech Lb 12:5--8 usunie tę trudność. \textbf{«Wtedy PAN zstąpił w słupie obłoku}, stanął u wejścia do namiotu i wezwał Aarona i Miriam; a oni przyszli oboje. I powiedział do nich: Słuchajcie teraz moich słów: Jeśli będzie wśród was prorok, ja, PAN, objawię mu się w widzeniu, będę mówił do niego we śnie. Lecz nie tak jest z moim sługą Mojżeszem, który jest wierny w całym moim domu. \textbf{Z ust do ust mówię do niego, \underline{jawnie}}».}[James S. White, PERGO 2.1; 1861][https://egwwritings.org/?ref=en\_PERGO.2.1&para=1471.6]

\othersQuoteNoGap{Wielki i straszny Bóg zstąpił otoczony obłokiem chwały. \textbf{Ten obłok można było zobaczyć, ale nie oblicze, które posiada jaśniejszy blask niż tysiąc słońc}. W tych okolicznościach Mojżeszowi pozwolono zbliżyć się i \textbf{rozmawiać z Bogiem twarzą w twarz, czyli z ust do ust, \underline{jawnie}}.}[James S. White, PERGO 2.2; 1861][https://egwwritings.org/?ref=en\_PERGO.2.2&para=1471.7]

\othersQuoteNoGap{Mówi prorok Daniel: «I patrzyłem, aż trony zostały postawione, a \textbf{Odwieczny zasiadł}; jego szata była biała jak śnieg, a \textbf{włosy jego głowy jak czysta wełna}; \textbf{jego tron jak ogniste płomienie, a jego koła jak płonący ogień}». Rozdz. 7:9. «Widziałem też w nocnym widzeniu: Oto w obłokach nieba przybył ktoś podobny do Syna Człowieka, \textbf{przyszedł aż do Odwiecznego} i \textbf{przyprowadzono go przed niego}. I dano mu władzę, cześć i królestwo». Werset 13, 14.}[James S. White, PERGO 2.3; 1861][https://egwwritings.org/?ref=en\_PERGO.2.3&para=1471.8]

\othersQuoteNoGap{Oto wzniosły opis działania \textbf{dwóch osób}; a mianowicie \textbf{Boga Ojca i Jego Syna Jezusa Chrystusa}. \textbf{Jeśli zaprzeczy się ich osobowości, to w tych cytatach z Daniela nie ma żadnej wyraźnej myśli}. W związku z tym cytatem przeczytajmy stwierdzenie apostoła, że \textbf{Syn był wyrazem istoty swojego Ojca}. «Bóg, który wielokrotnie i na różne sposoby przemawiał niegdyś do ojców przez proroków, w tych ostatecznych dniach przemówił do nas przez swego Syna, którego ustanowił dziedzicem wszystkiego, przez którego też stworzył światy; \textbf{który, będąc blaskiem jego chwały i wyrazem jego istoty}». Hbr 1:1--3.}[James S. White, PERGO 3.1; 1861][https://egwwritings.org/?ref=en\_PERGO.3.1&para=1471.11]

\othersQuoteNoGap{Dodajemy tu świadectwo Chrystusa. «A Ojciec, który mnie posłał, on świadczył o mnie. Nigdy nie słyszeliście jego głosu \textbf{ani nie widzieliście jego postaci}». J 5:37. Zobacz także Flp 2:6. \textbf{Twierdzenie, że Ojciec nie ma osobowej postaci, wydaje się najbardziej wyraźnym zaprzeczeniem jasnych słów Pisma}. \\
ZARZUT. — «\textbf{\underline{Bóg jest duchem}}». J 4:24.}[James S. White, PERGO 3.2; 1861][https://egwwritings.org/?ref=en\_PERGO.3.2&para=1471.12]

\othersQuoteNoGap{ODPOWIEDŹ. — \textbf{Aniołowie też są duchami} [Ps 104:4], jednak ci, którzy odwiedzili Abrahama i Lota, położyli się, jedli i chwycili Lota za rękę. \textbf{Byli istotami duchowymi. Tak samo Bóg jest istotą duchową}.}[James S. White, PERGO 3.3; 1861][https://egwwritings.org/?ref=en\_PERGO.3.3&para=1471.13]

\othersQuoteNoGap{ZARZUT. — \textbf{Bóg jest wszędzie}. Dowód. Ps 139:1--8. \textbf{Jest On tak samo obecny w każdym miejscu, jak w jakimkolwiek pojedynczym miejscu}.}[James S. White, PERGO 3.4; 1861][https://egwwritings.org/?ref=en\_PERGO.3.4&para=1471.14]

\othersQuoteNoGap{ODPOWIEDŹ. — 1. \textbf{Bóg jest wszędzie poprzez swoją wszechwiedzę}, jak to widać w przytoczonych powyżej słowach Dawida. Wersety 1--6. «PANIE, \textbf{przeniknąłeś mnie i znasz mnie}. \textbf{Wiesz}, kiedy siedzę i wstaję, z daleka \textbf{znasz} moje myśli. Otaczasz moją ścieżkę i spoczynek, wszystkie moje drogi są ci \textbf{znane}. Zanim na moim języku pojawi się słowo, ty, PANIE, już je \textbf{znasz}. Otaczasz mnie z tyłu i z przodu i położyłeś na mnie twoją rękę. Zbyt cudowna jest dla mnie \textbf{twoja wiedza}; jest wzniosła, nie mogę jej pojąć».}[James S. White, PERGO 3.5; 1861][https://egwwritings.org/?ref=en\_PERGO.3.5&para=1471.15]

\othersQuoteNoGap{2. \textbf{Bóg jest \underline{wszędzie poprzez swojego Ducha}, \underline{który jest Jego przedstawicielem} i objawia się gdziekolwiek On zechce}, jak to widać w tych samych słowach, na które powołuje się oponent. Wersety 7--10. «\textbf{Dokąd ujdę przed \underline{twoim duchem}}? \textbf{Dokąd ucieknę przed \underline{twoim obliczem}}? Jeśli wstąpię do nieba, jesteś tam; jeśli przygotuję sobie posłanie w piekle, tam też jesteś. Gdybym wziął skrzydła zorzy porannej, aby zamieszkać na krańcu morza, i tam twoja ręka prowadziłaby mnie i twoja prawica by mnie podtrzymała».}[James S. White, PERGO 4.1; 1861][https://egwwritings.org/?ref=en\_PERGO.4.1&para=1471.18]

\othersQuoteNoGap{\textbf{Bóg jest w niebie}. Tego uczy nas Modlitwa Pańska. «\textbf{Ojcze nasz, który jesteś w niebie}». Mt 6:9; Łk 11:2. \textbf{Ale jeśli Bóg jest tak samo obecny w każdym miejscu, jak w jakimkolwiek pojedynczym miejscu, to niebo również jest tak samo obecne w każdym miejscu, jak w jakimkolwiek pojedynczym miejscu, a idea pójścia do nieba jest całkowitym błędem}. Wszyscy jesteśmy w niebie; a Modlitwa Pańska, według tej mglistej teologii, oznacza po prostu: Ojcze nasz, \textbf{który jesteś wszędzie}, niech będzie uświęcone twoje imię. Niech przyjdzie twoje królestwo, niech się dzieje twoja wola na ziemi, \textbf{tak jak wszędzie}.}[James S. White, PERGO 4.2; 1861][https://egwwritings.org/?ref=en\_PERGO.4.2&para=1471.19]

\othersQuoteNoGap{Ponadto czytelnicy Biblii dotąd wierzyli, że Henoch i Eliasz zostali rzeczywiście wzięci \textbf{do Boga w niebie}. \textbf{Lecz jeśli Bóg i niebo są tak samo obecne w każdym miejscu jak w jakimkolwiek pojedynczym miejscu, to jest to całkowitym błędem}. Nie zostali oni przeniesieni. A cała ta historia o ognistym rydwanie, ognistych koniach i towarzyszącym wichrze, która zabrała Eliasza do nieba, była niepotrzebnym widowiskiem. Oni po prostu wyparowali, a mglista para przeszła przez cały wszechświat. To wszystko, co umysł może pojąć o Henochu i Eliaszu, \textbf{przyjmując, że Bóg i niebo nie są bardziej obecni w jakimś pojedynczym miejscu niż wszędzie}. Ale o Eliaszu jest napisane, że «\textbf{wstąpił} wśród wichuru \textbf{do nieba}». 2Krl 2:11. A o Henochu jest napisane, że «chodził z Bogiem, a potem już go nie było, bo Bóg go zabrał». Rdz 5:24.}[James S. White, PERGO 4.3; 1861][https://egwwritings.org/?ref=en\_PERGO.4.3&para=1471.20]

\othersQuoteNoGap{\textbf{Jest powiedziane, że Jezus jest po prawicy Majestatu na wysokościach}. Hbr 1:3. «A gdy Pan przestał do nich mówić, \textbf{został \underline{wzięty do nieba}} \textbf{i zasiadł po prawicy Boga}». Mk 16:19. \textbf{Ale jeśli niebo jest wszędzie i Bóg jest wszędzie, to wniebowstąpienie Chrystusa do prawicy Ojca oznacza po prostu, że udał się wszędzie}! Został uniesiony tylko tam, gdzie obłok ukrył go przed wzrokiem uczniów, a potem wyparował i udał się wszędzie! Tak więc zamiast ukochanego Jezusa, tak pięknie opisanego w obu Testamentach, mamy tylko pewien rodzaj oparów rozproszonych po całym wszechświecie. A zgodnie z tą rozproszoną teologią powtórne przyjście Chrystusa, czyli Jego powrót, byłby kondensacją tych oparów w jakimś miejscu, powiedzmy na Górze Oliwnej! \textbf{Chrystus powstał z martwych w fizycznej postaci}. «Nie ma go tu», powiedział anioł, «powstał bowiem, jak powiedział». Mt 28:6.}[James S. White, PERGO 5.1; 1861][https://egwwritings.org/?ref=en\_PERGO.5.1&para=1471.23]

\othersQuoteNoGap{«Kiedy szły przekazać to jego uczniom, nagle Jezus wyszedł im na spotkanie i powiedział: Witajcie! A one podeszły, \textbf{objęły go za nogi} i oddały mu pokłon». Werset 9.}[James S. White, PERGO 5.2; 1861][https://egwwritings.org/?ref=en\_PERGO.5.2&para=1471.24]

\othersQuoteNoGap{«\textbf{Popatrzcie na moje ręce i nogi}», powiedział Jezus do tych, którzy wątpili w jego zmartwychwstanie, «że to jestem ja. \textbf{Dotknijcie mnie i zobaczcie, \underline{bo duch nie ma ciała ani kości}, jak widzicie, że ja mam}. Kiedy to powiedział, \textbf{pokazał im ręce i nogi}. Lecz gdy oni z radości jeszcze nie wierzyli i dziwili się, zapytał ich: Macie tu coś do jedzenia? I podali mu kawałek pieczonej ryby i plaster miodu. A on wziął i jadł przy nich». Łk 24:39--43.}[James S. White, PERGO 5.3; 1861][https://egwwritings.org/?ref=en\_PERGO.5.3&para=1471.25]

% odtąd tłumaczenie bezpośrednie cytatów z Biblii
\othersQuoteNoGap{Po tym, jak Jezus przemówił do swoich uczniów na Górze Oliwnej, \textbf{został uniesiony od nich}, a obłok zabrał go sprzed ich oczu. «A gdy patrzyli uważnie \textbf{w niebo, jak wstępował}, oto stanęli przy nich dwaj mężowie w białych szatach, którzy powiedzieli: Mężowie z Galilei, dlaczego stoicie, patrząc w niebo? Ten Jezus, który \textbf{został od was wzięty w górę do nieba}, przyjdzie tak samo, jak widzieliście go \textbf{wstępującego do nieba}». Dz 1:9--11. J. W.}[James S. White, PERGO 6.1; 1861][https://egwwritings.org/?ref=en\_PERGO.6.1&para=1471.27]

James White zwalcza ideę, że Bóg jest tylko duchem i jako taki jest obecny \others{tak samo w każdym miejscu, jak w jakimkolwiek pojedynczym miejscu}. Przedstawia jasne i jednoznaczne świadectwo z Pisma Świętego, że Bóg jest osobową istotą; te same poglądy widzimy w pismach Ellen White.

% to check: sądach
\egw{Potężna moc, która działa w całej naturze i podtrzymuje wszystkie rzeczy, nie jest, jak twierdzą niektórzy naukowcy, \textbf{jedynie wszechobecną zasadą}, energią napędową. \textbf{\underline{Bóg jest duchem, ale jest osobową istotą}}, \textbf{gdyż człowiek został stworzony na Jego obraz}. \textbf{Jako \underline{osobowa istota}}, Bóg objawił się w swoim Synu. Jezus, blask chwały Ojca, «i \textbf{\underline{obraz Jego osoby}}» (Hbr 1:3), na ziemię przyszedł w postaci człowieka. Jako \textbf{osobowy Zbawiciel} przyszedł na świat. Jako \textbf{osobowy Zbawiciel wstąpił \underline{na wysokości}}. Jako \textbf{osobowy Zbawiciel wstawia się \underline{w niebiańskich sądach}}. \textbf{Przed tronem Bożym} w naszym imieniu służy «Podobny do Syna człowieczego». Dn 7:13}[Ed 131.5; 1903][https://egwwritings.org/?ref=en\_Ed.131.5&para=29.632]

Ellen White i pionierzy adwentyzmu rozróżniali między terminem ‘\textit{duch}’ a ‘\textit{istota}’. Bóg jest osobową istotą, nie tylko duchem. On nie jest \others{tak samo obecny w każdym miejscu, jak w jakimkolwiek jednym miejscu}, ale jest \others{w jednym miejscu bardziej niż w innym}[John. N. Loughborough, “Is God a Person?” The Adventist Review and Sabbath Herald, September 18, 1855][https://documents.adventistarchives.org/Periodicals/RH/RH18550918-V07-06.pdf]. Jest w niebie, w swojej świątyni, siedząc na swoim tronie — osobiście — i jest wszędzie obecny przez swojego przedstawiciela, Ducha Świętego.

Oto kilka innych cytatów z siostry White, które są w harmonii z poglądami pionierów na temat \emcap{osobowości Boga}:

\egw{On \normaltext{[Jezus]} nauczał, że Bóg nagradza sprawiedliwych i karze przestępców. \textbf{Nie był On niematerialnym duchem}, ale żywym władcą wszechświata. \textbf{Ten łaskawy Ojciec} nieustannie działał dla dobra człowieka i troszczył się o wszystko, co go dotyczy...}[3SP 47.1; 1878][https://egwwritings.org/?ref=en\_3SP.47.1&para=142.195]

\egw{\textbf{Biblia ukazuje nam \underline{Boga w Jego wysokim i świętym miejscu}}, nie w stanie bezczynności, nie w ciszy i samotności, ale otacza Go dziesięć tysięcy razy dziesięć tysięcy i tysiące tysięcy świętych istot, wszystkie gotowe wypełniać Jego wolę. \textbf{Poprzez tych posłańców aktywnie komunikuje się z każdą częścią swojego królestwa}. \textbf{\underline{Przez swojego Ducha jest wszędzie obecny}}. \textbf{Poprzez działanie swojego Ducha i swoich aniołów} służy dzieciom ludzkim}[MH 417.2; 1905][https://egwwritings.org/?ref=en\_MH.417.2&para=135.2136]

\egw{Wielkość Boga jest dla nas niepojęta. «\textbf{Tron Pana jest w niebie}» (Ps 11:4); \textbf{\underline{jednak przez swojego Ducha jest On wszędzie obecny}}. \textbf{Ma On dokładną wiedzę} o wszystkich dziełach swoich rąk i osobiście się nimi interesuje}[Ed 132.2; 1903][https://egwwritings.org/?ref=en\_Ed.132.2&para=29.636]

\egw{Przez Jezusa Chrystusa, \textbf{Bóg—nie wonności, \underline{nie coś nieuchwytnego}, \underline{ale osobowy Bóg}}—stworzył człowieka i obdarzył go inteligencją i mocą.}

Kontynuując w broszurze Jamesa White'a, czytamy jego ostrą krytykę pojęcia niematerialnego Boga. Wcześniej przypomnijmy krótko argument dr. Kellogga, że\others{\textbf{\underline{Dyskusje dotyczące formy Boga są całkowicie bezużyteczne}}} ponieważ Bóg jest\others{\textbf{daleko poza naszym pojmowaniem }\textbf{\underline{tak jak granice przestrzeni i czasu}}}. Wierzył on, że osoba Boga nie jest ograniczona do jednego miejsca, ponieważ jest On\others{tak samo obecny w każdym miejscu jak w jakimkolwiek jednym miejscu} \footnote{W The Living Temple, dr Kellogg sprzeciwiał się temu, że Bóg nie może być wszędzie obecny naraz: "\textit{Ktoś mówi}, 'Bóg może być obecny przez swojego Ducha lub przez swoją moc, ale z pewnością sam Bóg \textit{nie może być obecny wszędzie naraz}.' Odpowiadamy: Jak można oddzielić moc od źródła mocy? Gdzie Duch Boży działa, gdzie moc Boża się objawia, tam sam Bóg jest \textit{rzeczywiście i prawdziwie obecny}…"}

\othersQuote{NIEMATERIALNOŚĆ}

\othersQuoteNoGap{\textbf{TO jest tylko inne określenie nicości}. \textbf{Jest to zaprzeczenie wszystkich} \textbf{rzeczy i} \textbf{\underline{istot}} - całego istnienia. Nie ma ani jednego dowodu potwierdzającego jej istnienie. Nie ma sposobu, by mogła się objawić jakiejkolwiek inteligencji w niebie czy na ziemi. \textbf{Ani Bóg, ani aniołowie, ani ludzie nie mogliby pojąć takiej substancji, istoty czy rzeczy}. \textbf{Nie posiada ona żadnej właściwości ani mocy, przez którą \underline{mogłaby się objawić jakiejkolwiek inteligentnej istocie} we wszechświecie}. Rozum i analogia nigdy jej nie badają, ani nawet nie pojmują. \textbf{Objawienie nigdy jej nie ujawnia, ani żaden z naszych zmysłów nie świadczy o jej istnieniu}. \textbf{Nie można jej zobaczyć, poczuć, usłyszeć, posmakować ani powąchać, nawet przy najsilniejszych organach czy najbardziej wyczulonych zmysłach}. Nie jest ani płynna, ani stała, miękka ani twarda - nie może się ani rozszerzać, ani kurczyć. Krótko mówiąc, nie może wywierać żadnego wpływu - nie może ani działać, ani być przedmiotem działania. A nawet jeśli istnieje, nie może być w żaden sposób użyteczna. Nie posiada ani jednej pożądanej właściwości, zdolności czy zastosowania, jednak, co dziwne, \textbf{niematerialność jest Bogiem współczesnego chrześcijanina}, \textbf{jego oczekiwanym niebem}, \textbf{jego nieśmiertelnym ja} - \textbf{jego wszystkim}!}

\othersQuoteNoGap{\textbf{O sekciarstwo! O ateizm!! O unicestwienie!!!} \textbf{Kto może dostrzec subtelne różnice między jednym a drugim?} Wydają się podobne, różnią się tylko nazwą. \textbf{Ateista nie ma Boga. \underline{Sekciarze mają Boga bez ciała i części}.} Kto może zdefiniować różnicę? Z naszej strony nie dostrzegamy różnicy nawet o włos; \textbf{obaj twierdzą, że są negacją wszystkich rzeczy, które istnieją} — i obaj są równie bezsilni i nieznani.}[James S. White, PERGO 6.3; 1861][https://egwwritings.org/?ref=en\_PERGO.6.3&para=1471.30]

\othersQuoteNoGap{\textbf{Ateista nie ma życia po śmierci ani świadomego istnienia po grobie. Sekciarz ma, \underline{ale jest ono niematerialne, jak jego Bóg; i bez ciała czy części ciała}. Tutaj znowu obaj są negatywni i obaj dochodzą do tego samego punktu}. Ich wiara i nadzieja sprowadzają się do tego samego; tylko wyrażone są innymi terminami.}[James S. White, PERGO 7.1; 1861][https://egwwritings.org/?ref=en\_PERGO.7.1&para=1471.33]

\othersQuoteNoGap{Ponownie, \textbf{ateista nie ma nieba w wieczności}. \textbf{Sekciarz ma, ale jest ono \underline{niematerialne we wszystkich swoich właściwościach}, i dlatego jest negacją wszelkich bogactw i substancji}. Tutaj znowu są równi i dochodzą do tego samego punktu.}[James S. White, PERGO 7.2; 1861][https://egwwritings.org/?ref=en\_PERGO.7.2&para=1471.34]

\othersQuoteNoGap{Ponieważ nie zazdrościmy im posiadania wszystkiego, co sobie roszczą, pozwolimy im teraz się tym cieszyć w spokoju i bez przeszkód i przejdziemy do zbadania części, którą może nadal cieszyć się znienawidzony materialista.}[James S. White, PERGO 7.3; 1861][https://egwwritings.org/?ref=en\_PERGO.7.3&para=1471.35]

\othersQuoteNoGap{\textbf{Czym jest Bóg? Jest On materialną, uporządkowaną duchową istotą, \underline{posiadającą zarówno ciało, jak i części ciała}. Człowiek jest na Jego obraz.}}[James S. White, PERGO 7.4; 1861][https://egwwritings.org/?ref=en\_PERGO.7.4&para=1471.36]

\othersQuoteNoGap{\textbf{Czym jest Jezus Chrystus? Jest Synem Bożym i jest \underline{jak jego Ojciec}, to znaczy «jasnością chwały Ojca i dokładnym obrazem jego osoby». \underline{Jest on materialną istotą duchową, z ciałem, częściami ciała} i uczuciami; posiadającą nieśmiertelne ciało i nieśmiertelne kości}.}[James S. White, PERGO 7.5; 1861][https://egwwritings.org/?ref=en\_PERGO.7.5&para=1471.37]

\othersQuoteNoGap{\textbf{Czym są ludzie?} Są potomstwem Adama. \textbf{Są zdolni do otrzymywania wiedzy i wywyższenia do takiego stopnia, aby byli \underline{wskrzeszeni z martwych z ciałem jak ciało Jezusa Chrystusa}, \underline{i posiadali nieśmiertelne ciało i kości}}. W ten sposób doprowadzeni do doskonałości, posiądą \textbf{materialny wszechświat}, to jest ziemię, jako ich «wieczne dziedzictwo». Mając przed sobą te nadzieje i perspektywy, mówimy chrześcijańskiemu światu, który trzyma się niematerialności, że mogą zachować swojego Boga — swoje życie — swoje niebo i wszystko inne. Roszczą sobie prawo tylko do tego, co my odrzucamy; a my rościmy sobie prawo tylko do tego, co oni odrzucają. \textbf{Dlatego nie ma podstaw do kłótni czy sporu między nami}.}[James S. White, PERGO 7.6; 1861][https://egwwritings.org/?ref=en\_PERGO.7.6&para=1471.38]

\othersQuoteNoGap{Dla nas materia — a nie-obiekty \\
Mogą nam wziąć mistyczne sekty; \\
Kto czego chce, niech przy tym trwa \\
I niezmąconą radość ma. \\
Oni chcą Boga mieć bez ciała, \\
Lecz dla nas taki Bóg nie działa; \\
\textbf{Też niby-piekła nam nie trzeba,} \\
\textbf{Chcemy materialnego nieba.} \\
\textbf{My ziemię, niebo, gwiazdy znamy,} \\
\textbf{Powietrze, którym oddychamy;} \\
\textbf{Złoto i srebro, i kosztowności,} \\
\textbf{A także ciała z krwi i kości.} \\
\textbf{Nadzieja nasza jest wciąż ta sama,} \\
\textbf{Po odjęciu winy Adama} \\
\textbf{Wszystko już będzie dla człowieka,} \\
\textbf{A on dla Pana na wiek wieka}.}[James S. White, PERGO 8.1; 1861][https://egwwritings.org/?ref=en\_PERGO.8.1&para=1471.41]

James White porównał poglądy o niematerialnym Bogu z sekciarstwem, ateizmem i anihilacjonizmem. „\textit{Niematerialny Bóg}” jest innym wyrażeniem na nieistnienie Boga. James White nigdy nie otrzymał żadnej nagany od siostry White za te poglądy; przeciwnie, były one wspierane przez jej pisma. Wielu twierdzi, że siostra White zmieniła swoje poglądy z czasem i później przyjęła doktrynę o Trójcy, ale nie jest to poparte szczegółowymi zapisami historycznymi. W 1905 roku siostra White przypomina sobie spotkanie z dr. Kelloggiem, gdy dwadzieścia lat wcześniej przyszedł do niej z tymi samymi poglądami dotyczącymi \emcap{osobowości Boga}, które James White i inni pionierzy obalali:

\egw{Zatem ten temat był trzymany przede mną przez ponad dwadzieścia lat. Mój mąż nie żyje od dwudziestu lat, a przed jego śmiercią pojawiły się pewne rzeczy. Dr Kellogg przyszedł do mojego pokoju; zajmowałam jeden z dużych pokoi w biurze jako mój dom. Miałam tam dwa lub trzy pokoje, i \textbf{otrzymał wielkie światło}; i usiadł i powiedział, jakie było jego światło: \textbf{to są dokładnie te same teorie czy błędy, te same sofizmaty, które przedstawia i przedstawiał w «The Living Temple».} Powiedziałam: «Dr. Kelloggu, \textbf{już się z tym spotkałam}». Spotkałam się z tym, gdy pierwszy raz zaczęłam podróżować. Spotkałam się z tym na Północy; spotkałam się z tym w New Hampshire. Widziałam przekleństwo tego wpływu w Massachusetts i \textbf{świadectwa, które mi dano, mówiły wprost, że nie możemy czegoś takiego nauczać w naszych kościołach}. I rozmawiałam z nim. \textbf{Przedstawiłam historię} — nie mam czasu, żeby przedstawić ją tutaj. \textbf{Przedstawiłam mu historię o tym, jak Duch Boży to traktował i jak my jako lud musimy uniknąć tych sofizmatów i złudzeń}. I to kaznodzieje zwodzili ludzi tymi sofizmatami. \textbf{Nie powiem wam, do czego to prowadziło} — \textbf{może to musi nadejść}; ale nie powiem wam teraz, do czego to prowadziło; \textbf{ale powiem wam, do czego prowadzą te sofizmaty:} \textbf{prowadzą do \underline{niebytu Chrystusa, do niebytu Boga}, \underline{Jego osobowości}, i wprowadzają — jak to nazwać — pewien rodzaj \underline{sfabrykowanej teorii o Bogu i Chrystusie}}}[Ms70a-1905.11; 1905][https://egwwritings.org/?ref=en\_Ms70a-1905.11&para=12696.17]

Poglądy Kellogga w \textit{The Living Temple} dotyczące \emcap{osobowości Boga} prowadzą do niebytu Chrystusa i niebytu Boga. Dlaczego? Ponieważ jego poglądy o Bogu zakładają niematerialnego Boga. Kościół mierzył się z takimi poglądami na początku swojej działalności. James White pisał o nich w swojej broszurze \textit{The Personality of God}, a siostra White przypominała te wczesne doświadczenia, gdy wraz z mężem zwalczali błąd, że Bóg jest niematerialnym, wszechogarniającym duchem.
