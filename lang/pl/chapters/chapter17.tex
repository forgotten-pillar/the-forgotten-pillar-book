\qrchapter{https://forgottenpillar.com/rsc/pl-fp-chapter17}{Odpowiedź na trynitarne poglądy Kellogga}

Jeśli spojrzymy na kryzys Kellogga przez pryzmat \emcap{osobowości Boga} i \emcap{Fundamentalnych Zasad}, cytaty siostry White nieuchronnie jaśnieją w nowym świetle. W tym świetle widzimy konflikt między prawdą, którą otrzymaliśmy na początku, dotyczącą \emcap{osobowości Boga}, a doktryną o Trójcy. Aby uniknąć rozbieżności, w interesie obrony doktryny o Trójcy uczeni zawsze nadmiernie podkreślają panteistyczną stronę problemu.

Chcielibyśmy zakwestionować tę tendencję do nadmiernego podkreślania panteistycznej strony sporu Kellogga. Siostra White ogólnie pisała zapobiegawczą prawdę; podchodziła do błędu poprzez wywyższanie prawdy. Dlatego napisała tak wiele o \emcap{osobowości Boga}. W większości jej cytatów na ten temat widzimy, jak rozwiewa trynitarny błąd, a nie błąd panteistyczny. Czytamy jeden z takich przykładów, gdzie ugruntowuje prawdę o \emcap{osobowości Boga}, odwołując się do siedemnastego rozdziału Ew. Jana.

\egw{\textbf{Osobowość Ojca i Syna, a także jedność, która między Nimi istnieje, są przedstawione w siedemnastym rozdziale Ew. Jana}, w modlitwie Chrystusa za Jego uczniów:}[MH 421.7; 1905][https://egwwritings.org/?ref=en\_MH.421.7&para=135.2173]

Jest wiele przypadków, gdzie siostra White cytuje Ew. Jana 17 w odniesieniu do kryzysu Kellogga. Ci, którzy twierdzą, że kryzys Kellogga dotyczył wyłącznie panteizmu, powinni zapytać, jak Ew. Jana 17 odnosi się do Boga w przyrodzie. I nie tylko Ew. Jana 17, ale także rozdziały 13--16. W swoim liście do Kellogga napisała:

\egw{\textbf{\underline{...przestudiuj rozdział trzynasty, czternasty, piętnasty, szesnasty i siedemnasty Ew. Jana}. Słowa tych rozdziałów wyjaśniają się same. «To jest życie wieczne», oświadczył Chrystus, «aby znali \underline{Ciebie, jedynego prawdziwego Boga}, i Jezusa Chrystusa, którego posłałeś». \underline{Te słowa jasno mówią o osobowości Boga i Jego Syna.} \underline{Osobowość jednego nie eliminuje konieczności osobowości drugiego}}}[Lt232-1903.48, 1903][https://egwwritings.org/?ref=en\_Lt232-1903.48&para=10197.57]

We wspomnianych rozdziałach Ew. Jana, Jan nie odniósł się do niczego związanego z Bogiem w przyrodzie. Treść tych rozdziałów obejmuje to, kto jest jedynym prawdziwym Bogiem, jak Ojciec i Syn są jedno, ich prawdziwą relację i jak Jezus może być wszędzie obecny, a jednak wstąpi do Ojca.

\egw{Jezus powiedział do Żydów: «Ojciec mój aż dotąd działa i Ja działam.... Syn nie może nic czynić sam z siebie, tylko to, co widzi, że Ojciec czyni; co bowiem On czyni, to samo i Syn czyni. Ojciec bowiem miłuje Syna i ukazuje Mu wszystko, co sam czyni». J 5:17-20.}[8T 268.4, 1904][https://egwwritings.org/?ref=en\_8T.268.4&para=112.1557]

\egwnogap{\textbf{Tutaj ponownie ukazana jest \underline{osobowość Ojca i Syna}, pokazująca jedność, która między nimi istnieje}.}[8T 269.1; 1904][https://egwwritings.org/?ref=en\_8T.269.1&para=112.1560]

\egwnogap{\textbf{Ta jedność jest wyrażona również w \underline{siedemnastym rozdziale Ew. Jana}}, w modlitwie Chrystusa za Jego uczniów:}[8T 269.2; 1904][https://egwwritings.org/?ref=en\_8T.269.2&para=112.1561]

\egwnogap{«Nie tylko za nimi proszę, ale i za tymi, którzy uwierzą we Mnie przez ich słowo; aby wszyscy byli jedno; \textbf{jak Ty, Ojcze, we Mnie, a Ja w Tobie, aby i oni byli jedno w Nas}; aby świat uwierzył, że Ty Mnie posłałeś. A \textbf{chwałę, którą Mi dałeś}, dałem im, \textbf{aby byli jedno, jak My jesteśmy jedno; Ja w nich, a Ty we Mnie, aby stali się doskonali w jedności}, aby świat poznał, że Ty Mnie posłałeś i że umiłowałeś ich, jak i Mnie umiłowałeś». J 17:20--23.}[8T 269.3; 1904][https://egwwritings.org/?ref=en\_8T.269.3&para=112.1562]

\egwnogap{Cudowne stwierdzenie! \textbf{Jedność, która istnieje między Chrystusem a Jego uczniami, \underline{nie niszczy osobowości żadnej ze stron}. Są jedno w celu, w umyśle, w charakterze, ale \underline{nie w osobie}. W ten sposób Bóg i Chrystus są jedno}}[8T 269.4; 1904][https://egwwritings.org/?ref=en\_8T.269.4&para=112.1563]

\egwnogap{\textbf{Relacja między Ojcem a Synem oraz osobowość ich obu są również wyjaśnione w tym fragmencie Pisma}:}[8T 269.5; 1904][https://egwwritings.org/?ref=en\_8T.269.5&para=112.1564]

\egwnodotnogap{Tak mówi \textbf{Jehowa Zastępów}:} \\
\egwnodot{Oto \textbf{mąż}, którego imię to \textbf{Latorośl}:} \\
\egwnodot{I wyrośnie ze swego miejsca;} \\
\egwnodot{\textbf{I zbuduje świątynię Jehowy;... }} \\
\egwnodot{\textbf{I będzie niósł chwałę,}} \\
\egwnodot{\textbf{I zasiądzie i będzie panował na swoim tronie;}} \\
\egwnodot{\textbf{I będzie kapłanem na swoim tronie;}} \\
\egw{\textbf{I \underline{rada pokoju będzie między Nimi oboma}}»}[8T 269.6; 1904][https://egwwritings.org/?ref=en\_8T.269.6&para=112.1565]

Wspomniane rozdziały Ew. Jana dotyczą \emcap{osobowości Boga}, która została wyrażona w pierwszych dwóch punktach \emcap{Fundamentalnych Zasad}. Jaki błąd zwalczała siostra White, gdy odwoływała się do wersetów o tym, jak Ojciec jest jedynym prawdziwym Bogiem i jak Ojciec i Syn nie są jedną osobą? Panteizm? Z pewnością nie; ale najprawdopodobniej trynitarne poglądy, czyli wiarę w Boga jednego-w-trzech, czy też trzech-w-jednym.

Brat J. N. Loughborough, jeden z pierwszych braci, którzy pisali o \emcap{osobowości Boga}, napisał następujący komentarz do 17. rozdziału Ew. Jana:

\others{\textbf{\underline{Siedemnasty rozdział Ew. Jana sam w sobie wystarczy, aby obalić doktrynę o Trójcy}}. \textbf{...\underline{Przeczytajcie siedemnasty rozdział Ewangelii Jana i zobaczcie, czy nie obala on całkowicie doktryny o Trójcy}}.}[John N. Loughborough, The Advent Review, and Sabbath Herald, 5 listopada 1861, str. 184.10][https://egwwritings.org/?ref=en\_ARSH.November.5.1861.p.184.1&para=1685.6615]

Zapobiegawcze pisma siostry White wspierające prawdę o \emcap{osobowości Boga} i Jego obecności są takie same jak innych pionierów adwentyzmu. Jeśli pionierzy adwentyzmu obalali doktrynę o Trójcy poprzez wywyższanie prawdy o \emcap{osobowości Boga} i obecności Boga, co sprawia, że myślimy, że Ellen White nie robiła tego samego, gdy podniesiona została teologiczna strona kwestii Trójcy? Mówiąc to, nie zaprzeczamy panteistycznej stronie sporu Kellogga, ale przez nadmierne jej podkreślanie, nie opisuje się dokładnie jego prawdziwego problemu. Właściwe zrozumienie sporu Kellogga można osiągnąć, skupiając się jedynie przede wszystkim na prawdzie, którą siostra White wywyższała, a nie na błędzie, czy to panteizmie, czy Trójcy. Tą prawdą, którą siostra White wywyższała, była prawda o \emcap{osobowości Boga} i tym, gdzie jest Jego obecność. Jest to wyrażone w pierwszym punkcie \emcap{Fundamentalnych Zasad}, które były oficjalnym streszczeniem i przedstawieniem wierzeń Adwentystów Dnia Siódmego w czasach Ellen White; prawdą, którą my jako Kościół \egwinline{otrzymaliśmy i słyszeliśmy, i głosiliśmy}[Ms124-1905.12; 1905][https://egwwritings.org/?ref=en\_Ms124-1905.12&para=9099.18] na początku.

\egw{\textbf{Błagam każdego, aby był jasny i stanowczy w sprawie pewnych prawd, które otrzymaliśmy i słyszeliśmy, i głosiliśmy. Stwierdzenia Słowa Bożego są jasne. Osadźcie mocno stopy na \underline{platformie wiecznej prawdy}. \underline{Odrzućcie każdą formę błędu}, nawet jeśli \underline{jest pokryta pozorem rzeczywistości, która zaprzecza osobowości Boga lub Chrystusa}}}[Ms124-1905.12; 1905][https://egwwritings.org/?ref=en\_Ms124-1905.12&para=9099.18]

Ostrzeżenie z poprzednich cytatów nie straciło na znaczeniu z biegiem czasu. Jest ono dziś nawet bardziej aktualne. Powinniśmy \egwinline{odrzucić każdą formę błędu, nawet jeśli jest ona pokryta pozorem prawdy, która zaprzecza osobowości Boga lub Chrystusa}.

\begin{titledpoem}

    \stanza{
        To światło prawdy, tak przejrzyste, \\
        W kryzysie już nieoczywiste. \\
        Panteizm w szeregi się wkrada, \\
        A prawda o Bogu upada.
    }

    \stanza{
        Lecz Ellen White w swojej twórczości \\
        Pisała o osobowości. \\
        O Trójcy nigdy nie mówiła \\
        I się z Kellogiem nie zgodziła.
    }

    \stanza{
        Trafiła za to w samo sedno: \\
        „Ojciec i Syn stanowią jedno”. \\
        O jedność celu tutaj chodzi, \\
        Jak w Ewangelii Jan dowodzi.
    }

    \stanza{
        Za Ellen White pionierzy stali, \\
        Osobę Boga wyznawali. \\
        Loughborough głosił przeciw triadzie \\
        W spisanym w „Review” swym wykładzie.
    }

    \stanza{
        Te Punkty Wiary, skarb nasz czysty, \\
        Wskazują w sposób tak oczywisty, \\
        Że błąd się kryje w Trójcy dogmacie, \\
        Lecz znając Boga, się wyzwalacie.
    }

    \stanza{
        Odrzućmy zatem błędy zawiłe, \\
        Wybierzmy Boga, a da nam siłę. \\
        Bóg jest osobą, ma kształt i formę — \\
        Na tym oprzyjmy naszą platformę.
    }

\end{titledpoem}

