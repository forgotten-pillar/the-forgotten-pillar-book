\chapter{Korekta „The Living Temple”}

W \textit{Testimonies to the Church Containing Letters to Physicians and Ministers Instruction to Seventh-Day Adventists}, w rozdziale dziesiątym, \textit{Fundament Naszej Wiary}, Bóg przekazał cenne lekcje na temat rozwoju i konsekwencji teorii Kellogga. Szersze i głębsze znaczenie tych cytatów można zrozumieć, gdy znamy ich historyczny kontekst. Przyjrzyjmy się najpierw pokrótce historycznemu kontekstowi książki Kellogga, \textit{The Living Temple}.

W wyniku opatrzności Bóg dał do zrozumienia, że książka \textit{The Living Temple} nie powinna być wydrukowana. Jednym z takich wydarzeń było spalenie budynku drukarni w Battle Creek właśnie w noc przed planowanym drukiem. Ostatecznie książka została wydrukowana gdzie indziej; wywołała wielki kryzys w Kościele Adwentystów Dnia Siódmego. 7 października 1903 roku w Waszyngtonie odbyło się doroczne spotkanie konferencji. Obecnych było wielu przywódców Kościoła Adwentystów Dnia Siódmego, w tym dr Kellogg i jego zwolennicy. Wokół tej książki toczyła się poważna kontrowersja i konflikt był nieunikniony. Na szczęście, w momencie narastającego konfliktu, do rady dostarczono list od siostry White. W niedzielę list dotarł do uszu zgromadzonych, na co odpowiedziano wieloma „amen” i „alleluja”. Był to bardzo napięty i poruszający poranek dla kościoła, który był na krawędzi rozłamu - by wreszcie otrzymać konkretne wskazówki od posłanniczki Pana:

\egw{Mam coś do powiedzenia naszym nauczycielom w odniesieniu do \textbf{nowej książki Living Temple}. \textbf{Bądźcie ostrożni w popieraniu \underline{poglądów tej książki dotyczących osobowości Boga}}. Zgodnie z tym, jak Pan przedstawia mi te sprawy, \textbf{te poglądy nie mają poparcia u Boga}. \textbf{Są one pułapką, którą wróg przygotował na te ostatnie dni}. Myślałam, że zostanie to z pewnością dostrzeżone i że nie będzie konieczne, abym cokolwiek o tym mówiła. \textbf{Ale ponieważ pojawia się twierdzenie, że nauki tej książki mogą być popierane wypowiedziami z moich pism, jestem zmuszona zaprzeczyć tym zarzutom}. W tej książce mogą być wyrażenia i poglądy, które są w harmonii z moimi pismami. I w moich pismach może być wiele wypowiedzi, które wyrwane z kontekstu i interpretowane zgodnie z umysłem autora \textit{The Living Temple}, mogłyby wydawać się być w harmonii z naukami tej książki. \textbf{Może to dawać pozorne poparcie twierdzeniu, że poglądy w Living Temple są w harmonii z moimi pismami}. \textbf{Ale nie daj Boże, aby ta opinia przeważyła}.}[Lt211-1903.1; 1903][https://egwwritings.org/?ref=en\_Lt211-1903.1]

Wielokrotnie Siostra White stwierdzała, że prawdziwym problemem książki były poglądy\egwinline{\textbf{dotyczące osobowości Boga}}. Te poglądy nie są poparte stwierdzeniami z pism Ellen White i te właśnie poglądy\egwinline{\textbf{są pułapką, którą wróg przygotował na te ostatnie dni}}.

Bóg, ponownie w Swojej opatrzności, rozwiązał ten konflikt. Kellogg przyjął naganę od posłanniczki Pana i przed zakończeniem rady oświadczył, że Living Temple zostanie wycofana z rynku\footnote{\href{https://forgottenpillar.com/wp-content/uploads/2022/04/Letter-A-G-Daniells-to-W-C-White-October-29-1903.pdf}{List: A. G. Daniells do W. C. White'a, 23 października 1903, str. 5}}. Jednak po konferencji rozmawiał prywatnie z przewodniczącym generalnej konferencji, bratem Arthurem G. Daniellsem, o swoich planach dotyczących książki. Poniżej przedstawiamy wybrane listy, ujawniające plany Kellogga dotyczące korekty \textit{The Living Temple}.

Ellen White nie była obecna na dorocznej konferencji w Waszyngtonie, ale jej syn, William C. White, w niej uczestniczył. Gdy konferencja się zakończyła, brat Arthur G. Daniells napisał poufny list do Williama C. White'a dotyczący planu dr Kellogga odnośnie korekty jego książki:

\others{29 października 1903}

\othersnogap{Odkąd \textbf{rada została zamknięta} Czułem,że muszę napisać do Ciebie \textbf{poufnie w sprawie planów Dr.Kellogga  dotyczących korekty i publikacji ‘The Living Temple’}…. On \normaltext{[Kellogg]} powiedział,że kilka dni przed przybyciem do rady, przemyślał tę sprawę i zaczął dostrzegać,że \textbf{popełnił niewielki błąd w wyrażaniu swoich poglądów.}.On powiedział,że przez całą drogę był zatroskany o to,jak określić charakter Boga i jego związek z dziełami stworzenia…}

\othersnogap{\textbf{Następnie stwierdził On, że jego poprzednie poglądy \underline{dotyczące trójcy} stanęły mu na drodze do złożenia jasnego i absolutnie poprawnego oświadczenia; ale, że w krótkim czasie \underline{zaczął wierzyć w trójcę } to mógł teraz całkiem wyraźnie zobaczyć gdzie tkwiła cała trudność, i wierzył,że może wyjaśnić sprawę w satysfakcjonujący sposób.}}

\othersnogap{\textbf{Powiedział mi, że teraz uwierzył w \underline{Boga Ojca, Boga Syna i Boga Ducha Świętego}; a jego pogląd był taki, że to Bóg Duch Święty  a nie Bóg Ojciec, wypełnił całą przestrzeń i każdą żywą istotę. Powiedział, że gdyby  wierzył \underline{w to } przed napisaniem książki, to wyraził by swoje poglądy bez wywoływania mylnego wrażenia jakie teraz wywiera owa książka.}}

\othersnogap{\textbf{Przedstawiłem mu zastrzeżenia jakie znalazłem w nauczaniu i próbowałem mu pokazać, że nauczanie to jest tak całkowicie sprzeczne z ewangelią, że nie widziałem jak można by je poprawić zmieniając tylko kilka wyrażeń.}}

\othersnogap{Dyskutowaliśmy nad tą sprawą dość długo w przyjaznej atmosferze; ale byłem przekonany, że gdy się rozstaliśmy, doktor nie rozumiał ani siebie, ani charakteru swojego nauczania. I nie byłem w stanie sobie wyobrazić, jak to mogło by być  możliwe, by zmienił swoje błędne poglądy i \textbf{ w ciągu kilku dni \underline{poprawił książkę} tak, by wszystko było w porządku}.}[Letter: A. G. Daniells to W. C. White. October 29, 1903. pp. 1, 2][https://forgotten-pillar.s3.us-east-2.amazonaws.com/Letter-A-G-Daniells-to-W-C-White-October-29-1903.pdf]

Kellogg nie dostrzegał błędu w swoich poglądach, ale raczej w sposobie ich wyrażania. Nie uważał, że jego poglądy były fałszywe, a jedynie,że ich sposób wyrażania, który sprawił, że książka dawała błędne wrażenie. Jednak najwyraźniej nie była to prawda. Jak stwierdziła Siostra White, Kellogg miał problem z poglądami dotyczącymi \emcap{osobowości Boga} i miejsca Jego obecności. Dlatego Kellogg zasugerował, że aby „\textit{poprawić książkę}” włączy trynitarne wyrażenia, ponieważ zaczął wierzyć w doktrynę \textit{Trójcy}. W tym czasie Kościół Adwentystów Dnia Siódmego nie był trynitarny — doktryna Trójcy nie była częścią \emcap{Fundamentalnych Zasad}, jak widzieliśmy wcześniej. Nie dziwi więc, że Brat Daniells sprzeciwił się i odrzucił trynitarne nauczanie,twierdząc,że było ono \others{tak całkowicie sprzeczne z ewangelią}. Poprawienie książki poprzez zmianę kilku wyrażeń nie rozwiązałoby głównego problemu książki: poglądów na temat \emcap{osobowości Boga}.

W opisanych wydarzeniach i w odpowiedzi Williama White'a do Brata Daniellsa możemy zobaczyć, dlaczego Siostra White napisała Specjalne Świadectwa. William White odpowiedział Bratu Daniellsowi 4 listopada 1903 roku:

\others{Drogi Bracie, --}

\othersnogap{\textbf{\underline{Matka i ja} właśnie przeczytaliśmy twój list z \underline{z 29 Października} w którym mówisz o \underline{różnych planach,które zostały zaproponowane w celu korekty i reprodukcji ‘The Living Temple}.’}}

\othersnogap{Byliśmy mile zaskoczeni ogłoszeniem,że  Dr. Kellogg wycofa tę książkę z rynku  \textbf{i jest nam przykro,że jego myśli powracają do planu jej korekty, \underline{Matka wyraziła się dość stanowczo w tej sprawie; uważa to za bezowocne przedsięwzięcie}}. Myślę, że wkrótce napisze do Ciebie wyrażając na ten temat swoje poglądy.}

\othersnogap{\textbf{... Uważam, że będzie konieczne \underline{wydanie wkrótce specjalnego Świadectwa}, które musi zawierać bardzo pełne i jasne stanowisko w pozytywnym aspekcie tej kwestii, jak również artykuły wskazujące na błędy w nauczaniu tych, którzy odeszli od prawdy poprzez fascynujące i zwodnicze teorie}.}[\href{https://ellenwhite.org/letterbooks/555}{List od W.C. White'a do A.G. Daniellsa, 4 listopada 1903,} (str. 458)]

Oto dowód, że siostra White była zaznajomiona z zamiarami dr. Kellogga dotyczącymi korekty \textit{The Living Temple} oraz jego wiarą w doktrynę Trójcy. Według słów Williama, wypowiedziała się bardzo stanowczo w tej sprawie. Uznała to za bezowocne przedsięwzięcie. Z tego powodu konieczne było wydanie wkrótce specjalnego Świadectwa. I tak się stało. W ten sposób w 1904 roku zostało opublikowane \textit{Testimonies for the Church Containing Letters to Physicians and Ministers Instruction to Seventh-Day Adventists}, zawierające listy do lekarzy i kaznodziejów związanych z kryzysem Kellogga.

Mówiąc\others{\textbf{\underline{Matka i ja}}\textbf{ właśnie przeczytaliśmy twój list z }\textbf{\underline{29 października}}}, William zaświadczył, że siostra White była w pełni świadoma zamiarów Kellogga i jego trynitarnych przekonań. Po przeczytaniu listu Daniellsa, napisała bezpośrednią odpowiedź do dr. Kellogga. Ten list to \textit{Lt253-1903}. Jest to bardzo znaczący i odkrywczy list, ponieważ jasno pokazuje, jak prorokini odnosiła się do doktryny Trójcy. Podkreśliła doktrynę o \emcap{osobowości Boga} zawartą w \emcap{Fundamentalnych Zasadach}. Istnieją uderzające podobieństwa między tym listem a dziesiątym rozdziałem \textit{Special Testimonies}, \textit{Fundament naszej Wiary}.
