\qrchapterstar{https://forgottenpillar.com/rsc/pl-fp-introduction}{Wprowadzenie}

\addcontentsline{toc}{chapter}{Wprowadzenie}

Niniejsza książka ma do osiągnięcia trzy cele. Pierwszym z nich jest ożywienie starego filaru naszej wiary zwanego „\textit{osobowością Boga}”. Drugim celem jest przywrócenie zaufania do pism Ellen White, a trzecim — przywrócenie pierwotnej tożsamości adwentystycznej.

Przed 22 października 1844 roku istniała ogromna liczba adwentystów oczekujących na powrót Chrystusa na obłokach nieba. Był to globalny ruch ludzi oczekujących na Jego powtórne przyjście. Dzień 22 października minął bez zstąpienia Chrystusa na obłokach, a ogromna większość opuściła ruch, gardząc nim, gardząc proroctwami, Biblią i Bogiem. Pozostało bardzo niewielu wiernych, pokornych mężczyzn i kobiet, którzy byli całkowicie pewni, że Bóg prowadzi ten ruch. Wiedzieli, że Bóg świeci światłem Prawdy, a ich serca pragnęły je przyjąć. W oczach świata byli jednak uznawani za fanatyków i marzycieli. To wielkie rozczarowanie można porównać do tego, które przeżyli uczniowie Jezusa po tym, jak zobaczyli swojego Pana złożonego do grobu. Byli całkowicie pewni, że Chrystus „\textit{był prorokiem potężnym w czynie i słowie przed Bogiem i całym ludem}”, ale gdy umarł na krzyżu, byli gorzko rozczarowani, ponieważ „\textit{ufali, że On był tym, który miał odkupić Izraela}”. Jednak w stanie rozpaczy, w stanie rozczarowania sobą, byli gotowi otrzymać moc, by podbić cały świat ewangelią. Spotkali Chrystusa, a później otrzymali Jego Ducha. To samo stało się z pionierami adwentyzmu. Byli małą grupą ludzi, gorzko rozczarowanych; szukali Pana z całego serca i otrzymali Go w mocy i Prawdzie. Prawdy objawione przez Boga w tym cennym czasie kryzysu stanowią fundament wiary Adwentystów Dnia Siódmego. Prawdy te zostały poddane próbie przez wszystkie uwodzicielskie, zwodnicze teorie świata, przez tych, którzy wyszydzali tę małą grupę, a jednak te wielkie prawdy zwyciężyły. W czasie największej potrzeby Jezus dał swoje świadectwo, wzbudzając dziewczynkę, najsłabszą ze słabych, aby potwierdzić wszystkie Jego prawdy. Ellen White nie miała być źródłem prawd, ale raczej wspierać braci, którzy szukali prawdy w Biblii. Bóg użył Ellen White do potwierdzenia ich badań i wskazania im na Biblię. Ostatecznym wynikiem było ustanowienie fundamentu wiary opartego na Biblii, który będzie pewny aż do skończenia świata.

Czy zdziwiłbyś się, gdybyś wiedział, że fundament wiary Adwentystów Dnia Siódmego, który został położony na początku naszego dzieła, różni się w znacznym stopniu od obecnego? Dziś, ponad półtora wieku później, zachwycamy się opisami doświadczeń naszych pionierów; lecz od tego czasu Kościół Adwentystów Dnia Siódmego znalazł się pod wpływem kilku nowych ruchów. Od tamtej pory Kościół doświadczył wielu zmian, w tym zmian w naszej doktrynie. Niektórzy twierdzą, że zmiany te są dobre i postępowe; inni twierdzą, że są destrukcyjne i zwodnicze. Przeniesienie uwagi na pierwotny adwentyzm dnia siódmego zapoczątkowuje w obecnych czasach wielki spór. Osobiście uczestniczymy w tym sporze już od ponad sześciu lat i widzimy, że będzie się on tylko nasilał, często z opłakanymi skutkami. Wiele osób po obu stronach tego sporu w taki czy inny sposób odrzuca Ducha Proroctwa. Niektórzy całkowicie opuścili Kościół Adwentystów Dnia Siódmego. Tożsamość adwentystyczna została utracona lub drastycznie zmieniona względem tej pierwotnej.

Obecnie jesteśmy świadkami przesiewu w Kościele Adwentystów Dnia Siódmego i widzimy, jak jest miotany jedną falą kryzysu za drugą. Wielu traci wiarę i adwentystyczną tożsamość. Wierzymy jednak w rozwiązanie, które Pan w swoim miłosierdziu już zapewnił. Rozwiązanie to można znaleźć w historii ruchu Adwentystów Dnia Siódmego.

\egw{\textbf{Analizując naszą dotychczasową historię}, po rozważeniu każdego kroku w przód do naszej obecnej pozycji, mogę powiedzieć: Chwała Bogu! Gdy widzę, czego dokonał Pan, jestem pełna zdumienia i zaufania do Chrystusa jako przywódcy. \textbf{Nie mamy się czego obawiać w przyszłości, \underline{chyba że zapomnimy} drogi, którą Pan nas prowadził, oraz \underline{Jego nauk} w naszej dotychczasowej historii}}[LS 196.2; 1915][https://egwwritings.org/?ref=en\_LS.196.2]

Nie musimy się bać! Jest to wielkie zapewnienie i wielka obietnica — choć warunkowa. Musimy \textit{pamiętać}, jak Pan nas prowadził i jaka była \textit{Jego nauka w naszej dotychczasowej historii}. Kiedy patrzymy na to, czego Pan nauczał nas w przeszłości, jesteśmy zaskoczeni, widząc, jak wiele się zmieniło. Zmiana ta wymagała kilku lat i wielu kryzysów. Aby ocenić te zmiany w doktrynie, bez względu na to, czy były dobre i postępowe, czy złe i wyniszczające, ocena powinna opierać się na doświadczeniach z przeszłości, jako że Pan wyraźnie prowadził swój Kościół.

Teraz pora wysunąć śmiałe stwierdzenie — takie, które ma sprawić, że będziesz trzymać tę książkę w rękach aż do ostatniej strony. Zachęceni przez rady Ellen White do przestudiowania naszej dotychczasowej historii, doszliśmy do wniosku, że \textit{zapomnieliśmy} o jednym kluczowym filarze naszej wiary, głównym przedmiocie kryzysu, który wywołał Kellogg — \emcap{osobowości Boga}. Jednym z największych kryzysów w Kościele ADS za czasów życia proroka był kryzys Kellogga. To właśnie z tego kryzysu wywodzi się wiele innych dzisiejszych kryzysów. W tym świetle temat \emcap{osobowości Boga} jest kluczowy w naszych obecnych czasach.

Siostra White napisała Kelloggowi, że \emcap{osobowość Boga} i \emcap{osobowość Chrystusa} są \textit{filarem naszej wiary} w ten sam sposób co poselstwo o świątyni:

\egw{Ci, którzy usiłują przesunąć \textbf{dawne granice}, są chwiejni; \textbf{\underline{nie pamiętają}, jak otrzymali i usłyszeli}. Ci, którzy próbują \textbf{\underline{wprowadzić} teorie, które usunęłyby \underline{filary naszej wiary} dotyczące świątyni, \underline{lub dotyczące osobowości Boga czy Chrystusa}, działają jak ślepcy}. Starają się wprowadzić niepewność i puścić lud Boży bez steru, niezakotwiczony}[Ms62-1905.14][https://egwwritings.org/?ref=en\_Ms62-1905.14]

\emcap{Osobowości Boga} poświęca się dziś bardzo mało uwagi jako tematowi, choć jest ona jednym z kluczowych elementów w radzeniu sobie z innymi doktrynami odnoszącymi się do adwentyzmu, takimi jak doktryna o Trójcy, służbie świątynnej, roku 1844, i wszelkimi innymi doktrynami dotyczącymi niebiańskiej rzeczywistości.

\emcap{Osobowość Boga} była filarem naszej wiary. Dziś jest prawie zapomniana. Proponujemy rozsądne wyjaśnienie tego faktu. Wynika to z ewolucji języka angielskiego. Co oznacza termin „\textit{osobowość Boga}”? Ogólne rozumienie angielskiego słowa ‘\textit{personality}’ (\textit{osobowość}) zmieniło się na przestrzeni lat. Dziś \textit{osobowość} jest ogólnie postrzegana jako „\textit{charakterystyczny zbiór zachowań, percepcji i wzorców emocjonalnych}”\footnote{\href{https://en.wikipedia.org/wiki/Personality}{https://en.wikipedia.org/wiki/Personality}}, ale w XIX i na początku XX wieku oznaczało to „\textit{właściwość lub stan \textbf{bycia osobą}}”\footnote{\href{https://www.merriam-webster.com/dictionary/personality}{Merriam-Webster Dictionary}: ‘personality’.} \footnote{Hunter Robert, The American Encyclopaedic Dictionary: ‘personality’ — „właściwość lub stan bycia osobowym”. Wspomniany słownik był w posiadaniu Ellen White (patrz \href{https://repo.adventistdigitallibrary.org/PDFs/adl-22/adl-22251050.pdf?_ga=2.116010630.1065317374.1621993520-1506151612.1617862694&fbclid=IwAR3vwmp8jxtnpPEKv0KD9mCv8dJpmRGoyIXW0CkbQAjbU0h6YaBGqhgBzbk}{EGW Private and Office Libraries}).}. Odczytujemy tę definicję jako podstawową definicję słowa ‘\textit{personality}’ ze słownika Merriam-Webster\footnote{\href{https://www.merriam-webster.com/dictionary/personality\#word-history}{Merriam-Webster Dictionary} zaznacza, że pierwsze użycie definicji „właściwość lub stan bycia osobą” odnotowano w XV wieku.}. Kiedy siostra White i nasi pionierzy pisali na temat \emcap{osobowości Boga}, odnosili się do \textit{właściwości lub stanu Boga jako osoby}. Innymi słowy, zajmowali się kwestią, „\textit{czy Bóg jest osobą}” i „\textit{co sprawia, że jest On osobą}” lub „\textit{jaka jest właściwość lub stan Boga jako osoby}”? Spróbuj przypomnieć sobie, kiedy ostatnio studiowałeś Biblię pod kątem pytania: „\textit{Czy Bóg jest osobą?}”. Zastanów się, jak możesz udowodnić sobie na podstawie Biblii, że Bóg jest osobą. Zastanów się nad tym. Jest to ważne pytanie. Od tego pytania zależą Twoje postrzeganie Boga i Twoja relacja z Nim. \emcap{Osobowość Boga} ma fundamentalne znaczenie dla prawdziwej duchowości; prawdziwa duchowość opiera się na osobistej relacji z Bogiem. Nie można nawiązać prawdziwej relacji z kimkolwiek, jeśli nie jest on osobą. Być może nigdy nie zadałeś sobie tego pytania, ponieważ nigdy nie czułeś potrzeby zastanawiania się, czy Bóg jest osobą i czym jest to (właściwość lub stan), co czyni Go osobą. A może powstrzymywałeś się od tego pytania, ponieważ czułeś, że może to być tajemnica, której Bóg nie zamierzał ujawniać. Być może zaskoczy cię fakt, że Bóg udzielił w swoim Słowie jednoznacznej i twierdzącej odpowiedzi na pytanie, „\textit{jaka jest właściwość lub stan Boga jako osoby}”. Jeszcze bardziej zaskakujące było dla nas to, że pionierzy adwentyzmu, w tym siostra White, mieli wyraźne światło na ten temat i uważali go za \textit{filar naszej wiary}, jako część fundamentu wiary Adwentystów Dnia Siódmego. Kiedy \emcap{osobowość Boga} jest właściwie rozumiana w świetle naszej historii, stare cytaty świecą w nowym świetle i przedstawiane są nowe fragmenty dowodów, które pogłębiają zrozumienie naszej przeszłości i obecnego kryzysu.

Problem u podstawy kryzysu Kellogga dotyczył \emcap{osobowości Boga}. Z pewnością ważne jest, aby ocenić kryzys Kellogga dotyczący \emcap{osobowości Boga}, używając znaczenia zamierzonego w tamtym czasie, to znaczy używając definicji ‘osobowości’ jako właściwości lub stanu Boga jako osoby. Mając na uwadze tę definicję, kryzys Kellogga ukazuje się w nowym świetle i przedstawione są nam nowe istotne dowody. W świetle tych dowodów widzimy, jak Bóg prowadził nas w przeszłości; dlatego nie powinniśmy obawiać się o przyszłość. Znajomość i zrozumienie tego, jak również tego znaczenia, pomagają nam nie dać się wstrząsnąć żadnej fali oszustwa w obecnych sporach. Kiedy siostra White zwracała uwagę Kellogga na znaczenie tego tematu, zwracała również naszą uwagę, ponieważ jest to wszystkim dla nas jako ludu.

[Pisząc do Kellogga] \egw{Nie masz pełnej jasności co do \textbf{osobowości Boga}, która jest \textbf{\underline{wszystkim} dla nas jako ludu}}[Lt300-1903.7][https://egwwritings.org/?ref=en\_Lt300-1903.7]

Te badania tematu \emcap{osobowości Boga} wywołają wiele nowych i trudnych pytań. Nie obiecujemy odpowiedzieć na wszystkie z nich i być może nie będziesz usatysfakcjonowany udzielonymi odpowiedziami, ale modlimy się, mamy nadzieję i wierzymy, że ta książka spełni trzy cele przedstawione na początku tego wprowadzenia. Wierzymy, że dzięki ożywieniu doktryny \emcap{osobowości Boga} wzmocni się Twoje zaufanie do Ducha Proroctwa i że głębiej zakorzenisz się w adwentystycznym przesłaniu — w tym, w czym odnajdujemy naszą tożsamość jako ludzie — co uczyni Cię wierniejszym adwentystą dnia siódmego. Co najważniejsze, chcemy, abyś stał się bardziej świadomy Boga jako swojego osobistego Boga. To z pewnością wzmocni i pogłębi Twoją relację z Nim.

Odpowiedzi w kwestii \emcap{osobowości Boga} znajdujemy w badaniu kryzysu związanego z Kelloggiem, gdzie siostra White dała najbardziej konkretne światło na \emcap{osobowość Boga} i na fundament wiary Adwentystów Dnia Siódmego. Poniżej znajduje się cały dziesiąty rozdział z książki \textit{Testimonies for the Church Containing Letters to Physicians and Ministers Instruction to Seventh-Day Adventists}. Rozdział ten, „\textit{Fundament naszej wiary}”, zawiera głęboki wgląd w historię kryzysu Kellogga. Daje on historyczny przegląd prawd, które Bóg dał jako fundament naszej wiary, i w tych prawdach odnajdujemy naszą tożsamość jako Adwentyści Dnia Siódmego — zachowywanie przykazań Bożych i wiary Jezusa.

% chapter finished (D)