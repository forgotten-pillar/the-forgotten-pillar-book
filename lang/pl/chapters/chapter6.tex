\qrchapter{https://forgottenpillar.com/rsc/pl-fp-chapter6}{Badanie próby}

W odpowiedzi siostry White na wiarę dr. Kellogga w doktrynę o Trójcy i jego próby \textit{łatania} \textit{The Living Temple} widzimy, że postrzegała ona doktrynę o Trójcy jako sprzeczną ze światłem, które otrzymała odnośnie do \emcap{osobowości Boga}. Gdyby rzeczywiście przyjęła doktrynę o Trójcy, spodziewalibyśmy się, że starannie oddzieli ją od panteizmu i zachowa jej uzasadnione aspekty. Jednak nie to widzimy w jej odpowiedzi. Zamiast tego, jej odpowiedzią było skontrastowanie doktryny o Trójcy z prawdą o \emcap{osobowości Boga} i przywołanie jej wcześniejszych wizji, które pokazały, że ta doktryna okradłaby lud Boży z jego doświadczeń w przeszłości. W jej reakcji przywołującej, jak Bóg ustanowił \emcap{fundamentalne zasady}, wskazała, że doktryna o Trójcy \textit{burzy filary naszej wiary} i \textit{sprowadza nas na manowce z fundamentalnych zasad}. Tę wyraźną różnicę można jasno dostrzec, porównując nasze obecne Fundamentalne Wierzenia z \emcap{Fundamentalnymi Zasadami} wyznawanymi w przeszłości.

Mając na uwadze odpowiedź siostry White na wiarę dr. Kellogga w doktrynę o Trójcy, przyjrzyjmy się charakterystykom teorii, które opisała w rozdziale „\textit{Fundament naszej wiary}”. Gdy siostra White mówi o teoriach Kellogga dotyczących Boga, nasze pytanie powinno brzmieć: „Czy jej cytaty mają sens, jeśli zastosujemy do nich kontekst doktryny o Trójcy?”. Zbadajmy każdą charakterystykę.

\subsection*{Czy Trójca „okrada lud Boży z jego doświadczeń w przeszłości”?}

\egw{\normaltext{[Spirytualistyczne teorie]} czynią bezskuteczną prawdę niebiańskiego pochodzenia i \textbf{okradają lud Boży z jego doświadczeń w przeszłości}, dając mu w zamian fałszywą naukę}[SpTB02 54.1; 1904][https://egwwritings.org/read?panels=p417.275]

\egw{Ten fundament został zbudowany przez Mistrza i przetrwa burzę i nawałnicę. Czy pozwolą temu człowiekowi \normaltext{[Kelloggowi]} przedstawiać \textbf{doktryny, które zaprzeczają doświadczeniom z przeszłości ludu Bożego}? Nadszedł czas, aby podjąć zdecydowane działanie}[SpTB02 54.2; 1904][https://egwwritings.org/read?panels=p417.276]

\egw{\textbf{Jaki wpływ skłania ludzi na tym etapie naszej historii do działania w podstępny, potężny sposób, aby \underline{zburzyć fundament naszej wiary} — fundament, który został położony \underline{na początku naszej pracy} poprzez pełne modlitwy studium Słowa i przez objawienie? Na tym fundamencie \underline{budujemy przez ostatnie pięćdziesiąt lat}}. Czy dziwicie się, że gdy widzę początek działalności, która \textbf{\underline{dąży do usunięcia niektórych z filarów naszej wiary}}, mam coś do powiedzenia? Muszę być posłuszna rozkazowi: «Przeciwstaw się temu!»}[SpTB02 58.1; 1904][https://egwwritings.org/read?panels=p417.295]

Według świadectwa siostry White fundamentem naszej wiary były \emcap{Fundamentalne Zasady}. Obecnie nie odzwierciedlają one naszych wierzeń. Największy sprzeciw budzi pierwszy punkt, który dotyczy tego, kim jest Bóg. Zamiast wierzenia, że jest jeden Bóg — Ojciec, osobowa duchowa istota, mamy nowe wierzenie, że jest jeden Bóg — Ojciec, Syn i Duch Święty, jedność trzech Osób. Czy z danego światła i z doświadczeń tego, jak Bóg ustanowił pierwszy punkt \emcap{Fundamentalnych Zasad}, nowo utworzona doktryna o tym, kim jest Bóg i czym On jest, okradła lud Boży z jego doświadczeń w przeszłości?

\subsection*{Czy Trójca burzy filary naszej wiary lub prowadzi z fundamentalnych zasad na manowce?}

\egw{Zostałam pouczona przez niebiańskiego posłańca, że część rozumowania w książce «The Living Temple» jest niepoprawna i że \textbf{to rozumowanie sprowadziłoby na manowce umysły tych, którzy nie są całkowicie utwierdzeni w fundamentalnych zasadach teraźniejszej prawdy}}[SpTB02 51.3; 1904][https://egwwritings.org/read?panels=p417.262]

\egw{Mniej więcej w czasie, gdy opublikowano «The Living Temple», w porze nocnej przeszły przede mną obrazy wskazujące, że \textbf{zbliża się jakieś niebezpieczeństwo}, i że muszę się na nie przygotować poprze spisanie rzeczy, które Bóg mi objawił \textbf{odnośnie do fundamentalnych zasad naszej wiary}}[SpTB02 52.3; 1904][https://egwwritings.org/read?panels=p417.267]

\egw{\textbf{Wróg dusz starał się wprowadzić przypuszczenie, że miała nastąpić wielka reforma wśród Adwentystów Dnia Siódmego, i że ta reforma miałaby polegać na \underline{porzuceniu doktryn, które stoją jako filary naszej wiary}, i zaangażowaniu się w proces reorganizacji}. Gdyby doszło do tej reformy, co by z tego wynikło? \textbf{Zasady prawdy}, które Bóg w swojej mądrości dał Kościołowi ostatków, \textbf{zostałyby odrzucone}. Nasza religia zostałaby zmieniona. \textbf{Fundamentalne zasady}, które podtrzymywały dzieło przez ostatnie pięćdziesiąt lat, \textbf{uznano by za błąd}. Ustanowiono by nową organizację. Napisano by książki nowego porządku. Wprowadzono by system filozofii intelektualnej}[SpTB02 54.3; 1904][https://egwwritings.org/read?panels=p417.277]

Teorie dr. Kellogga dotyczące \emcap{osobowości Boga}, gdyby zostały przyjęte, zainicjowałyby reformę w Kościele Adwentystów Dnia Siódmego. Oparte na filozofii intelektualnej, spowodowałyby, że musielibyśmy się wyrzec pewnych doktryn, które stoją jako filary naszej wiary, potępiając \emcap{fundamentalne zasady} jako błąd. Czy jest możliwe, że poprzez przyjęcie doktryny o Trójcy weszliśmy w nową organizację?

\egw{Krótko przed tym, jak wysłałam świadectwa \textbf{dotyczące wysiłków wroga zmierzających do podkopania fundamentu naszej wiary poprzez rozpowszechnianie zwodniczych teorii}, przeczytałam historię o statku we mgle napotykającym górę lodową}[SpTB02 55.3; 1904][https://egwwritings.org/read?panels=p417.282]

\egw{Poselstwa wszelkiego rodzaju i typu były \textbf{narzucane Adwentystom Dnia Siódmego, aby zastąpić prawdę, która \underline{punkt po punkcie} została odkryta przez pełne modlitwy studium i potwierdzona przez cudotwórczą moc Pana}. \textbf{Lecz drogowskazy, które uczyniły nas tym, kim jesteśmy, mają być zachowane i będą zachowane}, jak Bóg oznajmił przez swoje słowo i świadectwo swojego Ducha. \textbf{Wzywa nas, abyśmy trzymali się mocno}, z uściskiem wiary, \textbf{\underline{fundamentalnych zasad}, które są oparte na \underline{niepodważalnym autorytecie}}}[SpTB02 59.1; 1904][https://egwwritings.org/read?panels=p417.299]

\emcap{Osobowość Boga} była filarem naszej wiary\footnote{\href{https://egwwritings.org/?ref=en_Ms62-1905.14}{EGW, Ms62-1905.14; 1905}}. \emcap{Osobowość Boga} była wyrażona w pierwszym punkcie \emcap{Fundamentalnych Zasad}. Czy jest możliwe, że poprzez przyjęcie doktryny o Trójcy zburzyliśmy ten konkretny filar naszej wiary? Czy jest możliwe, że przyjmując doktrynę o Trójcy zostaliśmy sprowadzeni na manowce z tej podstawowej zasady—\emcap{osobowości Boga}?

\subsection*{Czy Trójca eliminuje osobowość Boga?}

\egw{\textbf{Wprowadza to \normaltext{[«The Living Temple»]}, co jest niczym innym jak \underline{spekulacją} w odniesieniu do \underline{osobowości Boga} i tego, gdzie jet Jego obecność}}[SpTB02 51.3; 1904][https://egwwritings.org/read?panels=p417.262]

\egw{\textbf{Spirytualistyczne teorie \underline{dotyczące osobowości Boga}, doprowadzone do logicznego końca, usuwają cały chrześcijański porządek}}[SpTB02 54.1; 1904][https://egwwritings.org/read?panels=p417.275]

\egw{«The Living Temple» zawiera alfę tych teorii. Wiedziałam, że omega nastąpi wkrótce; i drżałam o nasz lud. Wiedziałam, że \textbf{muszę ostrzec naszych braci i siostry, aby nie wdawali się w spory dotyczące \underline{obecności} i \underline{osobowości Boga}. Stwierdzenia zawarte w «The Living Temple» \underline{w tej kwestii są niepoprawne}. Tylko przez błędne zastosowanie Pisma można poprzeć przedstawioną tam doktrynę}}[SpTB02 53.2; 1904][https://egwwritings.org/read?panels=p417.271]

Teorie, które Kellogg przedstawił w \textit{The Living Temple} są spekulacjami w kwestii \emcap{osobowości Boga} i tego, gdzie jest Jego obecność. Teorie te dotyczą kwestii właściwości lub stanu Boga jako osoby\footnote{Definicja słowa ‘\textit{personality}’ według Słownika Merriam-Webster — „\textit{właściwość lub stan jako osoby}”.}. Bóg dał nam jednoznaczne światło w tej sprawie w naszych \emcap{Fundamentalnych Zasadach}. Czy możliwe jest, że doktryna o Trójcy podważa to jednoznaczne światło dotyczące \emcap{osobowości Boga}?

\subsection*{Czy doktryna o Trójcy jest przedstawiana tak, jakby pani White ją popierała?}

\egw{W sporze, który powstał wśród naszych braci \textbf{odnośnie do nauk tej książki}, ci, którzy opowiadali się za jej szerokim rozpowszechnianiem, \textbf{oświadczyli: «Zawiera ona dokładnie te same poglądy, których naucza siostra White». To stwierdzenie ugodziło mnie prosto w serce. Byłam załamana, ponieważ wiedziałam, że to przedstawienie sprawy nie było prawdziwe}}[SpTB02 53.1; 1904][https://egwwritings.org/read?panels=p417.270]

\egw{\textbf{Jestem zmuszona zaprzeczyć twierdzeniu, że nauki zawarte w «The Living Temple» można poprzeć stwierdzeniami z moich pism.}. Mogą być w tej książce wyrażenia i poglądy, które są w zgodzie z moimi pismami. I mogą być w moich pismach liczne stwierdzenia, które wyrwane z kontekstu i interpretowane zgodnie z zamysłem autora «The Living Temple» wydawałyby się być zgodne z naukami tej książki. Może to dawać pozorne poparcie twierdzeniu, że poglądy w «The Living Temple» są w zgodzie z moimi pismami. \textbf{Ale broń Boże, żeby ten pogląd przeważył}}[SpTB02 53.3; 1904][https://egwwritings.org/read?panels=p417.272]

W tym momencie mamy wiele nierozstrzygniętych pytań. Jednak w miarę studiowania pierwszego punktu \emcap{Fundamentalnych Zasad} znajdziemy odpowiedzi na wszystkie te pytania. Do tej pory, w świetle \emcap{Fundamentalnych Zasad}, wiara w doktrynę o Trójcy — dla adwentysty dnia siódmego — staje się bardzo wątpliwa. Aby bronić doktryny o Trójcy, trzeba podważyć autorytet \emcap{Fundamentalnych Zasad}. W dalszej części w skrócie przeanalizujemy ich autorytet, kontekst w historii adwentyzmu i Boży cel w ich przekazaniu. Przyjrzymy się również kwestii prawdziwego autorstwa \emcap{Fundamentalnych Zasad} i ich roli w dzisiejszych czasach.

\begin{titledpoem}

    \stanza{
        Wizja siostry White stoi przeciw fali, \\
        By doktryną o Trójcy prawdy nie zalali. \\
        Poprzez jej ostrzeżenia mocno potępiona, \\
        Już nie zaciemni światła, zniszczenia nie dokona.
   }

    \stanza{
        Starannie położone filary naszych wierzeń \\
        Mierzą się dziś z wyzwaniem według tamtych ostrzeżeń. \\
        Co wzniesiono modlitwą, a także objawieniem, \\
        Przeciwnik chce zrujnować doktryny nowej tchnieniem.
   }

    \stanza{
        Osobowość Boga — prawda czysta i szczera — \\
        Lecz po tej innowacji niebo się jej wypiera. \\
        Na szali są przeżycia z historii Bożych ludzi, \\
        Gdy do burzenia podstaw nowa doktryna budzi.
   }

    \stanza{
        Niech nasze drogowskazy stoją pośród nas wiecznie \\
        I jak gwiazda do celu prowadzą nas bezpiecznie. \\
        Trzymajmy się podstawy, niech z rąk się nie wydziera, \\
        Bo w pomieszania mgle zniknie nasza galera.
    }
    
\end{titledpoem}

