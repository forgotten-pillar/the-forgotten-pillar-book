
\chapter{Badanie testu}

W odpowiedzi siostry White na wiarę dr. Kellogga w doktrynę o Trójcy i jego próby \textit{łatania} The Living Temple, widzimy, że postrzegała ona doktrynę o Trójcy jako sprzeczną ze światłem, które otrzymała odnośnie \emcap{osobowości Boga}. Gdyby rzeczywiście przyjęła doktrynę o Trójcy, spodziewalibyśmy się, że starannie oddzieli ją od panteizmu i zachowa jej uzasadnione aspekty. Jednak nie to widzimy w jej odpowiedzi. Zamiast tego, jej odpowiedzią było skontrastowanie doktryny o Trójcy z prawdą o \emcap{osobowości Boga}, przywołując jej wcześniejsze wizje, które pokazały, że ta doktryna okradłaby lud Boży z ich wcześniejszych doświadczeń. W jej reaktywnym odwołaniu, jak Bóg ustanowił \emcap{fundamentalne zasady}, wskazała, że doktryna o Trójcy \textit{burzy filary naszej wiary} i \textit{prowadzi nas na manowce od podstawowych zasad}. Tę wyraźną różnicę można jasno dostrzec, porównując nasze obecne Fundamentalne Wierzenia z \emcap{Fundamentalnymi Zasadami} wyznawanymi w przeszłości.

Mając na uwadze odpowiedź Siostry White na wierzenie dr. Kellogga w doktrynę o Trójcy, przyjrzyjmy się cechom teorii, które opisała w rozdziale “\textit{Fundament naszej wiary}”. Gdy Siostra White mówi o teoriach Kellogga dotyczących Boga, nasze pytanie powinno brzmieć: “czy jej cytaty mają sens, jeśli zastosujemy do nich kontekst doktryny o Trójcy?” Zbadajmy każdą cechę.

\subsection*{Czy Trójca “pozbawia lud Boży z ich przeszłych doświadczeń”?}

\egw{One \normaltext{[teorie spirytualistyczne]} unieważniają prawdę niebiańskiego pochodzenia i \textbf{okradają lud Boży z ich przeszłych doświadczeń}, dając im w zamian fałszywą naukę.}[SpTB02 54.1; 1904][https://egwwritings.org/?ref=en\_SpTB02.54.1]

\egw{Ten fundament został zbudowany przez Mistrza i przetrwa burzę i nawałnicę. Czy pozwolą temu człowiekowi \normaltext{[Kelloggowi]} przedstawiać \textbf{doktryny, które zaprzeczają przeszłemu doświadczeniu ludu Bożego}? Nadszedł czas na zdecydowane działanie.}[SpTB02 54.2; 1904][https://egwwritings.org/?ref=en\_SpTB02.54.2]

\egw{\textbf{Jaki to wpływ, który skłaniałby ludzi na tym etapie naszej historii do działania w podstępny, potężny sposób, aby \underline{zburzyć fundament naszej wiary} — fundament, który został położony \underline{na początku naszego dzieła} poprzez modlitewne studiowanie Słowa i przez objawienie? Na tym fundamencie \underline{budowaliśmy przez ostatnie pięćdziesiąt lat}}. Czy dziwicie się, że gdy widzę początek dzieła, które \textbf{\underline{usunęłoby niektóre z filarów naszej wiary},} mam coś do powiedzenia? Muszę być posłuszna rozkazowi: ‘Przeciwstaw się temu!’}[SpTB02 58.1; 1904][https://egwwritings.org/?ref=en\_SpTB02.58.1]

Według świadectwa siostry White, fundamentem naszej wiary były \emcap{Fundamentalne Zasady}. Obecnie nie reprezentują one naszych wierzeń. Najbardziej problematyczny jest pierwszy punkt, dotyczący tego, kim jest Bóg. Zamiast wierzenia, że jest jeden Bóg — Ojciec, osobowa duchowa istota, mamy nowe wierzenie, że jest jeden Bóg — Ojciec, Syn i Duch Święty, jedność trzech Osób. Czy z danego światła i z doświadczeń tego, jak Bóg ustanowił pierwszy punkt \emcap{Fundamentalnych Zasad}, nowo utworzona doktryna o tym, kim jest Bóg i czym On jest, pozbawiła lud Boży z jego przeszłych doświadczeń?

\subsection*{Czy Trójca burzy filary naszej wiary lub prowadzi na manowce od fundamentalnych zasad?}

\egw{Zostałam pouczona przez niebiańskiego posłańca, że niektóre rozumowanie w książce ‘Living Temple’ jest błędne i że \textbf{to rozumowanie wprowadziłoby w błąd umysły tych, którzy nie są gruntownie utwierdzeni w fundamentalnych zasadach obecnej prawdy.}}[SpTB02 51.3; 1904][https://egwwritings.org/?ref=en\_SpTB02.51.3]

\egw{W czasie, gdy opublikowano ‘Living Temple’, przeszły przede mną w nocnym widzeniu obrazy wskazujące, że \textbf{zbliża się niebezpieczeństwo} i że muszę się do niego przygotować, spisując rzeczy, które Bóg mi objawił \textbf{odnośnie fundamentalnych zasad naszej wiary}.}[SpTB02 52.3; 1904][https://egwwritings.org/?ref=en\_SpTB02.52.3]

\egw{\textbf{Nieprzyjaciel dusz starał się wprowadzić pewne przypuszczenia, że wśród Adwentystów Dnia Siódmego miała nastąpić wielka reforma, i że ta reforma miała polegać na \underline{porzuceniu doktryn, które stoją jako filary naszej wiary,} i zaangażowaniu się w proces reorganizacji}. Gdyby ta reforma miała miejsce, co by z tego wynikło? \textbf{Zasady prawdy}, które Bóg w swojej mądrości dał Kościołowi Ostatków, \textbf{zostałyby odrzucone}. Nasza religia zostałaby zmieniona. \textbf{Fundamentalne zasady}, które podtrzymywały dzieło przez ostatnie pięćdziesiąt lat, \textbf{zostałyby uznane za błąd}. Powstałaby nowa organizacja. Napisano by książki nowego porządku. Wprowadzono by system filozofii intelektualnej.}[SpTB02 54.3; 1904][https://egwwritings.org/?ref=en\_SpTB02.54.3]

Teorie dr. Kellogga dotyczące \emcap{osobowości Boga}, gdyby zostały przyjęte, rozpaliłyby reformację w Kościele Adwentystów Dnia Siódmego. Oparte na filozofii intelektualnej, spowodowałyby wyrzeczenie się niektórych doktryn, które stoją jako filary naszej wiary, potępiając \emcap{fundamentalne zasady} jako błąd. Czy możliwe, że poprzez przyjęcie doktryny o Trójcy weszliśmy w nową organizację?

\egw{Krótko przed tym, jak wysłałam świadectwa \textbf{dotyczące wysiłków wroga zmierzających do podważenia fundamentu naszej wiary poprzez rozpowszechnianie zwodniczych teorii}, przeczytałam historię o statku we mgle spotykającym górę lodową...}[SpTB02 55.3; 1904][https://egwwritings.org/?ref=en\_SpTB02.55.3]

\egw{Poselstwa wszelkiego rodzaju  były \textbf{narzucane Adwentystom Dnia Siódmego, aby zastąpić prawdę, która \underline{punkt po punkcie}, została odkryta poprzez modlitewne studium i potwierdzona przez cudowną moc Pana}. \textbf{Ale znaki, które uczyniły nas tym, kim jesteśmy, mają być zachowane i będą zachowane}, jak Bóg wskazał przez swoje słowo i świadectwo swojego Ducha. \textbf{Wzywa nas do mocnego trzymania się}, z wiarą, \textbf{\underline{fundamentalnych zasad}, które są oparte na \underline{niepodważalnym autorytecie}}.}[SpTB02 59.1; 1904][https://egwwritings.org/?ref=en\_SpTB02.59.1]

\emcap{Osobowość Boga} była filarem naszej wiary\footnote{\href{https://egwwritings.org/?ref=en_Ms62-1905.14}{EGW, Ms62-1905.14; 1905}}. \emcap{Osobowość Boga} była wyrażona w pierwszym punkcie \emcap{fundamentalnych zasad}. Czy możliwe, że poprzez przyjęcie doktryny o Trójcy zburzyliśmy ten konkretny filar naszej wiary? Czy możliwe, że przyjmując doktrynę o Trójcy zostaliśmy sprowadzeni na manowce z tej podstawowej zasady—\emcap{osobowości Boga}?

\subsection*{Czy Trójca eliminuje osobowość Boga?}

\egw{\textbf{Wprowadza to \normaltext{[The Living Temple]} nic innego jak \underline{spekulacje} dotyczące \underline{osobowości Boga} i miejsca Jego obecności.}}[SpTB02 51.3; 1904][https://egwwritings.org/?ref=en\_SpTB02.51.3]


\egw{\textbf{The spiritualistic theories \underline{regarding the personality of God}, followed to their logical conclusion, sweep away the whole Christian economy.}}[SpTB02 54.1; 1904][https://egwwritings.org/?ref=en\_SpTB02.54.1]


\egw{\textbf{Spirytualistyczne teorie \underline{dotyczące osobowości Boga}, doprowadzone do ich logicznej konkluzji, niszczą całą chrześcijańską ekonomię.}}[SpTB02 54.1; 1904][https://egwwritings.org/?ref=en\_SpTB02.54.1]

\egw{‘Living Temple’ zawiera alfę tych teorii. Wiedziałam, że omega nastąpi wkrótce; i drżałam o nasz lud. Wiedziałam, że \textbf{muszę ostrzec naszych braci i siostry, aby nie wchodzili w spór dotyczący \underline{obecności} i \underline{osobowości Boga}. Stwierdzenia zawarte w ‘Living Temple’ \underline{odnośnie tego punktu są niepoprawne}. Pismo użyte do poparcia tam przedstawionej doktryny jest błędnie zastosowanym Pismem}.}[SpTB02 53.2; 1904][https://egwwritings.org/?ref=en\_SpTB02.53.2]

Teorie, które Kellogg przedstawił w The Living Temple są spekulatywne w odniesieniu do \emcap{osobowości Boga} i miejsca Jego obecności. Teorie te dotyczą kwestii właściwości lub stanu Boga jako osoby\footnote{Definicja słowa ‘\textit{personality}’ według Merriam-Webster - “\textit{właściwość lub stan jako osoby}”}. Bóg dał nam jednoznaczne światło w tej sprawie w naszych \emcap{fundamentalnych zasadach}. Czy możliwe jest, że doktryna o Trójcy podważa to jednoznaczne światło dotyczące \emcap{osobowości Boga}?

\subsection*{Czy doktryna o Trójcy jest przedstawiana tak, jakby pani White ją popierała?}

\egw{W sporze, który powstał wśród naszych braci \textbf{odnośnie nauk tej książki,} ci, którzy popierali jej szerokie rozpowszechnianie \textbf{oświadczyli: ‘Zawiera ona dokładnie te same poglądy, których nauczała Siostra White.’ To stwierdzenie ugodziło prosto w moje serce. Byłam załamana, ponieważ wiedziałam, że to przedstawienie sprawy nie było prawdziwe}.}[SpTB02 53.1; 1904][https://egwwritings.org/?ref=en\_SpTB02.53.1]

\egw{\textbf{Jestem zmuszona zaprzeczyć twierdzeniu, że nauki zawarte w ‘Living Temple’ mogą być poparte cytatami z moich pism}. Mogą być w tej książce wyrażenia i poglądy, które są w harmonii z moimi pismami. I mogą być w moich pismach liczne stwierdzenia, które wyrwane z kontekstu i interpretowane zgodnie z zamysłem autora ‘Living Temple’, wydawałyby się być w harmonii z naukami tej książki. Może to dawać pozorne wsparcie dla twierdzenia, że poglądy w ‘Living Temple’ są w harmonii z moimi pismami. \textbf{Ale nie daj Boże, aby ten pogląd przeważył}.}[SpTB02 53.3; 1904][https://egwwritings.org/?ref=en\_SpTB02.53.3]

W tym momencie mamy wiele nierozstrzygniętych pytań. Jednak w miarę studiowania pierwszego punktu \emcap{fundamentalnych zasad}, znajdziemy odpowiedzi na wszystkie te pytania. Do tej pory, w świetle \emcap{fundamentalnych zasad}, wiara w doktrynę o Trójcy — jako Adwentysta Dnia Siódmego — staje się bardzo wątpliwa. Aby bronić doktryny o Trójcy, autorytet \emcap{fundamentalnych zasad} musi zostać podważony. W dalszej części krótko przestudiujemy ich autorytet, kontekst w historii adwentyzmu i Boży cel w ich przekazaniu. Przyjrzymy się również prawdziwemu autorstwu \emcap{fundamentalnych zasad} i ich roli w dzisiejszych czasach.
