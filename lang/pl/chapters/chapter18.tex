\qrchapter{https://forgottenpillar.com/rsc/pl-fp-chapter18}{Niebiańskie Trio}

Do tej pory widzieliśmy dowody na to, że Ellen White znała trynitarne poglądy dr. Kellogga, i widzieliśmy, jak na nie odpowiedziała. Zawsze podkreślała prawdę o obecności i \emcap{osobowości Boga} i wzywała do powrotu do fundamentu naszej wiary — \emcap{Fundamentalnych Zasad}. Jednak kiedy adwentystyczni uczeni omawiają doktrynę o Trójcy i Ellen White, nie podchodzą do tego w taki sam sposób, jak robiła to Ellen White. \emcap{Fundamentalne Zasady} wraz z doktryną o \emcap{osobowości Boga} są pomniejszane, a przedstawiana jest zniekształcona historia, że Ellen White była trynitarna i odpowiedzialna za przyjęcie doktryny o Trójcy w nasze szeregi. Chcemy zakwestionować tę zniekształconą historię, przyglądając się dowodom, które często są używane do wspierania tej fałszywej narracji.

Jednym z najbardziej znanych cytatów używanych do poparcia twierdzenia, że siostra White była odpowiedzialna za przyjęcie doktryny o Trójcy w nasze szeregi, są jej pisma i komentarze do Mt 28:19\footnote{\bible{Idźcie więc i nauczajcie wszystkie narody, chrzcząc je w imię Ojca i Syna, i Ducha Świętego}[Mt 28:19]}. Najbardziej wyróżniającym się cytatem w obronie doktryny o Trójcy jest cytat o „\textit{Niebiańskim Trio}”:

\egw{\textbf{Istnieją \underline{trzy żyjące osoby} \underline{niebiańskiego trio}}; w imię tych trzech wielkich mocy — \textbf{Ojca, Syna i Ducha Świętego} — ci, którzy przyjmują Chrystusa przez żywą wiarę, są chrzczeni, a te moce będą współpracować z posłusznymi poddanymi nieba w ich wysiłkach, aby żyć nowym życiem w Chrystusie...}[Ev 615.1; 1946][https://egwwritings.org/read?panels=p30.3407]

Powtórzmy, ten cytat jest często przytaczany jako argument, że siostra White broniła doktryny o Trójcy i ją popierała. Lecz jeśli przyjrzymy się temu cytatowi w kontekście, w którym był napisany, zobaczymy, że w samym tym cytacie ona właściwie \textit{odrzuciła} tę doktrynę i wywyższyła prawdę o \emcap{osobowości Boga}. Dla niektórych jest to absurdalne twierdzenie, ale zapraszamy do dokonania własnego osądu na podstawie przedstawionych danych. Zbadajmy kontekst tego cytatu.

\egw{Jestem pouczona, aby powiedzieć: \textbf{Poglądy} tych, którzy poszukują zaawansowanych naukowych idei, \textbf{\underline{nie są godne zaufania}}. Czynione są następujące porównania: «\textbf{Ojciec jest jak niewidzialne światło; Syn jest jak światło ucieleśnione; Duch jest jak światło rozlane wszędzie}». «\textbf{Ojciec jest jak rosa, niewidzialna para; Syn jest jak rosa zebrana w pięknej formie; Duch jest jak rosa opadła na siedzibę życia}». Inne wyobrażenie: «\textbf{Ojciec jest jak niewidzialna para. Syn jest jak gęsta chmura. Duch jest spadającym deszczem, działającym w orzeźwiającej mocy}»}[Ms21-1906.8; 1906][https://egwwritings.org/read?panels=p9754.15]

Jakie poglądy nie są godne zaufania? Dane sugerują, że są to trynitarne idee \textit{jednego Boga w trzech osobach}. Skąd to wiemy? Widzimy to w literackim kontekście porównań, które siostra White cytowała. Wbrew powszechnemu przekonaniu, że odnosiła się do „\textit{fałszywej}” trójcy przedstawionej przez dr. Kellogga,\footnote{Whidden, Woodrow W, et al. \textit{The Trinity: Understanding God's Love, His Plan of Salvation, and Christian Relationships}. Hagerstown, Md, Review And Herald Pub. Association, 2002, str. 216.} w rzeczywistości odnosiła się do trynitarnej idei trzech żyjących osób jednego żyjącego Boga, propagowanej przez Williama Boardmana w jego książce \textit{Higher Christian Life}, którą cytowała. Kontekst ma znaczenie. Kontekst cytatów, które przytoczyła, pokazuje, że wyobrażenia Ojca, Syna i Ducha Świętego służą do zilustrowania poglądu o trzech żyjących osobach jednego Boga. To jest pogląd, co do którego zostaliśmy wyraźnie pouczeni przez Boga, aby mu nie ufać. Niech dane same się interpretują.

\section*{William Boardman, „The Higher Christian Life”}

Ellen White posiadała książkę Williama Boardmana \textit{The Higher Christian Life}. Była to dobra książka o chrześcijańskim uświęceniu, ale zawierała trynitarne poglądy, które siostra White, w sposób szczególny pouczona przez Boga, miała obnażyć. Jest to kolejny przykład dowodu, gdzie widzimy, że Ellen White znała trynitarne stanowisko i odnosiła się do niego bezpośrednio. Zapoznajmy się z trynitarnymi poglądami promowanymi przez Williama Boardmana.

Mówiąc o Trójjedynym Bogu, William Boardman pisze:

\othersQuote{Ponadto Ojciec jest autorem i twórcą planu zbawienia przez wiarę w Jego Syna; a kiedy ufamy Jego Synowi, czcimy Ojca, ponieważ akceptujemy Jego plan zbawienia dla nas, usprawiedliwiamy Jego mądrość i działamy zgodnie z Jego wolą w tej sprawie. \textbf{Spojrzenie na oficjalne i istotne relacje osób Świętej Trójcy wobec siebie nawzajem i wobec nas może rzucić dodatkowe światło na naszą drogę}. W tej kwestii lekkomyślność graniczyłaby z bluźnierstwem. To święty teren. Ten, kto nań wkracza, musi stąpać bosą stopą i z odkrytą, nisko pochyloną głową.}[William Boardman, The Higher Christian Life, str. 99; 1858][https://archive.org/details/higherchristian02boargoog/page/n106/]

Brat Boardman chce, abyśmy rzucili \others{spojrzenie na oficjalne i istotne relacje} trzech osób Świętej Trójcy. Twierdzi, że \textit{Bóg jest jeden, ale także trzema}–\textit{Trójjedyny}–przedstawiając oficjalne i istotne relacje osób Świętej Trójcy. Jego fundamentalne stwierdzenie i zarys jego tezy są następujące:

\othersQuote{\textbf{Ojciec jest pełnią Bóstwa \underline{niewidzialnie}, bez postaci, którego żadne stworzenie nie widziało ani nie może zobaczyć}. \\
\textbf{Syn jest pełnią Bóstwa \underline{ucieleśnioną}, aby Jego stworzenia mogły Go widzieć, poznać i Mu ufać}. \\
\textbf{Duch jest pełnią Bóstwa \underline{we wszystkich czynnych działaniach}, czy to stworzenia, opatrzności, objawienia lub zbawienia, przez które Bóg objawia się wszechświatowi i poprzez wszechświat}.}[William Boardman, The Higher Christian Life, str. 100][https://archive.org/details/higherchristian02boargoog/page/n108/]

To stwierdzenie jest fundamentem dla jego kolejnych stwierdzeń i ilustracji. W następnych akapitach William Boardman podaje biblijne motywy, aby zilustrować \others{oficjalne i istotne relacje Świętej Trójcy} — \textit{to znaczy, że Bóg jest jeden, ale jednak trzema}. Pisze on:

\othersQuote{Inne z imion Jezusa da te same analogie w świetle nie mniej uderzającym — \textbf{Słońce Sprawiedliwości}. \\
Całe światło słońca na niebie było kiedyś ukryte w niewidzialności pierwotnego mroku; a potem światło teraz płonące w kuli dnia stało się, gdy wyszło pierwsze polecenie: Niech stanie się światło! I stało się światło, co najwyżej tylko jako rozproszona mgła szarego świtu poranka stworzenia z ciemności chaotycznej nocy, bez postaci, ciała, centrum, blasku czy chwały. Lecz gdy zostało oddzielone od ciemności i skupione w słońcu, wtedy w swoim wspaniałym blasku stało się tak olśniewające, że tylko orle oko mogło znieść patrzenie mu w twarz. \\
Jednakże jego promienie padające ukośnie przez ziemską atmosferę i opary rozweselają cały świat tym samym światłem, rozpraszając zimę, chłód i ciemność; rozpoczynając wiosnę kwiecistym pięknem, lato wiosenną obfitością, a jesień obładowaną złotymi skarbami do spichlerza.
\textbf{Ojciec jest jak Światło niewidzialne}. \\
\textbf{Syn jest jak Światło ucieleśnione}. \\
\textbf{Duch jest jak Światło zesłane w dół}.}[William Boardman, The Higher Christian Life, str. 101, 102.][https://archive.org/details/higherchristian02boargoog/page/n108/]

Ta ilustracja Słońca Sprawiedliwości pokazuje, że Bóg Ojciec, który jest \textit{pełnią Bóstwa niewidzialnie}, może być symbolicznie zilustrowany jako Światło, które \others{było kiedyś ukryte w niewidzialności pierwotnego mroku}. Syn, który jest \textit{pełnią Bóstwa ucieleśnioną}, jest jak Światło, które jest ucieleśnione w \others{poranku stworzenia}. Duch Święty, który jest \textit{pełnią Bóstwa we wszystkich czynnych działaniach}, jest jak \others{Światło zesłane w dół}. William Boardman daje nam inną podobną ilustrację, aby wyjaśnić \others{oficjalne relacje osób Bóstwa}:

\othersQuote{Jedno z podobieństw dla błogosławionych wpływów Ducha, \textbf{jednocześnie dając te same oficjalne relacje osób Bóstwa, wobec siebie nawzajem i wobec nas}, może je jeszcze bardziej zilustrować — \textbf{Rosa} — \textbf{rosa Hermonu} — rosa na skoszonej łące. Zanim rosa w ogóle zbierze się w krople, wisi nad całym krajobrazem w niewidzialnej parze, wszechobecna, ale niewidoczna. Z czasem, gdy światło słabnie w poranek, i gdy temperatura opada i dotyka punktu rosy, niewidzialne staje się widzialnym, ucieleśnionym; a gdy słońce wschodzi, stoi w diamentowych kroplach drżących i błyszczących w młodych promieniach słońca w perłowym pięknie na liściu i kwiecie, na całej twarzy przyrody. \\
Na dodatek powstaje bryza, oddech nieba jest delikatnie niesiony, potrząsając liściem i kwiatem, i w jednej chwili perłowe krople są znowu niewidoczne. Ale teraz gdzie? Opadłe u korzenia ziela i kwiatu, aby przekazać nowe życie, świeżość, wigor wszystkiemu, czego dotykają. \\
\textbf{Ojciec jest jak rosa w niewidzialnej parze}. \\
\textbf{Syn jest jak rosa zebrana w pięknej postaci}. \\
\textbf{Duch jest jak rosa opadła na siedzibę życia}.}[William Boardman, The Higher Christian Life, str. 102, 103][https://archive.org/details/higherchristian02boargoog/page/n110/]

Ojciec, który jest \textit{pełnią Bóstwa niewidzialnie}, jest zilustrowany przez \others{rosę w niewidzialnej parze}. Syn, który jest \textit{pełnią Bóstwa ucieleśnioną}, jest zilustrowany przez \others{rosę zebraną w pięknej postaci}. Duch, który jest \textit{pełnią Bóstwa we wszystkich czynnych dziełach}, jest zilustrowany przez \others{rosę opadłą na siedzibę życia}. Kolejna ilustracja, która obrazuje oficjalne relacje trzech osobowości jednego Boga, jest poprzez inne biblijne porównanie — Deszcz.

\othersQuote{\textbf{Jeszcze jedno z tych biblijnych porównań} — bynajmniej nie ostatnie z możliwych — nie będzie niemile widziane ani bezużyteczne — \textbf{Deszcz}. \\
Deszcz, podobnie jak rosa, unosi się najpierw w niewidzialności i wszechobecności, nad wszystkim, wokół wszystkiego. Niewidziany przez nikogo. Dopóki pozostaje w swojej niewidzialności, ziemia wysycha, powstają grudy, gleba pęka, słońce wylewa swój palący żar, wiatry unoszą kurz w kręcących się wirach i toczących się chmurach, a głód, wychudzony i chciwy, kroczy przez ziemię, a za nim podąża zaraza i śmierć. Z czasem gorliwy obserwator widzi małą chmurę podobną do dłoni, wznoszącą się daleko nad morzem. Gromadzi się, gromadzi, gromadzi; nadchodzi i rozprzestrzenia się, gdy nadchodzi, w majestacie nad całym niebem - ale wszystko jest jeszcze wyschnięte, suche i martwe na ziemi. \\
Lecz teraz spada kropla, i kropla po kropli, szybciej, prędzej – mżawka, deszcz — pędząc i dając ziemi wszystkie skarby chmur — grudy otwierają się, bruzdy miękną, źródła, strumyki, rzeki pęcznieją i napełniają się, a cała kraina jest znowu uradowana przywróconą obfitością. \\
\textbf{Ojciec jest podobny do niewidzialnej pary}. \\
\textbf{Syn jest jak ociężała chmura i padający deszcz}. \\
\textbf{Duch jest Deszczem — opadłym i działającym w orzeźwiającej mocy}.}[William Boardman, The Higher Christian Life, str. 103, 104][https://archive.org/details/higherchristian02boargoog/page/n110/]

Wysłuchajmy dobrze Williama Boardmana. Nie mówi on, że Ojciec jest \others{niewidzialną parą}; raczej używa metafory deszczu i \others{niewidzialnej pary}, aby zilustrować swój główny punkt, że Ojciec jest niewidzialną pełnią Bóstwa. Tak samo jest z Synem, który, podobnie jak deszcz objawiony w ociężałych chmurach, jest całą pełnią Bóstwa objawioną. Aby upewnić się, że jego poglądy nie są potencjalnie błędnie przedstawiane, William Boardman wyjaśnił swoje stanowisko. To był właśnie ten pogląd, któremu poinstruowana przez Boga Ellen White miała nie ufać:

\othersQuote{\textbf{Te porównania są wszystkie niedoskonałe. Raczej ukrywają niż ilustrują \underline{trójosobowość jednego Boga}, ponieważ nie są osobami, lecz rzeczami, zbyt słabymi i ziemskimi w najlepszym razie, by przedstawić żyjące osobowości żyjącego Boga. Mogą jedynie zilustrować oficjalne relacje każdej wobec innych oraz każdej i wszystkich wobec nas. I nie tylko. Mogą również ilustrować prawdę, że cała pełnia Tego, który napełnia wszystko we wszystkim, mieszka w każdej osobie \underline{Trójjedynego Boga}}. \\
\textbf{Ojciec jest całą pełnią Bóstwa NIEWIDZIALNIE}. \\
\textbf{Syn jest całą pełnią Bóstwa OBJAWIONĄ}. \\
\textbf{Duch jest całą pełnią Bóstwa OBJAWIAJĄCĄ}. \\
\textbf{Osoby te nie są jedynie urzędami lub sposobami objawienia, ale żyjącymi osobami żyjącego Boga}.}[William Boardman, The Higher Christian Life, str. 104, 105][https://archive.org/details/higherchristian02boargoog/page/n112/]

Kluczowe jest podkreślenie, że kiedy Boardman używa tych biblijnych porównań z przyrody, mówi o ilustracjach, a nie o rzeczywistości. Te porównania ilustrują jego poglądy. Jak sam przyznaje, był to pogląd o trzech \others{żyjących osobowościach żyjącego Boga}. Chociaż te ilustracje są niedoskonałe, mogą \others{ilustrować oficjalne relacje} \others{trójosobowości jednego Boga} oraz \others{prawdę, że cała pełnia Tego, który napełnia wszystko we wszystkim, mieszka w każdej osobie Trójjedynego Boga}. Jeden Bóg w trzech osobach to pogląd, o który chodzi, a ten pogląd jest wspólny dla wszystkich typów i wersji doktryny o Trójcy — włącznie z naszym obecnym trynitarnym stanowiskiem w drugim punkcie Fundamentalnych Wierzeń.\footnote{\others{Jest \textbf{jeden Bóg}: Ojciec, Syn i Duch Święty, \textbf{jedność trzech} odwiecznie współistniejących  \textbf{Osób}…} 2. punkt Fundamentalnych Wierzeń}

Po krótkim spojrzeniu na poglądy Williama Boardmana jest jasne, że poglądy, które z pouczenia Boga Ellen White miała napiętnować, były poglądami o Trójjedynym Bogu, czyli o \textit{trzech żyjących osobach w Trójcy}. Mając na uwadze te informacje, przyjrzyjmy się odpowiedzi Ellen White.

\section*{Ellen White o poglądach Williama Boardmana}

W związku z cytatem o Niebiańskim Trio twierdzono, że Ellen White była trynitariańska. Dzieje się tak przez ignoranckie lub czasami celowe pomijanie kontekstu tego cennego cytatu. Czytając odpowiedź Ellen White, w której broni naszego postrzegania Boga, spróbujcie rozpoznać, do kogo się odnosi, gdy mówi o Bogu. Czy Bóg, którego broniła, to Trójca czy Ojciec? Odnosząc się do ilustracji Williama Boardmana, powiedziała:

\egw{\textbf{Wszystkie te \underline{spirytualistyczne} zobrazowania są po prostu nicością}. Są niedoskonałe, nieprawdziwe. Osłabiają i umniejszają Majestat, do którego nie można porównać żadnego ziemskiego podobieństwa. \textbf{Boga nie można porównywać z rzeczami, które stworzyły Jego ręce}. Są to zwykłe ziemskie rzeczy, cierpiące pod przekleństwem Boga z powodu grzechów człowieka. \textbf{Ojca nie można opisać rzeczami ziemskimi}. \textbf{Ojciec jest całą pełnią Bóstwa \underline{cieleśnie} i jest \underline{niewidzialny dla śmiertelnego wzroku}}}[Ms21-1906.9; 1906][https://egwwritings.org/read?panels=p9754.15]

Gdy widzimy kontekst, jest oczywiste, że siostra White podąża za tokiem rozumowania Boardmana i poprawia błędy. Dla lepszego porównania spójrzmy na ich pisma obok siebie:

\begin{table}[h!]
\centering
\renewcommand{\arraystretch}{1.5}
\setlength{\tabcolsep}{15pt}
\begin{tabular}{|p{0.4\textwidth}|p{0.4\textwidth}|}
\hline
\multicolumn{1}{|c|}{\textbf{William Boardman}} & \multicolumn{1}{c|}{\textbf{Ellen G. White}} \\ \hline
\othersQuote{Te porównania są wszystkie niedoskonałe. Raczej ukrywają niż \textbf{ilustrują trójosobowość \underline{jednego Boga}}, ponieważ nie są osobami, lecz rzeczami, zbyt słabymi i ziemskimi w najlepszym razie, by przedstawić \textbf{żyjące osobowości żyjącego Boga}. \textbf{Mogą jedynie zilustrować oficjalne relacje każdej wobec innych oraz każdej i wszystkich wobec nas. I nie tylko. Mogą również ilustrować prawdę, że cała pełnia Tego, który napełnia wszystko we wszystkim, mieszka w \underline{każdej osobie Trójjedynego Boga}}.}[str. 104, 105][https://archive.org/details/higherchristian02boargoog/page/n112] & 
\egw{\textbf{Wszystkie te \underline{spirytualistyczne} zobrazowania są po prostu nicością}. Są niedoskonałe, nieprawdziwe. Osłabiają i umniejszają Majestat, do którego nie można porównać żadnego ziemskiego podobieństwa. \textbf{Boga nie można porównywać z rzeczami, które stworzyły Jego ręce}. Są to zwykłe ziemskie rzeczy, cierpiące pod przekleństwem Boga z powodu grzechów człowieka. \textbf{Ojca nie można opisać rzeczami ziemskimi}.}[Ms21-1906.9; 1906][https://egwwritings.org/read?panels=p9754.15] \\ \hline
\end{tabular}
\end{table}

W tym porównaniu jasne jest, kim jest Bóg dla Williama Boardmana, a kim jest dla siostry White. Dla Boardmana Bóg jest Trójjedynym Bogiem, trójosobowością jednego Boga. Dla siostry White Bóg jest Ojcem. Dla Boardmana te zobrazowania są niedoskonałe, ponieważ \others{raczej ukrywają niż ilustrują trójosobowość jednego Boga}, a dla siostry White te zobrazowania są niedoskonałe, ponieważ \egwnodot{Ojca nie można opisać rzeczami ziemskimi}. Dla Boardmana Bóg to \textit{Trójjedyny Bóg}; dla siostry White Bóg to \textit{Ojciec}.

Jedynym spostrzeżeniem Boardmana, który Ellen White potwierdza, jest to, że te wyobrażenia są niedoskonałe. Z pewnością William Boardman nie zgodziłby się z Ellen White, że te przedstawienia są \textit{spirytualistyczne} i \textit{nieprawdziwe}. Przeciwnie, wierzy on, że te ilustracje \others{ilustrują prawdę, że cała pełnia Tego, który napełnia wszystko we wszystkim, mieszka w każdej osobie Trójjedynego Boga}. Twierdzenie, że Ellen White zgadzała się z takim poglądem, jest rażącym przekłamaniem.

Kontekst tego ważnego cytatu skłania do zadania istotnych pytań. Dlaczego prorok Boży odnosi się do wyobrażeń, które ilustrują \others{trójosobowość jednego Boga}, jak do \egwinline{spirytualistycznych wyobrażeń} ilustrujących pogląd, który \egwinline{nie jest godny zaufania}? Albo dlaczego prorok Boży odnosi się do wyobrażeń, które \others{przedstawiają żyjące osobowości żyjącego Boga} jako \egwinline{spirytualistyczne wyobrażenia}? Albo dlaczego prorok Boży, odnosząc się do wyobrażeń, które \others{ilustrują prawdę, że cała pełnia Tego, który napełnia wszystko we wszystkim, mieszka w każdej osobie Trójjedynego Boga}, nazywa je \egwinline{spirytualistycznymi wyobrażeniami}? Wszystkie te spirytualistyczne wyobrażenia ilustrują pogląd, który \egwinline{nie jest godny zaufania}. Ten pogląd jest wyraźnie poglądem trynitarnym.

Siostra White kontynuuje tok rozumowania Boardmana i koryguje błąd.

\begin{table}[h!]
\centering
\renewcommand{\arraystretch}{1.5}
\setlength{\tabcolsep}{15pt}
\begin{tabular}{|p{0.4\textwidth}|p{0.4\textwidth}|}
\hline
\multicolumn{1}{|c|}{\textbf{William Boardman}} & \multicolumn{1}{c|}{\textbf{Ellen G. White}} \\ \hline
\othersQuote{Ojciec jest pełnią Bóstwa \textbf{niewidzialnie}, \textbf{\underline{bez postaci}}, którego \textbf{żadne stworzenie nie widziało \underline{i nie może zobaczyć}}.}[str. 100][https://archive.org/details/higherchristian02boargoog/page/n108/]

\othersQuote{Ojciec jest pełnią Bóstwa \textbf{NIEWIDZIALNIE}.}[p.105][https://archive.org/details/higherchristian02boargoog/page/n112/] & 
\egw{Ojciec jest pełnią Bóstwa \textbf{\underline{cieleśnie}} i jest \textbf{niewidzialny dla śmiertelnego wzroku}}[Ms21-1906.9; 1906][https://egwwritings.org/read?panels=p9754.15] \\ \hline
\end{tabular}
\end{table}

Dla Boardmana Ojciec nie ma postaci ani ciała i jest niewidzialny dla wszystkich stworzeń. Dla siostry White Ojciec ma postać i ciało i jest niewidzialny tylko dla śmiertelnych istot ludzkich.\footnote{Kiedy siostra White mówi o śmiertelnikach, mówi o ludzkości skażonej grzechem. Po odnowieniu ludzkości, przy zmartwychwstaniu, Chrystus da swoje nieśmiertelne życie swoim dzieciom. Więcej informacji można znaleźć w \href{https://egwwritings.org/?ref=en_RH.July.5.1887.par.5}{EGW, RH July 5, 1887, par. 5; 1887}.}

Ten cytat jest jednym z najbardziej bezpośrednich cytatów dotyczących \emcap{osobowości Boga}. \egwinline{Ojciec jest pełnią Bóstwa \textbf{cieleśnie}}[Ms21-1906.9; 1906][https://egwwritings.org/read?panels=p9754.16].

Może być dla kogoś mylące, że Ojciec jest całą pełnią Bóstwa cieleśnie, ponieważ w \textit{Kol 2:9}, w odniesieniu do Jezusa, jest napisane, że \bible{w nim mieszka cała pełnia Bóstwa cieleśnie}. Pismo Święte nie przeczy samo sobie. \textit{Kol 2:9} nie wyklucza, że Ojciec jest całą pełnią Bóstwa cieleśnie. Różne miejsca w Biblii opisują Ojca jako mającego ciało (\textit{postać: Dn 7:9,10; Obj 4:2,3; 1Krl 22:19-22; kształt: J 5:37}). Ma On wygląd człowieka (\textit{Ez 1:26-28}). Ma twarz (\textit{Wj 33:20; Mt 18:10; Obj 22:3,4}). Jednak Biblia całkowicie milczy na temat natury jego substancji. Biblia uczy nas, że \bible{\textbf{Rzeczy tajemne należą do PANA, naszego Boga}, \textbf{ale te rzeczy, które \underline{są objawione}, należą do nas i do naszych dzieci na wieki}, abyśmy wypełniali wszystkie słowa tego prawa}[Pwt 29:29]. Zostało nam objawione, że Ojciec ma ciało, On jest całą pełnią Bóstwa cieleśnie. Zostało też objawione, że w Jezusie także mieszka cała pełnia Bóstwa cieleśnie, ponieważ \bible{upodobało się Ojcu, aby w nim zamieszkała wszelka pełnia}[Kol 1:19]. Nie jest to żadna sprzeczność, ponieważ Syn jest \bible{\textbf{dokładnym obrazem Jego osoby}}[Hbr 1:3].

\begin{table}[h!]
\centering
\renewcommand{\arraystretch}{1.5}
\setlength{\tabcolsep}{15pt}
\begin{tabular}{|p{0.4\textwidth}|p{0.4\textwidth}|}
\hline
\multicolumn{1}{|c|}{\textbf{William Boardman}} & \multicolumn{1}{c|}{\textbf{Ellen G. White}} \\ \hline
\othersQuote{Syn jest pełnią Bóstwa \textbf{ucieleśnioną, aby jego stworzenia mogły go oglądać, znać go i mu ufać}.}[str. 100][https://archive.org/details/higherchristian02boargoog/page/n108/]

\othersQuote{Syn jest pełnią Bóstwa \textbf{OBJAWIONĄ}.}[str. 105][https://archive.org/details/higherchristian02boargoog/page/n112/] & 
\egw{Syn jest całą pełnią Bóstwa \textbf{objawioną}. Słowo Boże oznajmia, że jest «\textbf{dokładnym obrazem Jego osoby}». «Bóg tak umiłował świat, że dał \textbf{swojego jednorodzonego Syna}, aby każdy, kto w Niego wierzy, nie zginął, ale miał życie wieczne». \textbf{Tutaj ukazana jest \underline{osobowość Ojca}}}[Ms21-1906.10; 1906][https://egwwritings.org/read?panels=p9754.17] \\ \hline
\end{tabular}
\end{table}

Siostra White skupiła się na \emcap{osobowości Boga}, która jest osobowością Ojca. W Chrystusie, który jest \egwinline{zrodzony na wyraźny obraz osoby Ojca}[ST May 30, 1895, par. 3; 1895][https://egwwritings.org/read?panels=p820.12891], ukazana jest osobowość Ojca. W taki sam sposób, jak Jezus jest osobą, jest nią Ojciec. Właściwość lub stan Chrystusa jako osoby jest tą samą właściwością lub stanem Ojca jako osoby. Tak jak Chrystus jest osobową istotą, tak jest i Ojciec. Tak jak cała pełnia Bóstwa cieleśnie mieszka w Chrystusie, tak mieszka i w Ojcu, ponieważ Chrystus jest zrodzony na wyraźny obraz osoby Ojca. W Nim ukazana jest osobowość Ojca. Te proste wnioski zostały potwierdzone przez Pismo Święte w J 3:16 i Hbr 1:3.

Czy to samo rozumowanie dotyczące osobowości Ojca i Syna odnosi się do Ducha Świętego? Mówiąc o Duchu Świętym, siostra White kontynuuje:

\egw{\textbf{Pocieszyciel, którego Chrystus} obiecał posłać po swoim wniebowstąpieniu, \textbf{jest Duchem \underline{w} całej pełni Bóstwa}, ukazującym moc boskiej łaski wszystkim, którzy przyjmują i wierzą w Chrystusa jako osobistego Zbawiciela.}[Ms21-1906.11; 1906][https://egwwritings.org/read?panels=p9754.18]

Siostra White wprowadza rozróżnienie między Ojcem i Synem, którzy \textbf{są}, indywidualnie, \textbf{całą} pełnią Bóstwa, a Duchem, który jest \textbf{w całej} pełni Bóstwa. Jest to wyraźny kontrast w stosunku do rozumowania Williama Boardmana, gdzie wszyscy trzej są pełnią Bóstwa. Siostra White nie podąża za tym trynitarnym wzorcem. Wyjaśnienie jest proste w świetle \emcap{osobowości Boga} i Chrystusa. Duch Święty jest duchem, a duch mieszka \textbf{w} ciele/organizmie. Duch Święty jest \textbf{w całej} pełni Bóstwa\footnote{Spójrz na cytat z \href{https://egwwritings.org/?ref=en_Ms128-1897.13&para=5426.19}{{EGW, Ms128-1897.13; 1897}}, gdzie siostra White stwierdza, że Ojciec i Syn są absolutnym Bóstwem.}.

Wreszcie cytat przechodzi do swojej najbardziej znanej części:

\begin{table}[h!]
\centering
\renewcommand{\arraystretch}{1.5}
\setlength{\tabcolsep}{15pt}
\begin{tabular}{|p{0.4\textwidth}|p{0.4\textwidth}|}
\hline
\multicolumn{1}{|c|}{\textbf{William Boardman}} & \multicolumn{1}{c|}{\textbf{Ellen G. White}} \\ \hline
\othersQuote{\textbf{Ojciec} jest całą pełnią Bóstwa NIEWIDZIALNIE.}
\othersQuote{\textbf{Syn} jest całą pełnią Bóstwa OBJAWIONĄ.}
\othersQuote{\textbf{Duch} jest całą pełnią Bóstwa OBJAWIAJĄCĄ.}
\othersQuote{\textbf{Osoby} nie są jedynie urzędami lub sposobami objawienia, \textbf{ale żyjącymi osobami żyjącego Boga}.}[str. 105][https://archive.org/details/higherchristian02boargoog/page/n112/] & 
\egw{Są \textbf{trzy żyjące osoby niebiańskiego trio}; w imię tych trzech wielkich mocy — \textbf{Ojca, Syna i Ducha Świętego} — ci, którzy przyjmują Chrystusa przez żywą wiarę, są chrzczeni, a te moce będą współpracować z posłusznymi poddanymi nieba w ich wysiłkach, aby żyć nowym życiem w Chrystusie}[Ms21-1906.11; 1906][https://egwwritings.org/read?panels=p9754.18] \\ \hline
\end{tabular}
\end{table}

W świetle kontekstu książki Williama Boardmana, widzimy wyraźny kontrast między \others{trzema żyjącymi osobami \textbf{jednego żywego Boga}}, co jest trynitarnym poglądem, a \egwinline{trzema żyjącymi osobami \textbf{niebiańskiego trio}}, co jest zgodne z prawdą o \emcap{osobowości Boga}.

Słowo ‘\textit{trio}’ po prostu wskazuje na grupę trzech. \textit{„Niebiańskie trio”} jest reprezentowane przez Ojca, Syna i Ducha Świętego. Jednak, wbrew powszechnemu przekonaniu, nie tworzą oni jednego żywego Boga. Trzy-w-jednym i jeden-w-trzech to koncepcje, które eliminują \emcap{osobowość Boga}. Dlatego siostra White określiła trynitarne poglądy jako poglądy, którym \egwinline{nie należy ufać}[Ms21-1906.8; 1906][https://egwwritings.org/read?panels=p9754.15].

Siostra White nigdy nie podążała za żadną trynitarną modą — ani w słowach i wyrażeniach, ani w poglądach. Zachęcamy do podjęcia niemal nic niekosztującego wysiłku badawczego: W pismach Ellen White poszukaj standardowych trynitarnych terminów, takich jak: „\textit{trzej są jednym}”, „\textit{jeden jest trzema}”, „\textit{jeden w trzech}”, „\textit{trzej w jednym}”, lub jakichkolwiek możliwych permutacji. W jej imponującym dorobku nie znajdziesz ani jednego wystąpienia żadnego z nich, a tym bardziej słowa ‘\textit{trójca}’ opisującego naszego Boga\footnote{Istnieje tylko jedno wystąpienie w pismach Ellen White słowa ‘\textit{trójca}’ odnoszące się do \egw{pożądliwości ciała, pożądliwości oczu i pychy życia}[Lt43-1898.25; 1898][https://egwwritings.org/read?panels=p4806.31]}. Nigdy nie używała tych fraz, które są niezbędne do wyjaśnienia trynitarnego poglądu. Analizując poniższy cytat, możemy zrozumieć, dlaczego nigdy nie powiedziała, że Bóg jest trójcą.

\egw{Tematu \textbf{\underline{spekulacji} dotyczących \underline{osobowości Boga} \underline{nie odważymy się wyrażać}}, \textbf{\underline{chyba że w języku Słowa, które przedstawia Jego osobowość}}. Nie powinno być dyskusji na ten temat, \textbf{aby Bóg nie dał jednoznacznego objawienia \underline{czym On jest}}, które zniszczyłoby tego, kto ośmiela się wkroczyć na święty teren w \textbf{swoich spekulatywnych teoriach}, jak niektórzy odważyli się to zrobić, otwierając arkę, aby zobaczyć, co w niej jest, jaka jest jej moc i jak Bóg się objawił. Ci mężczyźni zostali zabici za swoją naukę ciekawość.}[17LtMs, Ms 223, 1902, par. 16][https://egwwritings.org/read?panels=p14067.9124037&index=0]

Czy to zauważyłeś? Nie powinno być dyskusji na temat tego, czym Bóg jest, \egwinline{aby Bóg nie dał jednoznacznego objawienia} \egwinline{czym On jest}. Aby powiedzieć: „Bóg jest \_\_\_\_\_\_\_“, puste miejsce musi być wypełnione \egwinline{w języku Słowa, które przedstawia Jego osobowość}. Biblia wyraźnie naucza, że Bóg jest osobową, duchową istotą — prawda potwierdzona przez samego Chrystusa w Jego objawieniach dla Ellen White. To pasuje do biblijnego języka, który opisuje osobowość Boga. Jednak zgodnie z powyższym stwierdzeniem, czy możemy powiedzieć: „\textit{Bóg jest trójcą}?”. Nie! To nie jest \egwinline{język Słowa, które przedstawia Jego osobowość.} Dlatego, w badanym kontekście, możemy bezpiecznie stwierdzić, że trynitarny pogląd na Boga jest częścią \egwinline{spekulatywnych teorii} o tym, \egwinline{czym On jest}.

To powiedziawszy, fraza \egwinline{Niebiańskie Trio} nie jest definicją tego, czym jest Bóg. Naszym Bogiem jest Ojciec — nie \egwinline{Niebiańskie Trio}. Termin Niebiańskie Trio nie służy jako zamiennik trynitarnej idei \textit{trzech żyjących osób jednego Boga}. Staje się to oczywiste, gdy badamy kontekst. Ellen White została pouczona, aby ostrzegać nas przed trynitarnymi poglądami, a nie ufać im. Nie popierała ich.

Chociaż ilustracje, które Ellen White cytowała, nie pochodziły od dr. Kellogga, wydaje się, że zwolennicy Kellogga, jeśli nie sam Kellogg, bronili go poglądami Williama Boardmana. Nie mamy bezpośrednich danych, aby to potwierdzić, ale wiemy, że dr Kellogg poruszał \others{teologiczną stronę kwestii \textbf{trójcy i tego typu rzeczy}.}[Wywiad, J. H. Kellogg, G. W. Amadon i A. C. Bourdeau, 7 października 1907 przeprowadzony w rezydencji Kellogga][https://archive.org/details/KelloggVs.TheBrethrenHisLastInterviewAsAnAdventistoct71907/page/n37] Ostatnie trzy akapity w manuskrypcie o niebiańskim trio \href{https://egwwritings.org/?ref=en_Ms21-1906&para=9754.1}{(Ms21-1906; 1906)} ujawniają związek z dr. Kelloggiem, co jest kolejnym „koronnym dowodem” trynitarnego stanowiska dr. Kellogga.

\egw{Piszę to, ponieważ moje życie może się skończyć w każdej chwili. \textbf{Jeśli nie nastąpi zerwanie z wpływem, który przygotował Szatan, i \underline{ożywienie świadectw, które Bóg dał, dusze zginą w swoim złudzeniu}. Przyjmą fałsz za fałszem i w ten sposób będą podtrzymywać niezgodę, która zawsze będzie istnieć, dopóki ci, którzy zostali zwiedzeni, nie \underline{zajmą stanowiska na właściwej platformie}}. Cała ta wyższa edukacja, która jest planowana, zostanie wygaszona; ponieważ jest fałszywa. Im prostsze wykształcenie naszych pracowników, im mniej mają związku z ludźmi, których Bóg nie prowadzi, tym więcej zostanie osiągnięte. Praca będzie wykonywana w \textbf{\underline{prostocie} prawdziwej pobożności i dawne, dawne czasy powrócą, kiedy pod przewodnictwem Ducha Świętego tysiące nawracały się w ciągu dnia. Kiedy prawda w swojej prostocie będzie przeżywana w każdym miejscu, wtedy Bóg będzie działał przez swoich aniołów, jak działał w dniu Pięćdziesiątnicy, a serca zostaną zmienione tak zdecydowanie, że będzie manifestacja wpływu autentycznej prawdy, jak to jest przedstawione w zstąpieniu Ducha Świętego}}[Ms21-1906.18; 1906][https://egwwritings.org/read?panels=p9754.25]

\egwnogap{Duch Święty nigdy nie rozdzielił i nigdy w przyszłości nie rozdzieli medycznej pracy misyjnej od służby ewangelii. Nie mogą być rozdzielone. Związane z Jezusem Chrystusem, służba Słowa i uzdrawianie chorych są jednością}[Ms21-1906.19; 1906][https://egwwritings.org/read?panels=p9754.26]

\egwnogap{Pięćdziesiąty ósmy rozdział Księgi Izajasza zawiera instrukcje na dziś. \textbf{«Wołaj na całe gardło, nie powściągaj się, podnieś głos swój jak trąba i oznajmij mojemu ludowi jego występek, a domowi Jakuba jego grzech». Bóg nie akceptuje \underline{dr. Kellogga jako swojego pracownika}, chyba że teraz zerwie on z Szatanem}. Praca nie byłaby utrudniona, jak to miało miejsce przez ostatnich kilka lat, \textbf{gdyby dr Kellogg był nawróconym człowiekiem. «Przyjdźcie», wołam, «wyjdźcie i odłączcie się od niego i jego współpracowników, których zakwasił». Teraz przekazuję wiadomość, którą Bóg mi dał, aby przekazać wszystkim, którzy twierdzą, że wierzą w prawdę: \underline{«Wyjdźcie spośród nich i odłączcie się»}, w przeciwnym razie ich grzech usprawiedliwiania złych rzeczy i tworzenia oszustw będzie nadal ruiną dusz. Nie możemy sobie pozwolić na bycie po niewłaściwej stronie. Nie możemy sobie pozwolić na przykrywanie prawdy problemami naukowymi. Nalegamy, aby dokonano zdecydowanych zmian i nie umieszczano więcej kamieni potknięcia przed stopami ludu Bożego}. Niech każda dusza włoży buty ewangelii. \textbf{Niech każda dusza modli się i pracuje, stawiając swoje stopy na fundamencie, który położył Chrystus przez oddanie swojego życia za życie świata}}[Ms21-1906.20; 1906][https://egwwritings.org/read?panels=p9754.27]

Cytat o niebiańskim trio był częścią sporu dotyczącego Kellogga. Jest to dowód na to, że kontrowersja wokół Kellogga obejmowała doktrynę o Trójcy. Jesteśmy pouczeni, aby zerwać \egwinline{z wpływem Szatana} i ożywić \egw{świadectwo, które Bóg dał} nam, w przeciwnym razie nasze dusze zginą w złudzeniach. Te wpływy i złudzenia pochodzą od trynitarian, takich jak \textit{William Boardman} i \textit{dr John H. Kellogg}. Ona wskazuje nam, abyśmy wrócili i postawili nasze stopy na fundamencie, który został zbudowany przez Mistrza.\footnote{\href{https://egwwritings.org/?ref=en_SpTB02.54.2&para=417.276}{EGW, SpTB02 54.2; 1904}}

Mamy nadzieję, że ten kontekst demaskuje fałszywą narrację o poparciu przez Ellen White doktryny o Trójcy, propagowaną przez naszych adwentystycznych uczonych. Dr Kellogg był w odstępstwie przez odejście od fundamentu naszej wiary, a doktryna o Trójcy była jego usprawiedliwieniem. Mając takie informacje na uwadze, należy zapytać: Jeśli Trójca była prawdziwa, a Ellen White ją popierała, i ta „prawdziwa” Trójca była zmieszana z błędem dr. Kellogga, powinniśmy oczekiwać, że oddzieli ona Trójcę od błędu. Ale nie to zrobiła. Zamiast tego, konsekwentnie wskazywała nam z powrotem na fundament naszej wiary, gdzie mieliśmy jasne nauczanie o obecności i \emcap{osobowości Boga}. Lecz w przypadku Trójcy wiernie niosła przesłanie z Nieba: „\textit{\textbf{Jestem pouczona, aby powiedzieć}, że poglądy tych, którzy poszukują \textbf{trynitarnych idei, nie są godne zaufania}}”.

\begin{titledpoem}

    \stanza{
        W dziedzinie nieba prawda trwa, \\
        Przesłanie wieczną siłę ma. \\
        Bóg do nas przez Ellen przemawiał, \\
        I światło z góry nam objawiał.
    }

    \stanza{
        Nie wszystko jednak się rozumie, \\
        Gdy Boga słuchać się nie umie. \\
        Nie trójca, ale trzy osoby \\
        Różne na swoje trzy sposoby.
    }

    \stanza{
        I Ojciec, chociaż niewidzialny, \\
        Sam jak najbardziej jest realny. \\
        Ma kształt, Bóstwo na sposób ciała, \\
        W którym to mieszka pełnia cała.
    }

    \stanza{
        Zaś Syn to objawiona pełnia \\
        I boskość Go na wskroś wypełnia. \\
        Oblicze Ojca On objawia, \\
        Bez Jego łaski nie zostawia.
    }

    \stanza{
        A w Duchu pełnia zamieszkała, \\
        Choć tajemnica to niemała. \\
        Ojciec i Syn mają postacie, \\
        To z Nimi w Duchu się jednacie.
    }

    \stanza{
        Choć Ojciec z Synem postać mają \\
        I się od siebie odróżniają, \\
        Ich Duch obecny wszędzie żyje, \\
        Mówiąc, co się w Ich sercach kryje.
    }

    \stanza{
        Przesłanie cenne z wysokości \\
        O Ojca ogromnej miłości — \\
        A gdy tę prawdę już poznamy, \\
        To z naszej drogi nie zbaczamy.
    }

    \stanza{
        Zaś słowa Ellen są prawdziwe, \\
        Gdy się usunie zwierciadło krzywe. \\
        Trójcy w nich ona nie uczyła, \\
        Lecz o Niebiańskim Trio mówiła.
    }

    \stanza{
        Stoją filary, pewna platforma, \\
        Boga osoba to kształt i forma. \\
        A Boskie Trio, co jest niebiańskie, \\
        Obala trójcy kłamstwo szatańskie.
    }

\end{titledpoem}
