\chapter{Dr. Kellogg i pisma Ellen White}

Dr. Kellogg twierdził, że w książce The Living Temple przedstawił te same poglądy, które głosiła Siostra White. Podobnie, dzisiaj wielu twierdzi, że Siostra White była trynitarianką i była odpowiedzialna za przyjęcie przez kościół doktryny o Trójcy\footnote{William Johnsson, Adventist Review, 6 stycznia 1994, ‘\textit{Present Truth –Walking in God's Light}’}. Sama Siostra White oświadczyła, że takie oświadczenia są fałszywe.

\egw{\textbf{Wróg stara się \underline{wprowadzić} wśród ludu Bożego spirytualistyczne teorie, które \underline{gdyby zostały przyjęte, podważyłyby fundament wiary}, który uczynił nas tym, czym jesteśmy}. Prowadzi ludzi do głoszenia bajek pod przykrywką Pisma Świętego. \textbf{Są tacy, którzy twierdzą, że pisma Siostry White są w harmonii z tymi naukami}.\textbf{ \underline{Oświadczam, że to jest fałsz}. Ludzie mogą niewłaściwie stosować Pismo Święte; mogą błędnie interpretować moje słowa; ale Bóg rozumie ich zamysły}. Jakże jestem za to wdzięczna! Kiedy wróg nadejdzie jak powódź, \textbf{Duch Pana podniesie przeciwko niemu sztandar dla nas}.}[Ms137-1903.21; 1903][https://egwwritings.org/?ref=en\_Ms137-1903.21&para=9939.30]

Dr. Kellogg głosił teorie, które, gdyby zostały przyjęte, podważyłyby fundament naszej wiary. Kluczowe jest prawidłowe zrozumienie, co stanowi fundament naszej wiary, o którym mówiła Siostra White. Widzieliśmy, że odnosi się to do \emcap{Fundamentalnych Zasad}. Patrząc na jej pisma i pisma naszych pionierów, widzimy, że doktryna o Trójcy przeczy \emcap{osobowości Boga} i prawdzie o obecności Boga. Dziś, mając doktrynę o Trójcy jako część naszych wierzeń, uznajemy, że odeszliśmy od \emcap{Fundamentalnych Zasad} i stworzyliśmy inny fundament. Siostra White nie była odpowiedzialna za tę zmianę. Jest to wyłącznie błędna interpretacja jej dzieł. Jej pisma nie podważają fundamentu wiary, która to uczyniła nas tym, czym jesteśmy. Jej późniejsze prace są całkowicie zgodne z prawdą daną na początku.

\egw{\textbf{Ostatnie pięćdziesiąt lat nie przyćmiło ani jednej joty czy zasady naszej wiary, którą przyjęliśmy jako wielkie i wspaniałe dowody, które stały się dla nas pewne w 1844 roku, po upływie czasu.} ... \textbf{\underline{Ani jedno słowo nie zostało zmienione ani zaprzeczone}. To, co Duch Święty poświadczył jako prawdę po upływie czasu, w naszym wielkim rozczarowaniu, \underline{jest solidnym fundamentem prawdy}. \underline{Filary prawdy zostały objawione}, i przyjęliśmy \underline{podstawowe zasady}, które uczyniły nas tym, czym jesteśmy—Adwentystami Dnia Siódmego, przestrzegającymi przykazań Bożych i mając wiarę Jezusa.}}[Lt326-1905.3; 1905][https://egwwritings.org/?ref=en\_Lt326-1905.3&para=7678.9]

\section*{Błędne przedstawienie stanowiska kościoła}

Poprzez błędne przedstawianie pism Siostry White, Dr. Kellogg nie tylko fałszywie przedstawiał jej pracę, ale także oficjalne stanowisko kościoła wyrażone w \emcap{Fundamentalnych Zasadach}. Ellen White upomniała Kellogga za błędne przedstawianie stanowiska kościoła. Czytając to upomnienie, miejmy na uwadze obecne stanowisko kościoła dotyczące \emcap{osobowości Boga} w porównaniu z pierwszym punktem \emcap{Fundamentalnych Zasad}.

\egw{Nie jesteś \textbf{ugruntowany w prawdzie}. Twoje oświadczenia składane wierzącym i niewierzącym \textbf{błędnie przedstawiają nas jako lud, który nie zamienił prawdy na błąd}. Umniejszają one wpływ, \textbf{który Bóg chciałby, abyśmy mieli przed światem, ujawniając w jasnym, jednoznacznym języku, że jesteśmy \underline{wierni zasadom naszej wiary} i że zachowujemy początek naszej ufności mocny aż do końca}. Jesteśmy ściśle wyznaniowi. \textbf{Wierzymy w 1903 roku w te same prawdy, w które wierzyliśmy, gdy zakładaliśmy Sanatorium i Kolegium w Battle Creek, i \underline{wiemy, że nie mieliśmy żadnych “jeśli” ani “ale” w tej sprawie}}.}[Lt300-1903.4; 1903][https://egwwritings.org/?ref=en\_Lt300-1903.4&para=7705.10]

\egwnogap{Podczas gdy opowiadałeś te rzeczy i składałeś te oświadczenia przed niewierzącymi, moje serce było naprawdę smutne. \textbf{Dałeś dowód, że odstąpiłeś od wiary}. Te same oświadczenia, które złożyłeś przed wpływowymi ludźmi świata, jak gazety relacjonowały twoje słowa, zostały mi przedstawione wyraźnie z twoich ust, gdy je wypowiadałeś. Nie możemy pracować, aby dać ci wpływ jako komuś, komu możemy zaufać w świętej pracy związanej z naszymi instytucjami, ponieważ najpierw potrzebujesz nawrócenia i przewodnictwa.}[Lt300-1903.5; 1903][https://egwwritings.org/?ref=en\_Lt300-1903.5&para=7705.11]

\egwnogap{Nie jesteś ugruntowany w wierze. Stwierdziłam to w moim dzienniku miesiące temu. \textbf{Z pewnością przedstawiłeś lud Boży, którego Pan prowadził krok po kroku drogami prawdy i umieścił na \underline{solidnym fundamencie}, w fałszywym świetle przed niewierzącymi. Niektórzy odstąpili od wiary i \underline{będą nadal błędnie przedstawiać pracę, którą Bóg mi powierzył}}.}[Lt300-1903.6; 1903][https://egwwritings.org/?ref=en\_Lt300-1903.6&para=7705.12]

\egwnogap{\textbf{Kwestia świątyni jest jasną i określoną doktryną, którą wyznajemy jako lud. \underline{Nie masz pełnej jasności co do osobowości Boga, która jest wszystkim dla nas jako ludu}. \underline{Praktycznie zniszczyłeś samego Pana Boga}}.}[Lt300-1903.7; 1903][https://egwwritings.org/?ref=en\_Lt300-1903.7&para=7705.13]

\egwnogap{Dlaczego pozwoliłeś sobie na składanie takich oświadczeń, jakbyś miał do tego upoważnienie, skoro są one fałszywe? \textbf{Uczyniłeś fakty naszej wiary bezskutecznymi wobec niewierzących,} \textbf{a prawdę, która zawsze powinna być wyróżniona i wywyższona wśród tego ludu, praktycznie zaprzeczyłeś i zignorowałeś w swoich licznych oświadczeniach. Jak śmiałeś to zrobić?} \textbf{Zmusza nas to teraz do przedstawienia naszego prawdziwego stanowiska, które czyni nas Adwentystami Dnia Siódmego}. Jakikolwiek wpływ Bóg dał ci w przeszłości, było to z miłosierdzia dla ciebie, pozwalając, aby światło świeciło na ciebie.}[Lt300-1903.8; 1903][https://egwwritings.org/?ref=en\_Lt300-1903.8&para=7705.14]

\egwnogap{\textbf{Nie możemy ani na chwilę dopuścić do jakiegokolwiek błędnego przedstawienia tych uroczystych i ważnych tematów prawdy, które były wiarą naszego ludu od 1844 roku. To wiele dla nas znaczy.} Pan chciałby, abym powiedziała ci, że wróg poprzez swoje zwodnicze oszustwa umieścił swoją niewiarę w twoim umyśle, a ty ją realizujesz. \textbf{\underline{Wszyscy, którzy przyjmą twoje poglądy, wejdą na dziwne ścieżki, jeśli połączą się z tobą}}. \textbf{Wnosisz \underline{obcy, pospolity ogień}}, \textbf{a nie ogień rozpalony przez samego Boga}; i teraz \textbf{muszę mówić wyraźnie do naszego ludu, że Pan prowadził nas krok po kroku i pokazał nam jasne światło dotyczące niebiańskiej świątyni w najświętszym miejscu, gdzie \underline{Bóg objawił się} swoim wybranym.}}[Lt300-1903.9; 1903][https://egwwritings.org/?ref=en\_Lt300-1903.9&para=7705.15]

Dr. Kellogg błędnie przedstawił prawdę, która stanowiła fundament naszej wiary; a w szczególności błędnie przedstawił prawdę o \emcap{osobowości Boga}, która była dla nas jako ludu wszystkim. Jeśli w 1903 roku konieczne było \egwinline{\textbf{przedstawienie naszego prawdziwego stanowiska, które czyni nas Adwentystami Dnia Siódmego}}, o ile ważniejsze jest to dla nas dzisiaj? Siostra White wykonała swoją część w podtrzymywaniu fundamentu naszej wiary na początku, ale wydaje się, że o tym zapomnieliśmy.


%% Dr. Kellogg and Ellen White writings

\begin{titledpoem}
    \stanza{
        In faith's foundation, once so clear, \\
        Ellen White's words, we ought to revere. \\
        Kellogg claimed harmony, yet in deceit, \\
        Misrepresented truths, in his conceit.
    }

    \stanza{
        "The enemy seeks," she solemnly warned, \\
        To twist our beliefs, until they're scorned. \\
        Spiritualistic theories, cleverly dressed, \\
        In Scripture's garb, they falsely impressed.
    }

    \stanza{
        "False!" she declared, against the tide, \\
        Her writings and the Trinity, wrongly tied. \\
        Her defense was strong, her vision, broad, \\
        The Fundamental Principles, by God, lauded.
    }

    \stanza{    
        Not a word changed, nor principle dimmed, \\
        Since 1844, when light on faith brimmed. \\
        The pillars of truth, so firmly believed, \\
        By misinterpretations, were not deceived.
    }

    \stanza{
        Kellogg's stance, a misrepresentation, \\
        Of the faith's core, and its foundation. \\
        The personality of God, so crucial, so dear, \\
        By his theories, was muddled, we all fear.
    }

    \stanza{   
        Yet, Ellen White stood, unyielding, firm, \\
        Against falsehoods, her teachings confirm. \\
        Against the tide of Trinity's doctrine, she stood, \\
        Uplifting the truth of God and His Son, as she should.
    }

    \stanza{    
        Today, as then, let's hold the line, \\
        To the original faith, divine. \\
        Ellen White's legacy, let's rightly claim, \\
        In truth and spirit, always the same.
    }

    \stanza{    
        In battles of faith, let's not forget, \\
        Her defense of principles, firmly set. \\
        Against the currents of change and doubt, \\
        Her writings guide, without and within, throughout.
    }
\end{titledpoem}

% \begin{titledpoem}

    \stanza{
        Podstawy wiary tak przejrzyste \\
        I słowa Ellen wyraziste \\
        John Harvey w książce swej przytaczał \\
        I ich znaczenie przeinaczał.
    }

    \stanza{
        Wróg — droga Ellen ostrzegała — \\
        Wypaczy prawdę, by ściemniała. \\
        Teorie groźne i przebrane \\
        W płaszczyku Biblii są podane.
    }

    \stanza{
        „Fałsz!” — przeciw fali tak krzyczała, \\
        Błędne stwierdzenia piętnowała. \\
        Wyraźna była wizja z nieba, \\
        Że fundamentów bronić trzeba.
    }

    \stanza{
        Niech joty, kreski nie zmieniają \\
        W filarach, co od dawna trwają. \\
        Niech zaprzestaną tej „reformy” \\
        Przez Pana wzniesionej platformy.
    }

    \stanza{
        Gdyż błędna była myśl Kellogga \\
        Co do wyobrażenia Boga. \\
        Osobie Boga trójca szkodzi, \\
        Choć Kellogg śmiało jej dowodzi.
    }

    \stanza{
        Lecz Ellen nic się nie ugięła, \\
        W obronę prawdę dawną wzięła. \\
        Mimo za Trójcą głosów mnóstwa \\
        Trzymała się wciąż prawdy Bóstwa.
    }

    \stanza{
        Więc dziś się też nie uginajmy, \\
        Przy dawnej wierze obstawajmy, \\
        Bo prawda dana z objawienia \\
        Na wieki wieków się nie zmienia.
    }

    \stanza{
        Nie zapomnijmy drogi Boga, \\
        Niech na niej stanie nasza noga, \\
        A wiatr niewiary nie odbierze, \\
        Czego jej pismo wiernie strzeże.
    }

\end{titledpoem}
