\chapter{Dr. Kellogg and Ellen White writings}


\chapter{Dr. Kellogg i pisma Ellen White}


Dr. Kellogg asserted that in the Living Temple he represented the same sentiments advocated by Sister White. Likewise, today many claim that Sister White was trinitarian and was responsible for the church's acceptance of the Trinity doctrine\footnote{William Johnsson, Adventist Review, January 6th, 1994, ‘\textit{Present Truth –Walking in God’s Light}’}. Sister White, herself, declared such claims to be false.


Dr. Kellogg twierdził, że w książce The Living Temple przedstawił te same poglądy, które głosiła Siostra White. Podobnie, dzisiaj wielu twierdzi, że Siostra White była trynitarianką i była odpowiedzialna za przyjęcie przez kościół doktryny o Trójcy\footnote{William Johnsson, Adventist Review, 6 stycznia 1994, ‘\textit{Present Truth –Walking in God's Light}’}. Sama Siostra White oświadczyła, że takie twierdzenia są fałszywe.


\egw{\textbf{The enemy is seeking to \underline{bring in} among the people of God spiritualistic theories, which \underline{if accepted, would undermine the foundation of the faith} that has made us what we are}. He leads men to present fables clothed with Scripture. \textbf{There are those who assert that Sister White’s writings are in harmony with these teachings}.\textbf{ \underline{I declare this to be false}. Men may misapply Scripture; they may misinterpret my words; but God understands their devising}. How thankful I am for this! When the enemy comes in like a flood, \textbf{the Spirit of the Lord will lift up a standard for us against him}.}[Ms137-1903.21; 1903][https://egwwritings.org/?ref=en\_Ms137-1903.21&para=9939.30]


\egw{\textbf{Wróg stara się \underline{wprowadzić} wśród ludu Bożego spirytualistyczne teorie, które \underline{gdyby zostały przyjęte, podważyłyby fundament wiary}, która uczyniła nas tym, czym jesteśmy}. Prowadzi ludzi do przedstawiania bajek ubranych w Pismo Święte. \textbf{Są tacy, którzy twierdzą, że pisma Siostry White są w harmonii z tymi naukami}.\textbf{ \underline{Oświadczam, że to jest fałsz}. Ludzie mogą niewłaściwie stosować Pismo Święte; mogą błędnie interpretować moje słowa; ale Bóg rozumie ich zamysły}. Jakże jestem za to wdzięczna! Kiedy wróg nadchodzi jak powódź, \textbf{Duch Pana podniesie przeciwko niemu sztandar dla nas}.}[Ms137-1903.21; 1903][https://egwwritings.org/?ref=en\_Ms137-1903.21&para=9939.30]


Dr. Kellogg advocated the theories that, if accepted, would undermine the foundation of our faith. It is crucial to correctly understand what constitutes the foundation of our faith, which Sister White referred to. We have seen that it refers to the \emcap{Fundamental Principles}. Looking at her writings, and the writings of our pioneers, we see that the Trinity doctrine contradicts the \emcap{personality of God} and the truth about God’s presence. Today, with the Trinity doctrine as part of our belief, we recognize that we have moved away from the \emcap{Fundamental Principles} and formed another foundation. Sister White was not responsible for this transition. It is purely a misinterpretation of her works. Her writings do not undermine the foundation of the faith that has made us what we are. Her later work is completely in harmony with the truth given in the beginning.


Dr. Kellogg głosił teorie, które, gdyby zostały przyjęte, podważyłyby fundament naszej wiary. Kluczowe jest prawidłowe zrozumienie, co stanowi fundament naszej wiary, o którym mówiła Siostra White. Widzieliśmy, że odnosi się to do \emcap{Fundamentalnych Zasad}. Patrząc na jej pisma i pisma naszych pionierów, widzimy, że doktryna o Trójcy przeczy \emcap{osobowości Boga} i prawdzie o obecności Boga. Dziś, mając doktrynę o Trójcy jako część naszych wierzeń, uznajemy, że odeszliśmy od \emcap{Fundamentalnych Zasad} i stworzyliśmy inny fundament. Siostra White nie była odpowiedzialna za tę zmianę. Jest to wyłącznie błędna interpretacja jej dzieł. Jej pisma nie podważają fundamentu wiary, która uczyniła nas tym, czym jesteśmy. Jej późniejsze prace są całkowicie zgodne z prawdą daną na początku.


\egw{\textbf{The past fifty years have not dimmed one jot or principle of our faith as we received the great and wonderful evidences that were made certain to us in 1844, after the passing of the time.} ... \textbf{\underline{Not a word is changed or denied}. That which the Holy Spirit testified to as truth after the passing of the time, in our great disappointment, \underline{is the solid foundation of truth}. \underline{Pillars of truth were revealed}, and we accepted \underline{the foundation principles} that have made us what we are—Seventh-day Adventists, keeping the commandments of God and having the faith of Jesus.}}[Lt326-1905.3; 1905][https://egwwritings.org/?ref=en\_Lt326-1905.3&para=7678.9]


\egw{\textbf{Ostatnie pięćdziesiąt lat nie przyćmiło ani jednej joty czy zasady naszej wiary, którą przyjęliśmy jako wielkie i wspaniałe dowody, które stały się dla nas pewne w 1844 roku, po upływie czasu.} ... \textbf{\underline{Ani jedno słowo nie zostało zmienione ani zaprzeczone}. To, co Duch Święty poświadczył jako prawdę po upływie czasu, w naszym wielkim rozczarowaniu, \underline{jest solidnym fundamentem prawdy}. \underline{Filary prawdy zostały objawione}, i przyjęliśmy \underline{podstawowe zasady}, które uczyniły nas tym, czym jesteśmy—Adwentystami Dnia Siódmego, zachowującymi przykazania Boże i mającymi wiarę Jezusa.}}[Lt326-1905.3; 1905][https://egwwritings.org/?ref=en\_Lt326-1905.3&para=7678.9]


\section*{Misrepresentation of the church standpoint}


\section*{Błędne przedstawienie stanowiska kościoła}


By misrepresenting Sister White’s writings, Dr. Kellogg did not only misrepresent her work, but also the church’s official standpoint expressed in the \emcap{Fundamental Principles}. Ellen White rebuked Kellogg for misrepresenting the church’s standpoint. As we read this rebuke, let us keep in mind the church’s current standpoint on the \emcap{personality of God} as it compares to the first point of the \emcap{Fundamental Principles}.


Poprzez błędne przedstawianie pism Siostry White, Dr. Kellogg nie tylko zniekształcał jej pracę, ale także oficjalne stanowisko kościoła wyrażone w \emcap{Fundamentalnych Zasadach}. Ellen White upomniała Kellogga za błędne przedstawianie stanowiska kościoła. Czytając to upomnienie, miejmy na uwadze obecne stanowisko kościoła dotyczące \emcap{osobowości Boga} w porównaniu z pierwszym punktem \emcap{Fundamentalnych Zasad}.


\egw{You \textbf{are not sound in the truth}. Your statements made to believers and unbelievers \textbf{misrepresent us as a people who have not changed the truth for error}. They detract from the influence \textbf{God would have us possess before the world in revealing in plain, unmistakable language that we are \underline{true to the principles of our faith} and that we hold the beginning of our confidence firm unto the end}. We are strictly denominational. \textbf{We believe in 1903 the same truths we did believe when we established the Sanitarium and the College in Battle Creek, and \underline{we know that we had no ifs or ands about this matter}}.}[Lt300-1903.4; 1903][https://egwwritings.org/?ref=en\_Lt300-1903.4&para=7705.10]


\egw{Nie jesteś \textbf{ugruntowany w prawdzie}. Twoje oświadczenia składane wierzącym i niewierzącym \textbf{błędnie przedstawiają nas jako lud, który nie zamienił prawdy na błąd}. Umniejszają one wpływ, \textbf{który Bóg chciałby, abyśmy mieli przed światem, ujawniając w jasnym, jednoznacznym języku, że jesteśmy \underline{wierni zasadom naszej wiary} i że zachowujemy początek naszej ufności mocny aż do końca}. Jesteśmy ściśle wyznaniowi. \textbf{Wierzymy w 1903 roku w te same prawdy, w które wierzyliśmy, gdy zakładaliśmy Sanatorium i Kolegium w Battle Creek, i \underline{wiemy, że nie mieliśmy żadnych “jeśli” ani “ale” w tej sprawie}}.}[Lt300-1903.4; 1903][https://egwwritings.org/?ref=en\_Lt300-1903.4&para=7705.10]


\egwnogap{While you have told the things that you have and made the statements you have before unbelievers, my heart has been sad indeed. \textbf{You have evidenced that you have departed from the faith}. The very statements you have made before worldly men of influence, as the papers have reported your words, have been presented to me distinctly from your lips as you have spoken them. We cannot labor to give you influence as one whom we can trust with the sacred work connected with our institutions, for you need first to be converted and led.}[Lt300-1903.5; 1903][https://egwwritings.org/?ref=en\_Lt300-1903.5&para=7705.11]


\egwnogap{Podczas gdy opowiadałeś te rzeczy i składałeś te oświadczenia przed niewierzącymi, moje serce było naprawdę smutne. \textbf{Dałeś dowód, że odstąpiłeś od wiary}. Te same oświadczenia, które złożyłeś przed wpływowymi ludźmi świata, jak gazety relacjonowały twoje słowa, zostały mi przedstawione wyraźnie z twoich ust, gdy je wypowiadałeś. Nie możemy pracować, aby dać ci wpływ jako komuś, komu możemy zaufać w świętej pracy związanej z naszymi instytucjami, ponieważ najpierw potrzebujesz nawrócenia i przewodnictwa.}[Lt300-1903.5; 1903][https://egwwritings.org/?ref=en\_Lt300-1903.5&para=7705.11]


\egwnogap{You are not sound in the faith. I have stated this in my diary months ago. \textbf{You have certainly placed the people of God, whom the Lord has led step by step in the ways of truth and placed upon \underline{a solid foundation}, in a false showing before unbelievers. Some have departed from the faith and \underline{will continue to misrepresent the work God has given me}}.}[Lt300-1903.6; 1903][https://egwwritings.org/?ref=en\_Lt300-1903.6&para=7705.12]


\egwnogap{Nie jesteś ugruntowany w wierze. Stwierdziłam to w moim dzienniku miesiące temu. \textbf{Z pewnością przedstawiłeś lud Boży, którego Pan prowadził krok po kroku drogami prawdy i umieścił na \underline{solidnym fundamencie}, w fałszywym świetle przed niewierzącymi. Niektórzy odstąpili od wiary i \underline{będą nadal błędnie przedstawiać pracę, którą Bóg mi powierzył}}.}[Lt300-1903.6; 1903][https://egwwritings.org/?ref=en\_Lt300-1903.6&para=7705.12]


\egwnogap{\textbf{The sanctuary question is a clear and definite doctrine as we have held it as a people. \underline{You are not definitely clear on the personality of God, which is everything to us as a people}. \underline{You have virtually destroyed the Lord God Himself}}.}[Lt300-1903.7; 1903][https://egwwritings.org/?ref=en\_Lt300-1903.7&para=7705.13]


\egwnogap{\textbf{Kwestia świątyni jest jasną i określoną doktryną, którą wyznajemy jako lud. \underline{Nie jesteś wystarczająco jasny w kwestii osobowości Boga, która jest dla nas jako ludu wszystkim}. \underline{Praktycznie zniszczyłeś samego Pana Boga}}.}[Lt300-1903.7; 1903][https://egwwritings.org/?ref=en\_Lt300-1903.7&para=7705.13]


\egwnogap{Why should you take the liberty to make the statements which you have made, as though you had authority for thus stating, when they are falsehoods? \textbf{You have made the facts of our faith of none effect before unbelievers,} \textbf{and the truth which should ever be kept prominent and exalted with this people you have virtually denied and ignored in your many statements. How dared you to do this?} \textbf{It necessitates us now to present our true position which constitutes us Seventh-day Adventists}. Whatever influence God has given you in the past has been in mercy to you, letting the light shine upon you.}[Lt300-1903.8; 1903][https://egwwritings.org/?ref=en\_Lt300-1903.8&para=7705.14]


\egwnogap{Dlaczego pozwoliłeś sobie na składanie takich oświadczeń, jakbyś miał do tego upoważnienie, skoro są one fałszywe? \textbf{Uczyniłeś fakty naszej wiary bezskutecznymi wobec niewierzących,} \textbf{a prawdę, która zawsze powinna być wyróżniona i wywyższona wśród tego ludu, praktycznie zaprzeczyłeś i zignorowałeś w swoich licznych oświadczeniach. Jak śmiałeś to zrobić?} \textbf{Zmusza nas to teraz do przedstawienia naszego prawdziwego stanowiska, które czyni nas Adwentystami Dnia Siódmego}. Jakikolwiek wpływ Bóg dał ci w przeszłości, było to z miłosierdzia dla ciebie, pozwalając, aby światło świeciło na ciebie.}[Lt300-1903.8; 1903][https://egwwritings.org/?ref=en\_Lt300-1903.8&para=7705.14]


\egwnogap{\textbf{We cannot for a moment have any misrepresentation upon these solemn and important subjects of truth which have been the faith of our people since 1844. This means much to us.} The Lord would have me say to you that the enemy has, through his specious deceptions, placed his unbelief in your mind, and you have been working it out. \textbf{\underline{All who receive your presentations will enter upon strange paths if they connect with you}}. \textbf{You are \underline{bringing in} strange, common fire}, \textbf{but not the fire of God’s own kindling}; and now \textbf{I must speak plainly to our people that the Lord has led us step by step and shown us clear light upon the heavenly sanctuary in the most holy of holies where \underline{God revealed Himself} to His appointed ones.}}[Lt300-1903.9; 1903][https://egwwritings.org/?ref=en\_Lt300-1903.9&para=7705.15]


\egwnogap{\textbf{Nie możemy ani na chwilę dopuścić do jakiegokolwiek błędnego przedstawienia tych uroczystych i ważnych tematów prawdy, które były wiarą naszego ludu od 1844 roku. To wiele dla nas znaczy.} Pan chciałby, abym powiedziała ci, że wróg poprzez swoje zwodnicze oszustwa umieścił swoją niewiarę w twoim umyśle, a ty ją realizujesz. \textbf{\underline{Wszyscy, którzy przyjmą twoje prezentacje, wejdą na dziwne ścieżki, jeśli połączą się z tobą}}. \textbf{Wnosisz \underline{dziwny, pospolity ogień}}, \textbf{a nie ogień rozpalony przez samego Boga}; i teraz \textbf{muszę mówić wyraźnie do naszego ludu, że Pan prowadził nas krok po kroku i pokazał nam jasne światło dotyczące niebiańskiej świątyni w najświętszym miejscu, gdzie \underline{Bóg objawił się} swoim wybranym.}}[Lt300-1903.9; 1903][https://egwwritings.org/?ref=en\_Lt300-1903.9&para=7705.15]


Dr. Kellogg misrepresented the truth that constituted the foundation of our faith; most specifically, he misrepresented the truth on the \emcap{personality of God}, which was everything to us as people. If in 1903, it necessitated \egwinline{\textbf{to present our true position which constitutes us Seventh-day Adventists}}, how much more important is it for us today? Sister White did her part in upholding the foundation of our faith in the beginning, but it seems like we have forgotten.


Dr. Kellogg błędnie przedstawił prawdę, która stanowiła fundament naszej wiary; a w szczególności błędnie przedstawił prawdę o \emcap{osobowości Boga}, która była dla nas jako ludu wszystkim. Jeśli w 1903 roku konieczne było \egwinline{\textbf{przedstawienie naszego prawdziwego stanowiska, które czyni nas Adwentystami Dnia Siódmego}}, o ile ważniejsze jest to dla nas dzisiaj? Siostra White zrobiła swoją część w podtrzymywaniu fundamentu naszej wiary na początku, ale wydaje się, że o tym zapomnieliśmy.


% Dr. Kellogg and Ellen White writings

\begin{titledpoem}
    
    \stanza{
        In faith’s foundation, once so clear, \\
        Dear Ellen’s words, we should revere. \\
        J.H. agreed, yet in deceit, \\
        And twisted truth, in his conceit.
    }

    \stanza{
        The enemy—dear Ellen warned— \\
        Will twist beliefs, till they are scorned. \\
        These dangerous theories, wrongly dressed, \\
        In Scripture’s garb, the false impressed.
    }

    \stanza{
        "False!" she declared, against the tide, \\
        His crafty statements were denied. \\
        Defense was strong, her vision, broad \\
        The Fundamental were from God.
    }

    \stanza{    
        No word was changed, not peg nor pin, \\
        From when the pillars did begin. \\
        Now who would dare to move a board— \\
        This platform built up by the Lord.
    }

    \stanza{
        For Kellogg’s stance did not agree, \\
        Foundation was not trinity. \\
        God’s personality is true; \\
        Confusion came through Kellogg’s view.
    }

    \stanza{   
        Yet, Ellen stood, unyielding, firm, \\
        The early truth she did confirm. \\
        Against the tide of Trinity \\
        She held the truth of deity.
    }

    \stanza{    
        Today, as then, let’s hold the line, \\
        The early pioneer faith, divine. \\
        Truth’s legacy, let’s rightly claim, \\
        Unchanging, solid, still the same.
    }

    \stanza{    
        The way God led, let’s not forget, \\
        These principles are firmly set. \\
        Against the changing winds of doubt, \\
        Her writings guide, within, without.
    }
    
\end{titledpoem}

% \begin{titledpoem}

    \stanza{
        Podstawy wiary tak przejrzyste \\
        I słowa Ellen wyraziste \\
        John Harvey w książce swej przytaczał \\
        I ich znaczenie przeinaczał.
    }

    \stanza{
        Wróg — droga Ellen ostrzegała — \\
        Wypaczy prawdę, by ściemniała. \\
        Teorie groźne i przebrane \\
        W płaszczyku Biblii są podane.
    }

    \stanza{
        „Fałsz!” — przeciw fali tak krzyczała, \\
        Błędne stwierdzenia piętnowała. \\
        Wyraźna była wizja z nieba, \\
        Że fundamentów bronić trzeba.
    }

    \stanza{
        Niech joty, kreski nie zmieniają \\
        W filarach, co od dawna trwają. \\
        Niech zaprzestaną tej „reformy” \\
        Przez Pana wzniesionej platformy.
    }

    \stanza{
        Gdyż błędna była myśl Kellogga \\
        Co do wyobrażenia Boga. \\
        Osobie Boga trójca szkodzi, \\
        Choć Kellogg śmiało jej dowodzi.
    }

    \stanza{
        Lecz Ellen nic się nie ugięła, \\
        W obronę prawdę dawną wzięła. \\
        Mimo za Trójcą głosów mnóstwa \\
        Trzymała się wciąż prawdy Bóstwa.
    }

    \stanza{
        Więc dziś się też nie uginajmy, \\
        Przy dawnej wierze obstawajmy, \\
        Bo prawda dana z objawienia \\
        Na wieki wieków się nie zmienia.
    }

    \stanza{
        Nie zapomnijmy drogi Boga, \\
        Niech na niej stanie nasza noga, \\
        A wiatr niewiary nie odbierze, \\
        Czego jej pismo wiernie strzeże.
    }

\end{titledpoem}
