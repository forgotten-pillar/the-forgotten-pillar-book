% \qrchapterstar{https://forgottenpillar.com/rsc/en-fp-introduction}{Introduction}


\qrchapterstar{https://forgottenpillar.com/rsc/ro-fp-introduction}{Introducere}


\addcontentsline{toc}{chapter}{Introduction}


\addcontentsline{toc}{chapter}{Introducere}


This book has three objectives to fulfill. The first one is to revive the old pillar of our faith called, “\textit{the personality of God}”. The second objective is to re-establish trust in the writings of Ellen White, and the third is to re-establish the original Adventist identity.


Această carte are trei obiective de îndeplinit. Primul este să reînvie vechiul stâlp al credinței noastre numit „\textit{personalitatea lui Dumnezeu}”. Al doilea obiectiv este să restabilească încrederea în scrierile lui Ellen White, iar al treilea este să restabilească identitatea adventistă originală.


Prior to October 22, 1844, there was a great number of Adventists waiting for Christ to return on the clouds of heaven. It was a global movement of people awaiting His second coming. October 22 passed without Christ descending on the clouds and the great majority left the movement, scorning it, scorning the prophecies, the Bible, and God. Very few faithful, humble, men and women remained, who were unquestionably sure that God was leading this movement. They knew that God was shining the light of Truth and their hearts were eager to receive it. But in the eyes of the world, they were just demonstrated fanatics and dreamers. This great disappointment can be compared to the one Jesus’ disciples had after they saw their Lord being laid in the grave. They were unquestionably sure that Christ “\textit{was a prophet mighty in deed and word before God and all the people}”, but as He died on the cross, they were bitterly disappointed, because they “\textit{trusted that it had been He which should have redeemed Israel}.” Yet in their state of despair, in their state of self-disappointment, they were ready to receive the power to conquer the whole world with the Gospel. They met Christ and later received His Spirit. The same happened with the Adventist pioneers. They were a small group of people, bitterly disappointed; they sought the Lord with all their hearts and received Him in power and in Truth. The truths God revealed during this precious time of crisis constitute the foundation of Seventh-day Adventist faith. These truths were tested by all the seductive, deceptive theories of the world, by those scorning this small group, yet these grand truths prevailed. In the time of greatest need, Jesus gave His testimony by raising a little girl, the weakest of the weak, to approve all of His truths. Ellen White was not to be the source of the truths; rather, to support the brethren who were seeking the truth in the Bible. God used Ellen White to approve their studies and to point them to the Bible. The final result was the establishment of the foundation of faith based on the Bible, which standeth sure till the end of the world.


Înainte de 22 octombrie 1844, exista un număr mare de adventiști care așteptau ca Hristos să Se întoarcă pe norii cerului. Era o mișcare globală de oameni care așteptau a doua Sa venire. 22 octombrie a trecut fără ca Hristos să coboare pe nori și marea majoritate a părăsit mișcarea, batjocorindu-o, batjocorind profețiile, Biblia și pe Dumnezeu. Foarte puțini bărbați și femei credincioși, umili, au rămas, care erau neîndoielnic siguri că Dumnezeu conducea această mișcare. Ei știau că Dumnezeu făcea să strălucească lumina Adevărului și inimile lor erau dornice să o primească. Dar în ochii lumii, ei erau doar fanatici și visători dovediți. Această mare dezamăgire poate fi comparată cu cea pe care au avut-o ucenicii lui Isus după ce L-au văzut pe Domnul lor fiind pus în mormânt. Ei erau neîndoielnic siguri că Hristos „\textit{era un profet puternic în faptă și cuvânt înaintea lui Dumnezeu și a întregului popor}”, dar când a murit pe cruce, au fost amar dezamăgiți, pentru că „\textit{nădăjduiau că El era Acela care va izbăvi pe Israel}.” Totuși, în starea lor de disperare, în starea lor de auto-dezamăgire, erau gata să primească puterea de a cuceri întreaga lume cu Evanghelia. L-au întâlnit pe Hristos și mai târziu au primit Duhul Său. Același lucru s-a întâmplat cu pionierii adventiști. Erau un grup mic de oameni, amar dezamăgiți; L-au căutat pe Domnul din toată inima și L-au primit în putere și în Adevăr. Adevărurile pe care Dumnezeu le-a revelat în acest timp prețios de criză constituie fundamentul credinței adventiste de ziua a șaptea. Aceste adevăruri au fost testate de toate teoriile seducătoare, înșelătoare ale lumii, de cei care batjocoreau acest grup mic, totuși aceste adevăruri mărețe au biruit. În timpul celei mai mari nevoi, Isus Și-a dat mărturia ridicând o fetiță, cea mai slabă dintre cei slabi, pentru a aproba toate adevărurile Sale. Ellen White nu trebuia să fie sursa adevărurilor; ci mai degrabă, să-i sprijine pe frații care căutau adevărul în Biblie. Dumnezeu a folosit-o pe Ellen White pentru a aproba studiile lor și pentru a-i îndrepta către Biblie. Rezultatul final a fost stabilirea fundamentului credinței bazat pe Biblie, care stă sigur până la sfârșitul lumii.


Would you be surprised to know that the foundation of Seventh-day Adventist faith, which was laid at the beginning of our work, is in a fair degree different from what it is currently? Today, more than a century and a half later, we marvel in amazement over the accounts of the experiences of our pioneers; but since then, the Seventh-day Adventist Church has been subject to several new movements. Since then, the church has experienced many changes, including changes in our doctrine. Some argue that these changes are good and progressive; others argue that they are destructive and deceptive. Moving the spotlight to the original Seventh-day Adventism, it initiates the great controversy in the present days. We have personally been in this controversy for over 6 years now and we have seen that it will only get bigger and stronger, often with results of a sad record. Many people from both sides of this controversy are rejecting the Spirit of Prophecy in one way or another. Some have left the Seventh-day Adventist Church altogether. The Adventist identity is either lost or drastically changed from the initial one.


Ați fi surprinși să știți că fundamentul credinței adventiste de ziua a șaptea, care a fost pus la începutul lucrării noastre, este într-o măsură considerabilă diferit de ceea ce este în prezent? Astăzi, la mai mult de un secol și jumătate mai târziu, ne minunăm cu uimire de relatările experiențelor pionierilor noștri; dar de atunci, Biserica Adventistă de Ziua a Șaptea a fost supusă mai multor mișcări noi. De atunci, biserica a experimentat multe schimbări, inclusiv schimbări în doctrina noastră. Unii susțin că aceste schimbări sunt bune și progresive; alții susțin că sunt distructive și înșelătoare. Mutarea reflectorului asupra adventismului de ziua a șaptea original inițiază marea controversă din zilele prezente. Am fost personal în această controversă de peste 6 ani acum și am văzut că va deveni doar mai mare și mai puternică, adesea cu rezultate de o înregistrare tristă. Mulți oameni din ambele părți ale acestei controverse resping Spiritul Profetic într-un fel sau altul. Unii au părăsit cu totul Biserica Adventistă de Ziua a Șaptea. Identitatea adventistă este fie pierdută, fie drastic schimbată față de cea inițială.


We are currently witnessing the shaking of the Seventh-day Adventist church, seeing her tossed through one wave of crisis after another. Many are losing their faith and their identity as Seventh-day Adventists. But we believe in a solution that the Lord, in His mercy, has already provided. The solution can be found in the history of the Seventh-day Adventist movement.


În prezent suntem martori la zguduirea bisericii adventiste de ziua a șaptea, văzând-o aruncată prin val după val de criză. Mulți își pierd credința și identitatea ca adventiști de ziua a șaptea. Dar credem într-o soluție pe care Domnul, în mila Sa, a oferit-o deja. Soluția poate fi găsită în istoria mișcării adventiste de ziua a șaptea.


\egw{\textbf{In reviewing our past history}, having traveled over every step of advance to our present standing, I can say, Praise God! As I see what the Lord has wrought, I am filled with astonishment, and with confidence in Christ as leader. \textbf{We have nothing to fear for the future, \underline{except as we shall forget} the way the Lord has led us, and \underline{His teaching} in our past history}.}[LS 196.2; 1915][https://egwwritings.org/read?panels=p41.1083]


\egw{\textbf{Privind în urmă la istoria noastră trecută}, călătorind peste fiecare pas de înaintare până la poziția noastră actuală, pot spune: Laudă lui Dumnezeu! Când văd ce a lucrat Domnul, sunt umplută de uimire și de încredere în Hristos ca și conducător. \textbf{Nu avem nimic de temut pentru viitor, \underline{decât dacă vom uita} calea pe care ne-a condus Domnul și \underline{învățătura Sa} din istoria noastră trecută}.}[LS 196.2; 1915][https://egwwritings.org/read?panels=p41.1083]


We shall not fear! This is a great reassurance and promise—though conditional. We must \textit{remember} how the Lord has led us, and \textit{His teaching in our past history}. When we look at what the Lord has taught us in our past history, we are surprised to see how things have changed. The change has taken several years and many crises. To judge these changes in doctrine, whether good and progressive or bad and destructive, evaluation should be based on past experiences, as the Lord clearly led His church.


Nu ne vom teme! Aceasta este o mare asigurare și promisiune—deși condiționată. Trebuie să \textit{ne amintim} cum ne-a condus Domnul și \textit{învățătura Sa din istoria noastră trecută}. Când privim la ceea ce ne-a învățat Domnul în istoria noastră trecută, suntem surprinși să vedem cum s-au schimbat lucrurile. Schimbarea a durat mai mulți ani și multe crize. Pentru a judeca aceste schimbări în doctrină, dacă sunt bune și progresive sau rele și distructive, evaluarea ar trebui să se bazeze pe experiențele trecute, deoarece Domnul a condus în mod clar biserica Sa.


At this time, we put forth a bold claim—one that is supposed to make you hold this book until the end of its cover. Encouraged by the counsels of Ellen White to review our past history, we have concluded that we have forgotten one crucial pillar of our faith, which was the main subject of Kellogg’s controversy—the \emcap{personality of God}. One of the biggest crises that the SDA Church ever had in the time of the living prophet was the Kellogg crisis. It is out of this crisis that many other crises, today, find their roots. In this light, the subject of the \emcap{personality of God} is pivotal in our present time.


În acest moment, înaintăm o afirmație îndrăzneață—una care ar trebui să vă facă să țineți această carte până la sfârșitul coperții sale. Încurajați de sfaturile lui Ellen White de a ne revizui istoria trecută, am concluzionat că am uitat un stâlp crucial al credinței noastre, care a fost subiectul principal al controversei lui Kellogg—\emcap{personalitatea lui Dumnezeu}. Una dintre cele mai mari crize pe care Biserica AZȘ le-a avut vreodată în timpul profetului viu a fost criza Kellogg. Din această criză își găsesc rădăcinile multe alte crize de astăzi. În această lumină, subiectul \emcap{personalității lui Dumnezeu} este esențial în timpul nostru prezent.


Sister White wrote to Kellogg that the \emcap{personality of God} and the \emcap{personality of Christ} was a pillar of our faith in the same rank as is the sanctuary message:


Sora White i-a scris lui Kellogg că \emcap{personalitatea lui Dumnezeu} și \emcap{personalitatea lui Hristos} erau un stâlp al credinței noastre de același rang ca și solia sanctuarului:


\egw{Those who seek to remove \textbf{the old landmarks} are not holding fast; they \textbf{are \underline{not remembering} how they have received and heard}. Those who try to \textbf{\underline{bring in} theories that would remove \underline{the pillars of our faith} concerning the sanctuary, \underline{or concerning the personality of God or of Christ}, are working as blind men}. They are seeking to bring in uncertainties and to set the people of God adrift, without an anchor.}[Ms62-1905.14][https://egwwritings.org/read?panels=p14070.10026020]


\egw{Cei care caută să îndepărteze \textbf{pietrele de hotar vechi} nu se țin tare; ei \textbf{nu \underline{își amintesc} cum au primit și au auzit}. Cei care încearcă să \textbf{\underline{introducă} teorii care ar îndepărta \underline{stâlpii credinței noastre} cu privire la sanctuar, \underline{sau cu privire la personalitatea lui Dumnezeu sau a lui Hristos}, lucrează ca oameni orbi}. Ei caută să aducă incertitudini și să lase poporul lui Dumnezeu în derivă, fără ancoră.}[Ms62-1905.14][https://egwwritings.org/read?panels=p14070.10026020]


The \emcap{personality of God} receives very little attention today as a subject, yet it is one of the crucial elements in dealing with other doctrines pertaining to Adventism, such as the doctrine of Trinity, the Sanctuary service, 1844 and any other doctrine dealing with the Heavenly reality.


\emcap{Personalitatea lui Dumnezeu} primește foarte puțină atenție astăzi ca subiect, totuși este unul dintre elementele cruciale în abordarea altor doctrine referitoare la adventism, cum ar fi doctrina Trinității, serviciul Sanctuarului, 1844 și orice altă doctrină care se ocupă de realitatea Cerească.


The \emcap{personality of God} was a pillar of our faith. Today, it is almost forgotten. We propose a reasonable explanation for that. It is due to the evolution of the English language. What is meant by the term, “\textit{the personality of God}”? The general understanding of the English word ‘\textit{personality}’ has changed over the years. Today, ‘\textit{personality}’ is generally viewed as, “\textit{the characteristic set of behaviors, cognitions, and emotional patterns}”\footnote{Wikipedia Contributors. “\textit{Personality.}” Wikipedia, Wikimedia Foundation, 19 Apr. 2019, \href{https://en.wikipedia.org/wiki/Personality}{en.wikipedia.org/wiki/Personality}.}, but in the nineteenth, and beginning of the twentieth century, it meant “\textit{the quality or state of \textbf{being a person}}”\footnote{\href{https://www.merriam-webster.com/dictionary/personality}{Merriam-Webster Dictionary}, - ‘personality’} \footnote{\href{https://babel.hathitrust.org/cgi/pt?id=mdp.39015050663213&view=1up&seq=780}{Hunter Robert, The American encyclopaedic dictionary}, ‘\textit{personality}’ - “\textit{the quality or state of being personal}”; Mentioned dictionary was in possession of Ellen White (see \href{https://repo.adventistdigitallibrary.org/PDFs/adl-22/adl-22251050.pdf?_ga=2.116010630.1065317374.1621993520-1506151612.1617862694&fbclid=IwAR3vwmp8jxtnpPEKv0KD9mCv8dJpmRGoyIXW0CkbQAjbU0h6YaBGqhgBzbk}{EGW Private and Office Libraries})}. We read this definition as the primary definition of the word ‘\textit{personality}’ from the Merriam-Webster Dictionary\footnote{\href{https://www.merriam-webster.com/dictionary/personality\#word-history}{Merriam-Webster Dictionary} marks that the first record of the definition “the quality or state of being a person” is recorded in the 15th century.}. When Sister White and our pioneers wrote about the \emcap{personality of God}, they referred to \textit{the quality or state of God being a person}. In other words, they dealt with the question, “\textit{is God a person}”, and, “\textit{what is it that makes Him a person}” or “\textit{what is the quality or state of God being a person}”? Try to remember the last time you had a Bible study on the question, “\textit{is God a person?}” Think about how you can prove to yourself, from the Bible, that God is a person. Think about it. It is an important question. Upon this question hangs your view of God and your relationship toward Him. The \emcap{personality of God} is fundamental to true spirituality; true spirituality is based on your personal relationship with God. No real relationship of any kind can be formed with anyone unless he/she is a person. Maybe you have never asked yourself this question because you never felt a need to question if God is a person, and what is it (the quality or state) that makes Him a person. Or, maybe you were refraining from this question because you felt it might be a mystery that God did not intend to reveal. Maybe it will surprise you to know that God has given a definite and affirmative answer in His Word to the question “\textit{what is the quality or state of God being a person}”. What was even more surprising for us, was that the Adventist pioneers, including Sister White, had definite light regarding this topic, and they held it as a pillar of our faith, as part of the foundation of Seventh-day Adventist faith. When the \emcap{personality of God} is rightly understood in light of our historical past, old quotations shine in a new light and new shreds of evidence are presented, which will deepen the understanding of our past history and the present crisis.


\emcap{Personalitatea lui Dumnezeu} era un stâlp al credinței noastre. Astăzi, este aproape uitată. Propunem o explicație rezonabilă pentru aceasta. Se datorează evoluției limbii engleze. Ce se înțelege prin termenul „\textit{personalitatea lui Dumnezeu}”? Înțelegerea generală a cuvântului englez ‘\textit{personality}’ s-a schimbat de-a lungul anilor. Astăzi, ‘\textit{personalitate}’ este în general privită ca „\textit{setul caracteristic de comportamente, cogniții și tipare emoționale}”\footnote{Wikipedia Contributors. „\textit{Personality.}” Wikipedia, Wikimedia Foundation, 19 Apr. 2019, \href{https://en.wikipedia.org/wiki/Personality}{en.wikipedia.org/wiki/Personality}.}, dar în secolul al nouăsprezecelea și la începutul secolului al douăzecilea, însemna „\textit{caracteristica sau starea prin care cineva \textbf{este definit ca persoană}}”\footnote{\href{https://www.merriam-webster.com/dictionary/personality}{Merriam-Webster Dictionary}, - ‘personality’} \footnote{\href{https://babel.hathitrust.org/cgi/pt?id=mdp.39015050663213&view=1up&seq=780}{Hunter Robert, The American encyclopaedic dictionary}, ‘\textit{personality}’ - „\textit{caracteristica sau starea de a fi personal}”; Dicționarul menționat era în posesia lui Ellen White (vezi \href{https://repo.adventistdigitallibrary.org/PDFs/adl-22/adl-22251050.pdf?_ga=2.116010630.1065317374.1621993520-1506151612.1617862694&fbclid=IwAR3vwmp8jxtnpPEKv0KD9mCv8dJpmRGoyIXW0CkbQAjbU0h6YaBGqhgBzbk}{EGW Private and Office Libraries})}. Citim această definiție ca definiția primară a cuvântului ‘\textit{personalitate}’ din Dicționarul Merriam-Webster\footnote{\href{https://www.merriam-webster.com/dictionary/personality\#word-history}{Merriam-Webster Dictionary} marchează că prima înregistrare a definiției „caracteristica sau starea prin care cineva este definit ca persoană” este înregistrată în secolul al 15-lea.}. Când Sora White și pionierii noștri au scris despre \emcap{personalitatea lui Dumnezeu}, ei s-au referit la \textit{caracteristica sau starea prin care Dumnezeu este definit ca persoană}. Cu alte cuvinte, ei s-au ocupat de întrebarea „\textit{este Dumnezeu o persoană}” și „\textit{ce anume Îl face să fie o persoană}” sau „\textit{care este caracteristica sau starea prin care Dumnezeu este definit ca persoană}”? Încercați să vă amintiți ultima dată când ați avut un studiu biblic pe întrebarea „\textit{este Dumnezeu o persoană?}” Gândiți-vă cum puteți dovedi pentru voi înșivă, din Biblie, că Dumnezeu este o persoană. Gândiți-vă la aceasta. Este o întrebare importantă. De această întrebare depinde perspectiva voastră despre Dumnezeu și relația voastră cu El. \emcap{Personalitatea lui Dumnezeu} este fundamentală pentru adevărata spiritualitate; adevărata spiritualitate se bazează pe relația voastră personală cu Dumnezeu. Nicio relație reală de orice fel nu poate fi formată cu cineva decât dacă el/ea este o persoană. Poate că nu v-ați pus niciodată această întrebare pentru că nu ați simțit niciodată nevoia să vă întrebați dacă Dumnezeu este o persoană și ce anume (caracteristica sau starea) Îl face să fie o persoană. Sau, poate v-ați abținut de la această întrebare pentru că ați simțit că ar putea fi un mister pe care Dumnezeu nu a intenționat să îl descopere. Poate vă va surprinde să știți că Dumnezeu a dat un răspuns definitiv și afirmativ în Cuvântul Său la întrebarea „\textit{care este caracteristica sau starea prin care Dumnezeu este definit ca persoană}”. Ceea ce a fost și mai surprinzător pentru noi a fost că pionierii adventiști, inclusiv Sora White, au avut lumină definită cu privire la acest subiect și l-au considerat ca un stâlp al credinței noastre, ca parte din temelia credinței adventiste de ziua a șaptea. Când \emcap{personalitatea lui Dumnezeu} este înțeleasă corect în lumina trecutului nostru istoric, citatele vechi strălucesc într-o lumină nouă și sunt prezentate noi dovezi, care vor aprofunda înțelegerea istoriei noastre trecute și a crizei prezente.


The root problem of the Kellogg crisis was about the \emcap{personality of God}. It is certainly important to evaluate Kellogg's crisis over the \emcap{personality of God} using the meaning intended at that time; that is, using the definition of ‘\textit{personality},’ as the quality or state of God being a person. With this definition in mind, the Kellogg crisis comes into a new light and new relevant evidence is brought forth for us today. In light of this evidence, we see how God has led us in the past; thus, we should not fear for the future. Knowing and understanding this, as well as its importance, helps us to not be shaken by any wave of deception in present controversies. When Sister White was drawing Kellogg’s attention to the importance of this subject, she was drawing our attention also, as it is everything to us as a people.


Problema de bază a crizei Kellogg era despre \emcap{personalitatea lui Dumnezeu}. Este cu siguranță important să evaluăm criza lui Kellogg privind \emcap{personalitatea lui Dumnezeu} folosind sensul intenționat la acea vreme; adică, folosind definiția de ‘\textit{personalitate}’ ca fiind caracteristica sau starea prin care Dumnezeu este definit ca persoană. Cu această definiție în minte, criza Kellogg apare într-o lumină nouă și sunt aduse dovezi noi relevante pentru noi astăzi. În lumina acestor dovezi, vedem cum Dumnezeu ne-a condus în trecut; astfel, nu ar trebui să ne temem pentru viitor. Cunoașterea și înțelegerea acestui lucru, precum și importanța sa, ne ajută să nu fim clătinați de niciun val de înșelăciune în controversele prezente. Când Sora White îi atrăgea atenția lui Kellogg asupra importanței acestui subiect, ea ne atrăgea și nouă atenția, deoarece este totul pentru noi ca popor.


[Writing to Kellogg] \egw{You are not definitely clear on \textbf{the personality of God}, which is \textbf{\underline{everything} to us as a people}.}[Lt300-1903.7][https://egwwritings.org/read?panels=p14068.7705013]


[Scriind lui Kellogg] \egw{Nu ești clar în mod definitiv cu privire la \textbf{personalitatea lui Dumnezeu}, care este \textbf{\underline{totul} pentru noi ca popor}.}[Lt300-1903.7][https://egwwritings.org/read?panels=p14068.7705013]


These studies on the \emcap{personality of God} will prompt a lot of new and hard questions. We do not promise to answer all of them, and perhaps you won’t be satisfied with the answers provided, but we pray, hope and believe that this book will fulfill the three objectives proposed in the beginning of this introduction. Through the reviving of the doctrine on the \emcap{personality of God}, we believe that your confidence in the Spirit of Prophecy will strengthen, and that you’ll find yourself rooted deeper in the Adventist message—where we find our identity as people—making you a more faithful Seventh-day Adventist. Most importantly, we want you to become more aware of God as your personal God. This will surely strengthen and deepen your relationship with Him.


Aceste studii despre \emcap{personalitatea lui Dumnezeu} vor genera multe întrebări noi și dificile. Nu promitem să răspundem la toate și poate nu veți fi mulțumiți de răspunsurile oferite, dar ne rugăm, sperăm și credem că această carte va îndeplini cele trei obiective propuse la începutul acestei introduceri. Prin reînvierea doctrinei despre \emcap{personalitatea lui Dumnezeu}, credem că încrederea voastră în Spiritul Profetic se va întări și că vă veți găsi înrădăcinați mai adânc în mesajul adventist—unde ne găsim identitatea ca popor—făcându-vă un adventist de ziua a șaptea mai credincios. Cel mai important, vrem să deveniți mai conștienți de Dumnezeu ca Dumnezeul vostru personal. Aceasta va întări și va aprofunda cu siguranță relația voastră cu El.


We find answers to the issue on the \emcap{personality of God} in examining the Kellogg crisis, where Sister White gave the most definite light on the \emcap{personality of God} and on the foundation of Seventh-day Adventist faith. The following is the complete tenth chapter from the book, \textit{Testimonies for the Church Containing Letters to Physicians and Ministers Instruction to Seventh-Day Adventists}. This chapter, \textit{The Foundation of our Faith}, contains deep insight into the history of Kellogg’s crisis. It gives a historical overview of the truths God gave as the foundation of our faith and in these truths we find our identity as Seventh-day Adventists— keeping the commandments of God and having the faith of Jesus.


Găsim răspunsuri la problema privind \emcap{personalitatea lui Dumnezeu} examinând criza Kellogg, unde Sora White a dat cea mai clară lumină asupra \emcap{personalității lui Dumnezeu} și asupra temeliei credinței adventiste de ziua a șaptea. Următorul este capitolul zece complet din cartea \textit{Mărturii pentru Biserică – Scrisori către medici și pastori, îndrumări pentru adventiștii de ziua a șaptea}. Acest capitol, \textit{Temelia credinței noastre}, conține o perspectivă profundă asupra istoriei crizei lui Kellogg. Oferă o privire istorică asupra adevărurilor pe care Dumnezeu le-a dat ca temelie a credinței noastre și în aceste adevăruri ne găsim identitatea ca adventiști de ziua a șaptea—păzind poruncile lui Dumnezeu și având credința lui Isus.
