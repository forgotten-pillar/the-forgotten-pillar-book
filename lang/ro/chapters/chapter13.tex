% \qrchapter{https://forgottenpillar.com/rsc/en-fp-chapter13}{The Sabbath God vs. Sunday God - J. B. Frisbie}


\qrchapter{https://forgottenpillar.com/rsc/ro-fp-chapter13}{Dumnezeul Sabatului vs. Dumnezeul duminicii - J. B. Frisbie}


There are other articles written on the \emcap{personality of God} by our pioneers and it would be too much to include everything here, but we would like to add one more testimony from brother J. B. Frisbie’s article where he compares the Sabbath God with the Sunday God. He compares the truth on the \emcap{personality of God} expressed in the first point of the \emcap{Fundamental Principles} with the Trinity doctrine. Let us take a look at a portion of his article, “\textit{The Seventh Day-Sabbath Not Abolished}” from the Review and Herald, March 7, 1854.


Există și alte articole scrise despre \emcap{personalitatea lui Dumnezeu} de către pionierii noștri și ar fi prea mult să includem totul aici, dar am dori să adăugăm încă o mărturie din articolul fratelui J. B. Frisbie unde compară Dumnezeul Sabatului cu Dumnezeul duminicii. El compară adevărul despre \emcap{personalitatea lui Dumnezeu} exprimat în primul punct al \emcap{Principiilor Fundamentale} cu doctrina Trinității. Să aruncăm o privire asupra unei porțiuni din articolul său, “\textit{Sabatul de ziua a șaptea nu a fost desființat}” din Review and Herald, 7 martie 1854.


\begin{figure}[hp]
    \centering
    \includegraphics[width=1\linewidth]{images/j-b-frisbie.jpg}
    \caption*{John Byington Frisbie (1816-1882)}
    \label{fig:j-b-frisbie}
\end{figure}


\begin{figure}[hp]
    \centering
    \includegraphics[width=1\linewidth]{images/j-b-frisbie.jpg}
    \caption*{John Byington Frisbie (1816-1882)}
    \label{fig:j-b-frisbie}
\end{figure}


\section*{The Sabbath God}


\section*{Dumnezeul Sabatului}


\others{After we know and remember God, by keeping his holy Sabbath, \textbf{then the Bible will teach of his personality and dwelling place}. \textbf{Man is in the image and likeness of God}. Genesis 1:26. ‘And God said, Let us (speaking to his son) make man in our image, after our likeness’. Chap 2:7. ‘And the Lord God formed man of the dust of the ground, and breathed into his nostrils the breath of life: and man became a living soul’. Genesis 9:6; 1 Corinthians 11:7; James 3:9. \textbf{That which was made in \underline{the image and likeness of God} was made of the dust of the ground called man}.}


\others{După ce Îl cunoaștem și ne amintim de Dumnezeu, păzind Sabatul Său sfânt, \textbf{atunci Biblia ne va învăța despre personalitatea Sa și locul Său de locuire}. \textbf{Omul este după chipul și asemănarea lui Dumnezeu}. Geneza 1:26. ‘Apoi Dumnezeu a zis: «Să facem (vorbind către fiul Său) om după chipul Nostru, după asemănarea Noastr㻑. Cap. 2:7. ‘Domnul Dumnezeu a făcut pe om din țărâna pământului, i-a suflat în nări suflare de viață, și omul s-a făcut astfel un suflet viu’. Geneza 9:6; 1 Corinteni 11:7; Iacov 3:9. \textbf{Ceea ce a fost făcut după \underline{chipul și asemănarea lui Dumnezeu} a fost făcut din țărâna pământului și numit om}.}


\othersnogap{This is known to be the true sense from other testimonies that may be given from the Bible. \textbf{Jesus was in the form of a man and the express image of his Father’s person}.}


\othersnogap{Aceasta se știe că este sensul adevărat din alte mărturii care pot fi date din Biblie. \textbf{Isus a fost în forma unui om și Întipărirea persoanei Tatălui Său}.}


\othersnogap{Philippians 2:6-8. \textbf{Christ Jesus}: ‘Who, being in \textbf{the form of God}, thought it not robbery to be \textbf{equal with God}. But made himself of no reputation, and took upon him \textbf{the form of a servant}, and was \textbf{made in the likeness of men’}. 2 Corinthians 4:4. \textbf{‘And being formed in fashion as a man’}, etc. Colossians 1:15. ‘\textbf{Who is the image of the invisible God}’.}


\othersnogap{Filipeni 2:6-8. \textbf{Hristos Isus}: ‘El, măcar că avea \textbf{chipul lui Dumnezeu}, totuși n-a crezut ca un lucru de apucat să fie \textbf{deopotrivă cu Dumnezeu}, ci S-a dezbrăcat pe Sine însuși și a luat \textbf{un chip de rob}, făcându-Se \textbf{asemenea oamenilor’}. 2 Corinteni 4:4. \textbf{‘La înfățișare a fost găsit ca un om’}, etc. Coloseni 1:15. ‘\textbf{El este chipul Dumnezeului celui nevăzut}’.}


\othersnogap{Hebrews 1:3. \textbf{The Son; ‘Who being the brightness of his glory, and the express image of his person’}. In this sense could Jesus say to Philip in truth, ‘He that hath seen me hath seen the Father.’ John 14:9. Some seem to suppose it argues \textbf{against the personality of God, \underline{because he is a Spirit, and say that he is without body, or parts}}. John 4:24. ‘\textbf{God is a Spirit}’. Hebrews 1:7. ‘\textbf{Who maketh his angels spirits}’. \textbf{Who would pretend to say that angels have no bodies or parts because they are spirits}. \textbf{\underline{None the less is God a spiritual being having body and parts as we may learn by his having a dwelling place and because he has and may be seen}}. Exodus 33:23. ‘And I will take away mine hand, and thou shalt\textbf{ see my back parts}, but my \textbf{face shall not be seen}’. Matthew 5:8. ‘Blessed are the pure in heart, for \textbf{they shall see God}’. Hebrews 12:14. ‘Follow peace with all men, and holiness, without which \textbf{no man shall see the Lord}’. Matthew 18:10. ‘That in heaven their angels do \textbf{always behold the face of my Father which is in heaven}’. Matthew 6:9. ‘After this manner therefore pray ye, \textbf{Our Father which art in heaven}’, etc. John 6:38. ‘For I \textbf{came down from heaven} not to do mine own will, but the will of him that sent me’. Chap 16:28. ‘\textbf{I came forth from the Father, and am come into the world}: again I \textbf{leave the world, and go to the Father}’.}


\othersnogap{Evrei 1:3. \textbf{Fiul; ‘El, care este oglindirea slavei Lui și Întipărirea persoanei Lui’}. În acest sens putea Isus să-i spună lui Filip în adevăr: ‘Cine M-a văzut pe Mine a văzut pe Tatăl.’ Ioan 14:9. Unii par să presupună că aceasta pledează \textbf{împotriva personalității lui Dumnezeu, \underline{pentru că El este Duh, și spun că El este fără trup sau părți}}. Ioan 4:24. ‘\textbf{Dumnezeu este Duh}’. Evrei 1:7. ‘\textbf{El face din îngerii Lui duhuri}’. \textbf{Cine ar pretinde să spună că îngerii nu au trupuri sau părți pentru că sunt duhuri}. \textbf{\underline{Cu nimic mai puțin Dumnezeu este o ființă spirituală având trup și părți, după cum putem învăța din faptul că are un loc de locuire și pentru că El există și poate fi văzut}}. Exod 33:23. ‘Apoi Îmi voi trage mâna la o parte, și \textbf{Mă vei vedea pe dinapoi}, dar \textbf{Fața nu Mi se poate vedea}’. Matei 5:8. ‘Ferice de cei cu inima curată, căci \textbf{ei vor vedea pe Dumnezeu}’. Evrei 12:14. ‘Urmăriți pacea cu toți oamenii și sfințirea, fără care \textbf{nimeni nu va vedea pe Domnul}’. Matei 18:10. ‘Feriți-vă să nu disprețuiți pe vreunul din acești micuți; căci vă spun că îngerii lor în ceruri \textbf{văd pururea fața Tatălui Meu care este în ceruri}’. Matei 6:9. ‘Iată dar cum trebuie să vă rugați: \textbf{«Tatăl nostru care ești în ceruri»}’, etc. Ioan 6:38. ‘Căci \textbf{M-am coborât din cer} ca să fac nu voia Mea, ci voia Celui ce M-a trimis’. Cap. 16:28. ‘\textbf{Am ieșit de la Tatăl și am venit în lume}; acum \textbf{las lumea și Mă duc la Tatăl}’.}


\othersnogap{\textbf{Does not God say he fills immensity of space? \underline{We answer, No}}. Psalm 139:7, 8. ‘Whither shall I go \textbf{from thy Spirit}? or whither shall I flee \textbf{from thy presence}? If I ascend up into heaven, thou art there’, etc. \textbf{\underline{God by his Spirit may fill heaven and earth}}, etc. \textbf{Some confound God with his Spirit, which makes confusion}. Psalm 11:4. ‘\textbf{The Lord is in his holy temple, the Lord’s throne is in heaven}: his eyes behold’, etc. Habakkuk 2:20; Psalm 102:19. ‘For he hath looked \textbf{down from the height of his Sanctuary}; \textbf{\underline{from heaven} did the Lord behold the earth’}. 1 Peter 3:12. ‘For the eyes of the Lord are over the righteous, and his ears are open unto their prayers’, etc. Psalm 80:1. ‘Give ear, O Shepherd of Israel, thou that leadest Joseph like a flock; thou \textbf{that dwellest between the cherubims}, shine forth’. Psalm 99:1; Isaiah 37:16.}


\othersnogap{\textbf{Nu spune Dumnezeu că umple imensitatea spațiului? \underline{Răspundem: Nu}}. Psalm 139:7, 8. ‘Unde mă voi duce departe \textbf{de Duhul Tău}? Și unde voi fugi departe \textbf{de Fața Ta}? Dacă mă voi sui în cer, Tu ești acolo’, etc. \textbf{\underline{Dumnezeu prin Duhul Său poate umple cerul și pământul}}, etc. \textbf{Unii Îl confundă pe Dumnezeu cu Duhul Său, ceea ce creează confuzie}. Psalm 11:4. ‘\textbf{Domnul este în Templul Lui cel sfânt, Domnul Își are scaunul de domnie în ceruri}. Ochii Lui privesc’, etc. Habacuc 2:20; Psalm 102:19. ‘Căci El privește \textbf{din înălțimea sfințeniei Lui}; \textbf{\underline{din ceruri} Domnul Își aruncă privirile spre pământ’}. 1 Petru 3:12. ‘Căci ochii Domnului sunt peste cei neprihăniți, și urechile Lui iau aminte la rugăciunile lor’, etc. Psalm 80:1. ‘Ia aminte, Păstorul lui Israel, Tu, care povățuiești pe Iosif ca pe o turmă! Arată-Te în strălucirea Ta, Tu, care \textbf{șezi pe heruvimi}’. Psalm 99:1; Isaia 37:16.}


\othersnogap{John 14:2. ‘In my Father’s house are many mansions. I go to prepare a place for you’. Revelation 21:2-5; Hebrews 11:6. ‘For he that cometh to God must believe that he is’, etc. \textbf{This testimony we deem highly important at this time, to know that there is a God. We have no doubt that if our eyes could be opened in vision, or see as angels see, we should see God in heaven sitting on his throne, and is present to all that exists, however distant from him in his creation}.}[\href{https://documents.adventistarchives.org/Periodicals/RH/RH18540307-V05-07.pdf}{Adventist Review and Sabbath Herald, March 7, 1854}, J. B. Frisbie, “The Seventh-Day Sabbath Not Abolished”, p. 50]


\othersnogap{Ioan 14:2. ‘În casa Tatălui Meu sunt multe locașuri. Mă duc să vă pregătesc un loc’. Apocalipsa 21:2-5; Evrei 11:6. ‘Și fără credință este cu neputință să fim plăcuți Lui! Căci cine se apropie de Dumnezeu trebuie să creadă că El este’, etc. \textbf{Această mărturie o considerăm foarte importantă în acest timp, să știm că există un Dumnezeu. Nu avem nicio îndoială că dacă ochii noștri ar putea fi deschiși în viziune, sau să vedem cum văd îngerii, L-am vedea pe Dumnezeu în cer șezând pe tronul Său, și este prezent la tot ce există, oricât de departe de El în creația Sa}.}[\href{https://documents.adventistarchives.org/Periodicals/RH/RH18540307-V05-07.pdf}{Adventist Review and Sabbath Herald, 7 martie 1854}, J. B. Frisbie, “Sabatul de ziua a șaptea nu a fost desființat”, p. 50]


Here we see the same argument and reasoning, that God is a personal spiritual Being. This God is the Sabbath God. Brother Frisbie compares this God with the Sunday God, who is a trinitarian God.


Aici vedem același argument și raționament, că Dumnezeu este o Ființă spirituală personală. Acest Dumnezeu este Dumnezeul Sabatului. Fratele Frisbie compară acest Dumnezeu cu Dumnezeul duminicii, care este un Dumnezeu trinitarian.


\section*{The Sunday God}


\section*{Dumnezeul duminicii}


\others{We will make a few extracts, that the reader may \textbf{see the broad contrast between \underline{the God of the Bible} brought to light through Sabbath-keeping, and the god in the dark through Sunday-keeping}. Catholic Catechism Abridged by the Rt. Rev. John Dubois, Bishop of New York. Page 5. ‘\textbf{Ques. Where is God? Ans. God is everywhere}. Q. Does God see and know all things? A. Yes, he does know and see all things. \textbf{Q. Has God any body? A. \underline{No; God has no body, he is a pure Spirit}}. \textbf{Q. Are there more Gods than one? A. No; there is but one God. Q. Are there more persons than one in God? A. \underline{Yes; in God there are three persons}. Q. Which are they? A. God the Father, God the Son and God the Holy Ghost. Q. Are there not three Gods? A. No; the Father, the Son and the Holy Ghost, are all but one and the same God}’.}


\others{Vom face câteva extrase, pentru ca cititorul să poată \textbf{vedea contrastul larg între \underline{Dumnezeul Bibliei} adus la lumină prin păzirea Sabatului, și dumnezeul din întuneric prin păzirea duminicii}. Catehismul Catolic Prescurtat de Preasfințitul Rev. John Dubois, Episcop de New York. Pagina 5. ‘\textbf{Întrebare. Unde este Dumnezeu? Răspuns. Dumnezeu este pretutindeni}. Î. Vede și știe Dumnezeu toate lucrurile? R. Da, El știe și vede toate lucrurile. \textbf{Î. Are Dumnezeu vreun trup? R. \underline{Nu; Dumnezeu nu are trup, El este un Duh pur}}. \textbf{Î. Sunt mai mulți Dumnezei decât unul? R. Nu; există doar un singur Dumnezeu. Î. Sunt mai multe persoane decât una în Dumnezeu? R. \underline{Da; în Dumnezeu sunt trei persoane}. Î. Care sunt ele? R. Dumnezeu Tatăl, Dumnezeu Fiul și Dumnezeu Duhul Sfânt. Î. Nu sunt trei Dumnezei? R. Nu; Tatăl, Fiul și Duhul Sfânt, sunt toți unul și același Dumnezeu}’.}


\othersnogap{The first article of the Methodist Religion, p. 8. \textbf{‘There is but one living and true God}, everlasting, \textbf{without body or parts}, of infinite power, wisdom and goodness: the maker and preserver of all things, visible and invisible. \textbf{And in unity of this God-head, there are three persons of one substance, power and eternity; the Father, the Son, and the Holy Ghost}.’}


\othersnogap{Primul articol al Religiei Metodiste, p. 8. \textbf{‘Există doar un singur Dumnezeu viu și adevărat}, veșnic, \textbf{fără trup sau părți}, cu putere, înțelepciune și bunătate infinite: făcătorul și păstrătorul tuturor lucrurilor, vizibile și invizibile. \textbf{Și în unitatea acestei Dumnezeiri, sunt trei persoane de o singură substanță, putere și eternitate; Tatăl, Fiul și Duhul Sfânt}.’}


\othersnogap{In this article like the Catholic doctrine, \textbf{we are taught that there are three persons of one substance,} power and eternity making\textbf{ in all one living and true God}, everlasting \textbf{without body or parts}. But in all this we are not told \textbf{what became of the body of Jesus who had a body when he ascended, who went to God who ‘is everywhere’ or nowhere}. Doxology.}


\othersnogap{În acest articol, asemenea doctrinei catolice, \textbf{suntem învățați că există trei persoane de o singură substanță,} putere și eternitate formând\textbf{ în total un singur Dumnezeu viu și adevărat}, veșnic \textbf{fără trup sau părți}. Dar în toate acestea nu ni se spune \textbf{ce s-a întâmplat cu trupul lui Isus care avea un trup când S-a înălțat, care S-a dus la Dumnezeu care ‘este pretutindeni’ sau nicăieri}. Doxologie.}


\othersnogap{‘\textbf{To God the Father, God the Son,}} \\
\others{\textbf{God the Spirit, three in one.}’} \\
\others{Again} \\
\others{‘Warms in the sun, refreshes in the breeze,} \\
\others{Glows in the stars, and blossoms in the trees.} \\
\others{\textbf{Lives through all life, extends through all extent},} \\
\others{Spreads undivided and operates unspent.’ - Pope.}


\othersnogap{‘\textbf{Lui Dumnezeu Tatăl, Dumnezeu Fiul,}} \\
\others{\textbf{Dumnezeu Duhul, trei în unul.}’} \\
\others{Din nou} \\
\others{‘Se încălzește în soare, înviorează în briză,} \\
\others{Strălucește în stele și înflorește în copaci.} \\
\others{\textbf{Trăiește prin toată viața, se extinde prin toată întinderea},} \\
\others{Se răspândește neîmpărțit și operează necheltuind.’ - Pope.}


\othersnogap{These ideas well accord with those heathen philosophers. One says, ‘That water was the principle of all things, and that God is that intelligence, by whom all things are formed out of water.’ Another, ‘That air is God, that it is produced, that it is immense and infinite,’ etc. A third, ‘That God is a soul diffused throughout all beings of nature,’ etc. \textbf{Some, who had the idea of \underline{a pure Spirit}}. Last of all, ‘That God is an eternal substance.’}


\othersnogap{Aceste idei se potrivesc bine cu cele ale filozofilor păgâni. Unul spune: ‘Că apa era principiul tuturor lucrurilor, și că Dumnezeu este acea inteligență, prin care toate lucrurile sunt formate din apă.’ Altul: ‘Că aerul este Dumnezeu, că este produs, că este imens și infinit,’ etc. Un al treilea: ‘Că Dumnezeu este un suflet răspândit prin toate ființele naturii,’ etc. \textbf{Unii, care aveau ideea de \underline{un Duh pur}}. În cele din urmă: ‘Că Dumnezeu este o substanță eternă.’}


\othersnogap{These extracts are taken from Rollin’s History, Vol. II, pp. 597-8, published by Harpers. \textbf{We should rather mistrust that the Sunday god came from the same source that Sunday-keeping did}. ‘Sunday was a name given by the heathens to the first day of the week, because it was the day on which they worshiped the sun.’ - Union Bible Dictionary. \textbf{Afterward modified by the Roman Catholic Church, in the form we now find it taught through the land}.}


\othersnogap{Aceste extrase sunt luate din Istoria lui Rollin, Vol. II, pp. 597-8, publicată de Harpers. \textbf{Ar trebui mai degrabă să bănuim că dumnezeul duminicii a venit din aceeași sursă din care a venit și păzirea duminicii}. ‘Duminica era un nume dat de păgâni primei zile a săptămânii, pentru că era ziua în care ei se închinau soarelui.’ - Dicționarul Biblic Union. \textbf{Ulterior modificat de Biserica Romano-Catolică, în forma în care îl găsim acum învățat prin țară}.}


\othersnogap{It is very natural to suppose when \textbf{the Pope set himself up to be God in the temple of God}, [2 Thessalonians 2:4] that he should have a day sanctified to his worship. This he has done. - Douay Catechism, p. 59. ‘Q. What is the best means to sanctify Sunday? A. By hearing mass, etc. This saying mass is for the priest to gabble over Latin, drink some wine, and give the people a wafer to eat.’}”“\others{But God sanctified his day because he had rested on it. Another day for a very different purpose. Genesis 2:3.}


\othersnogap{Este foarte natural să presupunem că atunci când \textbf{Papa s-a ridicat să fie Dumnezeu în templul lui Dumnezeu}, [2 Tesaloniceni 2:4] el ar trebui să aibă o zi sfințită pentru închinarea sa. Aceasta a făcut-o. - Catehismul Douay, p. 59. ‘Î. Care este cel mai bun mijloc de a sfinți duminica? R. Prin ascultarea slujbei, etc. Această slujbă constă în a bolborosi preotul în latină, a bea puțin vin și a da poporului o napolitană de mâncat.’}”“\others{Dar Dumnezeu a sfințit ziua Sa pentru că Se odihnise în ea. O altă zi pentru un scop foarte diferit. Geneza 2:3.}


\othersnogap{In days before the moral fall of Babylon God directed the minds of his honest children right in their prayers, whatever they might think at other times, but now since the apostasy the mind reaches to no god but to the people only, there are many prayers to men we know by their effect and eloquence. \textbf{We are truly thankful to our heavenly Father that \underline{he has led our minds from such folly}, to know, and remember \underline{his holy name} by keeping his holy day that we might love, serve and worthily \underline{glorify him through our great High Priest in the heavenly Sanctuary in this day of atonement}}.}[Ibid.][https://documents.adventistarchives.org/Periodicals/RH/RH18540307-V05-07.pdf]


\othersnogap{În zilele dinaintea căderii morale a Babilonului, Dumnezeu îndrepta mințile copiilor Săi cinstiți corect în rugăciunile lor, orice ar fi gândit ei în alte momente, dar acum de la apostazie mintea nu ajunge la niciun dumnezeu ci doar la oameni, există multe rugăciuni către oameni pe care le cunoaștem prin efectul și elocvența lor. \textbf{Suntem cu adevărat recunoscători Tatălui nostru ceresc că \underline{ne-a condus mințile departe de o asemenea nebunie}, să cunoaștem și să ne amintim \underline{numele Său sfânt} păzind ziua Sa sfântă pentru ca să putem iubi, sluji și \underline{glorifica în mod vrednic pe El prin Marele nostru Preot din Sanctuarul ceresc în această zi a ispășirii}}.}[Ibid.][https://documents.adventistarchives.org/Periodicals/RH/RH18540307-V05-07.pdf]


Before becoming a Seventh-day Adventist, Frisbie was a Methodist preacher and a bitter opponent of Adventist beliefs. In 1853, after a debate on the Sabbath with Joseph Bates, he reversed his position and began to keep the Sabbath and preach the Seventh-day Adventist doctrine. He renounced Sunday, the Trinity, and accepted the Seventh-day Sabbath and the truth about God, that the Seventh-day Adventist’s taught in the first point of the \emcap{Fundamental Principles}.


Înainte de a deveni adventist de ziua a șaptea, Frisbie era un predicator metodist și un oponent înverșunat al credințelor adventiste. În 1853, după o dezbatere despre Sabat cu Joseph Bates, și-a schimbat poziția și a început să păzească Sabatul și să predice doctrina adventistă de ziua a șaptea. A renunțat la duminică, la Trinitate și a acceptat Sabatul de ziua a șaptea și adevărul despre Dumnezeu, pe care adventiștii de ziua a șaptea îl învățau în primul punct al \emcap{Principiilor Fundamentale}.


Do other Adventist pioneers see discordance between the Trinity doctrine and the \emcap{personality of God} expressed in the first point of the \emcap{Fundamental Principles}?


Văd alți pionieri adventiști o discordanță între doctrina Trinității și \emcap{personalitatea lui Dumnezeu} exprimată în primul punct al \emcap{Principiilor Fundamentale}?


% The Sabbath God vs. Sunday God - J. B. Frisbie

\begin{titledpoem}
    
    \stanza{
        On seventh day or first we kneel, \\
        But deeper truths these days reveal. \\
        Not just when we choose to pray, \\
        But which God we serve each day.
    }

    \stanza{
        The Sabbath God, a Being clear, \\
        With form and place, both far and near. \\
        In His image we were made, \\
        His Son the perfect likeness displayed. \\
    }

    \stanza{
        The Son, the Father's image bright, \\
        Shows us the path to truth and light. \\
        "Who's seen me has seen the Father too," \\
        Christ's words both powerful and true.
    }

    \stanza{
        The Sunday God, a trinity, \\
        Three persons in strange unity. \\
        Without body, without part, \\
        A concept born from human art.
    }

    \stanza{
        One God with face and hands and form, \\
        Who rested when creation's storm \\
        Had ceased its work on seventh day, \\
        This God commands we rest and pray.
    }

    \stanza{
        Not some essence spreading wide, \\
        Formless spirit with no side. \\
        But a Person on a throne, \\
        With His Son, yet not alone.
    }

    \stanza{
        So choose not merely when to kneel, \\
        But which God your heart finds real. \\
        The day we keep reveals our view \\
        Of which God we believe is true.
    }
    
\end{titledpoem}