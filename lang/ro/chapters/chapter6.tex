% \qrchapter{https://forgottenpillar.com/rsc/en-fp-chapter6}{Examining the test}


\qrchapter{https://forgottenpillar.com/rsc/ro-fp-chapter6}{Examinarea testului}


In Sister White's reply, to Dr. Kellogg's belief on the Trinity doctrine and his attempts to \textit{patch up} the Living Temple, we see that she viewed the Trinity doctrine as contradicting the light given her regarding \emcap{the personality of God}. If she had actually embraced the Trinity doctrine, we would expect her to carefully separate it from pantheism and preserve its legitimate aspects. However, this is not what we see in her response. Instead, her response was to contrast the Trinity doctrine with the truth about the \emcap{personality of God}, recalling her past visions which showed that this doctrine would rob God's people of their past experiences. In her reactive recalling of how God established the \emcap{fundamental principles}, she indicated that the Trinity doctrine \textit{tears down the pillars of our faith} and \textit{leads us astray from the foundation principles}. This stark difference can be clearly seen by comparing our current Fundamental Beliefs with the \emcap{Fundamental Principles} held in the past.


În răspunsul surorii White către credința Dr. Kellogg despre doctrina Trinității și încercările sale de a \textit{petici} Templul viu, vedem că ea considera doctrina Trinității ca fiind în contradicție cu lumina dată ei cu privire la \emcap{personalitatea lui Dumnezeu}. Dacă ar fi îmbrățișat cu adevărat doctrina Trinității, ne-am aștepta ca ea să o separe cu grijă de panteism și să păstreze aspectele sale legitime. Cu toate acestea, aceasta nu este ceea ce vedem în răspunsul ei. În schimb, răspunsul ei a fost să pună în contrast doctrina Trinității cu adevărul despre \emcap{personalitatea lui Dumnezeu}, amintindu-și de viziunile ei din trecut care arătau că această doctrină va jefui poporul lui Dumnezeu de experiențele lor din trecut. În reamintirea ei reactivă despre cum Dumnezeu a stabilit \emcap{principiile fundamentale}, ea a indicat că doctrina Trinității \textit{dărâmă stâlpii credinței noastre} și \textit{ne duce în rătăcire de la principiile de bază}. Această diferență marcantă poate fi văzută clar comparând Punctele noastre Fundamentale de Credință actuale cu \emcap{Principiile Fundamentale} deținute în trecut.


Keeping in mind Sister White’s reply to Dr. Kellogg's belief on the Trinity doctrine, let us review the characteristics of the theories she described in the chapter “\textit{The Foundation of our Faith}”. When Sister White is speaking of Kellogg’s theories of God, our question should be, “do her quotations make sense if the Trinity doctrine is applied to their context?” Let’s examine each characteristic.


Având în minte răspunsul surorii White la credința Dr. Kellogg despre doctrina Trinității, să revizuim caracteristicile teoriilor pe care le-a descris în capitolul “\textit{Temelia credinței noastre}”. Când sora White vorbește despre teoriile lui Kellogg despre Dumnezeu, întrebarea noastră ar trebui să fie: “au sens citatele ei dacă doctrina Trinității este aplicată contextului lor?” Să examinăm fiecare caracteristică.


\subsection*{Does the Trinity “rob the people of God of their past experience”?}


\subsection*{Jefuiește Trinitatea “poporul lui Dumnezeu de experiența lor din trecut”?}


\egw{They \normaltext{[the spiritualistic theories]} make of no effect the truth of heavenly origin, and \textbf{rob the people of God of their past experience}, giving them instead a false science.}[SpTB02 54.1; 1904][https://egwwritings.org/read?panels=p417.275]


\egw{Ele \normaltext{[teoriile spiritualiste]} fac fără efect adevărul de origine cerească și \textbf{jefuiesc poporul lui Dumnezeu de experiența lor din trecut}, dându-le în schimb o știință falsă.}[SpTB02 54.1; 1904][https://egwwritings.org/read?panels=p417.275]


\egw{This foundation was built by the Masterworker, and will stand storm and tempest. Will they permit this man \normaltext{[Kellogg]} to present \textbf{doctrines that deny the past experience of the people of God}? The time has come to take decided action.}[SpTB02 54.2; 1904][https://egwwritings.org/read?panels=p417.276]


\egw{Această temelie a fost construită de Meșterul Lucrător și va rezista furtunii și vijeliei. Vor permite ei acestui om \normaltext{[Kellogg]} să prezinte \textbf{doctrine care neagă experiența din trecut a poporului lui Dumnezeu}? A venit timpul să luăm măsuri hotărâte.}[SpTB02 54.2; 1904][https://egwwritings.org/read?panels=p417.276]


\egw{\textbf{What influence is it that would lead men at this stage of our history to work in an underhanded, powerful way to \underline{tear down the foundation of our faith},—the foundation that was laid \underline{at the beginning of our work} by prayerful study of the word and by revelation? Upon this foundation \underline{we have been building for the past fifty years}}. Do you wonder that when I see the beginning of a work \textbf{that would \underline{remove some of the pillars of our faith},} I have something to say? I must obey the command, ‘Meet it!’}[SpTB02 58.1; 1904][https://egwwritings.org/read?panels=p417.295]


\egw{\textbf{Ce influență este aceea care i-ar conduce pe oameni în această etapă a istoriei noastre să lucreze într-un mod ascuns, puternic pentru a \underline{dărâma temelia credinței noastre},—temelia care a fost pusă \underline{la începutul lucrării noastre} prin studiul cu rugăciune al cuvântului și prin revelație? Pe această temelie \underline{am construit în ultimii cincizeci de ani}}. Vă mirați că atunci când văd începutul unei lucrări \textbf{care ar \underline{îndepărta unii dintre stâlpii credinței noastre},} am ceva de spus? Trebuie să ascult de porunca: ‘Înfruntă-o!’}[SpTB02 58.1; 1904][https://egwwritings.org/read?panels=p417.295]


According to Sister White’s testimony, the foundation of our faith was the \emcap{Fundamental Principles}. Currently, these do not represent our beliefs. Most objectionable is the first point, concerning who God is. Instead of the belief that there is one God—the Father, a personal spiritual being, we have a new belief that there is one God—Father, Son, and Holy Spirit, a unity of three Persons. From the light and the experiences of how God established the first point of the \emcap{Fundamental Principles}, does the newly formed doctrine about who God is and what He is, has robbed the people of God of their past experience?


Conform mărturiei surorii White, temelia credinței noastre erau \emcap{Principiile Fundamentale}. În prezent, acestea nu reprezintă credințele noastre. Cel mai inacceptabil este primul punct, referitor la cine este Dumnezeu. În loc de credința că există un singur Dumnezeu—Tatăl, o ființă spirituală personală, avem o nouă credință că există un singur Dumnezeu—Tatăl, Fiul și Duhul Sfânt, o unitate a trei Persoane. Din lumina și experiențele despre cum Dumnezeu a stabilit primul punct al \emcap{Principiilor Fundamentale}, a jefuit doctrina nou formată despre cine este Dumnezeu și ce este El poporul lui Dumnezeu de experiența lor din trecut?


\subsection*{Does the Trinity tear down the pillars of our faith, or lead astray from foundation principles?}


\subsection*{Dărâmă Trinitatea stâlpii credinței noastre sau duce în rătăcire de la principiile de bază?}


\egw{I have been instructed by the heavenly messenger that some of the reasoning in the book, ‘Living Temple,’ is unsound and that \textbf{this reasoning would lead astray the minds of those who are not thoroughly established on the foundation principles of present truth.}}[SpTB02 51.3; 1904][https://egwwritings.org/read?panels=p417.262]


\egw{Am fost instruită de mesagerul ceresc că unele dintre raționamentele din cartea ‘Templul viu’ sunt nesănătoase și că \textbf{aceste raționamente ar duce în rătăcire mințile celor care nu sunt pe deplin stabiliți pe principiile de bază ale adevărului prezent.}}[SpTB02 51.3; 1904][https://egwwritings.org/read?panels=p417.262]


\egw{About the time that ‘Living Temple’ was published, there passed before me in the night season, representations indicating that some \textbf{danger was approaching}, and that I must prepare for it by writing out the things God has revealed to me \textbf{regarding the foundation principles of our faith}.}[SpTB02 52.3; 1904][https://egwwritings.org/read?panels=p417.267]


\egw{Cam în timpul când a fost publicat „Templul viu”, au trecut înaintea mea în timpul nopții reprezentări care indicau că se apropie \textbf{un pericol}, și că trebuie să mă pregătesc pentru el scriind lucrurile pe care Dumnezeu mi le-a descoperit \textbf{cu privire la principiile de bază ale credinței noastre}.}[SpTB02 52.3; 1904][https://egwwritings.org/read?panels=p417.267]


\egw{\textbf{The enemy of souls has sought to bring in the supposition that a great reformation was to take place among Seventh-day Adventists, and that this reformation would consist in \underline{giving up the doctrines which stand as the pillars of our faith,} and engaging in a process of reorganization}. Were this reformation to take place, what would result? \textbf{The principles of truth} that God in His wisdom has given to the remnant church, \textbf{would be discarded}. Our religion would be changed. \textbf{The fundamental principles} that have sustained the work for the last fifty years \textbf{would be accounted as error}. A new organization would be established. Books of a new order would be written. A system of intellectual philosophy would be introduced.}[SpTB02 54.3; 1904][https://egwwritings.org/read?panels=p417.277]


\egw{\textbf{Vrăjmașul sufletelor a căutat să aducă presupunerea că o mare reformă urma să aibă loc printre adventiștii de ziua a șaptea, și că această reformă ar consta în \underline{renunțarea la doctrinele care stau ca stâlpii credinței noastre,} și angajarea într-un proces de reorganizare}. Dacă această reformă ar avea loc, care ar fi rezultatul? \textbf{Principiile adevărului} pe care Dumnezeu în înțelepciunea Sa le-a dat bisericii rămășiței, \textbf{ar fi abandonate}. Religia noastră ar fi schimbată. \textbf{Principiile fundamentale} care au susținut lucrarea în ultimii cincizeci de ani \textbf{ar fi considerate ca eroare}. O nouă organizație ar fi înființată. Cărți de un nou ordin ar fi scrise. Un sistem de filozofie intelectuală ar fi introdus.}[SpTB02 54.3; 1904][https://egwwritings.org/read?panels=p417.277]


Dr. Kellogg’s theories on the \emcap{personality of God}, if accepted, would ignite a reformation within the Seventh-day Adventist Church. Based on intellectual philosophy, they would cause us to renounce some of the doctrines that stand as the pillars of our faith, condemning the \emcap{Fundamental Principles} as error. Could it be that by adhering to the Trinity doctrine we entered into a new organization?


Teoriile Dr. Kellogg despre \emcap{personalitatea lui Dumnezeu}, dacă ar fi acceptate, ar aprinde o reformă în cadrul Bisericii Adventiste de Ziua a Șaptea. Bazate pe filozofie intelectuală, ele ne-ar face să renunțăm la unele dintre doctrinele care stau ca stâlpii credinței noastre, condamnând \emcap{Principiile Fundamentale} ca eroare. Oare prin aderarea la doctrina Trinității am intrat într-o nouă organizație?


\egw{Shortly before I sent out the testimonies \textbf{regarding the efforts of the enemy to undermine the foundation of our faith through the dissemination of seductive theories}, I had read an incident about a ship in a fog meeting an iceberg…}[SpTB02 55.3; 1904][https://egwwritings.org/read?panels=p417.282]


\egw{Cu puțin timp înainte de a trimite mărturiile \textbf{cu privire la eforturile vrăjmașului de a submina temelia credinței noastre prin diseminarea teoriilor seducătoare}, citisem un incident despre o corabie în ceață care întâlnește un aisberg...}[SpTB02 55.3; 1904][https://egwwritings.org/read?panels=p417.282]


\egw{Messages of every order and kind have been \textbf{urged upon Seventh-day Adventists, to take the place of the truth which, \underline{point by point}, has been sought out by prayerful study, and testified to by the miracle-working power of the Lord}. \textbf{But the way-marks which have made us what we are, are to be preserved, and they will be preserved}, as God has signified through His word and the testimony of His Spirit. \textbf{He calls upon us to hold firmly}, with the grip of faith, \textbf{to \underline{the fundamental principles} that are based upon \underline{unquestionable authority}}.}[SpTB02 59.1; 1904][https://egwwritings.org/read?panels=p417.299]


\egw{Mesaje de orice fel și natură au fost \textbf{impuse asupra adventiștilor de ziua a șaptea, pentru a lua locul adevărului care, \underline{punct cu punct}, a fost căutat prin studiu plin de rugăciune și mărturisit prin puterea făcătoare de minuni a Domnului}. \textbf{Dar pietrele de hotar care ne-au făcut ceea ce suntem trebuie să fie păstrate, și ele vor fi păstrate}, așa cum Dumnezeu a indicat prin Cuvântul Său și mărturia Duhului Său. \textbf{El ne cheamă să ținem cu tărie}, cu strânsoarea credinței, \textbf{de \underline{principiile fundamentale} care sunt bazate pe \underline{autoritate de necontestat}}.}[SpTB02 59.1; 1904][https://egwwritings.org/read?panels=p417.299]


The \emcap{personality of God} was the pillar of our faith\footnote{\href{https://egwwritings.org/?ref=en_Ms62-1905.14}{EGW, Ms62-1905.14; 1905}}. The \emcap{personality of God} was expressed in the first point of the \emcap{Fundamental Principles}. Could it be that by adhering to the Trinity doctrine we have torn down this particular pillar of our faith? Is it possible that by accepting the Trinity doctrine we were led astray from this foundation principle—the \emcap{personality of God}?


\emcap{Personalitatea lui Dumnezeu} era stâlpul credinței noastre\footnote{\href{https://egwwritings.org/?ref=en_Ms62-1905.14}{EGW, Ms62-1905.14; 1905}}. \emcap{Personalitatea lui Dumnezeu} era exprimată în primul punct al \emcap{Principiilor Fundamentale}. Oare prin aderarea la doctrina Trinității am dărâmat acest stâlp particular al credinței noastre? Este posibil ca prin acceptarea doctrinei Trinității să fi fost duși în rătăcire de la acest principiu de bază—\emcap{personalitatea lui Dumnezeu}?


\subsection*{Does the Trinity do away with the personality of God?}


\subsection*{Anulează Trinitatea personalitatea lui Dumnezeu?}


\egw{\textbf{It \normaltext{[The Living Temple]} introduces that which is naught but \underline{speculation} in regard to \underline{the personality of God} and where His presence is.}}[SpTB02 51.3; 1904][https://egwwritings.org/read?panels=p417.262]


\egw{\textbf{Ea \normaltext{[Templul viu]} introduce ceea ce nu este altceva decât \underline{speculație} cu privire la \underline{personalitatea lui Dumnezeu} și unde este prezența Sa.}}[SpTB02 51.3; 1904][https://egwwritings.org/read?panels=p417.262]


\egw{\textbf{The spiritualistic theories \underline{regarding the personality of God}, followed to their logical conclusion, sweep away the whole Christian economy.}}[SpTB02 54.1; 1904][https://egwwritings.org/read?panels=p417.275]


\egw{\textbf{Teoriile spiritualiste \underline{cu privire la personalitatea lui Dumnezeu}, urmărite până la concluzia lor logică, mătură întreaga economie creștină.}}[SpTB02 54.1; 1904][https://egwwritings.org/read?panels=p417.275]


\egw{‘Living Temple’ contains the alpha of these theories. I knew that the omega would follow in a little while; and I trembled for our people. I knew that \textbf{I must warn our brethren and sisters not to enter into controversy over \underline{the presence} and \underline{personality of God}. The statements made in ‘Living Temple’ \underline{in regard to this point are incorrect}. The scripture used to substantiate the doctrine there set forth, is scripture misapplied}.}[SpTB02 53.2; 1904][https://egwwritings.org/read?panels=p417.271]


\egw{„Templul viu” conține alfa acestor teorii. Știam că omega va urma în scurt timp; și am tremurat pentru poporul nostru. Știam că \textbf{trebuie să-i avertizez pe frații și surorile noastre să nu intre în controversă asupra \underline{prezenței} și \underline{personalității lui Dumnezeu}. Afirmațiile făcute în „Templul viu” \underline{cu privire la acest punct sunt incorecte}. Scriptura folosită pentru a susține doctrina expusă acolo este Scriptură aplicată greșit}.}[SpTB02 53.2; 1904][https://egwwritings.org/read?panels=p417.271]


The theories Kellogg presented in the Living Temple are speculative in regard to the \emcap{personality of God} and where His presence is. These theories deal with the question of the quality or state of God being a person\footnote{The Merriam-Webster definition of ‘\textit{personality}’ - “\textit{the quality or state of being a person}”}. God has given us definite light regarding this issue in our \emcap{Fundamental Principles}. Could it be that the Trinity doctrine is casting doubt on this definite light regarding the \emcap{personality of God}?


Teoriile pe care Kellogg le-a prezentat în Templul viu sunt speculative în ceea ce privește \emcap{personalitatea lui Dumnezeu} și unde este prezența Sa. Aceste teorii se ocupă cu problema caracteristicii sau stării prin care Dumnezeu este definit ca persoană\footnote{Definiția Merriam-Webster pentru ‘\textit{personalitate}’ - “\textit{caracteristica sau starea prin care cineva este definit ca persoană}”}. Dumnezeu ne-a dat lumină clară cu privire la această problemă în \emcap{Principiile noastre fundamentale}. Oare s-ar putea ca doctrina Trinității să pună la îndoială această lumină clară cu privire la \emcap{personalitatea lui Dumnezeu}?


\subsection*{Is the Trinity doctrine presented as if Mrs. White supported it?}


\subsection*{Este doctrina Trinității prezentată ca și cum doamna White ar fi susținut-o?}


\egw{In the controversy that arose among our brethren \textbf{regarding the teachings of this book,} those in favor of giving it a wide circulation \textbf{declared: ‘It contains the very sentiments that Sister White has been teaching.’ This assertion struck right to my heart. I felt heart-broken; for I knew that this representation of the matter was not true}.}[SpTB02 53.1; 1904][https://egwwritings.org/read?panels=p417.270]


\egw{În controversa care a apărut între frații noștri \textbf{cu privire la învățăturile acestei cărți,} cei în favoarea unei circulații largi a ei \textbf{au declarat: ‘Conține chiar opiniile pe care sora White le-a învățat.’ Această afirmație m-a lovit direct în inimă. M-am simțit cu inima frântă; căci știam că această reprezentare a situației nu era adevărată}.}[SpTB02 53.1; 1904][https://egwwritings.org/read?panels=p417.270]


\egw{\textbf{I am compelled to speak in denial of the claim that the teachings of ‘Living Temple’ can be sustained by statements from my writings}. There may be in this book expressions and sentiments that are in harmony with my writings. And there may be in my writings many statements which, taken from their connection, and interpreted according to the mind of the writer of ‘Living Temple,’ would seem to be in harmony with the teachings of this book. This may give apparent support to the assertion that the sentiments in ‘Living Temple’ are in harmony with my writings. \textbf{But God forbid that this sentiment should prevail}.}[SpTB02 53.3; 1904][https://egwwritings.org/read?panels=p417.272]


\egw{\textbf{Sunt constrânsă să vorbesc negând afirmația că învățăturile din ‘Templul viu’ pot fi susținute prin declarații din scrierile mele}. Pot exista în această carte expresii și opinii care sunt în armonie cu scrierile mele. Și pot exista în scrierile mele multe declarații care, luate din contextul lor și interpretate conform minții autorului cărții ‘Templul viu’, ar părea să fie în armonie cu învățăturile acestei cărți. Aceasta poate da un sprijin aparent afirmației că opiniile din ‘Templul viu’ sunt în armonie cu scrierile mele. \textbf{Dar ferească Dumnezeu ca această opinie să prevaleze}.}[SpTB02 53.3; 1904][https://egwwritings.org/read?panels=p417.272]


At this point, we have many unanswered questions. But, as we continue to study the first point of the \emcap{Fundamental Principles}, we will find answers to all of these questions. So far, in light of the \emcap{Fundamental Principles}, belief in the Trinity doctrine—as a Seventh-day Adventist—becomes very questionable. In order to defend the Trinity doctrine, the authority of the \emcap{Fundamental Principles} must be compromised. In what follows, we will briefly study their authority, context in Adventist history, and God’s purpose in giving them. We will also look at the true authorship of the \emcap{Fundamental Principles} and their role in present days.


În acest punct, avem multe întrebări fără răspuns. Dar, pe măsură ce continuăm să studiem primul punct al \emcap{Principiilor fundamentale}, vom găsi răspunsuri la toate aceste întrebări. Până acum, în lumina \emcap{Principiilor fundamentale}, credința în doctrina Trinității—ca adventist de ziua a șaptea—devine foarte îndoielnică. Pentru a apăra doctrina Trinității, autoritatea \emcap{Principiilor fundamentale} trebuie compromisă. În cele ce urmează, vom studia pe scurt autoritatea lor, contextul în istoria adventistă și scopul lui Dumnezeu în a le da. Vom analiza, de asemenea, adevărata paternitate a \emcap{Principiilor fundamentale} și rolul lor în zilele prezente.


% Examining Test

\begin{titledpoem}
    
    \stanza{
        The visions stand against the tide \\
        And all false doctrines are denied \\
        The testimony, clear and bright \\
        Expose the false, and bring forth light.
    }

    \stanza{
        The pillars which were set with care \\
        Now face a challenge, so beware \\
        The platform built by God’s wise plan \\
        Is weakened now by wayward man.
    }

    \stanza{
        God is a person, God’s church knew \\
        But since forgot, by words untrue \\
        Our past experience was robbed \\
        Untempered mortar has been daubed.
    }

    \stanza{
        The waymarks made us what we are, \\
        Should guide us still, our guiding star. \\
        Hold principles with faith’s strong grip, \\
        Lest in the fog we lose our ship.
    }
    
\end{titledpoem}