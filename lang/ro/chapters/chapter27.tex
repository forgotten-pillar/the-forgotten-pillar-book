% \qrchapter{https://forgottenpillar.com/rsc/en-fp-chapter27}{Steps to apostasy}


\qrchapter{https://forgottenpillar.com/rsc/ro-fp-chapter27}{Pași către apostazie}


In the following quotation, brother J. N. Loughborough, who was one of the pioneers of the Seventh-day Adventist Church, warned us about the five steps to apostasy.


În citatul următor, fratele J. N. Loughborough, care a fost unul dintre pionierii Bisericii Adventiste de Ziua a Șaptea, ne-a avertizat despre cei cinci pași către apostazie.


\others{\textbf{The} \textbf{first step} of apostasy is to \textbf{get up a creed}, telling us what we shall believe. \textbf{The second} is to \textbf{make that creed a test of fellowship}. \textbf{The third} is to \textbf{try members by that creed}. \textbf{The fourth} is to \textbf{denounce as heretics those who do not believe that creed}. And \textbf{fifth}, to \textbf{commence persecution against such}. I plead that we are not patterning after the churches in any unwarrantable sense in the step proposed.}[John N. Loughborough, Review and Herald, Oct. 8, 1861.][https://egwwritings.org/read?panels=p1685.5326]


\others{\textbf{Primul pas} al apostaziei este să \textbf{stabilești un crez}, spunându-ne ce să credem. \textbf{Al doilea} este să \textbf{faci din acel crez un test al părtășiei}. \textbf{Al treilea} este să \textbf{încerci membrii prin acel crez}. \textbf{Al patrulea} este să \textbf{denunți ca eretici pe cei care nu cred acel crez}. Și \textbf{al cincilea}, să \textbf{începi persecuția împotriva acestora}. Pledez că nu urmăm modelul bisericilor în niciun sens nejustificat în pasul propus.}[John N. Loughborough, Review and Herald, 8 oct. 1861.][https://egwwritings.org/read?panels=p1685.5326]


These principles are important to have in mind, and we ought to ask ourselves if we, today, are patterning after the churches in any unwarrantable sense in the step proposed. What would happen to a Seventh-day Adventist who would reject the Trinity doctrine in favor of the \emcap{Fundamental Principles}? Do we have a creed set up in our church? Do we test our membership by it?


Aceste principii sunt importante de avut în minte și ar trebui să ne întrebăm dacă noi, astăzi, urmăm modelul bisericilor în vreun sens nejustificat în pasul propus. Ce s-ar întâmpla cu un adventist de ziua a șaptea care ar respinge doctrina Trinității în favoarea \emcap{Principiilor Fundamentale}? Avem un crez stabilit în biserica noastră? Ne testăm membrii prin el?


The \emcap{Fundamental Principles} had a different nature and role in the Seventh-day Adventist Church contrary to that of the pattern held by other churches. The \emcap{Fundamental Principles} were not designed as a creed. In the preface of the 1872 statement, we read about their nature:


\emcap{Principiile Fundamentale} au avut o natură și un rol diferit în Biserica Adventistă de Ziua a Șaptea, contrar modelului deținut de alte biserici. \emcap{Principiile Fundamentale} nu au fost concepute ca un crez. În prefața declarației din 1872, citim despre natura lor:


\others{In presenting to the \textbf{public} this \textbf{synopsis of our faith}, we wish to have it distinctly understood that \textbf{\underline{we have no articles of faith, creed}, or discipline, \underline{aside from the Bible}}. We \textbf{do not} put forth this \textbf{\underline{as having any authority with our people}}, \textbf{nor is it designed to secure uniformity among them}, \textbf{as a system of faith}, \textbf{but is a brief statement of what is, and has been, with great unanimity, held by them}.}[A Declaration of the Fundamental Principles, Taught and Practiced by the Seventh-Day Adventists, 1872]


\others{Prezentând \textbf{publicului} această \textbf{sinteză a credinței noastre}, dorim să fie înțeles clar că \textbf{\underline{nu avem articole de credință, crez} sau disciplină, \underline{în afară de Biblie}}. \textbf{Nu} prezentăm aceasta \textbf{\underline{ca având vreo autoritate asupra poporului nostru}}, \textbf{nici nu este concepută pentru a asigura uniformitatea între ei}, \textbf{ca un sistem de credință}, \textbf{ci este o scurtă declarație a ceea ce este și a fost, cu mare unanimitate, susținut de ei}.}[O declarație a principiilor fundamentale, crezute și practicate de adventiștii de ziua a șaptea, 1872]


In the preface of the 1889 statement, we read similar sentiments:


În prefața declarației din 1889, citim opinii similare:


\others{As elsewhere stated, Seventh-day Adventists \textbf{have no creed but the Bible}; but they hold to \textbf{certain well-defined points of faith}, for which they \textbf{feel prepared to give a reason ‘to every man that asketh’ them}. The following propositions may be taken as a summary of \textbf{the principal features of their religious faith}, upon which there is, so far as we know, \textbf{entire unanimity throughout the body}.}[Seventh-day Adventist Year Book of statistics for 1889, pg. 147, The Fundamental Principles of Seventh-day Adventists]


\others{După cum s-a afirmat în altă parte, adventiștii de ziua a șaptea \textbf{nu au alt crez decât Biblia}; dar ei susțin \textbf{anumite puncte de credință bine definite}, pentru care se \textbf{simt pregătiți să dea socoteală ‘oricui le cere’}. Propozițiile următoare pot fi luate ca un rezumat al \textbf{principalelor caracteristici ale credinței lor religioase}, asupra cărora există, din câte știm, \textbf{unanimitate deplină în întregul corp}.}[Anuarul statistic al adventiștilor de ziua a șaptea pentru 1889, pg. 147, Principiile fundamentale ale adventiștilor de ziua a șaptea]


The \emcap{Fundamental Principles} were not designed to dictate someone’s faith. The believers, led by the Holy Spirit, freely rendered their consciences to the Word of God; under the influence of the Holy Spirit, they came to the same conclusions. There was entire unanimity throughout the body. All believers felt “\textit{prepared to give a reason to every man that asketh them}” regarding their faith.


\emcap{Principiile Fundamentale} nu au fost concepute pentru a dicta credința cuiva. Credincioșii, conduși de Duhul Sfânt, și-au supus liber conștiința Cuvântului lui Dumnezeu; sub influența Duhului Sfânt, au ajuns la aceleași concluzii. Exista unanimitate deplină în întregul corp. Toți credincioșii se simțeau “\textit{pregătiți să dea socoteală oricui le cere}” cu privire la credința lor.


Today we see a striking difference in the principles and practice of Adventist beliefs compared to our pioneers. We are keeping the spirit of unity by disciplining our members for the denial of the Fundamental Beliefs. In our church manual, under the section “\textit{Reason for Disciplines}”, we read the first point which states the discipline for denial of faith in the Seventh-day Adventist Fundamental Beliefs.


Astăzi vedem o diferență izbitoare în principiile și practica credințelor adventiste comparativ cu pionierii noștri. Păstrăm spiritul unității disciplinându-ne membrii pentru negarea Punctelor Fundamentale de Credință. În manualul nostru bisericesc, sub secțiunea “\textit{Motive pentru disciplină}”, citim primul punct care prevede disciplina pentru negarea credinței în Punctele Fundamentale de Credință ale Adventiștilor de Ziua a Șaptea.


\others{Reasons for Discipline}


\others{Motive pentru disciplină}


\others{1. \textbf{Denial of faith} in the fundamentals of the gospel and \textbf{in the Fundamental Beliefs of the Church} or \textbf{teaching doctrines contrary to the same}.}[SDA Church Manual, 20th edition, Revised 2022, p. 67][https://www.adventist.org/wp-content/uploads/2023/07/2022-Seventh-day-Adventist-Church-Manual.pdf]


\others{1. \textbf{Negarea credinței} în principiile fundamentale ale Evangheliei și \textbf{în Punctele Fundamentale de Credință ale Bisericii} sau \textbf{învățarea unor doctrine contrare acestora}.}[Manualul Bisericii AZȘ, ediția a 20-a, Revizuită 2022, p. 67][https://www.adventist.org/wp-content/uploads/2023/07/2022-Seventh-day-Adventist-Church-Manual.pdf]


To discipline someone over their faith is nothing else than coercion of conscience. We are to render our conscience to the Bible alone—not to any man, councils or church creed(s). Disciplining members for their denial of the Fundamental Beliefs is clear evidence that we, indeed, have a creed besides the Bible. We cannot exercise the freedom of our conscience in subjection to the Word of God while confined to a set of beliefs that, if questioned with the authority of the Bible, will be disciplined. In our practice we have forgotten the foundation of protestantism and reformation. All reformers have had their conscience coerced to the extent of their lives. Martin Luther had famously put this principle in action in his defense before the Diet of Worms.


A disciplina pe cineva pentru credința sa nu este altceva decât constrângerea conștiinței. Noi trebuie să ne supunem conștiința doar Bibliei—nu vreunui om, consiliu sau crez(uri) bisericesc(ști). Disciplinarea membrilor pentru negarea Punctelor Fundamentale de Credință este o dovadă clară că noi, într-adevăr, avem un crez în afară de Biblie. Nu putem exercita libertatea conștiinței noastre în supunere față de Cuvântul lui Dumnezeu în timp ce suntem constrânși la un set de credințe care, dacă sunt puse la îndoială cu autoritatea Bibliei, vor fi disciplinate. În practica noastră am uitat fundamentul protestantismului și al reformei. Toți reformatorii au avut conștiința constrânsă până la a-și risca viața. Martin Luther a pus în mod faimos acest principiu în acțiune în apărarea sa înaintea Dietei de la Worms.


\others{Unless I am \textbf{convicted by Scripture} and plain reason—I do not accept the authority of popes and councils, for they have contradicted each other—\textbf{\underline{my conscience is captive to the Word of God}}. I cannot and I will not recant anything, for \textbf{to go against conscience is neither right nor safe}. Here I stand, I cannot do otherwise. God help me. Amen.}[Bainton, 182]


\others{Dacă nu sunt \textbf{convins prin Scriptură} și rațiune simplă—nu accept autoritatea papilor și a conciliilor, pentru că s-au contrazis unii pe alții—\textbf{\underline{conștiința mea este captivă Cuvântului lui Dumnezeu}}. Nu pot și nu voi retracta nimic, pentru că \textbf{a merge împotriva conștiinței nu este nici drept, nici sigur}. Aici stau, nu pot face altfel. Dumnezeu să mă ajute. Amin.}[Bainton, 182]


If one Seventh-day Adventist member has his conscience captive to the Word of God and is not in harmony with the Seventh-day Adventist Fundamental Beliefs, his conscience should not be coerced by church discipline. We know that in the end of time, the whole Seventh-day Adventist Church will be coerced over the issue of the Sabbath. We have been fighting for religious freedom, yet we’re allowing ourselves to coerce the conscience of those who are not in harmony with the Fundamental Beliefs. If today we discipline our members for not subjecting their consciences to men, councils and creeds, how shall we act tomorrow when the government will discipline their citizens for not subjecting their conscience to its power, when they will force obedience to legislation contrary to the Scriptures?


Dacă un membru adventist de ziua a șaptea are conștiința captivă Cuvântului lui Dumnezeu și nu este în armonie cu Punctele Fundamentale de Credință ale Adventiștilor de Ziua a Șaptea, conștiința lui nu ar trebui constrânsă prin disciplină bisericească. Știm că la sfârșitul timpului, întreaga Biserică Adventistă de Ziua a Șaptea va fi constrânsă în privința problemei Sabatului. Am luptat pentru libertatea religioasă, totuși ne permitem să constrângem conștiința celor care nu sunt în armonie cu Punctele Fundamentale de Credință. Dacă astăzi ne disciplinăm membrii pentru că nu își supun conștiința oamenilor, consiliilor și crezurilor, cum vom acționa mâine când guvernul își va disciplina cetățenii pentru că nu își supun conștiința puterii sale, când vor forța ascultarea față de o legislație contrară Scripturilor?


Adventist pioneers were very much aware of the dangers of extorting church members’ consciences. The expression of their beliefs was not designed to form unity. They were ready to justify their faith, from the Bible, when asked. The Bible was their only creed and article of faith.


Pionierii adventiști erau foarte conștienți de pericolele constrângerii conștiinței membrilor bisericii. Exprimarea credințelor lor nu a fost concepută pentru a forma unitate. Erau gata să își justifice credința, din Biblie, când erau întrebați. Biblia era singurul lor crez și articol de credință.


In 1883, there was a suggestion to introduce the church manual into the Seventh-day Adventist Church. This proposal was rejected after close investigation of the committee appointed by the General Conference. In the article “\textit{No Church Manual}”, we read their reasons for not accepting the proposed church manual.


În 1883, a existat o sugestie de a introduce manualul bisericii în Biserica Adventistă de Ziua a Șaptea. Această propunere a fost respinsă după o investigație atentă a comitetului numit de Conferința Generală. În articolul “\textit{Fără Manual al Bisericii}”, citim motivele lor pentru care nu au acceptat manualul bisericii propus.


\others{\textbf{While brethren who have favored a manual have ever contended that such a work was not to be anything like a creed or a discipline, or to have authority to settle disputed points}, but was only to be considered as a book containing hints for the help of those of little experience, \textbf{yet it must be evident that such a work, issued under the auspices of the General Conference, would at once carry with it much weight of authority, and would be consulted by most of our younger ministers}. \textbf{\underline{It would gradually shape and mold the whole body}}; \textbf{and those who did not follow it would be considered out of harmony with established principles of church order}. \textbf{And, really, is this not the object of the manual?} And what would be the use of one if not to accomplish such a result? But would this result, on the whole, be a benefit? Would our ministers be broader, more original, more self-reliant men? Could they be better depended on in great emergencies? Would their spiritual experiences likely be deeper and their judgment more reliable? \textbf{We think the tendency all the other way}.}[No Church Manual, The Review and Herald, November 27, 1883, pg. 745][https://documents.adventistarchives.org/Periodicals/RH/RH18831127-V60-47.pdf]


\others{\textbf{În timp ce frații care au favorizat un manual au susținut întotdeauna că o astfel de lucrare nu trebuia să fie ceva asemănător unui crez sau unei discipline, sau să aibă autoritate pentru a rezolva puncte disputate}, ci trebuia doar să fie considerată ca o carte conținând sugestii pentru ajutorarea celor cu puțină experiență, \textbf{totuși trebuie să fie evident că o astfel de lucrare, emisă sub auspiciile Conferinței Generale, ar purta imediat cu ea multă greutate de autoritate și ar fi consultată de majoritatea pastorilor noștri mai tineri}. \textbf{\underline{Ea ar modela și forma treptat întregul corp}}; \textbf{iar cei care nu ar urma-o ar fi considerați în dezarmonie cu principiile stabilite ale ordinii bisericești}. \textbf{Și, într-adevăr, nu este acesta obiectul manualului?} Și care ar fi utilitatea unuia dacă nu pentru a realiza un astfel de rezultat? Dar ar fi acest rezultat, în ansamblu, un beneficiu? Ar fi pastorii noștri oameni mai largi la minte, mai originali, mai încrezători în sine? S-ar putea conta mai bine pe ei în mari urgențe? Ar fi experiențele lor spirituale probabil mai profunde și judecata lor mai de încredere? \textbf{Credem că tendința este în cealaltă direcție}.}[Fără Manual al Bisericii, The Review and Herald, 27 noiembrie 1883, pg. 745][https://documents.adventistarchives.org/Periodicals/RH/RH18831127-V60-47.pdf]


\others{\textbf{The Bible contains our creed and discipline. It \underline{thoroughly} furnishes the man of God unto all good works}. What it has not revealed relative to church organization and management, the duties of officers and ministers, and kindred subjects, should not be strictly defined and drawn out into minute specifications for the sake of uniformity, \textbf{but rather be left to individual judgment under the guidance of the Holy Spirit}. \textbf{Had it been best to have a book of directions of this sort, the Spirit would doubtless have gone further, and left one on record with the stamp of inspiration upon it}.}[Ibid.][https://documents.adventistarchives.org/Periodicals/RH/RH18831127-V60-47.pdf]


\others{\textbf{Biblia conține crezul și disciplina noastră. Ea îl înzestrează \underline{pe deplin} pe omul lui Dumnezeu pentru orice lucrare bună}. Ceea ce nu a revelat ea cu privire la organizarea și administrarea bisericii, îndatoririle funcționarilor și pastorilor, și subiecte înrudite, nu ar trebui să fie strict definit și elaborat în specificații minuțioase de dragul uniformității, \textbf{ci mai degrabă să fie lăsat judecății individuale sub călăuzirea Duhului Sfânt}. \textbf{Dacă ar fi fost mai bine să avem o carte de instrucțiuni de acest fel, Duhul ar fi mers fără îndoială mai departe și ar fi lăsat una înregistrată cu pecetea inspirației asupra ei}.}[Ibid.][https://documents.adventistarchives.org/Periodicals/RH/RH18831127-V60-47.pdf]


Since 1883, the Seventh-day Adventist Church had grown considerably; so, for the sake of convenience, in 1931, the General Conference Committee voted to publish a church manual.\footnote{Maratas, Prince. “Church Manual.” General Conference of Seventh-Day Adventists, 20 Aug. 2023, \href{https://gc.adventist.org/church-manual/}{gc.adventist.org/church-manual/}. Accessed 3 Feb. 2025.} The church, as an organized body, should exercise order and discipline, in the matters of organization and plans of the prosperity of the Church's mission. But no committee should exercise authority over someone’s conscience and someone’s belief. Only God holds the right to this authority. This is why the Bible is our only creed. We render our conscience to the Word of God, not a man, nor a group of men or committee. Contrary to this, many believe that God vested this authority to the general assembly of the General Conference. But such an idea is based on misrepresentation of one particular quotation. Let us read this quotation carefully.


Din 1883, Biserica Adventistă de Ziua a Șaptea crescuse considerabil; astfel, de dragul comodității, în 1931, Comitetul Conferinței Generale a votat să publice un manual al bisericii.\footnote{Maratas, Prince. “Church Manual.” General Conference of Seventh-Day Adventists, 20 Aug. 2023, \href{https://gc.adventist.org/church-manual/}{gc.adventist.org/church-manual/}. Accesat 3 feb. 2025.} Biserica, ca organism organizat, ar trebui să exercite ordine și disciplină, în chestiunile de organizare și planurile pentru prosperitatea misiunii Bisericii. Dar niciun comitet nu ar trebui să exercite autoritate asupra conștiinței cuiva și asupra credinței cuiva. Doar Dumnezeu deține dreptul la această autoritate. De aceea Biblia este singurul nostru crez. Ne supunem conștiința Cuvântului lui Dumnezeu, nu unui om, nici unui grup de oameni sau comitet. Contrar acestui lucru, mulți cred că Dumnezeu a învestit această autoritate adunării generale a Conferinței Generale. Dar o astfel de idee se bazează pe reprezentarea greșită a unui citat particular. Să citim acest citat cu atenție.


\egw{At times, when a small group of men entrusted with \textbf{the general management of the work} have, in the name of the General Conference, sought to carry out unwise plans and to restrict God’s work, I have said that I could no longer regard the voice of the General Conference, represented by these few men, as the voice of God. \textbf{But this is not saying that the decisions of a General Conference composed of an assembly of duly appointed, representative men from all parts of the field should not be respected}. \textbf{God has ordained that the representatives of His church from all parts of the earth, when assembled in a General Conference, \underline{shall have authority}}. The error that some are in danger of committing is in giving to the mind and judgment of one man, or of a small group of men, \textbf{the full measure of authority and influence that God has vested in His church in the judgment and voice of the General Conference assembled \underline{to plan for the prosperity and advancement of His work}}.}[9T 260.2; 1909][https://egwwritings.org/read?panels=p115.1474]


\egw{Uneori, când un grup mic de bărbați încredințați cu \textbf{conducerea generală a lucrării} au căutat, în numele Conferinței Generale, să ducă la îndeplinire planuri neînțelepte și să restricționeze lucrarea lui Dumnezeu, am spus că nu mai pot privi vocea Conferinței Generale, reprezentată de acești câțiva oameni, ca vocea lui Dumnezeu. \textbf{Dar aceasta nu înseamnă că deciziile unei Conferințe Generale compuse dintr-o adunare de bărbați numiți în mod corespunzător, reprezentanți din toate părțile câmpului, nu ar trebui respectate}. \textbf{Dumnezeu a rânduit ca reprezentanții bisericii Sale din toate părțile pământului, când sunt adunați într-o Conferință Generală, \underline{să aibă autoritate}}. Greșeala pe care unii sunt în pericol să o comită este de a da minții și judecății unui singur om, sau unui grup mic de oameni, \textbf{măsura deplină de autoritate și influență pe care Dumnezeu a investit-o în biserica Sa în judecata și vocea Conferinței Generale adunate \underline{pentru a planifica prosperitatea și înaintarea lucrării Sale}}.}[9T 260.2; 1909][https://egwwritings.org/read?panels=p115.1474]


Sister White pointed out that the world wide assembly of the General Conference meeting does have authority as the voice of God, yet she is very particular over what matters it has this authority. The authority God vested in the assembly of the General Conference is \egwinline{to plan for the prosperity and advancement of His work}. It is about making mission plans, not about managing beliefs or the conscience. God’s church does have His voice regarding beliefs; the voice of God pertaining to the faith is the Bible. The Bible is fully sufficient for us and we are free to render our conscience to it. No synopsis of any denominational faith has authority to dictate someone's faith; neither do \emcap{Fundamental Principles}, or current Fundamental Beliefs.\footnote{Although the Fundamental Principles were not designed to have authority over the people, nor were they designed to secure uniformity among them, as a system of faith, there is some evidence to the contrary. In his article, “\textit{Seventh-day Adventists and the Doctrine of the Trinity}”, of the “\textit{Christian Workers Magazine}”, 1915, D.M. Canright gave evidence that a Conference president used the \emcap{Fundamental Principles} as a test of fellowship in 1911. Such practice is not constructive to the Truth, neither is it beneficial for believers.} Sister White was very clear about the Bible being the only rule of faith, and every doctrine should be questioned with Scripture. In the Great Controversy, we read the following:


Sora White a subliniat că adunarea mondială a sesiunii Conferinței Generale are într-adevăr autoritate ca voce a lui Dumnezeu, totuși ea este foarte precisă asupra chestiunilor în care are această autoritate. Autoritatea pe care Dumnezeu a investit-o în adunarea Conferinței Generale este \egwinline{pentru a planifica prosperitatea și înaintarea lucrării Sale}. Este vorba despre realizarea planurilor de misiune, nu despre gestionarea credințelor sau a conștiinței. Biserica lui Dumnezeu are într-adevăr vocea Sa cu privire la credințe; vocea lui Dumnezeu referitoare la credință este Biblia. Biblia este pe deplin suficientă pentru noi și suntem liberi să ne supunem conștiința ei. Niciun rezumat al vreunei credințe denominaționale nu are autoritate să dicteze credința cuiva; nici \emcap{Principiile Fundamentale}, sau actualele Puncte Fundamentale de Credință.\footnote{Deși Principiile Fundamentale nu au fost concepute să aibă autoritate asupra oamenilor, nici nu au fost concepute să asigure uniformitate între ei, ca sistem de credință, există unele dovezi contrare. În articolul său, “\textit{Adventiștii de ziua a șaptea și doctrina Trinității}”, din “\textit{Christian Workers Magazine}”, 1915, D.M. Canright a dat dovezi că un președinte de Conferință a folosit \emcap{Principiile Fundamentale} ca test de părtășie în 1911. O astfel de practică nu este constructivă pentru Adevăr, nici nu este benefică pentru credincioși.} Sora White a fost foarte clară despre faptul că Biblia este singura regulă de credință, și fiecare doctrină ar trebui pusă la îndoială cu Scriptura. În Tragedia Veacurilor, citim următoarele:


\egw{But God will have a people upon the earth \textbf{to maintain the Bible, and \underline{the Bible only}}, \textbf{as the standard of all doctrines and the basis of all reforms}. \textbf{The opinions of learned men, the deductions of science, \underline{the creeds or decisions of ecclesiastical councils}, as numerous and discordant as are the churches which they represent, the voice of the majority - not one nor all of these should be regarded as evidence for or against any point of religious faith.} \textbf{Before accepting any doctrine or precept, we should demand a plain ‘Thus saith the Lord’ in its support.}}[GC 595.1; 1888][https://egwwritings.org/read?panels=p132.2689]


\egw{Dar Dumnezeu va avea un popor pe pământ \textbf{care să mențină Biblia, și \underline{numai Biblia}}, \textbf{ca standard al tuturor doctrinelor și baza tuturor reformelor}. \textbf{Opiniile oamenilor învățați, deducțiile științei, \underline{crezurile sau deciziile conciliilor ecleziastice}, atât de numeroase și discordante precum sunt bisericile pe care le reprezintă, vocea majorității - nici una și nici toate acestea nu ar trebui considerate ca dovadă pentru sau împotriva vreunui punct de credință religioasă.} \textbf{Înainte de a accepta orice doctrină sau precept, ar trebui să cerem un simplu ‘Așa zice Domnul’ în sprijinul său.}}[GC 595.1; 1888][https://egwwritings.org/read?panels=p132.2689]


The liberty of conscience is the basics of protestantism and reformation. We hope and believe that every Seventh-day Adventist can exercise freedom to render his conscience to the Bible without being coerced by discipline, or any other means. The issue of the church's creed and discipline becomes more relevant today, when we have the promise that God will re-establish the original foundation of our faith. We hope and pray that the evidence brought up here will bring light to the church leadership and encourage them to eradicate the false practices in our midst. As the religious leaders in Christ’s time were entrusted with the duty to preserve the Truth and to recognize the time of God’s visitation, so it is today with the leaders of the Seventh-day Adventist Church. In what follows, we will present the prophecies God specifically gave to the Seventh-day Adventist Church. In our time, the end-time, all the pillars of our faith that were held in the beginning will be re-established. May every member of the Seventh-day Adventist Church recognize the importance of the revival that God is about to establish.


Libertatea conștiinței este baza protestantismului și reformei. Sperăm și credem că fiecare adventist de ziua a șaptea poate exercita libertatea de a-și supune conștiința Bibliei fără a fi constrâns prin disciplină sau orice alte mijloace. Problema crezului și disciplinei bisericii devine mai relevantă astăzi, când avem promisiunea că Dumnezeu va restabili temelia originală a credinței noastre. Sperăm și ne rugăm ca dovezile aduse aici să aducă lumină conducerii bisericii și să îi încurajeze să eradicheze practicile false din mijlocul nostru. Așa cum conducătorii religioși din timpul lui Hristos au fost încredințați cu datoria de a păstra Adevărul și de a recunoaște timpul vizitării lui Dumnezeu, la fel este astăzi cu liderii Bisericii Adventiste de Ziua a Șaptea. În cele ce urmează, vom prezenta profețiile pe care Dumnezeu le-a dat în mod specific Bisericii Adventiste de Ziua a Șaptea. În timpul nostru, timpul sfârșitului, toți stâlpii credinței noastre care au fost ținuți la început vor fi restabiliți. Fie ca fiecare membru al Bisericii Adventiste de Ziua a Șaptea să recunoască importanța reînvierii pe care Dumnezeu este pe cale să o stabilească.


% Steps to Apostasy

\begin{titledpoem}
    \stanza{
        A creed established beyond God's Word, \\
        The voice of conscience no longer heard. \\
        Fellowship tested by human decree, \\
        From Bible authority we slowly flee.
    }

    \stanza{
        Those who dissent labeled heretics, lost, \\
        Their faith and conviction at terrible cost. \\
        Persecution follows for standing apart, \\
        When creeds replace Scripture within the heart. \\
    }

    \stanza{
        The Bible alone should guide our belief, \\
        All other authorities bringing grief. \\
        Our conscience surrenders to God's Word divine, \\
        Not to councils of men who draw the line.
    }

    \stanza{
        The pioneers knew this freedom well, \\
        Against human creeds they chose to rebel. \\
        For truth must flourish where conscience is free, \\
        As God intended His church to be.
    }
\end{titledpoem}