% \qrchapter{https://forgottenpillar.com/rsc/en-fp-chapter12}{Heaven's reality}


\qrchapter{https://forgottenpillar.com/rsc/ro-fp-chapter12}{Realitatea cerului}


The \emcap{personality of God} deals with the quality or state of God being a person. Whenever we look at the pioneer's work on the \emcap{personality of God}, we see that they were all in harmony with the view that God is a tangible \textit{being}, possessing both body and parts. We always see the same underlying reasoning, which differentiates the term ‘\textit{spirit}’ and the term ‘\textit{being}’. By differentiating these terms, they explain the quality or state of God being a person\footnote{\href{https://www.merriam-webster.com/dictionary/personality}{Merriam-Webster Dictionary} defines the word ‘\textit{personality}’ as “\textit{quality or state of being a person}”.}—a \emcap{personality of God}. All their conclusions are summed up in the first point of the \emcap{Fundamental Principles}. \others{There is \textbf{one God}, a \textbf{personal}, \textbf{spiritual being}, the Creator of all things, omnipotent, omniscient, … and \textbf{every-where present by his representative, the Holy Spirit}. Psalm 139:7.}[FPSDA 1.2][https://egwwritings.org/read?panels=p1299.6]


\emcap{Personalitatea lui Dumnezeu} se referă la caracteristica sau starea prin care Dumnezeu este definit ca persoană. Ori de câte ori privim la lucrarea pionierilor despre \emcap{personalitatea lui Dumnezeu}, vedem că toți erau în armonie cu perspectiva că Dumnezeu este o \textit{ființă} tangibilă, posedând atât trup, cât și părți. Vedem întotdeauna același raționament de bază, care diferențiază termenul ‘\textit{spirit}’ de termenul ‘\textit{ființă}’. Prin diferențierea acestor termeni, ei explică caracteristica sau starea prin care Dumnezeu este definit ca persoană\footnote{\href{https://www.merriam-webster.com/dictionary/personality}{Dicționarul Merriam-Webster} definește cuvântul ‘\textit{personalitate}’ ca “\textit{calitatea sau starea de a fi o persoană}”.}—o \emcap{personalitate a lui Dumnezeu}. Toate concluziile lor sunt rezumate în primul punct al \emcap{Principiilor Fundamentale}. \others{Există \textbf{un singur Dumnezeu}, o \textbf{ființă spirituală}, \textbf{personală}, Creatorul tuturor lucrurilor, atotputernic, atotștiutor, … și \textbf{prezent pretutindeni prin reprezentantul Său, Duhul Sfânt}. Psalmul 139:7.}[FPSDA 1.2][https://egwwritings.org/read?panels=p1299.6]


So far, in the pioneers’ work, we have seen that the \emcap{personality of God} is tightly connected to the reality of God’s presence. God is a personal spiritual being, having a body and shape; as such, His presence is cumbered to one locality—as the Bible says, in His temple, at His throne where He is surrounded with unapproachable glory. But He is everywhere present by His representative, the Holy Spirit. Obviously, the Holy Spirit is a spirit, and not a being, \bible{for a spirit hath not flesh and bones as ye see me have}, said Jesus (Luke 24:39). Christ is also a Being, like His Father. He is an express image of the Father’s person; therefore, He bears the same personality, or the quality or state of being a person, as His Father does.


Până acum, în lucrarea pionierilor, am văzut că \emcap{personalitatea lui Dumnezeu} este strâns legată de realitatea prezenței lui Dumnezeu. Dumnezeu este o ființă spirituală personală, având un trup și o formă; ca atare, prezența Sa este limitată la o singură localitate—așa cum spune Biblia, în templul Său, la tronul Său unde este înconjurat de o glorie de neapropriat. Dar El este prezent pretutindeni prin reprezentantul Său, Duhul Sfânt. Evident, Duhul Sfânt este un spirit, și nu o ființă, \bible{căci un duh n-are nici carne, nici oase, cum vedeți că am Eu}, a spus Isus (Luca 24:39). Hristos este de asemenea o Ființă, ca Tatăl Său. El este întipărirea persoanei Tatălui; prin urmare, El poartă aceeași personalitate, sau calitatea sau starea de a fi o persoană, ca și Tatăl Său.


In our experience, when we present the original Seventh-day Adventist beliefs on the \emcap{personality of God} to our trinitarian brothers, as expressed in the first two points of the \emcap{Fundamental Principles}, they often claim that the statements in the \emcap{Fundamental Principles} are correct in some way, but the understanding attributed to the terms “\textit{personal spiritual being}” are false. They usually attempt to harmonize the \emcap{Fundamental Principles} with the Trinity doctrine by twisting the words “\textit{spiritual being}”, as if the word ‘\textit{spiritual}’ means something mysterious, suitable to equalize the \emcap{personality of God} and of Christ with the personality of the Holy Ghost\footnote{The quality or state of the Holy Spirit being a person is bearing witness, not having the form of a person. \egw{\textbf{The Holy Spirit has a personality}, \textbf{\underline{else} He could not \underline{bear witness} to our spirits} and with our spirits that we are the children of God. \textbf{He must also be a divine person}, \textbf{\underline{else} He could not \underline{search out} the secrets which lie hidden in the mind of God}. ‘For what man knoweth the things of a man save the spirit of man, which is in him; even so the things of God knoweth no man, but the Spirit of God.’ [1 Corinthians 2:11.]}[21LtMs, Ms 20, 1906, par. 32][https://egwwritings.org/read?panels=p14071.10296041&index=0]. It is crystal clear that the Holy Spirit is a person, yet not in the same way as the Father and the Son, as the Holy Spirit does not possess the quality of an outward physical personage like the Father and the Son do.}. The underlying problem comes down to the understanding of the heavenly realities. The Bible is not silent about heaven, and its realities, and our pioneers understood it well. Below we read about the explanation of the terms “\textit{spiritual being}” from James White and Uriah Smith in their book, “\textit{The Biblical Institute}”. The Bible explains these terms using the example of angels, which are “\textit{spiritual beings}”.


În experiența noastră, când prezentăm credințele adventiste de ziua a șaptea originale despre \emcap{personalitatea lui Dumnezeu} fraților noștri trinitarieni, așa cum sunt exprimate în primele două puncte ale \emcap{Principiilor Fundamentale}, ei susțin adesea că afirmațiile din \emcap{Principiile Fundamentale} sunt corecte într-un fel, dar înțelegerea atribuită termenilor “\textit{ființă spirituală personală}” este falsă. De obicei încearcă să armonizeze \emcap{Principiile Fundamentale} cu doctrina Trinității răsucind cuvintele “\textit{ființă spirituală}”, ca și cum cuvântul ‘\textit{spiritual}’ înseamnă ceva misterios, potrivit pentru a egaliza \emcap{personalitatea lui Dumnezeu} și a lui Hristos cu personalitatea Duhului Sfânt\footnote{Calitatea sau starea Duhului Sfânt de a fi o persoană constă în a da mărturie, nu în a avea forma unei persoane. \egw{\textbf{Duhul Sfânt are o personalitate}, \textbf{\underline{altfel} nu ar putea \underline{da mărturie} duhurilor noastre} și împreună cu duhurile noastre că suntem copii ai lui Dumnezeu. \textbf{El trebuie să fie de asemenea o persoană divină}, \textbf{\underline{altfel} nu ar putea \underline{cerceta} tainele care zac ascunse în mintea lui Dumnezeu}. ‘Căci cine dintre oameni cunoaște lucrurile omului, afară de duhul omului, care este în el? Tot așa: nimeni nu cunoaște lucrurile lui Dumnezeu, afară de Duhul lui Dumnezeu.’ [1 Corinteni 2:11.]}[21LtMs, Ms 20, 1906, par. 32][https://egwwritings.org/read?panels=p14071.10296041&index=0]. Este clar ca lumina zilei că Duhul Sfânt este o persoană, totuși nu în același fel ca Tatăl și Fiul, deoarece Duhul Sfânt nu posedă calitatea unei înfățișări fizice exterioare ca Tatăl și Fiul.}. Problema de bază se reduce la înțelegerea realităților cerești. Biblia nu tace despre cer și realitățile sale, iar pionierii noștri au înțeles-o bine. Mai jos citim despre explicația termenilor “\textit{ființă spirituală}” de la James White și Uriah Smith în cartea lor, “\textit{Institutul Biblic}”. Biblia explică acești termeni folosind exemplul îngerilor, care sunt “\textit{ființe spirituale}”.


\begin{figure}[hp]
    \centering
    \includegraphics[width=1\linewidth]{images/uriah-smith.jpg}
    \caption*{Uriah Smith (1832-1903)}
    \label{fig:uriah-smith}
\end{figure}


\begin{figure}[hp]
    \centering
    \includegraphics[width=1\linewidth]{images/uriah-smith.jpg}
    \caption*{Uriah Smith (1832-1903)}
    \label{fig:uriah-smith}
\end{figure}


\others{\textbf{Angels are real beings}. They are described in the Bible as \textbf{possessing face, feet, wings} \&x. Ezekiel says of the cherubim, ‘\textbf{Their whole \underline{body} and their backs and their hands and their wings},’ \&c. Eze. 10:12. Angels \textbf{appeared }unto Abraham. Gen. 18:1-8. They talked and ate with him. They went on to Sodom and communed with Lot, who, entering into his house baked unleavened bread for them and they did eat. \textbf{These person were called angels}. David speaks of the manna as the corn of Heaven and angel’s food. Ps. 78:23-25.}


\others{\textbf{Îngerii sunt ființe reale}. Ei sunt descriși în Biblie ca \textbf{posedând față, picioare, aripi} etc. Ezechiel spune despre heruvimi, ‘\textbf{Tot \underline{trupul} lor, spatele lor, mâinile lor și aripile lor},’ etc. Ezec. 10:12. Îngerii \textbf{i-au apărut} lui Avraam. Gen. 18:1-8. Au vorbit și au mâncat cu el. Au mers mai departe la Sodoma și au stat de vorbă cu Lot, care, intrând în casa lui, a copt pâine nedospită pentru ei și au mâncat. \textbf{Aceste persoane au fost numite îngeri}. David vorbește despre mană ca despre grâul Cerului și hrana îngerilor. Ps. 78:23-25.}


\othersnogap{The case of Balaam, Num. 22:22-31, is an interesting incident. The angel \textbf{appeared }to Balaam with a sword \textbf{drawn in his hand}. The question is sometimes asked \textbf{how angels can be \underline{material beings since we cannot see them}. This case illustrates it}. The record says the \textbf{Lord opened the eyes of Balaam and he saw the angel}. \textbf{The angel did not create a body for that occasion}.\textbf{ He was just the same as he was before Balaam saw him; \underline{but the change took place in Balaam}. His eyes were opened, then he beheld the angel}. It was the same with the servant of Elisha when he and his master were brought into a straight place, surrounded by the army of the king of Syria. 2 Kings 6:17. Elisha prayed that \textbf{the eyes of his servant might be opened}; and he immediately saw the whole mountain full of horses and chariots round about Elisha.}


\othersnogap{Cazul lui Balaam, Num. 22:22-31, este un incident interesant. Îngerul \textbf{i-a apărut} lui Balaam cu o sabie \textbf{scoasă în mână}. Întrebarea care se pune uneori este \textbf{cum pot îngerii să fie \underline{ființe materiale de vreme ce nu îi putem vedea}. Acest caz o ilustrează}. Relatarea spune că \textbf{Domnul a deschis ochii lui Balaam și el a văzut îngerul}. \textbf{Îngerul nu și-a creat un trup pentru acea ocazie}.\textbf{ El era exact la fel ca înainte ca Balaam să-l vadă; \underline{dar schimbarea a avut loc în Balaam}. Ochii lui au fost deschiși, apoi a văzut îngerul}. La fel s-a întâmplat cu slujitorul lui Elisei când el și stăpânul său au fost aduși într-o strâmtoare, înconjurați de armata regelui Siriei. 2 Împărați 6:17. Elisei s-a rugat ca \textbf{ochii slujitorului său să fie deschiși}; și el a văzut imediat tot muntele plin de cai și care de foc în jurul lui Elisei.}


\othersnogap{\textbf{This may be further illustrated referring to things which we know are material and yet which we cannot see}. Air is material, light is material, even thought itself is only the result of material organizations — matter acting upon matter — and yet we can see none of these things. \textbf{Just so with the angels}.}


\othersnogap{\textbf{Aceasta poate fi ilustrată în continuare referindu-ne la lucruri pe care știm că sunt materiale și totuși pe care nu le putem vedea}. Aerul este material, lumina este materială, chiar și gândul însuși este doar rezultatul organizărilor materiale — materia acționând asupra materiei — și totuși nu putem vedea niciunul dintre aceste lucruri. \textbf{La fel este și cu îngerii}.}


\othersnogap{\textbf{It is further objected to the materiality of the angels that they are called spirits. }Heb. 1:13, 14.\textbf{\underline{But this is no objection to their being literal beings}}. \textbf{They are simply spiritual beings organized differently from these earthly bodies which we possess}. Paul says, 1 Cor. 15:44, ‘\textbf{There is a natural body and there is \underline{a spiritual body}}.’ \textbf{The natural body we now have; the spiritual body we shall have in the resurrection}. ‘\textbf{It is raised a spiritual body}.’ Verse 44. \textbf{But then we are equal unto the angels}, Luke 20:36; \textbf{then we have bodies like unto Christ’s most glorious body}. Phil. 3:4\footnote{Typo: It should be Philippians 3:21} \textbf{and Christ is no less a spirit than the angels}. \textbf{We read that God is a spirit, that is, simply \underline{a spiritual being}}.}[James White and Uriah Smith, The Biblical Institute (Kindle Locations 2537-2553). Kindle Edition.]


\othersnogap{\textbf{Se obiectează în continuare la materialitatea îngerilor că ei sunt numiți duhuri. }Evr. 1:13, 14.\textbf{\underline{Dar aceasta nu este o obiecție la faptul că sunt ființe literale}}. \textbf{Ei sunt pur și simplu ființe spirituale organizate diferit de aceste trupuri pământești pe care le posedăm}. Pavel spune, 1 Cor. 15:44, ‘\textbf{Este un trup natural și este \underline{un trup spiritual}}.’ \textbf{Trupul natural îl avem acum; trupul spiritual îl vom avea la înviere}. ‘\textbf{Înviază trup spiritual}.’ Versetul 44. \textbf{Dar atunci suntem egali cu îngerii}, Luca 20:36; \textbf{atunci avem trupuri asemenea trupului preaslăvit al lui Hristos}. Fil. 3:4\footnote{Greșeală de tipar: Ar trebui să fie Filipeni 3:21} \textbf{și Hristos nu este mai puțin duh decât îngerii}. \textbf{Citim că Dumnezeu este duh, adică, pur și simplu \underline{o ființă spirituală}}.}[James White și Uriah Smith, Institutul Biblic (Kindle Locations 2537-2553). Kindle Edition.]


The Bible gives us the insight that angels are spiritual beings that possess material bodies, but are still unseen to us, unless the Lord opens our eyes to see them. When the righteous will rise up in their new glorified bodies, they will rise in a spiritual body, an incorruptible one. This body will be tangible and material just as the new Earth will be tangible and material. And with our spiritual bodies we will possess the renewed Earth, we will replenish it \bible{and subdue it: and have dominion over the fish of the sea, and over the fowl of the air, and over every living thing that moveth upon the earth}[Genesis 1:28].


Biblia ne oferă perspectiva că îngerii sunt ființe spirituale care posedă trupuri materiale, dar sunt totuși nevăzuți pentru noi, dacă Domnul nu ne deschide ochii să-i vedem. Când cei drepți vor învia în noile lor trupuri glorificate, vor învia într-un trup spiritual, unul nestricăcios. Acest trup va fi tangibil și material la fel cum noul Pământ va fi tangibil și material. Și cu trupurile noastre spirituale vom stăpâni Pământul înnoit, îl vom umple \bible{și-l vom supune; și vom stăpâni peste peștii mării, peste păsările cerului și peste orice viețuitoare care se mișcă pe pământ}[Geneza 1:28].


% Heaven's reality

\begin{titledpoem}
    
    \stanza{
        God is not a vapor unknown, \\
        Not as a mystery on His throne. \\
        In just one place as beings are, \\
        Yet by His Spirit spread afar.
    }

    \stanza{
        Christ bears God’s image as His Son, \\
        Two beings divine, not joined as one. \\
        Angels have bodies, yet unseen, \\
        Physical forms with heav’nly sheen.
    }

    \stanza{
        We cannot see their spirit frame, \\
        Until immortal life we claim. \\
        We’ll resurrect like them to be, \\
        And dwell with them eternally.
    }

    \stanza{
        God is not three in mystic blend, \\
        But Father, Son, Their Spirit send. \\
        The Father, Son are not obscure, \\
        They’re personal, we know for sure.
    }
    
\end{titledpoem}