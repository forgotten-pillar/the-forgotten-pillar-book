% \qrchapter{https://forgottenpillar.com/rsc/en-fp-chapter8}{The constructive criticism}


\qrchapter{https://forgottenpillar.com/rsc/ro-fp-chapter8}{Critica constructivă}


The first point of the \emcap{Fundamental Principles} answers the questions: who is God, what is His personality, and how do we understand His presence?


Primul punct al \emcap{Principiilor Fundamentale} răspunde la întrebările: cine este Dumnezeu, care este personalitatea Lui și cum înțelegem prezența Lui?


\others{I. That there is \textbf{one God}, \textbf{a personal, spiritual }\textbf{\underline{being}}, \textbf{the creator of all things}, omnipotent, omniscient, and eternal; infinite in wisdom, holiness, justice, goodness, truth, and mercy; unchangeable, and \textbf{everywhere present by his representative, the Holy Spirit}. Ps. 139:7.}[FP1889 147.2; 1889][https://egwwritings.org/read?panels=p931.6]


\others{I. Că există \textbf{un singur Dumnezeu}, \textbf{o }\textbf{\underline{ființă}} \textbf{personală, spirituală}, \textbf{creatorul tuturor lucrurilor}, atotputernic, atotștiutor și etern; infinit în înțelepciune, sfințenie, dreptate, bunătate, adevăr și milă; neschimbător și \textbf{prezent pretutindeni prin reprezentantul Său, Duhul Sfânt}. Ps. 139:7.}[FP1889 147.2; 1889][https://egwwritings.org/read?panels=p931.6]


The one God, the Creator, is identified as the Father, because the second point of the \emcap{Fundamental Principles} states that Jesus Christ, the Son of the Eternal Father, is the one by whom God created all things\footnote{\href{https://egwwritings.org/?ref=en_FP1889.147.3&para=931.7}{FP1889 147.3; 1889}}. The \emcap{personality of God} is expressed in the term “\textit{personal spiritual being}”. We will soon see that this term denotes that the Father has a material body, a physical manifestation. Thus, in His personality, He is present only where He dwells physically. But, His presence is not constrained to His personality because He is \others{everywhere present by his representative, the Holy Spirit}. During our past history, this understanding and reasoning of the \emcap{personality of God}, as expressed in the first point of the \emcap{Fundamental Principles}, received constructive criticism; by “constructive criticism” we refer to the criticism supported by the Bible.


Singurul Dumnezeu, Creatorul, este identificat ca fiind Tatăl, deoarece al doilea punct al \emcap{Principiilor Fundamentale} afirmă că Isus Hristos, Fiul Tatălui Etern, este cel prin care Dumnezeu a creat toate lucrurile\footnote{\href{https://egwwritings.org/?ref=en_FP1889.147.3&para=931.7}{FP1889 147.3; 1889}}. \emcap{Personalitatea lui Dumnezeu} este exprimată în termenul „\textit{ființă personală spirituală}”. Vom vedea în curând că acest termen denotă că Tatăl are un corp material, o manifestare fizică. Astfel, în personalitatea Sa, El este prezent doar acolo unde locuiește fizic. Dar prezența Sa nu este constrânsă la personalitatea Sa, deoarece El este \others{prezent pretutindeni prin reprezentantul Său, Duhul Sfânt}. De-a lungul istoriei noastre trecute, această înțelegere și raționament al \emcap{personalității lui Dumnezeu}, așa cum este exprimată în primul punct al \emcap{Principiilor Fundamentale}, a primit critici constructive; prin „critici constructive” ne referim la critica susținută de Biblie.


We now present to you the following citations, some constructive criticism, from a prominent trinitarian brother in the Seventh-day Adventist world. Interestingly, he had acknowledged the authority of the \emcap{Fundamental Principles}, yet simultaneously believed in the Trinity doctrine. We find this document a very important element in the change of our beliefs from the fundamental principles to current Seventh-day Adventist Trinitarian belief.


Vă prezentăm acum următoarele citate, niște critici constructive, de la un frate trinitarian proeminent din lumea adventistă de ziua a șaptea. Interesant este că el recunoscuse autoritatea \emcap{Principiilor Fundamentale}, dar în același timp credea în doctrina Trinității. Considerăm acest document un element foarte important în schimbarea credințelor noastre de la principiile fundamentale la credința trinitariană adventistă de ziua a șaptea actuală.


This prominent brother was met with the question, “\textit{Do you not believe in a personal, definite God?}”:


Acest frate proeminent a fost întâmpinat cu întrebarea: „\textit{Nu crezi într-un Dumnezeu personal, definit?}”:


\others{\textbf{Most certainly. An infinite, divine, personal being is essential religion}. Worship requires someone to love, to obey, to trust. \textbf{Belief in a personal God is the very core of the Christian religion}. The conception of God as the All-Energy, the infinite Power, an all-pervading Presence, is too vast for the human mind to grasp; there must be something more \textbf{tangible}, more \textbf{\underline{restricted}}, upon which to center the mind in worship. \textbf{It is for this reason that Christ came to us in the image of God's }\textbf{\underline{personality}}\textbf{, the second Adam, to show us by his life of love and self-sacrifice the character and }\textbf{\underline{the personality of God}}. We can approach God only through Christ.}


\others{\textbf{Cu siguranță. O ființă infinită, divină, personală este esențială religiei}. Închinarea necesită pe cineva de iubit, de ascultat, în care să ai încredere. \textbf{Credința într-un Dumnezeu personal este însăși miezul religiei creștine}. Concepția despre Dumnezeu ca Toată-Energia, Puterea infinită, o Prezență atotpătrunzătoare, este prea vastă pentru ca mintea umană să o cuprindă; trebuie să existe ceva mai \textbf{tangibil}, mai \textbf{\underline{restrâns}}, pe care să se centreze mintea în închinare. \textbf{Din acest motiv Hristos a venit la noi în chipul }\textbf{\underline{personalității}} \textbf{lui Dumnezeu, al doilea Adam, pentru a ne arăta prin viața Sa de dragoste și jertfire de sine caracterul și }\textbf{\underline{personalitatea lui Dumnezeu}}. Putem să ne apropiem de Dumnezeu doar prin Hristos.}


\othersnogap{‘Who being the brightness of his glory, and \textbf{the express image of his person}, and upholding all things by the word of his power, when he had by himself purged our sins, sat down on the right hand of the Majesty on high.’}


\othersnogap{„Care, fiind strălucirea slavei Lui și \textbf{întipărirea persoanei Lui} și care ține toate lucrurile cu cuvântul puterii Lui, a făcut curățirea păcatelor și a șezut la dreapta Măririi, în locurile preaînalte.”}


\othersnogap{‘Who being the effulgence of his glory, and the impress of his substance, and upholding all things by the word of his power.’}


\othersnogap{„Care, fiind strălucirea slavei Lui și întipărirea ființei Lui și care ține toate lucrurile cu cuvântul puterii Lui.”}


\othersnogap{The apostle says, ‘But we all, with open face \textbf{beholding as in a glass} the glory of the Lord, are changed into the same image from glory to glory, even as by the Spirit of the Lord.’ 2 Cor. 3: 18. How apt and beautiful is this figure!... So, \textbf{in beholding Christ} in his miracles, his temptations, his exhortations, his life of self-abnegation, his ‘going about doing good,’ \textbf{we may behold the personality and power of God}. And what a great hope there is for us in the fact that \textbf{in Christ we find qualities not strange and foreign to humanity}, but kindred mental and moral characteristics; so that we are able to see and grasp an actual, rather than merely a theological or abstract or figurative truth, in the declaration of the apostle, ‘Now are we the sons of God.’ 1 John 3:2.}


\othersnogap{Apostolul spune: „Noi toți \textbf{privim ca într-o oglindă}, cu fața descoperită, slava Domnului și suntem schimbați în același chip al Lui, din slavă în slavă, prin Duhul Domnului.” 2 Cor. 3:18. Cât de potrivită și frumoasă este această figură!... Astfel, \textbf{privind la Hristos} în minunile Sale, ispitele Sale, îndemnurile Sale, viața Sa de lepădare de sine, „umblarea Sa făcând bine”, \textbf{putem privi personalitatea și puterea lui Dumnezeu}. Și ce mare speranță există pentru noi în faptul că \textbf{în Hristos găsim calități care nu sunt străine și neobișnuite umanității}, ci caracteristici mentale și morale înrudite; astfel încât suntem capabili să vedem și să înțelegem un adevăr real, mai degrabă decât doar unul teologic sau abstract sau figurativ, în declarația apostolului: „Acum suntem copii ai lui Dumnezeu.” 1 Ioan 3:2.}


\othersnogap{\textbf{The fact that God is so great that we cannot form a clear mental picture of his }\textbf{\underline{physical appearance}}\textbf{ need not lessen in our minds the reality of }\textbf{\underline{His personality}}\textbf{, neither does this conception disagree with that of a special expression of God in some }\textbf{\underline{particular form or place}}. \textbf{\underline{Indeed, there are scriptures which present God in this definite, and one may say circumscribed, form as sitting upon a throne in heaven, or as dwelling in the temple at Jerusalem}}, 1. Kings 22:19; Ps. 11:4; Matt. 21:12, 13.}


\othersnogap{\textbf{Faptul că Dumnezeu este atât de mare încât nu putem forma o imagine mentală clară a }\textbf{\underline{înfățișării Sale fizice}}\textbf{ nu trebuie să diminueze în mintea noastră realitatea }\textbf{\underline{personalității Lui}}\textbf{, nici această concepție nu este în dezacord cu aceea a unei expresii speciale a lui Dumnezeu într-o }\textbf{\underline{formă sau loc anume}}. \textbf{\underline{Într-adevăr, există scripturi care Îl prezintă pe Dumnezeu în această formă definită și, s-ar putea spune, circumscrisă, ca stând pe un tron în cer, sau ca locuind în templul din Ierusalim}}, 1 Împărați 22:19; Ps. 11:4; Mat. 21:12, 13.}


\othersnogap{The human mind is finite and cannot grasp infinity. \textbf{We naturally desire to form a definite, clearly defined conception of the being whom we worship}. \textbf{The Bible supplies this human need as well as all other of our spiritual requirements, and }\textbf{\underline{in the fortieth chapter of Isaiah}}\textbf{ the prophet deals with this question of God's personal appearance in a marvelous way}. ‘O Jerusalem, that bringest good tiding, lift up thy voice with strength; lift it up, be not afraid; say unto the cities of Judah, \textbf{Behold your God}! He shall feed his flock like a shepherd: he shall gather the lambs in his arms, and carry them in his bosom.’}


\othersnogap{Mintea umană este finită și nu poate cuprinde infinitul. \textbf{Dorim în mod natural să formăm o concepție definită, clar conturată despre ființa pe care o adorăm}. \textbf{Biblia satisface această nevoie umană precum și toate celelalte cerințe spirituale ale noastre, și }\textbf{\underline{în capitolul patruzeci din Isaia}}\textbf{ profetul tratează această chestiune a înfățișării personale a lui Dumnezeu într-un mod minunat}. „Ierusalime, care aduci vești bune, înalță-ți glasul cu putere; înalță-l, nu te teme; spune cetăților lui Iuda: \textbf{Iată Dumnezeul vostru}! El Își va paște turma ca un păstor: va lua mieii în brațele Lui și-i va purta la sânul Lui.”}


\othersnogap{‘Who hath measured the waters in the hollow of \textbf{his hand}, and meted out heaven with the span, and comprehended the dust of the earth in a measure, and weighed the mountains in scales, and the hills in a balance? \textbf{To whom then will ye liken God?} \textbf{Or what likeness will ye compare unto him?} Have ye not known? have ye not heard? hath it not been told you from the beginning? have ye not understood from the foundations of the earth? \textbf{It is he that sitteth upon the circle of the earth}, and the inhabitants thereof are as grasshoppers; \textbf{that stretcheth out the heavens as a curtain, and spreadeth them out as a tent to dwell in}: \textbf{\underline{To whom then will ye liken me, or shall I be equal? saith the Holy One}}. Lift up your eyes on high, and behold who hath created these things, that bringeth out their host by number: he calleth them all by names by the greatness of his might, for that he is strong in power; not one faileth. Hast thou not known? hast thou not heard, that the everlasting God, the Lord, the Creator of the ends of the earth, fainteth not, neither is weary? There is no searching of his understanding. He giveth power to the faint and to them that have no might he increaseth strength. Even the youths shall faint and be weary, and the young men shall utterly fall: but they that wait upon the Lord shall renew their strength; they shall mount up with wings as eagles; they shall run, and not be weary; and they shall walk, and not faint.’ Isa. 40:9,11,12,18,21,22,25,26,28-31.}


\othersnogap{„Cine a măsurat apele în căușul \textbf{mâinii sale} și a măsurat cerurile cu palma și a strâns țărâna pământului într-o măsură și a cântărit munții în cumpene și dealurile într-o balanță? \textbf{Cu cine Îl veți asemăna pe Dumnezeu?} \textbf{Și cu ce asemănare Îl veți compara?} Nu știți? N-ați auzit? Nu vi s-a spus de la început? N-ați priceput de la întemeierea pământului? \textbf{El șade pe cercul pământului}, și locuitorii lui sunt ca lăcustele; \textbf{El întinde cerurile ca o perdea și le desfășoară ca un cort de locuit}: \textbf{\underline{Cu cine Mă veți asemăna deci, sau cu cine voi fi egal? zice Cel Sfânt}}. Ridicați-vă ochii în sus și priviți cine a făcut aceste lucruri, care scoate oștirea lor după număr: El le cheamă pe toate pe nume prin măreția puterii Lui, căci El este tare în putere; nici una nu lipsește. Nu știi? N-ai auzit că Dumnezeul cel veșnic, Domnul, Creatorul marginilor pământului, nu obosește, nici nu osteneşte? Priceperea Lui nu poate fi cercetată. El dă putere celui slab și celor ce n-au tărie le mărește puterea. Chiar tinerii obosesc și ostenesc, și tinerii voinici se clatină: dar cei ce așteaptă pe Domnul își înnoiesc puterea; se vor înălța cu aripi ca vulturii; vor alerga și nu vor obosi; vor umbla și nu vor osteni.” Isaia 40:9,11,12,18,21,22,25,26,28-31.}


\othersnogap{\textbf{Here is a most marvelous description of God. His hand, his arm, his bosom are mentioned}. He is described as ‘sitting on the circle of the earth,’ he metes out heaven with the span, he holds the waters in the hollow of his hand; \textbf{\underline{so there can be no question that God is a definite, real, personal being}}. \textbf{A mere abstract principle, a law, a force could not have a hand, an arm. \underline{God is a person}, though too great for us to comprehend, as Job says}, ‘God is great and we know him not.’ Job 36:26...}


\othersnogap{\textbf{Iată o descriere foarte minunată a lui Dumnezeu. Sunt menționate mâna Lui, brațul Lui, sânul Lui}. El este descris ca „șezând pe cercul pământului”, El măsoară cerurile cu palma, El ține apele în căușul mâinii Sale; \textbf{\underline{așa că nu poate fi nicio îndoială că Dumnezeu este o ființă definită, reală, personală}}. \textbf{Un simplu principiu abstract, o lege, o forță nu ar putea avea o mână, un braț. \underline{Dumnezeu este o persoană}, deși prea mare pentru noi să-L cuprindem, după cum spune Iov}, „Dumnezeu este mare și noi nu-L cunoaștem.” Iov 36:26...}


\othersnogap{\textbf{\underline{This great being} is represented as sitting on the circle of the earth}. The orbit of the earth is nearly two hundred million miles in diameter. \textbf{A being so great as to occupy a seat of such proportions is quite \underline{beyond our comprehension as regards his form}}. \textbf{The prophet recognizes this, and so \underline{diverts our attention away from speculation respecting the exact size and form of God} by showing us the absurdity of trying to form even a mental image, \underline{intimating that this is closely akin to idolatry}. See verses 18-21}. He then shows us where to find a true conception of God, pointing us to the things which he has made: ‘Lift up your eyes on high and behold who hath created these things.’ This also was Paul's idea : ‘For the invisible things of him from the creation of the world are clearly seen, being understood by the things that are made, \textbf{even his eternal power and \underline{Godhead}}; so that they are without excuse.’ Rom. 1:20.}


\othersnogap{\textbf{\underline{Această ființă măreață} este reprezentată ca șezând pe cercul pământului}. Orbita pământului are aproape două sute de milioane de mile în diametru. \textbf{O ființă atât de mare încât să ocupe un scaun de astfel de proporții este cu totul \underline{dincolo de înțelegerea noastră în ceea ce privește forma sa}}. \textbf{Profetul recunoaște aceasta și astfel \underline{ne îndepărtează atenția de la speculații privind mărimea și forma exactă a lui Dumnezeu} arătându-ne absurditatea încercării de a forma chiar și o imagine mentală, \underline{sugerând că aceasta este strâns înrudită cu idolatria}. Vezi versetele 18-21}. El ne arată apoi unde să găsim o concepție adevărată despre Dumnezeu, îndreptându-ne către lucrurile pe care le-a făcut: „Ridicați-vă ochii în sus și priviți cine a făcut aceste lucruri.” Aceasta a fost și ideea lui Pavel: „În adevăr, însușirile nevăzute ale Lui, de la facerea lumii, se văd lămurit, când te uiți cu băgare de seamă la ele în lucrurile făcute de El, \textbf{și anume puterea Lui veșnică și \underline{Dumnezeirea Lui}}; așa că nu se pot dezvinovăți.” Romani 1:20.}


\othersnogap{\textbf{\underline{Discussions respecting the form of God are utterly unprofitable}, and serve only to belittle our conceptions of him who is above all things}, \textbf{and hence not to be compared in form or size or glory or majesty with anything which man has ever seen or which it is within his power to conceive}. In the presence of questions like these, we have only to acknowledge our foolishness and incapacity, and bow our heads with awe and reverence \textbf{in the presence of a Personality, an Intelligent Being} to the existence of which all nature bears definite and positive testimony, \textbf{but which is as far beyond our comprehension \underline{as are the bounds of space and time}}.}


\othersnogap{\textbf{\underline{Discuțiile privind forma lui Dumnezeu sunt cu totul nefolositoare} și servesc doar să ne micșoreze concepțiile despre El care este mai presus de toate lucrurile}, \textbf{și prin urmare să nu fie comparat în formă sau mărime sau slavă sau maiestate cu nimic din ceea ce omul a văzut vreodată sau din ceea ce este în puterea lui să conceapă}. În fața unor întrebări ca acestea, nu avem decât să ne recunoaștem nebunia și incapacitatea și să ne plecăm capetele cu teamă și reverență \textbf{în prezența unei Personalități, a unei Ființe Inteligente} la a cărei existență toată natura aduce mărturie clară și pozitivă, \textbf{dar care este cu mult dincolo de înțelegerea noastră \underline{precum sunt limitele spațiului și timpului}}.}


As mentioned before, this brother acknowledges the \emcap{Fundamental Principles}, yet believes in the Trinity. Here is a short summary of His constructive criticism regarding the \emcap{personality of God}: God is a definite, real, personal being, having a form—\others{\textbf{Indeed, there are scriptures which present God in \underline{this definite}, and one may say \underline{circumscribed}, form as sitting upon a throne in heaven}}. He advocates this because he believes it is necessary for us, finite human beings, to have a definite object of worship. But he expands the idea of a “\textit{circumscribed} God by the testimony from Isaiah chapter 40, which proves that God is\others{\textbf{\underline{beyond our comprehension as regards his form}}}. Any kind of conceptualization of God’s being, in any form, is akin to idolatry. \others{\textbf{\underline{Discussions respecting the form of God are utterly unprofitable}}}. The true matter of the personality of infinite God is beyond our comprehension. God’s true personality is more than a mystery to our finite minds. This is because God is\others{\textbf{far beyond our comprehension \underline{as are the bounds of space and time}}}. For this brother, understanding God’s personality merely as a definite being is in one way true, but in another way false. It is true that God presented Himself in \others{\textbf{\underline{particular form or place}}}, because \others{there must be something more \textbf{tangible}, more \textbf{\underline{restricted}}, upon which to center the mind in worship}. A simple understanding of God as a definite and tangible being is restrictive for God. The summary of his criticism is that we should form our conceptions of God outside of \others{\textbf{the bounds of space and time}}.


După cum s-a menționat mai înainte, acest frate recunoaște \emcap{Principiile Fundamentale}, totuși crede în Trinitate. Iată un scurt rezumat al criticii sale constructive privind \emcap{personalitatea lui Dumnezeu}: Dumnezeu este o ființă definită, reală, personală, având o formă—\others{\textbf{Într-adevăr, există scripturi care Îl prezintă pe Dumnezeu în \underline{această formă definită} și, s-ar putea spune, \underline{circumscrisă}, ca stând pe un tron în cer}}. El susține aceasta pentru că crede că este necesar pentru noi, ființe umane finite, să avem un obiect definit de închinare. Dar el extinde ideea unui Dumnezeu „\textit{circumscris}” prin mărturia din Isaia capitolul 40, care dovedește că Dumnezeu este\others{\textbf{\underline{dincolo de înțelegerea noastră în ceea ce privește forma sa}}}. Orice fel de conceptualizare a ființei lui Dumnezeu, în orice formă, este înrudită cu idolatria. \others{\textbf{\underline{Discuțiile privind forma lui Dumnezeu sunt cu totul nefolositoare}}}. Adevărata chestiune a personalității lui Dumnezeu infinit este dincolo de înțelegerea noastră. Adevărata personalitate a lui Dumnezeu este mai mult decât un mister pentru mințile noastre finite. Aceasta pentru că Dumnezeu este\others{\textbf{cu mult dincolo de înțelegerea noastră \underline{precum sunt limitele spațiului și timpului}}}. Pentru acest frate, înțelegerea personalității lui Dumnezeu doar ca o ființă definită este într-un fel adevărată, dar în alt fel falsă. Este adevărat că Dumnezeu S-a prezentat pe Sine în \others{\textbf{\underline{formă sau loc anume}}}, pentru că \others{trebuie să existe ceva mai \textbf{tangibil}, mai \textbf{\underline{restrâns}}, asupra căruia să se concentreze mintea în închinare}. O simplă înțelegere a lui Dumnezeu ca o ființă definită și tangibilă este restrictivă pentru Dumnezeu. Rezumatul criticii sale este că ar trebui să ne formăm concepțiile despre Dumnezeu în afara \others{\textbf{limitelor spațiului și timpului}}.


Please, candidly examine the reasons behind this brother’s faith. The reasoning behind his arguments is important to understand because it played an important role in Seventh-day Adventist history, as a bold step away from the \emcap{Fundamental Principles}. These arguments are not trivial; they are very persuasive and we urge you to their contemplation. Perhaps you might agree with them, but please allow us to unmask the deception. These citations are from Dr. Kellogg’s book “\textit{The Living Temple}”\footnote{\href{https://archive.org/details/J.H.Kellogg.TheLivingTemple1903}{Dr. J. H. Kellogg, The Living Temple, p.29-33.}}. From the section titled “\textit{Infinite Intelligence a Personal being}”, pages 29 to 33, the passages express Kellogg’s position on the \emcap{personality of God}, which was the main problem with his book.


Vă rugăm, examinați sincer motivele din spatele credinței acestui frate. Raționamentul din spatele argumentelor sale este important de înțeles pentru că a jucat un rol important în istoria adventistă de ziua a șaptea, ca un pas îndrăzneț departe de \emcap{Principiile Fundamentale}. Aceste argumente nu sunt triviale; ele sunt foarte convingătoare și vă îndemnăm să le contemplați. Poate că ați putea fi de acord cu ele, dar vă rugăm să ne permiteți să demascăm înșelăciunea. Aceste citate sunt din cartea Dr. Kellogg „\textit{Templul viu}”\footnote{\href{https://archive.org/details/J.H.Kellogg.TheLivingTemple1903}{Dr. J. H. Kellogg, Templul viu, p.29-33.}}. Din secțiunea intitulată „\textit{Inteligența Infinită o ființă Personală}”, paginile 29 până la 33, pasajele exprimă poziția lui Kellogg privind \emcap{personalitatea lui Dumnezeu}, care a fost principala problemă cu cartea sa.


That which you just read was exactly what Sister White referred to when she said: \egwinline{I have some things to say to our teachers \textbf{in reference to the new book The Living Temple}. \textbf{Be careful how you sustain the sentiments of this book \underline{regarding the personality of God}}. As the Lord presents matters to me, \textbf{these sentiments do not bear the endorsement of God}. \textbf{They are a snare that the enemy has prepared for these last days}...}[Lt211-1903.1; 1903][https://egwwritings.org/read?panels=p9598.8]


Ceea ce tocmai ați citit a fost exact la ce se referea sora White când a spus: \egwinline{Am câteva lucruri de spus învățătorilor noștri \textbf{cu privire la noua carte Templul viu}. \textbf{Fiți atenți cum susțineți opiniile din această carte \underline{privind personalitatea lui Dumnezeu}}. După cum Domnul îmi prezintă lucrurile, \textbf{aceste opinii nu poartă aprobarea lui Dumnezeu}. \textbf{Ele sunt o cursă pe care vrăjmașul a pregătit-o pentru aceste zile din urmă}...}[Lt211-1903.1; 1903][https://egwwritings.org/read?panels=p9598.8]


In the present Seventh-day Adventist controversy over the Trinity doctrine, we have personally been trying to shift the controversy from the Trinity doctrine to the \emcap{personality of God}. We’ve presented the position of the first point of the \emcap{Fundamental Principles} and have encountered arguments that greatly overlap with Dr. Kellogg’s sentiment on the \emcap{personality of God}, advocated in “\textit{Living Temple}”. We’ve seen this repeatedly. When the focus is drawn from the Trinity issue to the \emcap{personality of God}, Kellogg’s views regarding the \emcap{personality of God} frequently echoe from the lips of Trinitarian advocates. The quality or state of God being a person is a mystery in the Trinity doctrine, and often Kellogg’s sentiment on the \emcap{personality of God} resonates with Trinitarian understanding of God’s person.


În actuala controversă adventistă de ziua a șaptea privind doctrina Trinității, noi personal am încercat să mutăm controversa de la doctrina Trinității la \emcap{personalitatea lui Dumnezeu}. Am prezentat poziția primului punct al \emcap{Principiilor Fundamentale} și am întâlnit argumente care se suprapun mult cu opinia Dr. Kellogg privind \emcap{personalitatea lui Dumnezeu}, susținută în „\textit{Templul viu}”. Am văzut aceasta în mod repetat. Când atenția este atrasă de la problema Trinității la \emcap{personalitatea lui Dumnezeu}, viziunile lui Kellogg privind \emcap{personalitatea lui Dumnezeu} răsună frecvent de pe buzele susținătorilor Trinității. Caracteristica sau starea prin care Dumnezeu este definit ca persoană este un mister în doctrina Trinității, și adesea opinia lui Kellogg privind \emcap{personalitatea lui Dumnezeu} rezonează cu înțelegerea trinitariană a persoanei lui Dumnezeu.


\begin{figure}[hp]
    \centering
    \includegraphics[width=1\linewidth]{images/TLT.jpg}
    \caption*{The Living Temple by Dr. J. H. Kellogg, 1903}
    \label{fig:tlt}
\end{figure}


\begin{figure}[hp]
    \centering
    \includegraphics[width=1\linewidth]{images/TLT.jpg}
    \caption*{Templul viu de Dr. J. H. Kellogg, 1903}
    \label{fig:tlt}
\end{figure}


Some people find Dr. Kellogg’s understanding of God’s personality resonates with their understanding, yet they are tempted to think that there are other things objectionable with the Living Temple. The following evidence suggests the very opposite. There is a letter from Dr. Kellogg to William C. White, where Dr. Kellogg proposes to \others{cutting out a few leaves} from the three thousand copies of the Living Temple—those very leaves containing the \others{specially objectionable things appear, such as the comment on Isaiah 40} and the sentiments regarding the \emcap{personality of God} (the pages we have read).


Unii oameni consideră că înțelegerea Dr. Kellogg despre personalitatea lui Dumnezeu rezonează cu înțelegerea lor, totuși sunt tentați să creadă că există alte lucruri de obiectat în Templul viu. Dovezile următoare sugerează exact opusul. Există o scrisoare de la Dr. Kellogg către William C. White, unde Dr. Kellogg propune să \others{taie câteva foi} din cele trei mii de exemplare ale cărții Templul viu—tocmai acele foi care conțin \others{lucrurile deosebit de condamnabile, cum ar fi comentariul asupra Isaia 40} și opiniile privind \emcap{personalitatea lui Dumnezeu} (paginile pe care le-am citit).


\others{The Sanitarium has on hand, I find, \textbf{two or three thousand books which were sold}, but which have come back since the book was condemned. The question has been raised, what shall be done with these? \textbf{It has occurred to me that perhaps they might be saved \underline{by cutting out a few leaves} in which the \underline{specially objectionable things appear}, such as the \underline{comment on Isaiah 40}, which I borrowed from A.T. Jones, and the page on which the unfortunate heading appears, ‘\underline{The Personality of God},’ and tipping in leaves embodying a clear statement of the Bible view of God as a person presented in Elder Haskell’s article in the ‘Review’ a few weeks ago}. These books would be sold to old patients who are making a great demand for the book for Christmas presents…}[Letter from Dr. J.H. Kellogg to W.C.White; December 6, 1903, Chicago][https://174625.selcdn.ru/ellenwhite/EWhite/17226/17226.pdf]


\others{Sanatoriul are la dispoziție, constat, \textbf{două sau trei mii de cărți care au fost vândute}, dar care s-au întors de când cartea a fost condamnată. S-a ridicat întrebarea, ce să facem cu acestea? \textbf{Mi-a venit ideea că poate ar putea fi salvate \underline{prin tăierea câtorva foi} în care apar \underline{lucrurile deosebit de condamnabile}, cum ar fi \underline{comentariul asupra Isaia 40}, pe care l-am împrumutat de la A.T. Jones, și pagina pe care apare titlul nefericit, ‘\underline{Personalitatea lui Dumnezeu},’ și lipirea unor foi care să cuprindă o declarație clară a perspectivei biblice despre Dumnezeu ca persoană prezentată în articolul fratelui Haskell din ‘Review’ de acum câteva săptămâni}. Aceste cărți ar fi vândute pacienților vechi care cer cu insistență cartea pentru cadouri de Crăciun…}[Scrisoare de la Dr. J.H. Kellogg către W.C.White; 6 decembrie 1903, Chicago][https://174625.selcdn.ru/ellenwhite/EWhite/17226/17226.pdf]


What is the real issue with the reasoning in the Living Temple? We will study the matter to its very depth; superficially, we clearly see that the issue is the stepping off of the foundation of our faith—the \emcap{Fundamental Principles}—regarding the \emcap{personality of God} and where His presence is.


Care este adevărata problemă cu raționamentul din Templul viu? Vom studia problema până în profunzime; superficial, vedem clar că problema este îndepărtarea de la temelia credinței noastre—\emcap{Principiile Fundamentale}—privind \emcap{personalitatea lui Dumnezeu} și unde este prezența Sa.


\egw{\textbf{I have been instructed by the heavenly messenger} that \textbf{some of the reasoning} in the book, ‘Living Temple’, is unsound and that \textbf{this reasoning would lead astray} the minds of those who are not thoroughly established on \textbf{the foundation principles} of present truth. \textbf{It introduces that which is naught but speculation} in \textbf{regard to the personality of God and where His presence is}.}[SpTB02 51.3; 1904][https://egwwritings.org/read?panels=p417.262]


\egw{\textbf{Am fost instruită de mesagerul ceresc} că \textbf{o parte din raționamentul} din cartea ‘Templul viu’ este nesănătos și că \textbf{acest raționament ar duce în rătăcire} mințile celor care nu sunt temeinic stabiliți pe \textbf{principiile de bază} ale adevărului prezent. \textbf{Introduce ceea ce nu este altceva decât speculație} în privința \textbf{personalității lui Dumnezeu și unde este prezența Sa}.}[SpTB02 51.3; 1904][https://egwwritings.org/read?panels=p417.262]


Dr. Kellogg introduced the thought \egwinline{which is naught but speculation in regard to the personality of God}, by which he stepped off of the foundation of our faith—the \emcap{Fundamental Principles}. Discordance between Dr. Kellogg’s teaching and the \emcap{Fundamental Principles} is in the first statement of the principles where we are taught that\others{That there is \textbf{one God}, \textbf{a personal, spiritual \underline{being}}, \textbf{the creator of all things}, ... and \textbf{everywhere present by his representative, the Holy Spirit}. Ps. 139:7.}


Dr. Kellogg a introdus gândul \egwinline{care nu este altceva decât speculație în privința personalității lui Dumnezeu}, prin care s-a îndepărtat de la temelia credinței noastre—\emcap{Principiile Fundamentale}. Discordanța dintre învățătura Dr. Kellogg și \emcap{Principiile Fundamentale} se află în prima declarație a principiilor unde suntem învățați că \others{Există \textbf{un singur Dumnezeu}, \textbf{o ființă personală, spirituală}, \textbf{creatorul tuturor lucrurilor}, ... și \textbf{prezent pretutindeni prin reprezentantul său, Duhul Sfânt}. Ps. 139:7.}


Sister White directly warned us of the sentiments expressed in the Living Temple regarding the \emcap{personality of God}. They are not in harmony with the first point of the \emcap{Fundamental Principles}, which were part of the foundation of our faith.


Sora White ne-a avertizat direct despre opiniile exprimate în Templul viu privind \emcap{personalitatea lui Dumnezeu}. Ele nu sunt în armonie cu primul punct al \emcap{Principiilor Fundamentale}, care făceau parte din temelia credinței noastre.


\egw{\textbf{I have had to write much concerning the strange doctrines and theories expressed in Living Temple. \underline{Were these theories accepted by our people, the strong pillars of our faith and the truths that have made Seventh-day Adventists what they are would be swept away}. I have had to show the fallacy of these doctrines, presenting them \underline{as a species of last-day heresy}. We are told by the Word of God that just such teaching \underline{will be brought in at this time}.}}[Lt250-1903.2; 1903][https://egwwritings.org/read?panels=p9337.8]


\egw{\textbf{Am fost nevoită să scriu mult despre doctrinele și teoriile ciudate exprimate în Templul viu. \underline{Dacă aceste teorii ar fi acceptate de poporul nostru, stâlpii puternici ai credinței noastre și adevărurile care au făcut din adventiștii de ziua a șaptea ceea ce sunt ar fi măturate}. A trebuit să arăt falsitatea acestor doctrine, prezentându-le \underline{ca o specie de erezie a zilelor din urmă}. Suntem spuși de Cuvântul lui Dumnezeu că exact o astfel de învățătură \underline{va fi adusă în acest timp}.}}[Lt250-1903.2; 1903][https://egwwritings.org/read?panels=p9337.8]


Today we witness the widespread acceptance of Kellogg’s theories regarding the \emcap{personality of God}. The fact that the first point of the \emcap{Fundamental Principles} is no longer present in our beliefs proves that Kellogg’s theories regarding the \emcap{personality of God} have had an influence in shaping our beliefs.


Astăzi suntem martori la acceptarea pe scară largă a teoriilor lui Kellogg privind \emcap{personalitatea lui Dumnezeu}. Faptul că primul punct al \emcap{Principiilor Fundamentale} nu mai este prezent în credințele noastre dovedește că teoriile lui Kellogg privind \emcap{personalitatea lui Dumnezeu} au avut o influență în modelarea credințelor noastre.


\egw{One and another come to me, asking me to \textbf{explain the positions taken in “Living Temple.”} I reply, “They are unexplainable.” \textbf{The sentiments expressed do not give a true knowledge of God.} \textbf{All through the book are passages of scripture}. \textbf{These scriptures are brought in in such a way \underline{that error is made to appear as truth}}. \textbf{Erroneous theories are presented in so pleasing a way that unless care is taken, many will be misled}.}[SpTB02 52.1; 1904][https://egwwritings.org/read?panels=p417.265]


\egw{Unul și altul vin la mine, cerându-mi să \textbf{explic pozițiile luate în “Templul viu.”} Răspund: “Sunt inexplicabile.” \textbf{Opiniile exprimate nu oferă o cunoaștere adevărată despre Dumnezeu.} \textbf{Prin toată cartea sunt pasaje din Scriptură}. \textbf{Aceste scripturi sunt aduse în așa fel \underline{încât eroarea este făcută să apară ca adevăr}}. \textbf{Teoriile eronate sunt prezentate într-un mod atât de plăcut încât, dacă nu se ia seama, mulți vor fi înșelați}.}[SpTB02 52.1; 1904][https://egwwritings.org/read?panels=p417.265]


The error is being made to appear as truth, and many are misled.


Eroarea este făcută să pară ca adevăr, iar mulți sunt înșelați.


It is worth emphasizing, for some careless reader, that the real issue of Dr. Kellogg, and his book “\textit{Living Temple}”, is not the Trinity but the small step he took off of the \emcap{Fundamental Principles}. In order to understand the real issue of his book, it would be wrong to focus on its overlapping sentiments with the Trinity doctrine. Rather, we should focus on the point that constituted this small step he made; and this includes having a deep understanding of the \emcap{fundamental principles} just as our pioneers had. Who better to ask than the Adventist pioneers themselves?


Merită să subliniem, pentru vreun cititor neglijent, că adevărata problemă a Dr. Kellogg și a cărții sale „\textit{Templul viu}” nu este Trinitatea, ci micul pas pe care l-a făcut în afara \emcap{Principiilor Fundamentale}. Pentru a înțelege adevărata problemă a cărții sale, ar fi greșit să ne concentrăm asupra opiniilor sale care se suprapun cu doctrina Trinității. Mai degrabă, ar trebui să ne concentrăm asupra punctului care a constituit acest mic pas pe care l-a făcut; și aceasta include o înțelegere profundă a \emcap{principiilor fundamentale} exact așa cum le-au avut pionierii noștri. Pe cine să întrebăm mai bine decât pe pionierii adventiști înșiși?


% Constructive Criticism

\begin{titledpoem}
    
    \stanza{
        A person, God in heav’n, enthroned, \\
        In this our founding truths were zoned. \\
        All-Present by His Spirit’s might, \\
        These truths stood as our guiding light.
    }

    \stanza{
        False words that seemed so wise and deep, \\
        A subtle shift made faithful weep. \\
        "God’s form beyond all thought," they claimed, \\
        This mystery could not be named.
    }

    \stanza{
        "Discussions of God’s form," he said, \\
        "Are futile paths that lie ahead." \\
        Yet this deceit, so smoothly spun, \\
        Was Satan’s snare, and souls were won.
    }

    \stanza{
        The error dressed as truth so fair, \\
        And twisted in a clever snare. \\
        Just One small step from truths we held, \\
        By One giant leap our faith was felled.
    }

    \stanza{
        Beware the mind that seems too wise, \\
        To see deception in disguise. \\
        The truth is—God is personal \\
        This truth the Doctor would conceal.
    }
    
\end{titledpoem}