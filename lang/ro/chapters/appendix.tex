\qrchapterstar{https://forgottenpillar.com/rsc/ro-fp-appendix}{Apendix} \label{chap:appendix}

\addcontentsline{toc}{chapter}{Apendix}

\section*{Principiile fundamentale 1889}


As elsewhere stated, Seventh-day Adventists have no creed but the Bible; but they hold to certain well-defined points of faith for which they feel prepared to give a reason “to every man that asketh” them. The following propositions may be taken as a summary of the principal features of their religious faith, upon which there is, so far as we know, entire unanimity throughout the body. They believe,—


Așa cum s-a menționat în alte locuri, adventiștii de ziua a șaptea nu au un crez decât Biblia; dar ei se țin de anumite puncte bine definite de credință pentru care se simt pregătiți să dea un motiv „oricărui om care îi întreabă”. Următoarele propoziții pot fi luate ca un rezumat al caracteristicilor principale ale credinței lor religioase, asupra căreia există, pe cât știm noi, unanimitate deplină în întreaga comunitate. Ei cred,—


\lettrine{I.} That there is one God, a personal, spiritual being, the creator of all things, omnipotent, omniscient, and eternal; infinite in wisdom, holiness, justice, goodness, truth, and mercy; unchangeable, and everywhere present by his representative, the Holy Spirit. Psalm 139:7.


\lettrine{I.} Că există un Dumnezeu, o ființă personală și spirituală, creatorul tuturor lucrurilor, omnipotent, omniscient și etern; infinit în înțelepciune, sfințenie, dreptate, bunătate, adevăr și milă; neschimbător și prezent peste tot prin reprezentantul Său, Duhul Sfânt. Psalmul 139:7.


\lettrine{II.} That there is one Lord Jesus Christ, the Son of the Eternal Father, the one by whom he created all things, and by whom they do consist; that he took on him the nature of the seed of Abraham for the redemption of our fallen race; that he dwelt among men, full of grace and truth, lived our example, died our sacrifice, was raised for our justification, ascended on high to be our only mediator in the sanctuary in heaven, where, through the merits of his shed blood, he secures the pardon and forgiveness of the sins of all those who penitently come to him; and as the closing portion of his work as priest, before he takes his throne as king, he will make the great atonement for the sins of all such, and their sins will then be blotted out (Acts 3:19) and borne away from the sanctuary, as shown in the service of the Levitical priesthood, which foreshadowed and prefigured the ministry of our Lord in heaven. See Leviticus 16; Hebrews 8:4, 5; 9:6, 7; etc.


\lettrine{II.} Că există un Domn Isus Hristos, Fiul Tatălui Etern, cel prin care a creat toate lucrurile și prin care ele subsistă; că a luat asupra Sa natura seminței lui Avraam pentru răscumpărarea neamului nostru căzut; că a locuit între oameni, plin de har și adevăr, a trăit ca exemplul nostru, a murit ca jertfa noastră, a fost înviat pentru justificarea noastră, a urcat sus pentru a fi singurul nostru mijlocitor în sanctuarul din cer, unde, prin meritele sângelui Său vărsat, asigură iertarea și ștergerea păcatelor tuturor celor care se întorc cu pocăință la El; și ca partea finală a lucrării Sale de preot, înainte de a-și lua tronul ca împărat, va face marea ispășire pentru păcatele tuturor acestora, și păcatele lor vor fi apoi șterse (Fapte 3:19) și îndepărtate din sanctuar, așa cum se arată în slujba preoției levitice, care a prefigurat și a prevestit slujba Domnului nostru în cer. Vezi Leviticul 16; Evrei 8:4, 5; 9:6, 7; etc.


\lettrine{III.} That the Holy Scriptures of the Old and New Testaments were given by inspiration of God, contain a full revelation of his will to man, and are the only infallible rule of faith and practice.


\lettrine{III.} Că Sfintele Scripturi ale Vechiului și Noului Testament au fost date prin inspirația lui Dumnezeu, conțin o revelaţie deplină a voinţei Sale către om și sunt singura regulă infailibilă a credinței și practicii.


\lettrine{IV.} That baptism is an ordinance of the Christian church, to follow faith and repentance,—an ordinance by which we commemorate the resurrection of Christ, as by this act we show our faith in his burial and resurrection, and through that, in the resurrection of all the saints at the last day; and that no other mode more fitly represents these facts than that which the Scriptures prescribe, namely, immersion. Romans 6:3-5; Colossians 2:12.


\lettrine{IV.} Că botezul este o ordonanță a bisericii creștine, care trebuie să urmeze credinței și pocăinței,—o ordonanță prin care comemorăm învierea lui Hristos, deoarece prin acest act arătăm credința noastră în înmormântarea și învierea Sa, și prin aceasta, în învierea tuturor sfinților în ziua de apoi; și că niciun alt mod nu reprezintă mai potrivit aceste fapte decât cel pe care Scripturile îl prescriu, și anume, imersiunea. Romani 6:3-5; Coloseni 2:12.


\lettrine{V.} That the new birth comprises the entire change necessary to fit us for the kingdom of God, and consists of two parts; First, a moral change wrought by conversion and a Christian life (John 3:3, 5); second, a physical change at the second coming of Christ, whereby, if dead, we are raised incorruptible, and if living, are changed to immortality in a moment, in the twinkling of an eye. Luke 20:36; 1 Corinthians 15:51, 52.


\lettrine{V.} Că noua naștere cuprinde întreaga schimbare necesară pentru a ne pregăti pentru împărăția lui Dumnezeu și constă din două părți; În primul rând, o schimbare morală efectuată prin convertire și o viață creștină (Ioan 3:3, 5); în al doilea rând, o schimbare fizică la a doua venire a lui Hristos, prin care, dacă suntem morți, suntem înviați incorruptibili, și dacă suntem vii, suntem schimbați în nemurire într-o clipă, într-o clipire din ochi. Luca 20:36; 1 Corinteni 15:51, 52.


\lettrine{VI.} That prophecy is a part of God’s revelation to man; that it is included in that Scripture which is profitable for instruction (2 Timothy 3:16); that it is designed for us and our children (Deuteronomy 29:29); that so far from being enshrouded in impenetrable mystery, it is that which especially constitutes the word of God a lamp to our feet and a light to our path (Psalm 119:105; 2 Peter 1:19); that a blessing is pronounced upon those who study it (Revelation 1:1-3); and that, consequently, it is to be understood by the people of God sufficiently to show them their position in the world’s history and the special duties required at their hands.


\lettrine{VI.} Că profeția este o parte a revelaţiei lui Dumnezeu către om; că este inclusă în acea Scriptură care este folositoare pentru învățătură (2 Timotei 3:16); că este destinată nouă și copiilor noștri (Deuteronom 29:29); că departe de a fi învăluită în mister impenetrabil, ea este cea care constituie în special cuvântul lui Dumnezeu o lampă la picioarele noastre și o lumină pe cărări noastre (Psalmul 119:105; 2 Petru 1:19); că o binecuvântare este pronunțată asupra celor care o studiază (Apocalipsa 1:1-3); și că, prin urmare, trebuie să fie înțeleasă de poporul lui Dumnezeu suficient pentru a le arăta poziția lor în istoria lumii și datoriile speciale cerute de la ei.


\lettrine{VII.} That the world’s history from specified dates in the past, the rise and fall of empires, and the chronological succession of events down to the setting up of God’s everlasting kingdom, are outlined in numerous great chains of prophecy; and that these prophecies are now all fulfilled except the closing scenes.


\lettrine{VII.} Că istoria lumii din date specificate din trecut, ridicarea și căderea imperiilor, și succesiunea cronologică a evenimentelor până la întemeierea împărăției veșnice a lui Dumnezeu, sunt prezentate în numeroase mari lanțuri de profeție; și că aceste profeții sunt acum toate împlinite cu excepția scenelor finale.


\lettrine{VIII.} That the doctrine of the world’s conversion and a temporal millennium is a fable of these last days, calculated to lull men into a state of carnal security, and cause them to be overtaken by the great day of the Lord as by a thief in the night (1 Thessalonians 5:3); that the second coming of Christ is to precede, not follow, the millennium; for until the Lord appears, the papal power, with all its abominations, is to continue (2 Thessalonians 2:8), the wheat and tares grow together (Matthew 13:29, 30, 39), and evil men and seducers wax worse and worse, as the word of God declares. 2 Timothy 3:1, 13.


\lettrine{VIII.} Că doctrina conversiunii lumii și a unui mileniu temporal este o fabulă a acestor zile din urmă, calculată pentru a adormi pe oameni într-o stare de securitate carnală, și pentru a-i surprinde prin marea zi a Domnului ca de un hoț în noapte (1 Tesaloniceni 5:3); că a doua venire a lui Hristos trebuie să preceadă, nu să urmeze, mileniului; căci până când apare Domnul, puterea papală, cu toate abominațiunile ei, trebuie să continue (2 Tesaloniceni 2:8), grâul și neghina cresc împreună (Matei 13:29, 30, 39), și oamenii răi și seducătorii se fac din ce în ce mai răi, după cum declară cuvântul lui Dumnezeu. 2 Timotei 3:1, 13.


\lettrine{IX.} That the mistake of Adventists in 1844 pertained to the nature of the event then to transpire, not to the time; that no prophetic period is given to reach to the second advent, but that the longest one, the two thousand and three hundred days of Daniel 8:14, terminated in 1844, and brought us to an event called the cleansing of the sanctuary.


\lettrine{IX.} Că greșeala Adventiștilor în 1844 se referea la natura evenimentului care urma să se întâmple atunci, nu la timp; că nicio perioadă profetică nu este dată pentru a ajunge la a doua venire, dar cea mai lungă, cei doi mii trei sute de zile din Daniel 8:14, s-a încheiat în 1844, și ne-a adus la un eveniment numit curățirea Sanctuarului.


\lettrine{X.} That the sanctuary of the new covenant is the tabernacle of God in heaven, of which Paul speaks in Hebrews 8 and onward, and of which our Lord, as great high priest, is minister; that this sanctuary is the antitype of the Mosaic tabernacle, and that the priestly work of our Lord, connected therewith, is the antitype of the work of the Jewish priests of the former dispensation (Hebrews 8:1-5, etc.); that this, and not the earth, is the sanctuary to be cleansed at the end of the two thousand and three hundred days, what is termed its cleansing being in this case, as in the type, simply the entrance of the high priest into the most holy place, to finish the round of service connected therewith, by making the atonement and removing from the sanctuary the sins which had been transferred to it by means of the ministration in the first apartment (Leviticus 16; Hebrews 9:22, 23); and that this work in the antitype, beginning in 1844, consists in actually blotting out the sins of believers (Acts 3:19), and occupies a brief but indefinite space of time, at the conclusion of which the work of mercy for the world will be finished, and the second advent of Christ will take place.


\lettrine{X.} Că sanctuarul noului legământ este tabernacolul lui Dumnezeu în cer, despre care vorbește Pavel în Evrei 8 și mai departe, și despre care Domnul nostru, ca mare preot, este slujitor; că acest sanctuar este antitipul tabernacolului mosaician, și că lucrarea preoțească a Domnului nostru, legată de acesta, este antitipul lucrării preoților iudei din dispensația anterioară (Evrei 8:1-5, etc.); că acesta, și nu pământul, este sanctuarul care trebuie curățit la sfârșitul celor doi mii trei sute de zile, ceea ce se numește curățirea lui fiind în acest caz, ca și în tip, pur și simplu intrarea marelui preot în locul preasfânt, pentru a termina seria de serviciu legată de aceasta, prin a face ispășirea și a îndepărta din sanctuar păcatele care fuseseră transferate în el prin intermediul slujbei din apartamentul prim (Levitic 16; Evrei 9:22, 23); și că această lucrare în antitip, începând din 1844, constă în ștergerea efectivă a păcatelor credincioșilor (Fapte 3:19), și ocupă un spațiu de timp scurt dar nedefinit, la sfârșitul căruia lucrarea de milă pentru lume va fi terminată, și a doua venire a lui Hristos va avea loc.


\lettrine{XI.} That God’s moral requirements are the same upon all men in all dispensations; that these are summarily contained in the commandments spoken by Jehovah from Sinai, engraven on the tables of stone, and deposited in the ark, which was in consequence called the “ark of the covenant,” or testament (Numbers 10:33; Hebrews 9:4, etc.); that this law is immutable and perpetual, being a transcript of the tables deposited in the ark in the true sanctuary on high, which is also, for the same reason, called the ark of God’s testament; for under the sounding of the seventh trumpet we are told that “the temple of God was opened in heaven, and there was seen in his temple the ark of his testament.” Revelation 11:19.


\lettrine{XI.} Că cerințele morale ale lui Dumnezeu sunt aceleași pentru toți oamenii în toate dispensațiile; că acestea sunt cuprinse pe scurt în poruncile rostite de Iehova din Sinai, săpate pe tablele de piatră, și depuse în chivot, care din această cauză a fost numit „chivotul legământului,” sau testamentul (Numeri 10:33; Evrei 9:4, etc.); că această lege este imuabilă și perpetuă, fiind o copie a tablelor depuse în chivot în adevăratul sanctuar de sus, care este de asemenea, din același motiv, numit chivotul testamentului lui Dumnezeu; căci sub sunetul celei de-a șaptea trâmbe ni se spune că „templul lui Dumnezeu s-a deschis în cer, și s-a văzut în templul Lui chivotul testamentului Lui.” Apocalipsa 11:19.


\lettrine{XII.} That the fourth commandment of this law requires that we devote the seventh day of each week, commonly called Saturday, to abstinence from our own labor, and to the performance of sacred and religious duties; that this is the only weekly Sabbath known to the Bible, being the day that was set apart before Paradise was lost (Genesis 2:2, 3), and which will be observed in Paradise restored (Isaiah 66:22, 23); that the facts upon which the Sabbath institution is based confine it to the seventh day, as they are not true of any other day; and that the terms Jewish Sabbath, as applied to the seventh day, and Christian Sabbath, as applied to the first day of the week, are names of human invention, unscriptural in fact, and false in meaning.


\lettrine{XII.} Că a patra poruncă a acestei legi cere ca noi să dedicăm a șaptea zi a fiecărei săptămâni, numită în mod obișnuit sâmbătă, abstinență de la munca noastră, și la îndeplinirea datoriilor sacre și religioase; că aceasta este singura Sabat săptămânală cunoscută Bibliei, fiind ziua care a fost pusă deoparte înainte ca Paradisul să fie pierdut (Geneza 2:2, 3), și care va fi observată în Paradisul restaurat (Isaia 66:22, 23); că faptele pe care se bazează instituția Sabatului o limitează la a șaptea zi, deoarece nu sunt adevărate pentru nicio altă zi; și că termenii Sabat iudaic, aplicați a șaptea zi, și Sabat creștin, aplicați primei zile a săptămânii, sunt nume de invenție umană, nebiblice în fapt, și false în înțeles.


\lettrine{XIII.} That as the man of sin, the papacy, has thought to change times and laws (the law of God, Daniel 7:25), and has misled almost all Christendom in regard to the fourth commandment, we find a prophecy of a reform in this respect to be wrought among believers just before the coming of Christ. Isaiah 56:1, 2; 1 Peter 1:5; Revelation 14:12, etc.


\lettrine{XIII.} Că omul păcatului, papacia, a gândit să schimbe timpurile și legile (legea lui Dumnezeu, Daniel 7:25), și a indus în eroare aproape toată creștinătatea cu privire la a patra poruncă, găsim o profeție a unei reforme în această privință care trebuie să fie săvârșită printre credincioși chiar înainte de venirea lui Hristos. Isaia 56:1, 2; 1 Petru 1:5; Apocalipsa 14:12, etc.


\lettrine{XIV.} That the followers of Christ should be a peculiar people, not following the maxims, nor conforming to the ways, of the world; not loving its pleasures nor countenancing its follies; inasmuch as the apostle says that “whosoever therefore will be” in this sense, “a friend of the world, is the enemy of God” (James 4:4); and Christ says that we cannot have two masters, or, at the same time, serve God and mammon. Matthew 6:24.


\lettrine{XIV.} Că urmașii lui Hristos trebuie să fie un popor deosebit, care nu urmează maximele, nici nu se conformează căilor lumii; care nu iubesc plăcerile ei nici nu aprobă nebuniile ei; întrucât apostolul spune că „oricine deci va vrea” în acest sens, „prieten al lumii, este vrăjmaș al lui Dumnezeu” (Iacov 4:4); și Hristos spune că nu putem avea doi stăpâni, sau, în același timp, să slujim lui Dumnezeu și lui Mamona. Matei 6:24.


\lettrine{XV.} That the Scriptures insist upon plainness and modesty of attire as a prominent mark of discipleship in those who profess to be the followers of Him who was, “meek and lowly in heart,” that the wearing of gold, pearls, and costly array, or anything designed merely to adorn the person and foster the pride of the natural heart, is to be discarded, according to such scriptures as 1 Timothy 2:9, 10; 1 Peter 3:3, 4.


\lettrine{XV.} Că Scripturile insistă asupra simplității și modestiei în îmbrăcăminte ca o marcă proeminentă a discipolatului în cei care se pretind a fi urmașii Celui care era „blând și smerit la inimă,” că purtarea aurului, perlelor, și a veșmintelor scumpe, sau orice lucru conceput doar pentru a orna persoana și a cultiva mândria inimii naturale, trebuie să fie respingă, conform unor scripturi precum 1 Timotei 2:9, 10; 1 Petru 3:3, 4.


\lettrine{XVI.} That means for the support of evangelical work among men should be contributed from love to God and love of souls, not raised by church lotteries, or occasions designed to contribute to the fun-loving, appetite-indulging propensities of the sinner, such as fairs, festivals, oyster suppers, tea, broom, donkey, and crazy socials, etc., which are a disgrace to the professed church of Christ; that the proportion of one’s income required in former dispensation can be no less under the gospel; that it is the same as Abraham (whose children we are, if we are Christ’s, Galatians 3:29) paid to Melchisedec (type of Christ) when he gave him a tenth of all (Hebrews 7:1-4); the title is the Lord’s (Leviticus 27:30); and this tenth of one’s income is also to be supplemented by offerings from those who are able, for the support of the gospel. 2 Corinthians 9:6; Malachi 3:8, 10.


\lettrine{XVI.} Că mijloacele pentru susținerea lucrării evanghelice în rândul oamenilor trebuie să fie contribuite din dragoste pentru Dumnezeu și dragoste pentru suflete, nu ridicate prin lozuri de biserică, sau ocazii concepute pentru a contribui la propensiunile iubitoare de distracție și indulgență în apetit ale păcătosului, cum ar fi târguri, festivaluri, cine cu stridii, ceai, mătură, măgar, și socializări nebunești, etc., care sunt o rușine pentru biserica pretinsă a lui Hristos; că proporția din venitul cuiva cerută în dispensația anterioară nu poate fi mai mică sub evanghelie; că este aceeași ca a lui Avraam (ai cărui copii suntem, dacă suntem ai lui Hristos, Galateni 3:29) a plătit lui Melchisedec (tip al lui Hristos) când i-a dat o zecime din toate (Evrei 7:1-4); zecimea este a Domnului (Levitic 27:30); și această zecime din venitul cuiva trebuie de asemenea să fie completată prin ofrandele din cei care sunt în stare, pentru susținerea evangheliei. 2 Corinteni 9:6; Maleahi 3:8, 10.


\lettrine{XVII.} That as the natural or carnal heart is at enmity with God and his law, this enmity can be subdued only by a radical transformation of the affections, the exchange of unholy for holy principles; that this transformation follows repentance and faith, is the special work of the Holy Spirit, and constitutes regeneration, or conversion.


\lettrine{XVII.} Că, deoarece inima firească sau carnală este în vrăjmășie cu Dumnezeu și legea Lui, această vrăjmășie poate fi înfrânată numai printr-o transformare radicală a afecțiunilor, prin înlocuirea principiilor necurate cu principii sfinte; că această transformare urmează pocăinței și credinței, este lucrarea specială a Duhului Sfânt, și constituie renașterea, sau convertirea.


\lettrine{XVIII.} That as all have violated the law of God, and cannot of themselves render obedience to his just requirements, we are dependent on Christ, first, for justification from our past offenses, and, secondly, for grace whereby to render acceptable obedience to his holy law in time to come.


\lettrine{XVIII.} Că, deoarece toți au încălcat legea lui Dumnezeu și nu pot de la sine să dea ascultare cerințelor Lui drepte, suntem dependenți de Hristos, întâi, pentru justificarea de păcatele noastre trecute, și, în al doilea rând, pentru harul prin care să dăm o ascultare acceptabilă legii Lui sfinte în viitorul apropiat.


\lettrine{XIX.} That the Spirit of God was promised to manifest itself in the church through certain gifts, enumerated especially in 1 Corinthians 12 and Ephesians 4; that these gifts are not designed to supersede, or take the place of, the Bible, which is sufficient to make us wise unto salvation, any more than the Bible can take the place of the Holy Spirit; that, in specifying the various channels of its operation, that Spirit has simply made provision for its own existence and presence with the people of God to the end of time, to lead to an understanding of that word which it had inspired, to convince of sin, and to work a transformation in the heart and life; and that those who deny to the Spirit its place and operation, do plainly deny that part of the Bible which assigns to it this work and position.


\lettrine{XIX.} Că Duhul lui Dumnezeu a fost promis să se manifeste în biserică prin anumite daruri, enumerate în special în 1 Corinteni 12 și Efeseni 4; că aceste daruri nu sunt destinate să înlocuiască, sau să ia locul Bibliei, care este suficientă pentru a ne face înțelepți pentru mântuire, nici mai mult decât Biblia poate lua locul Duhului Sfânt; că, în specificarea diferitelor canale ale operației sale, acel Duh a făcut pur și simplu prevederi pentru propria sa existență și prezență cu poporul lui Dumnezeu până la sfârșitul timpului, pentru a conduce la o înțelegere a cuvântului pe care l-a inspirat, pentru a convinge de păcat, și pentru a lucra o transformare în inima și viața; și că cei care neagă Duhului locul și operația sa, neagă în mod clar acea parte a Bibliei care îi atribuie acestei lucrări și poziție.


\lettrine{XX.} That God, in accordance with his uniform dealings with the race, sends forth a proclamation of the approach of the second advent of Christ; and that this work is symbolized by the three messages of Revelation 14, the last one bringing to view the work of reform on the law of God, that his people may acquire a complete readiness for that event.


\lettrine{XX.} Că Dumnezeu, în conformitate cu tratamentul Lui uniform cu rasa umană, trimite o proclamație a apropierii celui de-al doilea venire a lui Hristos; și că această lucrare este simbolizată de cele trei mesaje din Apocalipsa 14, cea din urmă aducând în vedere lucrarea de reformă asupra legii lui Dumnezeu, pentru ca poporul Lui să dobândească o pregătire deplină pentru acel eveniment.


\lettrine{XXI.} That the time of the cleansing of the sanctuary (See proposition X.), synchronizing with the time of the proclamation of the third message (Revelation 14:9, 10), is a time of investigative judgment, first, with reference to the dead, and secondly, at the close of probation, with reference to the living, to determine who of the myriads now sleeping in the dust of the earth are worthy of a part in the first resurrection, and who of its living multitudes are worthy of translation,—points which must be determined before the Lord appears.


\lettrine{XXI.} Că timpul curățirii sanctuarului (Vezi propoziția X.), sincronizând cu timpul proclamării celui de-al treilea mesaj (Apocalipsa 14:9, 10), este un timp de judecată investigativă, întâi, cu privire la cei morți, și în al doilea rând, la încheierea perioadei de probă, cu privire la cei vii, pentru a determina care dintre miriadele care acum dorm în praful pământului sunt vrednici de o parte în prima învierea, și care dintre multitudinile sale vii sunt vrednici de translație,—puncte care trebuie determinate înainte ca Domnul să apară.


\lettrine{XXII.} That the grave, whether we all tend, expressed by the Hebrew word sheol and the Greek word hades, is a place, or condition, in which there is no work, device, wisdom, nor knowledge. Ecclesiastes 9:10.


\lettrine{XXII.} Că mormântul, oriunde ne îndreptăm cu toții, exprimat prin cuvântul ebraic sheol și cuvântul grecesc hades, este un loc, sau o condiție, în care nu există lucrare, dispozitiv, înțelepciune, nici cunoaștere. Eclesiastul 9:10.


\lettrine{XXIII.} That the state to which we are reduced by death is one of silence, inactivity, and entire unconsciousness. Psalm 146:4; Ecclesiastes 9:5, 6; Daniel 12:2.


\lettrine{XXIII.} Că starea în care suntem reduși de moarte este una de tăcere, inactivitate și inconștiență deplină. Psalmul 146:4; Eclesiastul 9:5, 6; Daniel 12:2.


\lettrine{XXIV.} That out of this prison-house of the grave, mankind are to be brought by a bodily resurrection; the righteous having part in the first resurrection, which takes place at the second coming of Christ; the wicked, in the second resurrection, which takes place in a thousand years thereafter. Revelation 20:4-6.


\lettrine{XXIV.} Că din această închisoare a mormântului, omenirea trebuie să fie adusă printr-o învierea trupească; cei drepți având parte în prima învierea, care are loc la cea de-a doua venire a lui Hristos; cei stricați, în cea de-a doua învierea, care are loc o mie de ani mai târziu. Apocalipsa 20:4-6.


\lettrine{XXV.} That at the last trump, the living righteous are to be changed in a moment, in the twinkling of an eye, and with the risen righteous are to be caught up to meet the Lord in the air, so forever to be with the Lord. 1 Thessalonians 4:16, 17; 1 Corinthians 15:51, 52.


\lettrine{XXV.} Că la trâmbița din urmă, cei drepți vii trebuie să fie schimbați într-o clipă, într-o clipire de ochi, și cu cei drepți înviați trebuie să fie răpiți să întâmpine pe Domnul în aer, pentru a fi pentru totdeauna cu Domnul. 1 Tesaloniceni 4:16, 17; 1 Corinteni 15:51, 52.


\lettrine{XXVI.} That these immortalized ones are then taken to heaven, to the New Jerusalem, the Father’s house, in which there are many mansions (John 14:1-3), where they reign with Christ a thousand years, judging the world and fallen angels, that is, apportioning the punishment to be executed upon them at the close of the one thousand years (Revelation 20:4; 1 Corinthians 6:2, 3); that during this time the earth lies in a desolate and chaotic condition (Jeremiah 4:23-27), described, as in the beginning, by the Greek term abussos— “bottom-less pit” (Septuagint of Genesis 1:2); and that here Satan is confined during the thousand years (Revelation 20:1, 2), and here finally destroyed (Revelation 20:10; Malachi 4:1); the theater of the ruin he has wrought in the universe being appropriately made, for a time, his gloomy prison-house, and then the place of his final execution.


\lettrine{XXVI.} Că acești nemuritori sunt apoi duși la cer, la Ierusalimul Nou, casa Tatălui, în care sunt multe locuințe (Ioan 14:1-3), unde domnesc cu Hristos o mie de ani, judecând lumea și îngerii căzuți, adică repartizând pedeapsa care trebuie executată asupra lor la încheierea celor o mie de ani (Apocalipsa 20:4; 1 Corinteni 6:2, 3); că în acest timp pământul se află într-o condiție desolată și haotică (Ieremia 4:23-27), descrisă, ca la început, prin termenul grecesc abussos— „gropă fără fund” (Septuaginta din Geneza 1:2); și că aici Satan este închis în cei o mie de ani (Apocalipsa 20:1, 2), și aici în sfârșit distrus (Apocalipsa 20:10; Maleahi 4:1); teatrul ruinei pe care a cauzat-o în univers fiind în mod corespunzător făcut, pentru o vreme, închisoarea sa întunecată, și apoi locul execuției sale finale.


\lettrine{XXVII.} That at the end of the thousand years the Lord descends with his people and the New Jerusalem (Revelation 21:2), the wicked dead are raised, and come up on the surface of the yet unrenewed earth, and gather about the city, the camp of the saints (Revelation 20:9), and fire comes down from God out of heaven and devours them. They are then consumed, root and branch (Malachi 4:1), becoming as though they had not been. Obadiah 15, 16. In this everlasting destruction from the presence of the Lord (2 Thessalonians 1:9), the wicked meet the “everlasting punishment” threatened against them (Matthew 25:46), which is everlasting death. Romans 6:23; Revelation 20:14, 15. This is the perdition of ungodly men, the fire which consumes them being the fire for which “the heavens and the earth, which are now,... are kept in store.” which shall melt even the elements with its intensity, and purge the earth from the deepest stains of the curse of sin. 2 Peter 3:7-12.


\lettrine{XXVII.} Că la sfârșitul celor o mie de ani Domnul coboară cu poporul Său și Ierusalimul Nou (Apocalipsa 21:2), cei morți stricați sunt înviați, și ies pe suprafața pământului încă neînnoit, și se adună în jurul orașului, tabărei sfinților (Apocalipsa 20:9), și foc coboară de la Dumnezeu din cer și îi mistuie. Ei sunt atunci consumați, din rădăcină până la crengi (Maleahi 4:1), devenind ca și cum n-ar fi fost niciodată. Obadia 15, 16. În această distrugere veșnică din prezența Domnului (2 Tesaloniceni 1:9), cei stricați întâlnesc „pedeapsa veșnică” amenințată împotriva lor (Matei 25:46), care este moarte veșnică. Romani 6:23; Apocalipsa 20:14, 15. Aceasta este pieirea oamenilor necredincioși, focul care îi mistuie fiind focul pentru care „cerurile și pământul, care sunt acum,... sunt păstrate în zăvor.” care va topi chiar și elementele prin intensitatea sa, și va curăța pământul de cele mai adânci urme ale blestemului păcatului. 2 Petru 3:7-12.


\lettrine{XXVIII.} That new heavens and a new earth shall spring by the power of God from the ashes of the old, and this renewed earth, with the New Jerusalem for its metropolis and capital, shall be the eternal inheritance of the saints, the place where the righteous shall evermore dwell. 2 Peter 3:13; Psalm 37:11, 29; Matthew 5:5.


\lettrine{XXVIII.} Că ceruri noi și pământ nou vor răsări prin puterea lui Dumnezeu din cenușa celui vechi, și acest pământ înnoit, cu Ierusalimul Nou ca metropolă și capitală, va fi moștenirea veșnică a sfinților, locul unde cei drepți vor locui în veci. 2 Petru 3:13; Psalmul 37:11, 29; Matei 5:5.


\section*{Fundamental Principles - Timeline} \label{appendix:timeline}


\section*{Principii fundamentale - Cronologie} \label{appendix:timeline}


The following is a list of some appearances of the Declaration of Fundamental Principles in our publications. All links are accessible at \href{https://notefp.link/fp-timeline}{https://notefp.link/fp-timeline}.


Următoarea este o listă cu unele apariții ale Declarației de Principii Fundamentale în publicațiile noastre. Toate legăturile sunt accesibile la \href{https://notefp.link/fp-timeline}{https://notefp.link/fp-timeline}.


\leftsubsection{1872 - The first appearance}


\leftsubsection{1872 - Prima apariție}


\textit{“A Declaration of the Fundamental Principles Taught and Practiced by Seventh-day Adventists}” - printed as a pamphlet (\href{https://adventistdigitallibrary.org/islandora/object/adl:366607?link_only=true}{original scan} \href{https://forgotten-pillar.s3.us-east-2.amazonaws.com/A+declaration+of+the+fundamental+principles+taught+and+practiced+by+the+Seventh-day+Adventists++.pdf}{*}). They appeared anonymous, presented as a short public synopsis of what Seventh-day Adventists believe.


\textit{„O declarație a principiilor fundamentale, crezute și practicate de adventiștii de ziua a șaptea”} - tipărită ca broșură (\href{https://adventistdigitallibrary.org/islandora/object/adl:366607?link_only=true}{scan original} \href{https://forgotten-pillar.s3.us-east-2.amazonaws.com/A+declaration+of+the+fundamental+principles+taught+and+practiced+by+the+Seventh-day+Adventists++.pdf}{*}). Au apărut anonim, prezentate ca o scurtă sinteză publică a ceea ce cred adventiștii de ziua a șaptea.


\leftsubsection{1874 - The Signs of the Times}


\leftsubsection{1874 - Signs of the Times}


Original scan: \href{https://adventistdigitallibrary.org/adl-364148/signs-times-june-4-1874}{ST June 4, 1874, p.3.} \href{https://forgotten-pillar.s3.us-east-2.amazonaws.com/Signs+of+the+Times+_+June+4%2C+1874++.pdf}{*} James White stood behind the declaration as a main editor of the Signs of the Times at that time.


Scan original: \href{https://adventistdigitallibrary.org/adl-364148/signs-times-june-4-1874}{ST 4 iunie 1874, p.3.} \href{https://forgotten-pillar.s3.us-east-2.amazonaws.com/Signs+of+the+Times+_+June+4%2C+1874++.pdf}{*} James White a stat în spatele declarației ca redactor principal al Signs of the Times la acel moment.


\leftsubsection{1874 - The Advent Review and Herald of the Sabbath}


\leftsubsection{1874 - Advent Review and Herald of the Sabbath}


Original scan: \href{https://documents.adventistarchives.org/Periodicals/RH/RH18741124-V44-22.pdf}{RH November 24, 1874, p.171} \href{https://forgotten-pillar.s3.us-east-2.amazonaws.com/RH18741124-V44-22.pdf}{*} Uriah Smith signed the declaration as the main editor of the Review and Herald of the Sabbath periodical at that time.


Scan original: \href{https://documents.adventistarchives.org/Periodicals/RH/RH18741124-V44-22.pdf}{RH 24 noiembrie 1874, p.171} \href{https://forgotten-pillar.s3.us-east-2.amazonaws.com/RH18741124-V44-22.pdf}{*} Uriah Smith a semnat declarația ca redactor principal al periodicalului Review and Herald of the Sabbath la acel moment.


\leftsubsection{1874 - Part of a booklet: The Seventh-day Adventists: A Brief Sketch of Their Origin, Progress, and Principles}


\leftsubsection{1874 - Parte dintr-o broșură: Adventiștii de ziua a șaptea: O scurtă schiță a originii, progresului și principiilor lor}


Booklet was reprinted in 1876 and 1878 and later years. \\
Original scan: (\href{https://adventistdigitallibrary.org/islandora/object/adl%3A22250872?solr_nav%5Bid%5D=a09d3902c2540c98eb7f&solr_nav%5Bpage%5D=56&solr_nav%5Boffset%5D=3}{1878 copy})


Broșura a fost retipărită în 1876 și 1878 și în anii următori. \\
Scanare originală: (\href{https://adventistdigitallibrary.org/islandora/object/adl%3A22250872?solr_nav%5Bid%5D=a09d3902c2540c98eb7f&solr_nav%5Bpage%5D=56&solr_nav%5Boffset%5D=3}{copia din 1878})


\leftsubsection{1875 - The Signs of the Times}


\leftsubsection{1875 - Semnele timpurilor}


Original scan: \href{https://documents.adventistarchives.org/Periodicals/ST/ST18750128-V01-14.pdf#search=ST18750128}{ST January 28, 1875} \href{https://forgotten-pillar.s3.us-east-2.amazonaws.com/ST18750128-V01-14.pdf}{*} (p. 108, 109)


Scanare originală: \href{https://documents.adventistarchives.org/Periodicals/ST/ST18750128-V01-14.pdf#search=ST18750128}{ST 28 ianuarie 1875} \href{https://forgotten-pillar.s3.us-east-2.amazonaws.com/ST18750128-V01-14.pdf}{*} (p. 108, 109)


\leftsubsection{1878 - The Signs of the Times}


\leftsubsection{1878 - Semnele timpurilor}


Original scan: \href{https://documents.adventistarchives.org/Periodicals/ST/ST18780221-V04-08.pdf#search=%22As%20already%20stated%2C%20S%2E%20D%2E%20Adventists%22}{ST February 21, 1878} \href{https://forgotten-pillar.s3.us-east-2.amazonaws.com/ST18780221-V04-08.pdf}{*} (p. 59)


Scanare originală: \href{https://documents.adventistarchives.org/Periodicals/ST/ST18780221-V04-08.pdf#search=%22As%20already%20stated%2C%20S%2E%20D%2E%20Adventists%22}{ST 21 februarie 1878} \href{https://forgotten-pillar.s3.us-east-2.amazonaws.com/ST18780221-V04-08.pdf}{*} (p. 59)


\leftsubsection{1888 - Gospel Sickle, April 1, 1888}


\leftsubsection{1888 - Gospel Sickle, 1 aprilie 1888}


Original scan: \href{https://adventistdigitallibrary.org/adl-410336/gospel-sickle-april-1-1888?view_only=true&solr_nav%5Bid%5D=ff4d7f3f77b9bdf9e9ac&solr_nav%5Bpage%5D=0&solr_nav%5Boffset%5D=6}{Gospel Sickle, April 1, 1888}


Scanare originală: \href{https://adventistdigitallibrary.org/adl-410336/gospel-sickle-april-1-1888?view_only=true&solr_nav%5Bid%5D=ff4d7f3f77b9bdf9e9ac&solr_nav%5Bpage%5D=0&solr_nav%5Boffset%5D=6}{Gospel Sickle, 1 aprilie 1888}


\leftsubsection{1888 - The Present Truth, August 16, 1888}


\leftsubsection{1888 - The Present Truth, 16 august 1888}


Original scan: \href{https://adventistdigitallibrary.org/adl-402854/present-truth-august-16-1888?view_only=true&solr_nav%5Bid%5D=ff4d7f3f77b9bdf9e9ac&solr_nav%5Bpage%5D=0&solr_nav%5Boffset%5D=13}{PT18880816} (p. 250 - 252)


Scanare originală: \href{https://adventistdigitallibrary.org/adl-402854/present-truth-august-16-1888?view_only=true&solr_nav%5Bid%5D=ff4d7f3f77b9bdf9e9ac&solr_nav%5Bpage%5D=0&solr_nav%5Boffset%5D=13}{PT18880816} (p. 250 - 252)


\leftsubsection{1889 - SDA Yearbook for 1889}


\leftsubsection{1889 - Anuarul SDA pentru 1889}


Original scan: \href{https://documents.adventistarchives.org/Yearbooks/YB1889.pdf#search=Yearbook%201889}{YB1889} \href{https://forgotten-pillar.s3.us-east-2.amazonaws.com/YB1889.pdf}{*} (p. 145 - 151) Uriah Smith extended Fundamental Principles to 28 propositions. He added point on sanctification (point 14), dress reform (point 15) and tithing (point 16). Also he made small textual changes in some expressions, but semantics remained the same.


Scanul original: \href{https://documents.adventistarchives.org/Yearbooks/YB1889.pdf#search=Yearbook%201889}{YB1889} \href{https://forgotten-pillar.s3.us-east-2.amazonaws.com/YB1889.pdf}{*} (p. 145 - 151) Uriah Smith a extins Principiile fundamentale la 28 de propoziții. El a adăugat un punct despre sfințire (punctul 14), reforma îmbrăcămintei (punctul 15) și zecimea (punctul 16). De asemenea, a făcut mici modificări textuale în unele expresii, dar semantica a rămas aceeași.


\leftsubsection{1897 - Words of Truth - no. 5}


\leftsubsection{1897 - Cuvintele Adevărului - nr. 5}


Original scan: \href{https://adl.b2.adventistdigitallibrary.org/concern/published_works/4ffda25e-a06b-48d4-8ace-67cdcd33726f}{WoT no.5}
Word of Truth was a series of pamphlets with \href{https://adl.b2.adventistdigitallibrary.org/concern/parent/22267078_fundamental_principles_of_seventh_day_adventists/published_works/94a22141-33e8-4b9a-b397-2fe48c17bec4}{29 sections}.


Scanul original: \href{https://adl.b2.adventistdigitallibrary.org/concern/published_works/4ffda25e-a06b-48d4-8ace-67cdcd33726f}{WoT no.5}
Cuvintele Adevărului era o serie de broșuri cu \href{https://adl.b2.adventistdigitallibrary.org/concern/parent/22267078_fundamental_principles_of_seventh_day_adventists/published_works/94a22141-33e8-4b9a-b397-2fe48c17bec4}{29 de secțiuni}.


\leftsubsection{1905 - SDA Yearbook for 1905}


\leftsubsection{1905 - Anuarul SDA pentru 1905}


Original scan: \href{https://documents.adventistarchives.org/Yearbooks/YB1905.pdf#search=Yearbook%201905}{YB1905} \href{https://forgotten-pillar.s3.us-east-2.amazonaws.com/YB1905.pdf}{*} (p. 188 - 192)


Scanul original: \href{https://documents.adventistarchives.org/Yearbooks/YB1905.pdf#search=Yearbook%201905}{YB1905} \href{https://forgotten-pillar.s3.us-east-2.amazonaws.com/YB1905.pdf}{*} (p. 188 - 192)


\leftsubsection{1907 - SDA Yearbook for 1907}


\leftsubsection{1907 - Anuarul SDA pentru 1907}


Original scan: \href{https://documents.adventistarchives.org/Yearbooks/YB1907.pdf#search=Yearbook%201906}{YB1907} \href{https://forgotten-pillar.s3.us-east-2.amazonaws.com/YB1907.pdf}{*} (p. 175 - 179)


Scanul original: \href{https://documents.adventistarchives.org/Yearbooks/YB1907.pdf#search=Yearbook%201906}{YB1907} \href{https://forgotten-pillar.s3.us-east-2.amazonaws.com/YB1907.pdf}{*} (p. 175 - 179)


\leftsubsection{1908 - SDA Yearbook for 1908}


\leftsubsection{1908 - Anuarul SDA pentru 1908}


Original scan: \href{https://documents.adventistarchives.org/Yearbooks/YB1908.pdf#search=Yearbook%201906}{YB1908} \href{https://forgotten-pillar.s3.us-east-2.amazonaws.com/YB1908.pdf}{*} (p. 213 - 217)


Scanul original: \href{https://documents.adventistarchives.org/Yearbooks/YB1908.pdf#search=Yearbook%201906}{YB1908} \href{https://forgotten-pillar.s3.us-east-2.amazonaws.com/YB1908.pdf}{*} (p. 213 - 217)


\leftsubsection{1909 - SDA Yearbook for 1909}


\leftsubsection{1909 - Anuarul Biserica Adventistă de Ziua a Șaptea pentru 1909}


Original scan: \href{https://documents.adventistarchives.org/Yearbooks/YB1909.pdf#search=Yearbook%201909}{YB1909} \href{https://forgotten-pillar.s3.us-east-2.amazonaws.com/YB1909.pdf}{*} (p. 220 - 224)


Scanul original: \href{https://documents.adventistarchives.org/Yearbooks/YB1909.pdf#search=Yearbook%201909}{YB1909} \href{https://forgotten-pillar.s3.us-east-2.amazonaws.com/YB1909.pdf}{*} (p. 220 - 224)


\leftsubsection{1910 - SDA Yearbook for 1910}


\leftsubsection{1910 - Anuarul Biserica Adventistă de Ziua a Șaptea pentru 1910}


Original scan: \href{https://documents.adventistarchives.org/Yearbooks/YB1910.pdf#search=Yearbook%201910}{YB1910} \textbf{\href{https://forgotten-pillar.s3.us-east-2.amazonaws.com/YB1910.pdf}{*}} (p. 224 - 228)


Scanul original: \href{https://documents.adventistarchives.org/Yearbooks/YB1910.pdf#search=Yearbook%201910}{YB1910} \textbf{\href{https://forgotten-pillar.s3.us-east-2.amazonaws.com/YB1910.pdf}{*}} (p. 224 - 228)


\leftsubsection{1911 - SDA Yearbook for 1911}


\leftsubsection{1911 - Anuarul Biserica Adventistă de Ziua a Șaptea pentru 1911}


Original scan: \href{https://documents.adventistarchives.org/Yearbooks/YB1911.pdf#search=Yearbook%201910}{YB1911} \href{https://forgotten-pillar.s3.us-east-2.amazonaws.com/YB1911.pdf}{*} (p. 223 - 227)


Scanul original: \href{https://documents.adventistarchives.org/Yearbooks/YB1911.pdf#search=Yearbook%201910}{YB1911} \href{https://forgotten-pillar.s3.us-east-2.amazonaws.com/YB1911.pdf}{*} (p. 223 - 227)


\leftsubsection{1912 - Advent Review and Sabbath Herald, August 22, 1912}


\leftsubsection{1912 - Revista Adventistă și Heraldul Sabatului, 22 august 1912}


Original scan: \href{https://adventistdigitallibrary.org/adl-351682/advent-review-and-sabbath-herald-august-22-1912?view_only=true&solr_nav%5Bid%5D=ff4d7f3f77b9bdf9e9ac&solr_nav%5Bpage%5D=0&solr_nav%5Boffset%5D=15}{RH19120822} (p. 4 - 6)


Scanul original: \href{https://adventistdigitallibrary.org/adl-351682/advent-review-and-sabbath-herald-august-22-1912?view_only=true&solr_nav%5Bid%5D=ff4d7f3f77b9bdf9e9ac&solr_nav%5Bpage%5D=0&solr_nav%5Boffset%5D=15}{RH19120822} (p. 4 - 6)


\leftsubsection{1912 - SDA Yearbook for 1912}


\leftsubsection{1912 - Anuarul Biserica Adventistă de Ziua a Șaptea pentru 1912}


Original scan: \href{https://documents.adventistarchives.org/Yearbooks/YB1912.pdf#search=Yearbook%201910}{YB1912} \href{https://forgotten-pillar.s3.us-east-2.amazonaws.com/YB1912.pdf}{*} (p. 261 - 265)


Scanul original: \href{https://documents.adventistarchives.org/Yearbooks/YB1912.pdf#search=Yearbook%201910}{YB1912} \href{https://forgotten-pillar.s3.us-east-2.amazonaws.com/YB1912.pdf}{*} (p. 261 - 265)


\leftsubsection{1913 - SDA Yearbook for 1913}


\leftsubsection{1913 - Anuarul SDA pentru 1913}


Original scan: \href{https://documents.adventistarchives.org/Yearbooks/YB1913.pdf#search=Yearbook%201913}{YB1913} \href{https://forgotten-pillar.s3.us-east-2.amazonaws.com/YB1913.pdf}{*} (p. 281 -285 )


Scanare originală: \href{https://documents.adventistarchives.org/Yearbooks/YB1913.pdf#search=Yearbook%201913}{YB1913} \href{https://forgotten-pillar.s3.us-east-2.amazonaws.com/YB1913.pdf}{*} (p. 281 -285 )


\leftsubsection{1914 - SDA Yearbook for 1914}
Original scan: \href{https://documents.adventistarchives.org/Yearbooks/YB1914.pdf#search=Yearbook%201914}{YB1914} \href{https://forgotten-pillar.s3.us-east-2.amazonaws.com/YB1914.pdf}{*} (p. 293 - 297)


\leftsubsection{1914 - Anuarul SDA pentru 1914}
Scanare originală: \href{https://documents.adventistarchives.org/Yearbooks/YB1914.pdf#search=Yearbook%201914}{YB1914} \href{https://forgotten-pillar.s3.us-east-2.amazonaws.com/YB1914.pdf}{*} (p. 293 - 297)


\section*{Unauthenticated reports in Ellen White writings}


\section*{Rapoarte neautentificate în scrierile Ellen White}


\label{appendix:unauthenticated-reports}
We would like to present to you one Ellen White quotation that challenges the conclusion on the personality of the Holy Spirit. In this study, we have seen that the Holy Spirit is a spirit and not a being. In studying the \emcap{personality of God} and where His presence is, we have seen the distinction between the terms ‘being’ and ‘spirit’. We came to the conclusion that the Father and the Son are two distinct beings, thus constrained in space, while the Holy Spirit is a spirit, a means by which the Father and Son are everywhere present.


\label{appendix:unauthenticated-reports}
Dorim să vă prezentăm o citare din Ellen White care pune la îndoială concluzia privind personalitatea Duhului Sfânt. În acest studiu, am văzut că Duhul Sfânt este un duh și nu o ființă. În studierea \emcap{personalității lui Dumnezeu} și a locului unde Se află prezența Sa, am văzut distincția dintre termenii „ființă” și „duh”. Am ajuns la concluzia că Tatăl și Fiul sunt două ființe distincte, astfel constrânse în spațiu, în timp ce Duhul Sfânt este un duh, un mijloc prin care Tatăl și Fiul sunt pretutindeni prezenți.


The following quotation testifies that the Holy Spirit is also a being, just as the Father and Son are:


Următoarea citare mărturisește că Duhul Sfânt este de asemenea o ființă, la fel cum sunt Tatăl și Fiul:


\egw{Here is where the work of the Holy Ghost comes in, after your baptism. You are baptized in the name of \textbf{the Father, of the Son, and of the Holy Ghost}. You are raised up out of the water to live henceforth in newness of life—to live a new life. You are born unto God, and you stand under the sanction and \textbf{the power of the three holiest \underline{beings} in heaven}, who are able to keep you from falling.}[Ms95-1906.29; 1906][https://egwwritings.org/read?panels=p8872.35]


\egw{Aici intervine lucrarea Duhului Sfânt, după botezul tău. Ești botezat în numele \textbf{Tatălui, al Fiului și al Duhului Sfânt}. Ești ridicat din apă pentru a trăi de acum înainte în noutate de viață—pentru a trăi o viață nouă. Ești născut pentru Dumnezeu, și stai sub sancțiunea și \textbf{puterea celor trei \underline{ființe} cele mai sfinte din cer}, care sunt în stare să te păzească de cădere.}[Ms95-1906.29; 1906][https://egwwritings.org/read?panels=p8872.35]


Many have come across this quotation and presented it as proof that the Holy Spirit is a being rather than a spirit. In the following, we present our concerns.


Mulți au întâlnit această citare și au prezentat-o ca dovadă că Duhul Sfânt este o ființă mai degrabă decât un duh. În continuare, vă prezentăm preocupările noastre.


The source of this quotation is Manuscript 95, 1906.


Sursa acestei citări este Manuscrisul 95, 1906.


This quotation is actually a report from the sermon Sister White held in Oakland, California, on Sabbath afternoon, October 20, 1906. Many of Ellen White’s public sermons were stenographically reported and later rewritten for publication. When Sister White preached, she never had a written sermon. There were no tape recorders at that time that could accurately document word for word. The only reference we have from that time is the report by the stenographer. This opens the possibility for human error in reporting, or later editing, prior to publication. The plethora of evidence presented in this book makes it clear that this statement is not in harmony with the authenticated quotations. Plainly stated, it’s obvious that a mistake was made in the report of this sermon.


Această citare este de fapt un raport din predica pe care Sora White a ținut-o în Oakland, California, sâmbătă după-amiază, 20 octombrie 1906. Multe dintre predicile publice ale Ellen White au fost raportate stenografic și mai târziu rescrise pentru publicare. Când Sora White predica, nu avea niciodată o predică scrisă. Nu existau magnetofoane în acea vreme care să poată documenta cu exactitate cuvânt cu cuvânt. Singura referință pe care o avem din acea perioadă este raportul stenografului. Aceasta deschide posibilitatea erorii umane în raportare sau editare ulterioară, înainte de publicare. Abundența de dovezi prezentate în această carte face clar că această afirmație nu este în armonie cu citatele autentificate. Spus clar, este evident că a fost făcută o greșeală în raportul acestei predici.


In order to clear any such mistakes for the future generations, Sister White actually warns us when it comes to unauthenticated reports of what she may have said.


Pentru a clarifica orice greșeli de acest fel pentru generațiile viitoare, Sora White ne avertizează de fapt atunci când vine vorba de rapoarte neautentificate despre ceea ce ar fi putut spune.


\egw{And now to all who have a desire for truth I would say: \textbf{Do not give credence to \underline{unauthenticated reports} as to what Sister White has done or said or written}. If you desire to know what the Lord has revealed through her, \textbf{read her published works}. Are there any points of interest concerning which she has not written, do not eagerly catch up and report rumors as to what she has said.}[5T 696.1; 1889][https://egwwritings.org/read?panels=p113.3386]


\egw{Și acum tuturor celor care au dorința adevărului le-aș spune: \textbf{Nu dați crezare \underline{rapoartelor neautentificate} cu privire la ceea ce a făcut, a spus sau a scris Sora White}. Dacă doriți să știți ce a revelat Domnul prin ea, \textbf{citiți lucrările ei publicate}. Dacă există vreun punct de interes cu privire la care nu a scris, nu prinde cu ușurință și nu raporta zvonuri despre ceea ce a spus.}[5T 696.1; 1889][https://egwwritings.org/read?panels=p113.3386]


The published works of Ellen White during her life represent the accurate and authentic material from Sister White. The process of publication ensured that the final product was genuine. The weight of the evidence is that Sister White herself was involved in the process of the publishing and she would review manuscripts prior to printing.


Lucrările publicate ale Ellen White în timpul vieții ei reprezintă materialul exact și autentic de la Sora White. Procesul de publicare a asigurat că produsul final era autentic. Greutatea dovezilor este că Sora White însăși a fost implicată în procesul de publicare și ar fi revizuit manuscrisele înainte de tipărire.


\egw{I read over all that is copied, to see that everything is as it should be. I read all the book manuscript before it is sent to the printer.}[Lt133-1902.4; 1902][https://egwwritings.org/read?panels=p9791.10]


\egw{Citesc tot ceea ce se copiază, pentru a vedea că totul este cum trebuie. Citesc tot manuscrisul cărții înainte ca acesta să fie trimis tipografului.}[Lt133-1902.4; 1902][https://egwwritings.org/read?panels=p9791.10]


\egw{I have all my publications closely examined. I desire that nothing shall appear in print without careful investigation.}[Lt49-1894.11; 1894][https://egwwritings.org/read?panels=p5289.20]


\egw{Toate publicațiile mele sunt examinate cu atenție. Doresc ca nimic să nu apară în tipar fără investigație atentă.}[Lt49-1894.11; 1894][https://egwwritings.org/read?panels=p5289.20]


The statement that the Holy Spirit is a being was not part of the process of publishing because this statement appeared after the death of Ellen White. Thus, it is not authenticated. It does not belong to her “\textit{published work}”. We do not seek any conspiracy in this; we’re simply adhering to Ellen White’s own suggestion to not give credence to these reports. In 1990, Ellen White Estate published the collection of her sermons and talks and in 2015, they included the sermons and talks into the files of her Manuscripts. We do not understand why they did that since the sermons and talks do not contain manuscripts from Ellen White, but from some stenographers. Nevertheless, above every manuscript the EGW Estate annotated its source, whether a sermon or letter. This tells us if the quotation is authenticated or not.


Afirmația că Duhul Sfânt este o ființă nu a făcut parte din procesul de publicare deoarece această afirmație a apărut după moartea Ellen White. Prin urmare, nu este autentificată. Nu aparține „\textit{lucrării ei publicate}”. Nu căutăm vreo conspirație în aceasta; pur și simplu aderăm la propria sugestie a Ellen White de a nu da crezare acestor rapoarte. În 1990, Moștenirea Ellen White a publicat colecția predicilor și discursurilor ei, iar în 2015, au inclus predicile și discursurile în dosarele manuscriselor ei. Nu înțelegem de ce au făcut asta, deoarece predicile și discursurile nu conțin manuscrise de la Ellen White, ci de la niște stenografi. Cu toate acestea, deasupra fiecărui manuscris, Moștenirea EGW a anotat sursa acestuia, fie o predică, fie o scrisoare. Aceasta ne spune dacă citatul este autentificat sau nu.


\begin{figure}
    \centering
    \includegraphics[width=1\linewidth]{images/sermons-and-talks.png}
    \label{fig:enter-label}
\end{figure}


\begin{figure}
    \centering
    \includegraphics[width=1\linewidth]{images/sermons-and-talks.png}
    \label{fig:enter-label}
\end{figure}


For us, personally, these quotations are unauthenticated and, especially, invalid compared to Ellen White’s authenticated works. But if someone insists on weighing her unconfirmed reports and published writings equally, we will not stand in their way but even further push the conclusion of the Holy Spirit as a being. Let’s follow together.


Pentru noi, personal, aceste citate sunt neautentificate și, mai ales, nevalide în comparație cu lucrările autentificate ale Ellen White. Dar dacă cineva insistă să cântărească rapoartele ei neconfirmate și scrierile publicate în mod egal, nu vom sta în calea lor, dar vom împinge și mai departe concluzia că Duhul Sfânt este o ființă. Să mergem împreună.


Even compared with Ellen White’s authenticated works, such a Holy Spirit, a being, would not be one with God because Christ was \egwinline{\textbf{The only being who was one with God}}[Lt121-1897.7; 1897][https://egwwritings.org/read?panels=p7266.13]. This Holy Spirit, a being, could not \egwinline{\textbf{enter into all the counsels and purposes of God}}, because Christ was \egwinline{\textbf{the only being}}[PP 34.1; 1890][https://egwwritings.org/read?panels=p84.75] who could do that. This Being is not to be exalted because \egwinline{\textbf{The Father and the Son \underline{alone} are to be exalted}}[YI, July 7, 1898 par.2.; 1898][https://egwwritings.org/read?panels=p469.2964]. The Holy Spirit, as a being, would not fit in the order of heaven as the third being because Satan was \egwinline{\textbf{next to Christ the most exalted \underline{being}} in the heavenly courts}[RH August 9, 1898, par. 7; 1898][https://egwwritings.org/read?panels=p821.17145]. This Holy Spirit, a being, was not invested in the cost of salvation; neither was he in the covenant with Father and Son to save the world, nor dishonored by man’s transgression.


Chiar și în comparație cu lucrările autentificate ale Ellen White, un asemenea Duh Sfânt, o ființă, nu ar fi unu cu Dumnezeu, deoarece Hristos era \egwinline{\textbf{Singura ființă care era una cu Dumnezeu}}[Lt121-1897.7; 1897][https://egwwritings.org/read?panels=p7266.13]. Acest Duh Sfânt, o ființă, nu ar putea \egwinline{\textbf{intra în toate sfaturile și scopurile lui Dumnezeu}}, deoarece Hristos era \egwinline{\textbf{singura ființă}}[PP 34.1; 1890][https://egwwritings.org/read?panels=p84.75] care ar putea face asta. Această Ființă nu trebuie să fie înălțată, deoarece \egwinline{\textbf{Tatăl și Fiul \underline{singuri} trebuie să fie înălțați}}[YI, July 7, 1898 par.2.; 1898][https://egwwritings.org/read?panels=p469.2964]. Duhul Sfânt, ca o ființă, nu s-ar potrivi în ordinea cerului ca a treia ființă, deoarece Satan era \egwinline{\textbf{lângă Hristos cea mai înălțată \underline{ființă}} în curțile cerești}[RH August 9, 1898, par. 7; 1898][https://egwwritings.org/read?panels=p821.17145]. Acest Duh Sfânt, o ființă, nu a fost investit în costul mântuirii; nici nu a fost în legământul cu Tatăl și Fiul pentru a salva lumea, nici nu a fost dezonorat de încălcarea omului.


\egwinline{The great gift of salvation has been placed within our reach at an \textbf{infinite cost to the Father and the Son}.}[RH November 21, 1912, par. 2; 1912][https://egwwritings.org/read?panels=p821.33329]


\egwinline{Marele dar al mântuirii a fost pus la îndemâna noastră la un \textbf{cost infinit pentru Tatăl și Fiul}.}[RH November 21, 1912, par. 2; 1912][https://egwwritings.org/read?panels=p821.33329]


\egwinline{In the plan to save a lost world, the counsel was between them \textbf{\underline{both}}; \textbf{the covenant of peace was between the Father and the Son}.}[ST December 23, 1897, par. 2; 1897][https://egwwritings.org/read?panels=p820.14803]


\egwinline{În planul de a salva o lume pierdută, sfatul a fost între ei \textbf{\underline{amândoi}}; \textbf{legământul păcii a fost între Tatăl și Fiul}.}[ST December 23, 1897, par. 2; 1897][https://egwwritings.org/read?panels=p820.14803]


\egwinline{But in the transgression of man \textbf{\underline{both} the Father and the Son were dishonored}.}[ST December 12, 1895, par. 7; 1895][https://egwwritings.org/read?panels=p820.13243]


\egwinline{Dar în călcarea legii a omului \textbf{\underline{amândoi} Tatăl și Fiul au fost desonorați}.}[ST December 12, 1895, par. 7; 1895][https://egwwritings.org/read?panels=p820.13243]


Such a Holy Spirit, a being, does not fit into harmony with the authenticated reports of Ellen White, nor with the Scriptures. The Holy Spirit is called ‘\textit{spirit}’, so it is a spirit, exclusively.


Un Duh Sfânt atât de sfânt, o ființă, nu se potrivește în armonie cu rapoartele autentificate ale Ellen White, nici cu Scripturile. Duhul Sfânt este numit ‘\textit{duh}’, deci este un duh, exclusiv.


Many of Sister White’s quotations are sourced from sermons or talks that were published after her death. In what follows, we will present a few that are most often discussed in an effort to prove that Sister White was a trinitarian. We invite everyone to weigh these quotations with her authenticated and published work, those during her lifetime.


Multe din citările Surorii White sunt preluate din predici sau discursuri care au fost publicate după moartea ei. În ceea ce urmează, vom prezenta câteva dintre cele mai des discutate în încercarea de a dovedi că Sora White era trinitară. Invităm pe toți să cântărească aceste citate cu lucrarea ei autentificată și publicată, cea din timpul vieții ei.


“\textit{And then the golden harps are touched, and the music flows all through the heavenly host, and they fall down and worship the Father and the Son and the Holy Spirit}.”\footnote{\href{https://egwwritings.org/?ref=en_Ms139-1906.32&para=9579.38}{EGW; Ms139-1906.32; 1906}} [Sermon/Thoughts on Matthew 4. Oakland, California July 24, 1906; Previously unpublished.]


“\textit{Și atunci harpele de aur sunt atinse, și muzica curge prin toată oastea cerească, și ei se prosternează și se închină Tatălui și Fiului și Duhului Sfânt}.”\footnote{\href{https://egwwritings.org/?ref=en_Ms139-1906.32&para=9579.38}{EGW; Ms139-1906.32; 1906}} [Sermon/Thoughts on Matthew 4. Oakland, California July 24, 1906; Previously unpublished.]


“\textit{We need to realize that the Holy Spirit, who is as much a person as God is a person, is walking through these grounds.}”\footnote{\href{https://egwwritings.org/?ref=en_Ms66-1899.11&para=6622.19}{EGW; Ms66-1899.11: 1899}} [Talk/Extracts From Talks Given by Mrs. E. G. White at the Opening of College Hall, Avondale, and in the Avondale Church]


“\textit{Trebuie să realizăm că Duhul Sfânt, care este la fel de mult o persoană cum este Dumnezeu o persoană, se plimbă prin aceste locuri.}”\footnote{\href{https://egwwritings.org/?ref=en_Ms66-1899.11&para=6622.19}{EGW; Ms66-1899.11: 1899}} [Talk/Extracts From Talks Given by Mrs. E. G. White at the Opening of College Hall, Avondale, and in the Avondale Church]
