% \qrchapter{https://forgottenpillar.com/rsc/en-fp-chapter3}{The historical context}


\qrchapter{https://forgottenpillar.com/rsc/ro-fp-chapter3}{Contextul istoric}


Ellen White recalled encountering the same sentiments in \textit{The Living Temple} that she had warned against early in her ministry:


Ellen White și-a amintit că a întâlnit aceleași opinii în \textit{Templul viu} împotriva cărora avertizase la începutul slujirii sale:


\egw{As we read \normaltext{[The Living Temple]}, I recognized the very sentiments against which I had been bidden to speak in warning \textbf{during the early days of my public labors}. \textbf{When I first left \underline{the State of Maine, it was to go through Vermont and Massachusetts}}, to bear a testimony \textbf{against these sentiments}. ‘Living Temple’ contains the alpha of these theories. I knew that the omega would follow in a little while; and I trembled for our people. I knew that I must warn our brethren and sisters \textbf{not to enter into controversy over \underline{the presence and personality of God}}. The statements made in ‘Living Temple’ in regard to this point are incorrect. The scripture used to substantiate the doctrine there set forth, is scripture misapplied.}[SpTB02 53.2; 1904][https://egwwritings.org/read?panels=p417.271]


\egw{Pe măsură ce citeam \normaltext{[Templul viu]}, am recunoscut chiar acele opinii împotriva cărora fusesem îndrumată să vorbesc ca avertizare \textbf{în primele zile ale lucrării mele publice}. \textbf{Când am părăsit pentru prima dată \underline{statul Maine, a fost să merg prin Vermont și Massachusetts}}, să duc o mărturie \textbf{împotriva acestor opinii}. „Templul viu” conține alfa acestor teorii. Știam că omega va urma în scurt timp; și am tremurat pentru poporul nostru. Știam că trebuie să-mi avertizez frații și surorile \textbf{să nu intre în controversă asupra \underline{prezenței și personalității lui Dumnezeu}}. Afirmațiile făcute în „Templul viu” cu privire la acest punct sunt incorecte. Scriptura folosită pentru a susține doctrina expusă acolo este Scriptură aplicată greșit.}[SpTB02 53.2; 1904][https://egwwritings.org/read?panels=p417.271]


She pinpointed her first encounter with these views: \egwinline{When I first left \textbf{the State of Maine}, it was to go through Vermont and Massachusetts, \textbf{to bear a testimony against these sentiments.}} Her biography, written by her grandson Arthur Lacey White, provides further context on these sentiments. In \textit{Ellen White: The Early Years}, under the section \textit{Wrestling with the Views of the Spiritualizers}, her experiences in eastern Maine reveal more about the controversy over the personality of God and its implications.


Ea a identificat prima sa întâlnire cu aceste perspective: \egwinline{Când am părăsit pentru prima dată \textbf{statul Maine}, a fost să merg prin Vermont și Massachusetts, \textbf{să duc o mărturie împotriva acestor opinii.}} Biografia ei, scrisă de nepotul ei Arthur Lacey White, oferă context suplimentar despre aceste opinii. În \textit{Ellen White: Primii ani}, sub secțiunea \textit{Luptând cu perspectivele celor care spiritualizează}, experiențele ei din estul statului Maine dezvăluie mai multe despre controversa privind personalitatea lui Dumnezeu și implicațiile sale.


\othersQuote{\textbf{\underline{In eastern Maine} Ellen was traveling} and working \textbf{in the atmosphere of the spiritualizers who had \underline{allegorized away heaven, God, Jesus, and the Advent hope}}. In the vision at Exeter in mid-February she seemed to be \textbf{in the presence of Jesus, and she was eager to procure answers to some \underline{vital questions}}.}[ALW, 1BIO 79.4; 1985][https://egwwritings.org/read?panels=p668.582]


\othersQuote{\textbf{\underline{În estul statului Maine} Ellen călătorea} și lucra \textbf{în atmosfera celor care spiritualizează care \underline{alegorizaseră cerul, pe Dumnezeu, pe Isus și speranța Adventului}}. În viziunea de la Exeter de la mijlocul lui februarie, ea părea să fie \textbf{în prezența lui Isus și era dornică să obțină răspunsuri la câteva \underline{întrebări vitale}}.}[ALW, 1BIO 79.4; 1985][https://egwwritings.org/read?panels=p668.582]


\othersQuoteNoGap{I asked Jesus if \textbf{His Father had a form like Himself}. \textbf{He said He had}, but I could not behold it, for said He, ‘If you should once behold the glory of \textbf{His person}, you would cease to exist.’—Early Writings, 54.}[ALW, 1BIO 79.5; 1985][https://egwwritings.org/read?panels=p668.583]


\othersQuoteNoGap{L-am întrebat pe Isus dacă \textbf{Tatăl Său avea o formă ca a Sa}. \textbf{El a spus că avea}, dar nu o puteam privi, căci a spus El: „Dacă ai privi o dată slava \textbf{persoanei Lui}, ai înceta să exiști.”—Scrieri timpurii, 54.}[ALW, 1BIO 79.5; 1985][https://egwwritings.org/read?panels=p668.583]


\othersQuoteNoGap{This was not the only occasion Ellen was to converse with Jesus and the angel \textbf{about the \underline{person of Jesus} and concerning \underline{God being a personal being}}. \textbf{\underline{The answers satisfied her fully that the spiritualizers were in gross error}}.}[ALW, 1BIO 80.1; 1985][https://egwwritings.org/read?panels=p668.586]


\othersQuoteNoGap{Aceasta nu a fost singura ocazie când Ellen a conversat cu Isus și cu îngerul \textbf{despre \underline{persoana lui Isus} și despre \underline{Dumnezeu ca ființă personală}}. \textbf{\underline{Răspunsurile au satisfăcut-o pe deplin că cei care spiritualizează erau într-o eroare gravă}}.}[ALW, 1BIO 80.1; 1985][https://egwwritings.org/read?panels=p668.586]


The vision Arthur Lacey White referred to is known as the \textit{vision on the personality of God}, which we will examine later. This vision confirms that the doctrine of the \emcap{personality of God} teaches that God the Father has a form, just as Jesus does. It specifically addresses the \others{\textbf{person of Jesus} and concerning \textbf{God being a personal being}.}


Viziunea la care s-a referit Arthur Lacey White este cunoscută ca \textit{viziunea despre personalitatea lui Dumnezeu}, pe care o vom examina mai târziu. Această viziune confirmă că doctrina \emcap{personalității lui Dumnezeu} învață că Dumnezeu Tatăl are o formă, la fel ca Isus. Ea abordează în mod specific \others{\textbf{persoana lui Isus} și \textbf{Dumnezeu ca ființă personală}.}


\begin{figure}[t]
    \centering
    \includegraphics[width=0.65\linewidth]{images/ellen-white.jpg}
    \caption*{Ellen G. White}
    \label{fig:ellen-g-white}
\end{figure}


\begin{figure}[t]
    \centering
    \includegraphics[width=0.65\linewidth]{images/ellen-white.jpg}
    \caption*{Ellen G. White}
    \label{fig:ellen-g-white}
\end{figure}


Consider the first point of the \emcap{Fundamental Principles}, which states that Seventh-day Adventists believe in \others{one God, \textbf{a personal, spiritual being}.}[First point of the Fundamental Principles][https://forgotten-pillar.s3.us-east-2.amazonaws.com/A+declaration+of+the+fundamental+principles+taught+and+practiced+by+the+Seventh-day+Adventists++.pdf] This makes it clear that the central issue in the doctrine of the \emcap{personality of God} concerns the outward, bodily form of the Father. But why was this such a vital and significant question? What were the implications of God having a bodily, personal form?


Luați în considerare primul punct al \emcap{Principiilor fundamentale}, care afirmă că adventiștii de ziua a șaptea cred în \others{un singur Dumnezeu, \textbf{o ființă personală, spirituală}.}[Primul punct al Principiilor fundamentale][https://forgotten-pillar.s3.us-east-2.amazonaws.com/A+declaration+of+the+fundamental+principles+taught+and+practiced+by+the+Seventh-day+Adventists++.pdf] Acest lucru clarifică faptul că problema centrală în doctrina \emcap{personalității lui Dumnezeu} privește forma exterioară, corporală a Tatălui. Dar de ce era aceasta o întrebare atât de vitală și semnificativă? Care erau implicațiile faptului că Dumnezeu are o formă corporală, personală?


\othersQuote{But because the pioneers of the Seventh-day Adventist Church held that prophecy was fulfilled on October 22, 1844, and that an important work began in heaven in the Most Holy Place of the heavenly sanctuary at that time, and because the Adventists who had become \textbf{spiritualizers} took the position that Christ had come into their hearts on October 22, 1844, and that His kingdom was in their hearts, the founders of the church, and notably Ellen White, were classed by the world generally, and also by those that SDAs have termed first-day Adventists, as one and the same group. Here again the great enemy cast aspersion upon the true, paralleling it with a false, spurious experience.}[ALW, 1BIO 80.2; 1985][https://egwwritings.org/read?panels=p668.587]


\othersQuote{Dar pentru că pionierii Bisericii Adventiste de Ziua a Șaptea susțineau că profeția s-a împlinit pe 22 octombrie 1844 și că o lucrare importantă a început în cer în Locul Preasfânt al sanctuarului ceresc în acel moment, și pentru că adventiștii care deveniseră \textbf{cei care spiritualizează} adoptaseră poziția că Hristos venise în inimile lor pe 22 octombrie 1844 și că împărăția Lui era în inimile lor, fondatorii bisericii, și în special Ellen White, au fost clasificați de lume în general, și de asemenea de cei pe care AZȘ i-au numit adventiști de ziua întâi, ca fiind unul și același grup. Aici din nou marele vrăjmaș a aruncat calomnie asupra adevărului, punându-l în paralel cu o experiență falsă, neautentică.}[ALW, 1BIO 80.2; 1985][https://egwwritings.org/read?panels=p668.587]


\othersQuoteNoGap{Ellen White was to speak of this matter again, particularly in the closing paragraphs of her first little book, Experience and Views, published in 1851. As one reads this he will note the use of \textbf{the term spiritualism}, which must be taken in the light of the work of the spiritualizers and not in the light of what today is understood to be spiritualism or spiritism, although both emanate from the same source.}[ALW, 1BIO 80.3; 1985][https://egwwritings.org/read?panels=p668.588]


\othersQuoteNoGap{Ellen White urma să vorbească din nou despre această chestiune, în special în paragrafele de încheiere ale primei sale cărțulii, Experience and Views, publicată în 1851. Pe măsură ce cineva citește aceasta, va observa folosirea \textbf{termenului spiritualism}, care trebuie înțeles în lumina lucrării celor care spiritualizează și nu în lumina a ceea ce astăzi este înțeles a fi spiritualism sau spiritism, deși ambele emană din aceeași sursă.}[ALW, 1BIO 80.3; 1985][https://egwwritings.org/read?panels=p668.588]


\othersQuoteNoGap{We turn now to the statement written and published in 1851 as found in Ibid., 77, 78:}[ALW, 1BIO 80.4; 1985][https://egwwritings.org/read?panels=p668.589]


\othersQuoteNoGap{Ne întoarcem acum la declarația scrisă și publicată în 1851, așa cum se găsește în Ibid., 77, 78:}[ALW, 1BIO 80.4; 1985][https://egwwritings.org/read?panels=p668.589]


\othersQuoteNoGap{\textbf{I have frequently been falsely charged with teaching views peculiar to Spiritualism}. But before the editor of The Day-Star ran into that delusion, \textbf{the Lord \underline{gave me a view} of the sad and desolating effects that would be produced upon the flock by him and others \underline{in teaching the spiritual views}}.}[ALW, 1BIO 80.5; 1985][https://egwwritings.org/read?panels=p668.590]


\othersQuoteNoGap{\textbf{Am fost frecvent acuzată în mod fals că predau învățături specifice spiritualismului}. Dar înainte ca editorul The Day-Star să cadă în acea amăgire, \textbf{Domnul \underline{mi-a dat o viziune} despre efectele triste și devastatoare care urmau să fie produse asupra turmei de către el și alții \underline{prin predarea perspectivei spiritualiste}}.}[ALW, 1BIO 80.5; 1985][https://egwwritings.org/read?panels=p668.590]


\othersQuoteNoGap{I have often seen the lovely \textbf{Jesus, that He is a person}. I asked Him \textbf{\underline{if His Father was a person} and \underline{had a form} like Himself}. Said Jesus, ‘I am in \textbf{the express image of My Father’s person}.}[ALW, 1BIO 80.6; 1985][https://egwwritings.org/read?panels=p668.591]


\othersQuoteNoGap{L-am văzut adesea pe iubitul \textbf{Isus, că El este o persoană}. L-am întrebat \textbf{\underline{dacă Tatăl Său era o persoană} și \underline{avea o formă} ca a Lui}. Isus a spus: ‘Eu sunt \textbf{întipărirea persoanei Tatălui Meu}.}[ALW, 1BIO 80.6; 1985][https://egwwritings.org/read?panels=p668.591]


\othersQuoteNoGap{\textbf{I have often seen that \underline{the spiritual view} took away all the glory of heaven, and that in many minds the throne of David and the lovely person of Jesus have been burned up in the fire of Spiritualism.} I have seen that some who have been deceived and led into this error will be brought out into the light of truth, but it will be almost impossible for them to get entirely rid of \textbf{the deceptive power of Spiritualism}. Such should make thorough work in confessing their errors and leaving them forever.}[ALW, 1BIO 80.7; 1985][https://egwwritings.org/read?panels=p668.592]


\othersQuoteNoGap{\textbf{Am văzut adesea că \underline{perspectiva spiritualistă} a îndepărtat toată slava cerului și că în multe minți tronul lui David și persoana iubită a lui Isus au fost arse în focul spiritualismului.} Am văzut că unii care au fost înșelați și conduși în această eroare vor fi aduși afară în lumina adevărului, dar va fi aproape imposibil pentru ei să scape complet de \textbf{puterea înșelătoare a spiritualismului}. Astfel de persoane ar trebui să facă o lucrare temeinică în mărturisirea erorilor lor și părăsirea lor pentru totdeauna.}[ALW, 1BIO 80.7; 1985][https://egwwritings.org/read?panels=p668.592]


\othersQuoteNoGap{\textbf{The spiritualization of heaven, God, Christ, and the coming of Christ lay at the foundation of much of the fanatical teachings that 17-year-old Ellen Harmon was called upon by God to meet in those formative days. The visions firmly established \underline{the personality of God and Christ}, \underline{the reality of heaven} and the reward to the faithful, and the resurrection. This sound guidance saved the emerging church}.}[ALW, 1BIO 81.1; 1985][https://egwwritings.org/read?panels=p668.595]


\othersQuoteNoGap{\textbf{Spiritualizarea cerului, a lui Dumnezeu, a lui Hristos și a venirii lui Hristos stătea la temelia multora dintre învățăturile fanatice pe care Ellen Harmon, în vârstă de 17 ani, a fost chemată de Dumnezeu să le înfrunte în acele zile formative. Viziunile au stabilit ferm \underline{personalitatea lui Dumnezeu și a lui Hristos}, \underline{realitatea cerului} și răsplata pentru cei credincioși, și învierea. Această îndrumare solidă a salvat biserica emergentă}.}[ALW, 1BIO 81.1; 1985][https://egwwritings.org/read?panels=p668.595]


The mistake of the Millerite movement in 1844 lay in misunderstanding the nature of the event, not its timing. Daniel 7:13-14 describes Christ coming to the Ancient of Days in heaven to receive dominion, glory, and a kingdom—not His second coming to earth. This event, marking the beginning of Christ’s work in the Most Holy Place, occurred at the conclusion of the 2300-day prophecy in 1844. Unlike other Adventist groups, the emerging Seventh-day Adventist Church uniquely recognized this heavenly event.


Greșeala mișcării millerite din 1844 a constat în neînțelegerea naturii evenimentului, nu a momentului său. Daniel 7:13-14 descrie pe Hristos venind la Cel Îmbătrânit de zile în cer pentru a primi stăpânire, slavă și o împărăție—nu a doua Sa venire pe pământ. Acest eveniment, marcând începutul lucrării lui Hristos în Locul Preasfânt, a avut loc la încheierea profeției de 2300 de zile în 1844. Spre deosebire de alte grupuri adventiste, Biserica Adventistă de Ziua a Șaptea emergentă a recunoscut în mod unic acest eveniment ceresc.


This understanding is built on key premises:
\begin{itemize}
    \item Heaven is a real, literal place (John 14:1-3).
    \item There is a literal sanctuary in heaven where Christ ministers (Hebrews 8:2). 
    \item A real, physical throne exists in this sanctuary, occupied by God Himself (Daniel 7:9-10; Revelation 4:2-3; Ezekiel 1:26-28; Psalm 11:4).
\end{itemize}


Această înțelegere este construită pe premise cheie:
\begin{itemize}
    \item Cerul este un loc real, literal (Ioan 14:1-3).
    \item Există un sanctuar literal în cer unde Hristos slujește (Evrei 8:2).
    \item Un tron real, fizic există în acest sanctuar, ocupat de Dumnezeu Însuși (Daniel 7:9-10; Apocalipsa 4:2-3; Ezechiel 1:26-28; Psalmul 11:4).
\end{itemize}


Why is the question of the Father’s bodily form so important? If God were not a physical being, there would be no need for a literal throne, sanctuary, or heavenly ministry. A spiritualized interpretation undermines the foundation of Seventh-day Adventist theology, leading to a domino effect that erodes the doctrine of Christ’s priestly work.


De ce este atât de importantă întrebarea despre forma corporală a Tatălui? Dacă Dumnezeu nu ar fi o ființă fizică, nu ar fi nevoie de un tron literal, sanctuar sau slujire cerească. O interpretare spiritualizată subminează temelia teologiei adventiste de ziua a șaptea, ducând la un efect de domino care erodează doctrina lucrării preoțești a lui Hristos.


The doctrine of the \emcap{personality of God} was a simple yet foundational teaching, affirmed in the first point of the \emcap{Fundamental Principles}: \textit{“One God, a personal, spiritual being.”} As such, He is not omnipresent by Himself but through His representative, the Holy Spirit.\footnote{The first point of the Fundamental Principles: \othersQuote{That there is \textbf{one God}, \textbf{a \underline{personal, spiritual being}}, the creator of all things, omnipotent, … and \textbf{everywhere present by \underline{his representative}, the Holy Spirit}. Ps. 139:7.}} When Ellen White asked Jesus \egwinline{if His Father \textbf{was a person} and \textbf{had a \underline{form}} like Himself,}[EW 77.1; 1882][https://egwwritings.org/read?panels=p28.490&index=0] we see clearly that the \textit{outward bodily \textbf{form}} is \textit{the quality or state} defining God as a person. This understanding was central in addressing the Kellogg crisis regarding \textit{The Living Temple}, which deviated from this core belief.


Doctrina \emcap{personalității lui Dumnezeu} era o învățătură simplă, dar fundamentală, afirmată în primul punct al \emcap{Principiilor Fundamentale}: \textit{“Un singur Dumnezeu, o ființă personală, spirituală.”} Ca atare, El nu este omniprezent prin Sine Însuși, ci prin reprezentantul Său, Duhul Sfânt.\footnote{Primul punct al Principiilor Fundamentale: \othersQuote{Că există \textbf{un singur Dumnezeu}, \textbf{o \underline{ființă personală, spirituală}}, creatorul tuturor lucrurilor, atotputernic, … și \textbf{prezent pretutindeni prin \underline{reprezentantul Său}, Duhul Sfânt}. Ps. 139:7.}} Când Ellen White L-a întrebat pe Isus \egwinline{dacă Tatăl Său \textbf{era o persoană} și \textbf{avea o \underline{formă}} ca a Sa,}[EW 77.1; 1882][https://egwwritings.org/read?panels=p28.490&index=0] vedem clar că \textit{\textbf{forma} trupească exterioară} este \textit{caracteristica sau starea} care Îl definește pe Dumnezeu ca persoană. Această înțelegere a fost centrală în abordarea crizei Kellogg privind \textit{Templul viu}, care s-a abătut de la această credință de bază.


But do our current \textit{Fundamental Beliefs} still affirm this doctrine? Do they explicitly teach that God is a real person with a bodily form, whose literal presence is in heaven, while He is omnipresent through His Spirit? The doctrine of God’s presence and personality is absent from today’s official beliefs. While individually, we may still believe in it, why was such a vital teaching omitted? What were the reasons behind this shift? These are the questions we must explore further in the context of \textit{The Foundation of Our Faith}.


Dar afirmă \textit{Punctele noastre Fundamentale de Credință} actuale încă această doctrină? Învață ele în mod explicit că Dumnezeu este o persoană reală cu o formă trupească, a cărui prezență literală este în cer, în timp ce El este omniprezent prin Duhul Său? Doctrina prezenței și personalității lui Dumnezeu lipsește din credințele oficiale de astăzi. Deși individual, poate încă credem în ea, de ce a fost omisă o învățătură atât de vitală? Care au fost motivele din spatele acestei schimbări? Acestea sunt întrebările pe care trebuie să le explorăm mai departe în contextul \textit{Temeliei credinței noastre}.


% The Historical Context

\begin{titledpoem}
    \stanza{
        In visions clear, Ellen White stood firm, \\
        Against false views she did affirm. \\
        The Father's form, a truth profound, \\
        In this essential faith was found.
    }

    \stanza{
        The "spiritualizers" sought to claim \\
        That heaven's realm was but a name. \\
        Yet God has form, like Christ His Son, \\
        This truth our founders built upon.
    }

    \stanza{
        A personal God on throne divine, \\
        Not merely spirit, by design. \\
        This doctrine once our cornerstone, \\
        Has from our statements somehow flown.
    }
\end{titledpoem}