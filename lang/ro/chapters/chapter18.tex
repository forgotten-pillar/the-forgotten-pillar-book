% \qrchapter{https://forgottenpillar.com/rsc/en-fp-chapter18}{The Heavenly Trio}


\qrchapter{https://forgottenpillar.com/rsc/ro-fp-chapter18}{Trio-ul ceresc}


So far we have seen the evidence that Ellen White knew about Dr. Kellogg's trinitarian sentiments, and we have seen how she responded to it. She always uplifted the truth on the presence and the \emcap{personality of God}, and called to come back to the foundation of our faith—\emcap{Fundamental Principles}. However, when Adventist scholars discuss the doctrine of the Trinity and Ellen White, they do not approach it in the same manner as Ellen White did. The \emcap{Fundamental Principles} together with the doctrine on the \emcap{personality of God} is downplayed, and the twisted story is presented that Ellen White was trinitarian and responsible for the church's acceptance of the Trinity doctrine into our ranks. We want to challenge this twisted story by looking at the evidence that is often used to support this false narrative.


Până acum am văzut dovezile că Ellen White știa despre opiniile trinitariene ale Dr. Kellogg și am văzut cum a răspuns ea la acestea. Ea a înălțat întotdeauna adevărul despre prezența și \emcap{personalitatea lui Dumnezeu} și a chemat să ne întoarcem la temelia credinței noastre—\emcap{Principiile fundamentale}. Cu toate acestea, când învățații adventiști discută despre doctrina Trinității și Ellen White, ei nu o abordează în același mod în care a făcut-o Ellen White. \emcap{Principiile fundamentale} împreună cu doctrina despre \emcap{personalitatea lui Dumnezeu} sunt minimalizate, iar povestea denaturată prezentată este că Ellen White era trinitariană și responsabilă pentru acceptarea doctrinei Trinității de către biserică în rândurile noastre. Vrem să contestăm această poveste denaturată analizând dovezile care sunt adesea folosite pentru a susține această narațiune falsă.


One of the most prominent quotations to support the claim that Sister White was responsible for accepting the Trinity doctrine into our ranks is her writings and comments on Matthew 28:19\footnote{\bible{Go ye therefore, and teach all nations, baptizing them in the name of the Father, and of the Son, and of the Holy Ghost}[Matthew 28:19]}. The most prominent quotation to stand out in defense of the Trinity doctrine is “\textit{the Heavenly Trio}” quotation:


Unul dintre cele mai proeminente citate pentru a susține afirmația că sora White a fost responsabilă pentru acceptarea doctrinei Trinității în rândurile noastre sunt scrierile și comentariile ei despre Matei 28:19\footnote{\bible{Duceți-vă dar și faceți ucenici din toate neamurile, botezându-i în Numele Tatălui și al Fiului și al Sfântului Duh}[Matei 28:19]}. Cel mai proeminent citat care iese în evidență în apărarea doctrinei Trinității este citatul „\textit{Trio-ul ceresc}”:


\egw{\textbf{There are \underline{three living persons} of the \underline{heavenly trio}}; in the name of these three great powers—\textbf{the Father, the Son, and the Holy Spirit}—those who receive Christ by living faith are baptized, and these powers will co-operate with the obedient subjects of heaven in their efforts to live the new life in Christ...}[Ev 615.1; 1946][https://egwwritings.org/read?panels=p30.3407]


\egw{\textbf{Există \underline{trei persoane vii} ale \underline{trio-ului ceresc}}; în numele acestor trei mari puteri—\textbf{Tatăl, Fiul și Duhul Sfânt}—cei care Îl primesc pe Hristos prin credință vie sunt botezați, iar aceste puteri vor coopera cu supușii ascultători ai cerului în eforturile lor de a trăi viața nouă în Hristos...}[Ev 615.1; 1946][https://egwwritings.org/read?panels=p30.3407]


To reiterate, this quotation is often cited to argue that Sister White defended and advocated the Trinity doctrine. But, if we take a look at this quotation in its literary context, we see that within the quotation itself she actually \textit{refuted} this doctrine and exalted the truth on the \emcap{personality of God}. To some this is a ludicrous claim, but we invite you to make your judgment based on presented data. Let us examine the context of this quotation.


Pentru a reitera, acest citat este adesea menționat pentru a argumenta că sora White a apărat și a susținut doctrina Trinității. Dar, dacă ne uităm la acest citat în contextul său literar, vedem că în cadrul citatului însuși ea de fapt a \textit{respins} această doctrină și a înălțat adevărul despre \emcap{personalitatea lui Dumnezeu}. Pentru unii aceasta este o afirmație ridicolă, dar vă invităm să vă faceți judecata pe baza datelor prezentate. Să examinăm contextul acestui citat.


\egw{I am instructed to say, \textbf{The sentiments} of those who are searching for advanced scientific ideas \textbf{\underline{are not to be trusted}}. Such representations as the following are made: ‘\textbf{The Father is as the light invisible; the Son is as the light embodied; the Spirit as the light shed abroad.}’ ‘\textbf{The Father is like the dew, invisible vapor; the Son is like the dew gathered in beauteous form; the Spirit is like the dew fallen to the seat of life.}’ Another representation: ‘\textbf{The Father is like the invisible vapor. The Son is like the leaden cloud. The Spirit is rain fallen and working in refreshing power.}’}[Ms21-1906.8; 1906][https://egwwritings.org/read?panels=p9754.15]


\egw{Sunt instruită să spun: \textbf{Opiniile} celor care caută idei științifice avansate \textbf{\underline{nu trebuie să fie de încredere}}. Sunt făcute reprezentări precum următoarele: ‘\textbf{Tatăl este ca lumina invizibilă; Fiul este ca lumina întrupată; Duhul ca lumina răspândită.}’ ‘\textbf{Tatăl este ca roua, vapori invizibili; Fiul este ca roua adunată în formă frumoasă; Duhul este ca roua căzută pe scaunul vieții.}’ O altă reprezentare: ‘\textbf{Tatăl este ca vaporii invizibili. Fiul este ca norul de plumb. Duhul este ploaia căzută și lucrând cu putere înviorătoare.}’}[Ms21-1906.8; 1906][https://egwwritings.org/read?panels=p9754.15]


What sentiments are not to be trusted? The data suggest that those sentiments are trinitarian ideas of \textit{one God in three persons}. How do we know that? We see in the literary context of the representations Sister White was quoting. Contrary to the popular belief that she was referencing the “\textit{false}” trinity expressed by Dr. Kellogg,\footnote{Whidden, Woodrow W, et al. \textit{The Trinity : Understanding God’s Love, His Plan of Salvation, and Christian Relationships}. Hagerstown, Md, Review And Herald Pub. Association, 2002, p. 216.} she was actually referencing trinitarian idea of \textit{three living persons of one living God}, advocated by William Boardman, in his book “Higher Christian Life”, which she quoted. The context matters. The context of the quotations she quoted, shows that the representations of the Father, the Son, and the Holy Spirit are serving to illustrate the sentiment of three living persons of one God. That is the sentiment we have been clearly instructed by God, not to trust. Let the data be its own interpreter.


Ce opinii nu trebuie să fie de încredere? Datele sugerează că acele opinii sunt idei trinitariene despre \textit{un singur Dumnezeu în trei persoane}. Cum știm asta? Vedem în contextul literar al reprezentărilor pe care le cita sora White. Contrar credinței populare că ea făcea referire la trinitatea „\textit{falsă}” exprimată de Dr. Kellogg,\footnote{Whidden, Woodrow W, et al. \textit{The Trinity : Understanding God's Love, His Plan of Salvation, and Christian Relationships}. Hagerstown, Md, Review And Herald Pub. Association, 2002, p. 216.} ea de fapt făcea referire la ideea trinitariană despre \textit{trei persoane vii ale unui singur Dumnezeu viu}, susținută de William Boardman, în cartea sa „Higher Christian Life”, pe care ea a citat-o. Contextul contează. Contextul citatelor pe care le-a citat arată că reprezentările Tatălui, Fiului și Duhului Sfânt servesc pentru a ilustra opinia despre trei persoane vii ale unui singur Dumnezeu. Aceasta este opinia despre care am fost clar instruiți de Dumnezeu să nu avem încredere. Lăsați datele să fie propriul lor interpret.


\section*{The Higher Christian Life, William Boardman}


\section*{The Higher Christian Life, William Boardman}


Ellen White owned William Boardman's book “Higher Christian Life.” It was a good book about Christian sanctification, but in it there was trinitarian sentiment, which Sister White was particularly instructed by God to call out. This is another instance of evidence where we see that Ellen White was familiar with the trinitarian stance, and she was addressing it directly. Let's get familiar with the trinitarian sentiments promoted by William Boardman.


Ellen White deținea cartea lui William Boardman „Higher Christian Life”. Era o carte bună despre sfințirea creștină, dar în ea exista o opinie trinitariană, pe care sora White a fost în mod special instruită de Dumnezeu să o evidențieze. Aceasta este o altă dovadă unde vedem că Ellen White era familiarizată cu poziția trinitariană și o aborda direct. Să ne familiarizăm cu opiniile trinitariene promovate de William Boardman.


Speaking of Triune God, William Boardman writes:


Vorbind despre Dumnezeul Triunic, William Boardman scrie:


\othersQuote{And then, again, the Father is the author and planner of salvation through faith in his Son; and when we trust in his Son we honor the Father, because we accept of his plan of salvation for us, justify his wisdom, and act in accordance with his will in the matter. \textbf{A glance at the official and essential relations of the persons of the Holy Trinity to each other and to us, may throw additional light upon our pathway}. Upon this subject flippancy would border upon blasphemy. It is holy ground. He who ventures upon it may well tread with unshod foot, and uncovered head bowed low.}[William Boardman, The Higher Christian Life, p. 99; 1858][https://archive.org/details/higherchristian02boargoog/page/n106/]


\othersQuote{Și apoi, din nou, Tatăl este autorul și planificatorul mântuirii prin credința în Fiul Său; și când ne încredem în Fiul Său, Îl onorăm pe Tatăl, pentru că acceptăm planul Său de mântuire pentru noi, justificăm înțelepciunea Sa și acționăm în conformitate cu voia Sa în această privință. \textbf{O privire asupra relațiilor oficiale și esențiale ale persoanelor Sfintei Trinități între ele și cu noi, poate arunca lumină suplimentară asupra căii noastre}. Asupra acestui subiect, neseriozitatea s-ar învecinea cu blasfemia. Este pământ sfânt. Cel care se aventurează pe el poate foarte bine să calce cu piciorul descălțat și capul descoperit plecat adânc.}[William Boardman, The Higher Christian Life, p. 99; 1858][https://archive.org/details/higherchristian02boargoog/page/n106/]


Brother Boardman wants us to take \others{a glance at the official and essential relations} of the three persons of the Holy Trinity. He asserts that \textit{God is one but also three}–\textit{Triune}–by presenting official and essential relations of the persons of the Holy Trinity. His fundamental statement and outline for his thesis is as follows:


Fratele Boardman vrea să aruncăm \others{o privire asupra relațiilor oficiale și esențiale} ale celor trei persoane ale Sfintei Trinități. El afirmă că \textit{Dumnezeu este unul, dar și trei}–\textit{Trinitar}–prezentând relațiile oficiale și esențiale ale persoanelor Sfintei Trinități. Afirmația sa fundamentală și schița pentru teza sa este următoarea:


\othersQuote{\textbf{The Father is fullness of the Godhead \underline{invisibly}, without form, whom no creature hath seen or can see}. \\
\textbf{The Son is the fullness of the Godhead \underline{embodied}, that his creatures may see him, and know him, and trust him}. \\
\textbf{The Spirit is the fullness of the Godhead \underline{in all active workings}, whether of creation, providence, revelation, or salvation, by which God manifests himself to and through the universe}.}[William Boardman, The Higher Christian Life, p. 100][https://archive.org/details/higherchristian02boargoog/page/n108/]


\othersQuote{\textbf{Tatăl este plinătatea Dumnezeirii \underline{invizibil}, fără formă, pe care nicio creatură nu L-a văzut sau nu-L poate vedea}. \\
\textbf{Fiul este plinătatea Dumnezeirii \underline{întrupată}, pentru ca creaturile Sale să-L poată vedea, să-L cunoască și să se încreadă în El}. \\
\textbf{Duhul este plinătatea Dumnezeirii \underline{în toate lucrările active}, fie de creație, providență, revelație sau mântuire, prin care Dumnezeu Se manifestă către și prin univers}.}[William Boardman, The Higher Christian Life, p. 100][https://archive.org/details/higherchristian02boargoog/page/n108/]


This statement is foundational to his following statements and illustrations. In the following paragraphs, William Boardman gives the biblical motives to illustrate \others{the official and essential relations of the Holy Trinity}—\textit{that is, God being one, but yet three}. He writes:


Această afirmație este fundamentală pentru afirmațiile și ilustrațiile sale următoare. În paragrafele următoare, William Boardman oferă motivele biblice pentru a ilustra \others{relațiile oficiale și esențiale ale Sfintei Trinități}—\textit{adică, Dumnezeu fiind unul, dar totuși trei}. El scrie:


\othersQuote{Another of the names of Jesus will give the same analogies in a light not less striking - \textbf{The Sun of Righteousness}. \\
All the light of the sun in the heavens was once hidden in the invisibility of primal darkness; and after this, the light now blazing in the orb of day was, when first the command when forth, Let light be! and light was, at most only the diffused haze of the gray dawn of the morn of creation out of the darkness of chaotic night, without form, or body, or centre, or radiance, or glory. But when separated from the darkness and centered in the sun, then in its glorious glitter it became so resplendent that none but the eagle eye could bear to look it in the face. \\
But then again its rays falling aslant through earth’s atmosphere and vapors, gladdens all the world with the same light, dispelling the winter, and the cold, and the darkness; starting Spring forth in floral beauty, and Summer in vernal luxuriance, and Autumn laden with golden treasures for the garner.
\textbf{The Father is as the Light invisible}. \\
\textbf{The Son is as the Light embodied}. \\
\textbf{The Spirit is as the Light shed down}.}[William Boardman, The Higher Christian Life, p. 101,102][https://archive.org/details/higherchristian02boargoog/page/n108/]


\othersQuote{Un alt nume al lui Isus va oferi aceleași analogii într-o lumină nu mai puțin izbitoare - \textbf{Soarele Neprihănirii}. \\
Toată lumina soarelui din ceruri a fost odată ascunsă în invizibilitatea întunericului primar; și după aceasta, lumina care acum strălucește în globul zilei a fost, când prima dată a ieșit porunca, Să fie lumină! și a fost lumină, cel mult doar ceața difuză a zorilor cenușii ai dimineții creației din întunericul nopții haotice, fără formă, sau corp, sau centru, sau strălucire, sau glorie. Dar când a fost separată de întuneric și centrată în soare, atunci în strălucirea sa glorioasă a devenit atât de strălucitoare încât numai ochiul de vultur putea suporta să o privească în față. \\
Dar apoi din nou razele sale căzând oblic prin atmosfera și vaporii pământului, înveselesc toată lumea cu aceeași lumină, risipind iarna, și frigul, și întunericul; pornind Primăvara în frumusețe florală, și Vara în luxurianță verde, și Toamna încărcată cu comori aurii pentru hambar.
\textbf{Tatăl este ca Lumina invizibilă}. \\
\textbf{Fiul este ca Lumina întrupată}. \\
\textbf{Duhul este ca Lumina revărsată}.}[William Boardman, The Higher Christian Life, p. 101,102][https://archive.org/details/higherchristian02boargoog/page/n108/]


This illustration of the Sun of Righteousness shows that God the Father, who is \textit{the fullness of the Godhead invisible,} can be symbolically illustrated as a Light that \others{was once hidden in the invisibility of primal darkness}. The Son, who is \textit{the fullness of the Godhead embodied}, is like a Light that is embodied in \others{the morn of creation}. The Holy Spirit, who is \textit{the fullness of the Godhead in all active workings}, is like a \others{Light shed down}. William Boardman gives us another similar illustration to clarify the \others{official relations of the persons of the Godhead}:


Această ilustrație a Soarelui Neprihănirii arată că Dumnezeu Tatăl, care este \textit{plinătatea Dumnezeirii invizibile}, poate fi ilustrat simbolic ca o Lumină care \others{a fost odată ascunsă în invizibilitatea întunericului primar}. Fiul, care este \textit{plinătatea Dumnezeirii întrupate}, este ca o Lumină care este întrupată în \others{dimineața creației}. Duhul Sfânt, care este \textit{plinătatea Dumnezeirii în toate lucrările active}, este ca o \others{Lumină revărsată}. William Boardman ne oferă o altă ilustrație similară pentru a clarifica \others{relațiile oficiale ale persoanelor Dumnezeirii}:


\othersQuote{One of the similies for blessed influences of the Spirit, \textbf{while giving the self-same official relations of the persons of the Godhead, to each other and to us}, may illustrate them still further,—\textbf{The Dew},—\textbf{The dew of Hermon} - the dew on the mown meadow. Before the dew gathers at all in drops, it hangs over all the landscape in visible vapor, omnipresent but unseen. By and by as the light wanes into morning, and as the temperature sinks and touches the dew point the invisible becomes the visible, the embodied; and, as the sun rises, it stands in diamond drops trembling and glittering in the sun’s young beams in pearly beauty upon leaf and flower, over all the face of nature. \\
But now again, a breeze springs up, the breath of heaven is wafted gently along, shaking leaf and flower, and in a moment the pearly drops are invisible angina. But where now? Fallen at the root of herb and flower to impart new life, freshness, vigor to all it touches. \\
\textbf{The Father is like the dew in invisible vapor}. \\
\textbf{The Son is like the dew gathered in beauteous form}. \\
\textbf{The Spirit is like the dew fallen to the seat of life}.}[William Boardman, The Higher Christian Life, p. 102,103][https://archive.org/details/higherchristian02boargoog/page/n110/]


\othersQuote{Una dintre asemănările pentru influențele binecuvântate ale Duhului, \textbf{în timp ce oferă aceleași relații oficiale ale persoanelor Dumnezeirii, între ele și cu noi}, le poate ilustra și mai mult,—\textbf{Roua},—\textbf{Roua Hermonului} - roua pe pajiștea cosită. Înainte ca roua să se adune deloc în picături, ea atârnă peste tot peisajul în vapori vizibili, omniprezentă dar nevăzută. Pe măsură ce lumina scade spre dimineață și temperatura scade și atinge punctul de rouă, invizibilul devine vizibil, întrupat; și, pe măsură ce soarele răsare, stă în picături de diamant tremurând și sclipind în razele tinere ale soarelui în frumusețe sidefie pe frunză și floare, peste toată fața naturii. \\
Dar acum din nou, se ridică o briză, suflarea cerului este purtată ușor de-a lungul, scuturând frunza și floarea, și într-o clipă picăturile sidefii sunt invizibile din nou. Dar unde acum? Căzute la rădăcina ierbii și florii pentru a împărtăși viață nouă, prospețime, vigoare la tot ce atinge. \\
\textbf{Tatăl este ca roua în vapori invizibili}. \\
\textbf{Fiul este ca roua adunată în formă frumoasă}. \\
\textbf{Duhul este ca roua căzută la sediul vieții}.}[William Boardman, The Higher Christian Life, p. 102,103][https://archive.org/details/higherchristian02boargoog/page/n110/]


The Father, who is \textit{the fullness of the Godhead invisible,} is illustrated by the \others{dew in invisible vapor}. The Son, who is \textit{the fullness of the Godhead embodied}, is illustrated by \others{the dew gathered in beauteous form}. The Spirit, who is \textit{the fullness of the Godhead in all active works}, is illustrated by \others{the dew fallen to the seat of life}. The next illustration that exemplifies the official relations of the three personalities of one God is by another Bible likening—the Rain.


Tatăl, care este \textit{plinătatea Dumnezeirii invizibile}, este ilustrat prin \others{roua în vapori invizibili}. Fiul, care este \textit{plinătatea Dumnezeirii întrupate}, este ilustrat prin \others{roua adunată în formă frumoasă}. Duhul, care este \textit{plinătatea Dumnezeirii în toate lucrările active}, este ilustrat prin \others{roua căzută la sediul vieții}. Următoarea ilustrație care exemplifică relațiile oficiale ale celor trei personalități ale unui singur Dumnezeu este printr-o altă asemănare biblică—Ploaia.


\othersQuote{\textbf{Yet one more of these Bible likenings} – by no means exhausting them – will not be unwelcome, or useless, - \textbf{the Rain}. \\
Rain, like the dew, floats in invisibility, and omnipresence at the first, over all, around all. Seen by none. While it remains in its invisibility, the earth parches, clods cleave together, the ground cracks open, the sun pours down his burning heat, the winds lift up the dust in circling whirls, and rolling clouds, and famine gaunt and greedy stalks through the land, followed by pestilence and death. By and by, the eager watcher sees the little hand-like cloud rising far out over the sea. It gathers, gathers, gathers; comes and spreads as it comes, in majesty over the whole heavens: - But all is parched and dry and dead yet, upon earth. \\
But now comes a drop, and drop after drop, quicker, faster – the shower, the rain – sweeping on, and giving to earth all the treasures of the clouds – clods open, furrows soften, springs, rivulets, rivers, swell and fill, and all the land is gladdened again with restored abundance. \\
\textbf{The Father is like to the invisible vapor}. \\
\textbf{The Son is as the laden cloud and falling rain}. \\
\textbf{The Spirit is the Rain – fallen and working in refreshing power}.}[William Boardman, The Higher Christian Life, p. 103,104][https://archive.org/details/higherchristian02boargoog/page/n110/]


\othersQuote{\textbf{Încă una dintre aceste asemănări biblice} – nicidecum epuizându-le – nu va fi nedorită sau inutilă, - \textbf{Ploaia}. \\
Ploaia, ca și roua, plutește în invizibilitate și omniprezență la început, peste tot, în jurul tuturor. Văzută de nimeni. În timp ce rămâne în invizibilitatea sa, pământul se usucă, bulgării se lipesc împreună, pământul se crapă, soarele își revarsă căldura arzătoare, vânturile ridică praful în vârtejuri circulare și nori rostogolitori, și foametea slabă și lacomă pășește prin țară, urmată de ciumă și moarte. Pe măsură ce trece timpul, privitorul nerăbdător vede micul nor ca o mână ridicându-se departe peste mare. Se adună, se adună, se adună; vine și se întinde pe măsură ce vine, în maiestate peste toate cerurile: - Dar totul este încă uscat și arid și mort pe pământ. \\
Dar acum vine o picătură, și picătură după picătură, mai repede, mai rapid – aversă, ploaie – măturând înainte și dând pământului toate comorile norilor – bulgării se deschid, brazdele se înmoaie, izvoarele, pâraiele, râurile se umflă și se umplu, și toată țara se bucură din nou cu abundență restaurată. \\
\textbf{Tatăl este ca vaporii invizibili}. \\
\textbf{Fiul este ca norul încărcat și ploaia care cade}. \\
\textbf{Duhul este Ploaia – căzută și lucrând în putere răcoritoare}.}[William Boardman, The Higher Christian Life, p. 103,104][https://archive.org/details/higherchristian02boargoog/page/n110/]


Let's give William Boardman a fair hearing. He is not saying that the Father is \others{invisible vapor}; rather, he uses a metaphor of rain and \others{invisible vapor} to illustrate his main point that the Father is the invisible fullness of the Godhead. So it is with the Son, who, just like rain manifested in leaden clouds, is all the fullness of the Godhead manifested. To ensure his sentiments are not potentially misrepresented, William Boardman clarified his sentiment. This was the very sentiment that Ellen White was instructed by God not to trust:


Să-i acordăm lui William Boardman o audiere corectă. El nu spune că Tatăl este \others{vapori invizibili}; mai degrabă, el folosește o metaforă a ploii și a \others{vaporilor invizibili} pentru a ilustra punctul său principal că Tatăl este plinătatea invizibilă a Dumnezeirii. La fel este și cu Fiul, care, la fel ca ploaia manifestată în nori încărcați, este toată plinătatea Dumnezeirii manifestată. Pentru a se asigura că opiniile sale nu sunt potențial denaturate, William Boardman și-a clarificat opinia. Aceasta a fost chiar opinia pe care Ellen White a fost instruită de Dumnezeu să nu o accepte:


\othersQuote{\textbf{These likenings are all imperfect. They rather hide than illustrate \underline{the tri-personality of the one God}, for they are not persons but things, poor and earthly at best, to represent the living personalities of the living God. So much they may do, however, as to illustrate the official relations of each to the others and of each and all to us. And more. They may also illustrate the truth that all the fulness of Him who filleth all in all, dwells in each person of \underline{the Triune God}}. \\
\textbf{The Father is all the fulness of the Godhead INVISIBLE}. \\
\textbf{The Son is all the fulness of the Godhead MANIFESTED}. \\
\textbf{The Spirit is all the fulness of the Godhead MAKING MANIFEST}. \\
\textbf{The persons are not mere offices, or modes of revelation, but living persons of the living God}.}[William Boardman, The Higher Christian Life, p. 104,105][https://archive.org/details/higherchristian02boargoog/page/n112/]


\othersQuote{\textbf{Aceste asemănări sunt toate imperfecte. Ele mai degrabă ascund decât ilustrează \underline{tri-personalitatea unicului Dumnezeu}, pentru că ele nu sunt persoane ci lucruri, sărace și pământești în cel mai bun caz, pentru a reprezenta personalitățile vii ale Dumnezeului viu. Totuși, atât pot face ele, încât să ilustreze relațiile oficiale ale fiecăruia cu ceilalți și ale fiecăruia și tuturor cu noi. Și mai mult. Ele pot de asemenea ilustra adevărul că toată plinătatea Celui care umple toate în toți, locuiește în fiecare persoană a \underline{Dumnezeului Triun}}. \\
\textbf{Tatăl este toată plinătatea Dumnezeirii INVIZIBILE}. \\
\textbf{Fiul este toată plinătatea Dumnezeirii MANIFESTATE}. \\
\textbf{Duhul este toată plinătatea Dumnezeirii CARE FACE MANIFESTĂ}. \\
\textbf{Persoanele nu sunt simple funcții, sau moduri de revelație, ci persoane vii ale Dumnezeului viu}.}[William Boardman, The Higher Christian Life, p. 104,105][https://archive.org/details/higherchristian02boargoog/page/n112/]


It is crucial to emphasize that when Boardman uses these Bible likenings from nature, he speaks of the illustrations, and not reality. These representations are illustrating his sentiments. In his own admission, that was the sentiment of three \others{living personalities of the living God.} Though these illustrations are imperfect, they may \others{illustrate the official relations} of \others{the tri-personality of the one God} and \others{the truth that all the fullness of Him who filleth all in all dwells in each person of the Triune God.} One God in three persons is the sentiment in question, and that sentiment is common to all types and versions of the trinity doctrine—including our current trinitarian stance in the second point of the Fundamental Beliefs.\footnote{\others{There is \textbf{one God}: Father, Son, and Holy Spirit, \textbf{a unity of three} coeternal \textbf{Persons}…} 2nd point of the Fundamental Beliefs}


Este crucial să subliniem că atunci când Boardman folosește aceste asemănări biblice din natură, el vorbește despre ilustrații, și nu despre realitate. Aceste reprezentări ilustrează opiniile sale. În propria sa recunoaștere, aceasta era opinia a trei \others{personalități vii ale Dumnezeului viu.} Deși aceste ilustrații sunt imperfecte, ele pot \others{ilustra relațiile oficiale} ale \others{tri-personalității unicului Dumnezeu} și \others{adevărul că toată plinătatea Celui care umple toate în toți locuiește în fiecare persoană a Dumnezeului Triun.} Un Dumnezeu în trei persoane este opinia în discuție, și acea opinie este comună tuturor tipurilor și versiunilor doctrinei Trinității—inclusiv poziția noastră trinitariană actuală din al doilea punct al Punctelor Fundamentale de Credință.\footnote{\others{Există \textbf{un singur Dumnezeu}: Tatăl, Fiul și Duhul Sfânt, \textbf{o unitate a trei} \textbf{Persoane} coeterne…} Al doilea punct al Punctelor Fundamentale de Credință}


In this brief look at William Boardman's sentiments, it is clear that the sentiments in question which Ellen White was instructed by God to call out, were the sentiments of the Triune God, or \textit{three living persons in the Trinity}. With that data in mind, let's examine Ellen White's response.


În această scurtă privire asupra opiniilor lui William Boardman, este clar că opiniile în discuție pe care Ellen White a fost instruită de Dumnezeu să le denunțe, erau opiniile despre Dumnezeul Triun, sau \textit{trei persoane vii în Trinitate}. Cu aceste date în minte, să examinăm răspunsul lui Ellen White.


\section*{Ellen White on William Boardman’s sentiment}


\section*{Ellen White despre opinia lui William Boardman}


With the Heavenly Trio quotation, it has been asserted that Ellen White was trinitarian. This is done by ignorantly or sometimes purposely ignoring the context of this valuable quotation. When reading Ellen White’s response, in which she defends our perceptions of God, try to recognize whom she is addressing when she speaks of God. Was the God she defended the Trinity or the Father? Referencing William Boardmans illustrations she said:


Cu citatul despre Trio-ul Ceresc, s-a afirmat că Ellen White era trinitariană. Acest lucru se face ignorând din neștiință sau uneori în mod intenționat contextul acestui citat valoros. Când citiți răspunsul lui Ellen White, în care ea apără percepțiile noastre despre Dumnezeu, încercați să recunoașteți cui se adresează când vorbește despre Dumnezeu. Era Dumnezeul pe care l-a apărat Trinitatea sau Tatăl? Făcând referire la ilustrațiile lui William Boardman, ea a spus:


\egw{\textbf{All these \underline{spiritualistic} representations are simply nothingness}. They are imperfect, untrue. They weaken and diminish the Majesty which no earthly likeness can be compared to. \textbf{God cannot be compared with the things His hands have made}. These are mere earthly things, suffering under the curse of God because of the sins of man. \textbf{The Father cannot be described by the things of earth}. \textbf{The Father is all the fulness of the Godhead \underline{bodily} and is \underline{invisible to mortal sight}}.}[Ms21-1906.9; 1906][https://egwwritings.org/read?panels=p9754.15]


\egw{\textbf{Toate aceste reprezentări \underline{spiritualiste} sunt pur și simplu nimic}. Ele sunt imperfecte, neadevărate. Ele slăbesc și diminuează Maiestatea cu care nicio asemănare pământească nu poate fi comparată. \textbf{Dumnezeu nu poate fi comparat cu lucrurile pe care mâinile Lui le-au făcut}. Acestea sunt simple lucruri pământești, suferind sub blestemul lui Dumnezeu din cauza păcatelor omului. \textbf{Tatăl nu poate fi descris prin lucrurile de pe pământ}. \textbf{Tatăl este toată plinătatea Dumnezeirii \underline{trupește} și este \underline{invizibil vederii muritoare}}.}[Ms21-1906.9; 1906][https://egwwritings.org/read?panels=p9754.15]


By observing the context, it is obvious that Sister White follows Boardman’s line of reasoning and corrects the mistakes. For better comparison, let us look at their writings side by side:


Observând contextul, este evident că sora White urmează linia de raționament a lui Boardman și corectează greșelile. Pentru o comparație mai bună, să ne uităm la scrierile lor alăturate:


\begin{table}[H]
\centering
\renewcommand{\arraystretch}{1.5}
\setlength{\tabcolsep}{15pt}
\resizebox{\textwidth}{!}{
\begin{tabular}{|p{0.4\textwidth}|p{0.4\textwidth}|}
\hline
\multicolumn{1}{|c|}{\textbf{William Boardman}} & \multicolumn{1}{c|}{\textbf{Ellen G. White}} \\ \hline
\othersQuote{These likenings are all imperfect. They rather hide than \textbf{illustrate the tri-personality of the \underline{one God}}, for they are not persons but things, poor and earthly at best, to represent \textbf{the living personalities of the living God}. \textbf{So much they may do, however, as to illustrate the official relations of each to the other and of each and all to us. And more. They may also illustrate the truth that all the fulness of Him who filleth all in all, dwells in \underline{each person of Triune God}}.}[p. 104,105][https://archive.org/details/higherchristian02boargoog/page/n112] & 
\egw{\textbf{All these \underline{spiritualistic} representations are simply nothingness}. They are imperfect, untrue. They weaken and diminish the Majesty which no earthly likeness can be compared to. \textbf{God cannot be compared with the things His hands have made}. These are mere earthly things, suffering under the curse of God because of the sins of man. \textbf{The Father cannot be described by the things of earth}.}[Ms21-1906.9; 1906][https://egwwritings.org/read?panels=p9754.15] \\ \hline
\end{tabular}
}
\end{table}


\begin{table}[H]
\centering
\renewcommand{\arraystretch}{1.5}
\setlength{\tabcolsep}{15pt}
\resizebox{\textwidth}{!}{
\begin{tabular}{|p{0.4\textwidth}|p{0.4\textwidth}|}
\hline
\multicolumn{1}{|c|}{\textbf{William Boardman}} & \multicolumn{1}{c|}{\textbf{Ellen G. White}} \\ \hline
\othersQuote{Aceste asemănări sunt toate imperfecte. Ele mai degrabă ascund decât \textbf{ilustrează tri-personalitatea \underline{unicului Dumnezeu}}, pentru că ele nu sunt persoane ci lucruri, sărace și pământești în cel mai bun caz, pentru a reprezenta \textbf{personalitățile vii ale Dumnezeului viu}. \textbf{Totuși, atât pot face ele, încât să ilustreze relațiile oficiale ale fiecăruia cu celălalt și ale fiecăruia și tuturor cu noi. Și mai mult. Ele pot de asemenea ilustra adevărul că toată plinătatea Celui care umple toate în toți, locuiește în \underline{fiecare persoană a Dumnezeului Triun}}.}[p. 104,105][https://archive.org/details/higherchristian02boargoog/page/n112] & 
\egw{\textbf{Toate aceste reprezentări \underline{spiritualiste} sunt pur și simplu nimic}. Ele sunt imperfecte, neadevărate. Ele slăbesc și diminuează Maiestatea cu care nicio asemănare pământească nu poate fi comparată. \textbf{Dumnezeu nu poate fi comparat cu lucrurile pe care mâinile Lui le-au făcut}. Acestea sunt simple lucruri pământești, suferind sub blestemul lui Dumnezeu din cauza păcatelor omului. \textbf{Tatăl nu poate fi descris prin lucrurile de pe pământ}.}[Ms21-1906.9; 1906][https://egwwritings.org/read?panels=p9754.15] \\ \hline
\end{tabular}
}
\end{table}


In this comparison, it is clear who God is for William Boardman, and who He is for Sister White. For Boardman, God is the Triune God, a tri-personality of the one God. For Sister White, God is the Father. For Boardman, these representations are imperfect because they \others{rather hide than illustrate the tri-personality of the one God}, and for Sister White these representations are imperfect because \egw{The Father cannot be described by the things of earth}. For Boardman, God is the \textit{Triune God}; for Sister White, God is \textit{the Father}.


În această comparație, este clar cine este Dumnezeu pentru William Boardman și cine este El pentru sora White. Pentru Boardman, Dumnezeu este Dumnezeul Triun, o tri-personalitate a unicului Dumnezeu. Pentru sora White, Dumnezeu este Tatăl. Pentru Boardman, aceste reprezentări sunt imperfecte pentru că ele \others{mai degrabă ascund decât ilustrează tri-personalitatea unicului Dumnezeu}, iar pentru sora White aceste reprezentări sunt imperfecte pentru că \egw{Tatăl nu poate fi descris prin lucrurile de pe pământ}. Pentru Boardman, Dumnezeu este \textit{Dumnezeul Triun}; pentru sora White, Dumnezeu este \textit{Tatăl}.


Boardman’s only point that Ellen White affirms is that these representations are imperfect. Surely, William Boardman would not agree with Ellen White that these representations are \textit{spiritualistic} and \textit{untrue}. On the contrary, he believes that these illustrations \others{illustrate the truth that all the fulness of Him who filleth all in all, dwells in each person of Triune God}. To say that Ellen White agreed with such sentiment is gross misrepresentation.


Singurul punct al lui Boardman pe care Ellen White îl afirmă este că aceste reprezentări sunt imperfecte. Cu siguranță, William Boardman nu ar fi de acord cu Ellen White că aceste reprezentări sunt \textit{spiritualiste} și \textit{neadevărate}. Dimpotrivă, el crede că aceste ilustrații \others{ilustrează adevărul că toată plinătatea Celui care umple toate în toți, locuiește în fiecare persoană a Dumnezeului Triun}. A spune că Ellen White a fost de acord cu o astfel de opinie este o denaturare grosolană.


The context of this important quotation prompts important questions. Why does the prophet of God refer to the representations that illustrate the \others{tri-personality of the one God} as \egwinline{spiritualistic representations}, which illustrate the sentiment that \egwinline{is not to be trusted}? Or why does the prophet of God refer to the representations that \others{represent the living personalities of the living God} as \egwinline{spiritualistic representations}? Or why does the prophet of God, when referring to the representations that \others{illustrate the truth that all the fullness of Him who filleth all in all, dwells in each person of Triune God}, refer to them as \egwinline{spiritualistic representations}? All of these spiritualistic representations illustrate the sentiment that \egwinline{is not to be trusted}. This sentiment is clearly the trinitarian sentiment.


Contextul acestei citate importante ridică întrebări importante. De ce profetul lui Dumnezeu se referă la reprezentările care ilustrează \others{tri-personalitatea unicului Dumnezeu} ca \egwinline{reprezentări spiritualiste}, care ilustrează opinia care \egwinline{nu trebuie să fie de încredere}? Sau de ce profetul lui Dumnezeu se referă la reprezentările care \others{reprezintă personalitățile vii ale Dumnezeului viu} ca \egwinline{reprezentări spiritualiste}? Sau de ce profetul lui Dumnezeu, când se referă la reprezentările care \others{ilustrează adevărul că toată plinătatea Celui care umple totul în toți, locuiește în fiecare persoană a Dumnezeului Trinitar}, se referă la ele ca \egwinline{reprezentări spiritualiste}? Toate aceste reprezentări spiritualiste ilustrează opinia care \egwinline{nu trebuie să fie de încredere}. Această opinie este în mod clar opinia trinitariană.


Sister White continues to follow Boardman’s line of reasoning and corrects the error.


Sora White continuă să urmeze linia de raționament a lui Boardman și corectează eroarea.


\begin{table}[H]
\centering
\renewcommand{\arraystretch}{1.5}
\setlength{\tabcolsep}{15pt}
\resizebox{\textwidth}{!}{
\begin{tabular}{|p{0.4\textwidth}|p{0.4\textwidth}|}
\hline
\multicolumn{1}{|c|}{\textbf{William Boardman}} & \multicolumn{1}{c|}{\textbf{Ellen G. White}} \\ \hline
\othersQuote{The Father is fullness of the Godhead \textbf{invisibly}, \textbf{\underline{without form}}, whom \textbf{no creature hath seen \underline{or can see}}.}[p.100][https://archive.org/details/higherchristian02boargoog/page/n108/]


\begin{table}[H]
\centering
\renewcommand{\arraystretch}{1.5}
\setlength{\tabcolsep}{15pt}
\resizebox{\textwidth}{!}{
\begin{tabular}{|p{0.4\textwidth}|p{0.4\textwidth}|}
\hline
\multicolumn{1}{|c|}{\textbf{William Boardman}} & \multicolumn{1}{c|}{\textbf{Ellen G. White}} \\ \hline
\othersQuote{Tatăl este plinătatea Dumnezeirii \textbf{invizibil}, \textbf{\underline{fără formă}}, pe care \textbf{nicio creatură nu L-a văzut \underline{sau poate vedea}}.}[p.100][https://archive.org/details/higherchristian02boargoog/page/n108/]


\othersQuote{The Father is all the fullness of the Godhead \textbf{INVISIBLE}.}[p.105][https://archive.org/details/higherchristian02boargoog/page/n112/] & 
\egw{The Father is all the fulness of the Godhead \textbf{\underline{bodily}}, and is \textbf{invisible to mortal sight}.}[Ms21-1906.9; 1906][https://egwwritings.org/read?panels=p9754.15] \\ \hline
\end{tabular}
}
\end{table}


\othersQuote{Tatăl este toată plinătatea Dumnezeirii \textbf{INVIZIBIL}.}[p.105][https://archive.org/details/higherchristian02boargoog/page/n112/] & 
\egw{Tatăl este toată plinătatea Dumnezeirii \textbf{\underline{trupește}}, și este \textbf{invizibil pentru vederea muritorilor}.}[Ms21-1906.9; 1906][https://egwwritings.org/read?panels=p9754.15] \\ \hline
\end{tabular}
}
\end{table}


For Boardman, the Father does not have a form nor body and is invisible to all creatures. For Sister White, the Father has a form and body and is invisible only to mortal human beings.\footnote{When Sister White talks about mortals, she talks about sin polluted humanity. After the restoration of humanity, at the resurrection, Christ will give His immortal life to His children. For more information read \href{https://egwwritings.org/?ref=en_RH.July.5.1887.par.5}{EGW, RH July 5, 1887, par. 5; 1887}.}


Pentru Boardman, Tatăl nu are formă nici trup și este invizibil pentru toate creaturile. Pentru Sora White, Tatăl are formă și trup și este invizibil doar pentru ființele umane muritoare.\footnote{Când Sora White vorbește despre muritori, ea vorbește despre umanitatea poluată de păcat. După restaurarea umanității, la înviere, Hristos va da viața Sa nemuritoare copiilor Săi. Pentru mai multe informații citiți \href{https://egwwritings.org/?ref=en_RH.July.5.1887.par.5}{EGW, RH July 5, 1887, par. 5; 1887}.}


This quotation is one of the most direct quotations regarding the \emcap{personality of God}. \egwinline{The Father is all the fullness of the Godhead \textbf{bodily}}[Ms21-1906.9; 1906][https://egwwritings.org/read?panels=p9754.16].


Acest citat este unul dintre cele mai directe citate privind \emcap{personalitatea lui Dumnezeu}. \egwinline{Tatăl este toată plinătatea Dumnezeirii \textbf{trupește}}[Ms21-1906.9; 1906][https://egwwritings.org/read?panels=p9754.16].


It might be confusing to someone that the Father is all the fullness of the Godhead bodily because in \textit{Colossians 2:9}, when referring to Jesus, it is written that \bible{in him dwelleth all the fulness of the Godhead bodily.} Scripture does not contradict itself. \textit{Colossians 2:9} does not exclude the Father to be all the fulness of the Godhead bodily. Various places in the Bible describe the Father having a body (\textit{a form: Daniel 7:9,10; Revelation 4:2,3; 1 Kings 22:19-22; a shape: John 5:37}). He has the appearance of a man (\textit{Ezekiel 1:26-28}). He has a face (\textit{Exodus 33:20; Matthew 18:10; Revelation 22:3, 4}). However, the Bible is completely silent about the nature of its substance. The Bible teaches us that \bible{\textbf{The secret things belong unto the LORD our God}: \textbf{but those things which \underline{are revealed} belong unto us and to our children for ever}, that we may do all the words of this law}[Deuteronomy 29:29]. It is revealed to us that the Father has body, He is all the fulness of the Godhead bodily. Also, it is revealed that in Jesus also dwells all the fulness of the Godhead bodily, because \bible{it pleased the Father that in him should all fulness dwell}[Colossians 1:19]. This is not a contradiction whatsoever because the Son is \bible{the \textbf{express image of \underline{His person}}}[Hebrews 1:3].


Ar putea fi confuz pentru cineva că Tatăl este toată plinătatea Dumnezeirii trupește deoarece în \textit{Coloseni 2:9}, când se referă la Isus, este scris că \bible{în El locuiește trupește toată plinătatea Dumnezeirii.} Scriptura nu se contrazice. \textit{Coloseni 2:9} nu exclude ca Tatăl să fie toată plinătatea Dumnezeirii trupește. Diferite locuri din Biblie descriu Tatăl având un trup (\textit{o formă: Daniel 7:9,10; Apocalipsa 4:2,3; 1 Împărați 22:19-22; un chip: Ioan 5:37}). El are înfățișarea unui om (\textit{Ezechiel 1:26-28}). El are o față (\textit{Exod 33:20; Matei 18:10; Apocalipsa 22:3, 4}). Cu toate acestea, Biblia tace complet despre natura substanței sale. Biblia ne învață că \bible{\textbf{Lucrurile ascunse sunt ale Domnului Dumnezeului nostru}: \textbf{iar lucrurile \underline{descoperite} sunt ale noastre și ale copiilor noștri pe vecie}, ca să împlinim toate cuvintele legii acesteia}[Deuteronom 29:29]. Ne este descoperit că Tatăl are trup, El este toată plinătatea Dumnezeirii trupește. De asemenea, este descoperit că în Isus locuiește și toată plinătatea Dumnezeirii trupește, pentru că \bible{Tatăl a găsit cu cale să locuiască toată plinătatea în El}[Coloseni 1:19]. Aceasta nu este deloc o contradicție deoarece Fiul este \bible{\textbf{întipărirea persoanei Lui}}[Evrei 1:3].


\begin{table}[H]
\centering
\renewcommand{\arraystretch}{1.5}
\setlength{\tabcolsep}{15pt}
\resizebox{\textwidth}{!}{
\begin{tabular}{|p{0.4\textwidth}|p{0.4\textwidth}|}
\hline
\multicolumn{1}{|c|}{\textbf{William Boardman}} & \multicolumn{1}{c|}{\textbf{Ellen G. White}} \\ \hline
\othersQuote{The Son is the fullness of the Godhead \textbf{embodied, that his creatures may see him, and know him, and trust him}.}[p.100][https://archive.org/details/higherchristian02boargoog/page/n108/]


\begin{table}[H]
\centering
\renewcommand{\arraystretch}{1.5}
\setlength{\tabcolsep}{15pt}
\resizebox{\textwidth}{!}{
\begin{tabular}{|p{0.4\textwidth}|p{0.4\textwidth}|}
\hline
\multicolumn{1}{|c|}{\textbf{William Boardman}} & \multicolumn{1}{c|}{\textbf{Ellen G. White}} \\ \hline
\othersQuote{Fiul este plinătatea Dumnezeirii \textbf{întrupată, pentru ca creaturile Sale să-L poată vedea, și să-L cunoască, și să se încreadă în El}.}[p.100][https://archive.org/details/higherchristian02boargoog/page/n108/]


\othersQuote{The Son is all the fulness of the Godhead \textbf{MANIFESTED}.}[p.105][https://archive.org/details/higherchristian02boargoog/page/n112/] & 
\egw{The Son is all the fulness of the Godhead \textbf{manifested}. The Word of God declares Him to be ‘\textbf{the express image of His person}’. ‘God so loved the world that He gave \textbf{His only begotten Son}, that whosoever believeth in Him should not perish, but have everlasting life’. \textbf{Here is shown \underline{the personality of the Father}}.}[Ms21-1906.10; 1906][https://egwwritings.org/read?panels=p9754.17] \\ \hline
\end{tabular}
}
\end{table}


\othersQuote{Fiul este toată plinătatea Dumnezeirii \textbf{MANIFESTATĂ}.}[p.105][https://archive.org/details/higherchristian02boargoog/page/n112/] & 
\egw{Fiul este toată plinătatea Dumnezeirii \textbf{manifestată}. Cuvântul lui Dumnezeu Îl declară a fi ‘\textbf{întipărirea persoanei Lui}’. ‘Fiindcă atât de mult a iubit Dumnezeu lumea, că a dat pe \textbf{singurul Lui Fiu născut}, pentru ca oricine crede în El să nu piară, ci să aibă viață veșnică’. \textbf{Aici este arătată \underline{personalitatea Tatălui}}.}[Ms21-1906.10; 1906][https://egwwritings.org/read?panels=p9754.17] \\ \hline
\end{tabular}
}
\end{table}


Sister White focused on the \emcap{personality of God}, which is the personality of the Father. In Christ, who is \egwinline{begotten in the express image of the Father’s person}[ST May 30, 1895, par. 3; 1895][https://egwwritings.org/read?panels=p820.12891], is shown the personality of the Father. In the same way that Jesus is a person, so is the Father. The quality or state of Christ being a person is the same quality or state of the Father being a person. As Christ is a personal being, so is the Father. Just as all the fullness of the Godhead bodily dwells in Christ, so it does in the Father, because Christ is begotten in the express image of the Father’s person. In Him is shown the personality of the Father. These simple conclusions have been asserted by Scripture in John 3:16 and Hebrews 1:3.


Sora White s-a concentrat pe \emcap{personalitatea lui Dumnezeu}, care este personalitatea Tatălui. În Hristos, care este \egwinline{născut în întipărirea expresă a persoanei Tatălui}[ST May 30, 1895, par. 3; 1895][https://egwwritings.org/read?panels=p820.12891], este arătată personalitatea Tatălui. În același fel în care Isus este o persoană, la fel este și Tatăl. Caracteristica sau starea prin care Hristos este definit ca persoană este aceeași caracteristică sau stare prin care Tatăl este definit ca persoană. Așa cum Hristos este o ființă personală, la fel este și Tatăl. Așa cum toată plinătatea Dumnezeirii trupește locuiește în Hristos, la fel locuiește și în Tatăl, pentru că Hristos este născut în întipărirea expresă a persoanei Tatălui. În El este arătată personalitatea Tatălui. Aceste concluzii simple au fost afirmate de Scriptură în Ioan 3:16 și Evrei 1:3.


Does the same reasoning, of the personality of the Father and Son, apply to the Holy Spirit? Speaking of the Holy Spirit, Sister White continues:


Se aplică același raționament, al personalității Tatălui și Fiului, și Duhului Sfânt? Vorbind despre Duhul Sfânt, sora White continuă:


\egw{\textbf{The Comforter that Christ} promised to send after He ascended to heaven, \textbf{is the Spirit \underline{in} all the fulness of the Godhead}, making manifest the power of divine grace to all who receive and believe in Christ as a personal Saviour.}[Ms21-1906.11; 1906][https://egwwritings.org/read?panels=p9754.18]


\egw{\textbf{Mângâietorul pe care Hristos} a promis să-L trimită după ce S-a înălțat la cer, \textbf{este Duhul \underline{în} toată plinătatea Dumnezeirii}, făcând manifestă puterea harului divin tuturor celor care Îl primesc și cred în Hristos ca Mântuitor personal.}[Ms21-1906.11; 1906][https://egwwritings.org/read?panels=p9754.18]


Sister White draws a distinction between Father and Son who \textbf{are}, individually, \textbf{all} the fullness of the Godhead, and the Spirit that is \textbf{in all} the fullness of the Godhead. This is a marked contrast to William Boardman’s reasoning, where all three are the fullness of the Godhead. Sister White does not follow this trinitarian fashion. The explanation is simple in light of the \emcap{personality of God} and of Christ. The Holy Spirit is a spirit, and the spirit dwells \textbf{in} the flesh/body. The Holy Spirit is \textbf{in all} the fullness of the Godhead\footnote{Take a look at the quotation from \href{https://egwwritings.org/?ref=en_Ms128-1897.13&para=5426.19}{{EGW, Ms128-1897.13; 1897}}, where Sister White states that the Father and the Son are the absolute Godhead.}.


Sora White face o distincție între Tatăl și Fiul care \textbf{sunt}, individual, \textbf{toată} plinătatea Dumnezeirii, și Duhul care este \textbf{în toată} plinătatea Dumnezeirii. Aceasta este un contrast marcat față de raționamentul lui William Boardman, unde toți trei sunt plinătatea Dumnezeirii. Sora White nu urmează această modă trinitariană. Explicația este simplă în lumina \emcap{personalității lui Dumnezeu} și a lui Hristos. Duhul Sfânt este un duh, iar duhul locuiește \textbf{în} trup/corp. Duhul Sfânt este \textbf{în toată} plinătatea Dumnezeirii\footnote{Aruncați o privire la citatul din \href{https://egwwritings.org/?ref=en_Ms128-1897.13&para=5426.19}{{EGW, Ms128-1897.13; 1897}}, unde sora White afirmă că Tatăl și Fiul sunt Dumnezeirea absolută.}.


Finally, the quotation continues to its most renowned part:


În final, citatul continuă către partea sa cea mai renumită:


\begin{table}[H]
    \centering
    \renewcommand{\arraystretch}{1.5}
    \setlength{\tabcolsep}{15pt}
    \resizebox{\textwidth}{!}{
    \begin{tabular}{|p{0.4\textwidth}|p{0.4\textwidth}|}
    \hline
    \multicolumn{1}{|c|}{\textbf{William Boardman}} & \multicolumn{1}{c|}{\textbf{Ellen G. White}} \\ \hline
    \othersQuote{\textbf{The Father} is all the fulness of the Godhead INVISIBLE.}


\begin{table}[H]
    \centering
    \renewcommand{\arraystretch}{1.5}
    \setlength{\tabcolsep}{15pt}
    \resizebox{\textwidth}{!}{
    \begin{tabular}{|p{0.4\textwidth}|p{0.4\textwidth}|}
    \hline
    \multicolumn{1}{|c|}{\textbf{William Boardman}} & \multicolumn{1}{c|}{\textbf{Ellen G. White}} \\ \hline
    \othersQuote{\textbf{Tatăl} este toată plinătatea Dumnezeirii INVIZIBIL.}


\othersQuote{\textbf{The Son} is all the fulness of the Godhead MANIFESTED.}


\othersQuote{\textbf{Fiul} este toată plinătatea Dumnezeirii MANIFESTAT.}


\othersQuote{\textbf{The Spirit} is all the fulness of the Godhead MAKING MANIFEST.}


\othersQuote{\textbf{Duhul} este toată plinătatea Dumnezeirii FĂCÂND MANIFEST.}


\othersQuote{\textbf{The persons} are not mere offices, or modes of revelation, \textbf{but living persons of the living God}.}[p.105][https://archive.org/details/higherchristian02boargoog/page/n112/] & 
    \egw{There are \textbf{three living persons of the heavenly trio}; in the name of these three great powers—\textbf{the Father, the Son, and the Holy Spirit}—those who receive Christ by living faith are baptized, and these powers will co-operate with the obedient subjects of heaven in their efforts to live the new life in Christ.}[Ms21-1906.11; 1906][https://egwwritings.org/read?panels=p9754.18] \\ \hline
    \end{tabular}
    }
    \end{table}


\othersQuote{\textbf{Persoanele} nu sunt simple funcții, sau moduri de revelație, \textbf{ci persoane vii ale Dumnezeului viu}.}[p.105][https://archive.org/details/higherchristian02boargoog/page/n112/] & 
    \egw{Există \textbf{trei persoane vii ale trioului ceresc}; în numele acestor trei mari puteri—\textbf{Tatăl, Fiul și Duhul Sfânt}—cei care Îl primesc pe Hristos prin credință vie sunt botezați, și aceste puteri vor coopera cu supușii ascultători ai cerului în eforturile lor de a trăi viața nouă în Hristos.}[Ms21-1906.11; 1906][https://egwwritings.org/read?panels=p9754.18] \\ \hline
    \end{tabular}
    }
    \end{table}


In light of the context of William Boardman’s book, we see a marked contrast between \others{three living persons of \textbf{one living God}}, which is the trinitarian sentiment, and \egwinline{the three living persons of \textbf{the heavenly trio}}, which is in accordance with the truth on the \emcap{personality of God}.


În lumina contextului cărții lui William Boardman, vedem un contrast marcat între \others{trei persoane vii ale \textbf{unui singur Dumnezeu viu}}, care este opinia trinitariană, și \egwinline{cele trei persoane vii ale \textbf{trioului ceresc}}, care este în conformitate cu adevărul despre \emcap{personalitatea lui Dumnezeu}.


The word ‘\textit{trio}’ simply indicates the group of three. The \textit{“heavenly trio}” is represented by the Father, the Son, and the Holy Spirit. But, contrary to popular assumption, they do not make one living God. Three-in-one and one-in-three are concepts that do away with the \emcap{personality of God}. This is why Sister White referred to trinitarian sentiments as sentiments that \egwinline{are not to be trusted}[Ms21-1906.8; 1906][https://egwwritings.org/read?panels=p9754.15].


Cuvântul ‘\textit{trio}’ indică pur și simplu grupul de trei. \textit{“Trioul ceresc}” este reprezentat de Tatăl, Fiul și Duhul Sfânt. Dar, contrar presupunerii populare, ei nu formează un singur Dumnezeu viu. Trei-în-unul și unul-în-trei sunt concepte care înlătură \emcap{personalitatea lui Dumnezeu}. De aceea sora White s-a referit la opiniile trinitariene ca opinii care \egwinline{nu trebuie să fie de încredere}[Ms21-1906.8; 1906][https://egwwritings.org/read?panels=p9754.15].


Sister White never followed any trinitarian fashion—neither in words and expressions, nor in sentiments. There is an almost effortless research endeavor we encourage you to take: in the writings of Ellen White, search for standard trinitarian terms like “\textit{three are one},” “\textit{one are three},” “\textit{one in three},” “\textit{three in one},” or any of the permutations possible. In her impressive oeuvre you will not find a single occurrence of any of these, let alone the word ‘\textit{trinity}’ describing our God\footnote{There is but one occurrence, in the writings of Ellen White, of the word ‘\textit{trinity}’ referring to \egw{the lust of the flesh, the lust of the eyes and the pride of life}[Lt43-1898.25; 1898][https://egwwritings.org/read?panels=p4806.31]}. She never used these phrases that are necessary to explain the trinitarian sentiment. Examining the following quote, we can see why she never said that God is trinity.


Sora White nu a urmat niciodată vreo modă trinitariană—nici în cuvinte și expresii, nici în opinii. Există un efort de cercetare aproape fără efort pe care vă încurajăm să-l întreprindeți: în scrierile lui Ellen White, căutați termeni trinitarieni standard precum „\textit{trei sunt una}”, „\textit{una sunt trei}”, „\textit{una în trei}”, „\textit{trei în una}” sau oricare dintre permutările posibile. În opera sa impresionantă nu veți găsi nicio apariție a vreuneia dintre acestea, cu atât mai puțin cuvântul ‘\textit{trinitate}’ descriind pe Dumnezeul nostru\footnote{Există o singură apariție, în scrierile lui Ellen White, a cuvântului ‘\textit{trinitate}’ referindu-se la \egw{pofta cărnii, pofta ochilor și mândria vieții}[Lt43-1898.25; 1898][https://egwwritings.org/read?panels=p4806.31]}. Ea nu a folosit niciodată aceste expresii care sunt necesare pentru a explica opinia trinitariană. Examinând următorul citat, putem vedea de ce ea nu a spus niciodată că Dumnezeu este trinitate.


\egw{The subject of \textbf{\underline{speculation} regarding \underline{God’s personality} \underline{we will not venture} to express}, \textbf{\underline{except in the language of the Word which represents His personality}}. There is to be no discussion over this question \textbf{lest God would give unmistakable revelation of \underline{what He is}} that would extinguish the one who dares venture on the holy ground in \textbf{his speculative theories}, as some ventured to do in opening the ark to see what was in it as its power and how God was manifested. The men were slain for their curiosity science.}[17LtMs, Ms 223, 1902, par. 16][https://egwwritings.org/read?panels=p14067.9124037&index=0]


\egw{Subiectul \textbf{\underline{speculației} cu privire la \underline{personalitatea lui Dumnezeu} \underline{nu ne vom aventura} să-l exprimăm}, \textbf{\underline{decât în limbajul Cuvântului care reprezintă personalitatea Sa}}. Nu trebuie să existe nicio discuție asupra acestei chestiuni \textbf{ca nu cumva Dumnezeu să dea o revelație de necontestat despre \underline{ceea ce este El}} care l-ar nimici pe cel care îndrăznește să se aventureze pe pământul sfânt în \textbf{teoriile sale speculative}, așa cum unii s-au aventurat să facă deschizând chivotul pentru a vedea ce era în el ca putere a sa și cum S-a manifestat Dumnezeu. Oamenii au fost uciși pentru știința lor de curiozitate.}[17LtMs, Ms 223, 1902, par. 16][https://egwwritings.org/read?panels=p14067.9124037&index=0]


Did you catch that? There is to be no discussion over the question of what God is, \egwinline{lest God would give unmistakable revelation} of \egwinline{what He is}. To say “God is \_\_\_\_\_\_\_”, the blank must be filled with \egwinline{the language of the Word which represents His personality.} The Bible clearly teaches that God is a personal, spiritual being—a truth confirmed by Christ Himself in His revelations to Ellen White. This fits within the biblical language that describes God’s personality. However, according to above statement, can we say “\textit{God is trinity}?” No! That is not \egwinline{the language of the Word which represents His personality.} Therefore, within explored context, we can safely conclude that, the Trinitarian view of God is part of \egwinline{speculative theories} of \egwinline{what He is}.


Ați înțeles asta? Nu trebuie să existe nicio discuție asupra chestiunii despre ceea ce este Dumnezeu, \egwinline{ca nu cumva Dumnezeu să dea o revelație de necontestat} despre \egwinline{ceea ce este El}. Pentru a spune „Dumnezeu este \_\_\_\_\_\_\_“, spațiul gol trebuie completat cu \egwinline{limbajul Cuvântului care reprezintă personalitatea Sa.} Biblia învață clar că Dumnezeu este o ființă personală, spirituală—un adevăr confirmat de Hristos Însuși în revelațiile Sale către Ellen White. Aceasta se încadrează în limbajul biblic care descrie personalitatea lui Dumnezeu. Cu toate acestea, conform afirmației de mai sus, putem spune „\textit{Dumnezeu este trinitate}?” Nu! Acesta nu este \egwinline{limbajul Cuvântului care reprezintă personalitatea Sa.} Prin urmare, în contextul explorat, putem concluziona în siguranță că perspectiva trinitariană despre Dumnezeu face parte din \egwinline{teoriile speculative} despre \egwinline{ceea ce este El}.


This being said, the phrase \egwinline{Heavenly Trio} is not a definition of what God is. Our God is the Father—not \egwinline{the Heavenly Trio.} The term Heavenly Trio does not serve as a replacement for the Trinitarian idea of \textit{three living persons of one God}. This becomes obvious, when we examine the context. Ellen White was instructed to warn us against Trinitarian sentiments, not to trust them. She was not endorsing them.


Acestea fiind spuse, expresia \egwinline{Trio Ceresc} nu este o definiție a ceea ce este Dumnezeu. Dumnezeul nostru este Tatăl—nu \egwinline{Trio-ul Ceresc.} Termenul Trio Ceresc nu servește ca înlocuitor pentru ideea trinitariană de \textit{trei persoane vii ale unui singur Dumnezeu}. Acest lucru devine evident când examinăm contextul. Ellen White a fost instruită să ne avertizeze împotriva opiniilor trinitariene, nu să avem încredere în ele. Ea nu le susținea.


Although the illustrations Ellen White quoted were not from Dr. Kellogg, it seems that Kellogg's proponents, if not Kellogg himself, were defending him with William Boardman's sentiments. We do not have direct data to confirm this, but we do know that Dr. Kellogg raised \others{the theological side of questions of \textbf{the trinity and all that sort of things}.}[Interview, J. H. Kellogg, G. W. Amadon and A. C. Bourdeau, October 7th 1907 held at Kellogg’s residence][https://archive.org/details/KelloggVs.TheBrethrenHisLastInterviewAsAnAdventistoct71907/page/n37] The last three paragraphs in the heavenly trio manuscript \href{https://egwwritings.org/?ref=en_Ms21-1906&para=9754.1}{(Ms21-1906; 1906)} reveal the connection with Dr. Kellogg, which is another “smoking gun” of Dr. Kellogg's trinitarian stance.


Deși ilustrațiile pe care Ellen White le-a citat nu erau de la Dr. Kellogg, se pare că susținătorii lui Kellogg, dacă nu chiar Kellogg însuși, îl apărau cu opiniile lui William Boardman. Nu avem date directe pentru a confirma acest lucru, dar știm că Dr. Kellogg a ridicat \others{partea teologică a chestiunilor despre \textbf{trinitate și toate aceste lucruri}.}[Interviu, J. H. Kellogg, G. W. Amadon și A. C. Bourdeau, 7 octombrie 1907, ținut la reședința lui Kellogg][https://archive.org/details/KelloggVs.TheBrethrenHisLastInterviewAsAnAdventistoct71907/page/n37] Ultimele trei paragrafe din manuscrisul trio-ului ceresc \href{https://egwwritings.org/?ref=en_Ms21-1906&para=9754.1}{(Ms21-1906; 1906)} dezvăluie legătura cu Dr. Kellogg, care este o altă „dovadă zdrobitoare” a poziției trinitariene a Dr. Kellogg.


\egw{I write this because any moment my life may be ended. \textbf{Unless there is a breaking away from the influence that Satan has prepared, and a \underline{reviving of the testimonies that God has given, souls will perish in their delusion}. They will accept fallacy after fallacy and will thus keep up a disunion that will always exist until those who have been deceived take \underline{their stand on the right platform}}. All this higher education that is being planned will be extinguished; for it is spurious. The more simple the education of our workers, the less connection they have with the men whom God is not leading, the more will be accomplished. \textbf{Work will be done in the \underline{simplicity} of true godliness, and the old, old times will be back when, under the Holy Spirit’s guidance, thousands were converted in a day. When the truth in its simplicity is lived in every place, then God will work through His angels as He worked on the day of Pentecost, and hearts will be changed so decidedly that there will be a manifestation of the influence of genuine truth, as is represented in the descent of the Holy Spirit}.}[Ms21-1906.18; 1906][https://egwwritings.org/read?panels=p9754.25]


\egw{Scriu aceasta pentru că în orice moment viața mea poate fi încheiată. \textbf{Dacă nu va exista o rupere de influența pe care Satan a pregătit-o și o \underline{reînviere a mărturiilor pe care Dumnezeu le-a dat, sufletele vor pieri în amăgirea lor}. Ei vor accepta eroare după eroare și astfel vor menține o dezbinare care va exista întotdeauna până când cei care au fost înșelați își vor lua \underline{poziția pe platforma corectă}}. Toată această educație superioară care este planificată va fi stinsă; căci este falsă. Cu cât educația lucrătorilor noștri este mai simplă, cu cât au mai puțină legătură cu oamenii pe care Dumnezeu nu îi conduce, cu atât mai mult se va realiza. \textbf{Lucrarea va fi făcută în \underline{simplitatea} adevăratei evlavii, iar vremurile vechi, vechi vor reveni când, sub călăuzirea Duhului Sfânt, mii erau convertiți într-o zi. Când adevărul în simplitatea sa este trăit în fiecare loc, atunci Dumnezeu va lucra prin îngerii Săi așa cum a lucrat în ziua Cincizecimii, iar inimile vor fi schimbate atât de hotărât încât va exista o manifestare a influenței adevărului autentic, așa cum este reprezentat în coborârea Duhului Sfânt}.}[Ms21-1906.18; 1906][https://egwwritings.org/read?panels=p9754.25]


\egwnogap{The Holy Spirit never has and never will in the future divorce the medical missionary work from the gospel ministry. They cannot be divorced. Bound up with Jesus Christ, the ministry of the Word and the healing of the sick are one.}[Ms21-1906.19; 1906][https://egwwritings.org/read?panels=p9754.26]


\egwnogap{Duhul Sfânt nu a divorțat niciodată și nu va divorța niciodată în viitor lucrarea misionară medicală de slujirea Evangheliei. Ele nu pot fi divorțate. Legate de Isus Hristos, slujirea Cuvântului și vindecarea bolnavilor sunt una.}[Ms21-1906.19; 1906][https://egwwritings.org/read?panels=p9754.26]


\egwnogap{The fifty-eighth chapter of Isaiah contains instruction for today. \textbf{‘Cry aloud, spare not, lift up thy voice like a trumpet, and show My people their transgression, and the house of Jacob their sin.’ God does not accept \underline{Dr. Kellogg as His laborer}, unless he will now break with Satan}. The work would not have been hindered, as it has been for the past several years, \textbf{if Dr. Kellogg were a converted man. ‘Come,’ I call, ‘come ye out and be separate from him and his associates whom he has leavened.’ I am now giving the message God has given me, to give to all who claim to believe the truth, \underline{‘Come ye out from among them, and be separate},’ else their sin in justifying wrongs and framing deceits will continue to be the ruin of souls. We cannot afford to be on the wrong side. We cannot afford to cover the truth with scientific problems. We urge that decided changes be made and no more stumbling blocks be placed before the feet of the people of God}. Let every soul put on the gospel shoes. \textbf{Let every soul pray and work, placing their feet upon \underline{the foundation Christ laid} in giving His life for the life of the world}.}[Ms21-1906.20; 1906][https://egwwritings.org/read?panels=p9754.27]


\egwnogap{Capitolul cincizeci și opt din Isaia conține instrucțiuni pentru astăzi. \textbf{„Strigă în gura mare, nu te opri! Înalță-ți glasul ca o trâmbiță și vestește poporului Meu nelegiuirile lui, casei lui Iacov păcatele ei!” Dumnezeu nu îl acceptă pe \underline{Dr. Kellogg ca lucrător al Său}, decât dacă va rupe acum cu Satan}. Lucrarea nu ar fi fost împiedicată, așa cum a fost în ultimii câțiva ani, \textbf{dacă Dr. Kellogg ar fi fost un om convertit. „Veniți,” chem, „ieșiți și separați-vă de el și de asociații săi pe care i-a dospit.” Dau acum mesajul pe care Dumnezeu mi l-a dat, să-l dau tuturor celor care pretind că cred adevărul, \underline{„Ieșiți din mijlocul lor și despărțiți-vă”}, altfel păcatul lor de a justifica greșelile și de a urzi înșelăciuni va continua să fie ruina sufletelor. Nu ne putem permite să fim de partea greșită. Nu ne putem permite să acoperim adevărul cu probleme științifice. Îndemnăm să se facă schimbări hotărâte și să nu se mai pună pietre de poticnire înaintea picioarelor poporului lui Dumnezeu}. Fiecare suflet să îmbrace încălțămintea Evangheliei. \textbf{Fiecare suflet să se roage și să lucreze, punându-și picioarele pe \underline{temelia pe care a pus-o Hristos} dându-Și viața pentru viața lumii}.}[Ms21-1906.20; 1906][https://egwwritings.org/read?panels=p9754.27]


The heavenly trio quotation was part of Kellogg's controversy. This is evidence that Kellogg’s controversy included the Trinity doctrine. We are told to break \egwinline{away from the influence of Satan} and to revive the \egw{testimony that God has given} us, or else our souls will perish in delusions. These influences and delusions come from trinitarians such as \textit{William Boardman} and \textit{Dr. John H. Kellogg}. She is pointing us back to place our feet upon the foundation that was built by the Masterworker.\footnote{\href{https://egwwritings.org/?ref=en_SpTB02.54.2&para=417.276}{EGW, SpTB02 54.2; 1904}}


Citatul despre trio-ul ceresc a fost parte din controversa lui Kellogg. Aceasta este dovada că controversa lui Kellogg includea doctrina Trinității. Ni se spune să ne rupem \egwinline{de influența lui Satan} și să reînviem \egw{mărturia pe care Dumnezeu ne-a dat-o}, altfel sufletele noastre vor pieri în amăgiri. Aceste influențe și amăgiri vin de la trinitarieni precum \textit{William Boardman} și \textit{Dr. John H. Kellogg}. Ea ne îndreaptă înapoi să ne punem picioarele pe temelia care a fost construită de Meșterul Lucrător.\footnote{\href{https://egwwritings.org/?ref=en_SpTB02.54.2&para=417.276}{EGW, SpTB02 54.2; 1904}}


We hope that this context exposes the false narrative of Ellen White's endorsement of the Trinity doctrine, propagated by our Adventist scholars. Dr. Kellogg was in apostasy for stepping off from the foundation of our faith, and the Trinity doctrine was his justification. With such data in mind, one must ask: If the Trinity was true, and Ellen White endorsed it, and this “true” Trinity was mixed with Dr. Kellogg's error, we should expect her to separate the Trinity from error. But this is not what she did. Instead, she consistently pointed us back to the foundation of our faith, where we had a clear teaching on the presence and the \emcap{personality of God}. But for the case of Trinity, she faithfully bore the message from Heaven: “\textit{\textbf{I am instructed to say}, the sentiments of those who are searching for \textbf{trinitarian ideas are not to be trusted}}.”


Sperăm că acest context expune narațiunea falsă despre susținerea de către Ellen White a doctrinei Trinității, propagată de învățații noștri adventiști. Dr. Kellogg era în apostazie pentru că a pășit în afara temeliei credinței noastre, iar doctrina Trinității era justificarea sa. Cu astfel de date în minte, cineva trebuie să întrebe: Dacă Trinitatea era adevărată, și Ellen White o susținea, și această Trinitate „adevărată” era amestecată cu eroarea Dr. Kellogg, ne-am aștepta ca ea să separe Trinitatea de eroare. Dar aceasta nu este ceea ce a făcut ea. În schimb, ea ne-a îndreptat constant înapoi la temelia credinței noastre, unde aveam o învățătură clară despre prezența și \emcap{personalitatea lui Dumnezeu}. Dar în cazul Trinității, ea a purtat cu credincioșie mesajul din Cer: „\textit{\textbf{Sunt instruită să spun}, opiniile celor care caută \textbf{idei trinitariene nu sunt de încredere}}.”


\begin{titledpoem}
\stanza{
    In a realm divine, where truths unfold, \\
    A message of clarity, brave and bold. \\
    Ellen spoke, her voice clear and bright, \\
    Revealing the depths of heavenly light.
}

\stanza{
    Misunderstood by many who read, \\
    Her words hold a truth that all must heed. \\
    Not a triune God, but a trio divine, \\
    Three living persons, distinct in line.
}

\stanza{
    The Father, not formless, but full and bright, \\
    Invisible to mortal, yet real in might. \\
    He is the fullness, a presence complete, \\
    Invisible to sight, yet real and concrete.
}

\stanza{
    The Son, God's fullness, manifest and near, \\
    In Him, the divine becomes crystal clear. \\
    The personality of God, seen in His face, \\
    In Christ, we witness God's grace.
}

\stanza{
    The Holy Spirit, in fullness resides, \\
    Within the Godhead, where mystery abides. \\
    Father and Son, with bodies they stand, \\
    Holy Spirit, as spirit, spreads through the land.
}

\stanza{
    Distinct and clear, their roles unfold, \\
    Father and Son, in form behold. \\
    Yet everywhere present, the Spirit we find, \\
    Their representative, in heart and mind.
}

\stanza{
    Spiritual and bodily, presence defined, \\
    Understanding this truth, enlightenment kind. \\
    Ellen's message, profound and bright, \\
    Guides us through the heavenly light.
}

\stanza{
    Ellen's words, in context found, \\
    Show a truth profound and sound. \\
    Not the trinity she did embrace, \\
    But a trio divine, each in their place.
}

\stanza{
    In "heavenly Trio," a contrast is drawn, \\
    Between trinity's doctrine and faith's true dawn. \\
    The pillar stands firm, the personality of God, \\
    Distinct from trinity, where truth is awed.
}
\end{titledpoem}