% \qrchapter{https://forgottenpillar.com/rsc/en-fp-chapter10}{Is God a person? - by John N. Loughborough}


\qrchapter{https://forgottenpillar.com/rsc/ro-fp-chapter10}{Este Dumnezeu o persoană? - de John N. Loughborough}


One of the earliest articles on the \emcap{personality of God} is Loughborough’s article “\textit{Is God a person?}” where he discusses the \emcap{personality of God} and His presence. It is important to remember the definition of ‘personality’ according to the Merriam-Webster dictionary: “\textit{the quality or state of being a person}”\footnote{\href{https://www.merriam-webster.com/dictionary/personality}{Merriam-Webster Dictionary - ‘\textit{personality}’}}. We will look carefully at how Loughborough sees the quality or state of God being a person.


Unul dintre cele mai vechi articole despre \emcap{personalitatea lui Dumnezeu} este articolul lui Loughborough „\textit{Este Dumnezeu o persoană?}” în care el discută \emcap{personalitatea lui Dumnezeu} și prezența Sa. Este important să ne amintim definiția cuvântului „personalitate” conform dicționarului Merriam-Webster: „\textit{caracteristica sau starea prin care cineva este definit ca persoană}”\footnote{\href{https://www.merriam-webster.com/dictionary/personality}{Dicționarul Merriam-Webster - „\textit{personalitate}”}}. Vom examina cu atenție cum vede Loughborough caracteristica sau starea prin care Dumnezeu este definit ca persoană.


\begin{figure}[hp]
    \centering
    \includegraphics[width=1\linewidth]{images/john-n-loughborough.jpg}
    \caption*{John Norton Loughborough (1832-1924)}
    \label{fig:john-n-loughborough}
\end{figure}


\begin{figure}[hp]
    \centering
    \includegraphics[width=1\linewidth]{images/john-n-loughborough.jpg}
    \caption*{John Norton Loughborough (1832-1924)}
    \label{fig:john-n-loughborough}
\end{figure}


\others{Whatever may be the truth in this matter, it certainly cannot be wrong for us to examine what the Word says respecting it. \textbf{Many there are that would refrain from the investigation of unpopular truths because the cry of heresy is raised against them}. We shall not consider ourselves subjects of the appellation, \textbf{neither are we prying into the secrets of the Almighty, as we pursue the investigation of this matter}. The Bible certainly contains testimony upon this point, and we again repeat, ‘\textbf{Things which are revealed belong to us}.’ We inquire then, What saith the Scripture?}


\others{Orice ar fi adevărul în această materie, cu siguranță nu poate fi greșit pentru noi să examinăm ceea ce spune Cuvântul în privința acesteia. \textbf{Sunt mulți care ar refuza investigarea adevărurilor nepopulare pentru că se ridică strigătul de erezie împotriva lor}. Nu ne vom considera pe noi înșine ca fiind supuși acestei denumiri, \textbf{nici nu cercetăm secretele Atotputernicului, pe măsură ce continuăm investigarea acestei chestiuni}. Biblia cu siguranță conține mărturie asupra acestui punct, și repetăm din nou: „\textbf{Lucrurile care sunt revelate ne aparțin}.” Întrebăm atunci: Ce spune Scriptura?}


\othersnogap{\textbf{The very testimony we have been examining in regard to man’s being formed of the dust in \underline{the image of God}, proves conclusively that \underline{God has a form}, although the sentiment is contrary to what we have been taught, while children, from the catechism}:}


\othersnogap{\textbf{Chiar mărturia pe care am examinat-o cu privire la faptul că omul a fost format din praf în \underline{chipul lui Dumnezeu}, dovedește în mod concludent că \underline{Dumnezeu are o formă}, deși opinia este contrară cu ceea ce ni s-a învățat, în copilărie, din catehism}:}


\othersnogap{Question. ‘What is God?’}


\othersnogap{\textbf{Întrebare. „Ce este Dumnezeu?”}}


\othersnogap{Answer. ‘An infinite and eternal spirit; one that always was and always will be.’}


\othersnogap{\textbf{Răspuns. „Un spirit infinit și etern; unul care a fost întotdeauna și va fi întotdeauna.”}}


\othersnogap{Q. ‘Where is God?’}


\othersnogap{\textbf{Î. „Unde este Dumnezeu?”}}


\othersnogap{A. ‘Everywhere.’}


\othersnogap{\textbf{R. „Peste tot.”}}


\othersnogap{\textbf{But we inquire, \underline{Is not God in one place more than another}?} Oh no, say you: \textbf{the Bible says \underline{he is a spirit}, and if so he must be \underline{everywhere alike}}. Well, if when man dies his spirit goes to God, it must go everywhere. \textbf{But the Bible certainly represents God as located in heaven. ‘For he hath looked down from the height of his sanctuary: from heaven did the Lord behold the earth.’ Psalm 102:19}. \textbf{Then certainly heaven cannot be everywhere, for God is represented as looking down from it. ‘\underline{Elijah went up} by a whirlwind \underline{into heaven}.’ 2 Kings 2:11}. \textbf{But, says one, does not the Bible represent God \underline{as everywhere present}?} Psalm 139:8, 9, 10. ‘If I ascend up into heaven, \textbf{thou art there}: if I make my bed in hell, \textbf{behold, thou art there}; if I take the wings of the morning, and dwell in the uttermost parts of the sea,\textbf{ even there shall thy hand lead me}, and thy right hand shall hold me.’}


\othersnogap{\textbf{Dar noi întrebăm: \underline{Nu este Dumnezeu într-un loc mai mult decât în altul}?} Oh nu, spuneți voi: \textbf{Biblia spune că \underline{El este un spirit}, și dacă este așa, trebuie să fie \underline{peste tot la fel}}. Ei bine, dacă atunci când omul moare spiritul lui merge la Dumnezeu, trebuie să meargă peste tot. \textbf{Dar Biblia cu siguranță îl reprezintă pe Dumnezeu ca fiind localizat în cer. „Căci S-a uitat din înălțimea sanctuarului Său; din cer a privit Domnul pământul.” Psalmul 102:19}. \textbf{Atunci cu siguranță cerul nu poate fi peste tot, pentru că Dumnezeu este reprezentat ca privind în jos din el. „\underline{Ilie s-a suit} cu un vârtej \underline{în cer}.” 2 Împărați 2:11}. \textbf{Dar, spune cineva, nu reprezintă Biblia pe Dumnezeu \underline{ca fiind prezent peste tot}?} Psalmul 139:8, 9, 10. „Dacă mă sui în cer, \textbf{Tu ești acolo}: dacă-mi fac patul în iad, \textbf{iată, Tu ești acolo}; dacă iau aripile zorilor și locuiesc la marginile mării, \textbf{chiar și acolo mâna Ta mă va conduce}, și dreapta Ta mă va ține.”}


\othersnogap{We reply, \textbf{the subject is introduced in verse 7, as follows}: ‘\textbf{\underline{Whither shall I go from thy Spirit}?} \textbf{or whither shall I flee from \underline{thy presence}?}’ \textbf{The Spirit is \underline{God’s representative}}. \textbf{His power is manifested wherever he listeth, through the agency of his Spirit}. Christ, when giving the commission to the disciples, says, ‘Go ye into all the world, and preach the gospel to every creature, and lo! \textbf{I am with you alway, even unto the end of the world}.’ Now, no one would contend that Christ had been on the earth personally ever since the disciples commenced to fulfill this commission. \textbf{But his Spirit has been on the earth; the Comforter that he promised to send.} \textbf{So in the same manner God manifests himself \underline{by his Spirit} which is also the power through which he works}. ‘But if \textbf{the Spirit of him} that raised up Jesus from the dead dwell in you, \textbf{he that raised up Christ} from the dead shall also quicken your mortal bodies \textbf{\underline{by his Spirit} that dwelleth in you}.’ Romans 8:11. \textbf{\underline{Here is a plain distinction made between the Spirit, and God that raises the dead by that Spirit}}. \textbf{If the living God is a Spirit in the strictest sense of the term, and at the same time is in possession of a Spirit, then we have at once the novel idea of the Spirit of a Spirit, something it will take at least a Spiritualist to explain}.}[The Adventist Review and Sabbath Herald, September 18, 1855][https://documents.adventistarchives.org/Periodicals/RH/RH18550918-V07-06.pdf]


\othersnogap{Răspundem, \textbf{subiectul este introdus în versetul 7, după cum urmează}: ‘\textbf{\underline{Încotro să mă duc de la Duhul Tău}?} \textbf{Sau încotro să fug de la \underline{prezența Ta}?}’ \textbf{Duhul este \underline{reprezentantul lui Dumnezeu}}. \textbf{Puterea Lui se manifestă oriunde vrea El, prin intermediul Duhului Său}. Hristos, când a dat discipolilor comanda, spune: ‘Mergeți în toată lumea și propovăduiți evanghelia la fiecare făptură, și iată! \textbf{Eu sunt cu voi în toate zilele, până la sfârșitul lumii}.’ Acum, nimeni nu ar susține că Hristos a fost pe pământ personal de când discipolii au început să îndeplinească această poruncă. \textbf{Dar Duhul Lui a fost pe pământ; Mângâietorul pe care a promis să-l trimită.} \textbf{Deci în același mod Dumnezeu se manifestă \underline{prin Duhul Său} care este și puterea prin care El lucrează}. ‘Dar dacă \textbf{Duhul Celui} care L-a înviat pe Isus din morți locuiește în voi, \textbf{Cel care L-a înviat pe Hristos} din morți va învia și trupurile voastre muritor \textbf{\underline{prin Duhul Său} care locuiește în voi}.’ Romani 8:11. \textbf{\underline{Aici se face o distincție clară între Duh și Dumnezeu care înviază pe morți prin acel Duh}}. \textbf{Dacă Dumnezeul viu este un Duh în sensul cel mai strict al cuvântului, și în același timp posedă un Duh, atunci avem imediat ideea nouă a Duhului unui Duh, ceva ce va trebui cel puțin unui spiritualist să explice}.}[The Adventist Review and Sabbath Herald, September 18, 1855][https://documents.adventistarchives.org/Periodicals/RH/RH18550918-V07-06.pdf]


Allow us to make a short comment. We hope you recognize the specific topic being discussed here. The subject is the first point of the \emcap{Fundamental Principles} and the assertion is that God does have a form, for man is made in the image of God. Such understanding of God’s personality precludes the idea that God is everywhere present. Brother Loughborough gave the biblical reasons for God's omnipresence, together with the sentiment that “\textit{God is in one place more than another}”. God is everywhere present by His representative, the Holy Spirit, just as it is written in the first point of the \emcap{Fundamental Principles}. Further in this discussion, we will read that God is a spiritual being and possesses a tangible, material body, in contrast to the idea that He is purely a spirit.


Permiteți-ne să facem o scurtă observație. Sperăm că recunoașteți subiectul specific discutat aici. Subiectul este primul punct al \emcap{Principiilor Fundamentale} și afirmația este că Dumnezeu într-adevăr are o formă, căci omul este făcut după chipul lui Dumnezeu. O asemenea înțelegere a personalității lui Dumnezeu exclude ideea că Dumnezeu este prezent peste tot. Fratele Loughborough a dat motivele biblice pentru omniprezența lui Dumnezeu, împreună cu opinia că „\textit{Dumnezeu este într-un loc mai mult decât în altul}”. Dumnezeu este prezent peste tot prin reprezentantul Său, Duhul Sfânt, exact cum este scris în primul punct al \emcap{Principiilor Fundamentale}. Mai departe în această discuție, vom citi că Dumnezeu este o ființă spirituală și posedă un trup tangibil, material, în contrast cu ideea că El este pur și simplu un duh.


\others{There is at least one impassable difficulty in the way of \textbf{those who believe \underline{God is immaterial}, and heaven is not a literal, \underline{located place}: they are obliged to admit that \underline{Jesus is there bodily, a literal person}}; the same Jesus that was crucified, dead, and buried, was raised from the dead, \textbf{ascended up to heaven}, and is now \textbf{at the right hand of God}. \textbf{Jesus was possessed of flesh and bones after his resurrection}. Luke 24:39. ‘\textbf{Behold my hands and my feet, that it is I, myself; handle me, and see; \underline{for a spirit hath not flesh and bones as ye see me have}}.’ \textbf{If Jesus is there in heaven with a literal body of flesh and bones, may not heaven after all be a literal place, a habitation for a literal God, a literal Saviour, literal angels, and resurrected immortal saints?} \textbf{\underline{Oh no, says one, ‘God is a Spirit.’}} So Christ said to the woman of Samaria at the well. \textbf{It does not necessarily follow because God is a Spirit, \underline{that he has no body}}. In John 3:6, Christ says to Nicodemus, ‘\textbf{That which is born of the Spirit is spirit}.’ \textbf{If that which is born of the Spirit is spirit, then on the same principle, that which has a spiritual nature is spirit. God is \underline{a spirit being}, his nature is spirit, he is not of a mortal nature; }\textbf{\underline{but this does not exclude the idea of his having a body}}. David says, [Psalm 114:4,] ‘Who maketh \textbf{his angels spirits};’ yet \textbf{\underline{angels have bodies}}. Angels appeared to both Abraham and Lot, and ate with them. \textbf{We see the idea that angels are spirits, does not prove that they are not literal beings}.}


\others{Există cel puțin o dificultate de netrecut în calea \textbf{celor care cred că \underline{Dumnezeu este imaterial}, și cerul nu este un loc literal, \underline{localizat}: sunt obligați să admită că \underline{Isus este acolo în trup, o persoană literală}}; același Isus care a fost răstignit, mort și îngropat, a fost înviat din morți, \textbf{s-a suit la cer}, și este acum \textbf{la dreapta lui Dumnezeu}. \textbf{Isus avea carne și oase după învierea Lui}. Luca 24:39. ‘\textbf{Priviți mâinile și picioarele mele, că eu sunt; atingeți-mă și vedeți; \underline{căci un duh nu are carne și oase cum vedeți că am eu}}.’ \textbf{Dacă Isus este acolo în cer cu un trup literal din carne și oase, nu ar putea cerul totuși să fie un loc literal, o locuință pentru un Dumnezeu literal, un Mântuitor literal, îngeri literali și sfinți nemuritori înviați?} \textbf{\underline{Oh nu, spune cineva, ‘Dumnezeu este un Duh.’}} Așa a spus și Hristos femeii din Samaria la fântână. \textbf{Nu rezultă neapărat din faptul că Dumnezeu este un Duh, \underline{că nu are trup}}. În Ioan 3:6, Hristos spune lui Nicodim, ‘\textbf{Ceea ce s-a născut din Duh este duh}.’ \textbf{Dacă ceea ce s-a născut din Duh este duh, atunci după același principiu, ceea ce are o natură spirituală este duh. Dumnezeu este \underline{o ființă spirituală}, natura Lui este duh, nu este de natură muritor; }\textbf{\underline{dar aceasta nu exclude ideea că are un trup}}. David spune, [Psalmul 114:4,] ‘Care face \textbf{pe îngerii Săi duhuri};’ totuși \textbf{\underline{îngerii au trupuri}}. Îngerii s-au arătat atât lui Avraam cât și lui Lot, și au mâncat cu ei. \textbf{Vedem că ideea că îngerii sunt duhuri, nu dovedește că nu sunt ființe literale}.}


\othersnogap{It is inferred because the Bible says that God is a Spirit, that he is not a person. An inference should not be made the basis for an argument. Great Scripture truths are plainly stated, and it will not do for us to found a doctrine on inferences, \textbf{contrary to positive statements in the word of God}. If the Scripture states in positive \textbf{terms that God is a person, it will not answer for us to draw an inference from the text which says ‘God is a Spirit,’ \underline{that he has no body}}.}


\othersnogap{Se deduce din faptul că Biblia spune că Dumnezeu este un Duh, că nu este o persoană. O deducție nu ar trebui să fie baza unui argument. Adevărurile mari din Scriptură sunt clar enunțate, și nu ne convine să întemeiez o doctrină pe deducții, \textbf{contrare afirmațiilor pozitive din cuvântul lui Dumnezeu}. Dacă Scriptura afirmă în termeni pozitivi \textbf{că Dumnezeu este o persoană, nu ne va conveni să tragem o deducție din textul care spune ‘Dumnezeu este un Duh,’ \underline{că nu are trup}}.}


\othersnogap{We will now present a few texts \textbf{which prove that God is a person}. Exodus 33:18, 23. ‘And he (Moses) said, I beseech thee shew me thy glory.’ Verse 20. ‘And he said, \textbf{Thou canst not see \underline{my face}, for there shall no man see me and live}.’ Verses 21-23. ‘And the Lord said, Behold there is a place by me, and thou shalt stand upon a rock: and it shall come to pass while my glory passeth by, that I will put thee in a cleft of the rock; and \textbf{will cover thee with \underline{my hand} while I pass by}; and I will take away \textbf{mine hand}, and thou shalt \textbf{see my \underline{back parts}}; but \textbf{\underline{my face} shall not be seen.’} \textbf{If God is \underline{an immaterial Spirit}, then Moses could not see him; for we are told a spirit cannot be seen by natural eyes}. \textbf{There would then be no propriety for God to say he would put his hand over Moses’ face while he passed by, (seemingly to prevent him from seeing his face,) for he could not see him}. Neither do we conceive how an immaterial hand could obstruct the rays of light from passing to Moses’ eyes. \textbf{But if the position be true \underline{that God is immaterial}, and cannot be seen by the natural eye, the text above is all superfluous}. \textbf{What sense is there in saying God put his hand over Moses’ face, to prevent him from seeing that which could not be seen}.}


\othersnogap{Vom prezenta acum câteva texte \textbf{care dovedesc că Dumnezeu este o persoană}. Exod 33:18, 23. ‘Și el (Moise) a zis: Te rog, arată-mi gloria Ta.’ Versetul 20. ‘Și El a zis, \textbf{Nu poți vedea \underline{fața Mea}, căci nici un om nu poate să mă vadă și să trăiască}.’ Versetele 21-23. ‘Și Domnul a zis: Iată, este un loc lângă Mine, și tu vei sta pe o stâncă: și se va întâmpla, pe când va trece gloria Mea, că te voi pune într-o crăpătură a stâncii; și \textbf{te voi acoperi cu \underline{mâna Mea} pe când trec}; și voi lua \textbf{mâna Mea}, și vei \textbf{vedea \underline{spatele Meu}}; dar \textbf{\underline{fața Mea} nu va fi văzută}.’ \textbf{Dacă Dumnezeu este \underline{un Duh imaterial}, atunci Moise nu ar fi putut să-L vadă; căci ni se spune că un duh nu poate fi văzut cu ochii naturali}. \textbf{Nu ar fi fost nici o potrivire pentru Dumnezeu să spună că-și va pune mâna peste fața lui Moise pe când trece, (aparent pentru a-l împiedica să-și vadă fața,) căci nu ar fi putut să-L vadă}. Nici nu concepem cum o mână imaterială ar putea să oprească razele de lumină să ajungă la ochii lui Moise. \textbf{Dar dacă poziția ar fi adevărată \underline{că Dumnezeu este imaterial}, și nu poate fi văzut cu ochiul natural, textul de mai sus este cu totul de prisos}. \textbf{Ce sens are să spunem că Dumnezeu și-a pus mâna peste fața lui Moise, pentru a-l împiedica să vadă ceea ce nu ar putea fi văzut}.}


\othersnogap{Says one, I see we cannot harmonize the matter any other way, than that there was a literal body seen by Moses; but that was not God’s own body, \textbf{it was a body he took that he might show himself to Moses}. \textbf{Moses could form no just conceptions of God unless he assumed a form.} \textbf{So God took a body}. This throws a worse coloring on the matter than the first position; \textbf{for it charges God with deception; telling Moses he should see him, when in fact Moses according to this testimony did not see God, but another body}. A person must be given to doubt almost beyond recovery, that would attempt thus to mystify, and do away the force of this testimony.}[Ibid.][https://documents.adventistarchives.org/Periodicals/RH/RH18550918-V07-06.pdf]


\othersnogap{Spune cineva, văd că nu putem armoniza lucrurile în alt fel, decât că a fost văzut un trup literal de Moise; dar acela nu era propriul trup al lui Dumnezeu, \textbf{era un trup pe care l-a luat pentru a se arăta lui Moise}. \textbf{Moise nu ar fi putut forma concepții corecte despre Dumnezeu decât dacă ar fi asumat o formă.} \textbf{Deci Dumnezeu a luat un trup}. Aceasta dă o colorare și mai rea decât prima poziție; \textbf{căci o acuză pe Dumnezeu de înșelăciune; spunând lui Moise că ar trebui să-L vadă, când de fapt Moise conform acestei mărturii nu a văzut pe Dumnezeu, ci alt trup}. O persoană trebuie să fie dată la îndoială aproape dincolo de recuperare, care ar încerca să mistifice în acest fel și să anuleze forța acestei mărturii.}[Ibid.][https://documents.adventistarchives.org/Periodicals/RH/RH18550918-V07-06.pdf]


Do you recognize that Brother Loughborough is tackling the sentiment that Dr. Kellogg would present in the Living Temple 48 years later? Dr. Kellogg said that it is true that God presented Himself in a\others{\textbf{\underline{particular form or place}}}[Dr. John H. Kellogg, The Living Temple, p.31.][https://archive.org/details/J.H.Kellogg.TheLivingTemple1903/page/n31/] because \others{there must be something more \textbf{tangible}, more \textbf{\underline{restricted}}, upon which to center the mind in worship}[bid, p.30][https://archive.org/details/J.H.Kellogg.TheLivingTemple1903/page/n30/], but that He is, in reality,\others{\textbf{far beyond our comprehension \underline{as are the bounds of space and time}}}[Ibid, p.33][https://archive.org/details/J.H.Kellogg.TheLivingTemple1903/page/n33/]. Brother Loughborough reasonably objected to the idea that God is only manifesting Himself to man as a definite Being, but in reality, is not what He presents Himself to be. Such a claim\others{charges God with deception}. Brother Loughborough continues with the affirmative, Biblical testimony that God is a material being.


Recunoașteți că Fratele Loughborough abordează opinia pe care Dr. Kellogg ar prezenta-o în Templul viu 48 de ani mai târziu? Dr. Kellogg a spus că este adevărat că Dumnezeu S-a prezentat într-o\others{\textbf{\underline{formă sau loc particular}}}[Dr. John H. Kellogg, The Living Temple, p.31.][https://archive.org/details/J.H.Kellogg.TheLivingTemple1903/page/n31/] pentru că \others{trebuie să existe ceva mai mult \textbf{tangibil}, mai mult \underline{\textbf{restrâns}}, pe care să-și concentreze mintea în închinare}[bid, p.30][https://archive.org/details/J.H.Kellogg.TheLivingTemple1903/page/n30/], dar că El este, în realitate,\others{\textbf{departe dincolo de înțelegerea noastră \underline{ca și limitele spațiului și timpului}}}[Ibid, p.33][https://archive.org/details/J.H.Kellogg.TheLivingTemple1903/page/n33/]. Fratele Loughborough s-a opus în mod rezonabil ideii că Dumnezeu doar se manifestă omului ca o Ființă definită, dar în realitate nu este ceea ce se prezintă a fi. O asemenea pretenție\others{o acuză pe Dumnezeu de înșelăciune}. Fratele Loughborough continuă cu mărturie biblică afirmativă că Dumnezeu este o ființă materială.


\others{Exodus 24:9. ‘Then went up Moses and Aaron, Nadab and Abihu, and seventy of the elders of Israel: \textbf{and they saw the God of Israel}: and there was under \textbf{his feet} as it were a paved work of a sapphire stone, and as it were the body of heaven in its clearness.’ They were permitted to \textbf{see his feet}, but no \textbf{man can see his face and live}. \textbf{No \underline{mortal eye} can bear the dazzling brightness of that glory of the face of God}. It far exceeds the light of the sun. For the prophet says, ‘The light of the moon shall be as the light of the sun, and the light of the sun shall be \textbf{seven fold}, as the light of seven days, in the day that the Lord bindeth up the breach of his people, and healeth the stroke of their wound.’ Isaiah 30:26. Notwithstanding this seven-fold light that is then to shine, the prophet speaking of the scene says, ‘Then the moon shall be confounded, and the sun ashamed, when the Lord of hosts shall reign in mount Zion, and in Jerusalem, and before his ancients gloriously.’ Isaiah 24:23. The testimony of John is, [Revelation 21:23,] ‘And the city had no need of the sun, neither of the moon, to shine in it: for \textbf{the glory of God did lighten it,} and the Lamb is the light thereof.’}


\others{Exod 24:9. ‘Atunci s-au suit Moise și Aaron, Nadab și Abihu, și șaptezeci din bătrânii lui Israel: \textbf{și au văzut pe Dumnezeul lui Israel}: și era sub \textbf{picioarele Lui} ceva ca o lucrare pavată din piatră de safir, și ca trupul cerului în limpezimea lui.’ Le-a fost permis să \textbf{vadă picioarele Lui}, dar nici un \textbf{om nu poate vedea fața Lui și să trăiască}. \textbf{Nici un \underline{ochi muritor} nu poate suporta strălucirea orbitor a acelei glori a feței lui Dumnezeu}. Depășește cu mult lumina soarelui. Căci profetul spune: ‘Lumina lunii va fi ca lumina soarelui, și lumina soarelui va fi \textbf{de șapte ori mai mare}, ca lumina a șapte zile, în ziua în care Domnul va lega rana poporului Său, și va vindeca rănile lui.’ Isaia 30:26. Cu toate că această lumină de șapte ori mai mare va străluci atunci, profetul vorbind despre scena aceea spune: ‘Atunci luna va fi confundată, și soarele va fi rușinat, când Domnul oștirilor va domni pe muntele Sion, și în Ierusalim, și înaintea bătrânilor Săi cu glorie.’ Isaia 24:23. Mărturie lui Ioan este, [Apocalipsa 21:23,] ‘Și cetatea nu avea nevoie de soare, nici de lună, ca să lumineze în ea: căci \textbf{gloria lui Dumnezeu a luminat-o,} și Mielul este lumina ei.’}


\othersnogap{\textbf{Infidels claim that there is a contradiction in the testimony of Moses, because he said, he talked with God face to face}. \textbf{We reply, there was a cloud between them}, but God told Moses, ‘\textbf{No man shall see me and live}.’ The Testimony of the New Testament is in harmony with that of the Old upon this subject. ‘Follow peace with all men, and holiness without which \textbf{no man shall see the Lord}.’ Hebrews 12:14. \textbf{Who with \underline{mortal eyes} could behold a light that far outshines seven fold the brightness of the sun?} Surely none but the holy can behold him, \textbf{none but immortal eyes} could bear that radiant glory. Although the Word says we cannot see God now and live, the promise is, that the \textbf{pure in heart shall see him}. Matthew 5:3. ‘Blessed are the pure in heart, \textbf{for they shall see God}.’ Revelation 22:4. ‘And \textbf{they shall see his face}, and his name shall be in their foreheads.’}


\othersnogap{\textbf{Necredincioșii susțin că există o contradicție în mărturie lui Moise, pentru că el a spus că a vorbit cu Dumnezeu față în față}. \textbf{Răspundem, a fost un nor între ei}, dar Dumnezeu i-a spus lui Moise, ‘\textbf{Nici un om nu mă va vedea și va trăi}.’ Mărturie Noului Testament este în armonie cu cea a Vechiului Testament asupra acestui subiect. ‘Urmăriți pacea cu toți oamenii, și sfințenia fără care \textbf{nici un om nu va vedea pe Domnul}.’ Evrei 12:14. \textbf{Cine cu \underline{ochii muritori} ar putea privi o lumină care depășește cu mult strălucirea de șapte ori a soarelui?} Sigur că nimeni decât cei sfinți pot să-L vadă, \textbf{nimeni decât ochii nemuritori} ar putea suporta acea glorie strălucitoare. Deși Cuvântul spune că nu putem vedea pe Dumnezeu acum și să trăim, promisiunea este, că \textbf{cei curi de inimă vor vedea pe Dumnezeu}. Matei 5:3. ‘Fericiți cei curi de inimă, \textbf{căci ei vor vedea pe Dumnezeu}.’ Apocalipsa 22:4. ‘Și \textbf{vor vedea fața Lui}, și numele Lui va fi pe frunțile lor.’}


\othersnogap{Paul, [Colossians 1:15,] speaking of Christ, says, ‘Who is the image of \textbf{the invisible God}, the first born of every creature.’ Here Christ is said to be ‘\textbf{the image of the invisible God}.’ We have already shown, that\textbf{ Christ has a body composed of substance, flesh and bones; and he is said to be}, ‘\textbf{the image of the invisible God}.’ Well, says one, we admit his divine nature is in the image of God. If by his divine nature you mean the part that existed in glory with the Father before the world was, we reply, that which was in the beginning with God, (the Word,) \textbf{was made flesh, not came into flesh}, or as some state, \textbf{clothed upon with a human nature, but made flesh}. But says another, \textbf{God is said to be invisible}. \textbf{Because he is invisible now, it does not prove that he never will be seen}. The Word says, ‘The pure in heart \textbf{shall see him}’. Willing faith says, Amen.}


\othersnogap{Pavel, [Coloseni 1:15,] vorbind despre Hristos, spune: ‘Care este \textbf{chipul Dumnezeului invizibil}, întâiul născut din toată făptura.’ Aici Hristos este numit ‘\textbf{chipul Dumnezeului invizibil}.’ Am arătat deja, că\textbf{ Hristos are un trup compus din substanță, carne și oase; și El este numit}, ‘\textbf{chipul Dumnezeului invizibil}.’ Ei bine, spune cineva, admitem că natura divină a Lui este în chipul lui Dumnezeu. Dacă prin natura divină a Lui înțelegi partea care a existat în glorie cu Tatăl înainte ca lumea să fie, răspundem, ceea ce era la început cu Dumnezeu, (Cuvântul,) \textbf{s-a făcut carne, nu a venit în carne}, sau cum spun unii, \textbf{s-a îmbrăcat cu o natură umană, ci s-a făcut carne}. Dar spune altul, \textbf{Dumnezeu este spus a fi invizibil}. \textbf{Din faptul că este invizibil acum, nu rezultă că nu va fi văzut niciodată}. Cuvântul spune: ‘Cei curi de inimă \textbf{îl vor vedea}’. Credința dorind spune, Amin.}


\othersnogap{Paul’s testimony in Philippians 2:5, 6, shows plainly what may be understood by the statement, that Christ is the image of God. ‘Let this mind be in you which was in Christ Jesus: who \textbf{being in the form of God}, thought it not robbery to \textbf{be equal with God}.’ \textbf{How can Christ be said to be in the form of God, if God has no form?} Romans 8:3. ‘God sending his own Son in the likeness of sinful flesh.’ \textbf{Christ is in the form of God, and in the form of men. This at once reveals to us the form of God}.}


\othersnogap{Mărturisirea lui Paul în Filipeni 2:5, 6, arată clar ce se poate înțelege prin afirmația că Hristos este chipul lui Dumnezeu. „Gândirea aceasta să fie în voi, care a fost și în Hristos Iisus: care, \textbf{fiind în forma lui Dumnezeu}, nu a socotit că ar fi o jaf să fie \textbf{egal cu Dumnezeu}.” \textbf{Cum poate fi spus că Hristos este în forma lui Dumnezeu, dacă Dumnezeu nu are formă?} Romani 8:3. „Dumnezeu și-a trimis propriul Fiu în asemănarea cărnii păcătoase.” \textbf{Hristos este în forma lui Dumnezeu și în forma oamenilor. Aceasta ne dezvăluie imediat forma lui Dumnezeu}.}


\othersnogap{\textbf{\underline{Daniel speaking of God, calls him the Ancient of days}}. Daniel 7:9. ‘And the Ancient of days did sit, \textbf{whose garment was white as snow}, and \textbf{the hair of his head} like the pure wool.’ \textbf{This personage is said to have a head, and hair; this certainly could not be said of him} \textbf{\underline{if he was immaterial and had no form}}. \textbf{But Paul’s testimony in \underline{Hebrews 1:3}, ought to settle every candid mind in \underline{regard to the personality of God}}. Speaking of Christ, he says, ‘Who being the brightness of his glory, \textbf{and the express image of his (the \underline{Father’s person})}.’ \textbf{Here then it is plainly stated \underline{God has a person}. Christ is the express image of it.} Then we can understand Christ where he says, ‘\textbf{He that hath seen me, hath seen the Father}.’ John 14:19. \textbf{He could not have meant, that he was his own father; for when he prayed he addressed his Father as another person who had sent him into the world}. He styled himself \textbf{the Son of God}. \textbf{Then he could not be the Father of which he was the son}. When he says, ‘He that hath seen me hath seen the Father,’ he must mean, that as \textbf{he was the express image of the Father’s person, those who saw him saw the likeness of the Father in him}.}[The Adventist Review and Sabbath Herald, September 18, 1855][https://documents.adventistarchives.org/Periodicals/RH/RH18550918-V07-06.pdf]


\othersnogap{\textbf{\underline{Daniel, vorbind despre Dumnezeu, îl numește pe Cel Vechi de zile}}. Daniel 7:9. „Și Cel Vechi de zile S-a așezat, \textbf{al cărui veșmânt era alb ca zăpada}, și \textbf{părul capului Său} ca lâna curată.” \textbf{Se spune că această persoană are cap și păr; aceasta cu siguranță nu ar putea fi spus despre el} \textbf{\underline{dacă ar fi fost imaterial și nu ar fi avut formă}}. \textbf{Dar mărturisirea lui Paul în \underline{Evrei 1:3}, ar trebui să liniștească orice minte cinstită în \underline{privința personalității lui Dumnezeu}}. Vorbind despre Hristos, el spune: „Care este strălucirea gloriei Sale, și \textbf{Întipărirea persoanei Lui (a \underline{Tatălui})}.” \textbf{Aici atunci se afirmă clar \underline{că Dumnezeu are o persoană}. Hristos este Întipărirea ei.} Atunci putem înțelege Hristos când spune: „\textbf{Cine M-a văzut pe Mine, a văzut pe Tatăl}.” Ioan 14:19. \textbf{El nu ar fi putut să însemne că era propriul său tată; căci când se ruga, se adresa Tatălui Său ca unei alte persoane care l-a trimis în lume}. El se numea pe sine \textbf{Fiul lui Dumnezeu}. \textbf{Atunci el nu putea fi Tatăl din care era fiul}. Când spune: „Cine M-a văzut pe Mine a văzut pe Tatăl,” el trebuie să însemne că, deoarece \textbf{era Întipărirea persoanei Tatălui, cei care l-au văzut au văzut asemănarea Tatălui în El}.}[The Adventist Review and Sabbath Herald, September 18, 1855][https://documents.adventistarchives.org/Periodicals/RH/RH18550918-V07-06.pdf]


It is important to pay attention to the biblical evidence that brother Loughborough points out in the testimony that God has a body. Brother Loughborough reviews several Bible passages proving that God does have a material body, but it is invisible to our mortal eyes. Sister White wrote the same when she said\egwinline{\textbf{The Father is all the fulness of the Godhead \underline{bodily}} and \textbf{is invisible to mortal sight}}[Ms21-1906.9; 1906][https://egwwritings.org/read?panels=p9754.16]. No mortal eye can see the Father, but that does not prove that God can never be seen. Jesus said: \bible{\textbf{He that hath seen me, hath seen the Father}}[John 14:19]. Jesus explained these words two chapters prior: \bible{Jesus cried and said, He that believeth on me, believeth not on me, \textbf{but on him that sent me}. And \textbf{he that seeth me seeth him that sent me}}[John 12:44-45]. Jesus did not send Himself, neither is Jesus the Father, one and the same person; but we see the Father in Christ because He is the \textit{express image of the Father's person}. (Hebrews 1:3). As Jesus is a person, possessing a body, so is the Father. Brother Loughborough continues to prove his point that God is a person, possessing form and shape, because man was created in the image of God.


Este important să acordăm atenție dovezilor biblice pe care fratele Loughborough le evidențiază în mărturisire, că Dumnezeu are un trup. Fratele Loughborough trece în revistă mai multe pasaje biblice care dovedesc că Dumnezeu într-adevăr are un trup material, dar acesta este invizibil pentru ochii noștri muritori. Sora White a scris același lucru când a spus\egwinline{\textbf{Tatăl este toată plenitudinea Dumnezeirii \underline{în trup}} și \textbf{este invizibil pentru vederea muritorului}}[Ms21-1906.9; 1906][https://egwwritings.org/read?panels=p9754.16]. Niciun ochi muritor nu poate vedea pe Tatăl, dar aceasta nu dovedește că Dumnezeu nu poate fi văzut niciodată. Iisus a spus: \bible{\textbf{Cine M-a văzut pe Mine, a văzut pe Tatăl}}[Ioan 14:19]. Iisus a explicat aceste cuvinte două capitole mai devreme: \bible{Iisus a strigat și a zis: Cine crede în Mine, nu crede în Mine, \textbf{ci în Cel care M-a trimis}. Și \textbf{cine Mă vede pe Mine, vede pe Cel care M-a trimis}}[Ioan 12:44-45]. Iisus nu S-a trimis pe Sine, nici Iisus nu este Tatăl, o singură și aceeași persoană; dar vedem pe Tatăl în Hristos pentru că El este \textit{Întipărirea persoanei Tatălui}. (Evrei 1:3). Cum Iisus este o persoană, având un trup, la fel este și Tatăl. Fratele Loughborough continuă să-și dovedească punctul de vedere că Dumnezeu este o persoană, având formă și înfățișare, pentru că omul a fost creat după chipul lui Dumnezeu.


\others{But we will now return to the subject of The creation of man. \textbf{We have seen already that man’s being made in the image of God, could not refer to a moral image, for it would involve the absurdity that the lifeless clay of which man was formed, had a character like God}. \textbf{We now see the Scriptures clearly teach, that \underline{God is a person with a body and form}}. Then Genesis 1:26, may be understood to teach the fact, \textbf{that man was made in the form of God}. Other scriptures agree with this testimony. See Genesis 9:6. ‘Whoso sheddeth man’s blood, by man shall his blood be shed: \textbf{for in the image of God made he man}.’ \textbf{\underline{This testimony cannot apply to a spirit, or immaterial part of man: that which is in the image of God has blood}}. 1 Corinthians 11:7. ‘For a man indeed ought not to cover his head, \textbf{forasmuch as he is the image and glory of God}.’ James [Chap 3:9] speaking of the tongue says, ‘Therewith bless we God, even the Father; and therewith curse we men, \textbf{which are made after the similitude (likeness, resemblance – Webster) of God}.’ \textbf{The foregoing testimony settles the point, \underline{that the image of God does not refer to character but to form}}.}


\others{Dar acum vom reveni la subiectul creației omului. \textbf{Am văzut deja că faptul că omul a fost făcut după chipul lui Dumnezeu, nu ar putea să se refere la un chip moral, căci ar implica absurditatea că lutul fără viață din care a fost format omul, avea un caracter ca al lui Dumnezeu}. \textbf{Acum vedem că Scripturile învață clar că \underline{Dumnezeu este o persoană cu trup și formă}}. Atunci Geneza 1:26, poate fi înțeleasă ca învățând faptul că \textbf{omul a fost făcut în forma lui Dumnezeu}. Alte scripturi sunt de acord cu această mărturisire. Vezi Geneza 9:6. „Cine varsă sânge uman, sângele lui va fi vărsat de om: \textbf{căci după chipul lui Dumnezeu a făcut El pe om}.” \textbf{\underline{Această mărturisire nu poate fi aplicată unui spirit, sau unei părți imateriale a omului: ceea ce este după chipul lui Dumnezeu are sânge}}. 1 Corinteni 11:7. „Căci omul nu trebuie să-și acopere capul, \textbf{fiindcă este chipul și gloria lui Dumnezeu}.” Iacov [Cap 3:9] vorbind despre limbă spune: „Cu ea binecuvântăm pe Dumnezeu, chiar pe Tatăl; și cu ea blestemăm pe oameni, \textbf{care sunt făcuți după asemănarea (asemănare, similitudine – Webster) lui Dumnezeu}.” \textbf{Mărturisirea de mai sus rezolvă chestiunea că \underline{chipul lui Dumnezeu nu se referă la caracter ci la formă}}.}


\othersnogap{Genesis 2:7. ‘\textbf{And the Lord God formed man of the dust of the ground, and breathed into his nostrils the breath of life; and man became a living soul}.’}[The Adventist Review and Sabbath Herald, September 18, 1855][https://documents.adventistarchives.org/Periodicals/RH/RH18550918-V07-06.pdf]


\othersnogap{Geneza 2:7. „\textbf{Și Domnul Dumnezeu a format pe om din țărâna pământului, și i-a suflat în nări suflarea vieții; și omul a devenit suflet viu}”.}[The Adventist Review and Sabbath Herald, September 18, 1855][https://documents.adventistarchives.org/Periodicals/RH/RH18550918-V07-06.pdf]


God formed man in His own image. God is a person, having a body, shape and form, and He formed man into His own image. From this reasoning we derive the obvious meaning of the Scriptures’ testimony about the \emcap{personality of God}. If we make false conceptions regarding God’s person, we are in danger of misunderstanding the other truths which are connected with man’s nature (mortality of the soul, the state of the dead, etc.). In his article, Brother Loughborough continues on to explain the connection between false doctrine on the immortality of the soul and wrong conceptions regarding the \emcap{personality of God}. His article in the Review and Herald from September 18, was taken from his book “\textit{An Examination of the Scripture Testimony}\footnote{\href{https://egwwritings.org/?ref=en_MPC.2&para=961.2}{John Norton Loughborough, An Examination of the Scripture Testimony, 1855}}.


Dumnezeu a format omul după propriul Său chip. Dumnezeu este o persoană, având trup, înfățișare și formă, și El a format omul după propriul Său chip. Din acest raționament derivăm înțelesul evident al mărturiei Scripturilor despre \emcap{personalitatea lui Dumnezeu}. Dacă facem concepții false cu privire la persoana lui Dumnezeu, suntem în pericol să înțelegem greșit alte adevăruri care sunt conectate cu natura omului (mortalitatea sufletului, starea morților, etc.). În articolul său, Fratele Loughborough continuă să explice legătura dintre falsa doctrină privind nemurirea sufletului și concepțiile greșite cu privire la \emcap{personalitatea lui Dumnezeu}. Articolul său din Review and Herald din 18 septembrie, a fost preluat din cartea sa „\textit{O examinare a mărturiei Scripturilor}\footnote{\href{https://egwwritings.org/?ref=en_MPC.2&para=961.2}{John Norton Loughborough, An Examination of the Scripture Testimony, 1855}}.


% Is God a person? - by John N. Loughborough

\begin{titledpoem}
    \stanza{
        In heaven's realm, upon His throne, \\
        God dwells in form, not spirit alone. \\
        A tangible being with shape and face, \\
        Beyond our sight in that holy place.
    }

    \stanza{    
        His glory shines too bright to see, \\
        No mortal eyes bear such majesty. \\
        Yet through His Spirit, everywhere present, \\
        His power extends, divine and pleasant.
    }

    \stanza{
        In Christ we glimpse the Father's form, \\
        The express image, perfect and warm. \\
        For we are made in God's own shape, \\
        Not just in virtue, soul, or trait.
    }

    \stanza{
        The dust was fashioned by His hand, \\
        In His own image, as He planned. \\
        A person true with body real, \\
        Not formless mist, as some appeal.
    }

    \stanza{
        The Father bodily, yet unseen by eye, \\
        Waits for the pure in heart to draw nigh.
    }
\end{titledpoem}