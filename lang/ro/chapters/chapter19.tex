% \qrchapter{https://forgottenpillar.com/rsc/en-fp-chapter19}{Ellen White and Matthew 28:19}


\qrchapter{https://forgottenpillar.com/rsc/ro-fp-chapter19}{Ellen White și Matei 28:19}


Many assert that Ellen White promoted the Trinity doctrine, and that she is the one responsible for accepting it into our ranks. These claims do not consider that she defended the \emcap{personality of God} expressed in the first point of the \emcap{Fundamental Principles}. To support the claims that Ellen White was trinitarian, quotations are presented to her comment on Matthew 28:19:


Mulți afirmă că Ellen White a promovat doctrina Trinității și că ea este cea responsabilă pentru acceptarea acesteia în rândurile noastre. Aceste afirmații nu iau în considerare faptul că ea a apărat \emcap{personalitatea lui Dumnezeu} exprimată în primul punct al \emcap{Principiilor Fundamentale}. Pentru a susține afirmațiile că Ellen White era trinitariană, sunt prezentate citate din comentariul ei asupra versetului Matei 28:19:


\bible{Go ye therefore, and teach all nations, \textbf{baptizing them in the name of \underline{the Father}, and of \underline{the Son}, and of \underline{the Holy Ghost}}.}[Matthew 28:19]


\bible{Duceți-vă dar și faceți ucenici din toate neamurile, \textbf{botezându-i în Numele \underline{Tatălui} și al \underline{Fiului} și al \underline{Sfântului Duh}}.}[Matei 28:19]


This verse has been most prominent in support of the Trinity doctrine. The Trinity doctrine has propositions about the \emcap{personality of God} of which this text says nothing to support. This verse itself does not teach that the Father, the Son, and the Holy Ghost, comprise \textit{one} God, the God of the Bible. There are other explicit verses in the Bible that exclude such interpretation of the text, i.e. 1 Corinthians 8:4-6; John 17:3; Ephesians 4:4-6; 1 Timothy 2:5.


Acest verset a fost cel mai proeminent în susținerea doctrinei Trinității. Doctrina Trinității are propoziții despre \emcap{personalitatea lui Dumnezeu} despre care acest text nu spune nimic pentru a le susține. Acest verset în sine nu învață că Tatăl, Fiul și Duhul Sfânt alcătuiesc \textit{un singur} Dumnezeu, Dumnezeul Bibliei. Există alte versete explicite în Biblie care exclud o astfel de interpretare a textului, adică 1 Corinteni 8:4-6; Ioan 17:3; Efeseni 4:4-6; 1 Timotei 2:5.


Unfortunately, the same unsupported assumptions made about Matthew 28:19 are made about Sister White’s quotations dealing with this verse. For example, Sister White uses terms like \egwinline{three highest powers in heaven}[Lt253a-1903.18; 1903][https://egwwritings.org/read?panels=p10143.25], \egwinline{three great powers of heaven}[8T 254.1; 1904][https://egwwritings.org/read?panels=p112.1450], \egwinline{the three holy dignitaries of heaven}[Ms92-1901.26: 1901][https://egwwritings.org/read?panels=p10732.32] and similar expressions—none of these quotations justify the assumption that these three (the Father, the Son, and the Holy Spirit) make \textit{one} God. On the contrary, as discussed in the previous chapter, keeping William Boardman’s sentiments and \egwinline{the heavenly trio} in context, “\textit{three-in-one}” sentiments \egwinline{should not be trusted}[Ms21-1906.8; 1906][https://egwwritings.org/read?panels=p9754.15].


Din păcate, aceleași presupuneri nefondate făcute despre Matei 28:19 sunt făcute și despre citatele Sorei White care tratează acest verset. De exemplu, Sora White folosește termeni precum \egwinline{cele trei puteri supreme din cer}[Lt253a-1903.18; 1903][https://egwwritings.org/read?panels=p10143.25], \egwinline{cele trei mari puteri ale cerului}[8T 254.1; 1904][https://egwwritings.org/read?panels=p112.1450], \egwinline{cei trei demnitari sfinți ai cerului}[Ms92-1901.26: 1901][https://egwwritings.org/read?panels=p10732.32] și expresii similare—niciunul dintre aceste citate nu justifică presupunerea că acești trei (Tatăl, Fiul și Duhul Sfânt) formează \textit{un singur} Dumnezeu. Dimpotrivă, așa cum s-a discutat în capitolul anterior, păstrând opiniile lui William Boardman și \egwinline{trioul ceresc} în context, opiniile “\textit{trei-în-unul}” \egwinline{nu ar trebui să fie de încredere}[Ms21-1906.8; 1906][https://egwwritings.org/read?panels=p9754.15].


The heavenly trio (the group of three: the Father, the Son and the Holy Spirit) are also present in other Bible verses, in addition to Matthew 28:19. There are several other instances in the New Testament where the Father, the Son and the Holy Spirit are mentioned, and these verses should be used to interpret the meaning behind the heavenly trio. None of the verses on the heavenly trio prove a three-in-one God; rather, all of them refer to the Father as one God. In the following verses, the heavenly trio is bolded in order to better distinguish the Father, the Son and the Holy Spirit.


Trioul ceresc (grupul de trei: Tatăl, Fiul și Duhul Sfânt) este de asemenea prezent în alte versete biblice, pe lângă Matei 28:19. Există mai multe alte cazuri în Noul Testament unde sunt menționați Tatăl, Fiul și Duhul Sfânt, și aceste versete ar trebui folosite pentru a interpreta semnificația din spatele trioului ceresc. Niciunul dintre versetele despre trioul ceresc nu dovedește un Dumnezeu trei-în-unul; mai degrabă, toate se referă la Tatăl ca un singur Dumnezeu. În versetele următoare, trioul ceresc este îngroșat pentru a distinge mai bine Tatăl, Fiul și Duhul Sfânt.


\bible{There is one body, and \textbf{one Spirit}, even as ye are called in one hope of your calling; \textbf{One Lord}, one faith, one baptism, \textbf{One God and Father} of all, who is above all, and through all, and in you all.}[Ephesians 4:4-6]


\bible{Este un singur trup, \textbf{un singur Duh}, după cum și voi ați fost chemați la o singură nădejde a chemării voastre. Este \textbf{un singur Domn}, o singură credință, un singur botez. Este \textbf{un singur Dumnezeu și Tată} al tuturor, care este mai presus de toți, care lucrează prin toți și care este în toți.}[Efeseni 4:4-6]


\bible{Now there are diversities of gifts, but the \textbf{same Spirit}. And there are differences of administrations, but the \textbf{same Lord}. And there are diversities of operations, but it is \textbf{the same God} which worketh all in all.}[1 Corinthians 12:4-6]


\bible{Sunt felurite daruri, dar este același \textbf{Duh}; sunt felurite slujbe, dar este același \textbf{Domn}; sunt felurite lucrări, dar este același \textbf{Dumnezeu}, care lucrează totul în toți.}[1 Corinteni 12:4-6]


\bible{The grace of \textbf{the Lord Jesus Christ}, and the love of \textbf{God}, and the communion of \textbf{the Holy Ghost}, be with you all. Amen.}[2 Corinthians 13:14]


\bible{Harul \textbf{Domnului Isus Hristos}, și dragostea lui \textbf{Dumnezeu}, și împărtășirea \textbf{Sfântului Duh}, să fie cu voi cu toți! Amin.}[2 Corinteni 13:14]


\bible{For through \textbf{him} \normaltext{[Christ]} we both have access by one \textbf{Spirit} unto the \textbf{Father}.}[Ephesians 2:18]


\bible{Căci prin \textbf{El} \normaltext{[Hristos]} și unii și alții avem intrare la \textbf{Tatăl} într-un \textbf{Duh}.}[Efeseni 2:18]


\bible{But we are bound to give thanks alway to \textbf{God} for you, brethren beloved of \textbf{the Lord}, because \textbf{God} hath from the beginning chosen you to salvation through sanctification of \textbf{the Spirit} and belief of the truth.}[2 Thessalonians 2:13]


\bible{Dar noi suntem datori să mulțumim totdeauna lui \textbf{Dumnezeu} pentru voi, fraților preaiubiți de \textbf{Domnul}, pentru că de la început \textbf{Dumnezeu} v-a ales pentru mântuire, în sfințirea \textbf{Duhului} și credința adevărului.}[2 Tesaloniceni 2:13]


\bible{How much more shall the blood of \textbf{Christ}, who through the eternal \textbf{Spirit} offered himself without spot to \textbf{God}, purge your conscience from dead works to serve \textbf{the living God}?}[Hebrews 9:14]


\bible{Cu cât mai mult sângele lui \textbf{Hristos}, care, prin \textbf{Duhul} cel veșnic, S-a adus pe Sine însuși jertfă fără pată lui \textbf{Dumnezeu}, va curăți cugetul vostru de faptele moarte, ca să slujiți \textbf{Dumnezeului celui viu}!}[Evrei 9:14]


\bible{Elect according to the foreknowledge of \textbf{God the Father}, through sanctification of \textbf{the Spirit}, unto obedience and sprinkling of the blood of \textbf{Jesus Christ}: Grace unto you, and peace, be multiplied.}[1 Peter 1:2]


\bible{Aleși după știința mai dinainte a lui \textbf{Dumnezeu Tatăl}, prin sfințirea lucrată de \textbf{Duhul}, spre ascultarea și stropirea cu sângele lui \textbf{Isus Hristos}: Har și pace să vă fie înmulțite!}[1 Petru 1:2]


All of the above verses talk about the heavenly trio (the Father, the Son and the Holy Spirit), and all of them consistently testify that the Father is the one referred to as God.
The same reasoning holds ground for Ellen White’s interpretation of Matthew 28:19.


Toate versetele de mai sus vorbesc despre trioul ceresc (Tatăl, Fiul și Duhul Sfânt), și toate mărturisesc în mod consecvent că Tatăl este cel numit Dumnezeu.
Același raționament este valabil pentru interpretarea lui Ellen White a versetului Matei 28:19.


\egw{Christ gave His followers a positive promise that after His ascension He would send them His Spirit. ‘Go ye therefore,’ He said, ‘and teach all nations, baptizing them in the name of \textbf{the Father (a personal God),} and of \textbf{the Son (a personal Prince and Saviour),} and of \textbf{the Holy Ghost (sent from heaven to represent Christ);} teaching them to observe all things whatsoever I have commanded you, and, lo, I am with you alway, even unto the end of the world.’ Matthew 28:19, 20.}[RH October 26, 1897, par. 9; 1897][https://egwwritings.org/read?panels=p821.16317]


\egw{Hristos le-a dat urmașilor Săi o promisiune pozitivă că după înălțarea Sa le va trimite Duhul Său. „Duceți-vă dar”, a spus El, „și faceți ucenici din toate neamurile, botezându-i în numele \textbf{Tatălui (un Dumnezeu personal),} și al \textbf{Fiului (un Prinț și Mântuitor personal),} și al \textbf{Duhului Sfânt (trimis din cer pentru a-L reprezenta pe Hristos);} învățându-i să păzească tot ce v-am poruncit; și iată că Eu sunt cu voi în toate zilele, până la sfârșitul veacului.” Matei 28:19, 20.}[RH 26 octombrie 1897, par. 9; 1897][https://egwwritings.org/read?panels=p821.16317]


The brackets in this quotation are in the original manuscript written by Ellen White. Here, she gives her own interpretation of Matthew 28:19. The Father is a personal God, the Son is a personal Prince and Saviour, and the Holy Spirit is Christ’s representative. This interpretation is in harmony with the \emcap{personality of God} expressed in the first point of the \emcap{Fundamental Principles}. Matthew 28:19 is a matter of interpretation. The interpretation which makes the Heavenly Trio one God is not inspired. This is not what the text indicates. Rather, let's read Matthew 28:19 within inspired compound: “\textit{Go ye therefore, and teach all nations, baptizing them in the name of a personal God, a personal Prince and Savior, and of the Holy Ghost}.” If one would read the text as such, no one would ever assume that one God is a unity of three persons. Therefore, let's stick to the inspiration, rather than subterfuge\footnote{\href{https://egwwritings.org/?ref=en\_Lt232-1903.41&para=10197.50}{{EGW, Lt232-1903.41; 1903}}}.


Parantezele din acest citat sunt în manuscrisul original scris de Ellen White. Aici, ea oferă propria interpretare a versetului Matei 28:19. Tatăl este un Dumnezeu personal, Fiul este un Prinț și Mântuitor personal, iar Duhul Sfânt este reprezentantul lui Hristos. Această interpretare este în armonie cu \emcap{personalitatea lui Dumnezeu} exprimată în primul punct al \emcap{Principiilor Fundamentale}. Matei 28:19 este o chestiune de interpretare. Interpretarea care face din Trioul Ceresc un singur Dumnezeu nu este inspirată. Aceasta nu este ceea ce indică textul. Mai degrabă, să citim Matei 28:19 în cadrul inspirat: „\textit{Duceți-vă dar și faceți ucenici din toate neamurile, botezându-i în numele unui Dumnezeu personal, unui Prinț și Mântuitor personal, și al Duhului Sfânt}.” Dacă cineva ar citi textul astfel, nimeni nu ar presupune vreodată că un singur Dumnezeu este o unitate a trei persoane. Prin urmare, să rămânem la inspirație, mai degrabă decât la subterfugiu\footnote{\href{https://egwwritings.org/?ref=en\_Lt232-1903.41&para=10197.50}{{EGW, Lt232-1903.41; 1903}}}.


\egw{Let them be thankful to God for His manifold mercies and be kind to one another. \textbf{They have \underline{one God} and \underline{one Saviour}; and \underline{one Spirit}—\underline{the Spirit of Christ}—is to bring unity into their ranks}.}[9T 189.3; 1909][https://egwwritings.org/read?panels=p115.1057]


\egw{Să fie recunoscători lui Dumnezeu pentru îndurările Sale multiple și să fie buni unii cu alții. \textbf{Ei au \underline{un singur Dumnezeu} și \underline{un singur Mântuitor}; și \underline{un singur Duh}—\underline{Duhul lui Hristos}—care trebuie să aducă unitate în rândurile lor}.}[9T 189.3; 1909][https://egwwritings.org/read?panels=p115.1057]


In light of the presented evidence, we see that simply numbering the Father, the Son and the Holy Spirit, does not prove the \textit{three-in-one} assumption, nor is it in conflict with the \emcap{personality of God} expressed in the \emcap{Fundamental Principles}. There is no denial of three persons of the Godhead, but only a denial of the assumption that these Three Great Worthies make one God.


În lumina dovezilor prezentate, vedem că simpla numărare a Tatălui, Fiului și Duhului Sfânt nu dovedește presupunerea \textit{trei-într-unul}, nici nu este în conflict cu \emcap{personalitatea lui Dumnezeu} exprimată în \emcap{Principiile Fundamentale}. Nu există o negare a celor trei persoane ale Dumnezeirii, ci doar o negare a presupunerii că aceste Trei Mari Demnitari formează un singur Dumnezeu.


Matthew 28:19 is a valuable verse and it opens a new field of study within the Bible and the Spirit of Prophecy. In the context of the Living Temple, and referring to its sentiments, Sister White wrote that this verse should be studied most earnestly because it is not half understood.


Matei 28:19 este un verset valoros și deschide un nou câmp de studiu în Biblie și în Spiritul Profeției. În contextul cărții Templul viu, și referindu-se la opiniile acesteia, sora White a scris că acest verset ar trebui studiat cu cea mai mare seriozitate pentru că nu este pe jumătate înțeles.


\egw{Just before His ascension, Christ gave His disciples a wonderful presentation, \textbf{as recorded in the twenty-eighth chapter of Matthew}. \textbf{This chapter contains instruction} that our ministers, our \textbf{physicians}, our youth, and all our church members need to \textbf{study most \underline{earnestly}}. \textbf{Those who study this instruction as they should will \underline{not dare to advocate theories that have no foundation in the Word of God}}. My brethren and sisters, make the Scriptures, which contain the alpha and omega of knowledge, your study. \textbf{All through the Old Testament and the New, there are things \underline{that are not half understood}}. ‘And Jesus came and spake unto them, saying, All power is given unto Me in heaven and in earth. Go ye therefore, and teach all nations, \textbf{baptizing them in the name of the Father, and of the Son, and of the Holy Ghost}; teaching them to observe all things whatsoever I have commanded you; and, lo, I am with you alway, even unto the end of the world.’ [Verses 18-20.]}[Lt214-1906.10; 1906][https://egwwritings.org/read?panels=p10171.16]


\egw{Chiar înainte de înălțarea Sa, Hristos le-a dat ucenicilor Săi o prezentare minunată, \textbf{așa cum este consemnată în capitolul douăzeci și opt din Matei}. \textbf{Acest capitol conține instrucțiuni} de care pastorii noștri, \textbf{medicii} noștri, tinerii noștri și toți membrii bisericii noastre trebuie să le \textbf{studieze cu cea mai mare \underline{seriozitate}}. \textbf{Cei care studiază aceste instrucțiuni așa cum ar trebui \underline{nu vor îndrăzni să susțină teorii care nu au nicio bază în Cuvântul lui Dumnezeu}}. Frații și surorile mele, faceți din Scripturi, care conțin alfa și omega cunoașterii, studiul vostru. \textbf{În tot Vechiul Testament și în Noul, există lucruri care \underline{nu sunt pe jumătate înțelese}}. „Isus S-a apropiat de ei, a vorbit cu ei și le-a zis: «Toată puterea Mi-a fost dată în cer și pe pământ. Duceți-vă și faceți ucenici din toate neamurile, \textbf{botezându-i în Numele Tatălui și al Fiului și al Sfântului Duh}. Și învățați-i să păzească tot ce v-am poruncit. Și iată că Eu sunt cu voi în toate zilele, până la sfârșitul veacului.»“ [Versetele 18-20.]}[Lt214-1906.10; 1906][https://egwwritings.org/read?panels=p10171.16]


There is a reason why Ellen White pinpointed to Matthew 28:19 as a Scripture which is \egwinline{not half understood.} This statement is made in the context of 1906, where many ministers, and physicians were advocating the trinity doctrine. As we have seen, the understanding of God as a trinity, was not something Ellen White supported, and for this reason, herself, she dared not \egwinline{to advocate theories that have no foundation in the Word of God.}


Există un motiv pentru care Ellen White a indicat Matei 28:19 ca fiind o Scriptură care \egwinline{nu este pe jumătate înțeleasă.} Această afirmație este făcută în contextul anului 1906, când mulți pastori și medici susțineau doctrina Trinității. După cum am văzut, înțelegerea lui Dumnezeu ca o trinitate nu era ceva pe care Ellen White îl susținea, și din acest motiv, ea însăși nu a îndrăznit \egwinline{să susțină teorii care nu au nicio bază în Cuvântul lui Dumnezeu.}


\egw{The great Teacher held in His hand \textbf{the entire map of truth. In \underline{simple} language He \underline{made plain} to His disciples} the way to heaven and \textbf{the endless subjects of divine power}. \textbf{The question of \underline{the essence of God} was a subject on which He maintained a wise reserve}, for their entanglements and specifications would bring in science which could not be dwelt upon by unsanctified minds without confusion. \textbf{In regard to God and in regard to His personality, the Lord Jesus said}, ‘Have I been so long time with you, and yet hast thou not known Me, Philip? He that hath seen Me hath seen the Father.’ [John 14:9.] \textbf{Christ was the express image of His Father’s person}.}[19LtMs, Ms 45, 1904, par. 15][https://egwwritings.org/read?panels=p14069.9381023&index=0]


\egw{Marele Învățător ținea în mâna Sa \textbf{întreaga hartă a adevărului. În limbaj \underline{simplu}, ea \underline{a făcut clar} ucenicilor Săi} calea către cer și \textbf{subiectele nesfârșite ale puterii divine}. \textbf{Chestiunea \underline{esenței lui Dumnezeu} era un subiect asupra căruia ea a menținut o rezervă înțeleaptă}, căci încurcăturile și specificațiile lor ar aduce știință care nu putea fi abordată de minți nesfințite fără confuzie. \textbf{În privința lui Dumnezeu și în privința personalității Lui, Domnul Isus a spus}: „De atâta vreme sunt cu voi, și nu M-ai cunoscut, Filipe? Cine M-a văzut pe Mine a văzut pe Tatăl.” [Ioan 14:9.] \textbf{Hristos era întipărirea persoanei Tatălui Său}.}[19LtMs, Ms 45, 1904, par. 15][https://egwwritings.org/read?panels=p14069.9381023&index=0]


\egwnogap{The open path, the safe path of walking in the way of His commandments, is a path from which there is no safe departing. \textbf{And when men follow their own human theories dressed up in soft, fascinating representations, they make a snare in which to catch souls}. \textbf{\underline{In the place of devoting your powers to theorizing}}, Christ has given you a work to do. His commission is, Go <throughout the world> and make disciples of all nations, \textbf{baptizing them in the name of the Father, and of the Son, and of the Holy Ghost}. Before the disciples shall compass the threshold, there is to be the imprint of \textbf{the sacred name, baptizing the believers in \underline{the name of the threefold powers} in the heavenly world}. The human mind is impressed in this ceremony, the beginning of the Christian life. It means very much. The work of salvation is not a small matter, but so vast that \textbf{the highest authorities} are taken hold of by the expressed faith of the human agency. \textbf{The Father, the Son, and the Holy Ghost, \underline{the eternal Godhead} is involved in the action required to make assurance to the human agent to unite \underline{all heaven} to contribute to the exercise of human faculties to reach and embrace the fulness of \underline{the threefold powers} to unite in the great work appointed, confederating the heavenly powers with the human, that men may become, through heavenly efficiency, partakers of the divine nature and workers together with Christ}.}[19LtMs, Ms 45, 1904, par. 16][https://egwwritings.org/read?panels=p14069.9381024&index=0]


\egwnogap{Calea deschisă, calea sigură de a umbla pe calea poruncilor Sale, este o cale de la care nu există o îndepărtare sigură. \textbf{Și când oamenii urmează propriile lor teorii omenești îmbrăcate în reprezentări blânde și fascinante, ei fac o cursă în care să prindă suflete}. \textbf{\underline{În loc să vă dedicați puterile teoretizării}}, Hristos v-a dat o lucrare de făcut. Însărcinarea Sa este: Mergeți <în toată lumea> și faceți ucenici din toate neamurile, \textbf{botezându-i în numele Tatălui și al Fiului și al Sfântului Duh}. Înainte ca ucenicii să treacă pragul, trebuie să fie amprenta \textbf{numelui sacru, botezând credincioșii în \underline{numele puterilor întreit} din lumea cerească}. Mintea umană este impresionată în această ceremonie, începutul vieții creștine. Înseamnă foarte mult. Lucrarea mântuirii nu este o chestiune mică, ci atât de vastă încât \textbf{cele mai înalte autorități} sunt prinse de credința exprimată a agentului uman. \textbf{Tatăl, Fiul și Duhul Sfânt, \underline{Dumnezeirea eternă} este implicată în acțiunea necesară pentru a da asigurare agentului uman de a uni \underline{tot cerul} pentru a contribui la exercitarea facultăților umane pentru a ajunge și a îmbrățișa plinătatea \underline{puterilor întreit} pentru a se uni în marea lucrare desemnată, confederând puterile cerești cu cele umane, pentru ca oamenii să devină, prin eficiența cerească, părtași naturii divine și conlucrători cu Hristos}.}[19LtMs, Ms 45, 1904, par. 16][https://egwwritings.org/read?panels=p14069.9381024&index=0]


This quotation is yet another often misrepresented statement. It has been often used to argue that Ellen White advocated for the Trinity by referencing the Father, the Son and the Holy Spirit by term \egwinline{eternal Godhead.} However, we must peel back the layers of its context. Ellen White was explaining the meaning behind Matthew 28:19. She stated: \egwinline{In the place of devoting your powers to theorizing,} fulfill the commission given by Christ. Theorizing about what? Theorizing about \egwinline{the essence of God.} This is another “smoking gun” for the Trinity doctrine, especially when she referenced the \emcap{personality of God} by stating: \egwinline{\textbf{In regard to God and in regard to His personality}, the Lord Jesus said…[John 14:9.] Christ was the express image of His \textbf{Father’s person}.} John 14:9 does not mean that seeing the Father in Christ implies they are one and the same person, all part of one God. Rather, it affirms that Christ is the express image of the Father’s person. The “God” she referred to was the Father. Indeed, Jesus taught the truth about who and what God is. This is what He \egwinline{made plain} \egwinline{in the simple language.} To claim that by the term \egwinline{eternal Godhead} Ellen White was endorsing the Trinity would contradict the very caution she expressed in the context of this passage.


Acest citat este încă o altă afirmație adesea denaturată. A fost folosit adesea pentru a argumenta că Ellen White a susținut Trinitatea făcând referire la Tatăl, Fiul și Duhul Sfânt prin termenul \egwinline{Dumnezeirea eternă.} Cu toate acestea, trebuie să dezvăluim straturile contextului său. Ellen White explica semnificația din spatele Matei 28:19. Ea a afirmat: \egwinline{În loc să vă dedicați puterile teoretizării,} împliniți însărcinarea dată de Hristos. Teoretizare despre ce? Teoretizare despre \egwinline{esența lui Dumnezeu.} Aceasta este o altă „dovadă zdrobitoare” pentru doctrina Trinității, mai ales când ea a făcut referire la \emcap{personalitatea lui Dumnezeu} afirmând: \egwinline{\textbf{În privința lui Dumnezeu și în privința personalității Lui}, Domnul Isus a spus...[Ioan 14:9.] Hristos era întipărirea \textbf{persoanei Tatălui Său}.} Ioan 14:9 nu înseamnă că a-L vedea pe Tatăl în Hristos implică faptul că ei sunt una și aceeași persoană, toți parte dintr-un singur Dumnezeu. Mai degrabă, afirmă că Hristos este întipărirea persoanei Tatălui. „Dumnezeul” la care ea s-a referit era Tatăl. Într-adevăr, Isus a învățat adevărul despre cine și ce este Dumnezeu. Aceasta este ceea ce El \egwinline{a făcut clar} \egwinline{în limbaj simplu.} A pretinde că prin termenul \egwinline{Dumnezeirea eternă} Ellen White susținea Trinitatea ar contrazice chiar precauția pe care ea a exprimat-o în contextul acestui pasaj.


Unfortunately, the desperate desire of Trinitarians to paint Ellen White as a Trinitarian advocate has overshadowed the true, inspired meaning of Matthew 28:19. Her message was: \egwinline{In the place of devoting your powers to theorizing} about \egwinline{the essence of God,} Christ has given us the commission in Matthew 28:19. And she explained the meaning of Matthew 28:19. Her point was: The Father, Son, and Holy Spirit unite all of heaven’s resources with human effort so that, through divine power, people may share in God’s nature and work alongside Christ. That is the meaning of this \egwinline{threefold name.} She continued explaining:


Din păcate, dorința disperată a trinitarienilor de a o prezenta pe Ellen White ca o susținătoare a Trinității a umbrit adevăratul sens inspirat al Matei 28:19. Mesajul ei era: \egwinline{În loc să vă dedicați puterile teoretizării} despre \egwinline{esența lui Dumnezeu,} Hristos ne-a dat însărcinarea din Matei 28:19. Și ea a explicat semnificația Matei 28:19. Ideea ei era: Tatăl, Fiul și Duhul Sfânt unesc toate resursele cerului cu efortul uman, astfel încât, prin puterea divină, oamenii să poată împărtăși natura lui Dumnezeu și să lucreze alături de Hristos. Aceasta este semnificația acestui \egwinline{nume întreit.} Ea a continuat explicând:


\egw{\textbf{Man’s capabilities can multiply through the connection of human agencies with divine agencies}. \textbf{United with the heavenly powers}, the human capabilities increase according to that faith that works by love and purifies, sanctifies, and ennobles the whole man. \textbf{\underline{The heavenly powers} have \underline{pledged themselves} to minister to human agents to make the name of God and of Christ and of the Holy Spirit their living efficiency, working and energizing the sanctified man, to make this name above every other name}. \textbf{All the treasures of heaven are under obligation to do for man} infinitely more than human beings can comprehend by multiplying threefold the human with the heavenly agencies.}[19LtMs, Ms 45, 1904, par. 17][https://egwwritings.org/read?panels=p14069.9381026&index=0]


\egw{\textbf{Capacitățile omului se pot multiplica prin conexiunea agențiilor umane cu agențiile divine}. \textbf{Unite cu puterile cerești}, capacitățile umane cresc în funcție de acea credință care lucrează prin dragoste și purifică, sfințește și înnobilează întregul om. \textbf{\underline{Puterile cerești} s-au \underline{angajat} să slujească agenților umani pentru a face numele lui Dumnezeu și al lui Hristos și al Duhului Sfânt eficiența lor vie, lucrând și energizând omul sfințit, pentru a face acest nume mai presus de orice alt nume}. \textbf{Toate comorile cerului sunt obligate să facă pentru om} infinit mai mult decât pot înțelege ființele umane prin multiplicarea întreit a umanului cu agențiile cerești.}[19LtMs, Ms 45, 1904, par. 17][https://egwwritings.org/read?panels=p14069.9381026&index=0]


\egwnogap{\textbf{\underline{The three great and glorious heavenly characters} are present on the occasion of baptism. All the human capabilities are to be henceforth consecrated powers to do service for God in representing the Father, the Son, and the Holy Ghost upon whom they depend. \underline{All heaven is represented by these three} in covenant relation with the new life}. ‘If ye then be risen with Christ, seek those things that are above, where Christ sitteth at \textbf{the right hand of God}.’ [Colossians 3:1.]}[19LtMs, Ms 45, 1904, par. 18][https://egwwritings.org/read?panels=p14069.9381027&index=0]


\egwnogap{\textbf{\underline{Cele trei mari și glorioase caractere cerești} sunt prezente cu ocazia botezului. Toate capacitățile umane trebuie să fie de acum înainte puteri consacrate pentru a face slujire pentru Dumnezeu reprezentând Tatăl, Fiul și Duhul Sfânt de care depind. \underline{Tot cerul este reprezentat de acești trei} în relație de legământ cu noua viață}. „Dacă deci ați înviat împreună cu Hristos, să umblați după lucrurile de sus, unde Hristos șade la \textbf{dreapta lui Dumnezeu}.” [Coloseni 3:1.]}[19LtMs, Ms 45, 1904, par. 18][https://egwwritings.org/read?panels=p14069.9381027&index=0]


Many claim that Matthew 28:19 is uninspired because it was inserted by the Catholic Church\footnote{Note, 1 John 5:7 \bible{For there are three that bear record in heaven, the Father, the Word, and the Holy Ghost: and these three are one.} is an interpolation known as “\textit{Johannine Comma}”. Ellen White never used that verse. This was not the case with Matthew 28:19.}. Yet, here we have divine inspiration revealing its true meaning—the significance of baptism in the threefold name as a pledge made by these \egwinline{three great and glorious heavenly characters.} Their pledge is that \egwinline{\textbf{all the treasures of heaven are under obligation to do for man} infinitely more than human beings can comprehend by multiplying threefold the human with the heavenly agencies.}


Mulți susțin că Matei 28:19 nu este inspirat deoarece a fost inserat de Biserica Catolică\footnote{Notă, 1 Ioan 5:7 \bible{Căci trei sunt care mărturisesc în cer: Tatăl, Cuvântul și Duhul Sfânt, și aceștia trei una sunt.} este o interpolare cunoscută ca „\textit{Comma Johanneum}”. Ellen White nu a folosit niciodată acel verset. Acesta nu a fost cazul cu Matei 28:19.}. Cu toate acestea, aici avem inspirația divină care dezvăluie adevăratul său sens—semnificația botezului în numele întreit ca un angajament făcut de aceste \egwinline{trei mari și glorioase caractere cerești.} Angajamentul lor este că \egwinline{\textbf{toate comorile cerului sunt obligate să facă pentru om} infinit mai mult decât pot înțelege ființele umane prin multiplicarea întreit a umanului cu agențiile cerești.}


Ellen White frequently quoted Matthew 28:19, explaining the pledge of the Father, the Son, and the Holy Spirit. This pledge serves as a wonderful encouragement and a promise upheld by Heaven. A detailed study of this pledge is beyond the scope of this book, as it does not directly address the presence and \emcap{personality of God}. However, we encourage you to explore this topic for yourself. When you delve deeper into its meaning, you will come to understand the reality of the ministry of heavenly angels.


Ellen White a citat frecvent Matei 28:19, explicând angajamentul Tatălui, Fiului și Duhului Sfânt. Acest angajament servește ca o încurajare minunată și o promisiune susținută de Cer. Un studiu detaliat al acestui angajament depășește scopul acestei cărți, deoarece nu abordează direct prezența și \emcap{personalitatea lui Dumnezeu}. Cu toate acestea, vă încurajăm să explorați acest subiect singuri. Când veți aprofunda semnificația sa, veți ajunge să înțelegeți realitatea slujirii îngerilor cerești.


Sister White stated that \egwinline{all heaven is represented by these three in covenant relation with the new life.} These three are the Father, the Son, and the Holy Spirit. In another instance, she said:


Sora White a afirmat că \egwinline{tot cerul este reprezentat de acești trei în relație de legământ cu noua viață.} Acești trei sunt Tatăl, Fiul și Duhul Sfânt. Într-o altă ocazie, ea a spus:


\egw{\textbf{All heaven is interested in your home}. \textbf{God and Christ and \underline{the heavenly angels}} are intensely desirous that you shall so train your children that they will be prepared to enter the family of the redeemed.}[17LtMs, Ms 161, 1902, par. 11][https://egwwritings.org/read?panels=p14067.9877018&index=0]


\egw{\textbf{Întregul cer este interesat de căminul tău}. \textbf{Dumnezeu și Hristos și \underline{îngerii cerești}} sunt intens doritori ca tu să-ți instruiești copiii astfel încât ei să fie pregătiți să intre în familia celor răscumpărați.}[17LtMs, Ms 161, 1902, par. 11][https://egwwritings.org/read?panels=p14067.9877018&index=0]


This is not a contradiction. All of heaven is represented by the Father, the Son, and the Holy Spirit, and in this quote, she specifically mentioned \egwinline{God and Christ and \textbf{the heavenly angels}.} There is a close connection between the workings of the Holy Spirit and the ministry of angels. The Inspiration testifies:


Aceasta nu este o contradicție. Întregul cer este reprezentat de Tatăl, Fiul și Duhul Sfânt, iar în acest citat, ea a menționat în mod specific \egwinline{Dumnezeu și Hristos și \textbf{îngerii cerești}.} Există o legătură strânsă între lucrările Duhului Sfânt și slujirea îngerilor. Inspirația mărturisește:


\egw{A measure of \textbf{the Spirit} is given to every man to profit withal. \textbf{Through the ministry of the angels \underline{the Holy Spirit is enabled} to work upon the mind and heart of the human agent}, and draw him to Christ who has paid the ransom money for his soul, that the sinner may be rescued from the slavery of sin and Satan.}[8LtMs, Lt 71, 1893, par. 10][https://egwwritings.org/read?panels=p14058.6086016&index=0]


\egw{O măsură din \textbf{Duhul} este dată fiecărui om spre folosul lui. \textbf{Prin slujirea îngerilor \underline{Duhul Sfânt este în stare} să lucreze asupra minții și inimii agentului uman}, și să-l atragă la Hristos care a plătit prețul de răscumpărare pentru sufletul său, pentru ca păcătosul să poată fi salvat din sclavia păcatului și a lui Satan.}[8LtMs, Lt 71, 1893, par. 10][https://egwwritings.org/read?panels=p14058.6086016&index=0]


This angelic ministry is one of the elements in the baptismal pledge of Matthew 28:19. When Ellen White said, \egwinline{\textbf{The heavenly powers} have \textbf{pledged themselves} to minister to human agents…,} she was referring to the holy angels. The connection between the Holy Spirit and the holy angels is beyond the scope of this book, but you can explore this topic further in the sequel, \textit{Rediscovering the Pillar}\footnote{Download for free: \href{https://forgottenpillar.com/book/rediscovering-the-pillar}{https://forgottenpillar.com/book/rediscovering-the-pillar}}, in the section on the Holy Spirit\footnote{Also, see the study on the angels \href{https://notefp.link/angels}{https://notefp.link/angels}}.


Această slujire îngerească este unul dintre elementele din angajamentul botezului din Matei 28:19. Când Ellen White a spus, \egwinline{\textbf{Puterile cerești} s-au \textbf{angajat} să slujească agenților umani…,} ea se referea la îngerii sfinți. Legătura dintre Duhul Sfânt și îngerii sfinți depășește scopul acestei cărți, dar puteți explora această temă mai departe în continuare, \textit{Redescoperind Stâlpul}\footnote{Descărcați gratuit: \href{https://forgottenpillar.com/book/rediscovering-the-pillar}{https://forgottenpillar.com/book/rediscovering-the-pillar}}, în secțiunea despre Duhul Sfânt\footnote{De asemenea, vedeți studiul despre îngeri \href{https://notefp.link/angels}{https://notefp.link/angels}}.


% Ellen White and Matthew 28:19

\begin{titledpoem}
    
    \stanza{
        In threefold name we’re baptized true, \\
        Not trinity as some construe. \\
        The Father, Son, and Spirit’s role, \\
        Not one God formed of triple whole.
    }

    \stanza{
        Dear Ellen’s words make clear the case, \\
        This pledge assures us heaven’s grace. \\
        The powers three have pledged their might, \\
        To guide the faithful to the light.
    }

    \stanza{
        Not proof of essence three-in-one, \\
        But heaven’s promise, freely done. \\
        A covenant of help divine, \\
        As new believers cross the line.
    }

    \stanza{
        The Father – God, in person real, \\
        The Son – our Prince, our wounds to heal, \\
        The Spirit – representative, \\
        Through Him Christ does in us now live.
    }
    
\end{titledpoem}