% \qrchapter{https://forgottenpillar.com/rsc/en-fp-chapter15}{Dr. Kellogg and the Trinity doctrine}


\qrchapter{https://forgottenpillar.com/rsc/ro-fp-chapter15}{Dr. Kellogg și doctrina Trinității}


The key problem with the Kellogg controversy was the sentiments over the \emcap{personality of God}, which were departing from the foundation of our faith, that God established at the beginning of our work. We have been told that \egwinline{Many things of like character will in the future arise}[Ms137-1903.10; 1903][https://egwwritings.org/read?panels=p9939.17]. In the book, the Living Temple, we see the sentiments regarding the \emcap{personality of God} and where His presence is, which were stepping off of the \emcap{Fundamental Principles}. This step was never supposed to be made! But we raise the question, where was this step heading? We will see the evidence that this step was heading toward the Trinity doctrine. Sister White prophesied that Kellogg’s step would lead toward the Omega heresy. Can we see the connection between Kellogg’s controversy and the Trinity doctrine?


Problema cheie cu controversa Kellogg a fost opiniile despre \emcap{personalitatea lui Dumnezeu}, care se îndepărtau de la temelia credinței noastre, pe care Dumnezeu a stabilit-o la începutul lucrării noastre. Ni s-a spus că \egwinline{Multe lucruri de caracter asemănător vor apărea în viitor}[Ms137-1903.10; 1903][https://egwwritings.org/read?panels=p9939.17]. În cartea, Templul viu, vedem opiniile cu privire la \emcap{personalitatea lui Dumnezeu} și unde este prezența Lui, care se îndepărtau de \emcap{Principiile Fundamentale}. Acest pas nu trebuia făcut niciodată! Dar ridicăm întrebarea, încotro se îndrepta acest pas? Vom vedea dovada că acest pas se îndrepta spre doctrina Trinității. Sora White a profețit că pasul lui Kellogg va duce spre erezia Omega. Putem vedea legătura dintre controversa lui Kellogg și doctrina Trinității?


In the following section, we want to present you with the connection between Kellogg’s controversy and the doctrine of Trinity. It is important to emphasize that the Living Temple does not contain this doctrine as it is believed today. The main problem with Kellogg’s teaching was the \textit{stepping off} of the \emcap{Fundamental Principles}, which were the foundation of our faith. The information we will present to you reveals that Dr. Kellogg justified his actions in stepping off of the foundation through his belief in the doctrine of Trinity. This is not difficult to see when we recognize that the \emcap{Fundamental Principles} were a non-Trinitarian. Our main focus should not be in recognizing the Trinity doctrine in Kellogg's arguments, but rather in understanding the differences between Kellogg’s teachings and the teachings of the \emcap{Fundamental Principles} regarding \egwinline{the personality of God and where His presence is}[SpTB02 51.3; 1903][https://egwwritings.org/read?panels=p417.262]. In other words, what were the steps Kellogg made in stepping off of the foundation of our faith? This approach is advocated by the Spirit of Prophecy and it will help us to avoid speculations regarding Kellogg’s motives—it will help us to focus upon the truth. Ellen White tells us that there are many good things written in the Living Temple, but they are mingled with specious, deceptive theories regarding the \emcap{personality of God} and \emcap{of Christ}.


În secțiunea următoare, vrem să vă prezentăm legătura dintre controversa lui Kellogg și doctrina Trinității. Este important să subliniem că Templul viu nu conține această doctrină așa cum este crezută astăzi. Problema principală cu învățătura lui Kellogg a fost \textit{îndepărtarea} de \emcap{Principiile Fundamentale}, care erau temelia credinței noastre. Informațiile pe care vi le vom prezenta dezvăluie că Dr. Kellogg și-a justificat acțiunile de îndepărtare de temelie prin credința sa în doctrina Trinității. Acest lucru nu este dificil de văzut când recunoaștem că \emcap{Principiile Fundamentale} erau non-trinitariene. Atenția noastră principală nu ar trebui să fie în recunoașterea doctrinei Trinității în argumentele lui Kellogg, ci mai degrabă în înțelegerea diferențelor dintre învățăturile lui Kellogg și învățăturile \emcap{Principiilor Fundamentale} cu privire la \egwinline{personalitatea lui Dumnezeu și unde este prezența Lui}[SpTB02 51.3; 1903][https://egwwritings.org/read?panels=p417.262]. Cu alte cuvinte, care au fost pașii pe care Kellogg i-a făcut în îndepărtarea de temelia credinței noastre? Această abordare este susținută de Spiritul Profetic și ne va ajuta să evităm speculațiile cu privire la motivele lui Kellogg—ne va ajuta să ne concentrăm asupra adevărului. Ellen White ne spune că sunt multe lucruri bune scrise în Templul viu, dar ele sunt amestecate cu teorii specioase, înșelătoare cu privire la \emcap{personalitatea lui Dumnezeu} și \emcap{a lui Hristos}.


\begin{figure}[hp]
    \centering
    \includegraphics[width=1\linewidth]{images/john-h-kellogg.jpg}
    \caption*{John Harvey Kellogg (1852-1943)}
    \label{fig:john-h-kellogg}
\end{figure}


\begin{figure}[hp]
    \centering
    \includegraphics[width=1\linewidth]{images/john-h-kellogg.jpg}
    \caption*{John Harvey Kellogg (1852-1943)}
    \label{fig:john-h-kellogg}
\end{figure}


\egw{\textbf{The book Living Temple contains specious, \underline{deceptive sentiments regarding the personality of God and of Christ}}. The Lord opened before me the true meaning of these sentiments, showing me that unless they were steadfastly repudiated, they would deceive the very elect. \textbf{Precious truth and beautiful sentiments were woven in with false, misleading theories. Thus truth was used to substantiate the \underline{most dangerous errors}. The precious representations of God are so misconstrued as to appear to uphold falsehoods \underline{originated by the great apostate}. Sentiments that belong to the revealings of God are mingled with specious, deceptive theories of satanic agencies}.}[Lt146-1905.2; 1905][https://egwwritings.org/read?panels=p9430.8]


\egw{\textbf{Cartea Templul viu conține opinii specioase, \underline{înșelătoare cu privire la personalitatea lui Dumnezeu și a lui Hristos}}. Domnul mi-a deschis adevăratul înțeles al acestor opinii, arătându-mi că, dacă nu ar fi respinse cu fermitate, ele i-ar înșela chiar pe cei aleși. \textbf{Adevărul prețios și opiniile frumoase au fost țesute împreună cu teorii false, înșelătoare. Astfel, adevărul a fost folosit pentru a susține \underline{cele mai periculoase erori}. Reprezentările prețioase ale lui Dumnezeu sunt atât de greșit interpretate încât par să susțină falsitățile \underline{originate de marele apostat}. Opiniile care aparțin revelațiilor lui Dumnezeu sunt amestecate cu teorii specioase, înșelătoare ale agenților satanici}.}[Lt146-1905.2; 1905][https://egwwritings.org/read?panels=p9430.8]


\egwnogap{In the controversy over these theories \textbf{it has been asserted that I believed and taught the same things} that I have been instructed to condemn in the book Living Temple. \textbf{This I deny}. In the name of Jesus Christ of Nazareth, \textbf{I say that this is not so}.}[Lt146-1905.3; 1905][https://egwwritings.org/read?panels=p9430.9]


\egwnogap{În controversa asupra acestor teorii \textbf{s-a afirmat că am crezut și am învățat aceleași lucruri} pe care am fost instruită să le condamn în cartea Templul viu. \textbf{Aceasta o neg}. În numele lui Isus Hristos din Nazaret, \textbf{spun că nu este așa}.}[Lt146-1905.3; 1905][https://egwwritings.org/read?panels=p9430.9]


This mixture of truth and error makes the matter difficult. In the eyes of pro-trinitarian scholars, the problem is solely attributed to pantheism, and the evidence of Kellogg's belief in the Trinity doctrine is interpreted as belief in a false Trinity\footnote{Whidden, Woodrow W, et al. \textit{The Trinity : Understanding God's Love, His Plan of Salvation, and Christian Relationships}. Hagerstown, Md, Review And Herald Pub. Association, 2002., p. 217}. Sister White's rebuke is attributed to the defense of the “correct” Trinity, which she supposedly believed. Unfortunately, such interpretation does not acknowledge Sister White's defense of the \emcap{Fundamental Principles} regarding the \emcap{personality of God} and of Christ, thus it is a misinterpretation of her work. In the following sections, we will examine historical data on Dr. Kellogg's connection with the doctrine of Trinity from the perspective of the Adventist truth on the \emcap{personality of God}, which constituted the foundation of our faith. With this perspective, we believe that the historical data will shine in a new light and spark honest and constructive dialogue in our church.


Acest amestec de adevăr și eroare face problema dificilă. În ochii învățaților pro-trinitarieni, problema este atribuită exclusiv panteismului, iar dovada credinței lui Kellogg în doctrina Trinității este interpretată ca credință într-o Trinitate falsă\footnote{Whidden, Woodrow W, et al. \textit{The Trinity : Understanding God's Love, His Plan of Salvation, and Christian Relationships}. Hagerstown, Md, Review And Herald Pub. Association, 2002., p. 217}. Mustrarea Sorei White este atribuită apărării Trinității „corecte”, în care ea ar fi crezut. Din păcate, o astfel de interpretare nu recunoaște apărarea Sorei White a \emcap{Principiilor Fundamentale} cu privire la \emcap{personalitatea lui Dumnezeu} și a lui Hristos, astfel că este o interpretare greșită a lucrării ei. În secțiunile următoare, vom examina datele istorice despre legătura Dr. Kellogg cu doctrina Trinității din perspectiva adevărului adventist despre \emcap{personalitatea lui Dumnezeu}, care a constituit temelia credinței noastre. Cu această perspectivă, credem că datele istorice vor străluci într-o lumină nouă și vor declanșa un dialog onest și constructiv în biserica noastră.


\section*{Correspondence of Dr. Kellogg and Brother Butler}


\section*{Corespondența dintre Dr. Kellogg și Fratele Butler}


In the following section we briefly present you with the well-known correspondence between Dr. Kellogg and G. I. Butler over the book, the Living Temple. Here, we see Dr. Kellogg’s objections regarding the controversy. He wrote to Brother Butler:


În secțiunea următoare vă prezentăm pe scurt corespondența binecunoscută dintre Dr. Kellogg și G. I. Butler despre cartea, Templul viu. Aici, vedem obiecțiile Dr. Kellogg cu privire la controversă. El i-a scris Fratelui Butler:


\others{As far as I can fathom, the \textbf{difficulty }which is found \textbf{in ‘The Living Temple’,} \textbf{the whole thing may be simmered down to the question}: \textbf{\underline{Is the Holy Ghost a person}?} You say no. I had supposed the Bible said this for the reason that the personal pronoun ‘he’ is used in speaking of the Holy Ghost. \textbf{Sister White uses the pronoun ‘he’ and has said in so many words that the Holy Ghost is \underline{the third person of the Godhead}}. \textbf{How the Holy Ghost can be the third person and not be a person at all is difficult for me to see}.}[Letter: J. H. Kellogg to G. I. Butler. Oct 28. 1903][https://static1.squarespace.com/static/554c4998e4b04e89ea0c4073/t/5db9fbc96defed1e45b497a4/1572469707862/1903-10-28-Kellog-to-Butler.pdf]


\others{Din câte pot înțelege, \textbf{dificultatea} care se găsește \textbf{în „Templul viu”,} \textbf{întregul lucru poate fi redus la întrebarea}: \textbf{\underline{Este Duhul Sfânt o persoană}?} Tu spui nu. Am presupus că Biblia spunea aceasta din motivul că pronumele personal „el” este folosit când se vorbește despre Duhul Sfânt. \textbf{Sora White folosește pronumele „el” și a spus în atâtea cuvinte că Duhul Sfânt este \underline{a treia persoană a Dumnezeirii}}. \textbf{Cum poate Duhul Sfânt să fie a treia persoană și să nu fie deloc o persoană este dificil pentru mine să văd}.}[Scrisoare: J. H. Kellogg către G. I. Butler. 28 oct. 1903][https://static1.squarespace.com/static/554c4998e4b04e89ea0c4073/t/5db9fbc96defed1e45b497a4/1572469707862/1903-10-28-Kellog-to-Butler.pdf]


\begin{figure}[hp]
    \centering
    \includegraphics[width=1\linewidth]{images/george-ide-butler.jpg}
    \caption*{George Ide Butler (1834-1918)}
    \label{fig:g-i-butler}
\end{figure}


\begin{figure}[hp]
    \centering
    \includegraphics[width=1\linewidth]{images/george-ide-butler.jpg}
    \caption*{George Ide Butler (1834-1918)}
    \label{fig:g-i-butler}
\end{figure}


According to Dr. Kellogg’s perspective, the whole problem with the book ‘The Living Temple’ comes down to the question “\textit{Is the Holy Ghost a person?}”. Obviously, he does not advocate an impersonal God, as he is often accused of\footnote{Whidden, Woodrow W, et al. \textit{The Trinity : Understanding God's Love, His Plan of Salvation, and Christian Relationships}. Hagerstown, Md, Review And Herald Pub. Association, 2002., p. 217}. Moreover, he even believes that the Holy Ghost is a \textit{third person of the Godhead}. Also, he claims that Brother Butler does not believe that the Holy Ghost is a person. The problem obviously lies in the definition of the word \textit{‘person’}. On this point, Kellogg continues:


Conform perspectivei Dr. Kellogg, întreaga problemă cu cartea „Templul viu” se reduce la întrebarea „\textit{Este Duhul Sfânt o persoană?}”. Evident, el nu pledează pentru un Dumnezeu impersonal, așa cum este adesea acuzat\footnote{Whidden, Woodrow W, et al. \textit{The Trinity : Understanding God's Love, His Plan of Salvation, and Christian Relationships}. Hagerstown, Md, Review And Herald Pub. Association, 2002., p. 217}. Mai mult, el chiar crede că Duhul Sfânt este o \textit{a treia persoană a Dumnezeirii}. De asemenea, el susține că Fratele Butler nu crede că Duhul Sfânt este o persoană. Problema constă evident în definiția cuvântului \textit{„persoană”}. Pe acest punct, Kellogg continuă:


\others{I believe this Spirit of God to be a personality you don’t. But this is purely a question of definition. \textbf{I believe the Spirit of God is a personality}; you say, No, it is not a personality. Now the only reason why we differ is because we \textbf{differ in our ideas as to \underline{what a personality is}}. \textbf{Your idea of personality is perhaps that of \underline{semblance to a person} or a human being}.}[Letter: J. H. Kellogg to G. I. Butler. Oct 28. 1903][https://static1.squarespace.com/static/554c4998e4b04e89ea0c4073/t/5db9fbc96defed1e45b497a4/1572469707862/1903-10-28-Kellog-to-Butler.pdf]


\others{Eu cred că acest Duh al lui Dumnezeu este o personalitate, tu nu. Dar aceasta este pur și simplu o chestiune de definiție. \textbf{Eu cred că Duhul lui Dumnezeu este o personalitate}; tu spui: Nu, nu este o personalitate. Acum, singurul motiv pentru care diferim este pentru că \textbf{diferim în ideile noastre cu privire la \underline{ce este o personalitate}}. \textbf{Ideea ta despre personalitate este poate aceea de \underline{asemănare cu o persoană} sau o ființă umană}.}[Scrisoare: J. H. Kellogg către G. I. Butler. 28 oct. 1903][https://static1.squarespace.com/static/554c4998e4b04e89ea0c4073/t/5db9fbc96defed1e45b497a4/1572469707862/1903-10-28-Kellog-to-Butler.pdf]


Brother Butler replied:


Fratele Butler a răspuns:


\others{\textbf{So far as Sister White and you being in perfect agreement, I shall have to leave that entirely between you and Sister White. \underline{Sister White says there is not perfect agreement; you claim there is}. \underline{I know some of her remarks seem to give you strong ground for claiming that she does}. I am candid enough to say that, but I must give her the credit until she disowns it of saying there is a difference too, and I do not believe you can fully tell just what she means. \underline{God dwells in us by His Holy Spirit}, as a Comforter, as a Reprover, especially the former. When we come to Him we partake of Him in that sense, because the Spirit comes forth from Him; \underline{it comes forth from the Father and the Son}. It is not a person walking around on foot, or flying \underline{as a literal being}, \underline{in any such sense as Christ and the Father are} – at least, if it is, it is utterly beyond my comprehension of the meaning of language or words}.}[Letter: G. I. Butler to J. H. Kellogg. April 5. 1904]


\others{\textbf{În ceea ce privește faptul că Sora White și tu sunteți în acord perfect, va trebui să las aceasta în întregime între tine și Sora White. \underline{Sora White spune că nu există un acord perfect; tu susții că există}. \underline{Știu că unele dintre remarcile ei par să-ți ofere temei solid pentru a susține că ea este de acord}. Sunt suficient de sincer să spun asta, dar trebuie să-i acord credit până când ea o dezice că spune că există și o diferență, și nu cred că poți spune pe deplin ce vrea ea să spună. \underline{Dumnezeu locuiește în noi prin Duhul Său Sfânt}, ca un Mângâietor, ca un Mustrător, în special primul. Când venim la El, luăm parte din El în acest sens, pentru că Duhul vine de la El; \underline{vine de la Tatăl și Fiul}. Nu este o persoană care umblă pe jos sau care zboară \underline{ca o ființă literală}, \underline{în vreun sens asemănător cu Hristos și Tatăl} – cel puțin, dacă este, este cu totul dincolo de înțelegerea mea a sensului limbajului sau cuvintelor}.}[Scrisoare: G. I. Butler către J. H. Kellogg. 5 aprilie 1904]


The given correspondence is crucial for understanding the Kellogg controversy. Kellogg himself stated, \others{the whole thing may be simmered down to the question: \textbf{Is the Holy Ghost a person?}} Similarly Dr. Kellogg wrote to William White: \others{I have been studying very carefully to see what is \textbf{the real root of the difficulty with the Living Temple}, and as far as I can see \textbf{\underline{the whole question} resolves itself into this: \underline{Is the Holy Ghost, a person}?}}[Letter J. H. Kellogg to William White, October 28, 1903][https://drive.google.com/file/d/1\_S4S-Hc0K7Ka8gda9oRhPuAb9XzBTwmb/view] How does Kellogg's conclusion compare to the review and instruction of heavenly origin, which clearly told us that the reasoning in the Living Temple is \egwinline{naught but speculation in regard to \textbf{the personality of God and where His presence is}}[SpTB02 51.3; 1904][https://egwwritings.org/read?panels=p417.262]? In the writings of Ellen White and the pioneers, the term ‘\textit{personality of God}’ refers specifically to the personality of the Father. So, why does Kellogg claim that the real issue is the personality of the Holy Spirit, when God indicated that the issue concerns the personality of the Father?


Corespondența dată este crucială pentru înțelegerea controversei Kellogg. Kellogg însuși a declarat: \others{întregul lucru poate fi redus la întrebarea: \textbf{Este Duhul Sfânt o persoană?}} În mod similar, Dr. Kellogg i-a scris lui William White: \others{Am studiat foarte atent pentru a vedea care este \textbf{adevărata rădăcină a dificultății cu Templul viu}, și din câte pot vedea \textbf{\underline{întreaga chestiune} se rezolvă la aceasta: \underline{Este Duhul Sfânt o persoană}?}}[Scrisoare J. H. Kellogg către William White, 28 octombrie 1903][https://drive.google.com/file/d/1\_S4S-Hc0K7Ka8gda9oRhPuAb9XzBTwmb/view] Cum se compară concluzia lui Kellogg cu revizuirea și instrucțiunea de origine cerească, care ne-a spus clar că raționamentul din Templul viu este \egwinline{nimic altceva decât speculație cu privire la \textbf{personalitatea lui Dumnezeu și unde este prezența Sa}}[SpTB02 51.3; 1904][https://egwwritings.org/read?panels=p417.262]? În scrierile lui Ellen White și ale pionierilor, termenul „\textit{personalitatea lui Dumnezeu}” se referă în mod specific la personalitatea Tatălui. Deci, de ce susține Kellogg că adevărata problemă este personalitatea Duhului Sfânt, când Dumnezeu a indicat că problema privește personalitatea Tatălui?


Many assume that Dr. Kellogg is being manipulative, evading the core issue. However, under a particular premise, his arguments concerning the personality of the Holy Spirit logically support his controversial views on the \emcap{personality of God}. This premise becomes evident within the data itself when we closely follow his reasoning.


Mulți presupun că Dr. Kellogg este manipulator, evitând problema centrală. Cu toate acestea, sub o anumită premisă, argumentele sale privind personalitatea Duhului Sfânt susțin logic opiniile sale controversate despre \emcap{personalitatea lui Dumnezeu}. Această premisă devine evidentă în cadrul datelor înseși când urmărim îndeaproape raționamentul său.


As we have seen earlier, the doctrine on the \emcap{personality of God} teaches that God, the Father, possesses a form—a tangible, material body. Dr. Kellogg concurred that this assertion holds true within the bounds of our finite conception of God\footnote{\href{https://archive.org/details/J.H.Kellogg.TheLivingTemple1903/page/n33/}{Dr. John H. Kellogg, The Living Temple, p.31.}}. However, he argued that, in reality, God transcends our conceptions regarding His form, as He is beyond the constraints of space\footnote{\href{https://archive.org/details/J.H.Kellogg.TheLivingTemple1903/page/n33/}{Dr. John H. Kellogg, The Living Temple, p.33.}}. In this sense, Kellogg effectively does away with the reality of God’s physical, material body. The premise that would validate Dr. Kellogg’s viewpoint is the \textit{exclusive equivalence} in understanding the \emcap{personality of God} and that of the Holy Spirit. Is the Holy Spirit constrained by space? No, He is not. Does the Holy Spirit have a physical body? No! According to Jesus, \bible{for a spirit hath not flesh and bones}[Luke 24:39]. Is the Holy Ghost a person? The answer hinges on our interpretation of what it means to be a person. What is that quality or state of the Holy Spirit being a person?\footnote{Direct application of the definition on the word ‘\textit{personality}’ from the \href{https://www.merriam-webster.com/dictionary/personality}{Merriam Webster Dictionary}} When comparing Dr. Kellogg's belief in the personality of the Holy Spirit with Brother Butler's views, it becomes evident that the quality of the Holy Spirit being a person does not align with \others{that of \textbf{semblance to a person} or a human being}. Butler explicitly stated his criteria for this determination\footnote{In his letter to Dr. Kellogg, Brother Butler further asserted that there is no distinction between the person and the bodily presence. See \href{https://c7da.us/egwdl/Butler\%20to\%20Kellogg\%20Aug121904.pdf}{Letter from Butler to Kellogg, August 12, 1904, p.6}}: \others{\textbf{It is not a person walking around on foot, or flying \underline{as a literal being}, \underline{in any such sense as Christ and the Father are} – at least, if it is, it is utterly beyond my comprehension of the meaning of language or words}}.


După cum am văzut mai devreme, doctrina despre \emcap{personalitatea lui Dumnezeu} învață că Dumnezeu, Tatăl, posedă o formă—un corp tangibil, material. Dr. Kellogg a fost de acord că această afirmație este adevărată în limitele concepției noastre finite despre Dumnezeu\footnote{\href{https://archive.org/details/J.H.Kellogg.TheLivingTemple1903/page/n33/}{Dr. John H. Kellogg, The Living Temple, p.31.}}. Cu toate acestea, el a argumentat că, în realitate, Dumnezeu transcende concepțiile noastre cu privire la forma Sa, deoarece El este dincolo de constrângerile spațiului\footnote{\href{https://archive.org/details/J.H.Kellogg.TheLivingTemple1903/page/n33/}{Dr. John H. Kellogg, The Living Temple, p.33.}}. În acest sens, Kellogg elimină efectiv realitatea corpului fizic, material al lui Dumnezeu. Premisa care ar valida punctul de vedere al Dr. Kellogg este \textit{echivalența exclusivă} în înțelegerea \emcap{personalității lui Dumnezeu} și a celei a Duhului Sfânt. Este Duhul Sfânt constrâns de spațiu? Nu, nu este. Are Duhul Sfânt un corp fizic? Nu! Conform lui Isus, \bible{căci un duh n-are nici carne, nici oase}[Luca 24:39]. Este Duhul Sfânt o persoană? Răspunsul depinde de interpretarea noastră a ceea ce înseamnă să fii o persoană. Care este acea caracteristică sau stare prin care Duhul Sfânt este definit ca persoană?\footnote{Aplicarea directă a definiției cuvântului „\textit{personalitate}” din \href{https://www.merriam-webster.com/dictionary/personality}{Dicționarul Merriam Webster}} Când comparăm credința Dr. Kellogg în personalitatea Duhului Sfânt cu opiniile Fratelui Butler, devine evident că caracteristica prin care Duhul Sfânt este definit ca persoană nu se aliniază cu \others{aceea de \textbf{asemănare cu o persoană} sau o ființă umană}. Butler și-a declarat explicit criteriile pentru această determinare\footnote{În scrisoarea sa către Dr. Kellogg, Fratele Butler a afirmat în continuare că nu există nicio distincție între persoană și prezența corporală. Vezi \href{https://c7da.us/egwdl/Butler\%20to\%20Kellogg\%20Aug121904.pdf}{Scrisoare de la Butler către Kellogg, 12 august 1904, p.6}}: \others{\textbf{Nu este o persoană care umblă pe jos sau care zboară \underline{ca o ființă literală}, \underline{în vreun sens asemănător cu Hristos și Tatăl} – cel puțin, dacă este, este cu totul dincolo de înțelegerea mea a sensului limbajului sau cuvintelor}}.


Have you noticed that Brother Butler addressed Kellogg’s unspoken premise? Butler drew a distinction between the Father and Christ, in relation to the Holy Spirit. Brother Butler is correct. There exists a contrast between the personality of the Holy Spirit and that of God and Christ. Christ and the Father possess a physical form of a person, whereas the Holy Spirit does not. To do away with the physical form of a person of the Father is to \textit{exclusively equate} the understanding of the personality of the Father with that of the Holy Spirit. Kellogg’s approach is compelling, because it was backed by valid arguments regarding the personality of the Holy Spirit.


Ați observat că Fratele Butler a abordat premisa nespusă a lui Kellogg? Butler a făcut o distincție între Tatăl și Hristos, în relație cu Duhul Sfânt. Fratele Butler are dreptate. Există un contrast între personalitatea Duhului Sfânt și cea a lui Dumnezeu și Hristos. Hristos și Tatăl posedă o formă fizică de persoană, în timp ce Duhul Sfânt nu. A elimina forma fizică de persoană a Tatălui înseamnă a \textit{echivala exclusiv} înțelegerea personalității Tatălui cu cea a Duhului Sfânt. Abordarea lui Kellogg este convingătoare, deoarece a fost susținută de argumente valide privind personalitatea Duhului Sfânt.


Let us briefly examine the personality of the Holy Spirit. What is the quality or state of the Holy Spirit being a person?


Să examinăm pe scurt personalitatea Duhului Sfânt. Care este caracteristica sau starea prin care Duhul Sfânt este definit ca persoană?


\egw{\textbf{The Holy Spirit has a personality}, \textbf{\underline{else} }He could not \textbf{bear witness} to our spirits and with our spirits that we are the children of God. \textbf{He must also be a \underline{divine person}}, \textbf{\underline{else}} He could not \textbf{search out the secrets} which lie hidden \textbf{in the mind of God}.}[21LtMs, Ms 20, 1906, par. 32; 1906][https://egwwritings.org/read?panels=p14071.10296041&index=0]


\egw{\textbf{Duhul Sfânt are o personalitate}, \textbf{\underline{altfel} }El nu ar putea \textbf{mărturisi} duhurilor noastre și cu duhurile noastre că suntem copiii lui Dumnezeu. \textbf{El trebuie să fie de asemenea o \underline{persoană divină}}, \textbf{\underline{altfel}} El nu ar putea \textbf{cerceta tainele} care zac ascunse \textbf{în mintea lui Dumnezeu}.}[21LtMs, Ms 20, 1906, par. 32; 1906][https://egwwritings.org/read?panels=p14071.10296041&index=0]


\egw{\textbf{The Holy Spirit is a person}; \textbf{\underline{for}} He \textbf{beareth witness} with our spirits that we are the children of God.}[21LtMs, Ms 20, 1906, par. 31; 1906][https://egwwritings.org/read?panels=p14071.10296040&index=0]


\egw{\textbf{Duhul Sfânt este o persoană}; \textbf{\underline{căci}} El \textbf{mărturisește} cu duhurile noastre că suntem copiii lui Dumnezeu.}[21LtMs, Ms 20, 1906, par. 31; 1906][https://egwwritings.org/read?panels=p14071.10296040&index=0]


The qualities or states that define the Holy Spirit as a person are explicitly mentioned in the provided quotations. These include the ability to bear witness and search out the mind. Further support can be found in Scripture, which attributes actions to the Holy Spirit such as speaking (\textit{Acts 13:2}), teaching (\textit{John 14:26; 1 Corinthians 2:13}), making decisions (\textit{Acts 15:28}), and experiencing emotions (\textit{Ephesians 4:30}), among others. These \textit{qualities }collectively affirm the personality of the Holy Spirit. Can these same qualities be also applied to the Father and the Son? Most certainly. However, unlike the Father and the Son, the Holy Spirit is distinguished by the absence of a material, tangible form. When Ellen White questioned Christ about the \emcap{personality of God}, her inquiry specifically targeted the personal form as the defining quality of the Father's personality.


Calitățile sau stările care definesc Duhul Sfânt ca persoană sunt menționate explicit în citatele furnizate. Acestea includ capacitatea de a mărturisi și de a cerceta mintea. Sprijin suplimentar poate fi găsit în Scriptură, care atribuie acțiuni Duhului Sfânt precum vorbirea (\textit{Faptele Apostolilor 13:2}), învățarea (\textit{Ioan 14:26; 1 Corinteni 2:13}), luarea deciziilor (\textit{Faptele Apostolilor 15:28}) și experimentarea emoțiilor (\textit{Efeseni 4:30}), printre altele. Aceste \textit{calități} afirmă în mod colectiv personalitatea Duhului Sfânt. Pot fi aceste aceleași calități aplicate și Tatălui și Fiului? Cu siguranță. Cu toate acestea, spre deosebire de Tatăl și Fiul, Duhul Sfânt se distinge prin absența unei forme materiale, tangibile. Când Ellen White l-a întrebat pe Hristos despre \emcap{personalitatea lui Dumnezeu}, întrebarea ei a vizat în mod specific forma personală ca fiind calitatea definitorie a personalității Tatălui.


\egw{I have often \textbf{seen }the lovely Jesus, that \textbf{He is a person}. \textbf{I asked Him if His Father \underline{was a person} and \underline{had a form} like Himself}. Said Jesus, ‘\textbf{I am in the express image of My Father's person}.’}[EW 77.1; 1882][https://egwwritings.org/read?panels=p28.490&index=0]


\egw{L-am \textbf{văzut} adesea pe iubitul Isus, că \textbf{El este o persoană}. \textbf{L-am întrebat dacă Tatăl Său \underline{era o persoană} și \underline{avea o formă} ca El Însuși}. Isus a spus: ‘\textbf{Eu sunt întipărirea persoanei Tatălui Meu}.’}[EW 77.1; 1882][https://egwwritings.org/read?panels=p28.490&index=0]


This brings us to a profound distinction in how the personality of the Holy Spirit is understood, as opposed to that of the Father and the Son. Ellen White describes the Holy Spirit as a spiritual manifestation of Christ, drawing a clear line between the outward, visible manifestation of Christ and His spiritual manifestation. This contrast underscores the unique nature of the Holy Spirit's presence and action in the world, distinct from the physical presence of Christ and the Father. Pay attention to the contrast between the outward, visible manifestation of Christ, and His spiritual manifestation:


Aceasta ne aduce la o distincție profundă în modul în care este înțeleasă personalitatea Duhului Sfânt, spre deosebire de cea a Tatălui și a Fiului. Ellen White descrie Duhul Sfânt ca o manifestare spirituală a lui Hristos, trasând o linie clară între manifestarea exterioară, vizibilă a lui Hristos și manifestarea Sa spirituală. Acest contrast subliniază natura unică a prezenței și acțiunii Duhului Sfânt în lume, distinctă de prezența fizică a lui Hristos și a Tatălui. Fiți atenți la contrastul dintre manifestarea exterioară, vizibilă a lui Hristos, și manifestarea Sa spirituală:


\egw{That \textbf{Christ }should \textbf{manifest Himself} to them, and yet \textbf{be invisible to the world}, was a mystery to the disciples. They could not understand \textbf{the words of Christ in their \underline{spiritual sense}}. \textbf{They were thinking of \underline{the outward, visible manifestation}}. They could not take in the fact that they could have \textbf{the presence of Christ with them}, and \textbf{yet He be unseen by the world}. \textbf{They did not understand the meaning of \underline{a spiritual manifestation}}.}[ST November 18, 1897, par. 6; 1897][https://egwwritings.org/read?panels=p820.14727&index=0]


\egw{Că \textbf{Hristos} ar trebui să \textbf{Se manifeste} lor, și totuși să \textbf{fie invizibil pentru lume}, era un mister pentru ucenici. Ei nu puteau înțelege \textbf{cuvintele lui Hristos în sensul lor \underline{spiritual}}. \textbf{Ei se gândeau la \underline{manifestarea exterioară, vizibilă}}. Ei nu puteau înțelege faptul că puteau avea \textbf{prezența lui Hristos cu ei}, și \textbf{totuși El să fie nevăzut de lume}. \textbf{Ei nu înțelegeau sensul \underline{unei manifestări spirituale}}.}[ST November 18, 1897, par. 6; 1897][https://egwwritings.org/read?panels=p820.14727&index=0]


The Holy Spirit is not a person in the physical sense but is manifested in a spiritual sense. If the exclusive understanding of the personality of the Holy Spirit is applied to the Father, then consequently His physical form of a person is done away. His personality is spiritualized. This is why Ellen White critically labeled Kellogg's perspective as spiritualism. Do you know which doctrine, in particular, has a core tenet, that the Father and the Holy Spirit are co-equal in their personalities? It is \textit{the doctrine of the trinity}. Could it be possible that Dr. Kellogg was actually raising the theological side of questions of the trinity?


Duhul Sfânt nu este o persoană în sens fizic, ci este manifestat în sens spiritual. Dacă înțelegerea exclusivă a personalității Duhului Sfânt este aplicată Tatălui, atunci în consecință forma Sa fizică de persoană este eliminată. Personalitatea Sa este spiritualizată. De aceea Ellen White a etichetat critic perspectiva lui Kellogg ca spiritualism. Știți care doctrină, în special, are ca principiu de bază că Tatăl și Duhul Sfânt sunt co-egali în personalitățile lor? Este \textit{doctrina Trinității}. Ar putea fi posibil ca Dr. Kellogg să fi ridicat de fapt partea teologică a întrebărilor despre Trinitate?


\section*{Kellogg’s confession about the Living Temple}


\section*{Mărturisirea lui Kellogg despre Templul viu}


In his interview with G. W. Amadon and A. C. Bourdeau, one month after being disfellowshipped, he confessed that he unintentionally brought the theological side of the question of the Trinity into his book “The Living Temple”.


În interviul său cu G. W. Amadon și A. C. Bourdeau, la o lună după ce a fost exclus din biserică, el a mărturisit că a adus neintenționat partea teologică a întrebării despre Trinitate în cartea sa „Templul viu”.


\others{\textbf{Now, I thought I had cut out entirely the theological side of questions of \underline{the trinity and all that sort of things}}. \textbf{I didn't mean to \underline{put it in} at all}, and I took pains to state in the preface that I did not. I never dreamed of such a thing as \textbf{any theological question being} \textbf{\underline{brought into it}}. I only wanted to show that \textbf{the heart does not beat of its own motion but that it is the power of God that keeps it going}.}[Kellogg vs. The Brethren: His Last Interview as an Adventist, p. 58.][https://forgotten-pillar.s3.us-east-2.amazonaws.com/1990\_kellogg\_vs\_brethren\_lastInterview\_oct7\_1907\_spectrum\_v20\_n3-4.pdf]


\others{\textbf{Acum, am crezut că am eliminat complet partea teologică a întrebărilor despre \underline{Trinitate și toate aceste lucruri}}. \textbf{Nu am intenționat să \underline{o includ} deloc}, și m-am străduit să declar în prefață că nu am făcut-o. Nu am visat niciodată un astfel de lucru ca \textbf{vreo întrebare teologică să fie} \textbf{\underline{adusă în ea}}. Am vrut doar să arăt că \textbf{inima nu bate din propria ei mișcare, ci că este puterea lui Dumnezeu care o menține în funcțiune}.}[Kellogg vs. The Brethren: His Last Interview as an Adventist, p. 58.][https://forgotten-pillar.s3.us-east-2.amazonaws.com/1990\_kellogg\_vs\_brethren\_lastInterview\_oct7\_1907\_spectrum\_v20\_n3-4.pdf]


If we were to look in his book for trinitarian expressions, we would not find any. Would that be a proof that Kellogg is disingenuous in his confession? The only thing we find is the teaching that is stepping off of the foundation of our faith—the \emcap{fundamental principles}—regarding the \emcap{personality of God} and where His presence is. The trinitarian expressions are not there but his sentiments regarding the \emcap{personality of God} are in line with the trinitarian sentiments on God’s person. These sentiments are deceptive and Kellogg was rebuked for them. When he wanted to explicitly state the belief in the Trinity doctrine, in hopes of fixing the book, he was again rebuked by the words, \egwinline{\textbf{Patchwork theories} cannot be accepted by those who are loyal to the faith} and to the \emcap{Fundamental Principles}\footnote{\href{https://egwwritings.org/?ref=en_Lt253-1903.28&para=9980.36}{EGW, Lt253-1903.28; 1903}}. The crucial problem of the Trinity doctrine, in regard to the \emcap{personality of God}, is the underlying assumption that all Three, the Father, the Son, and the Holy Spirit, possess the same type of personality in such a way that They make one monotheistic God. In this light, we may understand Kellogg's assertions over the personality of the Holy Spirit, that the Holy Spirit is the third person of the Godhead. Dr. Kellogg quoted Ellen White when asserting his claims; although he used the same words, he had a wrong sentiment. In light of Dr. Kellogg’s confession, for including \others{\textbf{the theological side of questions of \underline{the trinity}}}, and His assertion that \others{\textbf{the whole thing may be simmered down to the question}: \textbf{\underline{Is the Holy Ghost a person}}?}, we may see the unspoken premise that the Father and the Son are in the same way persons as is the Holy Spirit. This is why Brother Butler wrote to him regarding the personality of the Holy Spirit: \others{\textbf{It is not a person walking around on foot, or flying \underline{as a literal being}, \underline{in any such sense as Christ and the Father are} – at least, if it is, it is utterly beyond my comprehension of the meaning of language or words.}}[Letter from G. I. Butler to J. H. Kellogg, April 5 1904.]


Dacă am căuta în cartea sa expresii trinitariene, nu am găsi niciuna. Ar fi aceasta o dovadă că Kellogg este nesincer în mărturisirea sa? Singurul lucru pe care îl găsim este învățătura care se îndepărtează de la temelia credinței noastre—\emcap{principiile fundamentale}—cu privire la \emcap{personalitatea lui Dumnezeu} și unde este prezența Sa. Expresiile trinitariene nu sunt acolo, dar opiniile sale cu privire la \emcap{personalitatea lui Dumnezeu} sunt în concordanță cu opiniile trinitariene despre persoana lui Dumnezeu. Aceste opinii sunt înșelătoare și Kellogg a fost mustrat pentru ele. Când a vrut să afirme în mod explicit credința în doctrina Trinității, în speranța de a repara cartea, a fost din nou mustrat prin cuvintele, \egwinline{\textbf{Teoriile de peticire} nu pot fi acceptate de cei care sunt loiali credinței} și \emcap{Principiilor Fundamentale}\footnote{\href{https://egwwritings.org/?ref=en_Lt253-1903.28&para=9980.36}{EGW, Lt253-1903.28; 1903}}. Problema crucială a doctrinei Trinității, în ceea ce privește \emcap{personalitatea lui Dumnezeu}, este presupunerea de bază că toți Trei, Tatăl, Fiul și Duhul Sfânt, posedă același tip de personalitate în așa fel încât Ei formează un singur Dumnezeu monoteist. În această lumină, putem înțelege afirmațiile lui Kellogg despre personalitatea Duhului Sfânt, că Duhul Sfânt este a treia persoană a Dumnezeirii. Dr. Kellogg a citat-o pe Ellen White când și-a susținut afirmațiile; deși a folosit aceleași cuvinte, a avut o opinie greșită. În lumina mărturisirii Dr. Kellogg, pentru includerea \others{\textbf{laturii teologice a chestiunilor despre \underline{trinitate}}}, și afirmația sa că \others{\textbf{întreaga chestiune poate fi redusă la întrebarea}: \textbf{\underline{Este Duhul Sfânt o persoană}}?}, putem vedea premisa nespusă că Tatăl și Fiul sunt în același fel persoane precum este Duhul Sfânt. De aceea Fratele Butler i-a scris cu privire la personalitatea Duhului Sfânt: \others{\textbf{Nu este o persoană care umblă pe jos, sau care zboară \underline{ca o ființă literală}, \underline{în același sens în care sunt Hristos și Tatăl} – cel puțin, dacă este, este cu totul dincolo de înțelegerea mea a sensului limbajului sau cuvintelor.}}[Scrisoare de la G. I. Butler către J. H. Kellogg, 5 aprilie 1904.]


\section*{The presence of God manifested in nature}


\section*{Prezența lui Dumnezeu manifestată în natură}


From the works of our pioneers, we have seen that the personality of the Holy Ghost is most clearly expressed in terms of God's presence. Sister White told us that the Living Temple \egwinline{introduces that which is naught but speculation in \textbf{regard to the personality of God and where His presence is}.}[SpTB02 51.3; 1904][https://egwwritings.org/read?panels=p417.262] The \emcap{personality of God} and where His presence is are two mutually inclusive doctrines; one affirms the other. Deny one, and you deny the other. This notion is clearly seen in the book, the Living Temple. In the previous sections, we read Kellogg's arguments for the \emcap{personality of God} taken from his book. He argued that it is unprofitable to talk about God's shape or any tangible form. He raised skepticism in the reality of God as a definite, material, and tangible Being. If God is spirit, possessing no form nor body, then He is not restricted in His presence to one locality; this was the sentiment Kellogg advocated in the Living Temple.


Din lucrările pionierilor noștri, am văzut că personalitatea Duhului Sfânt este cel mai clar exprimată în termenii prezenței lui Dumnezeu. Sora White ne-a spus că Templul viu \egwinline{introduce ceea ce nu este altceva decât speculație cu privire la \textbf{personalitatea lui Dumnezeu și unde este prezența Sa}.}[SpTB02 51.3; 1904][https://egwwritings.org/read?panels=p417.262] \emcap{Personalitatea lui Dumnezeu} și unde este prezența Sa sunt două doctrine care se includ reciproc; una o afirmă pe cealaltă. Neagă una și o negi pe cealaltă. Această noțiune este văzută clar în cartea Templul viu. În secțiunile anterioare, am citit argumentele lui Kellogg pentru \emcap{personalitatea lui Dumnezeu} luate din cartea sa. El a argumentat că este neprofitabil să vorbim despre forma lui Dumnezeu sau despre orice formă tangibilă. El a ridicat scepticism în realitatea lui Dumnezeu ca Ființă definită, materială și tangibilă. Dacă Dumnezeu este spirit, neavând nici formă, nici trup, atunci El nu este restricționat în prezența Sa la o singură localitate; aceasta a fost opinia pe care Kellogg a susținut-o în Templul viu.


\others{Says one, ‘\textbf{God may be \underline{present by his Spirit}, or by his power, but \underline{certainly God himself} cannot be present everywhere at once}.’ We answer: How can power be separated from the source of power? \textbf{Where God's Spirit is at work}, where God's power is manifested, \textbf{God \underline{himself} is actually and truly present}…}[John H. Kellogg, The Living Temple, p.28.][https://archive.org/details/J.H.Kellogg.TheLivingTemple1903/page/n29/]


\others{Spune cineva, ‘\textbf{Dumnezeu poate fi \underline{prezent prin Duhul Său}, sau prin puterea Sa, dar \underline{cu siguranță Dumnezeu însuși} nu poate fi prezent pretutindeni deodată}.’ Răspundem: Cum poate fi puterea separată de sursa puterii? \textbf{Unde lucrează Duhul lui Dumnezeu}, unde se manifestă puterea lui Dumnezeu, \textbf{Dumnezeu \underline{însuși} este efectiv și cu adevărat prezent}...}[John H. Kellogg, Templul viu, p.28.][https://archive.org/details/J.H.Kellogg.TheLivingTemple1903/page/n29/]


When Dr. Kellogg wrote \others{Says one, ‘God may be present by His Spirit…’}, he referred to the sentiments of our pioneers who were loyal to the \emcap{Fundamental Principles}. This is the most obvious point where Dr. Kellogg stepped off from the \emcap{Fundamental Principles}. Is this step in harmony with the doctrine of the Trinity? Examining our current stance in the Fundamental Beliefs \#2, we see that one God, as a unity of three persons, is not everywhere present through the agency of the Holy Spirit, but rather is everywhere present by Himself.


Când Dr. Kellogg a scris \others{Spune cineva, ‘Dumnezeu poate fi prezent prin Duhul Său...’}, el s-a referit la opiniile pionierilor noștri care erau loiali \emcap{Principiilor Fundamentale}. Acesta este cel mai evident punct unde Dr. Kellogg s-a îndepărtat de \emcap{Principiile Fundamentale}. Este acest pas în armonie cu doctrina Trinității? Examinând poziția noastră actuală în Punctele Fundamentale de Credință \#2, vedem că un singur Dumnezeu, ca o unitate a trei persoane, nu este prezent pretutindeni prin mijlocirea Duhului Sfânt, ci mai degrabă este prezent pretutindeni prin El însuși.


\others{There is \textbf{one God}: Father, Son, and Holy Spirit, \textbf{a unity of three} coeternal \textbf{Persons}. God is immortal, all-powerful… and \textbf{ever present}.}[Fundamental Beliefs of Seventh-day Adventist, \#2 Trinity; 2020 Edition][https://www.adventist.org/wp-content/uploads/2020/06/ADV-28Beliefs2020.pdf]


\others{Există \textbf{un singur Dumnezeu}: Tatăl, Fiul și Duhul Sfânt, \textbf{o unitate a trei} \textbf{Persoane} coeterne. Dumnezeu este nemuritor, atotputernic... și \textbf{mereu prezent}.}[Punctele Fundamentale de Credință ale Adventiștilor de Ziua a Șaptea, \#2 Trinitatea; Ediția 2020][https://www.adventist.org/wp-content/uploads/2020/06/ADV-28Beliefs2020.pdf]


\section*{Dr. Kellogg's perception of God}


\section*{Percepția Dr. Kellogg despre Dumnezeu}


In examining the surrounding controversy over the Living Temple, we truly see that Dr. Kellogg raised \others{the theological side of questions of the trinity.}[Kellogg vs. The Brethren: His Last Interview as an Adventist, p. 58.][https://forgotten-pillar.s3.us-east-2.amazonaws.com/1990\_kellogg\_vs\_brethren\_lastInterview\_oct7\_1907\_spectrum\_v20\_n3-4.pdf] Another question we raise in examining Kellogg's sentiments with the \emcap{Fundamental Principles} is whom does he address in terms of “\textit{one God}”? There is no data to directly answer that question, but there is plenty of data which suggests that Dr. Kellogg's understanding of “\textit{one God}” was a Trinitarian understanding. His letter to W. W. Prescott is one piece of evidence supporting that notion:


Examinând controversa din jurul Templului viu, vedem cu adevărat că Dr. Kellogg a ridicat \others{latura teologică a chestiunilor despre trinitate.}[Kellogg vs. Frații: Ultimul său interviu ca adventist, p. 58.][https://forgotten-pillar.s3.us-east-2.amazonaws.com/1990\_kellogg\_vs\_brethren\_lastInterview\_oct7\_1907\_spectrum\_v20\_n3-4.pdf] O altă întrebare pe care o ridicăm examinând opiniile lui Kellogg cu \emcap{Principiile Fundamentale} este la cine se adresează el în termenii de “\textit{un singur Dumnezeu}”? Nu există date pentru a răspunde direct la această întrebare, dar există multe date care sugerează că înțelegerea Dr. Kellogg despre “\textit{un singur Dumnezeu}” era o înțelegere trinitariană. Scrisoarea sa către W. W. Prescott este o dovadă care susține această noțiune:


\others{The difference is this: \textbf{When we say God} is in the tree, the word ‘\textbf{God}’ \textbf{is understood in its most comprehensive sense}, and people understand the meaning to be \textbf{that the Godhead} is in the tree, \textbf{God the Father, God the Son, and God the Holy Spirit}, whereas the proper understanding in order \textbf{that wholesome conceptions} should be preserved in our minds, is that God the Father sits upon his throne in heaven where God the Son is also; \textbf{while God's life, or spirit or presence is the all-pervading power which is carrying out the will of God in all the universe}.}[Letter: Dr. Kellogg to Prof. W. W. Prescott, Oct. 25, 1903][https://forgotten-pillar.s3.us-east-2.amazonaws.com/1903-10-25-JHKellogg-to-W.W.Prescott.pdf]


\others{Diferența este aceasta: \textbf{Când spunem că Dumnezeu} este în copac, cuvântul ‘\textbf{Dumnezeu}’ \textbf{este înțeles în sensul său cel mai cuprinzător}, și oamenii înțeleg că sensul este \textbf{că Dumnezeirea} este în copac, \textbf{Dumnezeu Tatăl, Dumnezeu Fiul și Dumnezeu Duhul Sfânt}, în timp ce înțelegerea corectă pentru \textbf{ca concepțiile sănătoase} să fie păstrate în mintea noastră, este că Dumnezeu Tatăl șade pe tronul Său în cer unde este și Dumnezeu Fiul; \textbf{în timp ce viața lui Dumnezeu, sau spiritul sau prezența Sa este puterea atotpătrunzătoare care îndeplinește voia lui Dumnezeu în tot universul}.}[Scrisoare: Dr. Kellogg către Prof. W. W. Prescott, 25 oct. 1903][https://forgotten-pillar.s3.us-east-2.amazonaws.com/1903-10-25-JHKellogg-to-W.W.Prescott.pdf]


In the next chapter, we will make our case: if the given \others{wholesome conception} of God advocated by Dr. Kellogg was true, then his clarification of the Holy Spirit being \others{God's life, or spirit or presence is the all-pervading power which is carrying out the will of God in all the universe} would truly solve the entire difficulty of the Living Temple. But that was not the case. Dr. Kellogg's true problem was his perception of God, and his trinitarian stance was not solving the real issue—the \emcap{personality of God}.


În capitolul următor, vom prezenta cazul nostru: dacă \others{concepția sănătoasă} dată despre Dumnezeu susținută de Dr. Kellogg ar fi fost adevărată, atunci clarificarea sa că Duhul Sfânt este \others{viața lui Dumnezeu, sau spiritul sau prezența Sa este puterea atotpătrunzătoare care îndeplinește voia lui Dumnezeu în tot universul} ar rezolva cu adevărat întreaga dificultate a Templului viu. Dar nu a fost cazul. Adevărata problemă a Dr. Kellogg era percepția sa despre Dumnezeu, iar poziția sa trinitariană nu rezolva problema reală—\emcap{personalitatea lui Dumnezeu}.


There is another revealing letter showing us the consequences of raising \others{the theological side of questions of the trinity.} Writing to his friend Dr. Hayward, Dr. Kellogg reflected:


Există o altă scrisoare revelatoare care ne arată consecințele ridicării \others{aspectului teologic al chestiunilor legate de trinitate.} Scriind prietenului său Dr. Hayward, Dr. Kellogg reflecta:


\others{\textbf{These theologians} have sought to darken the minds of the people and to make \textbf{this sweet and beautiful truth \underline{appear loathsome} to them, by drawing into it \underline{the old controversy about the Trinity}}.}


\others{\textbf{Acești teologi} au căutat să întunece mințile oamenilor și să facă \textbf{acest adevăr dulce și frumos să \underline{pară dezgustător} pentru ei, atrăgând în el \underline{vechea controversă despre Trinitate}}.}


\othersnogap{I never raised the question as to \textbf{which part of God is present in a man}, whether it was \textbf{God, the Father};\textbf{ God, the Son}; or \textbf{God, the Holy Spirit}. The only point was that it is God and not man.}[Letter: Dr. J. H. Kellogg to Dr. Hayward, Aug., 15. 1905][https://forgotten-pillar.s3.us-east-2.amazonaws.com/1903-08-15-kellogg-to-hayward.pdf]


\othersnogap{Nu am ridicat niciodată întrebarea cu privire la \textbf{care parte a lui Dumnezeu este prezentă într-un om}, dacă era \textbf{Dumnezeu Tatăl};\textbf{ Dumnezeu Fiul}; sau \textbf{Dumnezeu Duhul Sfânt}. Singurul punct era că este Dumnezeu și nu omul.}[Scrisoare: Dr. J. H. Kellogg către Dr. Hayward, 15 august 1905][https://forgotten-pillar.s3.us-east-2.amazonaws.com/1903-08-15-kellogg-to-hayward.pdf]


Here we see the tensions between Dr. Kellogg and certain Seventh-day Adventist theologians of that time, where Dr. Kellogg's \others{sweet and beautiful truth} of God's divine immanence got entangled with \others{the old controversy about the Trinity}. This tells us that in the days of Dr. Kellogg, the doctrine of Trinity was controversial, and certainly it was not regarded as something positive, but rather as something which made Kellogg's teachings \others{loathsome}. But who were these theologians Dr. Kellogg referred to? He did not name anyone in his letter to Dr. Hayward, but we can get the idea of whom \others{these theologians} were based on his letter sent 10 days earlier to I. G. Butler\footnote{\href{https://forgotten-pillar.s3.us-east-2.amazonaws.com/1905-08-05-kellogg-butler.pdf}{Letter: J. H. Kellogg to I. G. Butler, Aug., 5. 1905}}, venting his frustration with the General Conference's bidding with him. These were A. G. Daniells, W. C. White, and W. W. Prescott. We can also include G. I. Butler himself to that group, since he also was a theologian participating in this \others{old controversy about the Trinity}. All of these people held leading positions within the Seventh-day Adventist church, and all of them were non-Trinitarians. The argument is being made that the issue with Dr. Kellogg's teaching lies somewhere other than his trinitarian sentiments, because supposedly the church was trinitarian at that time, and supposedly Ellen White was trinitarian herself. \footnote{This is currently the popular narrative promoted by laity.} If this was the case, and in this mix of truth and error, should we not have at least some defense of the trinity doctrine, dissecting it from error? We have not found any such data. Instead, all data we have is in defense of the \emcap{Fundamental Principles}, and the doctrine on the presence and the \emcap{personality of God}, which both are opposed to the doctrine of the Trinity. Ellen White said of the truth: the Trinity doctrine \egwinline{cannot be accepted by those who are \textbf{loyal to the faith and to the principles} that have withstood all the opposition of satanic influences.}[Lt253-1903.28; 1903][https://egwwritings.org/read?panels=p14068.9980036]


Aici vedem tensiunile dintre Dr. Kellogg și anumiți teologi adventiști de ziua a șaptea din acea vreme, unde \others{adevărul dulce și frumos} al Dr. Kellogg despre imanența divină a lui Dumnezeu s-a încurcat cu \others{vechea controversă despre Trinitate}. Aceasta ne spune că în zilele Dr. Kellogg, doctrina Trinității era controversată și cu siguranță nu era privită ca ceva pozitiv, ci mai degrabă ca ceva care făcea învățăturile lui Kellogg \others{dezgustătoare}. Dar cine erau acești teologi la care se referea Dr. Kellogg? El nu a numit pe nimeni în scrisoarea sa către Dr. Hayward, dar putem să ne facem o idee despre cine erau \others{acești teologi} pe baza scrisorii sale trimise cu 10 zile mai devreme către I. G. Butler\footnote{\href{https://forgotten-pillar.s3.us-east-2.amazonaws.com/1905-08-05-kellogg-butler.pdf}{Scrisoare: J. H. Kellogg către I. G. Butler, 5 august 1905}}, exprimându-și frustrarea față de cerințele Conferinței Generale față de el. Aceștia erau A. G. Daniells, W. C. White și W. W. Prescott. Putem include și pe G. I. Butler însuși în acel grup, deoarece și el era un teolog care participa la această \others{veche controversă despre Trinitate}. Toți acești oameni dețineau poziții de conducere în cadrul bisericii adventiste de ziua a șaptea și toți erau non-trinitarieni. Se susține argumentul că problema cu învățătura Dr. Kellogg se află în altă parte decât în opiniile sale trinitariene, deoarece se presupune că biserica era trinitariană în acel moment și se presupune că Ellen White însăși era trinitariană. \footnote{Aceasta este în prezent narațiunea populară promovată de laici.} Dacă acesta era cazul, și în acest amestec de adevăr și eroare, nu ar trebui să avem cel puțin o apărare a doctrinei trinității, disecând-o de eroare? Nu am găsit astfel de date. În schimb, toate datele pe care le avem sunt în apărarea \emcap{Principiilor Fundamentale} și a doctrinei despre prezența și \emcap{personalitatea lui Dumnezeu}, care ambele sunt opuse doctrinei Trinității. Ellen White a spus despre adevăr: doctrina Trinității \egwinline{nu poate fi acceptată de cei care sunt \textbf{loiali credinței și principiilor} care au rezistat tuturor opoziției influențelor satanice.}[Lt253-1903.28; 1903][https://egwwritings.org/read?panels=p14068.9980036]


In this short reflection on differences between Dr. Kellogg's sentiments and the \emcap{Fundamental Principles} from which he stepped off, we can recognize the following characteristics which are akin to the Trinity doctrine:


În această scurtă reflecție asupra diferențelor dintre opiniile Dr. Kellogg și \emcap{Principiile Fundamentale} de la care s-a îndepărtat, putem recunoaște următoarele caracteristici care sunt asemănătoare cu doctrina Trinității:


\begin{itemize}
    \item The word ‘God’ represents the wholesome conception of God as God the Father, God the Son, and God the Holy Spirit.
    \item God is everywhere present by Himself.
    \item The quality or state of the Father being a person is equalized to that of the Holy Spirit.\footnote{\href{https://www.adventist.org/wp-content/uploads/2020/06/ADV-28Beliefs2020.pdf}{Fundamental Beliefs \#5}: \others{He \normaltext{[the Holy Spirit]} \textbf{is as much a person} \underline{as} are \textbf{the Father} and the Son}; \href{https://www.adventist.org/wp-content/uploads/2020/06/ADV-28Beliefs2020.pdf}{Fundamental Beliefs \#3}: \others{\textbf{The qualities} and powers \textbf{exhibited in} the Son and \textbf{the Holy Spirit are \underline{also} those of the Father}}}
\end{itemize}


\begin{itemize}
    \item Cuvântul ‘Dumnezeu’ reprezintă concepția completă a lui Dumnezeu ca Dumnezeu Tatăl, Dumnezeu Fiul și Dumnezeu Duhul Sfânt.
    \item Dumnezeu este prezent pretutindeni prin Sine Însuși.
    \item Caracteristica sau starea prin care Tatăl este definit ca persoană este egalizată cu cea a Duhului Sfânt.\footnote{\href{https://www.adventist.org/wp-content/uploads/2020/06/ADV-28Beliefs2020.pdf}{Puncte Fundamentale de Credință \#5}: \others{El \normaltext{[Duhul Sfânt]} \textbf{este tot atât de mult o persoană} \underline{ca} și \textbf{Tatăl} și Fiul}; \href{https://www.adventist.org/wp-content/uploads/2020/06/ADV-28Beliefs2020.pdf}{Puncte Fundamentale de Credință \#3}: \others{\textbf{Calitățile} și puterile \textbf{manifestate în} Fiul și \textbf{Duhul Sfânt sunt \underline{de asemenea} cele ale Tatălui}}}
\end{itemize}


These three characteristics of Dr. Kellogg's sentiments depart from the foundation of our faith—the \emcap{Fundamental Principles}—but are in harmony with the teachings of the Trinity. In saying this, we are not claiming that Dr. Kellogg is responsible for the acceptance of the Trinity doctrine into our ranks, but rather that the Trinity doctrine was Kellogg's justification for stepping off from the foundation of our faith, established at the beginning of our work. The true problem was \textit{stepping off} from the \emcap{fundamental principles}, and both Dr. Kellogg and we as a church have made those steps. The difference is that Dr. Kellogg landed in pantheism, while we landed on the \#2 point of the Fundamental Beliefs.


Aceste trei caracteristici ale opiniilor Dr. Kellogg se îndepărtează de la temelia credinței noastre—\emcap{Principiile Fundamentale}—dar sunt în armonie cu învățăturile Trinității. Spunând aceasta, nu susținem că Dr. Kellogg este responsabil pentru acceptarea doctrinei Trinității în rândurile noastre, ci mai degrabă că doctrina Trinității a fost justificarea lui Kellogg pentru îndepărtarea de la temelia credinței noastre, stabilită la începutul lucrării noastre. Adevărata problemă a fost \textit{îndepărtarea} de la \emcap{principiile fundamentale}, și atât Dr. Kellogg, cât și noi ca biserică am făcut acești pași. Diferența este că Dr. Kellogg a ajuns în panteism, în timp ce noi am ajuns la punctul \#2 al Punctelor Fundamentale de Credință.


In the following chapter, we will examine Dr. Kellogg's teaching that God sustains all life, and how this truth, in combination with a false perception of God and His personality, led him to become a pantheist.


În capitolul următor, vom examina învățătura Dr. Kellogg că Dumnezeu susține toată viața și cum acest adevăr, în combinație cu o percepție falsă despre Dumnezeu și personalitatea Sa, l-a condus să devină panteist.


% Dr. Kellogg and the Trinity doctrine

\begin{titledpoem}
    
    \stanza{
        In Kellogg’s quest, the question posed, \\
        "The Spirit – how is He composed?" \\
        The issue stirred a great debate, \\
        How does this mystery relate?
    }

    \stanza{
        The question was beyond the seen \\
        To trinity J.H. did lean \\
        The Father wasn’t bound by space? \\
        Without a body or a face?
    }

    \stanza{
        To Ellen, Jesus did inform \\
        Like Him, His Father had a form \\
        “I am His image as express, \\
        Revealing form and righteousness.”
    }

    \stanza{
        In vision was the truth revealed \\
        The inspiration, it was sealed \\
        The Father’s form upon the throne \\
        And Christ with form just like His own.
    }

    \stanza{
        The Spirit’s personality \\
        In actions and in quality \\
        A role distinct, within us dwells. \\
        The mind of Christ the Spirit tells.
    }

    \stanza{
        God’s power and His presence show \\
        Wherever God would have it go \\
        And thus, He’s present everywhere \\
        Invisible, His Spirit there.
    }

    \stanza{
        The Living Temple showed a flaw \\
        A dangerous error Ellen saw \\
        The wayward theories in his mind \\
        Blocked him from truth he could not find.
    }

    \stanza{
        He went off searching on his own \\
        And did not follow what was shown \\
        If he had stayed where God had led, \\
        His teaching would have never spread.
    }
    
\end{titledpoem}