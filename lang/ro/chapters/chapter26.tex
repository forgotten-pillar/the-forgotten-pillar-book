% \qrchapter{https://forgottenpillar.com/rsc/en-fp-chapter26}{The steps to Omega}


\qrchapter{https://forgottenpillar.com/rsc/ro-fp-chapter26}{Pașii către Omega}


In our study so far, we have seen evidence that Kellogg’s controversy was connected to the Trinity doctrine and the \emcap{personality of God} expressed in the first point of the \emcap{Fundamental Principles}. Unfortunately, today we do not stand on that foundation regarding the \emcap{personality of God}; we have built another foundation that has changed the truth on the \emcap{personality of God} to a mysterious Triune God. Sister White was clearly against this reorganization and she prophesied that in the closing of His work, God will rehearse the history of the Advent movement and re-establish every pillar of our faith that was held in the beginning.


În studiul nostru de până acum, am văzut dovezi că controversa lui Kellogg era legată de doctrina Trinității și de \emcap{personalitatea lui Dumnezeu} exprimată în primul punct al \emcap{Principiilor Fundamentale}. Din păcate, astăzi nu stăm pe acea temelie cu privire la \emcap{personalitatea lui Dumnezeu}; am construit o altă temelie care a schimbat adevărul despre \emcap{personalitatea lui Dumnezeu} într-un Dumnezeu Trinitar misterios. Sora White a fost în mod clar împotriva acestei reorganizări și ea a profețit că în încheierea lucrării Sale, Dumnezeu va repeta istoria mișcării adventiste și va restabili fiecare stâlp al credinței noastre care a fost susținut la început.


\egw{\textbf{\underline{The Lord has declared that the history of the past shall be rehearsed as we enter upon the closing work}. \underline{Every truth} that He has given for these last days is to be proclaimed to the world. \underline{Every pillar} that He has established \underline{is to be strengthened}. We cannot now step off the foundation that God has established. We cannot now enter into any new organization; for this would mean apostasy from the truth}.}[Ms129-1905.6; 1905][https://egwwritings.org/read?panels=p9797.13]


\egw{\textbf{\underline{Domnul a declarat că istoria trecutului va fi repetată pe măsură ce intrăm în lucrarea de încheiere}. \underline{Fiecare adevăr} pe care El l-a dat pentru aceste zile din urmă trebuie să fie proclamat lumii. \underline{Fiecare stâlp} pe care El l-a stabilit \underline{trebuie să fie întărit}. Nu putem acum să pășim în afara temeliei pe care Dumnezeu a stabilit-o. Nu putem acum să intrăm în vreo organizație nouă; căci aceasta ar însemna apostazie de la adevăr}.}[Ms129-1905.6; 1905][https://egwwritings.org/read?panels=p9797.13]


Comparing the \emcap{Fundamental Principles} with the current Fundamental Beliefs of Seventh-day Adventists, we see that we have entered into a new organization. God’s warning, given through Sister White, to re-establish all pillars of our faith in these last days, is becoming imperative. As we traced the Trinity doctrine from Kellogg's controversy, we came across Ellen White’s warnings against alpha and omega apostasy, which will enter into our church.


Comparând \emcap{Principiile Fundamentale} cu actualele Puncte Fundamentale de Credință ale Adventiștilor de Ziua a Șaptea, vedem că am intrat într-o organizație nouă. Avertizarea lui Dumnezeu, dată prin Sora White, de a restabili toți stâlpii credinței noastre în aceste zile din urmă, devine imperativă. Pe măsură ce am urmărit doctrina Trinității de la controversa lui Kellogg, am întâlnit avertizările lui Ellen White împotriva apostaziei alfa și omega, care va intra în biserica noastră.


\egw{\textbf{‘Living Temple’ contains the alpha of these theories. I knew that \underline{the omega would follow in a little while}; and I trembled for our people}. I knew that \textbf{I must warn our brethren and sisters not to enter into controversy \underline{over the presence and personality of God}. The statements made in ‘Living Temple’ \underline{in regard to this point are incorrect}. }The scripture used to substantiate the doctrine there set forth, is scripture misapplied.}[SpTB02 53.2; 1904][https://egwwritings.org/read?panels=p417.271]


\egw{\textbf{„Templul viu” conține alfa acestor teorii. Știam că \underline{omega va urma într-un timp scurt}; și am tremurat pentru poporul nostru}. Știam că \textbf{trebuie să îi avertizez pe frații și surorile noastre să nu intre în controversă \underline{asupra prezenței și personalității lui Dumnezeu}. Afirmațiile făcute în „Templul viu” \underline{cu privire la acest punct sunt incorecte}. }Scriptura folosită pentru a susține doctrina prezentată acolo, este Scriptură aplicată greșit.}[SpTB02 53.2; 1904][https://egwwritings.org/read?panels=p417.271]


In the context of Seventh-day Adventist reorganization, we identify several steps that were necessary to accomplish this reorganization and are necessary to uphold it.


În contextul reorganizării Adventiștilor de Ziua a Șaptea, identificăm mai mulți pași care au fost necesari pentru a realiza această reorganizare și sunt necesari pentru a o susține.


\subsection*{Step 1: Deny the Fundamental Principles to be the foundation of our faith and the official, and accurate, representation of Seventh-day Adventist beliefs}


\subsection*{Pasul 1: Negarea Principiilor Fundamentale ca fiind temelia credinței noastre și reprezentarea oficială și exactă a credințelor Adventiștilor de Ziua a Șaptea}


The first step necessary is to hide the original foundation of our faith by unlinking it with the \emcap{Fundamental Principles}.


Primul pas necesar este să ascundem temelia originală a credinței noastre prin deconectarea ei de \emcap{Principiile Fundamentale}.


\egw{\textbf{As a people, we are to \underline{stand firm on the platform of eternal truth} that has withstood test and trial. We are to \underline{hold to the sure pillars of our faith}. \underline{The principles of truth} that God has revealed to us \underline{are our only true foundation}. They have made us what we are. The lapse of time has not lessened their value. \underline{It is the constant effort of the enemy to remove these truths from their setting}, and to put in their place \underline{spurious theories}. He \underline{will bring in} everything that he possibly can to carry out his deceptive designs.}}[SpTB02 51.2; 1904][https://egwwritings.org/read?panels=p417.261]


\egw{\textbf{Ca popor, trebuie să \underline{stăm fermi pe platforma adevărului etern} care a rezistat testului și încercării. Trebuie să \underline{ținem de stâlpii siguri ai credinței noastre}. \underline{Principiile adevărului} pe care Dumnezeu ni le-a revelat \underline{sunt singura noastră temelie adevărată}. Ele ne-au făcut ceea ce suntem. Trecerea timpului nu le-a diminuat valoarea. \underline{Este efortul constant al vrăjmașului de a îndepărta aceste adevăruri din cadrul lor}, și de a pune în locul lor \underline{teorii false}. El \underline{va aduce} tot ce îi este posibil pentru a-și duce la îndeplinire planurile înșelătoare.}}[SpTB02 51.2; 1904][https://egwwritings.org/read?panels=p417.261]


\egw{\textbf{Messages of every order and kind have been urged upon Seventh-day Adventists, to take the place of the truth which, \underline{point by point}, has been sought out by prayerful study, and testified to by the miracle-working power of the Lord}. \textbf{But \underline{the way-marks} \underline{which have made us what we are}, \underline{are to be preserved}, and they \underline{will be preserved}, as God has signified through His word and the testimony of His Spirit}. \textbf{He calls upon us to \underline{hold firmly}, with the grip of faith, to \underline{the fundamental principles} that are \underline{based upon unquestionable authority}}.}[SpTB02 59.1; 1904][https://egwwritings.org/read?panels=p417.299]


\egw{\textbf{Mesaje de orice fel și natură au fost impuse asupra Adventiștilor de Ziua a Șaptea, pentru a lua locul adevărului care, \underline{punct cu punct}, a fost căutat prin studiu plin de rugăciune și mărturisit prin puterea făcătoare de minuni a Domnului}. \textbf{Dar \underline{pietrele de hotar} \underline{care ne-au făcut ceea ce suntem}, \underline{trebuie să fie păstrate}, și ele \underline{vor fi păstrate}, așa cum Dumnezeu a indicat prin cuvântul Său și mărturia Duhului Său}. \textbf{El ne cheamă să \underline{ținem ferm}, cu strânsoarea credinței, de \underline{principiile fundamentale} care sunt \underline{bazate pe autoritate de necontestat}}.}[SpTB02 59.1; 1904][https://egwwritings.org/read?panels=p417.299]


The \emcap{Fundamental Principles} were the truths God revealed to the pioneers after the passing of time in 1844. We have seen the testimonies of our pioneers, including Ellen White, regarding the first point of the \emcap{Fundamental Principles}. All of them were in harmony regarding these particular points of our faith. In 1863, Seventh-day Adventists organized themselves into a church, as an organized body. Since then, many were misrepresenting the position of the Seventh-day Adventist Church and the pioneers found it necessary to meet inquiries, \others{and sometimes to correct false statements circulated against} the church’s beliefs and practices. Consequently, in 1872, the pioneers issued the document called “\textit{A Declaration of the Fundamental Principles, Taught and Practiced by the Seventh-Day Adventists}”\footnote{“A Declaration of the Fundamental Principles, Taught and Practiced by the Seventh-Day Adventists (1872) : MVT : Free Download, Borrow, and Streaming : Internet Archive.” Internet Archive, 2025, \href{https://archive.org/details/ADeclarationOfTheFundamentalPrinciplesTaughtAndPracticedByThe}{archive.org/details/ADeclarationOfTheFundamentalPrinciplesTaughtAndPracticedByThe}. Accessed 3 Feb. 2025.}. This declaration presented the public with \others{a brief statement of what is, and has been, with great unanimity, held by}[The preface of the Fundamental Principles in 1872.] Seventh-day Adventists.


\emcap{Principiile Fundamentale} au fost adevărurile pe care Dumnezeu le-a descoperit pionierilor după trecerea timpului din 1844. Am văzut mărturiile pionierilor noștri, inclusiv Ellen White, cu privire la primul punct al \emcap{Principiilor Fundamentale}. Toți erau în armonie cu privire la aceste puncte particulare ale credinței noastre. În 1863, adventiștii de ziua a șaptea s-au organizat într-o biserică, ca un corp organizat. De atunci, mulți au prezentat greșit poziția Bisericii Adventiste de Ziua a Șaptea și pionierii au considerat necesar să răspundă întrebărilor, \others{și uneori să corecteze afirmațiile false răspândite împotriva} credințelor și practicilor bisericii. În consecință, în 1872, pionierii au emis documentul numit „\textit{O declarație a principiilor fundamentale, crezute și practicate de adventiștii de ziua a șaptea}”\footnote{“A Declaration of the Fundamental Principles, Taught and Practiced by the Seventh-Day Adventists (1872) : MVT : Free Download, Borrow, and Streaming : Internet Archive.” Internet Archive, 2025, \href{https://archive.org/details/ADeclarationOfTheFundamentalPrinciplesTaughtAndPracticedByThe}{archive.org/details/ADeclarationOfTheFundamentalPrinciplesTaughtAndPracticedByThe}. Accesat 3 feb. 2025.}. Această declarație a prezentat publicului \others{o scurtă expunere a ceea ce este și a fost, cu mare unanimitate, susținut de}[Prefața Principiilor Fundamentale din 1872.] adventiștii de ziua a șaptea.


In the chapter “\hyperref[chap:authority]{The Authority of the Fundamental Principles}”, we discussed how pro-Trinitarian scholars have been compromising the authority of the \emcap{Fundamental Principles}, denying their true value in our Adventist history.


În capitolul „\hyperref[chap:authority]{Autoritatea Principiilor Fundamentale}”, am discutat cum învățații pro-trinitarieni au compromis autoritatea \emcap{Principiilor Fundamentale}, negând adevărata lor valoare în istoria noastră adventistă.


Pro-trinitarian scholars argue that this declaration was not what it claims to be—a declaration of the \emcap{fundamental principles}, taught and practiced by the Seventh-day Adventists. This declaration was a summary of the principal features of Adventist’s faith, and no point is really as problematic or objectionable as the first point, dealing with the \emcap{personality of God} and where His presence is. But the evidence in favor of the \emcap{Fundamental Principles}, especially to the first point, is overwhelming.


Învățații pro-trinitarieni susțin că această declarație nu a fost ceea ce pretinde a fi—o declarație a \emcap{principiilor fundamentale}, crezute și practicate de adventiștii de ziua a șaptea. Această declarație a fost un rezumat al caracteristicilor principale ale credinței adventiștilor, și niciun punct nu este cu adevărat la fel de problematic sau de criticabil ca primul punct, care tratează \emcap{personalitatea lui Dumnezeu} și unde este prezența Sa. Dar dovezile în favoarea \emcap{Principiilor Fundamentale}, în special pentru primul punct, sunt copleșitoare.


All of these claims are easily refuted by the fact that the \emcap{Fundamental Principles} have been regularly issued and reprinted over the course of the entire life of Sister White, until 1914. If they were mere private opinions of a few individuals, as claimed by scholars\footnote{Ministry Magazine “Our Declaration of Fundamental Beliefs”: January 1958, Roy Anderson, J. Arthur Buckwalter, Louise Kleuser, Earl Cleveland and Walter Schubert}, would they have been consistently reprinted over the course of 42 years\footnote{For a detailed list of publications throughout these years, see the Appendix.}, publicly claiming to represent the synopsis of Seventh-day Adventist faith? If they had been issued only once, we could deem it a conspiracy by some individuals to purposely misrepresent Seventh-day Adventist faith. On the contrary, the \emcap{Fundamental Principles} were regularly reprinted, and they truly represented the official Seventh-day Adventist faith and practice.


Toate aceste afirmații sunt ușor de respins prin faptul că \emcap{Principiile Fundamentale} au fost emise și retipărite în mod regulat pe parcursul întregii vieți a sorei White, până în 1914. Dacă ar fi fost simple opinii private ale câtorva indivizi, așa cum susțin învățații\footnote{Ministry Magazine “Our Declaration of Fundamental Beliefs”: ianuarie 1958, Roy Anderson, J. Arthur Buckwalter, Louise Kleuser, Earl Cleveland și Walter Schubert}, ar fi fost ele retipărite în mod constant pe parcursul a 42 de ani\footnote{Pentru o listă detaliată a publicațiilor de-a lungul acestor ani, vezi Anexa.}, pretinzând public că reprezintă rezumatul credinței adventiste de ziua a șaptea? Dacă ar fi fost emise doar o singură dată, am putea considera că este o conspirație a unor indivizi pentru a prezenta în mod intenționat greșit credința adventistă de ziua a șaptea. Dimpotrivă, \emcap{Principiile Fundamentale} au fost retipărite în mod regulat și au reprezentat cu adevărat credința și practica oficială adventistă de ziua a șaptea.


Another argument is that Sister White approved the \emcap{Fundamental Principles} in her writings by explicitly referring to them, and also by teaching the same truths taught in the \emcap{Fundamental Principles}. The works of our pioneers are also in harmony with the statements in this Declaration of the \emcap{Fundamental Principles}. Considering all of these facts, it is inevitable that this declaration was truthful in its claims. This document was indeed a declaration of the \emcap{fundamental principles}, taught and practiced by the Seventh-day Adventist Church, representing a public \others{synopsis of our faith}, \others{a brief statement of what is, and has been, with great unanimity, held by} Seventh-day Adventists.\footnote{The preface of the Fundamental Principles in 1872.} As such, it accurately represents the Seventh-day Adventist belief and practice, and represents the foundation of Seventh-day Adventist faith in the time of Ellen White.


Un alt argument este că sora White a aprobat \emcap{Principiile Fundamentale} în scrierile ei, referindu-se explicit la ele și, de asemenea, învățând aceleași adevăruri învățate în \emcap{Principiile Fundamentale}. Lucrările pionierilor noștri sunt, de asemenea, în armonie cu afirmațiile din această Declarație a \emcap{Principiilor Fundamentale}. Luând în considerare toate aceste fapte, este inevitabil că această declarație a fost adevărată în afirmațiile sale. Acest document a fost într-adevăr o declarație a \emcap{principiilor fundamentale}, crezute și practicate de Biserica Adventistă de Ziua a Șaptea, reprezentând un \others{rezumat public al credinței noastre}, \others{o scurtă expunere a ceea ce este și a fost, cu mare unanimitate, susținut de} adventiștii de ziua a șaptea.\footnote{Prefața Principiilor Fundamentale din 1872.} Ca atare, reprezintă cu acuratețe credința și practica adventistă de ziua a șaptea și reprezintă fundamentul credinței adventiste de ziua a șaptea în timpul lui Ellen White.


Today, in defense of the Trinity doctrine, Adventist historians boldly claim that when our pioneers were studying Adventist truths such as the sanctuary, investigative judgment, the Sabbath and other doctrines, they \others{did not study the subject of the doctrine of God}. These Adventist historians falsely claim that the doctrine of God \others{was not the question that they dealt at that time}[Denis Kaiser. “From Antitrinitarianism to Trinitarianism: The Adventist story” and Panelist. The God We Worship: A Godhead Symposium. Central California Conference, Dinuba, CA. March 23-24, 2018.]. Following this false claim, they present historical data on how Adventist doctrine gradually moved toward Trinitarian understanding. The truth is, there are some instances early on\footnote{The earliest mention of the Trinity doctrine, in a positive sense, was when M.C. Wilcox reprinted a non-Adventist article by Samuel Spear in Signs of the Times, December 7th, 1891 and December 14th, 1891} when the Trinity doctrine is mentioned in a positive light in our literature. But when you consider the fact that the Adventist church did have a positive position on the subject of the doctrine of God, as it was expressed in the \emcap{Fundamental Principles}, these instances cannot be interpreted as progressiveness in understanding, but rather an intrusion of the Trinity doctrine into the Seventh-day Adventist Church.


Astăzi, în apărarea doctrinei Trinității, istoricii adventiști susțin cu îndrăzneală că atunci când pionierii noștri studiau adevărurile adventiste precum sanctuarul, judecata de cercetare, Sabatul și alte doctrine, ei \others{nu au studiat subiectul doctrinei despre Dumnezeu}. Acești istorici adventiști susțin în mod fals că doctrina despre Dumnezeu \others{nu a fost problema cu care s-au ocupat în acel moment}[Denis Kaiser. “From Antitrinitarianism to Trinitarianism: The Adventist story” și Panelist. The God We Worship: A Godhead Symposium. Central California Conference, Dinuba, CA. 23-24 martie 2018.]. În urma acestei afirmații false, ei prezintă date istorice despre cum doctrina adventistă s-a îndreptat treptat spre înțelegerea trinitariană. Adevărul este că există câteva cazuri timpurii\footnote{Cea mai timpurie mențiune a doctrinei Trinității, într-un sens pozitiv, a fost când M.C. Wilcox a retipărit un articol non-adventist de Samuel Spear în Signs of the Times, 7 decembrie 1891 și 14 decembrie 1891} când doctrina Trinității este menționată într-o lumină pozitivă în literatura noastră. Dar când luați în considerare faptul că biserica adventistă a avut o poziție pozitivă cu privire la subiectul doctrinei despre Dumnezeu, așa cum a fost exprimată în \emcap{Principiile Fundamentale}, aceste cazuri nu pot fi interpretate ca progresivitate în înțelegere, ci mai degrabă ca o intruziune a doctrinei Trinității în Biserica Adventistă de Ziua a Șaptea.


It is easy to refute the claim that Adventist pioneers did not understand the doctrine of God. If they did not understand it, they would have failed to proclaim the first angel’s message. We discussed this point in detail in the chapter “\hyperref[chap:remembering-the-beginning]{Remembering the beginning}”. The Seventh-day Adventist movement was not a failure, but a God-led, prophetic movement.


Este ușor să respingem afirmația că pionierii adventiști nu au înțeles doctrina despre Dumnezeu. Dacă nu ar fi înțeles-o, ar fi eșuat să proclame solia primului înger. Am discutat acest punct în detaliu în capitolul „\hyperref[chap:remembering-the-beginning]{Amintindu-ne de început}”. Mișcarea adventistă de ziua a șaptea nu a fost un eșec, ci o mișcare profetică condusă de Dumnezeu.


\subsection*{Step 2: Ignore the warnings of building a new foundation}


\subsection*{Pasul 2: Ignorarea avertizărilor despre construirea unei noi fundații}


When the \emcap{Fundamental Principles} are removed from the equation, many of Ellen White’s warnings fail to shine in their true light and their true meaning does not resonate with the reader.


Când \emcap{Principiile Fundamentale} sunt eliminate din ecuație, multe dintre avertizările lui Ellen White nu reușesc să strălucească în adevărata lor lumină și adevăratul lor sens nu rezonează cu cititorul.


We have cited many quotations where Sister White warned the church not to step off the \emcap{Fundamental Principles}. We dealt with them in the chapter “\hyperref[chap:apostasy]{The great apostasy is soon to be realized}”, but we will mention one of the most prominent quotations again.


Am citat multe citate în care sora White a avertizat biserica să nu se îndepărteze de \emcap{Principiile Fundamentale}. Ne-am ocupat de ele în capitolul „\hyperref[chap:apostasy]{Marea apostazie urmează să fie realizată curând}”, dar vom menționa din nou unul dintre cele mai proeminente citate.


\egw{\textbf{The enemy of souls has sought to bring in the supposition that a great reformation was to take place among Seventh-day Adventists, and that this reformation would \underline{consist in giving up the doctrines which stand as the pillars of our faith} and engaging in a process of reorganization}. Were this reformation to take place, what would result? \textbf{The principles of truth that God in His wisdom has given to the remnant church would be discarded. Our religion would be changed. \underline{The fundamental principles that have sustained the work for the last fifty years would be accounted as error}}. \textbf{A new organization would be established. Books of a new order would be written. A system of intellectual philosophy would be introduced}...}[Lt242-1903.13; 1903][https://egwwritings.org/read?panels=p7767.20]


\egw{\textbf{Vrăjmașul sufletelor a căutat să aducă presupunerea că o mare reformă urma să aibă loc printre adventiștii de ziua a șaptea și că această reformă ar \underline{consta în renunțarea la doctrinele care stau ca stâlpii credinței noastre} și angajarea într-un proces de reorganizare}. Dacă această reformă ar avea loc, care ar fi rezultatul? \textbf{Principiile adevărului pe care Dumnezeu în înțelepciunea Sa le-a dat bisericii rămășiței ar fi înlăturate. Religia noastră ar fi schimbată. \underline{Principiile fundamentale care au susținut lucrarea în ultimii cincizeci de ani ar fi considerate ca eroare}}. \textbf{O nouă organizație ar fi înființată. Cărți de un ordin nou ar fi scrise. Un sistem de filozofie intelectuală ar fi introdus}...}[Lt242-1903.13; 1903][https://egwwritings.org/read?panels=p7767.20]


\egwnogap{Who has authority to begin such a movement? \textbf{We have our Bibles. We have our experience, attested to by the miraculous working of the Holy Spirit}. \textbf{We have a truth that admits of no compromise.} \textbf{\underline{Shall we not repudiate everything that is not in harmony with this truth}?}}[Lt242-1903.14; 1903][https://egwwritings.org/read?panels=p7767.21]


\egwnogap{Cine are autoritatea să înceapă o astfel de mișcare? \textbf{Avem Bibliile noastre. Avem experiența noastră, atestată de lucrarea miraculoasă a Duhului Sfânt}. \textbf{Avem un adevăr care nu admite niciun compromis.} \textbf{\underline{Nu vom repudia oare tot ceea ce nu este în armonie cu acest adevăr}?}}[Lt242-1903.14; 1903][https://egwwritings.org/read?panels=p7767.21]


\subsection*{Step 3: Deny that the personality of God was the pillar of our faith and a part of the foundation of our faith}


\subsection*{Pasul 3: Negarea faptului că personalitatea lui Dumnezeu era stâlpul credinței noastre și o parte din temelia credinței noastre}


There is one Ellen White statement that apparently supports the claim that the \emcap{personality of God} was not a pillar of our faith. Another expression for “\textit{pillars of our faith}” is “\textit{landmarks}”. In the following quotations, Sister White lists several landmarks: the cleansing of the sanctuary, the three angels’ messages, the temple of God, the Sabbath and the non-immortality of the wicked.


Există o declarație a lui Ellen White care aparent susține afirmația că \emcap{personalitatea lui Dumnezeu} nu era un stâlp al credinței noastre. O altă expresie pentru „\textit{stâlpii credinței noastre}” este „\textit{pietre de hotar}”. În următoarele citate, sora White enumeră mai multe pietre de hotar: curățirea sanctuarului, soliile celor trei îngeri, templul lui Dumnezeu, Sabatul și nemuritoarea celor răi.


\egw{The passing of the time in 1844 was a period of great events, opening to our astonished eyes \textbf{the cleansing of the sanctuary transpiring in heaven}, and having decided relation to God’s people upon the earth, [also] \textbf{the first and second angels’ messages and the third}, unfurling the banner on which was inscribed, ‘The commandments of God and the faith of Jesus.’ [Revelation 14:12.] One of the landmarks under this message was \textbf{the temple of God}, seen by His truth-loving people in heaven, and the ark containing the law of God. The light of \textbf{the Sabbath} of the fourth commandment flashed its strong rays in the pathway of the transgressors of God’s law. The \textbf{non-immortality of the wicked} is an old landmark. \textbf{I can call to mind nothing more that can come under the head of the old landmarks}. All this cry about changing the old landmarks is all imaginary.}[Ms13-1889.9; 1889][https://egwwritings.org/read?panels=p4179.14]


\egw{Trecerea timpului în 1844 a fost o perioadă de mari evenimente, deschizând ochilor noștri uimiți \textbf{curățirea sanctuarului care se desfășoară în cer}, și având o relație hotărâtă cu poporul lui Dumnezeu de pe pământ, [de asemenea] \textbf{soliile primului și celui de-al doilea înger și a treia}, desfășurând steagul pe care era înscris: „Poruncile lui Dumnezeu și credința lui Isus”. [Apocalipsa 14:12.] Una dintre pietrele de hotar sub această solie era \textbf{templul lui Dumnezeu}, văzut de poporul Său iubitor de adevăr în cer, și chivotul conținând legea lui Dumnezeu. Lumina \textbf{Sabatului} poruncii a patra și-a aruncat razele puternice pe calea călcătorilor legii lui Dumnezeu. \textbf{Nemuritoarea celor răi} este o piatră de hotar veche. \textbf{Nu-mi pot aminti nimic altceva care să poată intra sub categoria vechilor pietre de hotar}. Toată această strigare despre schimbarea vechilor pietre de hotar este doar imaginară.}[Ms13-1889.9; 1889][https://egwwritings.org/read?panels=p4179.14]


At the end of this list of landmarks, or pillars of our faith, she states that she can recall nothing else that would fall under the category of the old landmarks. For many, this quotation serves as proof that the \emcap{personality of God} was neither an old landmark nor a pillar. It is true that in this quotation, Sister White did not explicitly mention the \emcap{personality of God}, but it would be implicitly included under the first angel’s message, as well as being an underlying doctrine of the Sanctuary message. Furthermore, there are other quotations from Sister White that explicitly include the \emcap{personality of God} as an old landmark or pillar of our faith.


La sfârșitul acestei liste de pietre de hotar, sau stâlpi ai credinței noastre, ea afirmă că nu-și poate aminti nimic altceva care ar intra în categoria vechilor pietre de hotar. Pentru mulți, acest citat servește ca dovadă că \emcap{personalitatea lui Dumnezeu} nu era nici o piatră de hotar veche, nici un stâlp. Este adevărat că în acest citat, sora White nu a menționat explicit \emcap{personalitatea lui Dumnezeu}, dar aceasta ar fi inclusă implicit sub solia primului înger, precum și fiind o doctrină fundamentală a soliei Sanctuarului. Mai mult, există alte citate de la sora White care includ explicit \emcap{personalitatea lui Dumnezeu} ca o piatră de hotar veche sau stâlp al credinței noastre.


\egw{Those who seek to remove the \textbf{old landmarks} are not holding fast; they \textbf{are not remembering how they have received and heard}. Those who try to \textbf{\underline{bring in} theories that would remove \underline{the pillars of our faith}} \textbf{concerning the sanctuary}, \textbf{\underline{or concerning the personality of God or of Christ}, are working as blind men}. They are seeking to bring in uncertainties and to set the people of God \textbf{adrift}, without an anchor.}[Ms62-1905.14; 1905][https://egwwritings.org/read?panels=p10026.20]


\egw{Cei care caută să îndepărteze \textbf{vechile pietre de hotar} nu țin tare; ei \textbf{nu-și amintesc cum au primit și au auzit}. Cei care încearcă să \textbf{\underline{aducă} teorii care ar îndepărta \underline{stâlpii credinței noastre}} \textbf{cu privire la sanctuar}, \textbf{\underline{sau cu privire la personalitatea lui Dumnezeu sau a lui Hristos}, lucrează ca oameni orbi}. Ei caută să aducă incertitudini și să pună poporul lui Dumnezeu \textbf{în derivă}, fără ancoră.}[Ms62-1905.14; 1905][https://egwwritings.org/read?panels=p10026.20]


Sister White also teaches us that the pillars of our faith constitute the foundation of our faith.


Sora White ne învață de asemenea că stâlpii credinței noastre constituie temelia credinței noastre.


\egw{\textbf{What influence is it that would lead men at this stage of our history to work in an underhanded, powerful way \underline{to tear down the foundation of our faith},—the foundation that was laid at the beginning of our work by prayerful study of the word and by revelation? Upon \underline{this foundation} we have been building for \underline{the past fifty years}. Do you wonder that when I see the beginning of a work that would \underline{remove some of the pillars of our faith}, I have something to say? I must obey the command, ‘Meet it!’}}[SpTB02 58.1; 1904][https://egwwritings.org/read?panels=p417.295]


\egw{\textbf{Ce influență este aceea care i-ar conduce pe oameni în această etapă a istoriei noastre să lucreze într-un mod ascuns, puternic \underline{pentru a dărâma temelia credinței noastre},—temelia care a fost pusă la începutul lucrării noastre prin studiul cu rugăciune al cuvântului și prin revelație? Pe \underline{această temelie} am construit în \underline{ultimii cincizeci de ani}. Vă mirați că atunci când văd începutul unei lucrări care ar \underline{îndepărta unii dintre stâlpii credinței noastre}, am ceva de spus? Trebuie să ascult porunca: „Înfruntă-o!”}}[SpTB02 58.1; 1904][https://egwwritings.org/read?panels=p417.295]


Removing some of the pillars of our faith means tearing down the foundation of our faith. Elsewhere, Sister White said that tearing down or undermining the foundation of our faith is done by indoctrination of the sentiments regarding the \emcap{personality of God}.


Îndepărtarea unora dintre stâlpii credinței noastre înseamnă dărâmarea temeliei credinței noastre. În altă parte, sora White a spus că dărâmarea sau subminarea temeliei credinței noastre se face prin îndoctrinarea opiniilor cu privire la \emcap{personalitatea lui Dumnezeu}.


\egw{The college was taken out of Battle Creek; yet students are still called there, and there they \textbf{become indoctrinated with the very sentiments regarding the personality of God and Christ that would undermine the foundation of our faith}.}[Lt72-1906.5; 1906][https://egwwritings.org/read?panels=p10013.11]


\egw{Colegiul a fost mutat din Battle Creek; totuși studenții sunt încă chemați acolo, și acolo ei \textbf{devin îndoctrinați cu chiar acele opinii privind personalitatea lui Dumnezeu și a lui Hristos care ar submina temelia credinței noastre}.}[Lt72-1906.5; 1906][https://egwwritings.org/read?panels=p10013.11]


In light of these quotations we see positive testimony that the \emcap{personality of God} was part of the foundation of our faith. Furthermore, in chapter 10 of the special testimonies, entitled “\textit{The foundation of our faith}”, Sister White mentioned “\textit{Fundamental Principles}” using the synonyms “\textit{pillars of our faith}”, “\textit{waymarks}”, and “\textit{landmarks}”, when addressing the foundation of our faith.


În lumina acestor citate vedem o mărturie pozitivă că \emcap{personalitatea lui Dumnezeu} era parte din temelia credinței noastre. Mai mult, în capitolul 10 al mărturiilor speciale, intitulat “\textit{Temelia credinței noastre}”, sora White a menționat “\textit{Principii fundamentale}” folosind sinonimele “\textit{stâlpii credinței noastre}”, “\textit{pietre de hotar}” și “\textit{puncte de reper}”, când s-a adresat temeliei credinței noastre.


\subsection*{Step 4: Alter the meaning of the term “the personality of God”}


\subsection*{Pasul 4: Alterarea sensului termenului “personalitatea lui Dumnezeu”}


The term ‘\textit{personality}’ has two different applications and the most common definition in everyday use is in the area of psychology. ‘\textit{Personality}’ is defined as “\textit{the characteristic sets of behaviors, cognitions, and emotional patterns that evolve from biological and environmental factors}”\footnote{Wikipedia Contributors. “Personality.” Wikipedia, Wikimedia Foundation, 19 Apr. 2019, \href{https://en.wikipedia.org/wiki/Personality}{en.wikipedia.org/wiki/Personality}.}. It is of utmost importance to recognize that when we are dealing with the pillar of our faith—“\textit{the personality of God}”—we are not in the realms of psychology. The accurate application of the word ‘\textit{personality}’ within the doctrine on the \emcap{personality of God} is found in the Merriam-Webster Dictionary: “\textit{the quality or state of being a person}”\footnote{\href{https://www.merriam-webster.com/dictionary/personality}{Merriam-Webster Dictionary} - ‘\textit{personality}’}. According to the Merriam-Webster Dictionary, this definition has been in use since the 15th century\footnote{See “\href{https://www.merriam-webster.com/dictionary/personality\#word-history}{First known use}” of the word ‘personality’ in Merriam Webster Dictionary}. In the 1828 edition of the Merriam Webster Dictionary we read definition of the word ‘\textit{personality}’ as: “\textit{that which constitutes an individual a distinct person}”\footnote{\href{https://archive.org/details/americandictiona02websrich/page/272/mode/2up}{Merriam-Webster Dictionary, 1828 edition} - ‘\textit{personality}’} \footnote{\href{https://archive.org/details/websterscomplete00webs/page/974/mode/2up}{The 1886 edition of Merriam-Webster Dictionary} defines the word ‘\textit{personality}’ as: “\textit{that which constitutes, or pertains to, a person}”}. Both of the definitions are found in The Encyclopaedic Dictionary, by Hunter Robert\footnote{\href{https://babel.hathitrust.org/cgi/pt?id=mdp.39015050663213&view=1up&seq=780}{Hunter Robert, The Encyclopaedic Dictionary} - ‘\textit{personality}’}—dictionary owned by Ellen White. The use of these definitions can be seen from the articles written on the \emcap{personality of God}.


Termenul ‘\textit{personalitate}’ are două aplicații diferite și definiția cea mai comună în uzul cotidian este în domeniul psihologiei. ‘\textit{Personalitatea}’ este definită ca “\textit{seturile caracteristice de comportamente, cogniții și tipare emoționale care evoluează din factori biologici și de mediu}”\footnote{Wikipedia Contributors. “Personality.” Wikipedia, Wikimedia Foundation, 19 Apr. 2019, \href{https://en.wikipedia.org/wiki/Personality}{en.wikipedia.org/wiki/Personality}.}. Este de cea mai mare importanță să recunoaștem că atunci când avem de-a face cu stâlpul credinței noastre—“\textit{personalitatea lui Dumnezeu}”—nu suntem în domeniul psihologiei. Aplicarea corectă a cuvântului ‘\textit{personalitate}’ în cadrul doctrinei despre \emcap{personalitatea lui Dumnezeu} se găsește în Dicționarul Merriam-Webster: “\textit{caracteristica sau starea prin care cineva este definit ca persoană}”\footnote{\href{https://www.merriam-webster.com/dictionary/personality}{Dicționarul Merriam-Webster} - ‘\textit{personalitate}’}. Conform Dicționarului Merriam-Webster, această definiție a fost în uz din secolul al 15-lea\footnote{Vezi “\href{https://www.merriam-webster.com/dictionary/personality\#word-history}{Prima utilizare cunoscută}” a cuvântului ‘personalitate’ în Dicționarul Merriam Webster}. În ediția din 1828 a Dicționarului Merriam Webster citim definiția cuvântului ‘\textit{personalitate}’ ca: “\textit{ceea ce constituie un individ ca persoană distinctă}”\footnote{\href{https://archive.org/details/americandictiona02websrich/page/272/mode/2up}{Dicționarul Merriam-Webster, ediția 1828} - ‘\textit{personalitate}’} \footnote{\href{https://archive.org/details/websterscomplete00webs/page/974/mode/2up}{Ediția din 1886 a Dicționarului Merriam-Webster} definește cuvântul ‘\textit{personalitate}’ ca: “\textit{ceea ce constituie sau aparține unei persoane}”}. Ambele definiții se găsesc în Dicționarul Enciclopedic, de Hunter Robert\footnote{\href{https://babel.hathitrust.org/cgi/pt?id=mdp.39015050663213&view=1up&seq=780}{Hunter Robert, Dicționarul Enciclopedic} - ‘\textit{personalitate}’}—dicționar deținut de Ellen White. Utilizarea acestor definiții poate fi văzută din articolele scrise despre \emcap{personalitatea lui Dumnezeu}.


In 1903, when Sister White wrote to Dr. Kellogg, \egwinline{I have \textbf{ever }had the same testimony to bear which I now bear \textbf{regarding the personality of God}}[Lt253-1903.9; 1903][https://egwwritings.org/read?panels=p9980.15], she recalled her vision when she saw the Father and the Son.


În 1903, când sora White i-a scris Dr. Kellogg, \egwinline{am avut \textbf{întotdeauna }aceeași mărturie de dat pe care o dau acum \textbf{privind personalitatea lui Dumnezeu}}[Lt253-1903.9; 1903][https://egwwritings.org/read?panels=p9980.15], ea și-a amintit viziunea când L-a văzut pe Tatăl și pe Fiul.


\egw{‘I have often seen the lovely Jesus, that\textbf{ He is a person}.\textbf{ I asked Him if His Father was a person, }and \textbf{had \underline{a form} like Himself}. Said Jesus, ‘\textbf{I am the express image of My Father’s person!}’ [Hebrews 1:3.]}[Lt253-1903.12; 1903][https://egwwritings.org/read?panels=p9980.18]


\egw{‘L-am văzut adesea pe iubitul Isus, că\textbf{ El este o persoană}.\textbf{ L-am întrebat dacă Tatăl Său era o persoană, }și \textbf{avea \underline{o formă} ca a Sa}. Isus a spus: ‘\textbf{Eu sunt Întipărirea persoanei Tatălui Meu!}’ [Evrei 1:3.]}[Lt253-1903.12; 1903][https://egwwritings.org/read?panels=p9980.18]


The quality or state that Sister White defines God to be a person is to have \textit{a form}—\textit{a physical appearance}. Dr. Kellogg follows the same application of the word \textit{‘personality’}, although through speculation.


Caracteristica sau starea prin care sora White Îl definește pe Dumnezeu ca fiind o persoană este să aibă \textit{o formă}—\textit{o înfățișare fizică}. Dr. Kellogg urmează aceeași aplicare a cuvântului ‘\textit{personalitate}’, deși prin speculație.


\others{The fact that God is so great that we cannot form a clear mental picture of \textbf{his physical appearance} need not lessen in our minds the reality of \textbf{His personality}...}[John H. Kellogg, The Living Temple, p. 31][https://archive.org/details/J.H.Kellogg.TheLivingTemple1903/page/n31/mode/2up]


\others{Faptul că Dumnezeu este atât de mare încât nu putem forma o imagine mentală clară a \textbf{înfățișării Sale fizice} nu trebuie să diminueze în mintea noastră realitatea \textbf{personalității Sale}...}[John H. Kellogg, Templul viu, p. 31][https://archive.org/details/J.H.Kellogg.TheLivingTemple1903/page/n31/mode/2up]


As we have previously seen, our Adventist pioneers also pinpointed the physical appearance as a quality that makes God a person. James White wrote, \others{Those who deny \textbf{the personality of God}, say that ‘image’ here does not mean \textbf{physical form}, but moral image...}[James S. White, PERGO 1.1; 1861][https://egwwritings.org/read?panels=p1471.3]. J. B. Frisbie wrote, \others{Some seem to suppose it argues against \textbf{the personality of God}, because he is a Spirit, and say that he is without \textbf{body, or parts}...}[\href{https://documents.adventistarchives.org/Periodicals/RH/RH18540307-V05-07.pdf}{Adventist Review and Sabbath Herald, March 7, 1854}, J. B. Frisbie, “The Seventh-Day Sabbath Not Abolished”, p. 50]


După cum am văzut anterior, pionierii noștri adventiști au indicat de asemenea înfățișarea fizică ca o calitate care Îl face pe Dumnezeu o persoană. James White a scris, \others{Cei care neagă \textbf{personalitatea lui Dumnezeu}, spun că ‘chipul’ aici nu înseamnă \textbf{formă fizică}, ci chip moral...}[James S. White, PERGO 1.1; 1861][https://egwwritings.org/read?panels=p1471.3]. J. B. Frisbie a scris, \others{Unii par să presupună că argumentează împotriva \textbf{personalității lui Dumnezeu}, pentru că El este un Duh, și spun că El este fără \textbf{trup sau părți}...}[\href{https://documents.adventistarchives.org/Periodicals/RH/RH18540307-V05-07.pdf}{Adventist Review and Sabbath Herald, 7 martie 1854}, J. B. Frisbie, “Sabatul de ziua a șaptea nu a fost desființat”, p. 50]


In light of the facts, we recognize the application of the word ‘\textit{personality}’. When the subject on the \emcap{personality of God} is presented in its connection to the Trinity doctrine, there is often a tendency to alter the meaning of the word ‘\textit{personality}’. It is also important to mention that the subject on the \emcap{personality of God} deals with the personality of the Father. This is clearly seen from the presented data.


În lumina faptelor, recunoaștem aplicarea cuvântului ‘\textit{personalitate}’. Când subiectul despre \emcap{personalitatea lui Dumnezeu} este prezentat în legătura sa cu doctrina Trinității, există adesea o tendință de a altera sensul cuvântului ‘\textit{personalitate}’. Este de asemenea important să menționăm că subiectul despre \emcap{personalitatea lui Dumnezeu} se ocupă de personalitatea Tatălui. Acest lucru se vede clar din datele prezentate.


\subsection*{Step 5: In examining the Kellogg crisis, shifting the main focus from the personality of God to pantheism}


\subsection*{Pasul 5: În examinarea crizei Kellogg, mutarea accentului principal de la personalitatea lui Dumnezeu la panteism}


The data on the Kellogg crisis, in connection with the Trinity doctrine, is overwhelming if the \emcap{personality of God} is accounted for in the equation. The only way to not connect the dots is to ignore the \emcap{personality of God} and shift focus to pantheism exclusively. We do not deny the pantheistic nature of Kellogg's controversy. We believe that the pantheistic nature of Kellogg's controversy cannot be rightly understood if it is not examined in the true light of the \emcap{personality of God}. But, unfortunately, in examination of the Kellogg crisis, the attention that pantheism receives supersedes the examination of the truth on the \emcap{personality of God}.


Datele despre criza Kellogg, în legătură cu doctrina Trinității, sunt copleșitoare dacă \emcap{personalitatea lui Dumnezeu} este luată în considerare în ecuație. Singura modalitate de a nu face legătura este să ignorăm \emcap{personalitatea lui Dumnezeu} și să mutăm atenția exclusiv asupra panteismului. Nu negăm natura panteistă a controversei lui Kellogg. Credem că natura panteistă a controversei lui Kellogg nu poate fi înțeleasă corect dacă nu este examinată în lumina adevărată a \emcap{personalității lui Dumnezeu}. Dar, din păcate, în examinarea crizei Kellogg, atenția pe care o primește panteismul depășește examinarea adevărului despre \emcap{personalitatea lui Dumnezeu}.


You can do a search of Ellen White’s compilations to see just how much more attention pantheism received than the \emcap{personality of God}. If you were to search her writings for ‘pantheism’ or ‘pantheistic’, excluding the compilations after her death, you would find 36 occurrences. Among them are several repetitive quotations that Sister White copied from one letter to another, or to the special testimonies for the church. If you were to count the distinct occurrences you would only find 12 distinct quotations containing words like ‘\textit{pantheism}’ or ‘\textit{pantheistic}’\footnote{On the \href{https://egwwritings.org/}{https://egwwritings.org/} search bar, input the word “\textit{pantheis*} ”; this will include all words beginning with the ‘\textit{pantheis...}’, (including ‘\textit{pantheism}’ and ‘\textit{pantheistic}’). The results can be compared in subsetting the corpus of Ellen White writings by including or excluding compilations after her death. This option is available in the dropdown menu under the search bar.}. If you conducted the same search, but only in the compilations issued after her death, you would find 140 occurrences! All of these fall into one of the twelve distinct instances Sister White wrote on the subject of pantheism.


Puteți face o căutare în compilațiile lui Ellen White pentru a vedea cât de multă atenție a primit panteismul în comparație cu \emcap{personalitatea lui Dumnezeu}. Dacă ați căuta în scrierile ei cuvintele ‘panteism’ sau ‘panteist’, excluzând compilațiile de după moartea ei, ați găsi 36 de apariții. Printre acestea sunt mai multe citate repetitive pe care sora White le-a copiat dintr-o scrisoare în alta, sau în mărturiile speciale pentru biserică. Dacă ați număra aparițiile distincte, ați găsi doar 12 citate distincte conținând cuvinte precum ‘\textit{panteism}’ sau ‘\textit{panteist}’\footnote{Pe bara de căutare \href{https://egwwritings.org/}{https://egwwritings.org/}, introduceți cuvântul “\textit{pantheis*}”; aceasta va include toate cuvintele care încep cu ‘\textit{pantheis...}’, (inclusiv ‘\textit{panteism}’ și ‘\textit{panteist}’). Rezultatele pot fi comparate prin selectarea corpusului de scrieri ale lui Ellen White incluzând sau excluzând compilațiile de după moartea ei. Această opțiune este disponibilă în meniul derulant de sub bara de căutare.}. Dacă ați efectua aceeași căutare, dar doar în compilațiile emise după moartea ei, ați găsi 140 de apariții! Toate acestea se încadrează în una dintre cele douăsprezece cazuri distincte în care sora White a scris pe tema panteismului.


In a search of Ellen White writings on the phrase “\textit{personality of God}”, excluding the compilations after her death, you would find 58 occurrences. Among them are also several repetitive quotations that Sister White copied to several different letters and to the testimonies for the church. Yet, if you were to search this phrase within the compilations that were issued after her death you would only find 52 occurrences.


Într-o căutare în scrierile lui Ellen White pentru expresia “\textit{personalitatea lui Dumnezeu}”, excluzând compilațiile de după moartea ei, ați găsi 58 de apariții. Printre acestea sunt, de asemenea, mai multe citate repetitive pe care sora White le-a copiat în mai multe scrisori diferite și în mărturiile pentru biserică. Totuși, dacă ați căuta această expresie în compilațiile care au fost emise după moartea ei, ați găsi doar 52 de apariții.


These simple statistics demonstrate the focus of the compilators after the death of Sister White. Such emphasis on pantheism changed our public opinion regarding Kellogg’s crisis. Forty-three, out of fifty-eight, quotations on the phrase “\textit{personality of God}” are found in letters and manuscripts, available to the public from 2015 onwards. This means that three quarters (\textit{74 percent}) of the quotation regarding the \emcap{personality of God}, prior to 2015, was not available to the public. Prior to 2015 we did not have much available data to study Kellogg's crisis in light of the \emcap{personality of God} and in its context.


Aceste statistici simple demonstrează accentul compilatorilor de după moartea sorei White. Un astfel de accent pe panteism a schimbat opinia noastră publică cu privire la criza lui Kellogg. Patruzeci și trei, din cincizeci și opt, de citate despre expresia “\textit{personalitatea lui Dumnezeu}” se găsesc în scrisori și manuscrise, disponibile publicului din 2015 încoace. Aceasta înseamnă că trei sferturi (\textit{74 la sută}) din citatele privind \emcap{personalitatea lui Dumnezeu}, înainte de 2015, nu erau disponibile publicului. Înainte de 2015 nu aveam multe date disponibile pentru a studia criza lui Kellogg în lumina \emcap{personalității lui Dumnezeu} și în contextul ei.


% Steps to Omega

\begin{titledpoem}
    
    \stanza{
        On pillars now, the shadows cast— \\
        A truth forsaken, from the past. \\
        In steps they chart the silent drift, \\
        Five marks of change, through sacred rift.
    }

    \stanza{
        Denial blooms when once truth stood, \\
        Foundations are not understood, \\
        The fundamentals, once held dear \\
        Obscured, as new creeds appear.
    }

    \stanza{
        Prophetic warnings have been dimmed, \\
        Pioneers are shunned, old hymns are trimmed. \\
        The testimonies once rang out \\
        But now they’re often tinged with doubt.
    }

    \stanza{
        “God is a person” cast aside, \\
        And now His essence they deride. \\
        Forgotten pillar once was strong \\
        Now a new pillar, which is wrong!
    }

    \stanza{
        Scholars now twist the sacred term, \\
        Words redefined, they now affirm. \\
        Gone is the quest to see God’s face, \\
        Dim the desire for His embrace.
    }

    \stanza{
        The Kellogg crisis point is missed, \\
        The alpha given untrue twist \\
        And thus, the lessons are not learned \\
        The church toward omega turned.
    }

    \stanza{
        Confusion reigns, we can’t perceive \\
        It is not clear what we believe \\
        Our history has been revised \\
        We wanted truth, but then they lied.
    }
    
\end{titledpoem}