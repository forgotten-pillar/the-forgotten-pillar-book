% \qrchapter{https://forgottenpillar.com/rsc/en-fp-chapter17}{Reply to Kellogg’s trinitarian sentiments}


\qrchapter{https://forgottenpillar.com/rsc/ro-fp-chapter17}{Răspuns la opiniile trinitariene ale lui Kellogg}


If we look at the Kellogg crisis through the perspective of the \emcap{personality of God} and the \emcap{Fundamental Principles}, Sister White’s quotations inevitably shine in a new light. In this light we see the conflict between the truth we have received in the beginning, on the \emcap{personality of God}, and the Trinity doctrine. In order to avoid discrepancy, in the interest of defending the Trinity doctrine, scholars always overemphasize the pantheistic side of the problem.


Dacă privim criza Kellogg prin perspectiva \emcap{personalității lui Dumnezeu} și a \emcap{Principiilor Fundamentale}, citatele surorii White strălucesc inevitabil într-o lumină nouă. În această lumină vedem conflictul dintre adevărul pe care l-am primit la început, despre \emcap{personalitatea lui Dumnezeu}, și doctrina Trinității. Pentru a evita discrepanța, în interesul apărării doctrinei Trinității, învățații supraaccentuează întotdeauna latura panteistă a problemei.


We would like to challenge this tendency to overemphasize the pantheistic side of Kellogg’s controversy. Sister White generally wrote proactive truth; she approached the error by uplifting the truth. This is why she wrote so much about the \emcap{personality of God}. In most of her quotations on this subject, we see her dispelling the Trinitarian error, rather than pantheistic error. We read one such example where she establishes the truth on the \emcap{personality of God} referencing the seventeenth chapter of John.


Am dori să contestăm această tendință de a supraaccentua latura panteistă a controversei lui Kellogg. Sora White a scris în general adevăr proactiv; ea a abordat eroarea înălțând adevărul. De aceea a scris atât de mult despre \emcap{personalitatea lui Dumnezeu}. În majoritatea citatelor ei pe acest subiect, o vedem risipind eroarea trinitariană, mai degrabă decât eroarea panteistă. Citim un astfel de exemplu unde ea stabilește adevărul despre \emcap{personalitatea lui Dumnezeu} făcând referire la capitolul al șaptesprezecelea din Ioan.


\egw{\textbf{The personality of the Father and the Son, also the unity that exists between Them, are presented in the seventeenth chapter of John}, in the prayer of Christ for His disciples:}[MH 421.7; 1905][https://egwwritings.org/read?panels=p135.2173]


\egw{\textbf{Personalitatea Tatălui și a Fiului, de asemenea unitatea care există între Ei, sunt prezentate în capitolul al șaptesprezecelea din Ioan}, în rugăciunea lui Hristos pentru ucenicii Săi:}[MH 421.7; 1905][https://egwwritings.org/read?panels=p135.2173]


There are many cases where Sister White quotes John 17 in regard to Kellogg’s crisis. Those who assert that Kellogg’s crisis was solely about pantheism should inquire how John 17 addresses God in nature. And it is not only John 17, but also chapters 13-16. In her letter to Kellogg, she wrote:


Există multe cazuri în care sora White citează Ioan 17 cu privire la criza lui Kellogg. Cei care afirmă că criza lui Kellogg a fost doar despre panteism ar trebui să întrebe cum abordează Ioan 17 pe Dumnezeu în natură. Și nu este doar Ioan 17, ci și capitolele 13-16. În scrisoarea ei către Kellogg, ea a scris:


\egw{\textbf{\underline{…study the thirteenth, fourteenth, fifteenth, sixteenth, and seventeenth chapters of John}. The words of these chapters explain themselves. ‘This is life eternal,’ Christ declared, ‘that they might know \underline{Thee the only true God}, and Jesus Christ, whom Thou hast sent.’ \underline{In these words the personality of God and of His Son is clearly spoken of.} \underline{The personality of the one does not do away with the necessity for the personality of the other}.}}[Lt232-1903.48, 1903][https://egwwritings.org/read?panels=p10197.57]


\egw{\textbf{\underline{...studiază capitolele al treisprezecelea, al paisprezecelea, al cincisprezecelea, al șaisprezecelea și al șaptesprezecelea din Ioan}. Cuvintele acestor capitole se explică singure. ‘Aceasta este viața veșnică,’ a declarat Hristos, ‘să \underline{Te cunoască pe Tine, singurul Dumnezeu adevărat}, și pe Isus Hristos, pe care L-ai trimis Tu.’ \underline{În aceste cuvinte personalitatea lui Dumnezeu și a Fiului Său este clar vorbită.} \underline{Personalitatea unuia nu elimină necesitatea personalității celuilalt}.}}[Lt232-1903.48, 1903][https://egwwritings.org/read?panels=p10197.57]


In the aforementioned chapters of John, John did not reference anything pertaining to God in nature. The content of those chapters covers who is the only true God, how the Father and the Son are one, their true relation, and how Jesus can be everywhere present yet will ascend to the Father.


În capitolele menționate mai sus din Ioan, Ioan nu a făcut referire la nimic referitor la Dumnezeu în natură. Conținutul acelor capitole acoperă cine este singurul Dumnezeu adevărat, cum Tatăl și Fiul sunt una, relația lor adevărată și cum Isus poate fi prezent pretutindeni și totuși Se va înălța la Tatăl.


\egw{Jesus said to the Jews: ‘My Father worketh hitherto, and I work.... The Son can do nothing of Himself, but what He seeth the Father do: for what things soever He doeth, these also doeth the Son likewise. For the Father loveth the Son, and showeth Him all things that Himself doeth.’ John 5:17-20.}[8T 268.4, 1904][https://egwwritings.org/read?panels=p112.1557]


\egw{Isus le-a spus iudeilor: ‘Tatăl Meu lucrează până acum; și Eu de asemenea lucrez.... Fiul nu poate face nimic de la Sine, ci numai ce vede pe Tatăl făcând; căci lucrurile pe care le face El, le face și Fiul întocmai. Căci Tatăl iubește pe Fiul și-I arată tot ce face El Însuși.’ Ioan 5:17-20.}[8T 268.4, 1904][https://egwwritings.org/read?panels=p112.1557]


\egwnogap{\textbf{Here again is brought to view the \underline{personality of the Father and the Son}, showing the unity that exists between them}.}[8T 269.1; 1904][https://egwwritings.org/read?panels=p112.1560]


\egwnogap{\textbf{Aici din nou este adusă în vedere \underline{personalitatea Tatălui și a Fiului}, arătând unitatea care există între ei}.}[8T 269.1; 1904][https://egwwritings.org/read?panels=p112.1560]


\egwnogap{\textbf{This unity is expressed also in \underline{the seventeenth chapter of John}}, in the prayer of Christ for His disciples:}[8T 269.2; 1904][https://egwwritings.org/read?panels=p112.1561]


\egwnogap{\textbf{Această unitate este exprimată de asemenea în \underline{capitolul al șaptesprezecelea din Ioan}}, în rugăciunea lui Hristos pentru ucenicii Săi:}[8T 269.2; 1904][https://egwwritings.org/read?panels=p112.1561]


\egwnogap{‘Neither pray I for these alone, but for them also which shall believe on Me through their word; that they all may be one; \textbf{as Thou, Father, art in Me, and I in Thee, that they also may be one in Us}: that the world may believe that Thou hast sent Me. And \textbf{the glory which Thou gavest Me} I have given them; \textbf{that they may be one, even as We are one: I in them, and Thou in Me, that they may be made perfect in one}; and that the world may know that Thou hast sent Me, and hast loved them, as Thou hast loved Me.’ John 17:20-23.}[8T 269.3; 1904][https://egwwritings.org/read?panels=p112.1562]


\egwnogap{‘Și Mă rog nu numai pentru aceștia, ci și pentru cei ce vor crede în Mine prin cuvântul lor; ca toți să fie una; \textbf{precum Tu, Tată, ești în Mine, și Eu în Tine, ca și ei să fie una în Noi}: ca lumea să creadă că Tu M-ai trimis. Și \textbf{slava pe care Mi-ai dat-o Tu} le-am dat-o lor; \textbf{ca să fie una, după cum și Noi suntem una: Eu în ei, și Tu în Mine, ca să fie desăvârșiți în unire}; și ca lumea să cunoască că Tu M-ai trimis și că i-ai iubit, cum M-ai iubit pe Mine.’ Ioan 17:20-23.}[8T 269.3; 1904][https://egwwritings.org/read?panels=p112.1562]


\egwnogap{Wonderful statement! \textbf{The unity that exists between Christ and His disciples \underline{does not destroy the personality of either}. They are one in purpose, in mind, in character, but \underline{not in person}. It is thus that God and Christ are one}.}[8T 269.4; 1904][https://egwwritings.org/read?panels=p112.1563]


\egwnogap{Declarație minunată! \textbf{Unitatea care există între Hristos și ucenicii Săi \underline{nu distruge personalitatea niciunuia}. Ei sunt una în scop, în minte, în caracter, dar \underline{nu în persoană}. Astfel sunt Dumnezeu și Hristos una}.}[8T 269.4; 1904][https://egwwritings.org/read?panels=p112.1563]


\egwnogap{\textbf{The relation between the Father and the Son, and the personality of both, are made plain in this scripture also}:}[8T 269.5; 1904][https://egwwritings.org/read?panels=p112.1564]


\egwnogap{\textbf{Relația dintre Tatăl și Fiul, și personalitatea ambilor, sunt clarificate și în acest pasaj din Scriptură}:}[8T 269.5; 1904][https://egwwritings.org/read?panels=p112.1564]


\egwnogap{Thus speaketh \textbf{Jehovah of hosts}, saying,} \\
\egw{Behold, \textbf{the man} whose name is\textbf{ the Branch}:} \\
\egw{And He shall grow up out of His place;} \\
\egw{\textbf{And He shall build the temple of Jehovah;... }} \\
\egw{\textbf{And He shall bear the glory,}} \\
\egw{\textbf{And shall sit and rule upon His throne;}} \\
\egw{\textbf{And He shall be a priest upon His throne;}} \\
\egw{\textbf{And \underline{the counsel of peace shall be between Them both}}.’}[8T 269.6; 1904][https://egwwritings.org/read?panels=p112.1565]


\egwnogap{Așa vorbește \textbf{Domnul oștirilor}, zicând,} \\
\egw{Iată, \textbf{omul} al cărui nume este\textbf{ Odrasla}:} \\
\egw{Și El va crește din locul Său;} \\
\egw{\textbf{Și El va zidi templul Domnului;... }} \\
\egw{\textbf{Și El va purta slava,}} \\
\egw{\textbf{Și va ședea și va domni pe tronul Său;}} \\
\egw{\textbf{Și El va fi preot pe tronul Său;}} \\
\egw{\textbf{Și \underline{sfatul de pace va fi între Ei amândoi}}.’}[8T 269.6; 1904][https://egwwritings.org/read?panels=p112.1565]


The aforementioned chapters of the Gospel of John deal with the \emcap{personality of God}, which had been expressed in the first two points of the \emcap{Fundamental Principles}. What error did Sister White combat when she referenced verses on how the Father was the only true God, and how the Father and the Son are not one in person? Pantheism? Certainly not; but most probably the trinitarian sentiments, or belief in a one-in-three, or three-in-one God.


Capitolele menționate mai sus din Evanghelia după Ioan tratează \emcap{personalitatea lui Dumnezeu}, care fusese exprimată în primele două puncte ale \emcap{Principiilor Fundamentale}. Ce eroare a combătut sora White când a făcut referire la versetele despre cum Tatăl era singurul Dumnezeu adevărat și cum Tatăl și Fiul nu sunt una în persoană? Panteismul? Cu siguranță nu; ci cel mai probabil opiniile trinitariene, sau credința într-un Dumnezeu unu-în-trei, sau trei-în-unu.


Brother J. N. Loughborough, one of the first brethren who wrote on the \emcap{personality of God}, wrote the following comment on John chapter 17:


Fratele J. N. Loughborough, unul dintre primii frați care au scris despre \emcap{personalitatea lui Dumnezeu}, a scris următorul comentariu despre capitolul 17 din Ioan:


\others{\textbf{\underline{The seventeenth chapter of John is alone sufficient to refute the doctrine of the Trinity}}. \textbf{...\underline{Read the seventeenth chapter of John, and see if it does not completely upset the doctrine of the Trinity}}.}[John N. Loughborough, The Adventist Review, and Sabbath Herald, November 5, 1861, p. 184.10][https://egwwritings.org/read?panels=p1685.6615]


\others{\textbf{\underline{Capitolul al șaptesprezecelea din Ioan este singur suficient pentru a respinge doctrina Trinității}}. \textbf{...\underline{Citiți capitolul al șaptesprezecelea din Ioan și vedeți dacă nu răstoarnă complet doctrina Trinității}}.}[John N. Loughborough, The Adventist Review, and Sabbath Herald, November 5, 1861, p. 184.10][https://egwwritings.org/read?panels=p1685.6615]


Sister White’s proactive writing in support of the truth on the \emcap{personality of God} and His presence is the same as other Adventist pioneers. If Adventist pioneers were debunking the Trinity doctrine by exalting the truth on the \emcap{personality of God} and God’s presence, what makes us think Ellen White was not doing the same, when the theological side of the question of the Trinity was raised? By stating this, we do not deny the pantheistic side of Kellogg’s controversy, but by overemphasizing it, it falls short of accurately describing its real issue. The correct understanding of the Kellogg controversy can only be accomplished by focusing primarily on the truth Sister White uplifted, rather than focusing on error, whether pantheism or Trinity. This truth that Sister White uplifted was the truth on the \emcap{personality of God} and where His presence is. This is expressed in the first point of the \emcap{Fundamental Principles}, which were the official synopsis and representation of Seventh-day Adventist beliefs in the time of Ellen White; the truth which we, as a church, \egwinline{have received and heard and advocated}[Ms124-1905.12; 1905][https://egwwritings.org/read?panels=p9099.18] in the beginning.


Scrierile proactive ale sorei White în sprijinul adevărului despre \emcap{personalitatea lui Dumnezeu} și prezența Sa sunt aceleași cu cele ale altor pionieri adventiști. Dacă pionierii adventiști demontau doctrina Trinității prin înălțarea adevărului despre \emcap{personalitatea lui Dumnezeu} și prezența lui Dumnezeu, ce ne face să credem că Ellen White nu făcea același lucru, când a fost ridicată partea teologică a chestiunii Trinității? Afirmând aceasta, nu negăm partea panteistă a controversei lui Kellogg, dar prin supraaccentuarea ei, nu reușim să descriem cu acuratețe problema reală. Înțelegerea corectă a controversei Kellogg poate fi realizată doar prin concentrarea în primul rând pe adevărul pe care sora White l-a înălțat, mai degrabă decât prin concentrarea pe eroare, fie panteism sau Trinitate. Acest adevăr pe care sora White l-a înălțat era adevărul despre \emcap{personalitatea lui Dumnezeu} și unde este prezența Sa. Aceasta este exprimată în primul punct al \emcap{Principiilor Fundamentale}, care erau rezumatul oficial și reprezentarea credințelor Adventiștilor de Ziua a Șaptea în timpul lui Ellen White; adevărul pe care noi, ca biserică, \egwinline{l-am primit și auzit și susținut}[Ms124-1905.12; 1905][https://egwwritings.org/read?panels=p9099.18] la început.


\egw{\textbf{I entreat every one to be clear and firm regarding the certain truths that we have received and heard and advocated. The statements of God’s Word are plain. Plant your feet firmly on \underline{the platform of eternal truth}. \underline{Reject every phase of error}, even \underline{though it be covered with a semblance of reality, which denies the personality of God or of Christ}}.}[Ms124-1905.12; 1905][https://egwwritings.org/read?panels=p9099.18]


\egw{\textbf{Îi implor pe toți să fie clari și fermi cu privire la adevărurile sigure pe care le-am primit și auzit și susținut. Declarațiile Cuvântului lui Dumnezeu sunt clare. Plantați-vă picioarele ferm pe \underline{platforma adevărului veșnic}. \underline{Respingeți orice fază a erorii}, chiar \underline{dacă este acoperită cu o aparență de realitate, care neagă personalitatea lui Dumnezeu sau a lui Hristos}}.}[Ms124-1905.12; 1905][https://egwwritings.org/read?panels=p9099.18]


The warning from the previous quotations did not lessen in the course of time. Today it is even more relevant. We should \egwinline{reject every phase of error, even though it be covered with a semblance of reality, which denies the personality of God or of Christ}. In the following chapter we want to point out to the specific phase of error that is covered with a semblance of reality, which denies the personality of God and of Christ—three living persons of \textit{one} God, as opposed to \egwinline{three living persons of the heavenly trio.}[Ms21-1906.11; 1906][https://egwwritings.org/read?panels=p9754.18]


Avertizarea din citatele anterioare nu s-a diminuat în decursul timpului. Astăzi este chiar mai relevantă. Ar trebui să \egwinline{respingem orice fază a erorii, chiar dacă este acoperită cu o aparență de realitate, care neagă personalitatea lui Dumnezeu sau a lui Hristos}. În capitolul următor vrem să indicăm faza specifică a erorii care este acoperită cu o aparență de realitate, care neagă personalitatea lui Dumnezeu și a lui Hristos—trei persoane vii ale \textit{unui singur} Dumnezeu, în opoziție cu \egwinline{trei persoane vii ale trioului ceresc.}[Ms21-1906.11; 1906][https://egwwritings.org/read?panels=p9754.18]


% Reply to Kellogg’s trinitarian sentiments

\begin{titledpoem}
    
    \stanza{
        The light of truth, so clear and bold, \\
        A crisis came, a story told. \\
        Not pantheism, dim and wide, \\
        But God’s persona, we confide.
    }

    \stanza{
        But God, through Ellen, did uphold \\
        God’s personality was told. \\
        Against the Trinity, she leaned, \\
        A unity, by John unseen.
    }

    \stanza{
        "The Father and the Son," she wrote, \\
        Are one in purpose was her quote. \\
        John seventeen, her chosen guide, \\
        Where God’s true nature cannot hide.
    }

    \stanza{
        The pioneers, with her agreed, \\
        Of God’s true person, they did plead. \\
        Loughborough echoed, his words clear, \\
        The Trinity dismissed, no fear.
    }

    \stanza{
        The Fundamental Points, so dear, \\
        They make it plain, we must revere. \\
        Not in the trinity’s wrong creed, \\
        But in His presence, faith is freed.
    }

    \stanza{
        So let us stand on truth so bright, \\
        Rejecting wrong, with all our might. \\
        God’s person, where we find our plea, \\
        Truth’s platform for eternity.
    }
    
\end{titledpoem}