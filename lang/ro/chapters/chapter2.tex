% \qrchapter{https://forgottenpillar.com/rsc/en-fp-chapter2}{The Fundamental Principles}


\qrchapter{https://forgottenpillar.com/rsc/ro-fp-chapter2}{Principiile fundamentale}


The real issue according to chapter ten of the Special Testimonies is diverting from the foundation of our faith, which was established at the beginning of our work.


Adevărata problemă conform capitolului zece din Mărturii Speciale este îndepărtarea de la temelia credinței noastre, care a fost stabilită la începutul lucrării noastre.


\egw{\textbf{This foundation was built by the Masterworker}, and \underline{will} stand storm and tempest. Will they permit this man to \textbf{present doctrines that deny the past experience} of the people of God? The time has come to take decided action.}[SpTB02 54.2; 1904][https://egwwritings.org/read?panels=p417.276]


\egw{\textbf{Această temelie a fost zidită de Meșterul Lucrător}, și \underline{va} rezista furtunii și vijeliei. Vor permite ei acestui om să \textbf{prezinte doctrine care neagă experiența trecută} a poporului lui Dumnezeu? A venit timpul să se ia măsuri hotărâte.}[SpTB02 54.2; 1904][https://egwwritings.org/read?panels=p417.276]


Kellogg presented doctrines that deny the past experience. In another place, she wrote about Kellogg:


Kellogg a prezentat doctrine care neagă experiența trecută. În alt loc, ea a scris despre Kellogg:


\egw{I am much worried about Dr. Kellogg. In many respects, his course is not pleasing to the Lord. It seems to be \textbf{so easy for him to drift away from \underline{foundation principles}}. He is in great danger \textbf{of not holding the beginning of his confidence} steadfast unto the end.}[Lt138-1902.5; 1902][https://egwwritings.org/read?panels=p9219.11]


\egw{Sunt foarte îngrijorată pentru Dr. Kellogg. În multe privințe, calea lui nu este pe placul Domnului. Se pare că este \textbf{atât de ușor pentru el să se îndepărteze de \underline{principiile fundamentale}}. El este în mare pericol \textbf{de a nu păstra începutul încrederii sale} neclintit până la sfârșit.}[Lt138-1902.5; 1902][https://egwwritings.org/read?panels=p9219.11]


The problem was the departing from the foundation principles—but not all people recognized that. Especially the key and prominent people in the work; they forgot the way the Lord led them and His teaching in the past.


Problema era îndepărtarea de principiile fundamentale—dar nu toți oamenii au recunoscut aceasta. În special oamenii cheie și proeminenți din lucrare; ei au uitat calea pe care Domnul i-a condus și învățătura Sa din trecut.


\egw{I have been hoping that there would be a thorough reformation, and that \textbf{the principles} for which we fought \textbf{in the early days}, and which were brought out in the power of the Holy Spirit, \textbf{would be maintained}.}[SpTB02 56.3; 1904][https://egwwritings.org/read?panels=p417.287]


\egw{Am sperat că va fi o reformă temeinică și că \textbf{principiile} pentru care am luptat \textbf{în zilele de început}, și care au fost aduse la lumină în puterea Duhului Sfânt, \textbf{vor fi menținute}.}[SpTB02 56.3; 1904][https://egwwritings.org/read?panels=p417.287]


What were the principles that we fought for in the early days? What was this foundation of our faith?


Care erau principiile pentru care am luptat în zilele de început? Care era această temelie a credinței noastre?


\egw{As a people, we are to \textbf{stand firm on the platform of eternal truth} that has withstood test and trial. We are to \textbf{hold to the sure pillars of our faith}. \textbf{The \underline{principles of truth}} that God has revealed to us \textbf{are our only true foundation}. They have made us what we are...}[SpTB02 51.2; 1904][https://egwwritings.org/read?panels=p417.261]


\egw{Ca popor, trebuie să \textbf{stăm fermi pe platforma adevărului veșnic} care a rezistat încercării și probei. Trebuie să \textbf{ne ținem de stâlpii siguri ai credinței noastre}. \textbf{\underline{Principiile adevărului}} pe care Dumnezeu ni le-a descoperit \textbf{sunt singura noastră temelie adevărată}. Ele ne-au făcut ceea ce suntem...}[SpTB02 51.2; 1904][https://egwwritings.org/read?panels=p417.261]


The \egwinline{principles of truth} that God has revealed \egwinline{is our only true foundation}. She is calling these principles the platform of eternal truth. She refers to these principles as the \egwinline{sure pillars of our faith}[SpTB02 51.2; 1904][https://egwwritings.org/read?panels=p417.261].


\egwinline{Principiile adevărului} pe care Dumnezeu le-a descoperit \egwinline{sunt singura noastră temelie adevărată}. Ea numește aceste principii platforma adevărului veșnic. Ea se referă la aceste principii ca \egwinline{stâlpii siguri ai credinței noastre}[SpTB02 51.2; 1904][https://egwwritings.org/read?panels=p417.261].


She recalls the past experience of our pioneers, like James White, Joseph Bates, Elder Edson, father Pierce, how God worked on them until \egwinline{point by point}, \egwinline{all \textbf{the principal points of our faith} were made clear}. She recalled how \egwinline{this foundation was built by the Masterworker,} and assures that it \egwinline{will stand storm and tempest}. In conclusion, she strongly affirms the will of God for us regarding these principles. God \egwinline{calls upon us to \textbf{hold firmly}, with the grip of faith, \textbf{to the fundamental principles that are based upon unquestionable authority}}.


Ea își amintește experiența din trecut a pionierilor noștri, precum James White, Joseph Bates, Elder Edson, tatăl Pierce, cum Dumnezeu a lucrat asupra lor până când \egwinline{punct cu punct}, \egwinline{toate \textbf{punctele principale ale credinței noastre} au fost clarificate}. Ea și-a amintit cum \egwinline{această temelie a fost construită de Maestrul lucrător,} și ne asigură că \egwinline{va rezista furtunii și vijeliei}. În concluzie, ea afirmă cu tărie voia lui Dumnezeu pentru noi cu privire la aceste principii. Dumnezeu \egwinline{ne cheamă să \textbf{ținem cu tărie}, cu strânsoarea credinței, \textbf{de principiile fundamentale care sunt bazate pe autoritate de necontestat}}.


We see several different expressions that Sister White used for the foundation of our faith: “\textit{the platform of eternal truth},” “\textit{pillars of our faith},” “\textit{principles of truth},” “\textit{principal points},” “\textit{waymarks},” “\textit{foundation principles},” and “\textit{fundamental principles}”. These expressions denote the same thing—the foundation of our faith. When we hear these expressions today, somehow they don’t convey any concrete information. But for Seventh-day Adventists in her time, this was very clear and a definite point. All of these terms are referring to the public synopsis of Seventh-day Adventist’s faith called the \emcap{Fundamental Principles}, further explained below.


Vedem mai multe expresii diferite pe care sora White le-a folosit pentru temelia credinței noastre: “\textit{platforma adevărului veșnic},” “\textit{stâlpii credinței noastre},” “\textit{principii ale adevărului},” “\textit{puncte principale},” “\textit{pietre de hotar},” “\textit{principii de bază},” și “\textit{principii fundamentale}”. Aceste expresii denotă același lucru—temelia credinței noastre. Când auzim aceste expresii astăzi, cumva ele nu transmit nicio informație concretă. Dar pentru adventiștii de ziua a șaptea din timpul ei, acest lucru era foarte clar și un punct definit. Toți acești termeni se referă la sinopsa publică a credinței adventiștilor de ziua a șaptea numită \emcap{Principiile Fundamentale}, explicată mai jos.


God \egwinline{calls upon us to \textbf{hold firmly}, with the grip of faith, to \textbf{the \underline{fundamental principles}} that  are \textbf{based upon unquestionable authority}.} This is a reference to principal features of Seventh-day Adventist faith which God revealed to Adventist pioneers \egwinline{after the passing of the time in 1844,} when a group of keen, noble, and true men \egwinline{searched for the truth as for hidden treasure.} This was \textit{the foundation of our faith}. Our pioneers officially established the Seventh-day Adventist Church in 1863, and they taught these truths which they called “\textit{fundamental principles}.” But often, Seventh-day Adventists were misrepresented publicly. For this reason, in 1872, our pioneers published a document called “\textit{A Declaration of the Fundamental Principles, Taught and Practiced by the Seventh-day Adventists}” in order to publicly, but briefly, declare what \emcap{fundamental principles} Seventh-day Adventists taught and practiced. These \emcap{Fundamental Principles} were regularly printed as a standalone pamphlet, were present in our papers, and were annually printed in Adventist Yearbooks throughout Ellen White's lifetime.\footnote{See \hyperref[appendix:timeline]{Fundamental Principles - Timeline} for more details.} Therefore, when Ellen White referenced the “\textit{fundamental principles},” this was not a vague or opaque statement, since the Seventh-day Adventist church had officially and publicly declared what these \emcap{fundamental principles} were. In the preface of this document, we read the purpose behind this document.


Dumnezeu \egwinline{ne cheamă să \textbf{ținem cu tărie}, cu strânsoarea credinței, de \textbf{\underline{principiile fundamentale}} care sunt \textbf{bazate pe autoritate de necontestat}.} Aceasta este o referire la caracteristicile principale ale credinței adventiștilor de ziua a șaptea pe care Dumnezeu le-a revelat pionierilor adventiști \egwinline{după trecerea timpului în 1844,} când un grup de oameni zeloși, nobili și adevărați \egwinline{au căutat adevărul ca pe o comoară ascunsă.} Aceasta a fost \textit{temelia credinței noastre}. Pionierii noștri au înființat oficial Biserica Adventistă de Ziua a Șaptea în 1863, și ei au învățat aceste adevăruri pe care le-au numit “\textit{principii fundamentale}.” Dar adesea, adventiștii de ziua a șaptea erau reprezentați greșit în public. Din acest motiv, în 1872, pionierii noștri au publicat un document numit “\textit{O declarație a principiilor fundamentale, crezute și practicate de adventiștii de ziua a șaptea}” pentru a declara public, dar pe scurt, ce \emcap{principii fundamentale} învățau și practicau adventiștii de ziua a șaptea. Aceste \emcap{Principii Fundamentale} erau tipărite regulat ca broșură independentă, erau prezente în publicațiile noastre și erau tipărite anual în Anuarele Adventiste pe tot parcursul vieții lui Ellen White.\footnote{Vezi \hyperref[appendix:timeline]{Principiile Fundamentale - Cronologie} pentru mai multe detalii.} Prin urmare, când Ellen White făcea referire la “\textit{principiile fundamentale},” aceasta nu era o afirmație vagă sau opacă, deoarece Biserica Adventistă de Ziua a Șaptea declarase oficial și public care erau aceste \emcap{principii fundamentale}. În prefața acestui document, citim scopul din spatele acestui document.


\begin{figure}
    \centering
    \includegraphics[width=1\linewidth]{images/declaration-of-the-fundamental-principles.PNG}
    \caption*{Scan of the Declaration of the Fundamental Principles, 1872.}
    \label{fig:declaration-of-the-fundamental-principles}
\end{figure}


\begin{figure}
    \centering
    \includegraphics[width=1\linewidth]{images/declaration-of-the-fundamental-principles.PNG}
    \caption*{Scan al Declarației Principiilor Fundamentale, 1872.}
    \label{fig:declaration-of-the-fundamental-principles}
\end{figure}


\others{In presenting to the \textbf{public} this \textbf{synopsis of our faith}, we wish to have it distinctly understood that \textbf{we have no articles of faith, creed, or discipline, }\textbf{\underline{aside from the Bible}}. We \textbf{do not} put forth this as \textbf{having any authority with our people}, \textbf{nor is it designed to secure uniformity among them}, \textbf{as a system of faith}, \textbf{but is a brief statement of \underline{what is, and has been, with great unanimity, held by them}}. We often find it necessary to meet inquiries on this subject, and sometimes to correct false statements circulated against us, and to remove erroneous impressions which have obtained with those who have not had an opportunity to become acquainted with our faith and practice. Our only object is to meet this necessity.}


\others{Prezentând \textbf{publicului} această \textbf{sinopsă a credinței noastre}, dorim să fie înțeles clar că \textbf{nu avem articole de credință, crez sau disciplină, }\textbf{\underline{în afară de Biblie}}. \textbf{Nu} prezentăm aceasta ca \textbf{având vreo autoritate asupra poporului nostru}, \textbf{nici nu este concepută pentru a asigura uniformitate între ei}, \textbf{ca sistem de credință}, \textbf{ci este o scurtă declarație a \underline{ceea ce este și a fost, cu mare unanimitate, susținut de ei}}. Găsim adesea necesar să răspundem întrebărilor pe acest subiect și uneori să corectăm declarații false răspândite împotriva noastră și să înlăturăm impresii eronate care s-au format la cei care nu au avut oportunitatea să se familiarizeze cu credința și practica noastră. Singurul nostru obiectiv este să răspundem acestei necesități.}


\othersnogap{\textbf{As Seventh-day Adventists we desire simply that our position shall be understood}; and we are the more solicitous for this because there are many who call themselves Adventists who hold views with which we can have no sympathy, some of which, we think, are subversive of the plainest and most important principles set forth in the word of God...}[The Fundamental Principles 1872, p. 3.1][https://egwwritings.org/read?panels=p928.8]


\othersnogap{\textbf{Ca adventiști de ziua a șaptea dorim pur și simplu ca poziția noastră să fie înțeleasă}; și suntem cu atât mai preocupați de aceasta pentru că sunt mulți care se numesc adventiști care susțin puncte de vedere cu care nu putem avea nicio simpatie, unele dintre ele, credem noi, sunt subversive față de cele mai clare și mai importante principii prezentate în cuvântul lui Dumnezeu...}[Principiile Fundamentale 1872, p. 3.1][https://egwwritings.org/read?panels=p928.8]


This synopsis of faith consisted of 25 points, which represented \others{what is, and has been, with great unanimity, held by} Seventh-day Adventists. These 25 points constituted \egwinline{\textbf{the foundation} that was \textbf{laid at the beginning} of our work \textbf{by prayerful study} of the word and by revelation}. In 1904, Sister White told us that \egwinline{upon \textbf{this foundation} we have been building for \textbf{the past fifty years}.} These are the \egwinline{\textbf{the fundamental principles that are based upon unquestionable authority}}, that God \egwinline{calls upon us to \textbf{hold firmly}, with the grip of faith}. In other words, she repeated, \egwinline{we are to \textbf{hold to the sure pillars of our faith}}.


Această sinopsă a credinței consta din 25 de puncte, care reprezentau \others{ceea ce este și a fost, cu mare unanimitate, susținut de} adventiștii de ziua a șaptea. Aceste 25 de puncte constituiau \egwinline{\textbf{temelia} care a fost \textbf{pusă la început} lucrării noastre \textbf{prin studiu cu rugăciune} a cuvântului și prin revelație}. În 1904, sora White ne-a spus că \egwinline{pe \textbf{această temelie} am construit în \textbf{ultimii cincizeci de ani}.} Acestea sunt \egwinline{\textbf{principiile fundamentale care sunt bazate pe autoritate de necontestat}}, pe care Dumnezeu \egwinline{ne cheamă să le \textbf{ținem cu tărie}, cu strânsoarea credinței}. Cu alte cuvinte, ea a repetat, \egwinline{trebuie să \textbf{ținem de stâlpii siguri ai credinței noastre}}.


In 1904, Sister White wrote about\egwinline{the \textbf{efforts of the enemy to undermine the foundation of our faith}}. She wrote about the movement that would\egwinline{consist in \textbf{giving up} the doctrines which stand as \textbf{the pillars of our faith}}. This reformation, if accepted, would discard\egwinline{\textbf{the principles of truth} that God in His wisdom has given to the remnant church} and\egwinline{\textbf{the fundamental principles} that have sustained the work for the last fifty years \textbf{would be accounted as error}}. This movement started about the time when Dr. John H. Kellogg published the book, “Living Temple”.


În 1904, sora White a scris despre \egwinline{\textbf{eforturile vrăjmașului de a submina temelia credinței noastre}}. Ea a scris despre mișcarea care ar \egwinline{consta în \textbf{renunțarea la} doctrinele care stau ca \textbf{stâlpii credinței noastre}}. Această reformă, dacă ar fi acceptată, ar înlătura \egwinline{\textbf{principiile adevărului} pe care Dumnezeu în înțelepciunea Sa le-a dat bisericii rămășiței} și \egwinline{\textbf{principiile fundamentale} care au susținut lucrarea în ultimii cincizeci de ani \textbf{ar fi considerate ca eroare}}. Această mișcare a început cam în timpul când Dr. John H. Kellogg a publicat cartea “Templul viu”.


\egw{About the time that ‘Living Temple’ was published, there passed before me in the night season, \textbf{representations indicating that some danger was approaching}, and that I must prepare for it by \textbf{writing out the things} God has revealed to me \textbf{regarding \underline{the foundation principles of our faith}}.}[SpTB02 52.3; 1904][https://egwwritings.org/read?panels=p417.267]


\egw{Cam în timpul când ‘Templul viu’ a fost publicat, au trecut înaintea mea în timpul nopții, \textbf{reprezentări indicând că se apropie un pericol}, și că trebuie să mă pregătesc pentru el \textbf{scriind lucrurile} pe care Dumnezeu mi le-a revelat \textbf{cu privire la \underline{principiile de bază ale credinței noastre}}.}[SpTB02 52.3; 1904][https://egwwritings.org/read?panels=p417.267]


By publishing “Living Temple”, \textbf{foundation principles of our faith} \textbf{would be undermined}\egwinline{through the dissemination of \textbf{seductive theories}} contained therein.


Prin publicarea cărții “Templul viu”, \textbf{principiile de bază ale credinței noastre} \textbf{ar fi subminate} \egwinline{prin diseminarea \textbf{teoriilor seducătoare}} conținute în ea.


\egw{I have been instructed by the heavenly messenger that some of the reasoning in the book, ‘Living Temple,’ is unsound and that \textbf{this reasoning would lead astray} the minds of those who are not thoroughly established on \textbf{the foundation principles} of present truth. It introduces that which is naught but speculation in \textbf{regard to \underline{the personality of God and where His presence is}}.}[SpTB02 51.3; 1904][https://egwwritings.org/read?panels=p417.262]


\egw{Am fost instruită de mesagerul ceresc că unele dintre raționamentele din cartea „Templul viu” sunt nefondate și că \textbf{aceste raționamente ar duce în rătăcire} mințile celor care nu sunt pe deplin stabiliți pe \textbf{principiile fundamentale} ale adevărului prezent. Ea introduce ceea ce nu este altceva decât speculație în \textbf{privința \underline{personalității lui Dumnezeu și a locului unde este prezența Sa}}.}[SpTB02 51.3; 1904][https://egwwritings.org/read?panels=p417.262]


Sister White is very particular in pointing out that the reasoning contained in the book Living Temple,\egwinline{\textbf{would lead astray}} from the\egwinline{\textbf{the foundation principles} of present truth}. These reasonings are in\egwinline{\textbf{regard to the personality of God and where His presence is}}.


Sora White este foarte precisă în a sublinia că raționamentele conținute în cartea Templul viu,\egwinline{\textbf{ar duce în rătăcire}} de la\egwinline{\textbf{principiile fundamentale} ale adevărului prezent}. Aceste raționamente sunt în\egwinline{\textbf{privința personalității lui Dumnezeu și a locului unde este prezența Sa}}.


As mentioned before, the word ‘\textit{personality’}, in the context of the nineteenth century, is defined as “\textit{the quality or state of being a person}”\footnote{\href{https://www.merriam-webster.com/dictionary/personality}{Merriam-Webster Dictionary}, word ‘\textit{personality}’}. In other words, this term conveys the answer to the question, “\textit{what is it that defines someone to be a person?}”, “\textit{What is the quality or state of someone being a person?}” In the case of the \emcap{personality of God}, the question is, “\textit{Is God a person and what is it that defines Him as being a person? What is the quality or state of God being a person?}”


După cum s-a menționat anterior, cuvântul ‘\textit{personalitate}’, în contextul secolului al nouăsprezecelea, este definit ca „\textit{caracteristica sau starea prin care cineva este definit ca persoană}”\footnote{\href{https://www.merriam-webster.com/dictionary/personality}{Dicționarul Merriam-Webster}, cuvântul ‘\textit{personalitate}’}. Cu alte cuvinte, acest termen transmite răspunsul la întrebarea: „\textit{ce anume definește pe cineva ca fiind o persoană?}”, „\textit{Care este caracteristica sau starea prin care cineva este o persoană?}” În cazul \emcap{personalității lui Dumnezeu}, întrebarea este: „\textit{Este Dumnezeu o persoană și ce anume Îl definește ca fiind o persoană? Care este caracteristica sau starea prin care Dumnezeu este definit ca persoană?}”


The reasoning of Dr. Kellogg regarding these questions expressed in the book Living Temple, is\egwinline{unsound}. The sentiments, in\egwinline{\textbf{regard to the personality of God and where His presence is}},\egwinline{advocated in the book, did not bear the indorsement of God, and that they were \textbf{a snare that the enemy had prepared for the last days}}. As we are living in the last days, we ought to ask ourselves these questions. Likewise, we are to question the biblical validity of the statements in the \emcap{Fundamental Principles} regarding the \emcap{personality of God} and where His presence is. How do the \emcap{Fundamental Principles} define God as being a person, and what do they say regarding God’s presence?


Raționamentul Dr. Kellogg cu privire la aceste întrebări exprimate în cartea Templul viu, este\egwinline{nefondat}. Opiniile, în\egwinline{\textbf{privința personalității lui Dumnezeu și a locului unde este prezența Sa}},\egwinline{susținute în carte, nu au purtau aprobarea lui Dumnezeu și că erau \textbf{o cursă pe care vrăjmașul a pregătit-o pentru zilele din urmă}}. Deoarece trăim în zilele din urmă, ar trebui să ne punem aceste întrebări. De asemenea, trebuie să punem sub semnul întrebării validitatea biblică a afirmațiilor din \emcap{Principiile Fundamentale} cu privire la \emcap{personalitatea lui Dumnezeu} și locul unde este prezența Sa. Cum definesc \emcap{Principiile Fundamentale} pe Dumnezeu ca fiind o persoană și ce spun ele despre prezența lui Dumnezeu?


The first point listed below deals with the \emcap{personality of God} and His presence. The second point gives the context to the first. Please consider a few questions while reading them: Who is referred to as one God? How is God defined as a person or in other words, what is the quality or state of Him being a person? How do these points talk about the presence of God?


Primul punct enumerat mai jos se ocupă de \emcap{personalitatea lui Dumnezeu} și de prezența Sa. Al doilea punct oferă contextul pentru primul. Vă rugăm să luați în considerare câteva întrebări în timp ce le citiți: Cine este menționat ca un singur Dumnezeu? Cum este definit Dumnezeu ca persoană sau, cu alte cuvinte, care este caracteristica sau starea prin care El este o persoană? Cum vorbesc aceste puncte despre prezența lui Dumnezeu?


\others{“I – That there is \textbf{one God}, \textbf{\underline{a personal, spiritual being}}, \textbf{the creator of all things}, omnipotent, omniscient, and eternal, infinite in wisdom, holiness, justice, goodness, truth, and mercy; unchangeable, and \textbf{\underline{everywhere present by his representative, the Holy Spirit}}. Ps. 139:7.”}


\others{„I – Că există \textbf{un singur Dumnezeu}, \textbf{\underline{o ființă personală, spirituală}}, \textbf{creatorul tuturor lucrurilor}, atotputernic, atotștiutor și etern, infinit în înțelepciune, sfințenie, dreptate, bunătate, adevăr și milă; neschimbător și \textbf{\underline{prezent pretutindeni prin reprezentantul Său, Duhul Sfânt}}. Ps. 139:7.”}


\othersnogap{II – That there is \textbf{one Lord Jesus Christ, }\textbf{\underline{the Son of the Eternal Father}}, the one \textbf{\underline{by}}\textbf{ whom God created all things}, and by whom they do consist; …”}[The Fundamental Principles 1889, point no. 1.,2.,.] \footnote{See \hyperref[chap:appendix]{Appendix} for the full list of the Fundamental Principles} \footnote{From 1872 until 1914, the Fundamental Principles remained constant and unchanged, with the exception in 1889, when James Smith added three new points. But during all those years, the points concerning “\textit{the personality of God}” and “\textit{where His presence is}” remained the same. }


\othersnogap{II – Că există \textbf{un singur Domn Isus Hristos, }\textbf{\underline{Fiul Tatălui Etern}}, cel \textbf{\underline{prin}}\textbf{ care Dumnezeu a creat toate lucrurile} și prin care ele subzistă; …“}[Principiile Fundamentale 1889, punctul nr. 1.,2.,.] \footnote{Vezi \hyperref[chap:appendix]{Anexa} pentru lista completă a Principiilor Fundamentale} \footnote{Din 1872 până în 1914, Principiile Fundamentale au rămas constante și neschimbate, cu excepția anului 1889, când James Smith a adăugat trei puncte noi. Dar în toți acești ani, punctele privind „\textit{personalitatea lui Dumnezeu}” și „\textit{locul unde este prezența Sa}” au rămas aceleași.}


In the time of Ellen White, Seventh-day Adventists believed in one God—a personal, spiritual being, the Creator of all things—and they believed that this God created everything by His Son Jesus Christ. They addressed the Father as one God, and they addressed Christ as the Son of God. The quality or state of God being a person is expressed in the term “\textit{personal, spiritual being}”. Regarding His presence, the \emcap{Fundamental Principles} state that He is everywhere present by His representative, the Holy Spirit. The meaning of these principles requires very special attention. Keeping within the historical context, this will be the subject of our following studies.


În timpul lui Ellen White, adventiștii de ziua a șaptea credeau într-un singur Dumnezeu—o ființă personală, spirituală, Creatorul tuturor lucrurilor—și credeau că acest Dumnezeu a creat totul prin Fiul Său Isus Hristos. Ei se adresau Tatălui ca un singur Dumnezeu și se adresau lui Hristos ca Fiul lui Dumnezeu. Caracteristica sau starea prin care Dumnezeu este definit ca persoană este exprimată în termenul „\textit{ființă personală, spirituală}”. În ceea ce privește prezența Sa, \emcap{Principiile Fundamentale} afirmă că El este prezent pretutindeni prin reprezentantul Său, Duhul Sfânt. Semnificația acestor principii necesită o atenție foarte specială. Păstrând contextul istoric, aceasta va fi subiectul studiilor noastre următoare.


\section*{The Test}


\section*{Testul}


Most obviously, these \emcap{fundamental principles} do not contain the doctrine of the Trinity! More precisely, the sentiments “\textit{three in one},” or “\textit{one in three}”, in reference to God, are nowhere to be found—which are present in today’s \textit{Fundamental Beliefs}. Only the Father is referred to as “\textit{one God’‘}. But before rushing to swift conclusions, and condemning the doctrine of the Trinity as\egwinline{\textbf{seductive theories,}} which\egwinline{\textbf{undermine the foundation of our faith}}, please bear in mind that Sister White presents a comprehensive list of characteristics that must be fulfilled in order for it to be deemed as such.


În mod evident, aceste \emcap{principii fundamentale} nu conțin doctrina Trinității! Mai precis, opiniile „\textit{trei în unul}” sau „\textit{unul în trei}”, cu referire la Dumnezeu, nu se găsesc nicăieri—care sunt prezente în \textit{Punctele Fundamentale de Credință} de astăzi. Doar Tatăl este menționat ca „\textit{un singur Dumnezeu}”. Dar înainte de a ne grăbi să tragem concluzii rapide și să condamnăm doctrina Trinității ca\egwinline{\textbf{teorii seducătoare,}} care\egwinline{\textbf{subminează temelia credinței noastre}}, vă rugăm să țineți cont că sora White prezintă o listă cuprinzătoare de caracteristici care trebuie îndeplinite pentru ca aceasta să fie considerată ca atare.


If the Trinity doctrine is questionable, then the trinitarian sentiments would need to:
\begin{itemize}
    \item rob the people of God of their past experience
    \item destroy the \emcap{personality of God}
    \item tear down the pillars of our faith or lead astray from the foundation principles
    \item be presented as if Mrs. White supported them
\end{itemize}


Dacă doctrina Trinității este îndoielnică, atunci opiniile trinitariene ar trebui să:
\begin{itemize}
    \item jefuiască poporul lui Dumnezeu de experiența lor din trecut
    \item distrugă \emcap{personalitatea lui Dumnezeu}
    \item dărâme stâlpii credinței noastre sau să abată de la principiile de bază
    \item fie prezentate ca și cum doamna White le-ar fi susținut
\end{itemize}


It is not our intention to deal with any of Kellogg’s seductive theories, but rather to study the \emcap{personality of God} in its historical background. As we do this, we will face the evidence of Sister White reactively warning the church of these characteristics.


Nu este intenția noastră să ne ocupăm de vreuna dintre teoriile seducătoare ale lui Kellogg, ci mai degrabă să studiem \emcap{personalitatea lui Dumnezeu} în contextul ei istoric. Pe măsură ce facem aceasta, vom înfrunta dovezile Sorei White avertizând reactiv biserica despre aceste caracteristici.


% The Fundamental Prinicples

\begin{titledpoem}
    
    \stanza{
        A strong foundation God has laid \\
        Without a doubt it cannot fade \\
        Through earnest prayer and study deep \\
        The truths revealed we now must keep
    }

    \stanza{
        There is one God, He’s personal \\
        With form, but also spiritual \\
        His Spirit present everywhere \\
        God is all-knowing, all-aware.
    }

    \stanza{
        Our Father, the Eternal One \\
        By Christ His dear begotten Son \\
        Created all that does exist. \\
        These precious truths let’s not resist
    }

    \stanza{
        When men step off the platform firm \\
        And strange new teachings they would learn \\
        Then doctrines false with truth combine \\
        God’s principles they undermine
    }

    \stanza{
        Hold firm the truth with faithful grip \\
        Let not these anchors ever slip \\
        For what God wrought through pioneer’s hands \\
        Through time and tempest ever stands.
    }
    
\end{titledpoem}