% \qrchapter{https://forgottenpillar.com/rsc/en-fp-chapter4}{Revision of “Living Temple”}


\qrchapter{https://forgottenpillar.com/rsc/ro-fp-chapter4}{Revizuirea cărții „Templul viu”}


In \textit{Testimonies for the Church Containing Letters to Physicians and Ministers Instruction to Seventh-Day Adventists}, the tenth chapter, \textit{The Foundation of our Faith,} God gave valuable lessons on the development and consequences of Kellogg's theories. The broader and deeper meaning of these quotations can be understood when we are familiar with their historical context. Let us first take a brief look at the historical context of Kellogg's book, The \textit{Living Temple}.


În \textit{Mărturii pentru Biserică – Scrisori către medici și pastori, îndrumări pentru adventiștii de ziua a șaptea}, capitolul al zecelea, \textit{Temelia credinței noastre}, Dumnezeu a dat lecții valoroase despre dezvoltarea și consecințele teoriilor lui Kellogg. Sensul mai larg și mai profund al acestor citate poate fi înțeles când suntem familiarizați cu contextul lor istoric. Să aruncăm mai întâi o scurtă privire asupra contextului istoric al cărții lui Kellogg, \textit{Templul viu}.


In a series of providence, God signified that “\textit{Living Temple}” should not be printed. One such event was the burning of Battle Creek's press building, just the night before it was to be printed. Finally, the book was printed elsewhere; it instigated a great crisis in the Seventh-day Adventist Church. On October 7, 1903, a annual meeting of the conference was held in Washington DC. Many Seventh-day Adventist church leaders were present, including Dr. Kellogg and his sympathizers. Major controversy was taking place over this book and the conflict was inevitable. Fortunately, on the brink of this escalating conflict, a letter from Sister White was delivered to the council. On Sunday, the letter fell upon the ears of all, to which there resounded many “amen's” and “halleluyah's”. It was a very tense and moving morning for the church that was on the verge of a split—to at last have concrete direction from the Lord's messenger:


Într-o serie de providențe, Dumnezeu a arătat că „\textit{Templul viu}” nu ar trebui tipărit. Un astfel de eveniment a fost arderea clădirii tipografiei din Battle Creek, chiar în noaptea dinaintea tipăririi. În cele din urmă, cartea a fost tipărită în altă parte; ea a stârnit o mare criză în Biserica Adventistă de Ziua a Șaptea. Pe 7 octombrie 1903, o întâlnire anuală a conferinței a avut loc în Washington DC. Mulți lideri ai Bisericii Adventiste de Ziua a Șaptea au fost prezenți, inclusiv Dr. Kellogg și simpatizanții săi. O controversă majoră avea loc în legătură cu această carte și conflictul era inevitabil. Din fericire, în pragul acestui conflict în escaladare, o scrisoare de la sora White a fost livrată consiliului. Duminică, scrisoarea a căzut asupra urechilor tuturor, la care au răsunat multe „amin”-uri și „aleluia”-uri. A fost o dimineață foarte tensionată și emoționantă pentru biserica care era în pragul unei scindări—să aibă în sfârșit o îndrumare concretă de la solul Domnului:


\begin{figure}[h]
    \centering
    \includegraphics[width=1\linewidth]{images/review-and-herlad.jpg}
    \caption*{Burning of Review and Herald press building, December 30, 1902.}
    \label{fig:review-and-herald}
\end{figure}


\begin{figure}[h]
    \centering
    \includegraphics[width=1\linewidth]{images/review-and-herlad.jpg}
    \caption*{Arderea clădirii tipografiei Review and Herald, 30 decembrie 1902.}
    \label{fig:review-and-herald}
\end{figure}


\egw{I have some things to say to our teachers in reference to \textbf{the new book The Living Temple}. \textbf{Be careful how you sustain \underline{the sentiments of this book regarding the personality of God}}. As the Lord presents matters to me, \textbf{these sentiments do not bear the endorsement of God}. \textbf{They are a snare that the enemy has prepared for these last days}. I thought that this would surely be discerned and that it would not be necessary for me to say anything about it. \textbf{But since the claim has been made that the teachings of this book can be sustained by statements from my writings, I am compelled to speak in denial of this claim}. There may be in this book expressions and sentiments that are in harmony with my writings. And there may be in my writings many statements which when taken from their connection, and interpreted according to the mind of the writer of Living Temple, would seem to be in harmony with the teachings of this book. \textbf{This may give apparent support to the assertion that the sentiments in Living Temple are in harmony with my writings}. \textbf{But God forbid that this opinion should prevail}.}[Lt211-1903.1; 1903][https://egwwritings.org/read?panels=p14068.9598008]


\egw{Am câteva lucruri de spus învățătorilor noștri cu privire la \textbf{noua carte Templul viu}. \textbf{Fiți atenți cum susțineți \underline{opiniile din această carte privind personalitatea lui Dumnezeu}}. După cum Domnul îmi prezintă lucrurile, \textbf{aceste opinii nu poartă aprobarea lui Dumnezeu}. \textbf{Ele sunt o cursă pe care vrăjmașul a pregătit-o pentru aceste zile din urmă}. Am crezut că aceasta va fi cu siguranță discernută și că nu va fi necesar să spun ceva despre ea. \textbf{Dar deoarece s-a pretins că învățăturile acestei cărți pot fi susținute prin declarații din scrierile mele, sunt constrânsă să vorbesc pentru a nega această afirmație}. Pot fi în această carte expresii și opinii care sunt în armonie cu scrierile mele. Și pot fi în scrierile mele multe declarații care, atunci când sunt luate din contextul lor și interpretate conform minții autorului cărții Templul viu, ar părea să fie în armonie cu învățăturile acestei cărți. \textbf{Aceasta poate da un sprijin aparent afirmației că opiniile din Templul viu sunt în armonie cu scrierile mele}. \textbf{Dar Dumnezeu să ferească să prevaleze această opinie}.}[Lt211-1903.1; 1903][https://egwwritings.org/read?panels=p14068.9598008]


Repeatedly, Sister White stated that the true problem of the book was the sentiments\egwinline{\textbf{regarding the personality of God}}. These sentiments are not sustained by statements from Ellen White's writings and these very sentiments\egwinline{\textbf{are a snare that the enemy has prepared for these last days}}.


În mod repetat, sora White a declarat că adevărata problemă a cărții erau opiniile\egwinline{\textbf{privind personalitatea lui Dumnezeu}}. Aceste opinii nu sunt susținute de declarații din scrierile lui Ellen White și chiar aceste opinii\egwinline{\textbf{sunt o cursă pe care vrăjmașul a pregătit-o pentru aceste zile din urmă}}.


God, again in His providence, solved this conflict. Kellogg accepted the reproof from the Lord's messenger and, before the council closed, he stated that the Living Temple would be taken from the market\footnote{\href{https://forgottenpillar.com/wp-content/uploads/2022/04/Letter-A-G-Daniells-to-W-C-White-October-29-1903.pdf}{Letter: A. G. Daniells to W. C. White, October 23, 1903, pp. 5}}. But after the conference, he spoke privately with the general conference president, Brother Arthur G. Daniells, about his plans for revising the book. The following is a look at select letters, revealing Kellogg's plans for revising “\textit{Living Temple}”.


Dumnezeu, din nou în providența Sa, a rezolvat acest conflict. Kellogg a acceptat mustrarea de la solul Domnului și, înainte de încheierea consiliului, a declarat că Templul viu va fi retras de pe piață\footnote{\href{https://forgottenpillar.com/wp-content/uploads/2022/04/Letter-A-G-Daniells-to-W-C-White-October-29-1903.pdf}{Scrisoare: A. G. Daniells către W. C. White, 23 octombrie 1903, pp. 5}}. Dar după conferință, el a vorbit în privat cu președintele conferinței generale, fratele Arthur G. Daniells, despre planurile sale de revizuire a cărții. Următoarea este o privire asupra unor scrisori selectate, care dezvăluie planurile lui Kellogg pentru revizuirea „\textit{Templului viu}”.


Ellen White was not present at the yearly conference in Washington DC but her son, William C. White, did attend. When the conference was over, brother Arthur G. Daniells wrote a confidential letter to William C. White regarding Dr. Kellogg's plan to revise his book:


Ellen White nu a fost prezentă la conferința anuală din Washington DC, dar fiul ei, William C. White, a participat. Când conferința s-a încheiat, fratele Arthur G. Daniells a scris o scrisoare confidențială lui William C. White cu privire la planul Dr. Kellogg de a-și revizui cartea:


\others{October 29, 1903}


\others{29 octombrie 1903}


\othersnogap{Ever since the \textbf{council closed} I have felt that I should write you \textbf{confidentially regarding Dr. Kellogg's plans for revising and republishing ‘The Living Temple’}…. He \normaltext{[Kellogg]} said that some days before coming to the council, he had been thinking the matter over, and began to see that \textbf{he had made a slight mistake in expressing his views}. He said that all the way along he had been troubled to know how to state the character of God and his relation to his creation works…}


\othersnogap{De când \textbf{s-a încheiat consiliul} am simțit că ar trebui să vă scriu \textbf{confidențial cu privire la planurile Dr. Kellogg de a revizui și republica „Templul viu”}…. El \normaltext{[Kellogg]} a spus că cu câteva zile înainte de a veni la consiliu, se gândise la această chestiune și a început să vadă că \textbf{a făcut o mică greșeală în exprimarea punctelor sale de vedere}. A spus că tot timpul a avut probleme să știe cum să exprime caracterul lui Dumnezeu și relația Sa cu lucrările Sale de creație…}


\begin{figure}[hp]
    \centering
    \includegraphics[width=1\linewidth]{images/daniels.jpg}
    \caption*{Arthur Grosvenor Daniells (1858-1935)}
    \label{fig:daniells}
\end{figure}


\begin{figure}[hp]
    \centering
    \includegraphics[width=1\linewidth]{images/daniels.jpg}
    \caption*{Arthur Grosvenor Daniells (1858-1935)}
    \label{fig:daniells}
\end{figure}


\othersnogap{\textbf{He then stated that his former views \underline{regarding the trinity} had stood in his way of making a clear and absolutely correct statement; but that within a short time \underline{he had come to believe in the trinity} and could now see pretty clearly where all the difficulty was, and believed that he could clear the matter up satisfactorily.}}


\othersnogap{\textbf{El a declarat apoi că vederile sale anterioare \underline{cu privire la trinitate} îl împiedicaseră să facă o afirmație clară și absolut corectă; dar că într-un timp scurt \underline{ajunsese să creadă în trinitate} și putea acum să vadă destul de clar unde era toată dificultatea și credea că poate clarifica problema în mod satisfăcător.}}


\othersnogap{\textbf{He told me that he now believed in \underline{God the Father, God the Son, and God the Holy Ghost}; and his view was that it was God the Holy Ghost, and not God the Father, that filled all space, and every living thing. He said if he had believed \underline{this} before writing the book, he could have expressed his views without giving the wrong impression the book now gives.}}


\othersnogap{\textbf{Mi-a spus că acum credea în \underline{Dumnezeu Tatăl, Dumnezeu Fiul și Dumnezeu Duhul Sfânt}; și vederea lui era că Dumnezeu Duhul Sfânt, și nu Dumnezeu Tatăl, era cel care umplea tot spațiul și fiecare ființă vie. A spus că dacă ar fi crezut \underline{aceasta} înainte de a scrie cartea, și-ar fi putut exprima vederile fără a da impresia greșită pe care cartea o dă acum.}}


\othersnogap{\textbf{I placed before him the objections I found in the teaching, and tried to show him that the teaching was so utterly contrary to the gospel that I did not see how it could be revised by changing a few expressions.}}


\othersnogap{\textbf{I-am prezentat obiecțiile pe care le-am găsit în învățătură și am încercat să-i arăt că învățătura era atât de complet contrară evangheliei încât nu vedeam cum ar putea fi revizuită prin schimbarea câtorva expresii.}}


\othersnogap{We argued the matter at some length in a friendly way; but I felt sure that when we parted, the doctor did not understand himself, nor the character of his teaching. And I could not see how it would be possible for him to flop over, \textbf{and in the course of a few days \underline{fix the books up} so that it would be all right}.}[Letter: A. G. Daniells to W. C. White, October 29, 1903. pp. 1, 2][https://forgotten-pillar.s3.us-east-2.amazonaws.com/Letter-A-G-Daniells-to-W-C-White-October-29-1903.pdf]


\othersnogap{Am discutat problema pe larg într-un mod prietenos; dar am simțit sigur că atunci când ne-am despărțit, doctorul nu se înțelegea pe sine însuși, nici caracterul învățăturii sale. Și nu puteam vedea cum ar fi posibil pentru el să se răstoarne, \textbf{și în decursul a câteva zile să \underline{repare cărțile} astfel încât totul să fie în regulă}.}[Scrisoare: A. G. Daniells către W. C. White, 29 octombrie 1903. pp. 1, 2][https://forgotten-pillar.s3.us-east-2.amazonaws.com/Letter-A-G-Daniells-to-W-C-White-October-29-1903.pdf]


Kellogg did not see the mistake in his sentiments; but rather, in expressing his views. He did not think that his views were false, merely his expression of those views, which led to the book giving a wrong impression. Yet, evidently, this was not true. As Sister White had stated, Kellogg had a problem with the sentiments regarding the \emcap{personality of God} and where His presence is. So, Kellogg suggested that in order to “\textit{fix the books up}” he would include the trinitarian expressions because he now started to believe in \textit{the Trinity} doctrine. At this point in time, the Seventh-day Adventist Church was not trinitarian—the doctrine of Trinity was not part of the \emcap{Fundamental Principles}, as we saw previously. Thus, it is no surprise that Brother Daniels objected and refuted Trinitarian teaching, claiming that it was\others{so utterly contrary to the gospel.} Revising the book, by changing a few expressions, would not change the main problem of the book: the sentiments on the \emcap{personality of God}.


Kellogg nu vedea greșeala în opiniile sale; ci mai degrabă, în exprimarea vederilor sale. El nu credea că vederile sale erau false, ci doar exprimarea acelor vederi, care a dus la faptul că cartea a dat o impresie greșită. Totuși, evident, aceasta nu era adevărat. După cum declarase sora White, Kellogg avea o problemă cu opiniile privind \emcap{personalitatea lui Dumnezeu} și unde este prezența Sa. Astfel, Kellogg a sugerat că pentru a „\textit{repara cărțile}” ar include expresiile trinitariene pentru că acum începuse să creadă în doctrina \textit{Trinității}. În acel moment, Biserica Adventistă de Ziua a Șaptea nu era trinitariană—doctrina Trinității nu făcea parte din \emcap{Principiile Fundamentale}, așa cum am văzut anterior. Astfel, nu este de mirare că fratele Daniells a obiectat și a respins învățătura trinitariană, susținând că era \others{atât de complet contrară evangheliei.} Revizuirea cărții, prin schimbarea câtorva expresii, nu ar schimba problema principală a cărții: opiniile despre \emcap{personalitatea lui Dumnezeu}.


In the described events, and in William White's response to Brother Daniells, we can see why Sister White wrote the Special Testimonies. William White responded to Brother Daniells on Nov. 4, 1903:


În evenimentele descrise și în răspunsul lui William White către fratele Daniells, putem vedea de ce sora White a scris Mărturiile Speciale. William White a răspuns fratelui Daniells pe 4 noiembrie 1903:


\others{Dear Brother, --}


\others{Dragă frate, --}


\othersnogap{\textbf{\underline{Mother and I} have just read your letter of \underline{October 29} in which you speak of the \underline{various plans that have been proposed for the revising and reproduction of ‘The Living Temple}.’}}


\othersnogap{\textbf{\underline{Mama și cu mine} tocmai am citit scrisoarea ta din \underline{29 octombrie} în care vorbești despre \underline{diferitele planuri care au fost propuse pentru revizuirea și reproducerea cărții „Templul viu}.”}}


\othersnogap{We were pleasantly surprised at the announcement that Dr. Kellogg would withdraw this book from the market, \textbf{and we are sorry indeed that his mind is swinging back to the plan of revising it, \underline{Mother expresses herself quite emphatically regarding this matter; she regards it as an unprofitable undertaking}}. I think she will write to you soon expressing her views regarding this.}


\othersnogap{Am fost plăcut surprinși de anunțul că Dr. Kellogg va retrage această carte de pe piață, \textbf{și ne pare foarte rău că mintea lui se întoarce înapoi la planul de a o revizui, \underline{Mama se exprimă destul de categoric cu privire la această chestiune; ea o consideră o întreprindere neprofitabilă}}. Cred că îți va scrie curând exprimându-și vederile cu privire la aceasta.}


\othersnogap{\textbf{… I believe it will be necessary \underline{to issue a special Testimony soon}, and this must contain a very full and clear statement on the positive side of this question, as well as articles pointing out the errors in the teaching of those who have departed from the truth through fascinating and deceptive theories}.}[\href{https://ellenwhite.org/letterbooks/555}{Letter from W.C. White to A.G. Daniells, Nov. 4, 1903,} (p. 458)]


\othersnogap{\textbf{… Cred că va fi necesar \underline{să publicăm în curând o Mărturie specială}, și aceasta trebuie să conțină o declarație foarte completă și clară asupra aspectului pozitiv al acestei chestiuni, precum și articole care să evidențieze erorile din învățătura celor care s-au îndepărtat de adevăr prin teorii fascinante și înșelătoare}.}[\href{https://ellenwhite.org/letterbooks/555}{Scrisoare de la W.C. White către A.G. Daniells, 4 nov. 1903,} (p. 458)]


\begin{figure}[h]
    \centering
    \includegraphics[width=1\linewidth]{images/correspondance.jpg}
    \caption*{Correspondence chain between A. G. Daniells, W. C. White, Ellen White and Dr. John H. Kellogg.}
    \label{fig:corespondance}
\end{figure}


\begin{figure}[h]
    \centering
    \includegraphics[width=1\linewidth]{images/correspondance.jpg}
    \caption*{Lanțul de corespondență între A. G. Daniells, W. C. White, Ellen White și Dr. John H. Kellogg.}
    \label{fig:corespondance}
\end{figure}


Here is evidence that Sister White was familiar with Dr. Kellogg's intentions to revise “\textit{Living Temple}” and her familiarity with his belief in the Trinity doctrine. In William's words, she expressed herself quite emphatically regarding this matter. She deemed it an unprofitable undertaking. For this reason, it was necessary to issue a special Testimony soon. And there it was. This is how the \textit{Testimonies for the Church Containing Letters to Physicians and Ministers Instruction to Seventh-Day Adventists} was published in 1904, containing letters to the physicians and ministers connected to Kellogg's crisis.


Iată dovada că sora White era familiarizată cu intențiile Dr. Kellogg de a revizui „\textit{Templul viu}” și cu familiaritatea ei cu credința lui în doctrina Trinității. În cuvintele lui William, ea s-a exprimat destul de categoric cu privire la această chestiune. Ea a considerat-o o întreprindere neprofitabilă. Din acest motiv, a fost necesar să se publice în curând o Mărturie specială. Și iată-o. Așa a fost publicată \textit{Mărturii pentru Biserică – Scrisori către medici și pastori, îndrumări pentru adventiștii de ziua a șaptea} în 1904, conținând scrisori către medici și pastori legate de criza lui Kellogg.


By saying \others{\textbf{\underline{Mother and I} have just read your letter of \underline{October 29}}}, William testified that Sister White was fully aware of Kellogg's intentions and trinitarian belief. After she read Daniells’ letter, she wrote a direct reply to Dr. Kellogg. This letter is \textit{Lt253-1903}. It is a very prominent and eye opening letter because it clearly exposes how the prophet dealt with the Trinity doctrine. She elevated the doctrine on the \emcap{personality of God} constituted in the \emcap{Fundamental Principles}. There are striking similarities between this letter and the tenth chapter of the Special Testimonies, \textit{The Foundation of our Faith}.


Spunând \others{\textbf{\underline{Mama și cu mine} tocmai am citit scrisoarea ta din \underline{29 octombrie}}}, William a mărturisit că sora White era pe deplin conștientă de intențiile lui Kellogg și de credința lui trinitariană. După ce a citit scrisoarea lui Daniells, ea a scris un răspuns direct către Dr. Kellogg. Această scrisoare este \textit{Lt253-1903}. Este o scrisoare foarte proeminentă și revelatoare deoarece expune clar modul în care profetul a tratat doctrina Trinității. Ea a ridicat doctrina despre \emcap{personalitatea lui Dumnezeu} constituită în \emcap{Principiile Fundamentale}. Există asemănări izbitoare între această scrisoare și capitolul al zecelea din Mărturiile Speciale, \textit{Temelia credinței noastre}.


% Revision of the Living Temple

\begin{titledpoem}
    
    \stanza{
        In Kellogg’s book, a subtle snare \\
        Though well-disguised through crafty care \\
        From Bible truth would lead away \\
        And cause some precious souls to stray.
    }

    \stanza{
        And though much scripture there was used \\
        The early truth became confused \\
        This error served to twist the mind \\
        But in God’s Word the truth we find.
    }

    \stanza{
        God’s personality has form \\
        To Bible truth we must conform \\
        On this the Doctor wasn’t clear \\
        But early Advent truth is dear
    }
    
\end{titledpoem}