% \qrchapter{https://forgottenpillar.com/rsc/en-fp-chapter21}{Remembering the beginning} \label{chap:remembering-the-beginning}


\qrchapter{https://forgottenpillar.com/rsc/ro-fp-chapter21}{Amintindu-ne de început} \label{chap:remembering-the-beginning}


\egw{\textbf{We cannot for a moment have any \underline{misrepresentation} upon these solemn and important subjects of truth which have been the faith of our people since 1844.}}[Lt300-1903.9; 1903][https://egwwritings.org/read?panels=p7705.15]


\egw{\textbf{Nu putem avea nici pentru o clipă vreo \underline{denaturare} asupra acestor subiecte solemne și importante ale adevărului care au fost credința poporului nostru din 1844.}}[Lt300-1903.9; 1903][https://egwwritings.org/read?panels=p7705.15]


The true meaning of the \emcap{Fundamental Principles} is a broader view of the three angels’ messages.


Adevăratul înțeles al \emcap{Principiilor Fundamentale} este o viziune mai largă asupra soliilor celor trei îngeri.


\egw{\textbf{We are God’s commandment-keeping people}. For the past fifty years every phase of heresy has been brought to bear upon us, to \textbf{becloud our minds regarding the teaching of the word,—\underline{especially concerning the ministration of Christ in the heavenly sanctuary}, and the message of heaven for these last days, as \underline{given by the angels of the fourteenth chapter of Revelation}}. Messages of every order and kind have been urged upon Seventh-day Adventists, to \textbf{take the place of the truth which}, \textbf{point by point}, has been sought out by prayerful study, and testified to by the miracle-working power of the Lord. \textbf{But the way-marks which have made us what we are, are to be preserved, and they will be preserved}, as God has signified through His word and the testimony of His Spirit. \textbf{He calls upon us to hold firmly, with the grip of faith, to \underline{the fundamental principles} that are based upon unquestionable authority}.}[SpTB02 59.1; 1904][https://egwwritings.org/read?panels=p417.299]


\egw{\textbf{Noi suntem poporul lui Dumnezeu care păzește poruncile}. În ultimii cincizeci de ani, fiecare formă de erezie a fost adusă asupra noastră, pentru a \textbf{ne întuneca mințile cu privire la învățătura cuvântului,—\underline{în special cu privire la slujirea lui Hristos în sanctuarul ceresc}, și solia cerului pentru aceste zile din urmă, așa cum \underline{este dată de îngerii din capitolul paisprezece din Apocalipsa}}. Solii de tot felul și de orice natură au fost impuse asupra adventiștilor de ziua a șaptea, pentru a \textbf{lua locul adevărului care}, \textbf{punct cu punct}, a fost căutat prin studiu plin de rugăciune și mărturisit prin puterea făcătoare de minuni a Domnului. \textbf{Dar pietrele de hotar care ne-au făcut ceea ce suntem, trebuie să fie păstrate, și ele vor fi păstrate}, așa cum Dumnezeu a arătat prin cuvântul Său și mărturia Duhului Său. \textbf{El ne cheamă să ținem cu tărie, cu strânsoarea credinței, de \underline{principiile fundamentale} care sunt bazate pe autoritate de necontestat}.}[SpTB02 59.1; 1904][https://egwwritings.org/read?panels=p417.299]


Here we see how Ellen White described the message of the \emcap{Fundamental Principles} as the messages of the three angels’, from the fourteenth chapter of Revelation, and as a message concerning the ministration of Christ in the heavenly sanctuary. The first point of the \emcap{Fundamental Principles}, which is widely discussed here, answers the important question given by the first angel in the fourteenth chapter of Revelation: \textit{who is the God we ought to worship}?


Aici vedem cum Ellen White a descris solia \emcap{Principiilor Fundamentale} ca soliile celor trei îngeri, din capitolul paisprezece din Apocalipsa, și ca o solie privind slujirea lui Hristos în sanctuarul ceresc. Primul punct al \emcap{Principiilor Fundamentale}, care este discutat pe larg aici, răspunde la întrebarea importantă dată de primul înger din capitolul paisprezece din Apocalipsa: \textit{cine este Dumnezeul pe care trebuie să-L adorăm}?


\bible{Fear \textbf{God}, and \textbf{give glory \underline{to him}}; for \textbf{the hour of \underline{his} judgment is come}: and \textbf{worship \underline{him}} that made heaven, and earth, and the sea, and the fountains of waters.}[Revelation 14:7]


\bible{Temeți-vă de \textbf{Dumnezeu} și \textbf{dați-\underline{I} slavă}, căci \textbf{a venit ceasul judecății \underline{Lui}}, și \textbf{închinați-vă Celui} ce a făcut cerul și pământul, marea și izvoarele apelor!}[Apocalipsa 14:7]


Who is the God we ought to worship, declared by the first angel? In the spectrum of time we find different answers to this question. Today the answer is the Triune God, or Trinity God, as presented in the Fundamental Beliefs of Seventh-day Adventists. But, we raise the question: who was the God that the Adventist pioneers worshipped? The first angel’s message is tied to prophetic time, which was fulfilled in the times of our pioneers. The entire purpose behind their labor was the proclamation of the three angels’ messages. In 1844, the hour of God’s judgment had come. If the Trinity God was the God whose hour had come, and our pioneers did not worship the Trinity, wouldn't they have failed in their purpose of creating this movement?


Cine este Dumnezeul pe care trebuie să-L adorăm, declarat de primul înger? În spectrul timpului găsim diferite răspunsuri la această întrebare. Astăzi răspunsul este Dumnezeul Trinitar, sau Dumnezeul Trinității, așa cum este prezentat în Punctele Fundamentale de Credință ale Adventiștilor de Ziua a Șaptea. Dar, ridicăm întrebarea: cine era Dumnezeul pe care L-au adorat pionierii adventiști? Solia primului înger este legată de timpul profetic, care s-a împlinit în vremea pionierilor noștri. Întregul scop din spatele lucrării lor a fost proclamarea soliilor celor trei îngeri. În 1844, venise ceasul judecății lui Dumnezeu. Dacă Dumnezeul Trinității era Dumnezeul al cărui ceas venise, iar pionierii noștri nu au adorat Trinitatea, nu ar fi eșuat ei în scopul lor de a crea această mișcare?


Let us examine the history of our prophetic movement with this question: did our pioneers worship the true God in proclaiming the message of the first angel? We read the explanation of the events in the passing of 1844.


Să examinăm istoria mișcării noastre profetice cu această întrebare: au adorat pionierii noștri pe adevăratul Dumnezeu în proclamarea soliei primului înger? Citim explicația evenimentelor din trecerea anului 1844.


\egw{\textbf{Like the first disciples, William Miller and his associates did not, themselves, fully comprehend the import of the message which they bore}. Errors that had been long established in the church prevented them from arriving at a correct interpretation of an important point in the prophecy. Therefore, though they proclaimed the message which God had committed to them to be given to the world, yet through a misapprehension of its meaning they suffered disappointment.}[GC 351.2; 1888][https://egwwritings.org/read?panels=p132.1604]


\egw{\textbf{Ca și primii ucenici, William Miller și asociații săi nu au înțeles, ei înșiși, pe deplin importanța soliei pe care o purtau}. Erorile care fuseseră stabilite de mult timp în biserică i-au împiedicat să ajungă la o interpretare corectă a unui punct important din profeție. Prin urmare, deși au proclamat solia pe care Dumnezeu le-o încredințase să fie dată lumii, totuși, printr-o înțelegere greșită a sensului ei, au suferit dezamăgire.}[GC 351.2; 1888][https://egwwritings.org/read?panels=p132.1604]


\egwnogap{In explaining Daniel 8:14, ‘Unto \textbf{two thousand and three hundred days; then shall \underline{the sanctuary be cleansed}},’ Miller, as has been stated, adopted the generally received view that the earth is the sanctuary, and he believed that the cleansing of the sanctuary represented the purification of the earth by fire at the coming of the Lord. When, therefore, he found that the close of the 2300 days was definitely foretold, he concluded that this revealed the time of the second advent. His error resulted from accepting the popular view as to what constitutes the sanctuary.}[GC 352.1; 1888][https://egwwritings.org/read?panels=p132.1607]


\egwnogap{În explicarea textului din Daniel 8:14, ‘Până vor trece \textbf{două mii trei sute de zile; apoi \underline{sfântul Locaș va fi curățit}},’ Miller, așa cum s-a afirmat, a adoptat viziunea general acceptată că pământul este sanctuarul, și el credea că curățirea sanctuarului reprezenta purificarea pământului prin foc la venirea Domnului. Când, prin urmare, a descoperit că sfârșitul celor 2300 de zile era prezis în mod clar, a concluzionat că aceasta dezvăluia timpul celei de-a doua veniri. Eroarea sa a rezultat din acceptarea viziunii populare cu privire la ceea ce constituie sanctuarul.}[GC 352.1; 1888][https://egwwritings.org/read?panels=p132.1607]


\egwnogap{In the typical system, which was a shadow of the sacrifice and \textbf{priesthood of Christ}, \textbf{the cleansing of the sanctuary was the last service performed by the high priest }in the yearly round of ministration.\textbf{ It was the closing work of the atonement—a removal or putting away of sin from Israel}. \textbf{It prefigured the closing work in the ministration of our High Priest in heaven, in the removal or blotting out of the sins of His people, which are registered in the heavenly records}. \textbf{This service involves a work of \underline{investigation, a work of judgment}; and it immediately precedes the coming of Christ} in the clouds of heaven with power and great glory; for when He comes, every case has been decided. Says Jesus: ‘My reward is with Me, to give every man according as his work shall be.’ Revelation 22:12. \textbf{It is this work of judgment, immediately preceding the second advent, that is \underline{announced in the first angel’s message of Revelation 14:7}: ‘Fear \underline{God}, and give glory to Him; \underline{for the hour of His judgment is come}.}’}[GC 352.2; 1888][https://egwwritings.org/read?panels=p132.1608]


\egwnogap{În sistemul tipic, care era o umbră a jertfei și \textbf{preoției lui Hristos}, \textbf{curățirea sanctuarului era ultima slujbă îndeplinită de marele preot} în ciclul anual de slujire.\textbf{ Era lucrarea de încheiere a ispășirii—o îndepărtare sau înlăturare a păcatului din Israel}. \textbf{Ea prefigura lucrarea de încheiere în slujirea Marelui nostru Preot în cer, în îndepărtarea sau ștergerea păcatelor poporului Său, care sunt înregistrate în rapoartele cerești}. \textbf{Această slujbă implică o lucrare de \underline{cercetare, o lucrare de judecată}; și ea precede imediat venirea lui Hristos} pe norii cerului cu putere și mare slavă; căci atunci când El vine, fiecare caz a fost decis. Isus spune: „Răsplata Mea este cu Mine, ca să dau fiecăruia după cum îi este lucrarea.” Apocalipsa 22:12. \textbf{Această lucrare de judecată, care precede imediat a doua venire, este \underline{anunțată în solia primului înger din Apocalipsa 14:7}: „Temeți-vă de \underline{Dumnezeu} și dați-I slavă, \underline{căci a venit ceasul judecății Lui}.}’}[GC 352.2; 1888][https://egwwritings.org/read?panels=p132.1608]


\egwnogap{\textbf{Those who proclaimed this warning gave the right message at the right time}. But as the early disciples declared, ‘The time is fulfilled, and the kingdom of God is at hand,’ based on the prophecy of Daniel 9, while they failed to perceive that the death of the Messiah was foretold in the same scripture, \textbf{so Miller and his associates preached the message based on \underline{Daniel 8:14 and Revelation 14:7}, and failed to see that there were still other messages brought to view in Revelation 14}, which were also to be given before the advent of the Lord. As the disciples were mistaken in regard to the kingdom to be set up at the end of the seventy weeks, so Adventists were mistaken in regard to the event to take place at the expiration of the 2300 days. In both cases there was an acceptance of, or rather an adherence to, popular errors that blinded the mind to the truth. Both classes fulfilled the will of God in delivering the message which He desired to be given, and both, through their own misapprehension of their message, suffered disappointment.}[GC 352.3; 1888][https://egwwritings.org/read?panels=p132.1609]


\egwnogap{\textbf{Cei care au proclamat această avertizare au dat solia potrivită la timpul potrivit}. Dar așa cum primii ucenici au declarat: „S-a împlinit vremea, și Împărăția lui Dumnezeu este aproape”, bazându-se pe profeția din Daniel 9, în timp ce nu au reușit să înțeleagă că moartea lui Mesia era prezisă în aceeași scriptură, \textbf{tot așa Miller și asociații săi au predicat solia bazată pe \underline{Daniel 8:14 și Apocalipsa 14:7}, și nu au reușit să vadă că erau și alte solii prezentate în Apocalipsa 14}, care trebuiau de asemenea să fie date înainte de venirea Domnului. Așa cum ucenicii s-au înșelat cu privire la împărăția care urma să fie întemeiată la sfârșitul celor șaptezeci de săptămâni, tot așa adventiștii s-au înșelat cu privire la evenimentul care urma să aibă loc la expirarea celor 2300 de zile. În ambele cazuri a existat o acceptare a, sau mai degrabă o aderare la, erori populare care au orbit mintea față de adevăr. Ambele clase au împlinit voia lui Dumnezeu în transmiterea soliei pe care El a dorit să fie dată, și ambele, prin propria lor neînțelegere a soliei lor, au suferit dezamăgire.}[GC 352.3; 1888][https://egwwritings.org/read?panels=p132.1609]


\begin{figure}[hp]
    \centering
    \includegraphics[width=1\linewidth]{images/william-miller.jpg}
    \caption*{William Miller (1782-1849)}
    \label{fig:w-miller}
\end{figure}


\begin{figure}[hp]
    \centering
    \includegraphics[width=1\linewidth]{images/william-miller.jpg}
    \caption*{William Miller (1782-1849)}
    \label{fig:w-miller}
\end{figure}


In reading the explanation of the great disappointment, did you see the answer to the question, “\textit{who is God whose judgment has come}?” The first angel’s message from Revelation 14:7 aligns exactly with the prophetic time declared in Daniel 8:14. The judgment that has come was the investigative judgment, which started in 1844. The Bible clearly describes whose hour of judgment has come in the first angel’s message. Let us read it in the Bible and see Ellen White’s comment.


Citind explicația marii dezamăgiri, ați văzut răspunsul la întrebarea „\textit{cine este Dumnezeu al cărui ceas al judecății a venit}?” Solia primului înger din Apocalipsa 14:7 se aliniază exact cu timpul profetic declarat în Daniel 8:14. Judecata care a venit a fost judecata de cercetare, care a început în 1844. Biblia descrie clar al cui ceas al judecății a venit în solia primului înger. Să citim în Biblie și să vedem comentariul lui Ellen White.


\egw{‘I beheld,’ says the prophet Daniel, \textbf{‘till thrones were placed, and One that was \underline{Ancient of Days} \underline{did sit}}: \textbf{His raiment} was white as snow, and \textbf{the hair of His head} like pure wool; \textbf{His throne was fiery flames}, and the wheels thereof burning fire. A fiery stream issued and came forth from before Him: thousand thousands ministered unto Him, and ten thousand times ten thousand stood before Him: \textbf{\underline{the judgment was set, and the books were opened}}.’ Daniel 7:9, 10, R.V.}[GC 479.1; 1888][https://egwwritings.org/read?panels=p132.2169]


\egw{„Am privit”, spune profetul Daniel, \textbf{„până când s-au așezat scaune de domnie și \underline{Cel Îmbătrânit de zile} \underline{a șezut jos}}: \textbf{Haina Lui} era albă ca zăpada și \textbf{părul capului Său} ca lâna curată; \textbf{scaunul Său de domnie era ca niște flăcări de foc}, și roțile lui ca un foc aprins. Un râu de foc curgea și ieșea dinaintea Lui. Mii de mii de slujitori Îi slujeau, și de zece mii de ori zece mii stăteau înaintea Lui: \textbf{\underline{judecata s-a ținut și cărțile s-au deschis}}.” Daniel 7:9, 10, R.V.}[GC 479.1; 1888][https://egwwritings.org/read?panels=p132.2169]


\egwnogap{\textbf{Thus was presented to the prophet’s vision the great and solemn day when the characters and the lives of men should pass in review before the Judge of all the earth, and to every man should be rendered ‘according to his works.’ \underline{The Ancient of Days is God the Father}.} Says the psalmist: \textbf{‘Before }the mountains were brought forth, or ever Thou hadst formed the earth and the world, even \textbf{from everlasting to everlasting}, \textbf{Thou art God}.’ Psalm 90:2. \textbf{\underline{It is He, the source of all being, and the fountain of all law, that is to preside in the judgment}}. And holy angels as ministers and witnesses, in number ‘ten thousand times ten thousand, and thousands of thousands,’ attend this great tribunal.}[GC 479.2; 1888][https://egwwritings.org/read?panels=p132.2170]


\egwnogap{\textbf{Astfel a fost prezentată viziunii profetului ziua cea mare și solemnă când caracterele și viețile oamenilor vor trece în revistă înaintea Judecătorului întregului pământ, și fiecărui om i se va răsplăti „după faptele sale”. \underline{Cel Îmbătrânit de zile este Dumnezeu Tatăl}.} Spune psalmistul: \textbf{„Mai înainte} ca să se fi născut munții și să fi făcut Tu pământul și lumea, \textbf{din veșnicie în veșnicie}, \textbf{Tu ești Dumnezeu}.” Psalmul 90:2. \textbf{\underline{El este Cel care, sursa întregii ființe și izvorul întregii legi, urmează să prezideze la judecată}}. Și îngerii sfinți ca slujitori și martori, în număr de „zece mii de ori zece mii și mii de mii”, participă la acest mare tribunal.}[GC 479.2; 1888][https://egwwritings.org/read?panels=p132.2170]


\egwnogap{\textbf{‘And, behold, one like \underline{the Son of man} came with the clouds of heaven, and came to \underline{the Ancient of Days}, and they \underline{brought Him near before Him}}. And there was given Him dominion, and glory, and a kingdom, that all people, nations, and languages, should serve Him: His dominion is an everlasting dominion, which shall not pass away.’ Daniel 7:13, 14. \textbf{The coming of Christ here described is not His second coming to the earth}. \textbf{\underline{He comes to the Ancient of Days in heaven} to receive dominion and glory and a kingdom}, \textbf{which will be given Him at the close of His work as a mediator}. \textbf{\underline{It is this coming, and not His second advent to the earth, that was foretold in prophecy to take place at the termination of the 2300 days in 1844}}. \textbf{Attended by heavenly angels, our great High Priest enters the holy of holies and there appears in \underline{the presence of God}} to engage in the last acts of His ministration in behalf of man—\textbf{to perform the work of investigative judgment} and to \textbf{make an atonement} for all who are shown to be entitled to its benefits.}[GC 479.3; 1888][https://egwwritings.org/read?panels=p132.2171]


\egwnogap{\textbf{„Și, iată, venea pe norii cerului unul ca un \underline{Fiu al omului} și a înaintat spre \underline{Cel Îmbătrânit de zile}, și \underline{L-au adus aproape înaintea Lui}}. I s-a dat stăpânire, slavă și o împărăție, ca să-I slujească toate popoarele, neamurile și limbile: stăpânirea Lui este o stăpânire veșnică, care nu va trece.” Daniel 7:13, 14. \textbf{Venirea lui Hristos descrisă aici nu este a doua Sa venire pe pământ}. \textbf{\underline{El vine la Cel Îmbătrânit de zile în cer} pentru a primi stăpânire și slavă și o împărăție}, \textbf{care Îi va fi dată la încheierea lucrării Sale ca mijlocitor}. \textbf{\underline{Această venire, și nu a doua Sa venire pe pământ, a fost cea prezisă în profeție să aibă loc la terminarea celor 2300 de zile în 1844}}. \textbf{Însoțit de îngeri cerești, Marele nostru Preot intră în Sfânta Sfintelor și acolo apare în \underline{prezența lui Dumnezeu}} pentru a se angaja în ultimele acte ale slujirii Sale în favoarea omului—\textbf{pentru a îndeplini lucrarea de judecată de cercetare} și pentru a \textbf{face o ispășire} pentru toți cei care se dovedesc a fi îndreptățiți să primească beneficiile ei.}[GC 479.3; 1888][https://egwwritings.org/read?panels=p132.2171]


The answer is simple and straightforward: The God of our pioneers was the Ancient of Days. \egwinline{The Ancient of Days is God the Father}. He is \textit{a personal}, \textit{spiritual being}. We see this in His description: \bible{Whose garment was white as snow, and the hair of his head like the pure wool: his throne was like the fiery flame, and his wheels as burning fire.}[Daniel 7:9]. In the termination of the 2300 days prophecy, in 1844, \bible{The hour of His judgment has come}[Revelation 14:7], \bible{the Ancient of days did sit} and \bible{the judgment was set, and the books were opened.}[Daniel 7:9,10]. The God from the first angel’s message is the Ancient of Days. Our pioneers were not ignorant regarding the truth about God. They believed \others{That there is \textbf{one God}, \textbf{\underline{a personal, spiritual being}}, \textbf{the creator of all things}, omnipotent, omniscient, and eternal, infinite in wisdom, holiness, justice, goodness, truth, and mercy; unchangeable, and \textbf{\underline{everywhere present by his representative, the Holy Spirit}}. Ps. 139:7.}[First point of the Fundamental Principles.] This one God is the Father, the Ancient of Days, \others{the creator of all things}, and we are to \bible{worship Him that made heaven, and earth, and the sea, and the fountains of waters}[Revelation 14:7]. He \bible{created all things by Jesus Christ}[Ephesians 3:9].


Răspunsul este simplu și direct: Dumnezeul pionierilor noștri era Cel Îmbătrânit de zile. \egwinline{Cel Îmbătrânit de zile este Dumnezeu Tatăl}. El este \textit{o ființă personală}, \textit{spirituală}. Vedem aceasta în descrierea Sa: \bible{Haina Lui era albă ca zăpada, și părul capului Său ca lâna curată; scaunul Lui de domnie era ca niște flăcări de foc, și roțile lui ca un foc aprins.}[Daniel 7:9]. La terminarea profeției celor 2300 de zile, în 1844, \bible{A venit ceasul judecății Lui}[Apocalipsa 14:7], \bible{Cel Îmbătrânit de zile a șezut jos} și \bible{judecata s-a ținut și cărțile s-au deschis.}[Daniel 7:9,10]. Dumnezeul din solia primului înger este Cel Îmbătrânit de zile. Pionierii noștri nu erau ignoranți cu privire la adevărul despre Dumnezeu. Ei credeau \others{Că există \textbf{un singur Dumnezeu}, \textbf{\underline{o ființă personală, spirituală}}, \textbf{creatorul tuturor lucrurilor}, atotputernic, atotștiutor și etern, infinit în înțelepciune, sfințenie, dreptate, bunătate, adevăr și milă; neschimbător și \textbf{\underline{prezent pretutindeni prin reprezentantul Său, Duhul Sfânt}}. Ps. 139:7.}[Primul punct al Principiilor Fundamentale.] Acest Dumnezeu unic este Tatăl, Cel Îmbătrânit de zile, \others{creatorul tuturor lucrurilor}, și noi trebuie să ne \bible{închinăm Celui ce a făcut cerul și pământul, marea și izvoarele apelor}[Apocalipsa 14:7]. El a \bible{creat toate lucrurile prin Isus Hristos}[Efeseni 3:9].


Today, the first angel’s message has not lost any of its importance. The messages of the second and third angel’s depend on the first message and only the first message requires action on our part. We are to worship God. More specifically, we are to worship the right God. In the last and final conflict, there will be two kinds of worshippers, as we have been told in Revelation 13 and 14.


Astăzi, solia primului înger nu și-a pierdut nimic din importanța sa. Soliile celui de-al doilea și al treilea înger depind de prima solie și doar prima solie necesită acțiune din partea noastră. Trebuie să ne închinăm lui Dumnezeu. Mai precis, trebuie să ne închinăm Dumnezeului adevărat. În conflictul ultim și final, vor fi două tipuri de închinători, așa cum ni s-a spus în Apocalipsa 13 și 14.


\bible{And all that dwell upon the earth shall \textbf{worship him} \normaltext{[the beast]}, \textbf{whose names are not written in the book of life of the Lamb} slain from the foundation of the world.}[Revelation 13:8]


\bible{Și toți locuitorii pământului i se vor \textbf{închina} \normaltext{[fiarei]}, toți aceia al căror \textbf{nume n-a fost scris de la întemeierea lumii în cartea vieții Mielului}, care a fost junghiat.}[Apocalipsa 13:8]


The group that worships the beast will receive the mark of the beast. The whole world will be compelled to worship the beast and his image with the threat of death.


Grupul care se închină fiarei va primi semnul fiarei. Întreaga lume va fi constrânsă să se închine fiarei și chipului ei sub amenințarea cu moartea.


\bible{And he \normaltext{[the beast]} had power to give life unto \textbf{the image of the beast}, that the image of the beast should both speak, and cause that \textbf{as many as would not worship the image of the beast should be killed}.}[Revelation 13:15]


\bible{Și i s-a dat \normaltext{[fiarei]} putere să dea suflare \textbf{chipului fiarei}, ca chipul fiarei să vorbească și să facă \textbf{să fie omorâți toți cei ce nu se vor închina chipului fiarei}.}[Apocalipsa 13:15]


We should not participate in this worship. Let us learn and have faith just like Daniel’s three friends who refused to worship the image of King Nebuchadnezzar. The beast represented in Revelation 13, that extorts the consciences of men by the peril of their lives, is the papacy. Dear friend, don't be fooled. The papal God is a Trinity God. Do not overlook that.


Nu ar trebui să participăm la această închinare. Să învățăm și să avem credință la fel ca cei trei prieteni ai lui Daniel care au refuzat să se închine chipului împăratului Nebucadnețar. Fiara reprezentată în Apocalipsa 13, care constrânge conștiințele oamenilor prin pericolul vieții lor, este papalitatea. Dragă prietene, nu te lăsa înșelat. Dumnezeul papal este un Dumnezeu Trinitate. Nu trece cu vederea acest lucru.


We should worship the Ancient of Days as it is proclaimed in the first angel’s message. This is God the Creator who created everything through His Son, Jesus Christ. This is God from the first point of the \emcap{Fundamental Principles}. Our pioneers got this right.


Ar trebui să ne închinăm Celui Îmbătrânit de zile așa cum este proclamat în solia primului înger. Acesta este Dumnezeu Creatorul care a creat totul prin Fiul Său, Isus Hristos. Acesta este Dumnezeu din primul punct al \emcap{Principiilor Fundamentale}. Pionierii noștri au înțeles corect acest lucru.


True understanding of the mission and purpose of the Seventh-day Adventist movement should be conclusive evidence that the Trinity doctrine is a foreign doctrine to us. We’ve ended up where we are today because we have forgotten \egwinline{\textbf{the way the Lord has led us, and \underline{His teaching} in our past history.}}[LS 196.2; 1915][https://egwwritings.org/read?panels=p41.1083] It is very sad to see how our Adventist scholars claim that our pioneers did not correctly understand the doctrine of God. If that were true, our pioneers would have failed to proclaim the first angel's message. They did not fail. We have failed.


Înțelegerea adevărată a misiunii și scopului mișcării Adventiste de Ziua a Șaptea ar trebui să fie o dovadă concludentă că doctrina Trinității este o doctrină străină pentru noi. Am ajuns unde suntem astăzi pentru că am uitat \egwinline{\textbf{calea pe care ne-a condus Domnul și \underline{învățătura Sa} din istoria noastră trecută.}}[LS 196.2; 1915][https://egwwritings.org/read?panels=p41.1083] Este foarte trist să vedem cum învățații noștri adventiști susțin că pionierii noștri nu au înțeles corect doctrina despre Dumnezeu. Dacă ar fi adevărat, pionierii noștri ar fi eșuat să proclame solia primului înger. Ei nu au eșuat. Noi am eșuat.


\others{\textbf{Most of the founders of Seventh-day Adventism would not be able to join the church today if they had to subscribe to the denomination's Fundamental Beliefs}.}\others{\textbf{More specifically, most would not be able to agree to belief number 2, which deals with the doctrine of the Trinity.} For Joseph Bates the Trinity was an unscriptural doctrine, for James White it was that “old Trinitarian absurdity,” and for M. E. Cornell it was a fruit of the great apostasy, along with such false doctrines as Sunday-keeping and the immortality of the soul.}[George Night, Ministry Magazine, October 1993][https://www.ministrymagazine.org/archive/1993/10/adventists-and-change]


\others{\textbf{Majoritatea fondatorilor Bisericii Adventiste de Ziua a Șaptea nu ar putea să se alăture bisericii astăzi dacă ar trebui să subscrie la Punctele Fundamentale de Credință ale denominațiunii}.}\others{\textbf{Mai precis, majoritatea nu ar putea fi de acord cu punctul de credință numărul 2, care tratează doctrina Trinității.} Pentru Joseph Bates, Trinitatea era o doctrină nescripturală, pentru James White era acea „veche absurditate trinitariană”, iar pentru M. E. Cornell era un rod al marii apostazii, alături de doctrine false precum păzirea duminicii și nemurirea sufletului.}[George Night, Ministry Magazine, octombrie 1993][https://www.ministrymagazine.org/archive/1993/10/adventists-and-change]


The doctrine of Trinity is the doctrine that undermines the foundation of our faith, the foundation that was laid at the beginning of our work. The distinction between truth and error lies in hermeneutics—the method of interpreting the Bible. Let us thoroughly investigate this issue.


Doctrina Trinității este doctrina care subminează temelia credinței noastre, temelia care a fost pusă la începutul lucrării noastre. Distincția dintre adevăr și eroare constă în hermeneutică—metoda de interpretare a Bibliei. Să investigăm temeinic această problemă.


% Remembering the beginning

\begin{titledpoem}
    
    \stanza{
        In faith’s first light, they sought His face, \\
        Through earnest prayer, they felt His grace. \\
        The pioneers, with vision clear, \\
        In 1840’s, held God dear.
    }

    \stanza{    
        "The judgment hour has come," they cried, \\
        To tell the world, both far and wide. \\
        The Ancient of Days, they did proclaim, \\
        Not a trinity, but a singular name.
    }

    \stanza{    
        Ellen White, with her pen in hand, \\
        Spoke of the heav’nly glorious land. \\
        A sanctuary to be cleansed, \\
        In love with Jesus, our best Friend.
    }

    \stanza{    
        First angel called us to revere, \\
        Our God the Father, we should fear. \\
        "Who is the God we should adore?" \\
        Not trinity—they did implore.
    }

    \stanza{    
        The Trinity, was unembraced, \\
        By pioneers, God’s word they traced. \\
        Father, the Ancient, they declare, \\
        His judgment right, beyond compare.
    }

    \stanza{
        Yet, whispers now, through time have spread, \\
        Trinity’s shadow, causing dread. \\
        If this was God they must declare, \\
        Their mission failed, caught in despair.
    }

    \stanza{
        But this is falsehood, error bold, \\
        A new belief, but wrongly told. \\
        The God once worshipped, with great zeal, \\
        Was the true God, their mission real.
    }

    \stanza{
        In unity, we seek His face, \\
        Embrace His truth, with fervent grace. \\
        The pio’neers’ vision, do not lose, \\
        For in their footsteps, we must choose.
    }

    \stanza{
        Worship the God, of days of old, \\
        Ancient of Days, as was foretold. \\
        Third angels’ message, clear and bright, \\
        Guiding us still, through darkest night.
    }
    
\end{titledpoem}