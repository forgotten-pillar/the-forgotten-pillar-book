% \qrchapter{https://forgottenpillar.com/rsc/en-fp-chapter7}{The authority of the Fundamental Principles} \label{chap:authority}


\qrchapter{https://forgottenpillar.com/rsc/ro-fp-chapter7}{Autoritatea Principiilor Fundamentale} \label{chap:authority}


In the 10th chapter of the Special Testimonies, we read how God established the foundation of our faith. Sister White used several different expressions for the foundation of our faith. Her references included: “\textit{a platform of eternal truth}, \textit{“pillars of our faith”}, \textit{“principles of truth”}, \textit{“principal points”}, \textit{“waymarks”}, and “\textit{foundation principles}—all of these refer to the \emcap{Fundamental Principles}. At the end of the chapter, she affirmed the will of God that \egwinline{He calls upon us to hold firmly, with the grip of faith, to \textbf{the fundamental principles} that are \textbf{based upon unquestionable \underline{authority}}.}[SpTB02 59.1; 1904][https://egwwritings.org/read?panels=p417.299]


În capitolul 10 din Mărturii Speciale, citim cum Dumnezeu a stabilit temelia credinței noastre. Sora White a folosit mai multe expresii diferite pentru temelia credinței noastre. Referințele ei au inclus: “\textit{o platformă de adevăr veșnic}, \textit{“stâlpii credinței noastre”}, \textit{“principii ale adevărului”}, \textit{“puncte principale”}, \textit{“pietre de hotar”}, și “\textit{principii de bază}—toate acestea se referă la \emcap{Principiile Fundamentale}. La sfârșitul capitolului, ea a afirmat voia lui Dumnezeu că \egwinline{El ne cheamă să ținem cu tărie, cu strânsoarea credinței, de \textbf{principiile fundamentale} care sunt \textbf{bazate pe \underline{autoritate} de necontestat}.}[SpTB02 59.1; 1904][https://egwwritings.org/read?panels=p417.299]


The authority on which the \emcap{fundamental principles} are established is unquestionable. They were the result of deep, earnest study in the time of great disappointment, when \egwinline{\textbf{\underline{point by point}, has been sought out by prayerful study, and testified to by the \underline{miracle-working power of the Lord}}}\footnote{Ibid.}. \egwinline{\textbf{Thus \underline{the leading points of our faith} as we hold them today were firmly established}. \textbf{\underline{Point after point} was clearly defined, and all the brethren came into harmony}.}[Lt253-1903.4; 1903][https://egwwritings.org/read?panels=p14068.9980010]


Autoritatea pe care sunt stabilite \emcap{principiile fundamentale} este de necontestat. Ele au fost rezultatul unui studiu profund și serios în timpul marii dezamăgiri, când \egwinline{\textbf{\underline{punct cu punct}, a fost căutat prin studiu plin de rugăciune și mărturisit prin \underline{puterea făcătoare de minuni a Domnului}}}\footnote{Ibid.}. \egwinline{\textbf{Astfel \underline{punctele principale ale credinței noastre} așa cum le ținem astăzi au fost ferm stabilite}. \textbf{\underline{Punct după punct} a fost clar definit, și toți frații au ajuns în armonie}.}[Lt253-1903.4; 1903][https://egwwritings.org/read?panels=p14068.9980010]


They were the result of the earnest Bible studies of our pioneers, after the passing of time in 1844. As the Seventh-day Adventist movement progressed, there came a need for instituting the organization, which was realized in 1863. In 1872, the Seventh-day Adventist Church issued the document called “\textit{A Declaration of the Fundamental Principles, Taught and Practiced by the Seventh-Day Adventists}. This was the first written document declaring the \emcap{fundamental principles} as public statements of the Seventh-day Adventist faith. This document was the public synopsis of Seventh-day Adventist faith and it declared \others{what is, and has been, with great unanimity, held by} the Seventh-day Adventist people. It was written \others{to meet inquiries} as to what was believed by Seventh-day Adventists, \others{to correct false statements circulated} and to \others{remove erroneous impressions}[FP1872 3.1; 1872][https://egwwritings.org/read?panels=p928.8].


Ele au fost rezultatul studiilor biblice serioase ale pionierilor noștri, după trecerea timpului în 1844. Pe măsură ce mișcarea adventistă de ziua a șaptea a progresat, a apărut nevoia de a institui organizația, care s-a realizat în 1863. În 1872, Biserica Adventistă de Ziua a Șaptea a emis documentul numit “\textit{O declarație a principiilor fundamentale, crezute și practicate de adventiștii de ziua a șaptea}. Acesta a fost primul document scris care declara \emcap{principiile fundamentale} ca declarații publice ale credinței adventiste de ziua a șaptea. Acest document a fost rezumatul public al credinței adventiste de ziua a șaptea și declara \others{ceea ce este și a fost, cu mare unanimitate, susținut de} poporul adventist de ziua a șaptea. A fost scris \others{pentru a răspunde întrebărilor} despre ceea ce credeau adventiștii de ziua a șaptea, \others{pentru a corecta declarațiile false răspândite} și pentru a \others{îndepărta impresiile eronate}[FP1872 3.1; 1872][https://egwwritings.org/read?panels=p928.8].


Today it is still debated who authored the synopsis because originally, in 1872, it was left anonymous. In 1874, James White issued it in Signs of the Times\footnote{\href{https://adventistdigitallibrary.org/adl-364148/signs-times-june-4-1874}{Signs of the Times, June 4, 1874}} and Uriah Smith in the Review and Herald\footnote{\href{http://documents.adventistarchives.org/Periodicals/RH/RH18741124-V44-22.pdf}{The Advent Review and Herald of the Sabbath, November 24, 1874}}—both signing with their own signatures. In 1889, Uriah Smith revised it by adding three points; it was issued in the Adventist Yearbook with his signature on it. Uriah Smith died in 1903 and all successive printings of the \emcap{Fundamental Principles} were printed under his name. They were printed in the Yearbooks—each year from 1905 until 1914\footnote{For more detailed timeline of Fundamental Principles, see \hyperref[appendix:timeline]{Appendix: Fundamental Principles - Timeline}}. Sister White died in 1915 and, for the next 17 years, the \emcap{fundamental principles} were not printed. Their next appearance was in the 1931 Yearbook when they received significant changes.


Astăzi încă se dezbate cine a fost autorul rezumatului deoarece inițial, în 1872, a fost lăsat anonim. În 1874, James White l-a publicat în Signs of the Times\footnote{\href{https://adventistdigitallibrary.org/adl-364148/signs-times-june-4-1874}{Signs of the Times, June 4, 1874}} și Uriah Smith în Review and Herald\footnote{\href{http://documents.adventistarchives.org/Periodicals/RH/RH18741124-V44-22.pdf}{The Advent Review and Herald of the Sabbath, November 24, 1874}}—ambii semnând cu propriile lor semnături. În 1889, Uriah Smith l-a revizuit adăugând trei puncte; a fost publicat în Anuarul Adventist cu semnătura sa pe el. Uriah Smith a murit în 1903 și toate tipăriturile succesive ale \emcap{Principiilor Fundamentale} au fost tipărite sub numele său. Au fost tipărite în Anuare—în fiecare an din 1905 până în 1914\footnote{Pentru o cronologie mai detaliată a Principiilor Fundamentale, vezi \hyperref[appendix:timeline]{Anexa: Principiile Fundamentale - Cronologie}}. Sora White a murit în 1915 și, pentru următorii 17 ani, \emcap{principiile fundamentale} nu au fost tipărite. Următoarea lor apariție a fost în Anuarul din 1931 când au primit schimbări semnificative.


In 1971, LeRoy Froom wrote about a statement from 1872: \others{Though appearing anonymously, it was actually composed by Smith}[Edwin Froom, LeRoy. Movement of Destiny. 1971., p. 160]. Unfortunately, he didn’t provide any data to support his claim. It is unfortunate to see how pro-trinitarian scholars consider the \emcap{Fundamental Principles} to be of very little importance. Their true value is starkly diminished by attributing these beliefs to those of a small group of people, mostly to James White’s or Uriah Smith’s personal belief, rather than belief which was \others{with great unanimity, held by}[Preface of the Fundamental Principles 1872] the Seventh-day Adventist people. In 1958, Ministry Magazine described the \emcap{Fundamental Principles} as follows:


În 1971, LeRoy Froom a scris despre o declarație din 1872: \others{Deși a apărut anonim, a fost de fapt compus de Smith}[Edwin Froom, LeRoy. Movement of Destiny. 1971., p. 160]. Din păcate, el nu a furnizat nicio dovadă pentru a-și susține afirmația. Este regretabil să vedem cum învățații pro-trinitarieni consideră \emcap{Principiile Fundamentale} ca fiind de foarte mică importanță. Valoarea lor adevărată este drastic diminuată prin atribuirea acestor credințe unui grup mic de oameni, mai ales credinței personale a lui James White sau Uriah Smith, mai degrabă decât credinței care era \others{cu mare unanimitate, susținută de}[Prefața Principiilor Fundamentale 1872] poporul adventist de ziua a șaptea. În 1958, Ministry Magazine a descris \emcap{Principiile Fundamentale} astfel:


\others{It is true that in 1872 a ‘Declaration of the Fundamental Principles Taught and Practiced by Seventhday Adventists’ was printed, \textbf{but it was never adopted by the denomination and therefore cannot be considered official}. Evidently a small group, \textbf{perhaps even one or two, endeavored to put into words what they thought were the views of the entire church…}}[Ministry Magazine “\textit{Our Declaration of Fundamental Beliefs}”, January 1958, Roy Anderson, J. Arthur Buckwalter, Louise Kleuser, Earl Cleveland and Walter Schubert]


\others{Este adevărat că în 1872 o ‘Declarație a Principiilor Fundamentale Crezute și Practicate de Adventiștii de Ziua a Șaptea’ a fost tipărită, \textbf{dar nu a fost niciodată adoptată de denominațiune și prin urmare nu poate fi considerată oficială}. Evident un grup mic, \textbf{poate chiar unul sau doi, s-au străduit să pună în cuvinte ceea ce ei credeau că erau vederile întregii biserici…}}[Ministry Magazine “\textit{Our Declaration of Fundamental Beliefs}”, Ianuarie 1958, Roy Anderson, J. Arthur Buckwalter, Louise Kleuser, Earl Cleveland și Walter Schubert]


Problematically, there is no evidence to support the claim that the \emcap{Fundamental Principles} were not the representation of faith of the whole body. We certainly know that Sister White endorsed them and, from her influence alone, we know that these beliefs were indeed accepted by the denomination—this is in addition to the fact that they were printed multiple times over the course of 42 years, during the life of Ellen White.


În mod problematic, nu există nicio dovadă care să susțină afirmația că \emcap{Principiile Fundamentale} nu erau reprezentarea credinței întregului corp. Cu siguranță știm că sora White le-a aprobat și, doar din influența ei, știm că aceste credințe au fost într-adevăr acceptate de denominațiune—aceasta pe lângă faptul că au fost tipărite de mai multe ori pe parcursul a 42 de ani, în timpul vieții lui Ellen White.


But there should be no controversy over the authorship of the \emcap{Fundamental Principles}. We have a quotation from Sister White about who authored them. When speaking of Uriah Smith, Sister White wrote:


Dar nu ar trebui să existe nicio controversă asupra autorului \emcap{Principiilor Fundamentale}. Avem un citat de la sora White despre cine le-a scris. Când vorbea despre Uriah Smith, sora White a scris:


\egw{\textbf{Brother Smith was with us in the rise of this work. He understands how \underline{we—my husband and myself}—have carried the work forward and upward step by step and have borne the hardships, the poverty, and the want of means. With us were those early workers. Elder Smith, especially, was one with my husband in his early manhood}. …}[Ms54-1890.6; 1890][https://egwwritings.org/read?panels=p7213.15]


\egw{\textbf{Fratele Smith a fost cu noi la începutul acestei lucrări. El înțelege cum \underline{noi—soțul meu și eu însămi}—am dus lucrarea înainte și în sus pas cu pas și am purtat greutățile, sărăcia și lipsa de mijloace. Cu noi au fost acei lucrători timpurii. Bătrânul Smith, în special, a fost una cu soțul meu în tinerețea sa timpurie}. …}[Ms54-1890.6; 1890][https://egwwritings.org/read?panels=p7213.15]


\egwnogap{\textbf{\underline{We have stood shoulder to shoulder with Elder Smith in this work while the Lord was laying the foundation principles}}. \textbf{We had to work constantly against one-idea men}, who thought correct business relations in regard to the work which had to be done were an evidence of worldly-mindedness, and the cranky ones who would present themselves as capable of bearing responsibilities, but could not be trusted to be connected with the work lest they swing it in wrong lines. \textbf{Step after step has had to be taken, \underline{not after the wisdom of men} but after the wisdom and instruction of One who is too wise to err and too good to do us harm}. \textbf{There have been so many elements that would have to be proved and tried. I thank the Lord that Elders Smith, Amadon, and Batchellor still live. They composed the members of our family in the most trying parts of our history}.}[Ms54-1890.7; 1890][https://egwwritings.org/read?panels=p7213.16]


\egwnogap{\textbf{\underline{Am stat umăr la umăr cu Bătrânul Smith în această lucrare în timp ce Domnul punea principiile fundamentale}}. \textbf{A trebuit să lucrăm constant împotriva oamenilor cu o singură idee}, care credeau că relațiile corecte de afaceri în privința lucrării care trebuia făcută erau o dovadă de lumesc, și împotriva celor ciudați care se prezentau ca fiind capabili să poarte responsabilități, dar nu li se putea încredința să fie conectați cu lucrarea ca să nu o îndrepte pe linii greșite. \textbf{Pas după pas a trebuit să fie făcut, \underline{nu după înțelepciunea oamenilor} ci după înțelepciunea și instrucțiunea Celui care este prea înțelept să greșească și prea bun să ne facă rău}. \textbf{Au fost atât de multe elemente care trebuiau să fie dovedite și încercate. Îi mulțumesc Domnului că Bătrânii Smith, Amadon și Batchellor încă trăiesc. Ei au compus membrii familiei noastre în cele mai încercătoare părți ale istoriei noastre}.}[Ms54-1890.7; 1890][https://egwwritings.org/read?panels=p7213.16]


According to this quotation, who laid down the foundation principles?


Conform acestui citat, cine a pus principiile fundamentale?


\egwinline{\textbf{\underline{We have stood shoulder to shoulder with Elder Smith in this work while the Lord was laying the foundation principles}}.} \textbf{It was the Lord}! But who wrote them down as a declaration of our faith? It was Elder Smith with James White and Sister White; we see that where Sister White says\egwinline{\textbf{we} have stood shoulder to shoulder with Elder Smith}. This \textit{‘we’} is explained in the previous paragraph: \egwinline{He \normaltext{[Elder Smith]} understands how\textbf{ we—my husband and myself}—have carried the work forward}. With this quotation, Sister White was clearly involved when the Lord was laying the \emcap{Fundamental Principles}.


\egwinline{\textbf{\underline{Am stat umăr la umăr cu Bătrânul Smith în această lucrare în timp ce Domnul punea principiile fundamentale}}.} \textbf{A fost Domnul}! Dar cine le-a scris ca o declarație a credinței noastre? A fost Bătrânul Smith cu James White și Sora White; vedem aceasta unde Sora White spune\egwinline{\textbf{noi} am stat umăr la umăr cu Bătrânul Smith}. Acest \textit{‘noi’} este explicat în paragraful anterior: \egwinline{El \normaltext{[Bătrânul Smith]} înțelege cum\textbf{ noi—soțul meu și eu însămi}—am dus lucrarea înainte}. Cu acest citat, Sora White a fost clar implicată când Domnul punea \emcap{Principiile Fundamentale}.


It is true that the Declaration of the \emcap{Fundamental Principles} was written by a small group of people, namely Elder Smith, James White and Ellen White, but they endeavored to put into words what was the true view of the entire church body. They accurately represented the \emcap{fundamental principles}—the truths received in the beginning of our work. If that were not the case, then this declaration is the very opposite of what it claims to be. They were written \others{to meet inquiries} as to what was believed by Seventh-day Adventists, \others{to correct false statements circulated} and to \others{remove erroneous impressions.}[FP1872 3.1; 1872][https://egwwritings.org/read?panels=p928.8] If this document misrepresented the Adventist position, why was its continual reprinting, over the course of 42 years, permitted? It was reprinted until the death of Ellen White. If this document misrepresented the church’s position, wouldn’t Ellen White have raised her voice against it? She always raised her voice against the misrepresentation of the Seventh-day Adventist position, as she did with D. M. Canright and Dr. Kellogg. If the \emcap{Fundamental Principles} were misrepresenting the Seventh-day Adventist position, then all subsequent reprinting should be attributed to a conspiracy theory. That would be the greatest conspiracy theory within the Seventh-day Adventist Church. Ever. The harmony between the writings of Ellen White, Adventist pioneers, and the claims made in the Declaration of the \emcap{Fundamental Principles}, testify of the fact that this declaration is an accurate \others{summary of the principal features of} Seventh-day Adventist \others{faith, upon which there is, so far as we know, entire unanimity throughout the body}[The preface of the Fundamental Principles 1889].


Este adevărat că Declarația \emcap{Principiilor Fundamentale} a fost scrisă de un grup mic de oameni, și anume Bătrânul Smith, James White și Ellen White, dar ei s-au străduit să pună în cuvinte care era adevărata perspectivă a întregului corp al bisericii. Ei au reprezentat cu acuratețe \emcap{principiile fundamentale}—adevărurile primite la începutul lucrării noastre. Dacă nu ar fi fost cazul, atunci această declarație este exact opusul a ceea ce pretinde să fie. Au fost scrise \others{pentru a răspunde întrebărilor} despre ceea ce credeau adventiștii de ziua a șaptea, \others{pentru a corecta declarațiile false răspândite} și pentru a \others{îndepărta impresiile eronate.}[FP1872 3.1; 1872][https://egwwritings.org/read?panels=p928.8] Dacă acest document a reprezentat greșit poziția adventistă, de ce a fost permisă retipărirea sa continuă, pe parcursul a 42 de ani? A fost retipărit până la moartea lui Ellen White. Dacă acest document a reprezentat greșit poziția bisericii, nu și-ar fi ridicat Ellen White vocea împotriva lui? Ea și-a ridicat întotdeauna vocea împotriva reprezentării greșite a poziției adventiste de ziua a șaptea, așa cum a făcut cu D. M. Canright și Dr. Kellogg. Dacă \emcap{Principiile Fundamentale} reprezentau greșit poziția adventistă de ziua a șaptea, atunci toată retipărirea ulterioară ar trebui atribuită unei teorii a conspirației. Aceasta ar fi cea mai mare teorie a conspirației din Biserica Adventistă de Ziua a Șaptea. Vreodată. Armonia dintre scrierile lui Ellen White, pionierii adventiști și afirmațiile făcute în Declarația \emcap{Principiilor Fundamentale}, mărturisesc despre faptul că această declarație este un \others{rezumat acurat al caracteristicilor principale ale} credinței adventiste de ziua a șaptea, \others{asupra căreia există, din câte știm, unanimitate deplină în întregul corp}[Prefața Principiilor Fundamentale 1889].


With the death of Sister White in 1915, printing of the \emcap{Fundamental Principles} ceased. From 1915 onward, the Yearbook did not print any statement of belief until 1931. At this time, the \emcap{Fundamental Principles} received substantial changes. For the first time, the Trinity was introduced to the \emcap{fundamental principles}. In points’ 2 and 3 we read:


Odată cu moartea Sorei White în 1915, tipărirea \emcap{Principiilor Fundamentale} a încetat. Din 1915 înainte, Anuarul nu a tipărit nicio declarație de credință până în 1931. În acest moment, \emcap{Principiile Fundamentale} au primit schimbări substanțiale. Pentru prima dată, Trinitatea a fost introdusă în \emcap{principiile fundamentale}. În punctele 2 și 3 citim:


\others{2. \textbf{That the Godhead, or Trinity, consists of the Eternal Father, a \underline{personal, spiritual Being}}, omnipotent, \textbf{\underline{omnipresent}}, omniscient, infinite in wisdom and love; \textbf{the Lord Jesus Christ, the Son of the Eternal Father}, \textbf{through whom all things were created} and through whom the salvation of the redeemed hosts will be accomplished; \textbf{the Holy Spirit, the third person of the Godhead}, the great regenerating power in the work of redemption. Matt. 28:19}


\others{2. \textbf{Că Dumnezeirea, sau Trinitatea, constă din Tatăl Etern, o \underline{Ființă personală, spirituală}}, atotputernic, \textbf{\underline{omniprezent}}, atotștiutor, infinit în înțelepciune și dragoste; \textbf{Domnul Isus Hristos, Fiul Tatălui Etern}, \textbf{prin care toate lucrurile au fost create} și prin care mântuirea oștirilor răscumpărate va fi împlinită; \textbf{Duhul Sfânt, a treia persoană a Dumnezeirii}, marea putere regeneratoare în lucrarea de răscumpărare. Mat. 28:19}


\others{3. \textbf{That Jesus Christ is very God, being of the same nature and essence as the Eternal Father}…}[Yearbook of the Seventh-day Adventist Denomination, 1931, page. 377][https://static1.squarespace.com/static/554c4998e4b04e89ea0c4073/t/59d17eec12abd9c6194cd26d/1506901758727/SDA-YB1931-22+\%28P.+377-380\%29.pdf]


\others{3. \textbf{Că Isus Hristos este însuși Dumnezeu, fiind de aceeași natură și esență ca Tatăl Etern}…}[Yearbook of the Seventh-day Adventist Denomination, 1931, page. 377][https://static1.squarespace.com/static/554c4998e4b04e89ea0c4073/t/59d17eec12abd9c6194cd26d/1506901758727/SDA-YB1931-22+\%28P.+377-380\%29.pdf]


This change, in favor of the Trinity, appeared sixteen years after the death of Sister White. A comparison of this statement with the original \emcap{Fundamental Principles} presents several striking differences. The Father is still a personal, spiritual Being, the creator of all things, but is not addressed as “\textit{one God}” any longer. Jesus Christ is still the Son of the Eternal Father, through whom the Father created all things; Jesus is, also, of the very same nature and essence of the Father. Although these were the same terms to describe the doctrine on the \emcap{personality of God} in the original \emcap{Fundamental Principles}, we ask about the meaning of the term “\textit{personal, spiritual being}” applied to the Father, if He is, by new statement, omnipresent by Himself? The Holy Spirit is not an instrument, or means of the Father’s omnipresence anymore. Although this statement uses similar rhetoric of the original \emcap{Fundamental Principles}, it steps away from the original doctrine on the presence and the \emcap{personality of God}.


Această schimbare, în favoarea Trinității, a apărut la șaisprezece ani după moartea Sorei White. O comparație a acestei declarații cu \emcap{Principiile Fundamentale} originale prezintă câteva diferențe izbitoare. Tatăl este încă o Ființă personală, spirituală, creatorul tuturor lucrurilor, dar nu mai este adresat ca “\textit{un singur Dumnezeu}”. Isus Hristos este încă Fiul Tatălui Etern, prin care Tatăl a creat toate lucrurile; Isus este, de asemenea, de aceeași natură și esență cu Tatăl. Deși aceștia erau aceiași termeni pentru a descrie doctrina despre \emcap{personalitatea lui Dumnezeu} în \emcap{Principiile Fundamentale} originale, ne întrebăm despre sensul termenului “\textit{ființă personală, spirituală}” aplicat Tatălui, dacă El este, prin noua declarație, omniprezent prin Sine Însuși? Duhul Sfânt nu mai este un instrument, sau mijloc al omniprezenței Tatălui. Deși această declarație folosește o retorică similară cu \emcap{Principiile Fundamentale} originale, se îndepărtează de doctrina originală despre prezența și \emcap{personalitatea lui Dumnezeu}.


According to LeRoy Froom, this statement was written entirely by Francis Wilcox, with the approval of three other brothers (C.H. Watson, M.E. Kern and E.R. Palmer).\footnote{Edwin Froom, LeRoy. Movement of Destiny. 1971., p. 411, 413, 414} In the unpublished paper of \textit{The Seventh-day Adventist Church in Mission: 1919-1979}, we read how Elder Wilcox made this statement contrary to the belief of the church body and published it without their approval.


Conform lui LeRoy Froom, această declarație a fost scrisă în întregime de Francis Wilcox, cu aprobarea altor trei frați (C.H. Watson, M.E. Kern și E.R. Palmer).\footnote{Edwin Froom, LeRoy. Movement of Destiny. 1971., p. 411, 413, 414} În lucrarea nepublicată \textit{The Seventh-day Adventist Church in Mission: 1919-1979}, citim cum Bătrânul Wilcox a făcut această declarație contrar credinței corpului bisericii și a publicat-o fără aprobarea lor.


\others{\textbf{Realizing that the General Conference Committee or any other church body would never accept the document in the form in which it was written}, Elder Wilcox, with full knowledge of the group \normaltext{[C.H. Watson, M.E. Kern and E.R. Palmer]}, handed the Statement directly to Edson Rogers, the General Conference statistician, who published it in the 1931 edition of the Yearbook, where it has appeared ever since. It was without the official approval of the General Conference Committee, therefore, and without any formal denominational adoption, that Elder Wilcox's statement became the accepted declaration of our faith.}[Dwyer, Bonnie. “A New Statement of Fundamental Beliefs (1980) - Spectrum Magazine.” \textit{Spectrum Magazine}, 7 June 2009, \href{https://spectrummagazine.org/news/new-statement-fundamental-beliefs-1980/}{spectrummagazine.org/news/new-statement-fundamental-beliefs-1980/}. Accessed 30 Jan. 2025.]


\others{\textbf{Realizând că Comitetul Conferinței Generale sau orice alt corp bisericesc nu ar accepta niciodată documentul în forma în care a fost scris}, Bătrânul Wilcox, cu deplina cunoștință a grupului \normaltext{[C.H. Watson, M.E. Kern și E.R. Palmer]}, a înmânat Declarația direct lui Edson Rogers, statisticianul Conferinței Generale, care a publicat-o în ediția din 1931 a Anuarului, unde a apărut de atunci. A fost fără aprobarea oficială a Comitetului Conferinței Generale, prin urmare, și fără nicio adoptare denominațională formală, că declarația Bătrânului Wilcox a devenit declarația acceptată a credinței noastre.}[Dwyer, Bonnie. “A New Statement of Fundamental Beliefs (1980) - Spectrum Magazine.” \textit{Spectrum Magazine}, 7 June 2009, \href{https://spectrummagazine.org/news/new-statement-fundamental-beliefs-1980/}{spectrummagazine.org/news/new-statement-fundamental-beliefs-1980/}. Accessed 30 Jan. 2025.]


In 1980, the final change to the public synopsis of the Seventh-day Adventist faith was made. The General Conference voted to adopt today’s official statement:


În 1980, a fost făcută ultima schimbare la rezumatul public al credinței Adventiștilor de Ziua a Șaptea. Conferința Generală a votat să adopte declarația oficială de astăzi:


\others{\textbf{There is one God: Father, Son and Holy Spirit, a unity of three coeternal Persons}. God is immortal, all-powerful, all-knowing, above all, and \textbf{ever present}. He is infinite and beyond human comprehension, yet known through His self-revelation. He is forever worthy of worship, adoration, and service by the whole creation.}[Seventh-day Adventists Believe: A Biblical Exposition of 27 Fundamental Doctrines, p. 16]


\others{\textbf{Există un singur Dumnezeu: Tatăl, Fiul și Duhul Sfânt, o unitate a trei Persoane coeterne}. Dumnezeu este nemuritor, atotputernic, atotștiutor, mai presus de toate și \textbf{mereu prezent}. El este infinit și dincolo de înțelegerea umană, totuși cunoscut prin auto-revelația Sa. El este pentru totdeauna vrednic de închinare, adorare și slujire din partea întregii creații.}[Adventiștii de Ziua a Șaptea cred: O expunere biblică a 27 de doctrine fundamentale, p. 16]


In this brief historical overview we see that the 1931 statement is a “middle step” between the original Adventist belief to the full trinitarian belief.


În această scurtă prezentare istorică vedem că declarația din 1931 este un „pas intermediar” între credința adventistă originală și credința trinitariană deplină.


The change in our beliefs has occurred over time with many discussions. Our Adventist history has left a trace of these changes. If we are honest truth seekers we should study this matter in detail. Can we see, in our Adventist history, why we have left the first point of the \emcap{Fundamental Principles} in favor of the Trinity doctrine? Most certainly! In the following studies we will look at some of the historical documents that show why we have moved from the first point of the \emcap{Fundamental Principles}, held in the early years, to accept the Trinity doctrine. During these studies, we bid you to prayerfully evaluate the changes with your own beliefs.


Schimbarea în credințele noastre a avut loc de-a lungul timpului cu multe discuții. Istoria noastră adventistă a lăsat o urmă a acestor schimbări. Dacă suntem căutători sinceri ai adevărului, ar trebui să studiem această chestiune în detaliu. Putem vedea, în istoria noastră adventistă, de ce am părăsit primul punct al \emcap{Principiilor Fundamentale} în favoarea doctrinei Trinității? Cu siguranță! În studiile următoare vom analiza unele dintre documentele istorice care arată de ce ne-am îndepărtat de la primul punct al \emcap{Principiilor Fundamentale}, susținut în primii ani, pentru a accepta doctrina Trinității. În timpul acestor studii, vă îndemnăm să evaluați cu rugăciune schimbările în raport cu propriile voastre credințe.


% The authority of the Fundamental Principles

\begin{titledpoem}

    \stanza{
        Our principles stand firm and true, \\
        Established by God’s chosen few. \\
        A platform built on sturdy might, \\
        As guiding waymarks in the night.
    }

    \stanza{
        The truth was sought with earnest prayer, \\
        Point after point laid down with care. \\
        Yet modern minds the truth exchanged, \\
        For pleasing myths the doctrines changed.
    }

    \stanza{
        Return, O church, to truths ordained, \\
        Not to beliefs that men have claimed. \\
        Stand firm! God’s truth cannot be moved, \\
        Those Fundamental’s God approved.
    }

    \stanza{
        Let not new scholars lead astray, \\
        From paths our founders led the way. \\
        The Lord laid down these truths of old, \\
        Embrace these truths with courage bold.
    }
    
\end{titledpoem}