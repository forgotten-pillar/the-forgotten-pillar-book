% \qrchapter{https://forgottenpillar.com/rsc/en-fp-chapter25}{Setting up the wrong Fundamental Principles}


\qrchapter{https://forgottenpillar.com/rsc/ro-fp-chapter25}{Stabilirea unor principii fundamentale greșite}


You might ask yourself: how could it be possible that we, as a church, have gone astray from the light God gave us in the beginning? The answer to this question is the same answer to the question why the Jews went astray from the light God gave them concerning His Son. Please, take a look at the driving force behind the church in Apostolic times and our time.


V-ați putea întreba: cum este posibil ca noi, ca biserică, să ne fi abătut de la lumina pe care Dumnezeu ne-a dat-o la început? Răspunsul la această întrebare este același răspuns la întrebarea de ce evreii s-au abătut de la lumina pe care Dumnezeu le-a dat-o cu privire la Fiul Său. Vă rog, priviți la forța motrice din spatele bisericii în timpurile apostolice și în timpul nostru.


\egw{‘The angel of the Lord by night opened the prison doors, and brought them forth, and said, Go, stand and speak in the temple to the people all the words of this life.’ [Acts 5:19, 20.] We see here that the men in authority are not always obeyed, even though they may profess to be teachers of Bible doctrines. \textbf{There are many today who feel indignant and aggrieved that any voice should be raised presenting ideas that differ from their own in regard to points of religious belief}. \textbf{Have they not long advocated their ideas as truth?} So the priests and rabbis reasoned in apostolic days. What mean these men who are unlearned, some of them mere fishermen, who are presenting ideas contrary to the doctrines which the learned priests and rulers are teaching the people? \textbf{They have no right to meddle with the fundamental principles of our faith}.}[Lt38-1896.23; 1896][https://egwwritings.org/read?panels=p5631.29]


\egw{„Îngerul Domnului a deschis ușile temniței noaptea, i-a scos afară și le-a zis: «Duceți-vă, stați în Templu și vestiți norodului toate cuvintele vieții acesteia!»“ [Faptele Apostolilor 5:19, 20.] Vedem aici că oamenii aflați în autoritate nu sunt întotdeauna ascultați, chiar dacă ei pot pretinde că sunt învățători ai doctrinelor biblice. \textbf{Există mulți astăzi care se simt indignați și nedreptățiți că vreo voce ar trebui să fie ridicată prezentând idei care diferă de ale lor în privința punctelor de credință religioasă}. \textbf{Nu au susținut ei de mult timp ideile lor ca adevăr?} Așa au raționat preoții și rabinii în zilele apostolice. Ce înseamnă acești oameni care sunt neînvățați, unii dintre ei simpli pescari, care prezintă idei contrare doctrinelor pe care preoții învățați și conducătorii le învață poporul? \textbf{Ei nu au niciun drept să se amestece cu principiile fundamentale ale credinței noastre}.}[Lt38-1896.23; 1896][https://egwwritings.org/read?panels=p5631.29]


\egwnogap{“\textbf{But we see that the God of heaven sometimes commissions men to \underline{teach that which is regarded as contrary to the established doctrines}. Because those who were once the depositaries of truth \underline{became unfaithful to their sacred trust}, the Lord chose others who would receive the bright beams of the Sun of Righteousness, and would advocate truths that were not in accordance with the ideas of the religious leaders. And then these leaders, in the blindness of their minds, give full sway to what is supposed to be righteous indignation against the ones who have set aside cherished fables. They act like men that have lost their reason. They do not consider the possibility that they themselves have not rightly understood the Word. They will not open their eyes to discern the fact that they have misinterpreted and misapplied the Scriptures, and have built up false theories, \underline{calling them fundamental doctrines of the faith}}.“}[Lt38-1896.24; 1896][https://egwwritings.org/read?panels=p5631.30]


\egwnogap{„\textbf{Dar vedem că Dumnezeul cerului însărcinează uneori oameni să \underline{învețe ceea ce este considerat contrar doctrinelor stabilite}. Pentru că cei care au fost odată depozitarii adevărului \underline{au devenit necredincioși încrederii lor sacre}, Domnul a ales pe alții care să primească razele strălucitoare ale Soarelui Neprihănirii și să susțină adevăruri care nu erau în conformitate cu ideile liderilor religioși. Și atunci acești lideri, în orbirea minților lor, dau frâu liber la ceea ce se presupune a fi indignare dreaptă împotriva celor care au pus deoparte fabulele prețuite. Ei acționează ca oameni care și-au pierdut rațiunea. Ei nu iau în considerare posibilitatea că ei înșiși să nu fi înțeles corect Cuvântul. Ei nu vor deschide ochii să discearnă faptul că au interpretat greșit și au aplicat greșit Scripturile și au construit teorii false, \underline{numindu-le doctrine fundamentale ale credinței}}.”}[Lt38-1896.24; 1896][https://egwwritings.org/read?panels=p5631.30]


\egwnogap{\textbf{But the Holy Spirit will from time to time reveal the truth through its own chosen agencies; and no man, not even a priest or ruler, has a right to say, You shall not give publicity to your opinions, because I do not believe them. That wonderful ‘I’ may attempt to put down the Holy Spirit’s teaching. Men may, for a time, attempt to smother it and kill it; but that will not make error truth or truth error. The inventive minds of men have advanced speculative opinions in various lines, and when the Holy Spirit lets light shine into human minds, it does not respect every point of man’s application of the word. God impressed his servants to speak the truth irrespective of what men had taken for granted as truth}.}[Lt38-1896.25; 1896][https://egwwritings.org/read?panels=p5631.31]


\egwnogap{\textbf{Dar Duhul Sfânt va descoperi din timp în timp adevărul prin propriile Sale agenții alese; și niciun om, nici măcar un preot sau conducător, nu are dreptul să spună: Nu veți da publicitate opiniilor voastre, pentru că eu nu le cred. Acel minunat „eu” poate încerca să înăbușe învățătura Duhului Sfânt. Oamenii pot, pentru un timp, să încerce să o sufoce și să o omoare; dar aceasta nu va face eroarea adevăr sau adevărul eroare. Mințile inventive ale oamenilor au avansat opinii speculative în diverse direcții, și când Duhul Sfânt lasă lumina să strălucească în mințile umane, aceasta nu respectă fiecare punct al aplicării cuvântului de către om. Dumnezeu i-a impresionat pe slujitorii Săi să vorbească adevărul indiferent de ceea ce oamenii au luat de bun ca adevăr}.}[Lt38-1896.25; 1896][https://egwwritings.org/read?panels=p5631.31]


\egwnogap{\textbf{\underline{Even Seventh-day Adventists are in danger of closing their eyes to truth as it is in Jesus}, because it contradicts something which they have taken for granted as truth, but which the Holy Spirit teaches is not truth. Let all be very modest, and seek most earnestly to put self out of the question, and to exalt Jesus.} \textbf{In most of the religious controversies, the foundation of the trouble is that self is striving for the supremacy}. About what? About matters which are not vital points at all, and which are regarded as such only because men have given importance to them. See Matthew 12:31-37; Mark 14:56; Luke 5:21; Matthew 9:3.}[Lt38-1896.26; 1896][https://egwwritings.org/read?panels=p5631.32]


\egwnogap{\textbf{\underline{Chiar și adventiștii de ziua a șaptea sunt în pericol de a-și închide ochii la adevărul așa cum este el în Isus}, pentru că contrazice ceva ce ei au luat de bun ca adevăr, dar pe care Duhul Sfânt îl învață că nu este adevăr. Toți să fie foarte modești și să caute cu cea mai mare seriozitate să scoată eul din discuție și să-L înalțe pe Isus.} \textbf{În majoritatea controverselor religioase, temelia problemei este că eul se luptă pentru supremație}. Despre ce? Despre chestiuni care nu sunt deloc puncte vitale și care sunt considerate ca atare doar pentru că oamenii le-au dat importanță. Vezi Matei 12:31-37; Marcu 14:56; Luca 5:21; Matei 9:3.}[Lt38-1896.26; 1896][https://egwwritings.org/read?panels=p5631.32]


The proud state of the heart resists the will of God and is the driving force behind apostasy; the humble heart is obedient to the will of God and is the driving force behind true reformation. The following quotations express future, concrete prophecies where the fanciful ideas of God will be brought in and \egwinline{many things of like character will in the future arise}[Ms137-1903.10; 1903][https://egwwritings.org/read?panels=p9939.17]. These ideas are of like character to the ideas contained in the Living Temple. They will do away with the \emcap{personality of God}. Ellen White gives warning after warning to adhere to the \emcap{Fundamental Principles}, and to be aware of the leaders who will tear down the old foundation.


Starea mândră a inimii rezistă voinței lui Dumnezeu și este forța motrice din spatele apostaziei; inima smerită este ascultătoare față de voința lui Dumnezeu și este forța motrice din spatele adevăratei reformațiuni. Următoarele citate exprimă profeții concrete despre viitor unde ideile fanteziste despre Dumnezeu vor fi aduse și \egwinline{multe lucruri de caracter asemănător vor apărea în viitor}[Ms137-1903.10; 1903][https://egwwritings.org/read?panels=p9939.17]. Aceste idei sunt de caracter asemănător cu ideile conținute în Templul viu. Ele vor desființa \emcap{personalitatea lui Dumnezeu}. Ellen White dă avertizare după avertizare să aderăm la \emcap{principiile fundamentale} și să fim conștienți de liderii care vor dărâma vechea temelie.


\egw{In view of these Scriptures, who will dare to interpret God and place in the minds of others the sentiments regarding Him that are contained in Living Temple? \textbf{These theories are the theories of the great deceiver, and in the lives of \underline{those who receive them there will be sad chapters}}. \textbf{This is Satan’s device \underline{to unsettle the foundation of our faith}, to shake our confidence in the Lord’s guidance and in the experience that He has given us. \underline{Many things of like character will in the future arise}}. I entreat our medical missionary workers to be afraid to trust the suppositions and devising of any human being who entertains the thought that \textbf{the path over which the people of God have been led for the last fifty years is a wrong path}. \textbf{\underline{Beware of those who}, not having had any decided experience in the leading of the Lord’s Spirit, \underline{would suppose that this leading is all a fallacy}; that we have not the truth}; that we are not the people of the Lord, gathered by Him from all countries and nations. \textbf{\underline{Beware of those who would tear down the foundation, upon which we have been building for the last fifty years, to establish a new doctrine}}. \textbf{I know that these new theories are from the enemy}.}[Ms137-1903.10; 1903][https://egwwritings.org/read?panels=p9939.17]


\egw{În lumina acestor Scripturi, cine va îndrăzni să-L interpreteze pe Dumnezeu și să plaseze în mințile altora opiniile cu privire la El care sunt conținute în Templul viu? \textbf{Aceste teorii sunt teoriile marelui înșelător, și în viețile \underline{celor care le primesc vor fi capitole triste}}. \textbf{Aceasta este strategia lui Satan \underline{să zdruncine temelia credinței noastre}, să ne zguduie încrederea în călăuzirea Domnului și în experiența pe care El ne-a dat-o. \underline{Multe lucruri de caracter asemănător vor apărea în viitor}}. Îi implor pe lucrătorii noștri misionari medicali să se teamă să aibă încredere în presupunerile și născocirile oricărei ființe umane care întreține gândul că \textbf{calea pe care poporul lui Dumnezeu a fost condus în ultimii cincizeci de ani este o cale greșită}. \textbf{\underline{Feriți-vă de cei care}, neavând nicio experiență decisivă în conducerea Duhului Domnului, \underline{ar presupune că această conducere este toată o iluzie}; că nu avem adevărul}; că nu suntem poporul Domnului, adunat de El din toate țările și națiunile. \textbf{\underline{Feriți-vă de cei care ar dărâma temelia pe care am construit în ultimii cincizeci de ani, pentru a stabili o nouă doctrină}}. \textbf{Știu că aceste teorii noi sunt de la vrăjmaș}.}[Ms137-1903.10; 1903][https://egwwritings.org/read?panels=p9939.17]


\egwnogap{\textbf{Let those who would \underline{bring in} fanciful ideas of God awake to a sense of their danger. This is too solemn a subject to be trifled with}.}[Ms137-1903.11; 1903][https://egwwritings.org/read?panels=p9939.18]


\egwnogap{\textbf{Să se trezească la simțul pericolului lor cei care ar \underline{aduce} idei fanteziste despre Dumnezeu. Acesta este un subiect prea solemn pentru a fi tratat cu ușurință}.}[Ms137-1903.11; 1903][https://egwwritings.org/read?panels=p9939.18]


\egwnogap{The root of idolatry is an evil heart of unbelief in departing from the living God. It is because men have not faith in the presence and power of God \textbf{that they have been putting their trust in their own wisdom}. They have been devising and planning to exalt themselves and find salvation in their own works. \textbf{\underline{A deceptive influence from satanic agencies is coming in}, because leaders whom the Lord has warned and entreated and counseled are choosing their own wisdom in the place of the wisdom of God}. To such ones the warning comes, ‘Talk no more exceedingly proudly; let not arrogancy come out of your mouth; for the Lord is a God of knowledge, and by Him actions are weighed.’}[Ms137-1903.12; 1903][https://egwwritings.org/read?panels=p9939.19]


\egwnogap{Rădăcina idolatriei este o inimă rea a necredinței în îndepărtarea de Dumnezeul cel viu. Pentru că oamenii nu au credință în prezența și puterea lui Dumnezeu, \textbf{ei și-au pus încrederea în propria lor înțelepciune}. Ei au conceput și au planificat să se înalțe pe ei înșiși și să găsească mântuirea în propriile lor lucrări. \textbf{\underline{O influență înșelătoare din partea agențiilor satanice vine}, pentru că liderii pe care Domnul i-a avertizat și i-a implorat și i-a sfătuit aleg propria lor înțelepciune în locul înțelepciunii lui Dumnezeu}. Unor astfel de persoane le vine avertizarea: „Nu mai vorbiți cu atâta mândrie; să nu vă mai iasă din gură cuvinte trufașe! Căci Domnul este un Dumnezeu care știe totul și de El sunt cântărite faptele.”}[Ms137-1903.12; 1903][https://egwwritings.org/read?panels=p9939.19]


The difference between the old \emcap{Fundamental Principles} and the new Fundamental Beliefs is in our \egwinline{ideas of God.} The Trinitarian idea of God was not part of the foundation of our faith, which Sister White defended. How did this change take place? It was done through the leaders who chose \egwinline{their own wisdom in the place of the wisdom of God.} We should \egwinline{Beware of those who would tear down the foundation, upon which we have been building for the last fifty years, to establish a new doctrine.} In this observation, we recognize that this new Trinitarian idea of God was \egwinline{a deceptive influence from satanic agency} that came into our ranks.


Diferența dintre vechile \emcap{Principii Fundamentale} și noile Puncte Fundamentale de Credință constă în \egwinline{ideile noastre despre Dumnezeu.} Ideea trinitariană despre Dumnezeu nu a făcut parte din temelia credinței noastre, pe care a apărat-o sora White. Cum a avut loc această schimbare? S-a făcut prin conducătorii care au ales \egwinline{propria lor înțelepciune în locul înțelepciunii lui Dumnezeu.} Ar trebui să ne \egwinline{ferim de cei care ar dărâma temelia pe care am construit în ultimii cincizeci de ani, pentru a stabili o nouă doctrină.} În această observație, recunoaștem că această nouă idee trinitariană despre Dumnezeu a fost \egwinline{o influență înșelătoare din partea agenției satanice} care a pătruns în rândurile noastre.


% Setting up the wrong Fundamental Principles

\begin{titledpoem}
    
    \stanza{
        Sadly, within our own church walls \\
        From our own pulpits, error falls \\
        Members want smooth words for their ears \\
        Don’t step on toes, Allay their fears.
    }

    \stanza{
        Pastors and elders do preside \\
        While sins remain, untouched inside \\
        Laodicean comfort zone \\
        But they will reap what they have sown.
    }

    \stanza{
        Ask for the old paths, walk therein \\
        From the old truth, don’t move a pin. \\
        They spurned the truth which brightly shone, \\
        The Spirit’s pow’r, to them unknown.
    }

    \stanza{
        Do not the humble hearts perceive \\
        Whispers of truth they should believe? \\
        Meanwhile the stories ease concern. \\
        What God would tell them, they won’t learn.
    }

    \stanza{
        Beware of error, thinly veiled, \\
        God’s Word is true, but men have failed. \\
        Beware of shadows leaders cast. \\
        To the foundations true, hold fast,
    }

    \stanza{
        Let not man’s wisdom lead astray, \\
        Let God’s own Spirit show the way. \\
        For in the Scripture’s glowing light, \\
        We find the path of safety bright.
    }

    \stanza{
        Let us, in faith, each day commence, \\
        God’s Word our shield, not man’s pretense. \\
        For truth in Christ alone is found, \\
        And on this rock, our faith is sound.
    }
    
\end{titledpoem}