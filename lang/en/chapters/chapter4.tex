\chapter{Revision of “Living Temple”}

In \textit{Testimonies for the Church Containing Letters to Physicians and Ministers Instruction to Seventh-Day Adventists}, the tenth chapter, \textit{The Foundation of our Faith,} God gave valuable lessons on the development and consequences of Kellogg’s theories. The broader and deeper meaning of these quotations can be understood when we are familiar with their historical context. Let us first take a brief look at the historical context of Kellogg's book, The \textit{Living Temple}.

In a series of providence, God signified that "\textit{Living Temple}" should not be printed. One such event was the burning of Battle Creek’s press building, just the night before it was to be printed. Finally, the book was printed elsewhere; it instigated a great crisis in the Seventh-day Adventist Church. On October 7, 1903, a yearly meeting of the conference was held in Washington DC. Many Seventh-day Adventist church leaders were present, including Dr. Kellogg and his sympathizers. Major controversy was taking place over this book and the conflict was inevitable. Fortunately, on the brink of this escalating conflict, a letter from Sister White was delivered to the council. On Sunday, the letter fell upon the ears of all, to which there resounded many “amen's” and “halleluyah's”. It was a very tense and moving morning for the church that was on the verge of a split—to at last have concrete direction from the Lord's messenger:

\egw{I have some things to say to our teachers in reference to \textbf{the new book The Living Temple}. \textbf{Be careful how you sustain \underline{the sentiments of this book regarding the personality of God}}. As the Lord presents matters to me, \textbf{these sentiments do not bear the endorsement of God}. \textbf{They are a snare that the enemy has prepared for these last days}. I thought that this would surely be discerned and that it would not be necessary for me to say anything about it. \textbf{But since the claim has been made that the teachings of this book can be sustained by statements from my writings, I am compelled to speak in denial of this claim}. There may be in this book expressions and sentiments that are in harmony with my writings. And there may be in my writings many statements which when taken from their connection, and interpreted according to the mind of the writer of Living Temple, would seem to be in harmony with the teachings of this book. \textbf{This may give apparent support to the assertion that the sentiments in Living Temple are in harmony with my writings}. \textbf{But God forbid that this opinion should prevail}.}[Lt211-1903.1; 1903][https://egwwritings.org/?ref=en\_Lt211-1903.1]

Repeatedly, Sister White stated that the true problem of the book was the sentiments\egwinline{\textbf{regarding the personality of God}}. These sentiments are not sustained by statements from Ellen White’s writings and these very sentiments\egwinline{\textbf{are a snare that the enemy has prepared for these last days}}.

God, again in His providence, solved this conflict. Kellogg accepted the reproof from the Lord’s messenger and, before the council closed, he stated that the Living Temple would be taken from the market\footnote{\href{https://forgottenpillar.com/wp-content/uploads/2022/04/Letter-A-G-Daniells-to-W-C-White-October-29-1903.pdf}{Letter: A. G. Daniells to W. C. White, October 23, 1903, pp. 5}}. But after the conference, he spoke privately with the general conference president, Brother Arthur G. Daniells, about his plans for revising the book. The following is a look at select letters, revealing Kellogg’s plans for revising “\textit{Living Temple}”.

Ellen White was not present at the yearly conference in Washington DC but her son, William C. White, did attend. When the conference was over, brother Arthur G. Daniells wrote a confidential letter to William C. White regarding Dr. Kellogg’s plan to revise his book:

\others{October 29, 1903}

\othersnogap{Ever since the \textbf{council closed} I have felt that I should write you \textbf{confidentially regarding Dr. Kellogg’s plans for revising and republishing ‘The Living Temple’}…. He \normaltext{[Kellogg]} said that some days before coming to the council, he had been thinking the matter over, and began to see that \textbf{he had made a slight mistake in expressing his views}. He said that all the way along he had been troubled to know how to state the character of God and his relation to his creation works…}

\othersnogap{\textbf{He then stated that his former views \underline{regarding the trinity} had stood in his way of making a clear and absolutely correct statement; but that within a short time \underline{he had come to believe in the trinity} and could now see pretty clearly where all the difficulty was, and believed that he could clear the matter up satisfactorily.}}

\othersnogap{\textbf{He told me that he now believed in \underline{God the Father, God the Son, and God the Holy Ghost}; and his view was that it was God the Holy Ghost, and not God the Father, that filled all space, and every living thing. He said if he had believed \underline{this} before writing the book, he could have expressed his views without giving the wrong impression the book now gives.}}

\othersnogap{\textbf{I placed before him the objections I found in the teaching, and tried to show him that the teaching was so utterly contrary to the gospel that I did not see how it could be revised by changing a few expressions.}}

\othersnogap{We argued the matter at some length in a friendly way; but I felt sure that when we parted, the doctor did not understand himself, nor the character of his teaching. And I could not see how it would be possible for him to flop over, \textbf{and in the course of a few days \underline{fix the books up} so that it would be all right}.}[Letter: A. G. Daniells to W. C. White. October 29, 1903. pp. 1, 2][https://forgotten-pillar.s3.us-east-2.amazonaws.com/Letter-A-G-Daniells-to-W-C-White-October-29-1903.pdf]

Kellogg did not see the mistake in his sentiments; but rather, in expressing his views. He did not think that his views were false, merely his expression of those views, which led to the book giving a wrong impression. Yet, evidently, this was not true. As Sister White had stated, Kellogg had a problem with the sentiments regarding the \emcap{personality of God} and where His presence is. So, Kellogg suggested that in order to “\textit{fix the books up}" he would include the trinitarian expressions because he now started to believe in \textit{the Trinity} doctrine. At this point in time, the Seventh-day Adventist Church was not trinitarian—the doctrine of Trinity was not part of the \emcap{Fundamental Principles}, as we saw previously. Thus, it is no surprise that Brother Daniels objected and refuted Trinitarian teaching, claiming that it was\others{so utterly contrary to the gospel.} Revising the book, by changing a few expressions, would not change the main problem of the book: the sentiments on the \emcap{personality of God}. 

In the described events, and in William White’s response to Brother Daniells, we can see why Sister White wrote the Special Testimonies. William White responded to Brother Daniells on Nov. 4, 1903:

\others{Dear Brother, --}

\othersnogap{\textbf{\underline{Mother and I} have just read your letter of \underline{October 29} in which you speak of the \underline{various plans that have been proposed for the revising and reproduction of ‘The Living Temple}.’}}

\othersnogap{We were pleasantly surprised at the announcement that Dr. Kellogg would withdraw this book from the market, \textbf{and we are sorry indeed that his mind is swinging back to the plan of revising it, \underline{Mother expresses herself quite emphatically regarding this matter; she regards it as an unprofitable undertaking}}. I think she will write to you soon expressing her views regarding this.}

\othersnogap{\textbf{… I believe it will be necessary \underline{to issue a special Testimony soon}, and this must contain a very full and clear statement on the positive side of this question, as well as articles pointing out the errors in the teaching of those who have departed from the truth through fascinating and deceptive theories}.}[\href{https://ellenwhite.org/letterbooks/555}{Letter from W.C. White to A.G. Daniells, Nov. 4, 1903,} (p. 458)]

Here is evidence that Sister White was familiar with Dr. Kellogg’s intentions to revise “\textit{Living Temple}" and her familiarity with his belief in the Trinity doctrine. In William’s words, she expressed herself quite emphatically regarding this matter. She deemed it an unprofitable undertaking. For this reason, it was necessary to issue a special Testimony soon. And there it was. This is how the \textit{Testimonies for the Church Containing Letters to Physicians and Ministers Instruction to Seventh-Day Adventists} was published in 1904, containing letters to the physicians and ministers connected to Kellogg’s crisis.  

In saying\others{\textbf{\underline{Mother and I} have just read your letter of \underline{October 29}}}, William testified that Sister White was fully aware of Kellogg’s intentions and trinitarian belief. After she read Daniells’ letter, she wrote a direct reply to Dr. Kellogg. This letter is \textit{Lt253-1903}. It is a very prominent and eye opening letter because it clearly exposes how the prophet dealt with the Trinity doctrine. She elevated the doctrine on the \emcap{personality of God} constituted in the \emcap{Fundamental Principles}. There are striking similarities between this letter and the tenth chapter of the Special Testimonies, \textit{The Foundation of our Faith}.