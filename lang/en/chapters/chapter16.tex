\chapter{Dr. Kellogg and pantheism}

From her personal diary, on January 5, 1902, Sister White wrote that Kellogg’s \egwinline{science of God in nature is \textbf{true}}.

\egw{I am having things presented to me that worry my mind. Dr. Kellogg is traveling the same road that he did soon after taking up his responsibilities in the Sanitarium. \textbf{Human science is a lie in regard to God not having a personality}. I know this is a falsehood, and yet if we can in any way help the doctor we must try to do this. What can be said? There is such an exaltation given him that he is about to topple over the precipice. What can any of us do? The Lord alone can save Dr. Kellogg. \textbf{\underline{His science of God in nature is true}}, but he has placed nature where God should be. Nature is not God, but God created nature. \textbf{\underline{This science of God in nature is correct in one sense}}. \textbf{God gives to nature its life, its living properties, its beauty}. [He] is the author of all nature’s loveliness, and while He gives us this evidence of mighty power, \textbf{He is a personal God and Christ is a personal Saviour}.}[Ms236-1902.1; 1902][https://egwwritings.org/?ref=en\_Ms236-1902.1&para=12779.6]

\egwnogap{\textbf{We take not the fallacies of man but the Word of God that man was created after the image of God and Christ}, for the Word declares ‘God, who at sundry times and in divers manners spake in time past unto the fathers by the prophets, hath in these last days spoken unto us by his son, whom he hath appointed heir of all things, \textbf{by whom also he made the worlds; who being the brightness of his glory, and \underline{the express image of his person}}, and upholding all things by the word of his power, when he had by himself purged our sins, \textbf{sat down on the right hand of the Majesty of heaven}.’ Hebrews 1:1-3.}[Ms236-1902.4; 1902][https://egwwritings.org/?ref=en\_Ms236-1902.4&para=12779.9]

Interestingly, Sister White also claimed that God is in nature, and He is giving life and the living properties. Kellogg is correct on this point and his claim is definitely supported by her writings. Based on this point, Kellogg defended himself, saying that The Living Temple is in harmony with Sister White’s writings. He wrote to brother G. I. Butler precisely where Sister White advocated the same sentiment as he did.

\others{Sister White has clearly taken the same position with reference to this matter which I have taken. You will find it, in her little work on \textbf{Education }in the chapters ‘\textbf{God in Nature}’ and ‘\textbf{Science and the Bible.}’ You will find it all through ‘\textbf{Desire of Ages,}’ and ‘\textbf{Patriarchs and Prophets.}’}[Letter from Dr. Kellogg to Eld. Butler, February 21, 1904]

Let’s take a look at “\textit{God in Nature}”, in the book Education, where we can find the same sentiment regarding God in Nature that Kellogg promoted.

\egw{\textbf{Upon all created things is seen the impress of the Deity}. Nature testifies of God. The susceptible mind, brought in contact with the miracle and mystery of the universe, cannot but recognize \textbf{the working of infinite power}. \textbf{\underline{Not by its own inherent energy} does the earth produce its bounties}, and year by year continue its motion around the sun. \textbf{An unseen hand guides the planets in their circuit of the heavens}. \textbf{\underline{A mysterious life pervades all nature—a life that sustains the unnumbered worlds throughout immensity}}, \textbf{that lives in the insect atom which floats in the summer breeze, that wings the flight of the swallow and feeds the young ravens which cry, that brings the bud to blossom and the flower to fruit}.}[Ed 99.1; 1903][https://egwwritings.org/?ref=en\_Ed.99.1&para=29.470]

\egwnogap{\textbf{The same \underline{power} that upholds nature, is working also in man}. \textbf{The same great laws that guide alike the star and the atom control human life}. \textbf{The laws that govern the heart’s action, regulating the flow of the current of life to the body, are the laws of the mighty Intelligence that has the jurisdiction of the soul}. \textbf{\underline{From Him all life proceeds}}. Only in harmony with Him can be found its true sphere of action. For all the objects of His creation the condition is the same—\textbf{a life sustained by receiving the life of God}, a life exercised in harmony with the Creator’s will...}[Ed 99.2; 1903][https://egwwritings.org/?ref=en\_Ed.99.2&para=29.471]

\egw{...The heart not yet hardened by contact with evil is quick to \textbf{recognize the \underline{Presence} that pervades all created things}...}[Ed 100.2; 1903][https://egwwritings.org/?ref=en\_Ed.100.2&para=29.475]

In his defense, Kellogg was also referring to the Patriarchs and Prophets. There we read the following:

\egw{Many teach that matter possesses vital power,—that certain properties are imparted to matter, and it is then left to act through its own inherent energy; and that the operations of nature are conducted in harmony with fixed laws, with which God himself cannot interfere. \textbf{This is false science, and is not sustained by the word of God}. Nature is the servant of her Creator. God does not annul his laws, or work contrary to them; \textbf{but he is continually using them as his instruments. Nature testifies of an intelligence, \underline{a presence}, \underline{an active energy}, that works in and through her laws. There is in nature the continual working of \underline{the Father and the Son}.} Christ says, ‘My Father worketh hitherto, and I work.’ John 5:17.}[PP 114.4; 1980][https://egwwritings.org/?ref=en\_PP.114.4&para=84.445]

These quotations are in harmony with the quotations from The Living Temple.

\others{The manifestations of life are as varied as the different individual animals and plants, and parts of animated things. Every leaf, every blade of grass, every flower, every bird, even every insect, as well as every beast or every tree, bears witness to the infinite versatility and inexhaustible resources of \textbf{the one all-pervading, all-creating, all-sustaining Life}.}[John H. Kellogg, The Living Temple p. 16][https://archive.org/details/J.H.Kellogg.TheLivingTemple1903/page/n15/]

\others{Intelligence is one of the forces of the universe, one of the manifestations of the \textbf{\underline{all-pervading life which} created and creates, \underline{animates and sustains}}.}[John H. Kellogg, The Living Temple p. 396][https://archive.org/details/J.H.Kellogg.TheLivingTemple1903/page/n425/]

If Kellogg’s understanding of God as the source that sustains and animates nature is correct, then where is his error? Why is he called a pantheist? Is it fair to call him a pantheist? He definitely doesn’t think so. Take a look at what he wrote to Elder Butler:

\others{\textbf{I abhor pantheism} as much as you do. \textbf{I have endeavored in my book to simply teach the fact that man is dependent upon God for everything, and that without the divine power working in him the Spirit of God operating upon the elements which compose his body, he would be dust}.}[Letter from Dr. Kellogg to Eld. Butler, February 21, 1904]

\others{I am willing to renounce all the awful doctrines you and others attribute to me. I am willing to confess that \textbf{I am not a pantheist} nor a spiritualist, and that I believe none of the doctrines taught by these people or \textbf{by pantheistic or spiritualistic writings}. I never read a pantheistic book in my life. I never read a book on ‘New Thought,’ or anything of that kind. Anybody who will read carefully the ‘Living Temple’ from the first page right straight through to the last, and will give the matter fair and consistent consideration, ought to see very clearly that \textbf{I have no accord whatever with these pantheistic and spiritualistic theories}.}[Ibid.]

This is a very hard puzzle to solve unless you encounter the truth on the \emcap{personality of God}, which we covered in the beginning of this book. Yes, God sustains life in nature. In nature, we \egwinline{\textbf{recognize \underline{the Presence} that pervades all created things}}[Ed 100.2; 1903][https://egwwritings.org/?ref=en\_Ed.100.2&para=29.475]. But God \textit{Himself}—in His personality—is not in nature, nor is nature God. God is a \textit{personal being}, and He is in His holy temple, sitting on His throne. God is everywhere present by His \textit{representative}, the Holy Spirit.

When Sister White said \egwinline{Human science is a lie in regard to God \textbf{not having a personality},}[Ms236-1902; 1902][https://egwwritings.org/?ref=en\_Ms236-1902.1&para=12779.6] she was particularly referencing God having a physical form of a person, as could be seen in the context of that quotation. But when Dr. Kellogg was addressing ‘\textit{personality},’ he was not addressing the form or shape of a person. In 1936 in his lecture, he expressed the same sentiments he held in the Living Temple, only more vividly:

\others{So you see it is impossible to conceive of infinite things. They are beyond us. They are \textbf{outside of comprehension} and the same thing is true of \textbf{the \underline{infinite personality}}. \textbf{We can not form any conception of its shape or its size or any limitations of any sort because it is infinite}. Now, perhaps that is a difficult idea for you to take in and \textbf{the difficulty of accepting this idea is the fact that \underline{we have not a clear idea of personality}}. \textbf{We think of personality \underline{as connected with form}}.}

\others{…\textbf{It gave me a new conception of personality}. \textbf{\underline{Personality does not mean a person, a man or a woman}}. It does not mean that sort of thing at all. \textbf{It means the possession of the power to will and to do and to think and to plan}.}[\href{https://forgotten-pillar.s3.us-east-2.amazonaws.com/Sanitarium+Lecture+1936.pdf}{Dr. Kellogg Sanitarium Lectures, 1936}; For transcript see \href{https://notefp.link/1938-kellogg-lecture}{https://notefp.link/1938-kellogg-lecture}]

Such a view of personality applied to God led Dr. Kellogg into pantheism. The doctrine of the \emcap{personality of God} deals with the correct perception of God. Dr. Kellogg's perception of God was a trinitarian perception.

\others{All I wanted to explain in Living Temple was that this work that is going on in the man here \textbf{is not going on by itself \underline{like a clock wound up}; but it is the power of God and \underline{the Spirit of God that is carrying it on}}. \textbf{Now, I thought I had cut out entirely the theological side of questions of \underline{the trinity and all that sort of things}}. \textbf{I didn't mean to put it in at all}, and I took pains to state in the preface that I did not. I never dreamed \textbf{of such a thing} as any theological question being \textbf{brought into it}. I only wanted to show that \textbf{\underline{the heart does not beat of its own motion} but that it is \underline{the power of God that keeps it going}}.}[Interview, J. H. Kellogg, G. W. Amadon and A. C. Bourdeau, October 7th 1907 held at Kellogg’s residence][https://archive.org/details/KelloggVs.TheBrethrenHisLastInterviewAsAnAdventistoct71907/page/n37]

The heart does not beat of its own motion; it is the power of God that keeps it going. In this, Kellogg was absolutely right.

\egw{\textbf{The physical organism of man is under the supervision of God, but \underline{it is not like a clock which is set in operation and must go of itself}}. \textbf{The heart beats, pulse succeeds pulse, breath succeeds breath, but bear in mind that the being is under the supervision of God}. Ye are God's husbandry, ye are God's building. \textbf{In God we live and move and have our being}. \textbf{Each heartbeat, each breath is the inspiration of that God who breathed into the nostrils of Adam the breath of life}, the inspiration of the ever present God, the great I AM.}[13LtMs, Ms 92, 1898, par. 7][https://egwwritings.org/read?panels=p14063.7342012&index=0]

Dr. Kellogg's \egwinline{science of God in nature is true.}[Ms236-1902; 1902][https://egwwritings.org/?ref=en\_Ms236-1902.1&para=12779.6] The Scriptures clearly teach it: \bible{If he \normaltext{[God]} set his heart upon man, \textbf{if he gather unto himself \underline{his spirit} and his breath}; \textbf{\underline{All flesh shall perish together}, and man shall turn again unto dust}.}[Job 34:14-15] \bible{...thy judgments are a great deep: \textbf{O Lord, thou \underline{preservest} man and beast}... \textbf{For with thee is the fountain of life}: in thy light shall we see light.}[Psalm 36:6b,9]

This evidence testifies that Dr. Kellogg's science of God in nature is true, but his problems were erroneous views on the personality of God, which were trinitarian views. Even when he clarified that \others{God the Father sits upon his throne in heaven where God the Son is also; while God's life, or spirit or presence is the all-pervading power which is carrying out the will of God in all the universe,}[Letter: Dr. Kellogg to W. W. Prescott, October 25, 1903][https://forgotten-pillar.s3.us-east-2.amazonaws.com/1903-10-25-JHKellogg-to-W.W.Prescott.pdf] still he held erroneous views on the personality of God—God in \others{comprehensive sense} as \others{the Godhead… God the Father, God the Son, and God the Holy Spirit}[Ibid.][https://forgotten-pillar.s3.us-east-2.amazonaws.com/1903-10-25-JHKellogg-to-W.W.Prescott.pdf]. His Trinitarian view could \textit{not} \others{clear the matter up satisfactorily.}[Letter: A. G. Daniells to W. C. White, October 29, 1903][https://forgotten-pillar.s3.us-east-2.amazonaws.com/Letter-A-G-Daniells-to-W-C-White-October-29-1903.pdf]

The conclusion is frightening. If you believe that the heart does not beat of its own motion but that it is the power of God that keeps it going, and you combine it with the belief that God Himself is not a tangible being but a spirit present everywhere, then in the eyes of the Spirit of Prophecy, you are a pantheist. The perception of the quality or state of God being a person makes the difference between the true believer and the pantheist.
