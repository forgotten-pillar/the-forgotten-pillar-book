\chapter{The Heavenly Trio}

So far we have seen the evidence that Ellen White knew about Dr. Kellogg's trinitarian sentiments, and we have seen how she responded to it. She always uplifted the truth on the presence and the personality of God, and called to come back to the foundation of our faith—\emcap{Fundamental Principles}. However, when Adventist scholars are faced with the given data, they do not act in harmony with Ellen White's behavior. The \emcap{Fundamental Principles} are downplayed, the truth on the \emcap{personality of God} is ignored, and the twisted story is presented that Ellen White was trinitarian and responsible for the church's acceptance of the Trinity doctrine into our ranks. We want to challenge this twisted story by looking at the evidence that is often used to support this claim.

One of the most prominent quotations to support the claim that Sister White was responsible for accepting the Trinity doctrine into our ranks is her writings and comments on Matthew 28:19\footnote{\bible{Go ye therefore, and teach all nations, baptizing them in the name of the Father, and of the Son, and of the Holy Ghost}[Matthew 28:19]}. The most prominent quotation to stand out in defense of the Trinity doctrine is “\textit{the Heavenly Trio}” quotation:

\egw{\textbf{There are \underline{three living persons} of the \underline{heavenly trio}}; in the name of these three great powers—\textbf{the Father, the Son, and the Holy Spirit}—those who receive Christ by living faith are baptized, and these powers will co-operate with the obedient subjects of heaven in their efforts to live the new life in Christ...}[Ev 615.1; 1946][https://egwwritings.org/?ref=en\_Ev.615.1&para=30.3407]

To reiterate, this quotation is often cited to argue that Sister White defended and advocated the Trinity doctrine. But, if we take a look at this quotation in its context, we see that within the quotation itself she actually refuted this doctrine and exalted the truth on the \emcap{personality of God}. Let us look closely. 

The original quotation is part of the manuscript Ms 21, 1906, titled “\textit{Come Out and Be Separate}.” A few paragraphs before stating “\textit{the heavenly trio},” Sister White actually quoted a Trinitarian book, “\textit{Higher Christian Life},” by \textit{William Boardman}. Even more fascinating is that the quotation was taken from a section dealing with the topic of “\textit{The Holy Trinity}”. In order to understand what Sister White is referring to, we will take a brief look through this section of Boardman’s book. This will help us to understand Sister’s White position on the doctrine of Trinity and how it compares to Matthew 28:19.

\section*{The Higher Christian Life, William Boardman}

\textit{William Edwin Boardman}\footnote{“Biography of William E. Boardman.” Healingandrevival.com, 2025, \href{https://healingandrevival.com/BioWEBoardman.htm}{healingandrevival.com/BioWEBoardman.htm}. Accessed 3 Feb. 2025.} was a Presbyterian pastor and the teacher who, through the international success of his book “\textit{The Higher Christian Life}”, helped to ignite the \textit{Higher Life movement}\footnote{Wikipedia Contributors. “\href{https://en.wikipedia.org/wiki/Higher_Life_movement}{Higher Life Movement}.” Wikipedia, Wikimedia Foundation, 15 Jan. 2025.}. This book was well known in the nineteenth and twentieth centuries. It was also present in the personal library of Ellen G. White\footnote{“EGW Private and Office Libraries | Loma Linda University Del E. Webb Memorial Library.” Llu.edu, 2025, \href{https://library.llu.edu/heritage-research-center/egw-estate-branch-office/egw-private-and-office-libraries?combine_op=contains&combine=Higher+Christian+Life}{library.llu.edu/heritage-research-center/egw-estate-branch-office/egw-private-and-office-libraries}. Accessed 3 Feb. 2025.}. Passages from this book, which will be presented here, are the passages that Ellen White has referred to in the context of the “\textit{heavenly trio},” quotation. The complete book is freely available from the \textit{archive.org}\footnote{“The Higher Christian Life : Boardman, W. E. (William Edwin), 1810-1886 : Free Download, Borrow, and Streaming : Internet Archive.” \textit{Internet Archive}, 2025, \href{https://archive.org/details/higherchristianlife00boarrich?view=theater\#page/98/mode/2up}{archive.org/details/higherchristianlife00boarrich}. Accessed 3 Feb. 2025.} website, due to its expired copyrights. It is interesting to see what Sister White read from “\textit{The Holy Trinity}” section and her direct comments on this topic, yet it is more important to realize what God shows us here regarding the \emcap{personality of God} and the Trinity doctrine.

Speaking of Triune God, William Boardman writes:

\othersQuote{And then, again, the Father is the author and planner of salvation through faith in his Son; and when we trust in his Son we honor the Father, because we accept of his plan of salvation for us, justify his wisdom, and act in accordance with his will in the matter. \textbf{A glance at the official and essential relations of the persons of the Holy Trinity to each other and to us, may throw additional light upon our pathway}. Upon this subject flippancy would border upon blasphemy. It is holy ground. He who ventures upon it may well tread with unshod foot, and uncovered head bowed low.}[William Boardman, The Higher Christian Life, p. 99; 1858][https://archive.org/details/higherchristian02boargoog/page/n106/]

Brother Boardman wants us to take \others{a glance at the official and essential relations} of the three persons of the Holy Trinity. He asserts that \textit{God is one but also three}–\textit{Triune}–by presenting official and essential relations of the persons of the Holy Trinity. His fundamental statement and outline for his thesis is as follows:

\othersQuote{\textbf{The Father is fullness of the Godhead \underline{invisibly}, without form, whom no creature hath seen or can see}. \\
\textbf{The Son is the fullness of the Godhead \underline{embodied}, that his creatures may see him, and know him, and trust him}. \\
\textbf{The Spirit is the fullness of the Godhead \underline{in all active workings}, whether of creation, providence, revelation, or salvation, by which God manifests himself to and through the universe}.}[William Boardman, The Higher Christian Life, p. 100][https://archive.org/details/higherchristian02boargoog/page/n108/]

This statement is foundational to his following statements and illustrations. In the following paragraphs, William Boardman gives the biblical motives to illustrate \others{the official and essential relations of the Holy Trinity}—\textit{that is, God being one, but yet three}. He writes:

\othersQuote{Another of the names of Jesus will give the same analogies in a light not less striking - \textbf{The Sun of Righteousness}. \\
All the light of the sun in the heavens was once hidden in the invisibility of primal darkness; and after this, the light now blazing in the orb of day was, when first the command when forth, Let light be! and light was, at most only the diffused haze of the gray dawn of the morn of creation out of the darkness of chaotic night, without form, or body, or centre, or radiance, or glory. But when separated from the darkness and centered in the sun, then in its glorious glitter it became so resplendent that none but the eagle eye could bear to look it in the face. \\
But then again its rays falling aslant through earth’s atmosphere and vapors, gladdens all the world with the same light, dispelling the winter, and the cold, and the darkness; starting Spring forth in floral beauty, and Summer in vernal luxuriance, and Autumn laden with golden treasures for the garner.
\textbf{The Father is as the Light invisible}. \\
\textbf{The Son is as the Light embodied}. \\
\textbf{The Spirit is as the Light shed down}.}[William Boardman, The Higher Christian Life, p. 101,102][https://archive.org/details/higherchristian02boargoog/page/n108/]

This illustration of the Sun of Righteousness shows that God the Father, who is \textit{the fullness of the Godhead invisible,} is like a Light that \others{was once hidden in the invisibility of primal darkness}. The Son, who is \textit{the fullness of the Godhead embodied}, is like a Light that is embodied in \others{the morn of creation}. The Holy Spirit, who is \textit{the fullness of the Godhead} \textit{in all active workings}, is like a \others{Light shed down}. William Boardman gives us another similar illustration to clarify the \others{official relations of the persons of the Godhead}:

\othersQuote{One of the similies for blessed influences of the Spirit, \textbf{while giving the self-same official relations of the persons of the Godhead, to each other and to us}, may illustrate them still further,—\textbf{The Dew},—\textbf{The dew of Hermon} - the dew on the mown meadow. Before the dew gathers at all in drops, it hangs over all the landscape in visible vapor, omnipresent but unseen. By and by as the light wanes into morning, and as the temperature sinks and touches the dew point the invisible becomes the visible, the embodied; and, as the sun rises, it stands in diamond drops trembling and glittering in the sun’s young beams in pearly beauty upon leaf and flower, over all the face of nature. \\
But now again, a breeze springs up, the breath of heaven is wafted gently along, shaking leaf and flower, and in a moment the pearly drops are invisible angina. But where now? Fallen at the root of herb and flower to impart new life, freshness, vigor to all it touches. \\
\textbf{The Father is like the dew in invisible vapor}. \\
\textbf{The Son is like the dew gathered in beauteous form}. \\
\textbf{The Spirit is like the dew fallen to the seat of life}.}[William Boardman, The Higher Christian Life, p. 102,103][https://archive.org/details/higherchristian02boargoog/page/n110/]

The Father, who is \textit{the fullness of the Godhead invisible,} is illustrated by the \others{dew in invisible vapor}. The Son, who is \textit{the fullness of the Godhead embodied}, is illustrated by \others{the dew gathered in beauteous form}. The Spirit, who is \textit{the fullness of the Godhead in all active works}, is illustrated by \others{the dew fallen to the seat of life}. The next illustration that exemplifies the official relations of the three personalities of one God is by another Bible likening—the Rain.

\othersQuote{\textbf{Yet one more of these Bible likenings} – by no means exhausting them – will not be unwelcome, or useless, - \textbf{the Rain}. \\
Rain, like the dew, floats in invisibility, and omnipresence at the first, over all, around all. Seen by none. While it remains in its invisibility, the earth parches, clods cleave together, the ground cracks open, the sun pours down his burning heat, the winds lift up the dust in circling whirls, and rolling clouds, and famine gaunt and greedy stalks through the land, followed by pestilence and death. By and by, the eager watcher sees the little hand-like cloud rising far out over the sea. It gathers, gathers, gathers; comes and spreads as it comes, in majesty over the whole heavens: - But all is parched and dry and dead yet, upon earth. \\
But now comes a drop, and drop after drop, quicker, faster – the shower, the rain – sweeping on, and giving to earth all the treasures of the clouds – clods open, furrows soften, springs, rivulets, rivers, swell and fill, and all the land is gladdened again with restored abundance. \\
\textbf{The Father is like to the invisible vapor}. \\
\textbf{The Son is as the laden cloud and falling rain}. \\
\textbf{The Spirit is the Rain – fallen and working in refreshing power}.}[William Boardman, The Higher Christian Life, p. 103,104][https://archive.org/details/higherchristian02boargoog/page/n110/]

It is crucial to understand Boardman’s sentiments considering these illustrations. To make sure that he is not being misunderstood, he writes the following:

\othersQuote{\textbf{These likenings are all imperfect. They rather hide than illustrate \underline{the tri-personality of the one God}, for they are not persons but things, poor and earthly at best, to represent the living personalities of the living God. So much they may do, however, as to illustrate the official relations of each to the others and of each and all to us. And more. They may also illustrate the truth that all the fulness of Him who filleth all in all, dwells in each person of \underline{the Triune God}}. \\
\textbf{The Father is all the fulness of the Godhead INVISIBLE}. \\
\textbf{The Son is all the fulness of the Godhead MANIFESTED}. \\
\textbf{The Spirit is all the fulness of the Godhead MAKING MANIFEST}. \\
\textbf{The persons are not mere offices, or modes of revelation, but living persons of the living God}.}[William Boardman, The Higher Christian Life, p. 104,105][https://archive.org/details/higherchristian02boargoog/page/n112/]

It is crucial to emphasize that when Boardman uses these Bible likenings from nature, he speaks of the illustrations, and not reality. These illustrations are poor representations of \others{the living personalities of the living God}; however, they \textit{may }illustrate the official \textit{relations }of the \textit{tri-personality of the one God} and the truth that all the fullness of Him who filleth all in all, dwells in each person of the \textit{Triune God}.

In this brief look at “\textit{The Holy Trinity}” section, it is clear that Boardman’s sentiments are the sentiments of the Triune God, \textit{three living persons in the Trinity}.

\section*{Ellen White on William Boardman’s sentiment}

Sister White was acquainted with Boardman’s trinitarian sentiment. For him, God was undoubtedly a person, yet he was three persons. This basic premise is common to every version of the Trinity doctrine. Boardman tried to give some Biblical ground to this notion. This is what Sister White had to say regarding these sentiments:

\egw{I am instructed to say, \textbf{The sentiments} of those who are searching for advanced scientific ideas \textbf{\underline{are not to be trusted}}. Such representations as the following are made: \textbf{‘The Father is as the light invisible; the Son is as the light embodied; the Spirit as the light shed abroad}.’ \textbf{‘The Father is like the dew, invisible vapor; the Son is like the dew gathered in beauteous form; the Spirit is like the dew fallen to the seat of life}.’ Another representation: \textbf{‘The Father is like the invisible vapor. The Son is like the leaden cloud. The Spirit is rain fallen and working in refreshing power}.’}[Ms21-1906.8; 1906][https://egwwritings.org/?ref=en\_Ms21-1906.8&para=9754.15]

Which sentiments, according to Sister White, are not to be trusted? She answers with quotations—those from William Boardman’s book, “\textit{Higher Christian Life}”, the section about “\textit{The Holy Trinity}”. These sentiments are trinitarian sentiments. The illustrations that Sister White quoted expressed \others{the official and essential relations of the Holy Trinity}[William Boardman, The Higher Christian Life, p. 99; 1858][https://archive.org/details/higherchristian02boargoog/page/n106/]. Many who have only read the “\textit{heavenly trio}” quotation, without knowing its context and the theological underpinnings of Boardman’s book, assume that these sentiments are pantheistic\footnote{This is assumed due to the context of the Kellogg crisis. Ms21-1906 is written in the context of the Kellogg crisis, and Sister White’s reaction to William Boardman’s trinitarian sentiments reveals another connection between the Trinity doctrine and Kellogg’s crisis.}, yet they are trinitarian.

It is important to note that when Sister White wrote \egwinline{I am instructed to say}, she did not merely convey her own personal opinion. She conveyed what God instructed her to say. \textit{She had instruction from God to warn us that} \textit{the trinitarian sentiments are not to be trusted}. In the detailed analysis of the “\textit{heavenly trio}” quotation, we can see differences between truth and error.

\egw{\textbf{All these \underline{spiritualistic} representations are simply nothingness}. They are imperfect, untrue. They weaken and diminish the Majesty which no earthly likeness can be compared to. \textbf{God cannot be compared with the things His hands have made}. These are mere earthly things, suffering under the curse of God because of the sins of man. \textbf{The Father cannot be described by the things of earth}. \textbf{The Father is all the fulness of the Godhead \underline{bodily} and is \underline{invisible to mortal sight}}.}[Ms21-1906.9; 1906][https://egwwritings.org/?ref=en\_Ms21-1906.8&para=9754.15]

By observing the context, it is obvious that Sister White follows Boardman’s line of reasoning and corrects the mistakes. For better comparison, let us look at their writings side by side:

\begin{table}[h!]
\centering
\renewcommand{\arraystretch}{1.5}
\setlength{\tabcolsep}{15pt}
\begin{tabular}{|p{0.4\textwidth}|p{0.4\textwidth}|}
\hline
\multicolumn{1}{|c|}{\textbf{William Boardman}} & \multicolumn{1}{c|}{\textbf{Ellen G. White}} \\ \hline
\othersQuote{These likenings are all imperfect. They rather hide than \textbf{illustrate the tri-personality of the \underline{one God}}, for they are not persons but things, poor and earthly at best, to represent \textbf{the living personalities of the living God}. \textbf{So much they may do, however, as to illustrate the official relations of each to the other and of each and all to us. And more. They may also illustrate the truth that all the fulness of Him who filleth all in all, dwells in \underline{each person of Triune God}}.}[p. 104,105][https://archive.org/details/higherchristian02boargoog/page/n112] & 
\egw{\textbf{All these \underline{spiritualistic} representations are simply nothingness}. They are imperfect, untrue. They weaken and diminish the Majesty which no earthly likeness can be compared to. \textbf{God cannot be compared with the things His hands have made}. These are mere earthly things, suffering under the curse of God because of the sins of man. \textbf{The Father cannot be described by the things of earth}.}[Ms21-1906.9; 1906][https://egwwritings.org/?ref=en\_Ms21-1906.8&para=9754.15] \\ \hline
\end{tabular}
\end{table}

The context of this important quotation prompts important questions. Why does the prophet of God refer to the representations that illustrate the \others{tri-personality of the one God} as \egwinline{spiritualistic representations}, which illustrate the sentiment that \egwinline{is not to be trusted}? Or why does the prophet of God refer to the representations that \others{represent the living personalities of the living God} as \egwinline{spiritualistic representations}? Or why does the prophet of God, when referring to the representations that \others{illustrate the truth that all the fullness of Him who filleth all in all, dwells in each person of Triune God}, refer to them as \egwinline{spiritualistic representations}? All of these spiritualistic representations illustrate the sentiment that \egwinline{is not to be trusted}. This sentiment is clearly the trinitarian sentiment.

Boardman’s only point that Ellen White affirms is that these representations are imperfect. Surely, William Boardman would not agree with Ellen White that these representations are \textit{spiritualistic} and \textit{untrue}. On the contrary, he believes that these illustrations \others{illustrate the truth that all the fulness of Him who filleth all in all, dwells in each person of Triune God}. Ellen White does not drink this wine. The question remains: why?

In this comparison, it is clear who God is for William Boardman, and who He is for Sister White. For Boardman, God is the Triune God, a tri-personality of the one God. For Sister White, God is the Father. For Boardman, these representations are imperfect because they \others{rather hide than illustrate the tri-personality of the one God}, and for Sister White these representations are imperfect because \egw{The Father cannot be described by the things of earth}. For Boardman, God is the \textit{Triune God}; for Sister White, God is \textit{the Father}.

Sister White continues to follow Boardman’s line of reasoning and corrects the error.

\begin{table}[h!]
\centering
\renewcommand{\arraystretch}{1.5}
\setlength{\tabcolsep}{15pt}
\begin{tabular}{|p{0.4\textwidth}|p{0.4\textwidth}|}
\hline
\multicolumn{1}{|c|}{\textbf{William Boardman}} & \multicolumn{1}{c|}{\textbf{Ellen G. White}} \\ \hline
\othersQuote{The Father is fullness of the Godhead \textbf{invisibly}, \textbf{\underline{without form}}, whom \textbf{no creature hath seen \underline{or can see}}.}[p.100][https://archive.org/details/higherchristian02boargoog/page/n108/]

\othersQuote{The Father is all the fullness of the Godhead \textbf{INVISIBLE}.}[p.105][https://archive.org/details/higherchristian02boargoog/page/n112/] & 
\egw{The Father is all the fulness of the Godhead \textbf{\underline{bodily}}, and is \textbf{invisible to mortal sight}.}[Ms21-1906.9; 1906][https://egwwritings.org/?ref=en\_Ms21-1906.8&para=9754.15] \\ \hline
\end{tabular}
\end{table}

For Boardman, the Father does not have a form nor body and is invisible to all creatures. For Sister White, the Father has a form and body and is invisible only to mortal human beings.\footnote{When Sister White talks about mortals, she talks about sin polluted humanity. After the restoration of humanity, at the resurrection, Christ will give His immortal life to His children. For more information read \href{https://egwwritings.org/?ref=en_RH.July.5.1887.par.5}{{EGW, RH July 5, 1887, par. 5; 1887}}.}

This quotation is one of the most direct quotations regarding the \emcap{personality of God}. \egwinline{The Father is all the fullness of the Godhead \textbf{bodily}}[Ms21-1906.9; 1906][https://egwwritings.org/?ref=en\_Ms21-1906.9&para=9754.16].

It might be confusing to someone that the Father is all the fullness of the Godhead bodily because in \textit{Colossians 2:9}, when referring to Jesus, it is written that \bible{in him dwelleth all the fulness of the Godhead bodily.} Scripture does not contradict itself. \textit{Colossians 2:9} does not exclude the Father to be all the fulness of the Godhead bodily. Various places in the Bible describe the Father having a body (\textit{a form: Daniel 7:9,10; Revelation 4:2,3; 1 Kings 22:19-22; a shape: John 5:37}). He has the appearance of a man (\textit{Ezekiel 1:26-28}). He has a face (\textit{Exodus 33:20; Matthew 18:10; Revelation 22:3, 4}). However, the Bible is completely silent about the nature of its substance. The Bible teaches us that \bible{\textbf{The secret things belong unto the LORD our God}: \textbf{but those things which \underline{are revealed} belong unto us and to our children for ever}, that we may do all the words of this law}[Deuteronomy 29:29]. It is revealed to us that the Father has body, He is all the fulness of the Godhead bodily. Also, it is revealed that in Jesus also dwells all the fulness of the Godhead bodily, because \bible{it pleased the Father that in him should all fulness dwell}[Colossians 1:19]. This is not a contradiction whatsoever because the Son is \bible{the \textbf{express image of \underline{His person}}}[Hebrews 1:3].

\begin{table}[h!]
\centering
\renewcommand{\arraystretch}{1.5}
\setlength{\tabcolsep}{15pt}
\begin{tabular}{|p{0.4\textwidth}|p{0.4\textwidth}|}
\hline
\multicolumn{1}{|c|}{\textbf{William Boardman}} & \multicolumn{1}{c|}{\textbf{Ellen G. White}} \\ \hline
\othersQuote{The Son is the fullness of the Godhead \textbf{embodied, that his creatures may see him, and know him, and trust him}.}[p.100][https://archive.org/details/higherchristian02boargoog/page/n108/]

\othersQuote{The Son is all the fulness of the Godhead \textbf{MANIFESTED}.}[p.105][https://archive.org/details/higherchristian02boargoog/page/n112/] & 
\egw{The Son is all the fulness of the Godhead \textbf{manifested}. The Word of God declares Him to be ‘\textbf{the express image of His person}’. ‘God so loved the world that He gave \textbf{His only begotten Son}, that whosoever believeth in Him should not perish, but have everlasting life’. \textbf{Here is shown \underline{the personality of the Father}}.}[Ms21-1906.10; 1906][https://egwwritings.org/?ref=en_Ms21-1906.10&para=9754.17] \\ \hline
\end{tabular}
\end{table}

Sister White focused on the \emcap{personality of God}, which is the personality of the Father. In Christ, who is \egwinline{begotten in the express image of the Father’s person}[ST May 30, 1895, par. 3; 1895][https://egwwritings.org/?ref=en\_ST.May.30.1895.par.3&para=820.12891], is shown the personality of the Father. In the same way that Jesus is a person, so is the Father. The quality or state of Christ being a person is the same quality or state of the Father being a person. As Christ is a personal being, so is the Father. Just as all the fullness of the Godhead bodily dwells in Christ, so it does in the Father, because Christ is begotten in the express image of the Father’s person. In Him is shown the personality of the Father. These simple conclusions have been asserted by Scripture in John 3:16 and Hebrews 1:3.

Does the same reasoning, of the personality of the Father and Son, apply to the Holy Spirit? Speaking of the Holy Spirit, Sister White continues:

\egw{\textbf{The Comforter that Christ} promised to send after He ascended to heaven, \textbf{is the Spirit \underline{in} all the fulness of the Godhead}, making manifest the power of divine grace to all who receive and believe in Christ as a personal Saviour.}[Ms21-1906.11; 1906][https://egwwritings.org/?ref=en\_Ms21-1906.11&para=9754.18]

Sister White draws a distinction between Father and Son who \textbf{are}, individually, \textbf{all }the fullness of the Godhead, and the Spirit that is \textbf{in all }the fullness of the Godhead. This is a marked contrast to William Boardman’s reasoning, where all three are the fullness of the Godhead. Sister White does not follow this trinitarian fashion. The explanation is simple in light of the \emcap{personality of God} and of Christ. The Holy Spirit is a spirit, and the spirit dwells \textbf{in} the flesh/body. The Holy Spirit is \textbf{in all} the fullness of the Godhead\footnote{Take a look at the quotation from \href{https://egwwritings.org/?ref=en_Ms128-1897.13&para=5426.19}{{EGW, Ms128-1897.13; 1897}}, where Sister White states that the Father and the Son are the absolute Godhead.}.

Finally, the quotation continues to its most renowned part:

\begin{table}[h!]
\centering
\renewcommand{\arraystretch}{1.5}
\setlength{\tabcolsep}{15pt}
\begin{tabular}{|p{0.4\textwidth}|p{0.4\textwidth}|}
\hline
\multicolumn{1}{|c|}{\textbf{William Boardman}} & \multicolumn{1}{c|}{\textbf{Ellen G. White}} \\ \hline
\othersQuote{\textbf{The Father} is all the fulness of the Godhead INVISIBLE.}

\othersQuote{\textbf{The Son} is all the fulness of the Godhead MANIFESTED.}

\othersQuote{\textbf{The Spirit} is all the fulness of the Godhead MAKING MANIFEST.}

\othersQuote{\textbf{The persons} are not mere offices, or modes of revelation, \textbf{but living persons of the living God}.}[p.105][https://archive.org/details/higherchristian02boargoog/page/n112/] & 
\egw{There are \textbf{three living persons of the heavenly trio}; in the name of these three great powers—\textbf{the Father, the Son, and the Holy Spirit}—those who receive Christ by living faith are baptized, and these powers will co-operate with the obedient subjects of heaven in their efforts to live the new life in Christ.}[Ms21-1906.11; 1906][https://egwwritings.org/?ref=en_Ms21-1906.11&para=9754.18] \\ \hline
\end{tabular}
\end{table}

In light of the context of William Boardman’s book, we see a marked contrast between \others{three living persons of \textbf{one living God}}, which is the trinitarian sentiment, and \egwinline{the three living persons of \textbf{the heavenly trio}}, which is in accordance with the truth on the \emcap{personality of God}.

The word ‘\textit{trio}’ simply indicates the number three. The \textit{“heavenly trio}” is represented by the Father, the Son, and the Holy Spirit. But, contrary to popular assumption, they do not make one living God. Three-in-one and one-in-three are concepts that do away with the \emcap{personality of God}. This is why Sister White referred to trinitarian sentiments as sentiments that \egwinline{are not to be trusted}[Ms21-1906.8; 1906][https://egwwritings.org/?ref=en\_Ms21-1906.8&para=9754.15].

Sister White never followed any trinitarian fashion—neither in words and expressions, nor in sentiments. There is an almost effortless research endeavor we suggest, even encourage, that you undertake: in the writings of Ellen White, search for standard trinitarian terms like “\textit{three are one},” “\textit{one are three},” “\textit{one in three},” or “\textit{three in one}”. In her impressive oeuvre you will not find a single occurrence of any of these, let alone the word ‘\textit{trinity}’ itself\footnote{There is but one occurrence, in the writings of Ellen White, of the word ‘trinity’ referring to \egw{the lust of the flesh, the lust of the eyes and the pride of life.}[Lt43-1898.25; 1898][https://egwwritings.org/?ref=en_Lt43-1898.25&para=4806.31]}. She never used these phrases that are necessary to explain the trinitarian sentiment. Simply put, Sister White was not a Trinitarian, and neither was the whole Seventh-day Adventist church, in her days.

The last three paragraphs in the heavenly trio manuscript \href{https://egwwritings.org/?ref=en_Ms21-1906&para=9754.1}{(Ms21-1906; 1906)} give us more details about the controversy that was taking place in 1906, in connection with Boardman’s book, Higher Christian Life, and its connection with the Kellogg crisis.

\egw{I write this because any moment my life may be ended. \textbf{Unless there is a breaking away from the influence that Satan has prepared, and a \underline{reviving of the testimonies that God has given, souls will perish in their delusion}. They will accept fallacy after fallacy and will thus keep up a disunion that will always exist until those who have been deceived take \underline{their stand on the right platform}}. All this higher education that is being planned will be extinguished; for it is spurious. The more simple the education of our workers, the less connection they have with the men whom God is not leading, the more will be accomplished. \textbf{Work will be done in the \underline{simplicity} of true godliness, and the old, old times will be back when, under the Holy Spirit’s guidance, thousands were converted in a day. When the truth in its simplicity is lived in every place, then God will work through His angels as He worked on the day of Pentecost, and hearts will be changed so decidedly that there will be a manifestation of the influence of genuine truth, as is represented in the descent of the Holy Spirit}.}[Ms21-1906.18; 1906][https://egwwritings.org/?ref=en\_Ms21-1906.18&para=9754.25]

\egwnogap{The Holy Spirit never has and never will in the future divorce the medical missionary work from the gospel ministry. They cannot be divorced. Bound up with Jesus Christ, the ministry of the Word and the healing of the sick are one.}[Ms21-1906.19; 1906][https://egwwritings.org/?ref=en\_Ms21-1906.19&para=9754.26]

\egwnogap{The fifty-eighth chapter of Isaiah contains instruction for today. \textbf{‘Cry aloud, spare not, lift up thy voice like a trumpet, and show My people their transgression, and the house of Jacob their sin.’ God does not accept \underline{Dr. Kellogg as His laborer}, unless he will now break with Satan}. The work would not have been hindered, as it has been for the past several years, \textbf{if Dr. Kellogg were a converted man. ‘Come,’ I call, ‘come ye out and be separate from him and his associates whom he has leavened.’ I am now giving the message God has given me, to give to all who claim to believe the truth, \underline{‘Come ye out from among them, and be separate},’ else their sin in justifying wrongs and framing deceits will continue to be the ruin of souls. We cannot afford to be on the wrong side. We cannot afford to cover the truth with scientific problems. We urge that decided changes be made and no more stumbling blocks be placed before the feet of the people of God}. Let every soul put on the gospel shoes. \textbf{Let every soul pray and work, placing their feet upon the foundation Christ laid in giving His life for the life of the world}.}[Ms21-1906.20; 1906][https://egwwritings.org/?ref=en\_Ms21-1906.20&para=9754.27]

The heavenly trio quotation was part of Kellogg's controversy. This is evidence that Kellogg’s controversy included the Trinity doctrine. We are told to break \egwinline{away from the influence of Satan} and to revive the \egw{testimony that God has given} us, or else our souls will perish in delusions. These influences and delusions come from trinitarians such as \textit{William Boardman} and \textit{Dr. John H. Kellogg}. She is pointing us back to place our feet upon the foundation that was built by the Masterworker.\footnote{\href{https://egwwritings.org/?ref=en_SpTB02.54.2&para=417.276}{EGW, SpTB02 54.2; 1904}}