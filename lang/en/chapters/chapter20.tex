\chapter{Dr. Kellogg and Ellen White writings}

Dr. Kellogg asserted that in the Living Temple he represented the same sentiments advocated by Sister White. Likewise, today many claim that Sister White was trinitarian and is responsible for the church's acceptance of the Trinity doctrine\footnote{William Johnsson, Adventist Review, January 6th, 1994, ‘\textit{Present Truth –Walking in God’s Light}’}. Sister White, herself, declared such claims to be false.

\egw{\textbf{The enemy is seeking to \underline{bring in} among the people of God spiritualistic theories, which \underline{if accepted, would undermine the foundation of the faith} that has made us what we are}. He leads men to present fables clothed with Scripture. \textbf{There are those who assert that Sister White’s writings are in harmony with these teachings}.\textbf{ \underline{I declare this to be false}. Men may misapply Scripture; they may misinterpret my words; but God understands their devising}. How thankful I am for this! When the enemy comes in like a flood, \textbf{the Spirit of the Lord will lift up a standard for us against him}.}[Ms137-1903.21; 1903][https://egwwritings.org/?ref=en\_Ms137-1903.21&para=9939.30]

Dr. Kellogg advocated the theories that, if accepted, would undermine the foundation of our faith. It is crucial to correctly understand what constitutes the foundation of our faith, which Sister White referred to. We have seen that it refers to the \emcap{Fundamental Principles}. Looking at her writings, and the writings of our pioneers, we see that the Trinity doctrine contradicts the \emcap{personality of God} and the truth about God’s presence. Today, with the Trinity doctrine as part of our belief, we recognize that we have moved away from the \emcap{Fundamental Principles} and formed another foundation. Sister White was not responsible for this transition. It is purely a misinterpretation of her works. Her writings do not undermine the foundation of the faith that has made us what we are. Her later work is completely in harmony with the truth given in the beginning.

\egw{\textbf{The past fifty years have not dimmed one jot or principle of our faith as we received the great and wonderful evidences that were made certain to us in 1844, after the passing of the time.} ... \textbf{\underline{Not a word is changed or denied}. That which the Holy Spirit testified to as truth after the passing of the time, in our great disappointment, \underline{is the solid foundation of truth}. \underline{Pillars of truth were revealed}, and we accepted \underline{the foundation principles} that have made us what we are—Seventh-day Adventists, keeping the commandments of God and having the faith of Jesus.}}[Lt326-1905.3; 1905][https://egwwritings.org/?ref=en\_Lt326-1905.3&para=7678.9]

\section*{Misrepresentation of the church standpoint}

By misrepresenting Sister White’s writings, Dr. Kellogg did not only misrepresent her work, but the church’s official standpoint expressed in the \emcap{Fundamental Principles}. Ellen White rebuked Kellogg for misrepresenting the church’s standpoint. As we read this rebuke, let us keep in mind the church’s current standpoint on the \emcap{personality of God} as it compares to the first point of the \emcap{Fundamental Principles}.

\egw{You \textbf{are not sound in the truth}. Your statements made to believers and unbelievers \textbf{misrepresent us as a people who have not changed the truth for error}. They detract from the influence \textbf{God would have us possess before the world in revealing in plain, unmistakable language that we are \underline{true to the principles of our faith} and that we hold the beginning of our confidence firm unto the end}. We are strictly denominational. \textbf{We believe in 1903 the same truths we did believe when we established the Sanitarium and the College in Battle Creek, and \underline{we know that we had no ifs or ands about this matter}}.}[Lt300-1903.4; 1903][https://egwwritings.org/?ref=en\_Lt300-1903.4&para=7705.10]

\egwnogap{While you have told the things that you have and made the statements you have before unbelievers, my heart has been sad indeed. \textbf{You have evidenced that you have departed from the faith}. The very statements you have made before worldly men of influence, as the papers have reported your words, have been presented to me distinctly from your lips as you have spoken them. We cannot labor to give you influence as one whom we can trust with the sacred work connected with our institutions, for you need first to be converted and led.}[Lt300-1903.5; 1903][https://egwwritings.org/?ref=en\_Lt300-1903.5&para=7705.11]

\egwnogap{You are not sound in the faith. I have stated this in my diary months ago. \textbf{You have certainly placed the people of God, whom the Lord has led step by step in the ways of truth and placed upon \underline{a solid foundation}, in a false showing before unbelievers. Some have departed from the faith and \underline{will continue to misrepresent the work God has given me}}.}[Lt300-1903.6; 1903][https://egwwritings.org/?ref=en\_Lt300-1903.6&para=7705.12]

\egwnogap{\textbf{The sanctuary question is a clear and definite doctrine as we have held it as a people. \underline{You are not definitely clear on the personality of God, which is everything to us as a people}. \underline{You have virtually destroyed the Lord God Himself}}.}[Lt300-1903.7; 1903][https://egwwritings.org/?ref=en\_Lt300-1903.7&para=7705.13]

\egwnogap{Why should you take the liberty to make the statements which you have made, as though you had authority for thus stating, when they are falsehoods? \textbf{You have made the facts of our faith of none effect before unbelievers,} \textbf{and the truth which should ever be kept prominent and exalted with this people you have virtually denied and ignored in your many statements. How dared you to do this?} \textbf{It necessitates us now to present our true position which constitutes us Seventh-day Adventists}. Whatever influence God has given you in the past has been in mercy to you, letting the light shine upon you.}[Lt300-1903.8; 1903][https://egwwritings.org/?ref=en\_Lt300-1903.8&para=7705.14]

\egwnogap{\textbf{We cannot for a moment have any misrepresentation upon these solemn and important subjects of truth which have been the faith of our people since 1844. This means much to us.} The Lord would have me say to you that the enemy has, through his specious deceptions, placed his unbelief in your mind, and you have been working it out. \textbf{\underline{All who receive your presentations will enter upon strange paths if they connect with you}}. \textbf{You are \underline{bringing in} strange, common fire}, \textbf{but not the fire of God’s own kindling}; and now \textbf{I must speak plainly to our people that the Lord has led us step by step and shown us clear light upon the heavenly sanctuary in the most holy of holies where \underline{God revealed Himself} to His appointed ones.}}[Lt300-1903.9; 1903][https://egwwritings.org/?ref=en\_Lt300-1903.9&para=7705.15]

Dr. Kellogg misrepresented the truth that constituted the foundation of our faith; most specifically, he misrepresented the truth on the \emcap{personality of God}, which was everything to us as people. If in 1903, it necessitated \egwinline{\textbf{to present our true position which constitutes us Seventh-day Adventists}}, how much more is it important for us today? Sister White did her part in upholding the foundation of our faith in the beginning, but it seems like we have forgotten.

% Dr. Kellogg and Ellen White writings

\begin{titledpoem}
    \stanza{
        In faith's foundation, once so clear, \\
        Ellen White's words, we ought to revere. \\
        Kellogg claimed harmony, yet in deceit, \\
        Misrepresented truths, in his conceit.
    }

    \stanza{
        "The enemy seeks," she solemnly warned, \\
        To twist our beliefs, until they're scorned. \\
        Spiritualistic theories, cleverly dressed, \\
        In Scripture's garb, they falsely impressed.
    }

    \stanza{
        "False!" she declared, against the tide, \\
        Her writings and the Trinity, wrongly tied. \\
        Her defense was strong, her vision, broad, \\
        The Fundamental Principles, by God, lauded.
    }

    \stanza{    
        Not a word changed, nor principle dimmed, \\
        Since 1844, when light on faith brimmed. \\
        The pillars of truth, so firmly believed, \\
        By misinterpretations, were not deceived.
    }

    \stanza{
        Kellogg's stance, a misrepresentation, \\
        Of the faith's core, and its foundation. \\
        The personality of God, so crucial, so dear, \\
        By his theories, was muddled, we all fear.
    }

    \stanza{   
        Yet, Ellen White stood, unyielding, firm, \\
        Against falsehoods, her teachings confirm. \\
        Against the tide of Trinity's doctrine, she stood, \\
        Uplifting the truth of God and His Son, as she should.
    }

    \stanza{    
        Today, as then, let's hold the line, \\
        To the original faith, divine. \\
        Ellen White's legacy, let's rightly claim, \\
        In truth and spirit, always the same.
    }

    \stanza{    
        In battles of faith, let's not forget, \\
        Her defense of principles, firmly set. \\
        Against the currents of change and doubt, \\
        Her writings guide, without and within, throughout.
    }
\end{titledpoem}