\chapter{Ellen White and Matthew 28:19}

Many assert that Ellen White promoted the Trinity doctrine, and that she is the one responsible for accepting it into our ranks. These claims do not consider that she defended the \emcap{personality of God} expressed in the first point of the \emcap{Fundamental Principles}. To support the claims that Ellen White was trinitarian, quotations are presented to her comment on Matthew 28:19:

\bible{Go ye therefore, and teach all nations, \textbf{baptizing them in the name of \underline{the Father}, and of \underline{the Son}, and of \underline{the Holy Ghost}}.}[Matthew 28:19]

This verse has been most compelling in support of the Trinity doctrine. The Trinity doctrine has propositions about the \emcap{personality of God} of which this text says nothing to support. This verse itself does not teach that the Father, the Son, and the Holy Ghost, comprise one God, the God of the Bible. There are other explicit verses in the Bible that exclude such interpretation of the text, i.e. 1 Corinthians 8:4-6; John 17:3; Ephesians 4:4-6; 1 Timothy 2:5.

Unfortunately, the same unsupported assumptions made about Matthew 28:19 are made about Sister White’s quotations dealing with this verse. For example, Sister White uses terms like \egwinline{three highest powers in heaven}[Lt253a-1903.18; 1903][https://egwwritings.org/?ref=en\_Lt253a-1903.18&para=10143.25], \egwinline{three great powers of heaven}[8T 254.1; 1904][https://egwwritings.org/?ref=en\_8T.254.1&para=112.1450], \egwinline{the three holy dignitaries of heaven}[Ms92-1901.26: 1901][https://egwwritings.org/?ref=en\_Ms92-1901.26&para=10732.32] and similar expressions—none of these quotations justify the assumption that these three (the Father, the Son, and the Holy Spirit) make one God. On the contrary, as discussed in the previous chapter, keeping William Boardman’s sentiments and “\textit{the heavenly trio}” in context, “\textit{three-in-one}” sentiments \egwinline{should not be trusted}[Ms21-1906.8; 1906][https://egwwritings.org/?ref=en\_Ms21-1906.8&para=9754.15].

The heavenly trio (the Father, the Son and the Holy Spirit) are also present in other Bible verses, in addition to Matthew 28:19. There are several other instances in the New Testament where the Father, the Son and the Holy Spirit are mentioned, and these verses should be used to interpret the meaning behind the heavenly trio. None of the verses on the heavenly trio prove a three-in-one God; rather, all of them refer to the Father as one God. In the following verses, the heavenly trio is bolded in order to better distinguish the Father, the Son and the Holy Spirit.

\bible{There is one body, and \textbf{one Spirit}, even as ye are called in one hope of your calling; \textbf{One Lord}, one faith, one baptism, \textbf{One God and Father} of all, who is above all, and through all, and in you all.}[Ephesians 4:4-6]

\bible{Now there are diversities of gifts, but the \textbf{same Spirit}. And there are differences of administrations, but the \textbf{same Lord}. And there are diversities of operations, but it is \textbf{the same God} which worketh all in all.}[1 Corinthians 12:4-6]

\bible{The grace of \textbf{the Lord Jesus Christ}, and the love of \textbf{God}, and the communion of \textbf{the Holy Ghost}, be with you all. Amen.}[2 Corinthians 13:14]

\bible{For through \textbf{him} \normaltext{[Christ]} we both have access by one \textbf{Spirit} unto the \textbf{Father}.}[Ephesians 2:18]

\bible{But we are bound to give thanks alway to \textbf{God} for you, brethren beloved of \textbf{the Lord}, because \textbf{God} hath from the beginning chosen you to salvation through sanctification of \textbf{the Spirit} and belief of the truth.}[2 Thessalonians 2:13]

\bible{How much more shall the blood of \textbf{Christ}, who through the eternal \textbf{Spirit} offered himself without spot to \textbf{God}, purge your conscience from dead works to serve \textbf{the living God}?}[Hebrews 9:14]

\bible{Elect according to the foreknowledge of \textbf{God the Father}, through sanctification of \textbf{the Spirit}, unto obedience and sprinkling of the blood of \textbf{Jesus Christ}: Grace unto you, and peace, be multiplied.}[1 Peter 1:2]

All of the above verses talk about the heavenly trio (the Father, the Son and the Holy Spirit), and all of them consistently testify that the Father is the one referred to as God.
The same reasoning holds ground for Ellen White’s interpretation of Matthew 28:19.

\egw{Christ gave His followers a positive promise that after His ascension He would send them His Spirit. ‘Go ye therefore,’ He said, ‘and teach all nations, baptizing them in the name of \textbf{the Father (a personal God),} and of \textbf{the Son (a personal Prince and Saviour),} and of \textbf{the Holy Ghost (sent from heaven to represent Christ);} teaching them to observe all things whatsoever I have commanded you, and, lo, I am with you alway, even unto the end of the world.’ Matthew 28:19, 20.}[RH October 26, 1897, par. 9; 1897][https://egwwritings.org/?ref=en\_RH.October.26.1897.par.9&para=821.16317]

The brackets in this quotation are in the original manuscript written by Ellen White. Here, she gives her own interpretation of Matthew 28:19. The Father is a personal God, the Son is a personal Prince and Saviour, and the Holy Spirit is Christ’s representative. This interpretation is in harmony with the \emcap{personality of God} expressed in the first point of the \emcap{Fundamental Principles}.

\egw{Let them be thankful to God for His manifold mercies and be kind to one another. \textbf{They have \underline{one God} and \underline{one Saviour}; and \underline{one Spirit}—\underline{the Spirit of Christ}—is to bring unity into their ranks}.}[9T 189.3; 1909][https://egwwritings.org/?ref=en\_9T.189.3&para=115.1057]

In light of the presented evidence, we see that simply numbering the Father, the Son and the Holy Spirit, does not prove the \textit{three-in-one} assumption, nor is it in conflict with the \emcap{personality of God} expressed in the \emcap{Fundamental Principles}. There is no denial of three persons of the Godhead, but there is denial of the claim that all three persons are defined as persons in the same way. The quality or state of being a person is different for the Holy Spirit when compared to the personality of the Father and of the Son. The Father and the Son are two distinct beings, while the Holy Spirit is a spirit.

Matthew 28:19 is a valuable verse and it opens a new field of study within the Bible and the Spirit of Prophecy. In the context of the Living Temple, and referring to its sentiments, Sister White wrote that this verse should be studied most earnestly because it is not half understood.

\egw{Just before His ascension, Christ gave His disciples a wonderful presentation, \textbf{as recorded in the twenty-eighth chapter of Matthew}. \textbf{This chapter contains instruction} that our ministers, our \textbf{physicians}, our youth, and all our church members need to \textbf{study most \underline{earnestly}}. \textbf{Those who study this instruction as they should will \underline{not dare to advocate theories that have no foundation in the Word of God}}. My brethren and sisters, make the Scriptures, which contain the alpha and omega of knowledge, your study. \textbf{All through the Old Testament and the New, there are things \underline{that are not half understood}}. ‘And Jesus came and spake unto them, saying, All power is given unto Me in heaven and in earth. Go ye therefore, and teach all nations, \textbf{baptizing them in the name of the Father, and of the Son, and of the Holy Ghost}; teaching them to observe all things whatsoever I have commanded you; and, lo, I am with you alway, even unto the end of the world.’ [Verses 18-20.]}[Lt214-1906.10; 1906][https://egwwritings.org/?ref=en\_Lt214-1906.10&para=10171.16]

Matthew 28:19 is truly a valuable verse, but since it is not directly connected to the topic of the \emcap{personality of God}, we leave this study for another occasion.