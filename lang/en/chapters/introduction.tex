\chapter*{Introduction}

This book has three objectives to fulfill. The first one is to revive the old pillar of our faith called, “\textit{the personality of God}”. The second objective is to re-establish trust in the writings of Ellen White, and the third is to re-establish the original Adventist identity.

Prior to October 22, 1844, there was a great number of Adventists waiting for Christ to return on the clouds of heaven. It was a global movement of people awaiting His second coming. October 22 passed without Christ descending on the clouds and the great majority left the movement, scorning it, scorning the prophecies, the Bible, and God. Very few faithful, humble, men and women remained, who were unquestionably sure that God was leading this movement. They knew that God was shining the light of Truth and their hearts were eager to receive it. But in the eyes of the world, they were just demonstrated fanatics and dreamers. This great disappointment can be compared to the one Jesus’ disciples had after they saw their Lord being laid in the grave. They were unquestionably sure that Christ “\textit{was a prophet mighty in deed and word before God and all the people}”, but as He died on the cross, they were bitterly disappointed, because they “\textit{trusted that it had been He which should have redeemed Israel}.” Yet in their state of despair, in their state of self-disappointment, they were ready to receive the power to conquer the whole world with the Gospel. They met Christ and later received His Spirit. The same happened with the Adventist pioneers. They were a small group of people, bitterly disappointed; they sought the Lord with all their hearts and received Him in power and in Truth. The truths God revealed during this precious time of crisis constitute the foundation of Seventh-day Adventist faith. These truths were tested by all the seductive, deceptive theories of the world, by those scorning this small group, yet these grand truths prevailed. In the time of greatest need, Jesus gave His testimony by raising a little girl, the weakest of the weak, to approve all of His truths. Ellen White was not to be the source of the truths; rather, to support the brethren who were seeking the truth in the Bible. God used Ellen White to approve their studies and to point them to the Bible. The final result was the establishment of the foundation of faith based on the Bible, which standeth sure till the end of the world.

Would you be surprised to know that the foundation of Seventh-day Adventist faith, which was laid at the beginning of our work, is in a fair degree different from what it is currently? Today, more than a century and a half later, we marvel in amazement over the accounts of the experiences of our pioneers; but since then, the Seventh-day Adventist Church has been subject to several new movements. Since then, the church has experienced many changes, including changes in our doctrine. Some argue that these changes are good and progressive; others argue that they are destructive and deceptive. Moving the spotlight to the original Seventh-day Adventism, it initiates the great controversy in the present days. We have personally been in this controversy for over 6 years now and we have seen that it will only get bigger and stronger, often with results of a sad record. Many people from both sides of this controversy are rejecting the Spirit of Prophecy in one way or another. Some have left the Seventh-day Adventist Church altogether. The Adventist identity is either lost or drastically changed from the initial one.

We are currently witnessing the shaking of the Seventh-day Adventist church, seeing her tossed through one wave of crisis after another. Many are losing their faith and their identity as Seventh-day Adventists. But we believe in a solution that the Lord, in His mercy, has already provided. The solution can be found in the history of the Seventh-day Adventist movement.

\egw{\textbf{In reviewing our past history}, having traveled over every step of advance to our present standing, I can say, Praise God! As I see what the Lord has wrought, I am filled with astonishment, and with confidence in Christ as leader. \textbf{We have nothing to fear for the future, \underline{except as we shall forget} the way the Lord has led us, and \underline{His teaching} in our past history}.}[LS 196.2; 1915][https://egwwritings.org/?ref=en\_LS.196.2]

We shall not fear! This is a great reassurance and promise—though conditional. We must \textit{remember} how the Lord has led us, and \textit{His teaching in our past history}. When we look at what the Lord has taught us in our past history, we are surprised to see how things have changed. The change has taken several years and many crises. To judge these changes in doctrine, whether good and progressive or bad and destructive, evaluation should be based on past experiences, as the Lord clearly led His church.

At this time, we put forth a bold claim—one that is supposed to make you hold this book until the end of its cover. Encouraged by the counsels of Ellen White to review our past history, we have concluded that we have forgotten one crucial pillar of our faith, which was the main subject of Kellogg’s controversy—the \emcap{personality of God}. One of the biggest crises that the SDA Church ever had in the time of the living prophet was the Kellogg crisis. It is out of this crisis that many other crises, today, find their roots. In this light, the subject of the \emcap{personality of God} is pivotal in our present time. 

Sister White wrote to Kellogg that the \emcap{personality of God} and the \emcap{personality of Christ} was a pillar of our faith in the same rank as is the sanctuary message:

\egw{Those who seek to remove \textbf{the old landmarks} are not holding fast; they \textbf{are \underline{not remembering} how they have received and heard}. Those who try to \textbf{\underline{bring in} theories that would remove \underline{the pillars of our faith} concerning the sanctuary, \underline{or concerning the personality of God or of Christ}, are working as blind men}. They are seeking to bring in uncertainties and to set the people of God adrift, without an anchor.}[Ms62-1905.14][https://egwwritings.org/?ref=en\_Ms62-1905.14]

The \emcap{personality of God} receives very little attention today as a subject, yet it is one of the crucial elements in dealing with other doctrines pertaining to Adventism, such as the doctrine of Trinity, the Sanctuary service, 1844 and any other doctrine dealing with the Heavenly reality.

The \emcap{personality of God} was a pillar of our faith. Today, it is almost forgotten. We propose a reasonable explanation for that. It is due to the evolution of the English language. What is meant by the term, “\textit{the personality of God}”? The general understanding of the English word ‘\textit{personality}’ has changed over the years. Today, ‘\textit{personality}’ is generally viewed as, “\textit{the characteristic set of behaviors, cognitions, and emotional patterns}”\footnote{Wikipedia Contributors. “\textit{Personality.}” Wikipedia, Wikimedia Foundation, 19 Apr. 2019, \href{https://en.wikipedia.org/wiki/Personality}{en.wikipedia.org/wiki/Personality}.}, but in the nineteenth, and beginning of the twentieth century, it meant “\textit{the quality or state of \textbf{being a person}}”\footnote{\href{https://www.merriam-webster.com/dictionary/personality}{Merriam-Webster Dictionary}, - ‘personality’} \footnote{\href{https://babel.hathitrust.org/cgi/pt?id=mdp.39015050663213&view=1up&seq=780}{Hunter Robert, The American encyclopaedic dictionary}, ‘\textit{personality}’ - “\textit{the quality or state of being personal}”; Mentioned dictionary was in possession of Ellen White (see \href{https://repo.adventistdigitallibrary.org/PDFs/adl-22/adl-22251050.pdf?_ga=2.116010630.1065317374.1621993520-1506151612.1617862694&fbclid=IwAR3vwmp8jxtnpPEKv0KD9mCv8dJpmRGoyIXW0CkbQAjbU0h6YaBGqhgBzbk}{EGW Private and Office Libraries})}. We read this definition as the primary definition of the word ‘\textit{personality}’ from the Merriam-Webster Dictionary\footnote{\href{https://www.merriam-webster.com/dictionary/personality\#word-history}{Merriam-Webster Dictionary} marks that the first record of the definition “the quality or state of being a person” is recorded in the 15th century.}. When Sister White and our pioneers wrote about the \emcap{personality of God}, they referred to \textit{the quality or state of God being a person}. In other words, they dealt with the question, “\textit{is God a person}”, and, “\textit{what is it that makes Him a person}” or “\textit{what is the quality or state of God being a person}”? Try to remember the last time you had a Bible study on the question, “\textit{is God a person?}” Think about how you can prove to yourself, from the Bible, that God is a person. Think about it. It is an important question. Upon this question hangs your view of God and your relationship toward Him. The \emcap{personality of God} is fundamental to true spirituality; true spirituality is based on your personal relationship with God. No real relationship of any kind can be formed with anyone unless he/she is a person. Maybe you have never asked yourself this question because you never felt a need to question if God is a person, and what is it (the quality or state) that makes Him a person. Or, maybe you were refraining from this question because you felt it might be a mystery that God did not intend to reveal. Maybe it will surprise you to know that God has given a definite and affirmative answer in His Word to the question “\textit{what is the quality or state of God being a person}”. What was even more surprising for us, was that the Adventist pioneers, including Sister White, had definite light regarding this topic, and they held it as a pillar of our faith, as part of the foundation of Seventh-day Adventist faith. When the \emcap{personality of God} is rightly understood in light of our historical past, old quotations shine in a new light and new shreds of evidence are presented, which will deepen the understanding of our past history and the present crisis.

The root problem of the Kellogg crisis was about the \emcap{personality of God}. It is certainly important to evaluate Kellogg's crisis over the \emcap{personality of God} using the meaning intended at that time; that is, using the definition of ‘\textit{personality},’ as the quality or state of God being a person. With this definition in mind, the Kellogg crisis comes into a new light and new relevant evidence is brought forth for us today. In light of this evidence, we see how God has led us in the past; thus, we should not fear for the future. Knowing and understanding this, as well as its importance, helps us to not be shaken by any wave of deception in present controversies. When Sister White was drawing Kellogg’s attention to the importance of this subject, she was drawing our attention also, as it is everything to us as a people.

[Writing to Kellogg] \egw{You are not definitely clear on \textbf{the personality of God}, which is \textbf{\underline{everything} to us as a people}.}[Lt300-1903.7][https://egwwritings.org/?ref=en\_Lt300-1903.7]

These studies on the \emcap{personality of God} will prompt a lot of new and hard questions. We do not promise to answer all of them, and perhaps you won’t be satisfied with the answers provided, but we pray, hope and believe that this book will fulfill the three objectives proposed in the beginning of this introduction. Through the reviving of the doctrine on the \emcap{personality of God}, we believe that your confidence in the Spirit of Prophecy will strengthen, and that you’ll find yourself rooted deeper in the Adventist message—where we find our identity as people—making you a more faithful Seventh-day Adventist. Most importantly, we want you to become more aware of God as your personal God. This will surely strengthen and deepen your relationship with Him. 

We find answers to the issue on the \emcap{personality of God} in examining the Kellogg crisis, where Sister White gave the most definite light on the \emcap{personality of God} and on the foundation of Seventh-day Adventist faith. The following is the complete tenth chapter from the book, \textit{Testimonies for the Church Containing Letters to Physicians and Ministers Instruction to Seventh-Day Adventists}. This chapter, \textit{The Foundation of our Faith}, contains deep insight into the history of Kellogg’s crisis. It gives a historical overview of the truths God gave as the foundation of our faith and in these truths we find our identity as Seventh-day Adventists— keeping the commandments of God and having the faith of Jesus.
