\chapter{Ellen White and Matthew 28:19}

Many assert that Ellen White promoted the Trinity doctrine, and that she is the one responsible for accepting it into our ranks. These claims do not consider that she defended the \emcap{personality of God} expressed in the first point of the \emcap{Fundamental Principles}. To support the claims that Ellen White was trinitarian, quotations are presented to her comment on Matthew 28:19:

\bible{Go ye therefore, and teach all nations, \textbf{baptizing them in the name of \underline{the Father}, and of \underline{the Son}, and of \underline{the Holy Ghost}}.}[Matthew 28:19]

This verse has been most prominent in support of the Trinity doctrine. The Trinity doctrine has propositions about the \emcap{personality of God} of which this text says nothing to support. This verse itself does not teach that the Father, the Son, and the Holy Ghost, comprise \textit{one} God, the God of the Bible. There are other explicit verses in the Bible that exclude such interpretation of the text, i.e. 1 Corinthians 8:4-6; John 17:3; Ephesians 4:4-6; 1 Timothy 2:5.

Unfortunately, the same unsupported assumptions made about Matthew 28:19 are made about Sister White’s quotations dealing with this verse. For example, Sister White uses terms like \egwinline{three highest powers in heaven}[Lt253a-1903.18; 1903][https://egwwritings.org/?ref=en\_Lt253a-1903.18&para=10143.25], \egwinline{three great powers of heaven}[8T 254.1; 1904][https://egwwritings.org/?ref=en\_8T.254.1&para=112.1450], \egwinline{the three holy dignitaries of heaven}[Ms92-1901.26: 1901][https://egwwritings.org/?ref=en\_Ms92-1901.26&para=10732.32] and similar expressions—none of these quotations justify the assumption that these three (the Father, the Son, and the Holy Spirit) make \textit{one} God. On the contrary, as discussed in the previous chapter, keeping William Boardman’s sentiments and \egwinline{the heavenly trio} in context, “\textit{three-in-one}” sentiments \egwinline{should not be trusted}[Ms21-1906.8; 1906][https://egwwritings.org/?ref=en\_Ms21-1906.8&para=9754.15].

The heavenly trio (the group of three: the Father, the Son and the Holy Spirit) are also present in other Bible verses, in addition to Matthew 28:19. There are several other instances in the New Testament where the Father, the Son and the Holy Spirit are mentioned, and these verses should be used to interpret the meaning behind the heavenly trio. None of the verses on the heavenly trio prove a three-in-one God; rather, all of them refer to the Father as one God. In the following verses, the heavenly trio is bolded in order to better distinguish the Father, the Son and the Holy Spirit.

\bible{There is one body, and \textbf{one Spirit}, even as ye are called in one hope of your calling; \textbf{One Lord}, one faith, one baptism, \textbf{One God and Father} of all, who is above all, and through all, and in you all.}[Ephesians 4:4-6]

\bible{Now there are diversities of gifts, but the \textbf{same Spirit}. And there are differences of administrations, but the \textbf{same Lord}. And there are diversities of operations, but it is \textbf{the same God} which worketh all in all.}[1 Corinthians 12:4-6]

\bible{The grace of \textbf{the Lord Jesus Christ}, and the love of \textbf{God}, and the communion of \textbf{the Holy Ghost}, be with you all. Amen.}[2 Corinthians 13:14]

\bible{For through \textbf{him} \normaltext{[Christ]} we both have access by one \textbf{Spirit} unto the \textbf{Father}.}[Ephesians 2:18]

\bible{But we are bound to give thanks alway to \textbf{God} for you, brethren beloved of \textbf{the Lord}, because \textbf{God} hath from the beginning chosen you to salvation through sanctification of \textbf{the Spirit} and belief of the truth.}[2 Thessalonians 2:13]

\bible{How much more shall the blood of \textbf{Christ}, who through the eternal \textbf{Spirit} offered himself without spot to \textbf{God}, purge your conscience from dead works to serve \textbf{the living God}?}[Hebrews 9:14]

\bible{Elect according to the foreknowledge of \textbf{God the Father}, through sanctification of \textbf{the Spirit}, unto obedience and sprinkling of the blood of \textbf{Jesus Christ}: Grace unto you, and peace, be multiplied.}[1 Peter 1:2]

All of the above verses talk about the heavenly trio (the Father, the Son and the Holy Spirit), and all of them consistently testify that the Father is the one referred to as God.
The same reasoning holds ground for Ellen White’s interpretation of Matthew 28:19.

\egw{Christ gave His followers a positive promise that after His ascension He would send them His Spirit. ‘Go ye therefore,’ He said, ‘and teach all nations, baptizing them in the name of \textbf{the Father (a personal God),} and of \textbf{the Son (a personal Prince and Saviour),} and of \textbf{the Holy Ghost (sent from heaven to represent Christ);} teaching them to observe all things whatsoever I have commanded you, and, lo, I am with you alway, even unto the end of the world.’ Matthew 28:19, 20.}[RH October 26, 1897, par. 9; 1897][https://egwwritings.org/?ref=en\_RH.October.26.1897.par.9&para=821.16317]

The brackets in this quotation are in the original manuscript written by Ellen White. Here, she gives her own interpretation of Matthew 28:19. The Father is a personal God, the Son is a personal Prince and Saviour, and the Holy Spirit is Christ’s representative. This interpretation is in harmony with the \emcap{personality of God} expressed in the first point of the \emcap{Fundamental Principles}. Matthew 28:19 is a matter of interpretation. The interpretation which makes the Heavenly Trio one God is not inspired. This is not what the text indicates. Rather, let's read Matthew 28:19 within inspired compound: "\textit{Go ye therefore, and teach all nations, baptizing them in the name of a personal God, a personal Prince and Savior, and of the Holy Ghost}." If one would read the text as such, no one would ever assume that one God is a unity of three persons. Therefore, let's stick to the inspiration, rather than subterfuges\footnote{\href{https://egwwritings.org/?ref=en\_Lt232-1903.41&para=10197.50}{{EGW, Lt232-1903.41; 1903}}}.

\egw{Let them be thankful to God for His manifold mercies and be kind to one another. \textbf{They have \underline{one God} and \underline{one Saviour}; and \underline{one Spirit}—\underline{the Spirit of Christ}—is to bring unity into their ranks}.}[9T 189.3; 1909][https://egwwritings.org/?ref=en\_9T.189.3&para=115.1057]

In light of the presented evidence, we see that simply numbering the Father, the Son and the Holy Spirit, does not prove the \textit{three-in-one} assumption, nor is it in conflict with the \emcap{personality of God} expressed in the \emcap{Fundamental Principles}. There is no denial of three persons of the Godhead, but only a denial of the assumption that these Three Great Worthies make one God.

Matthew 28:19 is a valuable verse and it opens a new field of study within the Bible and the Spirit of Prophecy. In the context of the Living Temple, and referring to its sentiments, Sister White wrote that this verse should be studied most earnestly because it is not half understood.

\egw{Just before His ascension, Christ gave His disciples a wonderful presentation, \textbf{as recorded in the twenty-eighth chapter of Matthew}. \textbf{This chapter contains instruction} that our ministers, our \textbf{physicians}, our youth, and all our church members need to \textbf{study most \underline{earnestly}}. \textbf{Those who study this instruction as they should will \underline{not dare to advocate theories that have no foundation in the Word of God}}. My brethren and sisters, make the Scriptures, which contain the alpha and omega of knowledge, your study. \textbf{All through the Old Testament and the New, there are things \underline{that are not half understood}}. ‘And Jesus came and spake unto them, saying, All power is given unto Me in heaven and in earth. Go ye therefore, and teach all nations, \textbf{baptizing them in the name of the Father, and of the Son, and of the Holy Ghost}; teaching them to observe all things whatsoever I have commanded you; and, lo, I am with you alway, even unto the end of the world.’ [Verses 18-20.]}[Lt214-1906.10; 1906][https://egwwritings.org/?ref=en\_Lt214-1906.10&para=10171.16]

There is a reason why Ellen White pipointed to Matthew 28:19 as a Scripture which is \egwinline{not half understood.} This statement is made in the context of 1906, where many ministers, and physicians were advocating the trinity doctrine. As we have seen, the understanding of God as a trinity, was not something Ellen White supported, and for this reason, herself, she dared not \egwinline{to advocate theories that have no foundation in the Word of God.}

\egw{The great Teacher held in His hand \textbf{the entire map of truth. In \underline{simple} language He \underline{made plain} to His disciples} the way to heaven and \textbf{the endless subjects of divine power}. \textbf{The question of \underline{the essence of God} was a subject on which He maintained a wise reserve}, for their entanglements and specifications would bring in science which could not be dwelt upon by unsanctified minds without confusion. \textbf{In regard to God and in regard to His personality, the Lord Jesus said}, ‘Have I been so long time with you, and yet hast thou not known Me, Philip? He that hath seen Me hath seen the Father.’ [John 14:9.] \textbf{Christ was the express image of His Father’s person}.}[19LtMs, Ms 45, 1904, par. 15][https://egwwritings.org/read?panels=p14069.9381023&index=0]

\egwnogap{The open path, the safe path of walking in the way of His commandments, is a path from which there is no safe departing. \textbf{And when men follow their own human theories dressed up in soft, fascinating representations, they make a snare in which to catch souls}. \textbf{\underline{In the place of devoting your powers to theorizing}}, Christ has given you a work to do. His commission is, Go <throughout the world> and make disciples of all nations, \textbf{baptizing them in the name of the Father, and of the Son, and of the Holy Ghost}. Before the disciples shall compass the threshold, there is to be the imprint of \textbf{the sacred name, baptizing the believers in \underline{the name of the threefold powers} in the heavenly world}. The human mind is impressed in this ceremony, the beginning of the Christian life. It means very much. The work of salvation is not a small matter, but so vast that \textbf{the highest authorities} are taken hold of by the expressed faith of the human agency. \textbf{The Father, the Son, and the Holy Ghost, \underline{the eternal Godhead} is involved in the action required to make assurance to the human agent to unite \underline{all heaven} to contribute to the exercise of human faculties to reach and embrace the fulness of \underline{the threefold powers} to unite in the great work appointed, confederating the heavenly powers with the human, that men may become, through heavenly efficiency, partakers of the divine nature and workers together with Christ}.}[19LtMs, Ms 45, 1904, par. 16][https://egwwritings.org/read?panels=p14069.9381024&index=0]

This quotation is yet another often misrepresented statement. It has been often used to argue that Ellen White advocated for the Trinity by referencing the Father, the Son and the Holy Spirit by term \egwinline{eternal Godhead.} However, we must peel back the layers of its context. Ellen White was explaining the meaning behind Matthew 28:19. She stated: \egwinline{In the place of devoting your powers to theorizing,} fulfill the commission given by Christ. Theorizing about what? Theorizing about \egwinline{the essence of God.} This is another “smoking gun” for the Trinity doctrine, especially when she referenced the \emcap{personality of God} by stating: \egwinline{\textbf{In regard to God and in regard to His personality}, the Lord Jesus said…[John 14:9.] Christ was the express image of His \textbf{Father’s person}.} John 14:9 does not mean that seeing the Father in Christ implies they are one and the same person, all part of one God. Rather, it affirms that Christ is the express image of the Father’s person. The “God” she referred to was the Father. Indeed, Jesus taught the truth about who and what God is. This is what He \egwinline{made plain} \egwinline{in the simple language.} To claim that by the term \egwinline{eternal Godhead} Ellen White was endorsing the Trinity would contradict the very caution she expressed in the context of this passage.

Unfortunately, the desperate desire of Trinitarians to paint Ellen White as a Trinitarian advocate has overshadowed the true, inspired meaning of Matthew 28:19. Her message was: \egwinline{In the place of devoting your powers to theorizing} about \egwinline{the essence of God,} Christ has given us the commission in Matthew 28:19. And she explained the meaning of Matthew 28:19. Her point was: The Father, Son, and Holy Spirit unite all of heaven’s resources with human effort so that, through divine power, people may share in God’s nature and work alongside Christ. That is the meaning of this \egwinline{threefold name.} She continued explaining:

\egw{\textbf{Man’s capabilities can multiply through the connection of human agencies with divine agencies}. \textbf{United with the heavenly powers}, the human capabilities increase according to that faith that works by love and purifies, sanctifies, and ennobles the whole man. \textbf{\underline{The heavenly powers} have \underline{pledged themselves} to minister to human agents to make the name of God and of Christ and of the Holy Spirit their living efficiency, working and energizing the sanctified man, to make this name above every other name}. \textbf{All the treasures of heaven are under obligation to do for man} infinitely more than human beings can comprehend by multiplying threefold the human with the heavenly agencies.}[19LtMs, Ms 45, 1904, par. 17][https://egwwritings.org/read?panels=p14069.9381026&index=0]

\egwnogap{\textbf{\underline{The three great and glorious heavenly characters} are present on the occasion of baptism. All the human capabilities are to be henceforth consecrated powers to do service for God in representing the Father, the Son, and the Holy Ghost upon whom they depend. \underline{All heaven is represented by these three} in covenant relation with the new life}. ‘If ye then be risen with Christ, seek those things that are above, where Christ sitteth at \textbf{the right hand of God}.’ [Colossians 3:1.]}[19LtMs, Ms 45, 1904, par. 18][https://egwwritings.org/read?panels=p14069.9381027&index=0]

Many claim that Matthew 28:19 is uninspired because it was inserted by the Catholic Church\footnote{Note, 1 John 5:7 \bible{For there are three that bear record in heaven, the Father, the Word, and the Holy Ghost: and these three are one.} is an interpolation known as “\textit{Johannine Comma}”. Ellen White never used that verse. This was not the case with Matthew 28:19.}. Yet, here we have divine inspiration revealing its true meaning—the significance of baptism in the threefold name as a pledge made by these \egwinline{three great and glorious heavenly characters.} Their pledge is that \egwinline{\textbf{all the treasures of heaven are under obligation to do for man} infinitely more than human beings can comprehend by multiplying threefold the human with the heavenly agencies.}

Ellen White frequently quoted Matthew 28:19, explaining the pledge of the Father, the Son, and the Holy Spirit. This pledge serves as a wonderful encouragement and a promise upheld by Heaven. A detailed study of this pledge is beyond the scope of this book, as it does not directly address the presence and \emcap{personality of God}. However, we encourage you to explore this topic for yourself. When you delve deeper into its meaning, you will come to understand the reality of the ministry of angels.

Sister White stated that \egwinline{all heaven is represented by these three in covenant relation with the new life.} These three are the Father, the Son, and the Holy Spirit. In another instance, she said:

\egw{\textbf{All heaven is interested in your home}. \textbf{God and Christ and \underline{the heavenly angels}} are intensely desirous that you shall so train your children that they will be prepared to enter the family of the redeemed.}[17LtMs, Ms 161, 1902, par. 11][https://egwwritings.org/read?panels=p14067.9877018&index=0]

This is not a contradiction. All of heaven is represented by the Father, the Son, and the Holy Spirit, and in this quote, she specifically mentioned \egwinline{God and Christ and \textbf{the heavenly angels}.} There is a close connection between the workings of the Holy Spirit and the ministry of angels. The Inspiration testifies:

\egw{A measure of \textbf{the Spirit} is given to every man to profit withal. \textbf{Through the ministry of the angels \underline{the Holy Spirit is enabled} to work upon the mind and heart of the human agent}, and draw him to Christ who has paid the ransom money for his soul, that the sinner may be rescued from the slavery of sin and Satan.}[8LtMs, Lt 71, 1893, par. 10][https://egwwritings.org/read?panels=p14058.6086016&index=0]

This angelic ministry is one of the elements in the baptismal pledge of Matthew 28:19. When Ellen White said, \egwinline{\textbf{The heavenly powers} have \textbf{pledged themselves} to minister to human agents…,} she was referring to the holy angels. The connection between the Holy Spirit and the holy angels is beyond the scope of this book, but you can explore this topic further in the sequel, \textit{Rediscovering the Pillar}\footnote{Download for free: \href{https://forgottenpillar.com/book/rediscovering-the-pillar}{https://forgottenpillar.com/book/rediscovering-the-pillar}}, in the section on the Holy Spirit\footnote{Also, see the study on the angles \href{https://notefp.link/angels}{https://notefp.link/angels}}.