\chapter{The patchwork theories - Lt253-1903}

\egw{Dear Brother,—}

\egwnogap{\textbf{I must tell you that your ideas in regard to some things \underline{have been decidedly wrong}.} I would that you could see your errors. \textbf{The book Living Temple \underline{is not to be patched up}, a few changes made in it, and then advertised and praised as a valuable production}. It would be better to present the physiological parts in another book under another title. \textbf{When you wrote that book}, \textbf{you were not under the inspiration of God}. There was by your side the one who inspired Adam to look at God in a false light. Your whole heart needs to be changed, thoroughly and entirely cleansed.}[Lt253-1903.1; 1903][https://egwwritings.org/?ref=en\_Lt253-1903.1&para=9980.7]

\egwnogap{\textbf{My brother, do not allow yourself to be alienated from your ministering brethren who tell you of your dangers. Those who faithfully and frankly tell you of your errors are your best friends.} I am sorry, very sorry, for your medical associates. They have been unfaithful to God and untrue to you in failing to tell you kindly but firmly where you were not working righteously.}[Lt253-1903.2; 1903][https://egwwritings.org/?ref=en\_Lt253-1903.2&para=9980.8]

\egwnogap{There are many things that you must overcome before you can be saved. In the heart that is not led by God, there is a something that leads it to desire to be sustained in its wrong course. The men who faithfully tell you the truth, pointing out your mistakes, you have regarded as your enemies. But often they are your best friends and, in telling you wherein you were walking in strange paths, were doing a very disagreeable duty. The Lord’s servants are not to flatter your pride; they are not to stand silent, fearing to say, ‘Why do ye thus?’ They are faithfully to warn you of your danger.}[Lt253-1903.3; 1903][https://egwwritings.org/?ref=en\_Lt253-1903.3&para=9980.9]

\egwnogap{\textbf{My husband, Elder Joseph Bates, Father Pierce, Elder Edson, and many others who were keen, noble, and true were among those who, after the passing of the time in 1844, searched for truth}. \textbf{At our important meetings, these men would meet together and search for the truth as for hidden treasure}. I met with them, and we studied and prayed earnestly; for we felt that we must learn God’s truth. Often we remained together until late at night, and sometimes through the entire night, praying for light and studying the Word. As we fasted and prayed, great power came upon us. But I could not understand the reasoning of the brethren. My mind was locked, as it were, and I could not comprehend what we were studying. Then the Spirit of God would come upon me, I would be taken off in vision, and a clear explanation of the passages we had been studying would be given me with instruction as to the position we were to take regarding truth and duty. Again and again this happened. \textbf{A line of truth extending from that time to the time when we shall enter the city of God was plainly marked out before me}, and I gave my brethren and sisters the instruction that the Lord had given me. They knew that when not in vision, I could not understand these matters, and they accepted as light direct from heaven the revelations given me. \textbf{Thus the leading points of our faith as we hold them today were firmly established}. \textbf{\underline{Point after point} was clearly defined, and all the brethren came into harmony}.}[Lt253-1903.4; 1903][https://egwwritings.org/?ref=en\_Lt253-1903.4]

\egwnogap{\textbf{The whole company of believers were united in the truth}. \textbf{There were those who came in with strange doctrines, but we were never afraid to meet them. Our experience was wonderfully established by the revelations of the Holy Spirit}.}[Lt253-1903.5; 1903][https://egwwritings.org/?ref=en\_Lt253-1903.5&para=9980.11]

\egwnogap{For two or three years my mind continued to be locked to the Scriptures. In 1846 I was married to Elder James White. It was some time after my second son was born that we were in great perplexity regarding certain points of doctrine. I was praying to the Lord to unlock my mind, that I might understand His Word. Suddenly I seemed to be enshrouded in clear, beautiful light, and ever since, \textbf{the Scriptures have been an open book to me}.}[Lt253-1903.6; 1903][https://egwwritings.org/?ref=en\_Lt253-1903.6]

\egwnogap{I was at that time in Paris, Maine. Old Father Andrews was very sick. For some time he had been a great sufferer from inflammatory rheumatism. He could not move without intense pain. We prayed for him. I laid my hands on his head, and said, “Father Andrews, the Lord Jesus maketh thee whole.” He was healed instantly. He got up and walked about the room, praising God, and saying, “I never saw it on this wise before. Angels of God are in this room.” The glory of God was revealed. \textbf{Light seemed to shine all through the house, and an angel’s hand was laid upon my head. From that time to this I have been able to understand the Word of God.}}[Lt253-1903.7; 1903][https://egwwritings.org/?ref=en\_Lt253-1903.7&para=9980.13]

\egwnogap{\textbf{After the passing of the time, we were opposed and cruelly falsified. Erroneous theories were pressed in upon us by men and women who had gone into fanaticism}. I was directed to go to the places where these people were advocating these erroneous theories, and as I went, the power of the Spirit was wonderfully displayed in rebuking the errors that were creeping in. \textbf{\underline{Satan himself, in the person of a man}, was working to make of no effect my testimony regarding the position that we now know to be substantiated by Scripture.}}[Lt253-1903.8; 1903][https://egwwritings.org/?ref=en\_Lt253-1903.8&para=9980.14]

\egwnogap{\textbf{Just such theories as you have presented in Living Temple were presented then}. \textbf{These subtle, deceiving sophistries have again and again sought to find place amongst us. \underline{But I have ever had the same testimony to bear which I now bear regarding the personality of God}}.}[Lt253-1903.9; 1903][https://egwwritings.org/?ref=en\_Lt253-1903.9&para=9980.15]

\egwnogap{In (Early Writings, 60, 66, 67)\footnote{It appears that the pages are incorrect. The mentioned paragraphs can be found in Early Writings on pages \href{https://egwwritings.org/read?panels=p28.462&index=0}{70.2}, \href{https://egwwritings.org/read?panels=p28.490&index=0}{77}, and \href{https://egwwritings.org/read?panels=p28.390&index=0}{54.2}.}, are the following statements:}[Lt253-1903.10; 1903][https://egwwritings.org/?ref=en\_Lt253-1903.10&para=9980.16]

\egwnogap{‘May 14, 1851, I saw the beauty and loveliness of Jesus. As I beheld His glory, the thought did not occur to me that I should ever be separated from His presence. \textbf{I saw a light coming from the glory that encircled the Father}, and as it approached near to me, my body shook and trembled like a leaf. I thought that if it should come near me, I would be struck out of existence; but the light passed me. \textbf{Then could I have some sense of the great and terrible \underline{God} with whom we have to do}.’}[Lt253-1903.11; 1903][https://egwwritings.org/?ref=en\_Lt253-1903.11&para=9980.17]

\egwnogap{‘I have often seen \textbf{the lovely Jesus, that He is a person}. \textbf{I asked Him if His Father was a person, and had \underline{a form} like Himself}. Said Jesus, ‘\textbf{I am the express image of My Father’s person!}’ [Hebrews 1:3.]}[Lt253-1903.12; 1903][https://egwwritings.org/?ref=en\_Lt253-1903.12&para=9980.18]

\egwnogap{‘\textbf{I have often seen that the spiritual view took away all the glory of heaven, and that in many minds the throne of David and the lovely person of Jesus have been burned up in the fire of spiritualism}. I have seen that some who have been deceived and led into this error, will be brought out into the light of truth, \textbf{but it will be almost impossible for them to get entirely rid of the deceptive power of spiritualism. Such should make thorough work in confessing their errors, and leaving them forever}.’}[Lt253-1903.13; 1903][https://egwwritings.org/?ref=en\_Lt253-1903.13&para=9980.19]

\egwnogap{\textbf{There is a strain of spiritualism \underline{coming in} among our people, and \underline{it will undermine the faith} of those who give place to it, leading them to give heed to seducing spirits and doctrines of devils}. Errors will be presented in a pleasing and flattering manner. The enemy desires to divert the minds of our brethren and sisters from the work of preparing a people to stand in these last days.}[Lt253-1903.14; 1903][https://egwwritings.org/?ref=en\_Lt253-1903.14&para=9980.21]

\egwnogap{I am instructed to warn our brethren and sisters \textbf{not to discuss the nature of our God}. Many of the curious who attempted to open the ark of the testament, to see what was inside, were punished for their presumption. \textbf{We are not to say that the Lord God of heaven is in a leaf, or in a tree; for He is not there. \underline{He sitteth upon His throne in the heavens}.}}[Lt253-1903.15; 1903][https://egwwritings.org/?ref=en\_Lt253-1903.15&para=9980.22]

\egwnogap{The work of the Creator as seen in nature reveals His power. But nature is not above God, nor is God in nature as some represent Him to be. God made the world, but the world is not God; it is but the work of His hands. \textbf{Nature reveals the work of a positive, \underline{personal God}, showing that God is, and that He is a rewarder of those who diligently seek Him}.}[Lt253-1903.16, 1903][https://egwwritings.org/?ref=en\_Lt253-1903.16&para=9980.23]

\egwnogap{I could say much regarding the sanctuary; the ark containing the law of God; the cover of the ark, which is the mercy seat; the angels at either end of the ark; and other things connected with the heavenly sanctuary and with the great day of atonement. I could say much regarding the mysteries of heaven; but my lips are closed. I have no inclination to try to describe them.}[Lt253-1903.17; 1903][https://egwwritings.org/?ref=en\_Lt253-1903.17&para=9980.25]

\egwnogap{\textbf{I would not dare to speak of God as you have spoken of Him}. He is high and lifted up, and His glory fills the heavens. “The voice of the Lord is mighty; it shaketh the cedars of Lebanon. \textbf{The Lord is in His holy temple}; let all the earth keep silence before Him.” [See Psalm 29:5; Habakkuk 2:20.]}[Lt253-1903.18; 1903][https://egwwritings.org/?ref=en\_Lt253-1903.18&para=9980.26]

\egwnogap{\textbf{My brother, when you are tempted to speak of God, \underline{where He is, or what He is}, remember that on this point silence is eloquence}. Take off your shoes from off your feet; for the ground on which you are placing your careless, unsanctified feet is holy ground.}[Lt253-1903.19; 1903][https://egwwritings.org/?ref=en\_Lt253-1903.19]

\egwnogap{\textbf{I am instructed to say that there is nothing in the Word of God to substantiate your spiritualistic theories. Will you not renounce these theories at once? Upon them your mind has been dwelling for a long time, but they have had no sanctifying, refining, ennobling influence upon your life. The Lord has no use for these theories, and He would not have His people vindicate or propagate them.}}[Lt253-1903.20; 1903][https://egwwritings.org/?ref=en\_Lt253-1903.20&para=9980.28]

\egwnogap{\textbf{The Father, the omniscient One, created the world \underline{through} Christ Jesus}. Christ is the light of the world, the way to eternal life. He, the anointed One, God gave to make an atonement for the sins of the world. You need to understand that unless you believe \textbf{in that atonement}, and know that you are bought with the price of the blood of \textbf{the only begotten Son of God}, you will assuredly be bound up with the wicked one. \textbf{If you continue to cherish the theories that you have been cherishing, you will be left to become the sport of Satan’s temptations}. He is playing the game of life for your soul. Remain for a little longer linked up with him, and be assured that you will lose your soul.}[Lt253-1903.21; 1903][https://egwwritings.org/?ref=en\_Lt253-1903.21&para=9980.29]

\egwnogap{By declaring that our institutions are undenominational, you have put our people and our work in a false position. You have been led over a terrible path, the dangers of which you have not known, but may sometime see. It is not yet too late for wrongs to be righted. There is hope for you. \textbf{You have followed the enemy step by step, striving to look into mysteries too high and holy for your comprehension}. \textbf{Then in your teaching the Holy One has been brought down to man’s \underline{scientific, spiritualistic ideas}}. You have been walking in crooked paths. You have lost the moral image of God. But there is hope for you. You may still turn your feet into the right path. Will you not now make straight paths for your feet, lest the lame be turned out of the way? Will you now refuse to sow one more seed of skepticism and sophistry in the minds of others? Will you now come to Christ and be healed?}[Lt253-1903.22; 1903][https://egwwritings.org/?ref=en\_Lt253-1903.22]

\egwnogap{\textbf{I have hesitated and delayed about the sending out of that which the Spirit of the Lord has impelled me to write}. I did not want to be compelled to present the satanic influence of these sophistries. But unless there is a decided change, in yourself and your associates, I shall have to do this, to save others from following the path that you have been following. I shall have to obey the command given me of God, “\textbf{Meet it}.” This is the only thing that I can do.}[Lt253-1903.23; 1903][https://egwwritings.org/?ref=en\_Lt253-1903.23&para=9980.31]

\egwnogap{I present to you the things that the Lord has presented to me. There is a great work to be done. We are to take hold of the work understandingly, praying, believing, and receiving the Holy Spirit. Thus only can we do the work given us. \textbf{I am required by God to bear testimony against Living Temple}. Whatever your associates may say concerning this book,\textbf{ I take the position now and forever that it is a snare}. \textbf{No union will be formed by our people as a whole upon the \underline{theories} that you have begun to present in that book}. \textbf{You may regard this as forever decided}. \textbf{As a people we shall stand firm \underline{on the platform that has withstood test and trial}. We shall hold to the \underline{sure pillars of our faith}. \underline{The principles of truth} that God has revealed to us are our only foundation. They have made us what we are. These new, fanciful theories are fascinating and misleading. They endanger the eternal interests of the soul. The Scriptures do not sustain them}. Clothed with the Christian armor, shod with the preparation of the gospel of peace, we shall stand \textbf{firm against these misleading theories}. You may turn and wrest the Word of God to your own destruction, but I entreat you not to do this.}[Lt253-1903.24; 1903][https://egwwritings.org/?ref=en\_Lt253-1903.24&para=9980.32]

\egwnogap{\textbf{Heaven is not a vapor. It is a place}. \textbf{Christ has gone to prepare mansions for those who love Him}, those who, in obedience to His commands, come out from the world and are separate. The principles of heaven must be brought into our experience, that we may be distinguished from the world. \textbf{There must be a marked contrast between us and the world; for we are God’s denominated people}.}[Lt253-1903.25; 1903][https://egwwritings.org/?ref=en\_Lt253-1903.25&para=9980.33]

\egwnogap{The Lord has given you an opportunity to make things right. \textbf{I rejoice that you have made a beginning. Do not think that we have no right to try to correct your errors and the results of these errors. As long as God gives me breath, and commissions me to use pen and voice in beating back this evil thing that has come in among us, I shall act my part in the warfare. Ever since I was seventeen years old, I have had to fight this battle against false theories, in defense of the truth}. \textbf{The history of our past experience is indelibly fixed in my mind, and I am determined that \underline{no theories of the order that you have been accepting} shall come into our ranks}. If you refuse to change, and labor to lead your associates after you, and they venture to follow your leading, the accountability rests with you and with them, not on my soul.}[Lt253-1903.26, 1903][https://egwwritings.org/?ref=en\_Lt253-1903.26&para=9980.34]

\egwnogap{\textbf{I speak decidedly, in order that you may know, that unless there is a decided change in you, there can be no hope of a union between you and those who are holding the beginning of their confidence firm unto the end.} You have made the division. \textbf{\underline{We must stand firm for the truths that the Lord has given us as the pillars of our faith}}.}[Lt253-1903.27; 1903][https://egwwritings.org/?ref=en\_Lt253-1903.27&para=9980.35]

\egwnogap{I entreat you to turn to the Lord with full purpose of heart, before it is forever too late. Separate yourself from the influences which have separated you from your brethren who are engaged in the gospel ministry and from the people whom God is leading. \textbf{\underline{Patchwork theories} cannot be accepted by those who are loyal to the faith and to \underline{the principles} that have withstood all the opposition of satanic influences}.}[Lt253-1903.28; 1903][https://egwwritings.org/?ref=en\_Lt253-1903.28&para=9980.36]

\egwnogap{If you will empty yourself of all that has separated you from Christ, and receive the Saviour into your heart, you will be transformed in character. Lay off responsibilities for a time, and go away somewhere with a few of your brethren, and with them search the Scriptures. Humble your heart before the Lord, and make thorough work for repentance. \textbf{The religion of Christ is the spiritual leaven that is to be introduced into the heart. This changes the life and character}. This religion is a heavenly principle, seen in the Christian’s life and conversation. It is revealed in Christian purity. The love of Christ is seen in the tenderness and grace of sanctified humanity. It is by the Word made flesh that we are saved. Our redemption was wrought out, \textbf{not by the Son of God’s remaining in heaven, but by the Son of God’s becoming incarnate—taking humanity upon Him and coming to this world}. Thus eternal life was brought to us. That which authority, commands, and promises could not do, God did by coming to this world in the likeness of sinful flesh.}[Lt253-1903.29; 1903][https://egwwritings.org/?ref=en\_Lt253-1903.29&para=9980.37]

\egwnogap{Christ came to the earth to live as a man among men, not to be spoiled by human frailty, but to place in the minds of men principles of truth that could never be obliterated, because they are eternally true. He came to bring a new life to fallen human beings—an excellence that could not be stained or deteriorated by sin.}[Lt253-1903.30; 1903][https://egwwritings.org/?ref=en\_Lt253-1903.30&para=9980.38]

\egwnogap{\textbf{My brother, I must tell you that you have little realization of whither your feet have been tending}. You have been binding yourself up with those who belong to the army of the great apostate. \textbf{Your mind has been as dark as Egypt}. \textbf{If you will fall on the Rock and be broken}, Christ will accept you. But you have been standing on the enemy’s ground, doing his work. \textbf{The religious world is fast going over the same road that you have been following. If you continue to follow this road, you will have plenty of company. But what will the end be?}}[Lt253-1903.31; 1903][https://egwwritings.org/?ref=en\_Lt253-1903.31]

\egwnogap{So long have you been walking in darkness, so long have you followed your own way, that you may be strongly tempted to resist this appeal that I make. If it were not that your \textbf{eternal interests are involved}, I would not speak to you on this subject. It would seem that I have written enough, that there is no need of my urging this subject upon you further. \textbf{But I tell you in truth that I clearly understand what I am doing}. Sufficient light has been given you. But for several years you have not heeded this light. If you had wished to know what the Lord has said, you could have known; \textbf{for you have the books that have been written under the guidance of His Spirit}. You have had all the directions that could be asked for to point out the right way. Direct light has been sent you. But you have looked upon this as of less importance than your own plans and devisings. If you had heeded the testimonies sent you, Living Temple would never have been written.}[Lt253-1903.32; 1903][https://egwwritings.org/?ref=en\_Lt253-1903.32&para=9980.40]

\egwnogap{Will you not make a thorough, determined, Christlike effort to break the spell that Satan has cast over you? He has had great power over your mind and has swayed you in wrong lines. He thinks that he can hold you now. Will you not defeat and disappoint him?}[Lt253-1903.33; 1903][https://egwwritings.org/?ref=en\_Lt253-1903.33&para=9980.41]

\egwnogap{I write to you as I would to a son. Break away from the enemy—the accuser of the brethren. Say to him, “Get thee behind me Satan. I have committed a grievous sin in heeding your suggestions. I will no longer listen to them.” I beg of you, for your soul’s sake, to resist the tempter, that he may flee from you. Draw near to God, and He will draw near to you. \textbf{You will lose heaven unless you fall on the Rock and are broken}.}[Lt253-1903.34; 1903][https://egwwritings.org/?ref=en\_Lt253-1903.34&para=9980.42]

Many things in this letter to Dr. Kellogg go without being said, yet are explained when the context is understood. Ellen White read the letter from Brother Daniells expressing how Dr. Kellogg wanted to revise the Living Temple because he\others{had been thinking the matter over, and began to see that he had made a slight mistake in \textbf{expressing }his views}, and\others{that within a short time \textbf{he had come to believe in the trinity} and could now see pretty clearly where all the difficulty was, and believed that he could clear the matter up satisfactorily}. Kellogg confessed,\others{that he now believed \textbf{in God the Father, God the Son, and God the Holy Ghost}}. In answer to that, Sister White personally wrote to him:\egwinline{The book Living Temple \textbf{is not to be patched up}, a few changes made in it, and then advertised and praised as a valuable production}. How did Kellogg want to patch up his book? According to A. G. Daniells’ testimony, he thought to change a few expressions by explicitly stating his trinitarian sentiment. But the expression of the views was not the real problem—it was the views themselves. Sister White did not spare rebuking him for his views of God, which were \textit{trinitarian} views. She told him that she is\egwinline{\textbf{determined that \underline{no theories of the order that you have been accepting} shall come into our ranks}}. This is a very strong statement. Could it be that, since Kellogg confessed that he was accepting the Trinity doctrine, Sister White was also including it in her statement? It seems unthinkable because this doctrine is in our ranks today. But her statement actually pinpoints the Trinity when she said:\egwinline{\textbf{Patchwork theories} cannot be accepted by those who are loyal \textbf{to the faith and to the principles} that have withstood all the opposition of satanic influences}. Kellogg wanted to patch up “\textit{Living Temple}" by explicitly mentioning the Trinity doctrine. Why was Sister White determined to keep this doctrine out of our ranks, yet it is in our ranks today? It is fair to point out that the Trinity was not part of Seventh-day Adventist faith in her time and it came into our ranks later. Today, many argue that it was because of her works that the Trinity is a part of our beliefs, but Ellen White’s reaction, and her answer to Kellogg’s belief in it, showcases how she dealt with such doctrine. What can we learn from that?

Taken in its context, this letter sheds new light on Kellogg’s controversy and demonstrates how we should deal with the Trinity doctrine. The first thing we question is why Sister White never used the word “Trinity” in her writings, even when she was directly dealing with this doctrine? Elsewhere, she answers: 

\egw{I was cautioned not to enter into controversy \textbf{regarding the question} that \textbf{\underline{will come up}} over \textbf{these things, because controversy \underline{might lead men to resort to subterfuges, and their minds would be led away from the truth of the Word of God to assumption and guesswork}}. \textbf{The more that fanciful theories are discussed, the \underline{less men will know of God and of the truth that sanctifies the soul}}.}[Lt232-1903.41; 1903][https://egwwritings.org/?ref=en\_Lt323-1903.41]

This is a very important lesson and principle that Sister White is teaching us here. When the controversy over Kellogg’s theories arose, she did not venture into the theories themselves, because this would lead the minds of men away from the truth of the Word of God to assumption and guesswork. Rather, she led the minds of men into the truth, which sanctifies the soul. She led by example, evident here in her letter to Dr. Kellogg. This truth that she led the minds of men to, was the truth on the \emcap{personality of God}. She rebuked Kellogg for his theories but, very importantly, we properly identify these theories by their context and her implicit expression of them. 

We see that she made a contrast between the Trinity and the \emcap{personality of God}. She made a contrast between the old principles of our faith and the new theories. First, she drew our minds back to the beginning of our spiritual heritage,\egwinline{after the passing of the time in 1844}, when her husband James White, Joseph Bates, Father Pierce, Elder Edson, and many others who were keen, noble, and true, searched for truth. She pointed back to the wonderful and mighty experiences of how the leading points of our faith, held in 1903, were firmly established. \egwinline{\textbf{Thus \underline{the leading points of our faith}} as we hold them today were firmly established.} \egwinline{\textbf{\underline{Point after point} was clearly defined, and all the brethren came into harmony}.} \egwinline{\textbf{The whole company of believers were united in the truth}}. Obviously, from the context of chapter 10 of the Special Testimonies, we know that these experiences explain \egwinline{\textbf{how firmly the foundation of our faith has been laid}}[SpTB02 56.4; 1904][https://egwwritings.org/?ref=en\_SpTB02.56.4\&para=417.288]. This foundation is expressed in the \emcap{Fundamental Principles}\footnote{\href{https://static1.squarespace.com/static/554c4998e4b04e89ea0c4073/t/59d17e24c027d84167e17617/1506901547915/SDA-YB1905+\%28P.+188-192\%29.pdf}{Yearbook Of Seventh-day Adventist denomination 1905, p. 188-192}}. This foundation is the truth which,\egwinline{\textbf{\underline{point by point}}, \textbf{has been sought out by prayerful study, and testified to by the miracle-working power of the Lord}}. God \egwinline{\textbf{calls upon us to \underline{hold firmly}, with the grip of faith, to \underline{the fundamental principles} that are \underline{based upon unquestionable authority}}.}[SpTB02 59.1; 1904][https://egwwritings.org/?ref=en\_SpTB02.59.1] In light of these experiences and the truth expressed in the \emcap{fundamental principles}, \egwinline{\textbf{\underline{Patchwork theories} cannot be accepted by those who are loyal \underline{to the faith} and \underline{to the principles} that have withstood all the opposition of satanic influences}}[Lt253-1903.28; 1903][https://egwwritings.org/?ref=en\_Lt253-1903.28]. From the historical record of these brethren who were keen, noble and true, we have evidence that they, too, have contrasted the Trinity doctrine with the truth on the \emcap{personality of God}. James White, in the Review and Herald article, listed \others{some of the popular fables of the age}, saying: \others{Here we might mention \textbf{the Trinity, which \underline{does away the personality of God, and of his Son Jesus Christ}}}[James White, Review \& Herald, December 11, 1855, p. 85.15][http://documents.adventistarchives.org/Periodicals/RH/RH18551211-V07-11.pdf]. J. N. Andrews said, \others{\textbf{The doctrine of the Trinity which was established in the church by the council of Nicea, A. D. 325}. \textbf{This doctrine \underline{destroys the personality of God, and his Son Jesus Christ our Lord}}...}[J. N. Andrews, Review \& Herald, March 6, 1855, p. 185][http://documents.adventistarchives.org/Periodicals/RH/RH18550306-V06-24.pdf] J. B. Frisbie, in his article “\textit{Seventh-day Sabbath not abolished}", compares the Sabbath God to the Sunday god; he describes the Sabbath God in light of the \emcap{personality of God} expressed in the first point of the \emcap{Fundamental Principles}. The Sunday god is described by the \others{unity of this God-head, there are three persons of one substance, power and eternity; the Father, the Son, and the Holy Ghost}[J. B. Frisbie, Review \& Herald March 7, 1854. p. 50][http://documents.adventistarchives.org/Periodicals/RH/RH18540307-V05-07.pdf]. He explained how the doctrine on the \emcap{personality of God} stands in conflict with the doctrine of Trinity, in the same way the Holy Sabbath stands in conflict with pagan Sunday worship. Also, brother J. N. Loughborough wrote the objections to the Trinity doctrine in the Adventist Review and Sabbath Herald\footnote{\href{https://adventistdigitallibrary.org/adl-349160/advent-review-and-sabbath-herald-november-5-1861}{J. N. Loughborough, November 5, 1861, Review \& Herald, vol. 18, p. 184, par. 1-11}}. In the other publication of the Review and Herald, he published the article “\textit{Is God a person?}”, explaining the position of Seventh-day Adventist belief on the \emcap{personality of God}, expressed in the first point of the \emcap{Fundamental Principles}\footnote{\href{http://documents.adventistarchives.org/Periodicals/RH/RH18550918-V07-06.pdf}{J. N. Loughborough, September 18. 1855, Review \& Herald, vol. 7, p. 6.}}. James White was also explaining the same position in his multiple print pamphlet, “\textit{The Personality of God}”\footnote{\href{https://egwwritings.org/?ref=en_PERGO.1.1&para=1471.3}{J. White, The Personality of God, June 18. 1861.}}. These are just a few examples where the Adventist pioneers explained the position on the \emcap{personality of God} expressed by the first point of the \emcap{fundamental principles}.

Sister White rebuked Kellogg:\egwinline{\textbf{But I tell you in truth that I clearly understand what I am doing}. \textbf{Sufficient light has been given you}. But for several years you have not heeded this light. If you had wished to know what the Lord has said, you could have known; \textbf{for \underline{you have the books} that have been written under the guidance of His Spirit}. You have had all the directions that could be asked for to point out the right way. Direct light has been sent you. But you have looked upon this as of less importance than your own plans and devisings. If you had heeded the testimonies sent you, Living Temple would never have been written.}[Lt253-1903.32; 1903][https://egwwritings.org/?ref=en\_Lt253-1903.32] The core issue of Dr. Kellogg’s controversy was \egwinline{the personality of God and where His presence is}[SpTB02 51.3; 1904][https://egwwritings.org/?ref=en\_SpTB02.51.3&para=417.262]. Dr. Kellogg had access to the pioneer writings, books and the church's \emcap{Fundamental Principles} that were testified to by the miracle working power of the Holy Spirit. 

Sister White recalled the experiences of how the \textit{leading points of our faith}, as were held in former times, were firmly established.\egwinline{\textbf{\underline{Point after point} was clearly defined, and all the brethren came into harmony}}[Lt253-1903.4; 1903][https://egwwritings.org/?ref=en\_Lt253-1903.4]. These leading points were the \emcap{Fundamental Principles}, of which the \emcap{personality of God} was one. This point, and Sister White’s testimony of it, remained the same during the course of her life.  She said\egwinline{\textbf{\underline{I have ever had the same testimony to bear which I now bear regarding the personality of God}}}[Lt253-1903.9; 1903][https://egwwritings.org/?ref=en\_Lt253-1903.9]. From Early Writings, she then quoted her visions of the Heavenly reality. She recalled how she had had the privilege to be in the presence of God, how God, encircled by the light of His glory, passed by her side. She did not see God from the light He was encircled by; she was afraid of Him, thinking that if He were to approach her she\egwinline{would be struck out of existence}. Then she saw\egwinline{\textbf{the lovely Jesus, that He is a person}. \textbf{I asked Him if His Father was a person, and had \underline{a form like} Himself}. Said Jesus, ‘\textbf{I am the express image of My Father’s person!}’}[Lt253-1903.12; 1903][https://egwwritings.org/?ref=en\_Lt253-1903.12]. The question she had was: \textit{is God a person, having a form like Jesus}? The answer was affirmative—with a strong biblical foundation. Her visions were not the source of the truth on the \emcap{personality of God}; rather, they confirmed the truth the pioneers had discovered through diligent study of God’s word.

Therefore, their final conclusion on the \emcap{personality of God} was,\others{That there is \textbf{one God}, \textbf{a personal, spiritual \underline{being}}, \textbf{the creator of all things}, omnipotent, omniscient, and eternal, infinite in wisdom, holiness, justice, goodness, truth, and mercy; unchangeable, and \textbf{everywhere present by his representative, the Holy Spirit}. Ps. 139:7; That there is one Lord Jesus Christ, \textbf{the Son of the Eternal Father, the one by whom he created all things, and by whom they do consist} …and as the closing portion of his work as priest, before he takes his throne as king, he will make \textbf{the great atonement} for the sins of all such, and their sins will then be blotted out (Acts 3:19) and borne away from the sanctuary, as shown in the service of the Levitical priesthood, which foreshadowed and prefigured the ministry of our Lord in heaven. See Lev. 16; Heb. 8: 4, 5; 9: 6, 7; etc.}[The first, and part of the second, point of the Fundamental Principles, 1905.]

Ellen White reminded Dr. Kellogg on this point of the \emcap{fundamental principles} by stating:\egwinline{\textbf{The Father, the omniscient One, created the world \underline{through} Christ Jesus}. Christ is the light of the world, the way to eternal life. \textbf{He, the anointed One, God gave to make an atonement for the sins of the world}...}[Lt253-1903.21; 1903][https://egwwritings.org/?ref=en\_Lt253-1903.21]

The question on the \emcap{personality of God} deals with the quality or state of God being a person. The Adventist pioneers gave an answer to it and God approved it through the writings of Ellen White: God is a \textit{personal spiritual Being} and He is our heavenly Father. Where is His presence?\egwinline{\textbf{We are not to say that the Lord God of heaven is in a leaf, or in a tree; for He is not there. \underline{He sitteth upon His throne in the heavens}}.}[Lt253-1903.15; 1903][https://egwwritings.org/?ref=en\_Lt253-1903.15] \\
His presence is on the throne in heaven. \\
\egwinline{\textbf{Heaven is not a vapor. It is a place}. \textbf{Christ has gone to prepare mansions for those who love Him}, those who, in obedience to His commands, come out from the world and are separate...}[EGW, Lt253-1903.25; 1903][https://egwwritings.org/?ref=en\_Lt253-1903.25]. \\
“...\egwinline{‘The voice of the Lord is mighty; it shaketh the cedars of Lebanon. \textbf{The Lord is in His holy temple}; let all the earth keep silence before Him.’ [See Psalm 29:5; Habakkuk 2:20.]}[Lt253-1903.18; 1903][https://egwwritings.org/?ref=en\_Lt253-1903.18]

According to Adventist pioneers and Sister White, our heavenly Father is one God. He is a personal Spiritual Being, present in heaven, on His throne. The throne of heaven is a real, physical throne, upon which sits a real Person (Being, having a form, just like Jesus)—our heavenly Father. That place is a real place; it is not a vapor, or any other spiritual view.

\egwinline{\textbf{I have often seen that the spiritual view took away all the glory of heaven, and that in many minds the throne of David and the lovely person of Jesus have been burned up in the fire of spiritualism}. I have seen that some who have been deceived and led into this error, will be brought out into the light of truth, \textbf{but it will be almost impossible for them to get entirely rid of the deceptive power of spiritualism. Such should make thorough work in confessing their errors, and leaving them forever}.}[Lt253-1903.13; 1903][https://egwwritings.org/?ref=en\_Lt253-1903.13]

The spiritual view of God’s person is an erroneous view. In the Bible we have testimonies of heaven, the heavenly throne, and God who is sitting upon it. If we accept these testimonies in their obvious meaning, then the Trinity doctrine cannot be sustained. The Bible and Spirit of Prophecy present one God in heaven, as a personal being, having a body and form just as Jesus has. This view is not in harmony with the doctrine of the Triune God, since it requires the Holy Spirit to be a Being\footnote{Please look at \hyperref[appendix:unauthenticated-reports]{the appendix} for more quotations which exclude the Holy Spirit to be a being, possessing physical body and form.}, having a body and form—this idea would compromise the Holy Spirit to be a means of the Father and Son by which They are everywhere present. In order to sustain the Trinity doctrine, the testimonies regarding the throne of God and of God’s person, need to be understood by some spiritual view. Here we have seen that Sister White contrasted the truth of the \emcap{personality of God} with the doctrine of Trinity. She contrasted the doctrine of Trinity with the first two points of the \emcap{Fundamental Principles}, which were the results of our pioneers studying the Word of God. Referring to the pioneers and the \emcap{Fundamental Principles}, she said: \egwinline{\textbf{\underline{Patchwork theories} cannot be accepted by those who are \underline{loyal to the faith and to the principles} that have withstood all the opposition of satanic influences.}}[Lt253-1903.28; 1903][https://egwwritings.org/?ref=en\_Lt253-1903.28]

The conclusion is straightforward and simple. Those who are loyal to the faith, and to the principles received in the beginning of the work, cannot accept patchwork theories. Put into context, the patchwork theory, which is the Trinity doctrine, cannot be accepted by those who are holding fast \egwinline{\textbf{to \underline{the fundamental principles} that are \underline{based upon unquestionable authority}}}[SpTB02 59.1; 1904][https://egwwritings.org/?ref=en\_SpTB02.59.1]. This conclusion leads us back to our first proposed test of the foundation of our faith.