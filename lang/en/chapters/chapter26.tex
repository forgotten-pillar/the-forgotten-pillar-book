\chapter{The steps to Omega}

In our study so far, we have seen evidence that Kellogg’s controversy was connected to the Trinity doctrine and the \emcap{personality of God} expressed in the first point of the \emcap{Fundamental Principles}. Unfortunately, today we do not stand on that foundation regarding the \emcap{personality of God}; we have built another foundation that has changed the truth on the \emcap{personality of God} to a mysterious Triune God. Sister White was clearly against this reorganization and she prophesied that in the closing of His work, God will rehearse the history of the Advent movement and re-establish every pillar of our faith that was held in the beginning.

\egw{\textbf{\underline{The Lord has declared that the history of the past shall be rehearsed as we enter upon the closing work}. \underline{Every truth} that He has given for these last days is to be proclaimed to the world. \underline{Every pillar} that He has established \underline{is to be strengthened}. We cannot now step off the foundation that God has established. We cannot now enter into any new organization; for this would mean apostasy from the truth}.}[Ms129-1905.6; 1905][https://egwwritings.org/?ref=en\_Ms129-1905.6&para=9797.13]

Comparing the \emcap{Fundamental Principles} with the current Fundamental Beliefs of Seventh-day Adventists, we see that we have entered into a new organization. God’s warning, given through Sister White, to re-establish all pillars of our faith in these last days, is becoming imperative. As we traced the Trinity doctrine from Kellogg's controversy, we came across Ellen White’s warnings against alpha and omega apostasy, which will enter into our church.

\egw{\textbf{‘Living Temple’ contains the alpha of these theories. I knew that \underline{the omega would follow in a little while}; and I trembled for our people}. I knew that \textbf{I must warn our brethren and sisters not to enter into controversy \underline{over the presence and personality of God}. The statements made in “Living Temple” \underline{in regard to this point are incorrect}. }The scripture used to substantiate the doctrine there set forth, is scripture misapplied.}[SpTB02 53.2; 1904][https://egwwritings.org/?ref=en\_SpTB02.53.2&para=417.271]

In the context of Seventh-day Adventist reorganization, we identify several steps that were necessary to accomplish this reorganization and are necessary to uphold it. 

\subsection*{Step 1: Deny the Fundamental Principles to be the foundation of our faith and the official, and accurate, representation of Seventh-day Adventist beliefs}

The first step necessary is to hide the original foundation of our faith by unlinking it with the \emcap{Fundamental Principles}.

\egw{\textbf{As a people, we are to \underline{stand firm on the platform of eternal truth} that has withstood test and trial. We are to \underline{hold to the sure pillars of our faith}. \underline{The principles of truth} that God has revealed to us \underline{are our only true foundation}. They have made us what we are. The lapse of time has not lessened their value. \underline{It is the constant effort of the enemy to remove these truths from their setting}, and to put in their place \underline{spurious theories}. He \underline{will bring in} everything that he possibly can to carry out his deceptive designs.}}[SpTB02 51.2; 1904][https://egwwritings.org/?ref=en\_SpTB02.51.2&para=417.261]

\egw{\textbf{Messages of every order and kind have been urged upon Seventh-day Adventists, to take the place of the truth which, \underline{point by point}, has been sought out by prayerful study, and testified to by the miracle-working power of the Lord}. \textbf{But \underline{the way-marks} \underline{which have made us what we are}, \underline{are to be preserved}, and they \underline{will be preserved}, as God has signified through His word and the testimony of His Spirit}. \textbf{He calls upon us to \underline{hold firmly}, with the grip of faith, to \underline{the fundamental principles} that are \underline{based upon unquestionable authority}}.}[SpTB02 59.1; 1904][https://egwwritings.org/?ref=en\_SpTB02.59.1&para=417.299]

The \emcap{Fundamental Principles} were the truths God revealed to the pioneers after the passing of time in 1844. We have seen the testimonies of our pioneers, including Ellen White, regarding the first point of the \emcap{Fundamental Principles}. All of them were in harmony regarding these particular points of our faith. In 1863, Seventh-day Adventists organized themselves into a church, as an organized body. Since then, many were misrepresenting the position of the Seventh-day Adventist Church and the pioneers found it necessary to meet inquiries, \others{and sometimes to correct false statements circulated against} the church’s beliefs and practices. Consequently, in 1872, the pioneers issued the document called “\textit{A Declaration of the Fundamental Principles, Taught and Practiced by the Seventh-Day Adventists}”\footnote{“A Declaration of the Fundamental Principles, Taught and Practiced by the Seventh-Day Adventists (1872) : MVT : Free Download, Borrow, and Streaming : Internet Archive.” Internet Archive, 2025, \href{https://archive.org/details/ADeclarationOfTheFundamentalPrinciplesTaughtAndPracticedByThe}{archive.org/details/ADeclarationOfTheFundamentalPrinciplesTaughtAndPracticedByThe}. Accessed 3 Feb. 2025.}. This declaration presented the public with \others{a brief statement of what is, and has been, with great unanimity, held by}[The preface of the Fundamental Principles in 1872.] Seventh-day Adventists.

In the chapter “\hyperref[chap:authority]{The Authority of the Fundamental Principles}”, we discussed how pro-Trinitarian scholars have been compromising the authority of the \emcap{Fundamental Principles}, denying their true value in our Adventist history.

Pro-trinitarian scholars argue that this declaration was not what it claims to be—a declaration of the \emcap{fundamental principles}, taught and practiced by the Seventh-day Adventist Church. This declaration was a summary of the principal features of Adventist’s faith, and no point is really as problematic or objectionable as the first point, dealing with the \emcap{personality of God} and where His presence is. But the evidence in favor of the \emcap{Fundamental Principles}, especially to the first point, is overwhelming.

All of these claims are easily refuted by the fact that the \emcap{Fundamental Principles} have been regularly issued and reprinted over the course of the entire life of Sister White, until 1914. If they were mere private opinions of a few individuals, as claimed by scholars\footnote{Ministry Magazine “Our Declaration of Fundamental Beliefs”: January 1958, Roy Anderson, J. Arthur Buckwalter, Louise Kleuser, Earl Cleveland and Walter Schubert}, would they have been consistently reprinted over the course of 42 years, publicly claiming to represent the synopsis of Seventh-day Adventist faith? If they had been issued only once, we could deem it a conspiracy by some individuals to purposely misrepresent Seventh-day Adventist faith. On the contrary, the \emcap{Fundamental Principles} were regularly reprinted, and they truly represented the official Seventh-day Adventist faith and practice.

Another argument is that Sister White approved the \emcap{Fundamental Principles} in her writings by explicitly referring to them, and also by teaching the same truths taught in the \emcap{Fundamental Principles}. The works of our pioneers are also in harmony with the statements in this Declaration of the \emcap{Fundamental Principles}. Considering all of these facts, it is inevitable that this declaration was truthful in its claims. This document was indeed a declaration of the \emcap{Fundamental Principles}, taught and practiced by the Seventh-day Adventist Church, representing a public \others{synopsis of our faith}, \others{a brief statement of what is, and has been, with great unanimity, held by} Seventh-day Adventists.\footnote{The preface of the Fundamental Principles in 1872.} As such, it accurately represents the Seventh-day Adventist belief and practice, and represents the foundation of Seventh-day Adventist faith in the time of Ellen White.

Today, in defense of the Trinity doctrine, Adventist historians boldly claim that when our pioneers were studying Adventist truths such as the sanctuary, investigative judgment, the Sabbath and other doctrines, they \others{did not study the subject of the doctrine of God}. These Adventist historians falsely claim that the doctrine of God \others{was not the question that they dealt at that time}[Denis Kaiser. "From Antitrinitarianism to Trinitarianism: The Adventist story" and Panelist. The God We Worship: A Godhead Symposium. Central California Conference, Dinuba, CA. March 23-24, 2018.]. Following this false claim, they present historical data on how Adventist doctrine gradually moved toward Trinitarian understanding. The truth is, there are some instances early on\footnote{The earliest mention of the Trinity doctrine, in a positive sense, was when M.C. Wilcox reprinted a non-Adventist article by Samuel Spear in Signs of the Times, December 7th, 1891 and December 14th, 1891} when the Trinity doctrine is mentioned in a positive light in our literature. But when you consider the fact that the Adventist church did have its positive position on the subject of the doctrine of God, as it was expressed in the \emcap{Fundamental Principles}, these instances cannot be interpreted as progressiveness in understanding, but rather an intrusion of the Trinity doctrine into the Seventh-day Adventist Church.

It is easy to refute the claim that Adventist pioneers did not understand the doctrine of God. If they did not understand it, they would have failed to proclaim the first angel’s message. We discussed this point in detail in the chapter “\hyperref[chap:remebering-the-beginning]{Remembering the beginning}”. The Seventh-day Adventist movement was not a failure, but a God-led, prophetic movement.

\subsection*{Step 2: Ignore the warnings of building a new foundation}

When the \emcap{Fundamental Principles} are removed from the equation, many of Ellen White’s warnings fail to shine in their true light and their true meaning does not resonate with the reader.

We have cited many quotations where Sister White warned the church not to step off the \emcap{Fundamental Principles}. We dealt with them in the chapter “\hyperref[chap:apostasy]{The great apostasy is soon to be realized}”, but we will mention one of the most prominent quotations again.

\egw{\textbf{The enemy of souls has sought to bring in the supposition that a great reformation was to take place among Seventh-day Adventists, and that this reformation would \underline{consist in giving up the doctrines which stand as the pillars of our faith} and engaging in a process of reorganization}. Were this reformation to take place, what would result? \textbf{The principles of truth that God in His wisdom has given to the remnant church would be discarded. Our religion would be changed. \underline{The fundamental principles that have sustained the work for the last fifty years would be accounted as error}}. \textbf{A new organization would be established. Books of a new order would be written. A system of intellectual philosophy would be introduced}...}[Lt242-1903.13; 1903][https://egwwritings.org/?ref=en\_Lt242-1903.13&para=7767.20]

\egwnogap{Who has authority to begin such a movement? \textbf{We have our Bibles. We have our experience, attested to by the miraculous working of the Holy Spirit}. \textbf{We have a truth that admits of no compromise.} \textbf{\underline{Shall we not repudiate everything that is not in harmony with this truth}?}}[Lt242-1903.14; 1903][https://egwwritings.org/?ref=en\_Lt242-1903.14&para=7767.21]

\subsection*{Step 3: Deny that the personality of God was the pillar of our faith and a part of the foundation of our faith}

There is one Ellen White statement that apparently supports the claim that the \emcap{personality of God} was not a pillar of our faith. Another expression for “\textit{pillars of our faith}” is “\textit{landmarks}”. In the following quotations, Sister White lists several landmarks: the cleansing of the sanctuary, the three angels’ messages, the temple of God, the Sabbath and the non-immortality of the wicked.

\egw{The passing of the time in 1844 was a period of great events, opening to our astonished eyes \textbf{the cleansing of the sanctuary transpiring in heaven}, and having decided relation to God’s people upon the earth, [also] \textbf{the first and second angels’ messages and the third}, unfurling the banner on which was inscribed, “The commandments of God and the faith of Jesus.” [Revelation 14:12.] One of the landmarks under this message was \textbf{the temple of God}, seen by His truth-loving people in heaven, and the ark containing the law of God. The light of \textbf{the Sabbath} of the fourth commandment flashed its strong rays in the pathway of the transgressors of God’s law. The \textbf{non-immortality of the wicked} is an old landmark. \textbf{I can call to mind nothing more that can come under the head of the old landmarks}. All this cry about changing the old landmarks is all imaginary.}[Ms13-1889.9; 1889][https://egwwritings.org/?ref=en\_Ms13-1889.9&para=4179.14]

At the end of this list of landmarks, or pillars of our faith, she says that she can call to mind nothing more that can come under the head of the old landmarks. For many, this quotation is proof that the \emcap{personality of God} was not an old landmark nor a pillar. It is true that in this quotation Sister White did not explicitly mention the \emcap{personality of God}, but it would be implicitly included under the first angel’s message. Furthermore, there are other quotations from Sister White that explicitly include the \emcap{personality of God} as an old landmark, or pillar of our faith.

\egw{Those who seek to remove the \textbf{old landmarks} are not holding fast; they \textbf{are not remembering how they have received and heard}. Those who try to \textbf{\underline{bring in} theories that would remove \underline{the pillars of our faith}} \textbf{concerning the sanctuary}, \textbf{\underline{or concerning the personality of God or of Christ}, are working as blind men}. They are seeking to bring in uncertainties and to set the people of God \textbf{adrift}, without an anchor.}[Ms62-1905.14; 1905][https://egwwritings.org/?ref=en\_Ms62-1905.14&para=10026.20]

Sister White also teaches us that the pillars of our faith constitute the foundation of our faith.

\egw{\textbf{What influence is it that would lead men at this stage of our history to work in an underhanded, powerful way \underline{to tear down the foundation of our faith},—the foundation that was laid at the beginning of our work by prayerful study of the word and by revelation? Upon \underline{this foundation} we have been building for \underline{the past fifty years}. Do you wonder that when I see the beginning of a work that would \underline{remove some of the pillars of our faith}, I have something to say? I must obey the command, ‘Meet it!’}}[SpTB02 58.1; 1904][https://egwwritings.org/?ref=en\_SpTB02.58.1&para=417.295]

Removing some of the pillars of our faith means tearing down the foundation of our faith. Elsewhere, Sister White said that tearing down or undermining the foundation of our faith is done by indoctrination of the sentiments regarding the \emcap{personality of God}.

\egw{The college was taken out of Battle Creek; yet students are still called there, and there they \textbf{become indoctrinated with the very sentiments regarding the personality of God and Christ that would undermine the foundation of our faith}.}[Lt72-1906.5; 1906][https://egwwritings.org/?ref=en\_Lt72-1906.5&para=10013.11]

In light of these quotations we see positive testimony that the \emcap{personality of God} was part of the foundation of our faith. Furthermore, in chapter 10 of the special testimonies, entitled “\textit{The foundation of our faith}”, Sister White mentioned “\textit{Fundamental Principles}” using the synonyms “\textit{pillars of our faith}”, “\textit{waymarks}”, and “\textit{landmarks}”, when addressing the foundation of our faith.

\subsection*{Step 4: Alter the meaning of the term “the personality of God”}

The term ‘\textit{personality}’ has two different applications and the most common definition in everyday use is in the area of psychology. ‘\textit{Personality}’ is defined as “\textit{the characteristic sets of behaviors, cognitions, and emotional patterns that evolve from biological and environmental factors}”\footnote{Wikipedia Contributors. “Personality.” Wikipedia, Wikimedia Foundation, 19 Apr. 2019, \href{https://en.wikipedia.org/wiki/Personality}{en.wikipedia.org/wiki/Personality}.}. It is of utmost importance to recognize that when we are dealing with the pillar of our faith—“\textit{the personality of God}”—we are not in the realms of psychology. The accurate application of the word ‘\textit{personality}’ within the doctrine on the \emcap{personality of God} is found in the Merriam-Webster Dictionary: “\textit{the quality or state of being a person}”\footnote{\href{https://www.merriam-webster.com/dictionary/personality}{Merriam-Webster Dictionary} - ‘\textit{personality}’}. According to the Merriam-Webster Dictionary, this definition has been in use since the 15th century\footnote{See “\href{https://www.merriam-webster.com/dictionary/personality\#word-history}{First known use}” of the word ‘personality’ in Merriam Webster Dictionary}. In the 1828 edition of the Merriam Webster Dictionary we read definition of the word ‘\textit{personality}’ as: “\textit{that which constitutes an individual a distinct person}”\footnote{\href{https://archive.org/details/americandictiona02websrich/page/272/mode/2up}{Merriam-Webster Dictionary, 1828 edition} - ‘\textit{personality}’} \footnote{\href{https://archive.org/details/websterscomplete00webs/page/974/mode/2up}{The 1886 edition of Merriam-Webster Dictionary} defines the word ‘\textit{personality}’ as: “\textit{that which constitutes, or pertains to, a person}”}. Both of the definitions are found in The Encyclopaedic Dictionary, by Hunter Robert\footnote{\href{https://babel.hathitrust.org/cgi/pt?id=mdp.39015050663213&view=1up&seq=780}{Hunter Robert, The Encyclopaedic Dictionary} - ‘\textit{personality}’}—dictionary owned by Ellen White. The use of these definitions can be seen from the articles written on the \emcap{personality of God}.

In 1903, when Sister White wrote to Dr. Kellogg, \egwinline{I have \textbf{ever }had the same testimony to bear which I now bear \textbf{regarding the personality of God}}[Lt253-1903.9; 1903][https://egwwritings.org/?ref=en\_Lt253-1903.9&para=9980.15], she recalled her vision when she saw the Father and the Son.

\egw{‘I have often seen the lovely Jesus, that\textbf{ He is a person}.\textbf{ I asked Him if His Father was a person, }and \textbf{had \underline{a form} like Himself}. Said Jesus, ‘\textbf{I am the express image of My Father’s person!}’ [Hebrews 1:3.]}[Lt253-1903.12; 1903][https://egwwritings.org/?ref=en\_Lt253-1903.12&para=9980.18]

The quality or state that Sister White defines God to be a person is to have \textit{a form}—\textit{a physical appearance}. Dr. Kellogg follows the same application of the word \textit{'personality'}, although through speculation.

\others{The fact that God is so great that we cannot form a clear mental picture of \textbf{his physical appearance} need not lessen in our minds the reality of \textbf{His personality}...}[John H. Kellogg, The Living Temple, p. 31][https://archive.org/details/J.H.Kellogg.TheLivingTemple1903/page/n31/mode/2up]

As we have previously seen, our Adventist pioneers also pinpointed the physical appearance as a quality that makes God a person. James White wrote, \others{Those who deny \textbf{the personality of God}, say that ‘image’ here does not mean \textbf{physical form}, but moral image...}[James S. White, PERGO 1.1; 1861][https://egwwritings.org/?ref=en\_PERGO.1.1&para=1471.3]. J. B. Frisbie wrote, \others{Some seem to suppose it argues against \textbf{the personality of God}, because he is a Spirit, and say that he is without \textbf{body, or parts}...}[\href{https://documents.adventistarchives.org/Periodicals/RH/RH18540307-V05-07.pdf}{Adventist Review and Sabbath Herald, March 7, 1854}, J. B. Frisbie, “The Seventh-Day Sabbath Not Abolished”, p. 50]

In light of the facts, we recognize the application of the word ‘\textit{personality}’. When the subject on the \emcap{personality of God} is presented in its connection to the Trinity doctrine, there is often a tendency to alter the meaning of the word ‘\textit{personality}’. It is also important to mention that the subject on the \emcap{personality of God} deals with the personality of the Father. This is clearly seen from the presented data.

\subsection*{Step 5: In examining the Kellogg crisis, shifting the main focus from the personality of God to pantheism}

The data on the Kellogg crisis, in connection with the Trinity doctrine, is overwhelming if the \emcap{personality of God} is accounted for in the equation. The only way to not connect the dots is to ignore the \emcap{personality of God} and shift focus to pantheism exclusively. We do not deny the pantheistic nature of Kellogg's controversy. We believe that the pantheistic nature of Kellogg's controversy cannot be rightly understood if it is not examined in the true light of the \emcap{personality of God}. But, unfortunately, in examination of the Kellogg crisis, the attention that pantheism receives supersedes the examination of the truth on the \emcap{personality of God}.

You can do a search of Ellen White’s compilations to see just how much more attention pantheism received than the \emcap{personality of God}. If you were to search her writings for ‘pantheism’ or ‘pantheistic’, excluding the compilations after her death, you would find 36 occurrences. Among them are several repetitive quotations that Sister White copied from one letter to another, or to the special testimonies for the church. If you were to count the distinct occurrences you would only find 12 distinct quotations containing words like ‘\textit{pantheism}’ or ‘\textit{pantheistic}’\footnote{On the \href{https://egwwritings.org/}{https://egwwritings.org/} search bar, input the word “\textit{pantheis*} ”; this will include all words beginning with the ‘\textit{pantheis..}’, (including ‘\textit{pantheism}’ and ‘\textit{pantheistic}’). The results can be compared in subsetting the corpus of Ellen White writings by including or excluding compilations after her death. This option is available in the dropdown menu under the search bar.}. If you conducted the same search, but only in the compilations issued after her death, you would find 140 occurrences! All of these fall into one of the twelve distinct instances Sister White wrote on the subject of pantheism.

In a search of Ellen White writings on the phrase “\textit{personality of God}”, excluding the compilations after her death, you would find 58 occurrences. Among them are also several repetitive quotations that Sister White copied to several different letters and to the testimonies for the church. Yet, if you were to search this phrase within the compilations that were issued after her death you would only find 52 occurrences.

These simple statistics demonstrate the focus of the compilators after the death of Sister White. Such emphasis on pantheism changed our public opinion regarding Kellogg’s crisis. Forty-three, out of fifty-eight, quotations on the phrase “\textit{personality of God}” are found in letters and manuscripts, available to the public from 2015 onwards. This means that three quarters (\textit{74 percent}) of the quotation regarding the \emcap{personality of God}, prior to 2015, was not available to the public. Prior to 2015 we did not have much available data to study Kellogg's crisis in light of the \emcap{personality of God} and in its context.