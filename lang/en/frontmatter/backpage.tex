\pagestyle{empty}

\cleardoublepage

% Back cover of the book
\begin{center}
\begin{tikzpicture}[remember picture,overlay]
    
    % Black background
    \fill[black] (current page.south west) rectangle (current page.north east);    

    % Central circle pattern
    \begin{scope}[shift={(current page.center)}, scale=1.0871]
        \foreach \x in {-20, -18, -16, -14, -12, -10, -8, -6, -4, -2, 0, 2, 4, 6, 8, 10, 12, 14, 16, 18} {
            \foreach \y in {-2, 2} {
                \begin{scope}[shift={(\x + 1, \y)}]
                    \foreach \r in {0.2, 0.4, 0.6, 0.8, 1.0} {
                        \draw[thick, darkgray] (0,0) circle (\r);
                    }
                \end{scope}
            }
            \foreach \y in {-3, -1, 1, 3} {
                \begin{scope}[shift={(\x, \y)}]
                    \foreach \r in {0.2, 0.4} {
                        \draw[thick, darkgray] (0,0) circle (\r);
                    }
                \end{scope}
            }
            \foreach \y in {-1, 1} {
                \begin{scope}[shift={(\x, 0)}]
                    \foreach \r in {0.2, 0.4, 0.6, 0.8, 1.0} {
                        \draw[thick, lightgray] (0,0) circle (\r);
                    }
                \end{scope}
            }
            \foreach \y in {-1, 1} {
                \begin{scope}[shift={(\x, \y)}]
                    \foreach \r in {0.2, 0.4, 0.6, 0.8, 1.0} {
                        \fill[black] (0,0) circle (\r);
                    }
                \end{scope}
            }
            \foreach \y in {-1, 1} {
                \begin{scope}[shift={(\x, \y)}]
                    \foreach \r in {0.2, 0.4, 0.6, 0.8, 1.0} {
                        \draw[thick, darkgray] (0,0) circle (\r);
                    }
                \end{scope}
            }
        }
    \end{scope}

    % Title
    \node[white, anchor=north, text width =0.9\linewidth, align = justify] at ([yshift=-48pt] current page.north) {\scshape{The Seventh-day Adventist Church is grappling with a persistent doctrinal challenge: the Trinity. While many focus on resolving the doctrine itself, the author argues that the real issue is a deeper character problem. The widespread antagonism between pro- and anti-Trinitarians fuels division. The author proposes a simple step toward constructive dialogue: finding Common Ground. Michael, co-founder of the Forgotten Pillar Project, aims to revive the crucial doctrine on the personality of God, offering a fresh perspective that could bridge the divide. Explore the Trinity controversy through this unique lens and discover the clarity that could bring unity to the church.}};

    % Publisher logo
    \node[anchor=north, above=4pt of publisher.north] {\includegraphics[width=0.21\textwidth]{frontmatter/logo-white.png}};

\end{tikzpicture}
\end{center}
