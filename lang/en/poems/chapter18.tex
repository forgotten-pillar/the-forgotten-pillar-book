% The Heavenly Trio

\begin{titledpoem}
    
    \stanza{
        In a realm divine, where truths unfold, \\
        A message of clarity, brave and bold. \\
        Ellen spoke, her voice clear and bright, \\
        Revealing the depths of heavenly light.
    }

    \stanza{
        Misunderstood by many who read, \\
        Her words hold a truth that all must heed. \\
        Not a triune God, but a trio divine, \\
        Three living persons, distinct in line.
    }

    \stanza{
        The Father, not formless, but full and bright, \\
        Invisible to mortal, yet real in might. \\
        He is the fullness, a presence complete, \\
        Invisible to sight, yet real and concrete.
    }

    \stanza{
        The Son, God's fullness, manifest and near, \\
        In Him, the divine becomes crystal clear. \\
        The personality of God, seen in His face, \\
        In Christ, we witness God's grace.
    }

    \stanza{
        The Holy Spirit, in fullness resides, \\
        Within the Godhead, where mystery abides. \\
        Father and Son, with bodies they stand, \\
        Holy Spirit, as spirit, spreads through the land.
    }

    \stanza{
        Distinct and clear, their roles unfold, \\
        Father and Son, in form behold. \\
        Yet everywhere present, the Spirit we find, \\
        Their representative, in heart and mind.
    }

    \stanza{
        Spiritual and bodily, presence defined, \\
        Understanding this truth, enlightenment kind. \\
        Ellen's message, profound and bright, \\
        Guides us through the heavenly light.
    }

    \stanza{
        Ellen's words, in context found, \\
        Show a truth profound and sound. \\
        Not the trinity she did embrace, \\
        But a trio divine, each in their place.
    }

    \stanza{
        In "heavenly Trio," a contrast is drawn, \\
        Between trinity's doctrine and faith's true dawn. \\
        The pillar stands firm, the personality of God, \\
        Distinct from trinity, where truth is awed.
    }
    
\end{titledpoem}