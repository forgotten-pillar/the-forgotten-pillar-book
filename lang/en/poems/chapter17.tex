% Reply to Kellogg’s trinitarian sentiments

\begin{titledpoem}
    
    \stanza{
        In the light of truth, so clear and bold, \\
        Through the crisis of Kellogg, a story told. \\
        Not of pantheism's shadow, dim and wide, \\
        But of God's persona, in whom we confide.
    }

    \stanza{
        Ellen White, with wisdom, did uphold \\
        The personality of God, repeatedly told. \\
        Against the Trinity, her words did lean, \\
        In John's gospel, a unity unseen.
    }

    \stanza{
        "The Father and the Son," she wrote with grace, \\
        Their unity, not in person, but in embrace. \\
        John seventeen, her chosen guide, \\
        Where God's true nature cannot hide.
    }

    \stanza{
        Loughborough echoed, with voice so clear, \\
        The Trinity, by John, dismissed without fear. \\
        Adventist pioneers, with Ellen, agreed, \\
        On God's true personality, they all did plead.
    }

    \stanza{
        The Fundamental Principles, so dear, \\
        The personality of God, we must revere. \\
        Not in the trinity's confusing creed, \\
        But in His presence, our faith is freed.
    }

    \stanza{
        So let us stand, on truth so bright, \\
        Rejecting error, with all our might. \\
        For in God's personality, we find our plea, \\
        A beacon of truth, that sets us free.
    }
    
\end{titledpoem}