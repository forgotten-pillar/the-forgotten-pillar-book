\qrchapter{https://forgottenpillar.com/rsc/en-fp-chapter13}{The Sabbath God vs. Sunday God - J. B. Frisbie}


\qrchapter{https://forgottenpillar.com/rsc/en-fp-chapter13}{Le Dieu du Sabbat contre le Dieu du Dimanche - J. B. Frisbie}


There are other articles written on the \emcap{personality of God} by our pioneers and it would be too much to include everything here, but we would like to add one more testimony from brother J. B. Frisbie’s article where he compares the Sabbath God with the Sunday God. He compares the truth on the \emcap{personality of God} expressed in the first point of the \emcap{Fundamental Principles} with the Trinity doctrine. Let us take a look at a portion of his article, “\textit{The Seventh Day-Sabbath Not Abolished}” from the Review and Herald, March 7, 1854.


Il existe d'autres articles écrits sur la \emcap{personnalité de Dieu} par nos pionniers et ce serait trop d'inclure tout ici, mais nous aimerions ajouter un témoignage supplémentaire de l'article du frère J. B. Frisbie où il compare le Dieu du Sabbat avec le Dieu du Dimanche. Il compare la vérité sur la \emcap{personnalité de Dieu} exprimée dans le premier point des \emcap{Principes Fondamentaux} avec la doctrine de la Trinité. Examinons une partie de son article, « \textit{Le Sabbat du Septième Jour n'est pas Aboli} » de la Review and Herald, 7 mars 1854.


\begin{figure}[hp]
    \centering
    \includegraphics[width=1\linewidth]{images/j-b-frisbie.jpg}
    \caption*{John Byington Frisbie (1816-1882)}
    \label{fig:j-b-frisbie}
\end{figure}


\begin{figure}[hp]
    \centering
    \includegraphics[width=1\linewidth]{images/j-b-frisbie.jpg}
    \caption*{John Byington Frisbie (1816-1882)}
    \label{fig:j-b-frisbie}
\end{figure}


\section*{The Sabbath God}


\section*{Le Dieu du Sabbat}


\others{After we know and remember God, by keeping his holy Sabbath, \textbf{then the Bible will teach of his personality and dwelling place}. \textbf{Man is in the image and likeness of God}. Genesis 1:26. ‘And God said, Let us (speaking to his son) make man in our image, after our likeness’. Chap 2:7. ‘And the Lord God formed man of the dust of the ground, and breathed into his nostrils the breath of life: and man became a living soul’. Genesis 9:6; 1 Corinthians 11:7; James 3:9. \textbf{That which was made in \underline{the image and likeness of God} was made of the dust of the ground called man}.}


\others{Après que nous connaissons et nous souvenons de Dieu, en gardant son saint Sabbat, \textbf{alors la Bible enseignera sur sa personnalité et son lieu de résidence}. \textbf{L'homme est à l'image et à la ressemblance de Dieu}. Genèse 1:26. « Et Dieu dit : Faisons (parlant à son fils) l'homme à notre image, selon notre ressemblance ». Chap 2:7. « Et l'Éternel Dieu forma l'homme de la poussière de la terre, et souffla dans ses narines un souffle de vie ; et l'homme devint une âme vivante ». Genèse 9:6 ; 1 Corinthiens 11:7 ; Jacques 3:9. \textbf{Ce qui fut fait à \underline{l'image et à la ressemblance de Dieu} fut fait de la poussière de la terre appelé homme}.}


\othersnogap{This is known to be the true sense from other testimonies that may be given from the Bible. \textbf{Jesus was in the form of a man and the express image of his Father’s person}.}


\othersnogap{Ceci est connu pour être le vrai sens d'après d'autres témoignages qui peuvent être donnés de la Bible. \textbf{Jésus était dans la forme d'un homme et l'empreinte de la personne de son Père}.}


\othersnogap{Philippians 2:6-8. \textbf{Christ Jesus}: ‘Who, being in \textbf{the form of God}, thought it not robbery to be \textbf{equal with God}. But made himself of no reputation, and took upon him \textbf{the form of a servant}, and was \textbf{made in the likeness of men’}. 2 Corinthians 4:4. \textbf{‘And being formed in fashion as a man’}, etc. Colossians 1:15. ‘\textbf{Who is the image of the invisible God}’.}


\othersnogap{Philippiens 2:6-8. \textbf{Christ Jésus} : « Lequel, étant en \textbf{forme de Dieu}, n'a point regardé comme une usurpation d'être \textbf{égal à Dieu}. Mais il s'est anéanti lui-même, ayant pris \textbf{la forme de serviteur}, et ayant été \textbf{fait à la ressemblance des hommes} ». 2 Corinthiens 4:4. \textbf{« Et étant trouvé en figure comme un homme »}, etc. Colossiens 1:15. « \textbf{Qui est l'image du Dieu invisible} ».}


\othersnogap{Hebrews 1:3. \textbf{The Son; ‘Who being the brightness of his glory, and the express image of his person’}. In this sense could Jesus say to Philip in truth, ‘He that hath seen me hath seen the Father.’ John 14:9. Some seem to suppose it argues \textbf{against the personality of God, \underline{because he is a Spirit, and say that he is without body, or parts}}. John 4:24. ‘\textbf{God is a Spirit}’. Hebrews 1:7. ‘\textbf{Who maketh his angels spirits}’. \textbf{Who would pretend to say that angels have no bodies or parts because they are spirits}. \textbf{\underline{None the less is God a spiritual being having body and parts as we may learn by his having a dwelling place and because he has and may be seen}}. Exodus 33:23. ‘And I will take away mine hand, and thou shalt\textbf{ see my back parts}, but my \textbf{face shall not be seen}’. Matthew 5:8. ‘Blessed are the pure in heart, for \textbf{they shall see God}’. Hebrews 12:14. ‘Follow peace with all men, and holiness, without which \textbf{no man shall see the Lord}’. Matthew 18:10. ‘That in heaven their angels do \textbf{always behold the face of my Father which is in heaven}’. Matthew 6:9. ‘After this manner therefore pray ye, \textbf{Our Father which art in heaven}’, etc. John 6:38. ‘For I \textbf{came down from heaven} not to do mine own will, but the will of him that sent me’. Chap 16:28. ‘\textbf{I came forth from the Father, and am come into the world}: again I \textbf{leave the world, and go to the Father}’.}


\othersnogap{Hébreux 1:3. \textbf{Le Fils ; « Qui étant la splendeur de sa gloire, et l'empreinte de sa personne »}. Dans ce sens Jésus pouvait-il dire à Philippe en vérité, « Celui qui m'a vu a vu le Père ». Jean 14:9. Certains semblent supposer que cela argumente \textbf{contre la personnalité de Dieu, \underline{parce qu'il est un Esprit, et disent qu'il est sans corps, ni parties}}. Jean 4:24. « \textbf{Dieu est Esprit} ». Hébreux 1:7. « \textbf{Qui fait de ses anges des esprits} ». \textbf{Qui prétendrait dire que les anges n'ont ni corps ni parties parce qu'ils sont des esprits}. \textbf{\underline{Dieu n'en est pas moins un être spirituel ayant corps et parties comme nous pouvons l'apprendre par le fait qu'il a un lieu de résidence et parce qu'il a et peut être vu}}. Exode 33:23. « Et je retirerai ma main, et tu \textbf{verras mon dos}, mais ma \textbf{face ne sera point vue} ». Matthieu 5:8. « Heureux ceux qui ont le cœur pur, car \textbf{ils verront Dieu} ». Hébreux 12:14. « Recherchez la paix avec tous, et la sanctification, sans laquelle \textbf{personne ne verra le Seigneur} ». Matthieu 18:10. « Qu'au ciel leurs anges \textbf{voient continuellement la face de mon Père qui est aux cieux} ». Matthieu 6:9. « Vous donc, priez ainsi : \textbf{Notre Père qui es aux cieux} », etc. Jean 6:38. « Car je \textbf{suis descendu du ciel} pour faire, non ma volonté, mais la volonté de celui qui m'a envoyé ». Chap 16:28. « \textbf{Je suis sorti du Père, et je suis venu dans le monde} ; je \textbf{laisse de nouveau le monde, et je vais au Père} ».}


\othersnogap{\textbf{Does not God say he fills immensity of space? \underline{We answer, No}}. Psalm 139:7, 8. ‘Whither shall I go \textbf{from thy Spirit}? or whither shall I flee \textbf{from thy presence}? If I ascend up into heaven, thou art there’, etc. \textbf{\underline{God by his Spirit may fill heaven and earth}}, etc. \textbf{Some confound God with his Spirit, which makes confusion}. Psalm 11:4. ‘\textbf{The Lord is in his holy temple, the Lord’s throne is in heaven}: his eyes behold’, etc. Habakkuk 2:20; Psalm 102:19. ‘For he hath looked \textbf{down from the height of his Sanctuary}; \textbf{\underline{from heaven} did the Lord behold the earth’}. 1 Peter 3:12. ‘For the eyes of the Lord are over the righteous, and his ears are open unto their prayers’, etc. Psalm 80:1. ‘Give ear, O Shepherd of Israel, thou that leadest Joseph like a flock; thou \textbf{that dwellest between the cherubims}, shine forth’. Psalm 99:1; Isaiah 37:16.}


\othersnogap{\textbf{Dieu ne dit-il pas qu'il remplit l'immensité de l'espace ? \underline{Nous répondons, Non}}. Psaume 139:7, 8. « Où irais-je loin \textbf{de ton Esprit} ? Et où fuirais-je loin \textbf{de ta face} ? Si je monte aux cieux, tu y es », etc. \textbf{\underline{Dieu par son Esprit peut remplir le ciel et la terre}}, etc. \textbf{Certains confondent Dieu avec son Esprit, ce qui crée de la confusion}. Psaume 11:4. « \textbf{L'Éternel est dans le temple de sa sainteté ; l'Éternel a son trône dans les cieux} ; ses yeux contemplent », etc. Habacuc 2:20 ; Psaume 102:19. « Car il a regardé \textbf{du haut de son Sanctuaire} ; \textbf{\underline{des cieux} l'Éternel a regardé sur la terre} ». 1 Pierre 3:12. « Car les yeux du Seigneur sont sur les justes, et ses oreilles sont attentives à leurs prières », etc. Psaume 80:1. « Prête l'oreille, berger d'Israël, toi qui conduis Joseph comme un troupeau ; toi \textbf{qui sièges entre les chérubins}, fais briller ta splendeur ». Psaume 99:1 ; Ésaïe 37:16.}


\othersnogap{John 14:2. ‘In my Father’s house are many mansions. I go to prepare a place for you’. Revelation 21:2-5; Hebrews 11:6. ‘For he that cometh to God must believe that he is’, etc. \textbf{This testimony we deem highly important at this time, to know that there is a God. We have no doubt that if our eyes could be opened in vision, or see as angels see, we should see God in heaven sitting on his throne, and is present to all that exists, however distant from him in his creation}.}[\href{https://documents.adventistarchives.org/Periodicals/RH/RH18540307-V05-07.pdf}{Adventist Review and Sabbath Herald, March 7, 1854}, J. B. Frisbie, “The Seventh-Day Sabbath Not Abolished”, p. 50]


\othersnogap{Jean 14:2. « Il y a plusieurs demeures dans la maison de mon Père. Je vais vous préparer une place ». Apocalypse 21:2-5 ; Hébreux 11:6. « Or, il faut que celui qui s'approche de Dieu croie qu'il existe », etc. \textbf{Nous estimons ce témoignage hautement important en ce temps, de savoir qu'il y a un Dieu. Nous n'avons aucun doute que si nos yeux pouvaient être ouverts en vision, ou voir comme les anges voient, nous verrions Dieu au ciel assis sur son trône, et il est présent à tout ce qui existe, aussi éloigné soit-il de lui dans sa création}.}[\href{https://documents.adventistarchives.org/Periodicals/RH/RH18540307-V05-07.pdf}{Adventist Review and Sabbath Herald, 7 mars 1854}, J. B. Frisbie, « Le Sabbat du Septième Jour n'est pas Aboli », p. 50]


Here we see the same argument and reasoning, that God is a personal spiritual Being. This God is the Sabbath God. Brother Frisbie compares this God with the Sunday God, who is a trinitarian God.


Ici, nous voyons le même argument et raisonnement, que Dieu est un Être spirituel personnel. Ce Dieu est le Dieu du Sabbat. Frère Frisbie compare ce Dieu avec le Dieu du dimanche, qui est un Dieu trinitaire.


\section*{The Sunday God}


\section*{Le Dieu du dimanche}


\others{We will make a few extracts, that the reader may \textbf{see the broad contrast between \underline{the God of the Bible} brought to light through Sabbath-keeping, and the god in the dark through Sunday-keeping}. Catholic Catechism Abridged by the Rt. Rev. John Dubois, Bishop of New York. Page 5. ‘\textbf{Ques. Where is God? Ans. God is everywhere}. Q. Does God see and know all things? A. Yes, he does know and see all things. \textbf{Q. Has God any body? A. \underline{No; God has no body, he is a pure Spirit}}. \textbf{Q. Are there more Gods than one? A. No; there is but one God. Q. Are there more persons than one in God? A. \underline{Yes; in God there are three persons}. Q. Which are they? A. God the Father, God the Son and God the Holy Ghost. Q. Are there not three Gods? A. No; the Father, the Son and the Holy Ghost, are all but one and the same God}’.}


\others{Nous ferons quelques extraits, afin que le lecteur puisse \textbf{voir le large contraste entre \underline{le Dieu de la Bible} mis en lumière par l'observation du Sabbat, et le dieu dans les ténèbres par l'observation du dimanche}. Catéchisme catholique abrégé par le très révérend John Dubois, évêque de New York. Page 5. ‘\textbf{Question. Où est Dieu ? Réponse. Dieu est partout}. Q. Dieu voit-il et connaît-il toutes choses ? R. Oui, il connaît et voit toutes choses. \textbf{Q. Dieu a-t-il un corps ? R. \underline{Non ; Dieu n'a pas de corps, il est un pur Esprit}}. \textbf{Q. Y a-t-il plus d'un Dieu ? R. Non ; il n'y a qu'un seul Dieu. Q. Y a-t-il plus d'une personne en Dieu ? R. \underline{Oui ; en Dieu il y a trois personnes}. Q. Quelles sont-elles ? R. Dieu le Père, Dieu le Fils et Dieu le Saint-Esprit. Q. N'y a-t-il pas trois Dieux ? R. Non ; le Père, le Fils et le Saint-Esprit ne sont tous qu'un seul et même Dieu}’.}


\othersnogap{The first article of the Methodist Religion, p. 8. \textbf{‘There is but one living and true God}, everlasting, \textbf{without body or parts}, of infinite power, wisdom and goodness: the maker and preserver of all things, visible and invisible. \textbf{And in unity of this God-head, there are three persons of one substance, power and eternity; the Father, the Son, and the Holy Ghost}.’}


\othersnogap{Le premier article de la religion méthodiste, p. 8. \textbf{‘Il n'y a qu'un seul Dieu vivant et vrai}, éternel, \textbf{sans corps ni parties}, d'une puissance, sagesse et bonté infinies : le créateur et conservateur de toutes choses, visibles et invisibles. \textbf{Et dans l'unité de cette Divinité, il y a trois personnes d'une seule substance, puissance et éternité ; le Père, le Fils et le Saint-Esprit}.’}


\othersnogap{In this article like the Catholic doctrine, \textbf{we are taught that there are three persons of one substance,} power and eternity making\textbf{ in all one living and true God}, everlasting \textbf{without body or parts}. But in all this we are not told \textbf{what became of the body of Jesus who had a body when he ascended, who went to God who ‘is everywhere’ or nowhere}. Doxology.}


\othersnogap{Dans cet article comme dans la doctrine catholique, \textbf{on nous enseigne qu'il y a trois personnes d'une seule substance,} puissance et éternité formant\textbf{ en tout un seul Dieu vivant et vrai}, éternel \textbf{sans corps ni parties}. Mais dans tout cela on ne nous dit pas \textbf{ce qu'est devenu le corps de Jésus qui avait un corps quand il est monté, qui est allé vers Dieu qui ‘est partout’ ou nulle part}. Doxologie.}


\othersnogap{‘\textbf{To God the Father, God the Son,}} \\
\others{\textbf{God the Spirit, three in one.}’} \\
\others{Again} \\
\others{‘Warms in the sun, refreshes in the breeze,} \\
\others{Glows in the stars, and blossoms in the trees.} \\
\others{\textbf{Lives through all life, extends through all extent},} \\
\others{Spreads undivided and operates unspent.’ - Pope.}


\othersnogap{‘\textbf{À Dieu le Père, Dieu le Fils,}} \\
\others{\textbf{Dieu l'Esprit, trois en un.}’} \\
\others{Encore} \\
\others{‘Réchauffe dans le soleil, rafraîchit dans la brise,} \\
\others{Brille dans les étoiles, et fleurit dans les arbres.} \\
\others{\textbf{Vit à travers toute vie, s'étend à travers toute étendue},} \\
\others{Se répand sans division et opère sans épuisement.’ - Pope.}


\othersnogap{These ideas well accord with those heathen philosophers. One says, ‘That water was the principle of all things, and that God is that intelligence, by whom all things are formed out of water.’ Another, ‘That air is God, that it is produced, that it is immense and infinite,’ etc. A third, ‘That God is a soul diffused throughout all beings of nature,’ etc. \textbf{Some, who had the idea of \underline{a pure Spirit}}. Last of all, ‘That God is an eternal substance.’}


\othersnogap{Ces idées s'accordent bien avec celles des philosophes païens. L'un dit : ‘Que l'eau était le principe de toutes choses, et que Dieu est cette intelligence, par laquelle toutes choses sont formées à partir de l'eau.’ Un autre : ‘Que l'air est Dieu, qu'il est produit, qu'il est immense et infini,’ etc. Un troisième : ‘Que Dieu est une âme diffusée à travers tous les êtres de la nature,’ etc. \textbf{Certains, qui avaient l'idée d’\underline{un pur Esprit}}. En dernier lieu : ‘Que Dieu est une substance éternelle.’}


\othersnogap{These extracts are taken from Rollin’s History, Vol. II, pp. 597-8, published by Harpers. \textbf{We should rather mistrust that the Sunday god came from the same source that Sunday-keeping did}. ‘Sunday was a name given by the heathens to the first day of the week, because it was the day on which they worshiped the sun.’ - Union Bible Dictionary. \textbf{Afterward modified by the Roman Catholic Church, in the form we now find it taught through the land}.}


\othersnogap{Ces extraits sont tirés de l'Histoire de Rollin, Vol. II, pp. 597-8, publié par Harpers. \textbf{Nous devrions plutôt nous méfier que le dieu du dimanche vienne de la même source que l'observation du dimanche}. ‘Le dimanche était un nom donné par les païens au premier jour de la semaine, parce que c'était le jour où ils adoraient le soleil.’ - Union Bible Dictionary. \textbf{Ensuite modifié par l'Église catholique romaine, sous la forme que nous le trouvons maintenant enseigné à travers le pays}.}


\othersnogap{It is very natural to suppose when \textbf{the Pope set himself up to be God in the temple of God}, [2 Thessalonians 2:4] that he should have a day sanctified to his worship. This he has done. - Douay Catechism, p. 59. ‘Q. What is the best means to sanctify Sunday? A. By hearing mass, etc. This saying mass is for the priest to gabble over Latin, drink some wine, and give the people a wafer to eat.’}”“\others{But God sanctified his day because he had rested on it. Another day for a very different purpose. Genesis 2:3.}


\othersnogap{Il est très naturel de supposer que quand \textbf{le Pape s'est établi pour être Dieu dans le temple de Dieu}, [2 Thessaloniciens 2:4] qu'il devrait avoir un jour sanctifié pour son culte. C'est ce qu'il a fait. - Catéchisme de Douay, p. 59. ‘Q. Quel est le meilleur moyen de sanctifier le dimanche ? R. En entendant la messe, etc. Cette célébration de la messe consiste pour le prêtre à baragouiner du latin, boire du vin, et donner au peuple une hostie à manger.’}”“\others{Mais Dieu a sanctifié son jour parce qu'il s'était reposé ce jour-là. Un autre jour pour un but très différent. Genèse 2:3.}


\othersnogap{In days before the moral fall of Babylon God directed the minds of his honest children right in their prayers, whatever they might think at other times, but now since the apostasy the mind reaches to no god but to the people only, there are many prayers to men we know by their effect and eloquence. \textbf{We are truly thankful to our heavenly Father that \underline{he has led our minds from such folly}, to know, and remember \underline{his holy name} by keeping his holy day that we might love, serve and worthily \underline{glorify him through our great High Priest in the heavenly Sanctuary in this day of atonement}}.}[Ibid.][https://documents.adventistarchives.org/Periodicals/RH/RH18540307-V05-07.pdf]


\othersnogap{Aux jours avant la chute morale de Babylone, Dieu dirigeait correctement l'esprit de ses enfants honnêtes dans leurs prières, quoi qu'ils puissent penser à d'autres moments, mais maintenant depuis l'apostasie, l'esprit n'atteint aucun dieu mais seulement les gens, il y a beaucoup de prières aux hommes que nous connaissons par leur effet et leur éloquence. \textbf{Nous sommes vraiment reconnaissants à notre Père céleste de ce qu’\underline{il a conduit nos esprits loin de telle folie}, pour connaître et se souvenir de \underline{son saint nom} en gardant son saint jour afin que nous puissions l'aimer, le servir et dignement \underline{le glorifier à travers notre grand Souverain Sacrificateur dans le Sanctuaire céleste en ce jour d'expiation}}.}[Ibid.][https://documents.adventistarchives.org/Periodicals/RH/RH18540307-V05-07.pdf]


Before becoming a Seventh-day Adventist, Frisbie was a Methodist preacher and a bitter opponent of Adventist beliefs. In 1853, after a debate on the Sabbath with Joseph Bates, he reversed his position and began to keep the Sabbath and preach the Seventh-day Adventist doctrine. He renounced Sunday, the Trinity, and accepted the Seventh-day Sabbath and the truth about God, that the Seventh-day Adventist’s taught in the first point of the \emcap{Fundamental Principles}.


Avant de devenir Adventiste du Septième Jour, Frisbie était un prédicateur méthodiste et un opposant acharné aux croyances adventistes. En 1853, après un débat sur le Sabbat avec Joseph Bates, il a renversé sa position et a commencé à observer le Sabbat et à prêcher la doctrine Adventiste du Septième Jour. Il a renoncé au dimanche, à la Trinité, et a accepté le Sabbat du septième jour et la vérité sur Dieu, que les Adventistes du Septième Jour enseignaient dans le premier point des \emcap{Principes Fondamentaux}.


Do other Adventist pioneers see discordance between the Trinity doctrine and the \emcap{personality of God} expressed in the first point of the \emcap{Fundamental Principles}?


D'autres pionniers adventistes voient-ils une discordance entre la doctrine de la Trinité et la \emcap{personnalité de Dieu} exprimée dans le premier point des \emcap{Principes Fondamentaux} ?


% The Sabbath God vs. Sunday God - J. B. Frisbie

\begin{titledpoem}
    
    \stanza{
        On seventh day or first we kneel, \\
        But deeper truths these days reveal. \\
        Not just when we choose to pray, \\
        But which God we serve each day.
    }

    \stanza{
        The Sabbath God, a Being clear, \\
        With form and place, both far and near. \\
        In His image we were made, \\
        His Son the perfect likeness displayed. \\
    }

    \stanza{
        The Son, the Father's image bright, \\
        Shows us the path to truth and light. \\
        "Who's seen me has seen the Father too," \\
        Christ's words both powerful and true.
    }

    \stanza{
        The Sunday God, a trinity, \\
        Three persons in strange unity. \\
        Without body, without part, \\
        A concept born from human art.
    }

    \stanza{
        One God with face and hands and form, \\
        Who rested when creation's storm \\
        Had ceased its work on seventh day, \\
        This God commands we rest and pray.
    }

    \stanza{
        Not some essence spreading wide, \\
        Formless spirit with no side. \\
        But a Person on a throne, \\
        With His Son, yet not alone.
    }

    \stanza{
        So choose not merely when to kneel, \\
        But which God your heart finds real. \\
        The day we keep reveals our view \\
        Of which God we believe is true.
    }
    
\end{titledpoem}