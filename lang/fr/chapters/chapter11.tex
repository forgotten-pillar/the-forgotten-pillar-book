\qrchapter{https://forgottenpillar.com/rsc/en-fp-chapter11}{The personality of God - by James S. White}


\qrchapter{https://forgottenpillar.com/rsc/en-fp-chapter11}{La personnalité de Dieu - par James S. White}


In what follows, we will examine James White’s pamphlet titled “\textit{The Personality of God}”. When we read this article, we will see that James White continues where Brother Loughborough left off, and that he expands and deepens the understanding behind the first point of the \emcap{Fundamental Principles}.


Dans ce qui suit, nous examinerons la brochure de James White intitulée « \textit{La Personnalité de Dieu} ». Lorsque nous lirons cet article, nous verrons que James White continue là où Frère Loughborough s'est arrêté, et qu'il élargit et approfondit la compréhension derrière le premier point des \emcap{Principes Fondamentaux}.


James White’s tract was printed multiple times, advertised 54 times, and reprinted twice in the Review and Herald publication. His view on the \emcap{personality of God} was well known and spread throughout Adventism. In this pamphlet, we will see clear criticism toward the ideas that Kellogg advocated in the Living Temple.


Le tract de James White a été imprimé plusieurs fois, annoncé 54 fois, et réimprimé deux fois dans la publication Review and Herald. Sa vision de la \emcap{personnalité de Dieu} était bien connue et répandue dans tout l'Adventisme. Dans cette brochure, nous verrons une critique claire envers les idées que Kellogg défendait dans le Temple Vivant.


\begin{figure}[hp]
    \centering
    \includegraphics[width=1\linewidth]{images/james-and-ellen-white.jpg}
    \caption*{James Springer White (1821-1881) and Ellen White (1827-1915)}
    \label{fig:james-and-ellen-white}
\end{figure}


\begin{figure}[hp]
    \centering
    \includegraphics[width=1\linewidth]{images/james-and-ellen-white.jpg}
    \caption*{James Springer White (1821-1881) et Ellen White (1827-1915)}
    \label{fig:james-and-ellen-white}
\end{figure}


\othersQuote{\textbf{MAN was made in the image of God}. ‘And God said, Let us make man in our image, after our likeness.’ ‘So God created man in his own image, in the image of God created he him.’ Genesis 1:26, 27. See also chap. 9:6; 1 Corinthians 11:7. \textbf{Those who deny the personality of God, say that ‘image’ here does not mean \underline{physical form}, but moral image, and they make this the grand starting point to prove the immortality of all men}. The argument stands thus: First, man was made in God’s moral image. Second, God is an immortal being. Third, therefore all men are immortal. But this mode of reasoning would also prove man omnipotent, omniscient, and omnipresent, and thus clothe mortal man with all the attributes of the deity. Let us try it: First, man was made in God’s moral image. Second, God is omnipotent, omniscient, and omnipresent. Third, therefore, man is omnipotent, omniscient, and omnipresent. That which proves too much, proves nothing to the point, therefore the position that the image of God means his moral image, cannot be sustained. \textbf{As proof that God is a person, read his own words to Moses}: ‘And the Lord said, Behold there is a place by me, and thou shalt stand upon a rock; and it shall come to pass, while my glory passeth by, that I will put thee in a cleft of the rock, and will cover thee \textbf{with my hand} while \textbf{I pass by}. And I will take away \textbf{mine hand} and thou shalt \textbf{see my back parts}; \textbf{but my face shall not be seen}.’ Exodus 33:21-23. See also chap. 24:9-11. \textbf{Here God tells Moses that he shall \underline{see his form}}. \textbf{To say that God made it appear to Moses that he saw his form, when he has no form, is charging God with adding to falsehood a sort of juggling deception upon his servant Moses}.}[James S. White, PERGO 1.1; 1861][https://egwwritings.org/read?panels=p1471.3]


\othersQuote{\textbf{L'HOMME fut fait à l'image de Dieu}. « Et Dieu dit : Faisons l'homme à notre image, selon notre ressemblance. » « Dieu donc créa l'homme à son image ; il le créa à l'image de Dieu. » Genèse 1:26, 27. Voir aussi chap. 9:6 ; 1 Corinthiens 11:7. \textbf{Ceux qui nient la personnalité de Dieu, disent qu’« image » ici ne signifie pas \underline{forme physique}, mais image morale, et ils en font le grand point de départ pour prouver l'immortalité de tous les hommes}. L'argument se présente ainsi : Premièrement, l'homme fut fait à l'image morale de Dieu. Deuxièmement, Dieu est un être immortel. Troisièmement, donc tous les hommes sont immortels. Mais ce mode de raisonnement prouverait aussi que l'homme est omnipotent, omniscient et omniprésent, et revêtirait ainsi l'homme mortel de tous les attributs de la divinité. Essayons : Premièrement, l'homme fut fait à l'image morale de Dieu. Deuxièmement, Dieu est omnipotent, omniscient et omniprésent. Troisièmement, donc, l'homme est omnipotent, omniscient et omniprésent. Ce qui prouve trop ne prouve rien sur le point, donc la position que l'image de Dieu signifie son image morale, ne peut être soutenue. \textbf{Comme preuve que Dieu est une personne, lisez ses propres paroles à Moïse} : « Et l'Éternel dit : Voici, il y a un lieu près de moi, et tu te tiendras sur le rocher ; et il arrivera que quand ma gloire passera, je te mettrai dans une fente du rocher, et je te couvrirai \textbf{de ma main} pendant que \textbf{je passerai}. Et j'ôterai \textbf{ma main}, et tu \textbf{verras mon dos} ; \textbf{mais ma face ne sera point vue}. » Exode 33:21-23. Voir aussi chap. 24:9-11. \textbf{Ici Dieu dit à Moïse qu'il \underline{verra sa forme}}. \textbf{Dire que Dieu fit paraître à Moïse qu'il voyait sa forme, alors qu'il n'a pas de forme, c'est accuser Dieu d'ajouter au mensonge une sorte de tromperie de jonglerie sur son serviteur Moïse}.}[James S. White, PERGO 1.1; 1861][https://egwwritings.org/read?panels=p1471.3]


\othersQuoteNoGap{But the skeptic thinks he sees a contradiction between verse 11, which says that the Lord spake unto Moses face to face, and verse 20, which states that Moses could not see his face. But let Numbers 12:5-8 remove the difficulty. \textbf{‘And the Lord came down in the pillar of the cloud}, and stood in the door of the tabernacle, and called Aaron and Miriam, and they both came forth. And he said, Hear now my words. If there be a prophet among you, I, the Lord, will make myself known unto him in a vision, and will speak unto him in a dream. My servant Moses is not so, who is faithful in all mine house. \textbf{With him will I speak mouth to mouth, even \underline{apparently}}.’}[James S. White, PERGO 2.1; 1861][https://egwwritings.org/read?panels=p1471.6]


\othersQuoteNoGap{Mais le sceptique pense voir une contradiction entre le verset 11, qui dit que l'Éternel parlait à Moïse face à face, et le verset 20, qui déclare que Moïse ne pouvait pas voir sa face. Mais laissons Nombres 12:5-8 lever la difficulté. \textbf{« Et l'Éternel descendit dans la colonne de nuée}, et se tint à l'entrée du tabernacle, et appela Aaron et Marie, et ils sortirent tous deux. Et il dit : Écoutez maintenant mes paroles. S'il y a un prophète parmi vous, moi, l'Éternel, je me ferai connaître à lui en vision, et je lui parlerai en songe. Il n'en est pas ainsi de mon serviteur Moïse, qui est fidèle dans toute ma maison. \textbf{Je parlerai avec lui bouche à bouche, et même \underline{en apparence}}. »}[James S. White, PERGO 2.1; 1861][https://egwwritings.org/read?panels=p1471.6]


\othersQuoteNoGap{The great and dreadful God came down, wrapped in a cloud of glory. \textbf{This cloud could be seen, but not the face which possesses more dazzling brightness than a thousand suns}. Under these circumstances Moses was permitted to draw near and \textbf{converse with God face to face, or mouth to mouth, even \underline{apparently}}.}[James S. White, PERGO 2.2; 1861][https://egwwritings.org/read?panels=p1471.7]


\othersQuoteNoGap{Le Dieu grand et redoutable descendit, enveloppé dans une nuée de gloire. \textbf{Cette nuée pouvait être vue, mais pas la face qui possède plus d'éclat éblouissant que mille soleils}. Dans ces circonstances, Moïse fut autorisé à s'approcher et \textbf{converser avec Dieu face à face, ou bouche à bouche, même \underline{en apparence}}.}[James S. White, PERGO 2.2; 1861][https://egwwritings.org/read?panels=p1471.7]


\othersQuoteNoGap{Says the prophet Daniel, ‘I beheld till the thrones were cast down, and \textbf{the Ancient of days did sit}, whose garment was white as snow, \textbf{and the hairs of his head like the pure wool}; \textbf{his throne was like the fiery flame, and his wheels as burning fire}.’ Chap. 7:9. ‘I saw in the night visions, and, behold, one like the Son of man came with the clouds of heaven, and \textbf{came to the Ancient of days}, and they brought \textbf{him near before him}, and there was given him dominion and glory and a kingdom.’ Verses 13, 14.}[James S. White, PERGO 2.3; 1861][https://egwwritings.org/read?panels=p1471.8]


\othersQuoteNoGap{Dit le prophète Daniel : « Je regardai jusqu'à ce que des trônes furent placés, et \textbf{l'Ancien des jours s'assit} ; son vêtement était blanc comme la neige, et \textbf{les cheveux de sa tête étaient comme de la laine pure} ; \textbf{son trône était comme une flamme de feu, et ses roues comme un feu ardent}. » Chap. 7:9. « Je regardais dans les visions de la nuit, et voici, quelqu'un comme un fils d'homme vint avec les nuées du ciel, et \textbf{il vint jusqu'à l'Ancien des jours}, et on le fit approcher \textbf{devant lui}. Et il lui fut donné la domination, la gloire et le règne. » Versets 13, 14.}[James S. White, PERGO 2.3; 1861][https://egwwritings.org/read?panels=p1471.8]


\othersQuoteNoGap{Here is a sublime description of the action of \textbf{two personages}; viz, \textbf{God the Father, and his Son Jesus Christ}. \textbf{Deny their personality, and there is not a distinct idea in these quotations from Daniel}. In connection with this quotation read the apostle’s declaration that \textbf{the Son was in the express image of his Father’s person}. ‘God, who at sundry times, and in divers manners, spake in time past unto the fathers by the prophets, hath in these last days spoken unto us by his Son, whom he hath appointed heir of all things, by whom also he made the worlds; \textbf{who being the brightness of his glory, and the express image of his person}.’ Hebrews 1:1-3.}[James S. White, PERGO 3.1; 1861][https://egwwritings.org/read?panels=p1471.11]


\othersQuoteNoGap{Voici une description sublime de l'action de \textbf{deux personnages} ; à savoir, \textbf{Dieu le Père, et son Fils Jésus-Christ}. \textbf{Niez leur personnalité, et il n'y a pas une idée distincte dans ces citations de Daniel}. En rapport avec cette citation, lisez la déclaration de l'apôtre que \textbf{le Fils était l'empreinte de la personne de son Père}. « Dieu ayant autrefois parlé à nos pères, à plusieurs reprises et en plusieurs manières, par les prophètes, nous a parlé en ces derniers jours par son Fils, qu'il a établi héritier de toutes choses ; par lequel aussi il a fait les mondes ; \textbf{qui, étant la splendeur de sa gloire et l'empreinte de sa personne}. » Hébreux 1:1-3.}[James S. White, PERGO 3.1; 1861][https://egwwritings.org/read?panels=p1471.11]


\othersQuoteNoGap{We here add the testimony of Christ. ‘And the Father himself which hath sent me, hath borne witness of me. Ye have neither heard his voice at any time, \textbf{nor seen his shape}.’ John 5:37. See also Philippians 2:6. \textbf{To say that the Father has not a personal shape, seems the most pointed contradiction of plain scripture terms}. \\
OBJECTION. - ‘\textbf{\underline{God is a Spirit}}.’ John 4:24.}[James S. White, PERGO 3.2; 1861][https://egwwritings.org/read?panels=p1471.12]


\othersQuoteNoGap{Nous ajoutons ici le témoignage du Christ. « Et le Père lui-même, qui m'a envoyé, a rendu témoignage de moi. Vous n'avez jamais entendu sa voix, ni \textbf{vu sa forme}. » Jean 5:37. Voir aussi Philippiens 2:6. \textbf{Dire que le Père n'a pas une forme personnelle, semble la contradiction la plus pointue des termes clairs de l'Écriture}. \\
OBJECTION. - « \textbf{\underline{Dieu est Esprit}}. » Jean 4:24.}[James S. White, PERGO 3.2; 1861][https://egwwritings.org/read?panels=p1471.12]


\othersQuoteNoGap{ANSWER. - \textbf{Angels are also spirits} [Psalm 104:4], yet those that visited Abram and Lot, lay down, ate, and took hold of Lot’s hand. \textbf{They were spirit beings. So is God a Spirit being}.}[James S. White, PERGO 3.3; 1861][https://egwwritings.org/read?panels=p1471.13]


\othersQuoteNoGap{RÉPONSE. - \textbf{Les anges sont aussi des esprits} [Psaume 104:4], pourtant ceux qui ont visité Abram et Lot, se sont couchés, ont mangé, et ont pris la main de Lot. \textbf{Ils étaient des êtres spirituels. Ainsi Dieu est-il un Être spirituel}.}[James S. White, PERGO 3.3; 1861][https://egwwritings.org/read?panels=p1471.13]


\othersQuoteNoGap{OBJ. - \textbf{God is everywhere}. Proof. Psalm 139:1-8. \textbf{He is as much in every place as in any one place}.}[James S. White, PERGO 3.4; 1861][https://egwwritings.org/read?panels=p1471.14]


\othersQuoteNoGap{OBJ. - \textbf{Dieu est partout}. Preuve. Psaume 139:1-8. \textbf{Il est autant dans chaque endroit que dans n'importe quel endroit}.}[James S. White, PERGO 3.4; 1861][https://egwwritings.org/read?panels=p1471.14]


\othersQuoteNoGap{ANS. - 1. \textbf{God is everywhere by virtue of his omniscience}, as will be seen by the very words of David referred to above. Verses 1-6. ‘O Lord, \textbf{thou hast searched me, and known me}. \textbf{Thou knowest} my down-sitting and mine uprising; \textbf{thou understandest} my thought afar off. Thou compassest my path and my lying down, and art \textbf{acquainted }with all my ways. For there is not a word in my tongue, but, lo, O Lord, \textbf{thou knowest it altogether}. Thou hast beset me behind and before, and laid thy hand upon me. \textbf{Such knowledge} is too wonderful for me. It is high; I cannot attain unto it.’}[James S. White, PERGO 3.5; 1861][https://egwwritings.org/read?panels=p1471.15]


\othersQuoteNoGap{RÉP. - 1. \textbf{Dieu est partout en vertu de son omniscience}, comme on le verra par les paroles mêmes de David mentionnées ci-dessus. Versets 1-6. ‘Ô Éternel, \textbf{tu m'as sondé, et tu m'as connu}. \textbf{Tu connais} quand je m'assieds et quand je me lève ; \textbf{tu comprends} de loin ma pensée. Tu marches sur mes sentiers et sur ma couche, et tu es \textbf{instruit} de toutes mes voies. Car il n'y a pas une parole sur ma langue, que voici, ô Éternel, \textbf{tu ne la connaisses tout entière}. Tu m'entoures par derrière et par devant, et tu mets ta main sur moi. \textbf{Une telle connaissance} est trop merveilleuse pour moi. Elle est élevée ; je ne puis l'atteindre.’}[James S. White, PERGO 3.5; 1861][https://egwwritings.org/read?panels=p1471.15]


\othersQuoteNoGap{2. \textbf{God is \underline{everywhere by virtue of his Spirit}, \underline{which is his representative}, and is manifested wherever he pleases}, as will be seen by the very words the objector claims, referred to above. Verses 7-10. ‘\textbf{Whither shall I go from \underline{thy Spirit}}? \textbf{or whither shall I flee from \underline{thy presence}}? If I ascend up into heaven, thou art there; if I make my bed in hell, behold, thou art there. If I take the wings of the morning, and dwell in the uttermost parts of the sea, even there shall thy hand lead me, and thy right hand shall hold me.’}[James S. White, PERGO 4.1; 1861][https://egwwritings.org/read?panels=p1471.18]


\othersQuoteNoGap{2. \textbf{Dieu est \underline{partout en vertu de son Esprit}, \underline{qui est son représentant}, et se manifeste partout où il lui plaît}, comme on le verra par les paroles mêmes que l'objecteur revendique, mentionnées ci-dessus. Versets 7-10. ‘\textbf{Où irais-je loin de \underline{ton Esprit}}? \textbf{et où fuirais-je loin de \underline{ta présence}}? Si je monte aux cieux, tu y es ; si je me couche au séjour des morts, voici, tu y es. Si je prends les ailes de l'aurore, et que j'aille habiter à l'extrémité de la mer, là aussi ta main me conduira, et ta droite me saisira.’}[James S. White, PERGO 4.1; 1861][https://egwwritings.org/read?panels=p1471.18]


\othersQuoteNoGap{\textbf{God is in heaven.} This we are taught in the Lord’s prayer. ‘\textbf{Our Father which art in heaven}.’ Matthew 6:9; Luke 11:2. \textbf{But if God is as much in every place as he is in any one place, then heaven is also as much in every place as it is in any one place, and the idea of going to heaven is all a mistake}. We are all in heaven; and the Lord’s prayer, according to this foggy theology simply means, Our Father \textbf{which art everywhere,} hallowed be thy name. Thy kingdom come, thy will be done, on earth, \textbf{as it is everywhere}.}[James S. White, PERGO 4.2; 1861][https://egwwritings.org/read?panels=p1471.19]


\othersQuoteNoGap{\textbf{Dieu est au ciel.} C'est ce qu'on nous enseigne dans la prière du Seigneur. ‘\textbf{Notre Père qui es aux cieux}.’ Matthieu 6:9 ; Luc 11:2. \textbf{Mais si Dieu est autant dans chaque endroit qu'il est dans n'importe quel endroit, alors le ciel est aussi autant dans chaque endroit qu'il est dans n'importe quel endroit, et l'idée d'aller au ciel est toute une erreur}. Nous sommes tous au ciel ; et la prière du Seigneur, selon cette théologie brumeuse signifie simplement, Notre Père \textbf{qui es partout,} que ton nom soit sanctifié. Que ton règne vienne, que ta volonté soit faite, sur la terre, \textbf{comme elle l'est partout}.}[James S. White, PERGO 4.2; 1861][https://egwwritings.org/read?panels=p1471.19]


\othersQuoteNoGap{Again, Bible readers have believed that Enoch and Elijah were really taken up \textbf{to God in heaven}. \textbf{But if God and heaven be as much in every place as in any one place, this is all a mistake}. They were not translated. And all that is said about the chariot of fire, and horses of fire, and the attending whirlwind to take Elijah up into heaven, was a useless parade. They only evaporated, and a misty vapor passed through the entire universe. This is all of Enoch and Elijah that the mind can possibly grasp, \textbf{admitting that God and heaven are no more in any one place than in every place}. But it is said of Elijah that he ‘\textbf{went up} by a whirlwind \textbf{into heaven}.’ 2 Kings 2:11. And of Enoch it is said that he ‘walked with God, and was not, for God took him.’ Genesis 5:24.}[James S. White, PERGO 4.3; 1861][https://egwwritings.org/read?panels=p1471.20]


\othersQuoteNoGap{Encore, les lecteurs de la Bible ont cru qu'Hénoc et Élie ont été réellement enlevés \textbf{vers Dieu au ciel}. \textbf{Mais si Dieu et le ciel sont autant dans chaque endroit que dans n'importe quel endroit, tout cela est une erreur}. Ils n'ont pas été enlevés. Et tout ce qui est dit au sujet du char de feu, et des chevaux de feu, et du tourbillon qui accompagnait pour enlever Élie au ciel, n'était qu'une parade inutile. Ils se sont seulement évaporés, et une vapeur brumeuse a traversé l'univers entier. C'est tout ce qu'il y a d'Hénoc et d'Élie que l'esprit peut possiblement saisir, \textbf{en admettant que Dieu et le ciel ne sont pas plus dans un endroit que dans chaque endroit}. Mais il est dit d'Élie qu'il ‘\textbf{monta} dans un tourbillon \textbf{au ciel}.’ 2 Rois 2:11. Et d'Hénoc il est dit qu'il ‘marcha avec Dieu, et il ne fut plus, car Dieu le prit.’ Genèse 5:24.}[James S. White, PERGO 4.3; 1861][https://egwwritings.org/read?panels=p1471.20]


\othersQuoteNoGap{\textbf{Jesus is said to be on the right hand of the Majesty on high}. Hebrews 1:3. ‘So, then, after the Lord had spoken unto them \textbf{he was received \underline{up into heaven}}, \textbf{and sat on the right hand of God}.’ Mark 16:19. \textbf{But if heaven be everywhere, and God everywhere, then Christ’s ascension up to heaven, at the Father’s right hand, simply means that he went everywhere}! He was only taken up where the cloud hid him from the gaze of his disciples, and then evaporated and went everywhere! So that instead of the lovely Jesus, so beautifully described in both Testaments, we have only a sort of essence dispersed through the entire universe. And in harmony with this rarified theology, Christ’s second advent, or his return, would be the condensation of this essence to some locality, say the mount of Olivet! \textbf{Christ arose from the dead with a physical form}. ‘He is not here,’ said the angel, ‘for he is risen as he said.’ Matthew 28:6.}[James S. White, PERGO 5.1; 1861][https://egwwritings.org/read?panels=p1471.23]


\othersQuoteNoGap{\textbf{Il est dit que Jésus est à la droite de la Majesté dans les lieux très hauts}. Hébreux 1:3. ‘Le Seigneur donc, après leur avoir parlé, \textbf{fut enlevé \underline{au ciel}}, \textbf{et il s'assit à la droite de Dieu}.’ Marc 16:19. \textbf{Mais si le ciel est partout, et Dieu partout, alors l'ascension du Christ au ciel, à la droite du Père, signifie simplement qu'il est allé partout}! Il a seulement été enlevé là où la nuée l'a caché du regard de ses disciples, et puis s'est évaporé et est allé partout ! De sorte qu'au lieu du beau Jésus, si magnifiquement décrit dans les deux Testaments, nous n'avons qu'une sorte d'essence dispersée à travers l'univers entier. Et en harmonie avec cette théologie raréfiée, le second avènement du Christ, ou son retour, serait la condensation de cette essence en quelque localité, disons le mont des Oliviers ! \textbf{Christ est ressuscité des morts avec une forme physique}. ‘Il n'est point ici,’ dit l'ange, ‘car il est ressuscité comme il l'avait dit.’ Matthieu 28:6.}[James S. White, PERGO 5.1; 1861][https://egwwritings.org/read?panels=p1471.23]


\othersQuoteNoGap{‘And as they went to tell his disciples, behold, Jesus met them, saying, All hail! And they came and \textbf{held him by the feet}, and they worshiped him.’ Verse 9.}[James S. White, PERGO 5.2; 1861][https://egwwritings.org/read?panels=p1471.24]


\othersQuoteNoGap{‘Et comme elles allaient pour l'annoncer à ses disciples, voici, Jésus vint au-devant d'elles, disant : Je vous salue ! Et elles s'approchèrent et \textbf{embrassèrent ses pieds}, et elles l'adorèrent.’ Verset 9.}[James S. White, PERGO 5.2; 1861][https://egwwritings.org/read?panels=p1471.24]


\othersQuoteNoGap{‘\textbf{Behold my hands and my feet},’ said Jesus to those who stood in doubt of his resurrection, ‘that it is I myself. \textbf{Handle me and see, \underline{for a spirit hath not flesh and bones} as ye see me have}. And when he had thus spoken, he \textbf{showed them his hands and his feet}. And while they yet believed not for joy, and wondered, he said unto them, Have ye here any meat? And they gave him a piece of broiled fish, and of an honey-comb, and he took it and did eat before them.’ Luke 24:39-43.}[James S. White, PERGO 5.3; 1861][https://egwwritings.org/read?panels=p1471.25]


\othersQuoteNoGap{‘\textbf{Voyez mes mains et mes pieds},’ dit Jésus à ceux qui doutaient de sa résurrection, ‘c'est bien moi. \textbf{Touchez-moi et voyez, \underline{car un esprit n'a ni chair ni os} comme vous voyez que j'ai}. Et en disant cela, il leur \textbf{montra ses mains et ses pieds}. Et comme ils ne croyaient point encore, tant ils étaient transportés de joie et d'admiration, il leur dit : Avez-vous ici quelque chose à manger ? Et ils lui présentèrent un morceau de poisson rôti et un rayon de miel, et l'ayant pris, il en mangea devant eux.’ Luc 24:39-43.}[James S. White, PERGO 5.3; 1861][https://egwwritings.org/read?panels=p1471.25]


\othersQuoteNoGap{After Jesus addressed his disciples on the mount of Olivet, he \textbf{was taken up from them}, and a cloud received him out of their sight. ‘And while they looked steadfastly \textbf{toward heaven as he went up,} behold two men stood by them in white apparel, which also said, Ye men of Galilee, why stand ye gazing up into heaven? This same Jesus which is \textbf{taken up from you into heaven}, shall so come in like manner as ye have seen him \textbf{go into heaven}.’ Acts 1:9-11. J. W.}[James S. White, PERGO 6.1; 1861][https://egwwritings.org/read?panels=p1471.27]


\othersQuoteNoGap{Après que Jésus eut parlé à ses disciples sur le mont des Oliviers, il \textbf{fut enlevé d'avec eux}, et une nuée le déroba à leurs yeux. ‘Et comme ils avaient les yeux attachés \textbf{au ciel pendant qu'il s'en allait}, voici, deux hommes se présentèrent à eux en vêtements blancs, et leur dirent : Hommes galiléens, pourquoi vous tenez-vous là à regarder au ciel ? Ce même Jésus, qui a été \textbf{enlevé d'avec vous dans le ciel}, viendra de la même manière que vous l'avez vu \textbf{aller au ciel}.’ Actes 1:9-11. J. W.}[James S. White, PERGO 6.1; 1861][https://egwwritings.org/read?panels=p1471.27]


James White fights the idea that God is just a spirit, and as such, is present \others{as much in every place as in any one place}. He gives plain and positive testimony from Scripture that God is a personal being; we see the very same sentiments in Ellen White’s writings.


James White combat l'idée que Dieu est juste un esprit, et en tant que tel, est présent \others{autant dans chaque endroit que dans n'importe quel endroit}. Il donne un témoignage clair et positif de l'Écriture que Dieu est un être personnel ; nous voyons les mêmes raisonnements dans les écrits d'Ellen White.


\egw{The mighty power that works through all nature and sustains all things is not, as some men of science claim, \textbf{merely an all-pervading principle}, an actuating energy. \textbf{\underline{God is a spirit; yet He is a personal being}}, \textbf{for man was made in His image}. \textbf{As \underline{a personal being}}, God has revealed Himself in His Son. Jesus, the outshining of the Father’s glory, “and \textbf{the express \underline{image of His person}}” (Hebrews 1:3), was on earth found in fashion as a man. As \textbf{a personal Saviour} He came to the world. As \textbf{a personal Saviour He ascended \underline{on high}}. As \textbf{a personal Saviour He intercedes \underline{in the heavenly courts}}. \textbf{Before the throne of God} in our behalf ministers “One like the Son of man.” Daniel 7:13.}[Ed 131.5; 1903][https://egwwritings.org/read?panels=p29.632]


\egw{La puissance formidable qui œuvre à travers toute la nature et soutient toutes choses n'est pas, comme certains hommes de science le prétendent, \textbf{simplement un principe omniprésent}, une énergie activatrice. \textbf{\underline{Dieu est un esprit ; pourtant Il est un être personnel}}, \textbf{car l'homme a été fait à Son image}. \textbf{En tant qu’\underline{être personnel}}, Dieu s'est révélé dans Son Fils. Jésus, le rayonnement de la gloire du Père, « et \textbf{l'empreinte de sa personne} » (Hébreux 1:3), a été trouvé sur terre sous la forme d'un homme. En tant que \textbf{Sauveur personnel}, Il est venu dans le monde. En tant que \textbf{Sauveur personnel, Il est monté \underline{en haut}}. En tant que \textbf{Sauveur personnel, Il intercède \underline{dans les cours célestes}}. \textbf{Devant le trône de Dieu} en notre faveur officie « Quelqu'un de semblable au Fils de l'homme. » Daniel 7:13.}[Ed 131.5; 1903][https://egwwritings.org/read?panels=p29.632]


Ellen White and the Adventist pioneers made a distinction between the terms ‘\textit{spirit}’ and ‘\textit{being}’. God is a personal being, not just a spirit. He is not\others{as much in every place as in any one place}, but He is\others{in one place more than another}[John. N. Loughborough, “Is God a Person?” The Adventist Review and Sabbath Herald, September 18, 1855][https://documents.adventistarchives.org/Periodicals/RH/RH18550918-V07-06.pdf]. He is in heaven, in His temple, sitting on His throne—in person—and He is everywhere present by His representative, the Holy Spirit.


Ellen White et les pionniers adventistes faisaient une distinction entre les termes « \textit{esprit} » et « \textit{être} ». Dieu est un être personnel, pas seulement un esprit. Il n'est pas \others{autant dans chaque endroit que dans n'importe quel endroit}, mais Il est \others{dans un endroit plus que dans un autre}[John. N. Loughborough, “Is God a Person?” The Adventist Review and Sabbath Herald, September 18, 1855][https://documents.adventistarchives.org/Periodicals/RH/RH18550918-V07-06.pdf]. Il est au ciel, dans Son temple, assis sur Son trône — en personne — et Il est partout présent par Son représentant, le Saint-Esprit.


Here are some other quotations from Sister White that are in harmony with the pioneers’ views on the \emcap{personality of God}:


Voici quelques autres citations de Sœur White qui sont en harmonie avec les vues des pionniers sur la \emcap{personnalité de Dieu} :


\egw{He \normaltext{[Jesus]} taught that God was a rewarder of the righteous, and a punisher of the transgressor. \textbf{He was not an intangible spirit}, but a living ruler of the universe. \textbf{This gracious Father} was constantly working for the good of man, and mindful of all that concerns him...}[3SP 47.1; 1878][https://egwwritings.org/read?panels=p142.195]


\egw{Il \normaltext{[Jésus]} enseignait que Dieu était celui qui récompense les justes, et celui qui punit les transgresseurs. \textbf{Il n'était pas un esprit intangible}, mais un dirigeant vivant de l'univers. \textbf{Ce Père plein de grâce} travaillait constamment pour le bien de l'homme, et attentif à tout ce qui le concerne...}[3SP 47.1; 1878][https://egwwritings.org/read?panels=p142.195]


\egw{\textbf{The Bible shows us \underline{God in His high and holy place}}, not in a state of inactivity, not in silence and solitude, but surrounded by ten thousand times ten thousand and thousands of thousands of holy beings, all waiting to do His will. \textbf{Through these messengers He is in active communication with every part of His dominion}. \textbf{\underline{By His Spirit He is everywhere present}}. \textbf{Through the agency of His Spirit and His angels} He ministers to the children of men.}[MH 417.2; 1905][https://egwwritings.org/read?panels=p135.2136]


\egw{\textbf{La Bible nous montre \underline{Dieu dans Son lieu élevé et saint}}, non dans un état d'inactivité, non dans le silence et la solitude, mais entouré de dix mille fois dix mille et de milliers de milliers d'êtres saints, tous attendant de faire Sa volonté. \textbf{Par ces messagers, Il est en communication active avec chaque partie de Son domaine}. \textbf{\underline{Par Son Esprit, Il est partout présent}}. \textbf{Par l'intermédiaire de Son Esprit et de Ses anges}, Il exerce Son ministère auprès des enfants des hommes.}[MH 417.2; 1905][https://egwwritings.org/read?panels=p135.2136]


\egw{The greatness of God is to us incomprehensible. ‘\textbf{The Lord’s throne is in heaven}’ (Psalm 11:4); \textbf{\underline{yet by His Spirit He is everywhere present}}. \textbf{He has an intimate knowledge} of, and a personal interest in, all the works of His hand.}[Ed 132.2; 1903][https://egwwritings.org/read?panels=p29.636]


\egw{La grandeur de Dieu nous est incompréhensible. « \textbf{Le trône de l'Éternel est dans les cieux} » (Psaume 11:4) ; \textbf{\underline{pourtant par Son Esprit Il est partout présent}}. \textbf{Il a une connaissance intime} de toutes les œuvres de Sa main, et un intérêt personnel pour elles.}[Ed 132.2; 1903][https://egwwritings.org/read?panels=p29.636]


\egw{Through Jesus Christ, \textbf{God—not a perfume, \underline{not something intangible}, \underline{but a personal God}}—created man and endowed him with intelligence and power.}[Ms117-1898.10; 1898][https://egwwritings.org/read?panels=p7182.15]


\egw{Par Jésus-Christ, \textbf{Dieu — non un parfum, \underline{non quelque chose d'intangible}, \underline{mais un Dieu personnel}} — a créé l'homme et l'a doté d'intelligence et de puissance.}[Ms117-1898.10; 1898][https://egwwritings.org/read?panels=p7182.15]


Continuing in James White’s pamphlet, we read his sharp criticism on the notion of an immaterial God. Before that, let’s briefly recall Dr. Kellogg’s argument that\others{\textbf{\underline{Discussions respecting the form of God are utterly unprofitable}}}[Dr. John H. Kellogg, The Living Temple, p.33.][https://archive.org/details/J.H.Kellogg.TheLivingTemple1903/page/n33/] because God is\others{\textbf{far beyond our comprehension }\textbf{\underline{as are the bounds of space and time}}}. He believed that God’s person is not constrained to one locality because He is in\others{as much in every place as in any one place}[James S. White, PERGO 4.3; 1861][https://egwwritings.org/read?panels=p1471.20] \footnote{In the Living Temple, Dr. Kellogg objected that God cannot be everywhere presente at once: “\textit{Says one}, ‘God may be present by his Spirit, or by his power, but certainly God himself \textit{cannot be present everywhere at once}.’ We answer: How can power be separated from the source of power? Where God’s Spirit is at work, where God’s power is manifested, God \textit{himself is actually and truly present}…” \href{https://archive.org/details/J.H.Kellogg.TheLivingTemple1903/page/n29/}{John H. Kellogg, The Living Temple, p.28}.}. If God in His personality were truly a definite being, having a tangible body, then He would not be able to be present\others{as much in every place as in any one place} and, thus, sustain life. James White continues against the reasoning that God is immaterial in His person.


En continuant dans la brochure de James White, nous lisons sa critique acerbe sur la notion d'un Dieu immatériel. Avant cela, rappelons brièvement l'argument du Dr Kellogg selon lequel \others{\textbf{\underline{Les discussions concernant la forme de Dieu sont totalement inutiles}}}[Dr. John H. Kellogg, The Living Temple, p.33.][https://archive.org/details/J.H.Kellogg.TheLivingTemple1903/page/n33/] parce que Dieu est \others{\textbf{bien au-delà de notre compréhension }\textbf{\underline{comme le sont les limites de l'espace et du temps}}}. Il croyait que la personne de Dieu n'est pas limitée à une localité parce qu'Il est \others{autant dans chaque endroit que dans n'importe quel endroit}[James S. White, PERGO 4.3; 1861][https://egwwritings.org/read?panels=p1471.20] \footnote{Dans Le Temple Vivant, le Dr Kellogg s'opposait à l'idée que Dieu ne peut pas être partout présent à la fois : « \textit{Quelqu'un dit}, ‘Dieu peut être présent par son Esprit, ou par sa puissance, mais certainement Dieu lui-même \textit{ne peut pas être présent partout à la fois}.’ Nous répondons : Comment la puissance peut-elle être séparée de la source de puissance ? Là où l'Esprit de Dieu est à l'œuvre, là où la puissance de Dieu est manifestée, Dieu \textit{lui-même est réellement et véritablement présent}… » \href{https://archive.org/details/J.H.Kellogg.TheLivingTemple1903/page/n29/}{John H. Kellogg, The Living Temple, p.28}.}. Si Dieu dans Sa personnalité était vraiment un être défini, ayant un corps tangible, alors Il ne serait pas capable d'être présent \others{autant dans chaque endroit que dans n'importe quel endroit} et, ainsi, de soutenir la vie. James White continue contre le raisonnement que Dieu est immatériel dans Sa personne.


\othersQuote{IMMATERIALITY}


\othersQuote{IMMATÉRIALITÉ}


\othersQuoteNoGap{\textbf{THIS is but another name for nonentity}. \textbf{It is the negative of all} \textbf{things and} \textbf{\underline{beings} }- of all existence. There is not one particle of proof to be advanced to establish its existence. It has no way to manifest itself to any intelligence in heaven or on earth. \textbf{Neither God, angels, nor men could possibly conceive of such a substance, being, or thing}. \textbf{It possesses no property or power by which \underline{to make itself manifest to any intelligent being} in the universe}. Reason and analogy never scan it, or even conceive of it. \textbf{Revelation never reveals it, nor do any of our senses witness its existence}. \textbf{It cannot be seen, felt, heard, tasted, or smelled, even by the strongest organs, or the most acute sensibilities}. It is neither liquid nor solid, soft nor hard - it can neither extend nor contract. In short, it can exert no influence whatever - it can neither act nor be acted upon. And even if it does exist, it can be of no possible use. It possesses no one, desirable property, faculty, or use, yet, strange to say, \textbf{immateriality is the modern Christian’s God}, \textbf{his anticipated heaven}, \textbf{his immortal self} - \textbf{his all}!}[James S. White, PERGO 6.2; 1861][https://egwwritings.org/read?panels=p1471.29]


\othersQuoteNoGap{\textbf{CECI n'est qu'un autre nom pour la non-entité}. \textbf{C'est le négatif de toutes} \textbf{choses et} \textbf{\underline{êtres}} - de toute existence. Il n'y a pas une seule particule de preuve à avancer pour établir son existence. Elle n'a aucun moyen de se manifester à aucune intelligence au ciel ou sur terre. \textbf{Ni Dieu, ni les anges, ni les hommes ne pourraient possiblement concevoir une telle substance, être, ou chose}. \textbf{Elle ne possède aucune propriété ou pouvoir par lequel \underline{se faire manifeste à tout être intelligent} dans l'univers}. La raison et l'analogie ne la scrutent jamais, ni même ne la conçoivent. \textbf{La révélation ne la révèle jamais, et aucun de nos sens ne témoigne de son existence}. \textbf{Elle ne peut être vue, sentie, entendue, goûtée, ou sentie, même par les organes les plus forts, ou les sensibilités les plus aiguës}. Elle n'est ni liquide ni solide, ni molle ni dure - elle ne peut ni s'étendre ni se contracter. En bref, elle ne peut exercer aucune influence quelconque - elle ne peut ni agir ni être agie. Et même si elle existe, elle ne peut être d'aucune utilité possible. Elle ne possède aucune propriété, faculté, ou utilité désirable, pourtant, chose étrange à dire, \textbf{l'immatérialité est le Dieu du chrétien moderne}, \textbf{son ciel anticipé}, \textbf{son moi immortel} - \textbf{son tout}!}[James S. White, PERGO 6.2; 1861][https://egwwritings.org/read?panels=p1471.29]


\othersQuoteNoGap{\textbf{O sectarianism! O atheism!! O annihilation!!!} \textbf{who can perceive the nice shades of difference between the one and the other?} They seem alike, all but in name. \textbf{The atheist has no God. \underline{The sectarian has a God without body or parts}.} Who can define the difference? For our part we do not perceive a difference of a single hair; \textbf{they both claim to be the negative of all things which exist} - and both are equally powerless and unknown.}[James S. White, PERGO 6.3; 1861][https://egwwritings.org/read?panels=p1471.30]


\othersQuoteNoGap{\textbf{Ô sectarisme! Ô athéisme!! Ô annihilation!!!} \textbf{qui peut percevoir les nuances subtiles de différence entre l'un et l'autre?} Ils semblent pareils, sauf en nom. \textbf{L'athée n'a pas de Dieu. \underline{Le sectaire a un Dieu sans corps ni parties}.} Qui peut définir la différence? Pour notre part, nous ne percevons pas une différence d'un seul cheveu; \textbf{ils prétendent tous deux être le négatif de toutes les choses qui existent} - et tous deux sont également impuissants et inconnus.}[James S. White, PERGO 6.3; 1861][https://egwwritings.org/read?panels=p1471.30]


\othersQuoteNoGap{\textbf{The atheist has no after life, or conscious existence beyond the grave. The sectarian has one, \underline{but it is immaterial, like his God; and without body or parts}. Here again both are negative, and both arrive at the same point}. Their faith and hope amount to the same; only it is expressed by different terms.}[James S. White, PERGO 7.1; 1861][https://egwwritings.org/read?panels=p1471.33]


\othersQuoteNoGap{\textbf{L'athée n'a pas de vie après la mort, ou d'existence consciente au-delà de la tombe. Le sectaire en a une, \underline{mais elle est immatérielle, comme son Dieu; et sans corps ni parties}. Ici encore, tous deux sont négatifs, et tous deux arrivent au même point}. Leur foi et leur espérance reviennent au même; seulement c'est exprimé par des termes différents.}[James S. White, PERGO 7.1; 1861][https://egwwritings.org/read?panels=p1471.33]


\othersQuoteNoGap{Again, \textbf{the atheist has no heaven in eternity}. \textbf{The sectarian has one, but it is \underline{immaterial in all its properties}, and is therefore the negative of all riches and substances}. Here again they are equal, and arrive at the same point.}[James S. White, PERGO 7.2; 1861][https://egwwritings.org/read?panels=p1471.34]


\othersQuoteNoGap{Encore, \textbf{l'athée n'a pas de ciel dans l'éternité}. \textbf{Le sectaire en a un, mais il est \underline{immatériel dans toutes ses propriétés}, et est donc le négatif de toutes richesses et substances}. Ici encore, ils sont égaux, et arrivent au même point.}[James S. White, PERGO 7.2; 1861][https://egwwritings.org/read?panels=p1471.34]


\othersQuoteNoGap{As we do not envy them the possession of all they claim, we will now leave them in the quiet and undisturbed enjoyment of the same, and proceed to examine the portion still left for the despised materialist to enjoy.}[James S. White, PERGO 7.3; 1861][https://egwwritings.org/read?panels=p1471.35]


\othersQuoteNoGap{Comme nous ne leur envions pas la possession de tout ce qu'ils réclament, nous allons maintenant les laisser dans la jouissance tranquille et non perturbée de cela, et procéder à examiner la portion encore laissée au matérialiste méprisé pour en jouir.}[James S. White, PERGO 7.3; 1861][https://egwwritings.org/read?panels=p1471.35]


\othersQuoteNoGap{\textbf{What is God? He is material, organized intelligence, \underline{possessing both body and parts}. Man is in his image.}}[James S. White, PERGO 7.4; 1861][https://egwwritings.org/read?panels=p1471.36]


\othersQuoteNoGap{\textbf{Qu'est-ce que Dieu? Il est une intelligence matérielle, organisée, \underline{possédant à la fois corps et parties}. L'homme est à son image.}}[James S. White, PERGO 7.4; 1861][https://egwwritings.org/read?panels=p1471.36]


\othersQuoteNoGap{\textbf{What is Jesus Christ? He is the Son of God, and is \underline{like his Father}, being ‘the brightness of his Father’s glory, and the express image of his person.’ \underline{He is a material intelligence, with body, parts}, and passions; possessing immortal flesh and immortal bones}.}[James S. White, PERGO 7.5; 1861][https://egwwritings.org/read?panels=p1471.37]


\othersQuoteNoGap{\textbf{Qu'est-ce que Jésus-Christ? Il est le Fils de Dieu, et est \underline{comme son Père}, étant ‘l'éclat de la gloire de son Père, et l'empreinte de sa personne.’ \underline{Il est une intelligence matérielle, avec corps, parties}, et passions; possédant une chair immortelle et des os immortels}.}[James S. White, PERGO 7.5; 1861][https://egwwritings.org/read?panels=p1471.37]


\othersQuoteNoGap{\textbf{What are men?} They are the offspring of Adam. \textbf{They are capable of receiving intelligence and exaltation to such a degree as to be \underline{raised from the dead with a body like that of Jesus Christ}, \underline{and to possess immortal flesh and bones}}. Thus perfected, they will possess \textbf{the material universe}, that is, the earth, as their ‘everlasting inheritance.’ With these hopes and prospects before us, we say to the Christian world who hold to immateriality, that they are welcome to their God - their life - their heaven, and their all. They claim nothing but that which we throw away; and we claim nothing but that which they throw away. \textbf{Therefore, there is no ground for quarrel or contention between us}.}[James S. White, PERGO 7.6; 1861][https://egwwritings.org/read?panels=p1471.38]


\othersQuoteNoGap{\textbf{Que sont les hommes?} Ils sont la progéniture d'Adam. \textbf{Ils sont capables de recevoir l'intelligence et l'exaltation à un tel degré qu'ils peuvent être \underline{ressuscités des morts avec un corps comme celui de Jésus-Christ}, \underline{et posséder une chair et des os immortels}}. Ainsi perfectionnés, ils posséderont \textbf{l'univers matériel}, c'est-à-dire la terre, comme leur ‘héritage éternel.’ Avec ces espoirs et perspectives devant nous, nous disons au monde chrétien qui tient à l'immatérialité, qu'ils sont les bienvenus à leur Dieu - leur vie - leur ciel, et leur tout. Ils ne réclament rien d'autre que ce que nous jetons; et nous ne réclamons rien d'autre que ce qu'ils jettent. \textbf{Par conséquent, il n'y a aucun motif de querelle ou de contention entre nous}.}[James S. White, PERGO 7.6; 1861][https://egwwritings.org/read?panels=p1471.38]


\othersQuoteNoGap{We choose all substance - what remains \\
The mystical sectarian gains; \\
All that each claims, each shall possess, \\
Nor grudge each other’s happiness. \\
An immaterial God they choose, \\
For such a God we have no use; \\
\textbf{An immaterial heaven and hell,} \\
\textbf{In such a heaven we cannot dwell.} \\
\textbf{We claim the earth, the air, and sky,} \\
\textbf{And all the starry worlds on high;} \\
\textbf{Gold, silver, ore, and precious stones,} \\
\textbf{And bodies made of flesh and bones.} \\
\textbf{Such is our hope, our heaven, our all,} \\
\textbf{When once redeemed from Adam’s fall;} \\
\textbf{All things are ours, and we shall be,} \\
\textbf{The Lord’s to all eternity}.}[James S. White, PERGO 8.1; 1861][https://egwwritings.org/read?panels=p1471.41]


\othersQuoteNoGap{Nous choisissons toute substance - ce qui reste \\
Le sectaire mystique gagne; \\
Tout ce que chacun réclame, chacun possédera, \\
Sans envier le bonheur de l'autre. \\
Un Dieu immatériel ils choisissent, \\
Pour un tel Dieu nous n'avons aucun usage; \\
\textbf{Un ciel et un enfer immatériels,} \\
\textbf{Dans un tel ciel nous ne pouvons demeurer.} \\
\textbf{Nous réclamons la terre, l'air, et le ciel,} \\
\textbf{Et tous les mondes étoilés en haut;} \\
\textbf{Or, argent, minerai, et pierres précieuses,} \\
\textbf{Et des corps faits de chair et d'os.} \\
\textbf{Tel est notre espoir, notre ciel, notre tout,} \\
\textbf{Une fois rachetés de la chute d'Adam;} \\
\textbf{Toutes choses sont à nous, et nous serons,} \\
\textbf{Au Seigneur pour toute l'éternité}.}[James S. White, PERGO 8.1; 1861][https://egwwritings.org/read?panels=p1471.41]


James White compared the sentiments on the immaterial God with sectarianism, atheism, and annihilation. “\textit{Immaterial God}” is another expression for the nonentity of God. James White never received any reproof from Sister White for these views; rather, they were supported by her writings. Many assert that Sister White changed her views over time and, later, accepted the Trinity doctrine, but this is not backed up by detailed historical records. In 1905, Sister White recalls the occasion with Dr. Kellogg when, twenty years prior, he came to her with the very sentiments regarding the \emcap{personality of God} that James White and other pioneers were refuting:


James White a comparé le raisonnement sur le Dieu immatériel avec le sectarisme, l'athéisme, et l'annihilation. “\textit{Dieu immatériel}” est une autre expression pour la non-entité de Dieu. James White n'a jamais reçu de reproche de Sœur White pour ces vues; au contraire, elles étaient soutenues par ses écrits. Beaucoup affirment que Sœur White a changé ses vues au fil du temps et, plus tard, a accepté la doctrine de la Trinité, mais cela n'est pas soutenu par des documents historiques détaillés. En 1905, Sœur White rappelle l'occasion avec le Dr Kellogg quand, vingt ans auparavant, il est venu à elle avec les mêmes raisonnements concernant la \emcap{personnalité de Dieu} que James White et d'autres pionniers réfutaient:


\egw{Now this subject has been kept before me for more than twenty years. My husband has been dead twenty years, and before he died, things came in. Dr. Kellogg came into my room; I was occupying one of the large rooms at the office as my home. I had two or three rooms there, and \textbf{he got a great light}; and he sat down and told what his light was: \textbf{it is just the same theories or errors, the same sophistries, that he is presenting, and did present in ‘Living Temple.’} I said, ‘Dr. Kellogg, \textbf{I have met that.}’ I met it when I first started out to travel. I met it in the North; I met it in New Hampshire. I saw the curse of its influence in Massachusetts, and \textbf{the testimonies that were given to me were right to the point that we were not to have anything of this kind to be taught in our churches}. And I talked with him. \textbf{I gave the history}—I have not time to give it to you here.\textbf{ I gave him the history of how that was treated by the Spirit of God, and how we as a people must escape the sophistries and delusions}. And it was ministers that were deceiving the people with these sophistries. \textbf{I will not tell you what they led to}—\textbf{it may have to come}; but I will not tell you now what they led to; \textbf{but I will tell you what this sophistry leads to:} \textbf{It leads to \underline{the nonentity of Christ, to the nonentity of God}, \underline{his personality}, and brings in,—what shall I call it?—a sort of \underline{manufactured theory of God and Christ}}.}[Ms70a-1905.11; 1905][https://egwwritings.org/read?panels=p12696.17]


\egw{Maintenant, ce sujet a été gardé devant moi pendant plus de vingt ans. Mon mari est mort il y a vingt ans, et avant qu'il ne meure, des choses sont arrivées. Le Dr Kellogg est entré dans ma chambre ; j'occupais l'une des grandes pièces du bureau comme ma maison. J'avais deux ou trois pièces là-bas, et \textbf{il a reçu une grande lumière} ; et il s'est assis et a dit quelle était sa lumière : \textbf{ce sont exactement les mêmes théories ou erreurs, les mêmes sophismes, qu'il présente, et qu'il a présentés dans ‘Le Temple Vivant.’} J'ai dit : ‘Dr Kellogg, \textbf{j'ai rencontré cela.}’ Je l'ai rencontré quand j'ai commencé à voyager. Je l'ai rencontré dans le Nord ; je l'ai rencontré dans le New Hampshire. J'ai vu la malédiction de son influence dans le Massachusetts, et \textbf{les témoignages qui m'ont été donnés étaient directement au point que nous ne devions rien avoir de ce genre à enseigner dans nos églises}. Et j'ai parlé avec lui. \textbf{J'ai donné l'histoire}—je n'ai pas le temps de vous la donner ici.\textbf{ Je lui ai donné l'histoire de comment cela a été traité par l'Esprit de Dieu, et comment nous, en tant que peuple, devons échapper aux sophismes et aux illusions}. Et c'étaient des pasteurs qui trompaient le peuple avec ces sophismes. \textbf{Je ne vous dirai pas à quoi ils ont mené}—\textbf{cela pourrait devoir venir} ; mais je ne vous dirai pas maintenant à quoi ils ont mené ; \textbf{mais je vous dirai à quoi ce sophisme mène :} \textbf{Il mène à \underline{la non-entité du Christ, à la non-entité de Dieu}, \underline{sa personnalité}, et apporte,—comment l'appellerai-je ?—une sorte de \underline{théorie fabriquée de Dieu et du Christ}}.}[Ms70a-1905.11; 1905][https://egwwritings.org/read?panels=p12696.17]


Kellogg’s sentiment in the Living Temple regarding the \emcap{personality of God} leads to the nonentity of Christ and the nonentity of God. Why? Because his views of God claim an immaterial God. The church was faced with such sentiments in the beginning of their work. James White wrote about them in his pamphlet “\textit{The Personality of God}”, and Sister White recalled these early experiences when she and her husband combatted the error that God is an immaterial, all-prevailing spirit.


Le raisonnement de Kellogg dans Le Temple Vivant concernant la \emcap{personnalité de Dieu} mène à la non-entité du Christ et à la non-entité de Dieu. Pourquoi ? Parce que ses vues de Dieu revendiquent un Dieu immatériel. L'église a été confrontée à de tels raisonnements au début de leur œuvre. James White a écrit à leur sujet dans sa brochure « \textit{La Personnalité de Dieu} », et Sœur White s'est rappelée de ces premières expériences quand elle et son mari ont combattu l'erreur que Dieu est un esprit immatériel, omniprésent.


I've examined the text and found no grammar issues that need correction. The text is well-formatted with proper LaTeX directives and maintains grammatical correctness throughout.

The text discusses James White's pamphlet “The Personality of God” and contains numerous quotations with proper formatting. All sentences are grammatically sound, with appropriate subject-verb agreement, proper use of articles, and correct punctuation.

The LaTeX formatting (such as \textbf{}, \underline{}, \emcap{}, etc.) is used consistently and correctly throughout the document, and I've preserved all of these elements as instructed.