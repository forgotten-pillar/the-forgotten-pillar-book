\qrchapter{https://forgottenpillar.com/rsc/en-fp-chapter7}{The authority of the Fundamental Principles} \label{chap:authority}


\qrchapter{https://forgottenpillar.com/rsc/en-fp-chapter7}{L'autorité des Principes Fondamentaux} \label{chap:authority}


In the 10th chapter of the Special Testimonies, we read how God established the foundation of our faith. Sister White used several different expressions for the foundation of our faith. Her references included: “\textit{a platform of eternal truth}, \textit{“pillars of our faith”}, \textit{“principles of truth”}, \textit{“principal points”}, \textit{“waymarks”}, and “\textit{foundation principles}—all of these refer to the \emcap{Fundamental Principles}. At the end of the chapter, she affirmed the will of God that \egwinline{He calls upon us to hold firmly, with the grip of faith, to \textbf{the fundamental principles} that are \textbf{based upon unquestionable \underline{authority}}.}[SpTB02 59.1; 1904][https://egwwritings.org/read?panels=p417.299]


Dans le 10ème chapitre des Témoignages Spéciaux, nous lisons comment Dieu a établi le fondement de notre foi. Sœur White a utilisé plusieurs expressions différentes pour le fondement de notre foi. Ses références incluaient : “\textit{une plateforme de vérité éternelle}, \textit{“piliers de notre foi”}, \textit{“principes de vérité”}, \textit{“points principaux”}, \textit{“repères”}, et “\textit{les principes fondamentaux}—tous ceux-ci se réfèrent aux \emcap{Principes Fondamentaux}. À la fin du chapitre, elle a affirmé la volonté de Dieu qu’\egwinline{Il nous appelle à tenir fermement, avec la prise de la foi, aux \textbf{principes fondamentaux} qui sont \textbf{basés sur une \underline{autorité} incontestable}.}[SpTB02 59.1; 1904][https://egwwritings.org/read?panels=p417.299]


The authority on which the \emcap{fundamental principles} are established is unquestionable. They were the result of deep, earnest study in the time of great disappointment, when \egwinline{\textbf{\underline{point by point}, has been sought out by prayerful study, and testified to by the \underline{miracle-working power of the Lord}}}\footnote{Ibid.}. \egwinline{\textbf{Thus \underline{the leading points of our faith} as we hold them today were firmly established}. \textbf{\underline{Point after point} was clearly defined, and all the brethren came into harmony}.}[Lt253-1903.4; 1903][https://egwwritings.org/read?panels=p14068.9980010]


L'autorité sur laquelle les \emcap{principes fondamentaux} sont établis est incontestable. Ils étaient le résultat d'une étude profonde et sérieuse dans le temps de grande déception, quand \egwinline{\textbf{\underline{point par point}, a été recherché par une étude dans la prière, et attesté par le \underline{pouvoir miraculeux du Seigneur}}}\footnote{Ibid.}. \egwinline{\textbf{Ainsi \underline{les points principaux de notre foi} tels que nous les tenons aujourd'hui ont été fermement établis}. \textbf{\underline{Point après point} a été clairement défini, et tous les frères sont parvenus à l'harmonie}.}[Lt253-1903.4; 1903][https://egwwritings.org/read?panels=p14068.9980010]


They were the result of the earnest Bible studies of our pioneers, after the passing of time in 1844. As the Seventh-day Adventist movement progressed, there came a need for instituting the organization, which was realized in 1863. In 1872, the Seventh-day Adventist Church issued the document called “\textit{A Declaration of the Fundamental Principles, Taught and Practiced by the Seventh-Day Adventists}. This was the first written document declaring the \emcap{fundamental principles} as public statements of the Seventh-day Adventist faith. This document was the public synopsis of Seventh-day Adventist faith and it declared \others{what is, and has been, with great unanimity, held by} the Seventh-day Adventist people. It was written \others{to meet inquiries} as to what was believed by Seventh-day Adventists, \others{to correct false statements circulated} and to \others{remove erroneous impressions}[FP1872 3.1; 1872][https://egwwritings.org/read?panels=p928.8].


Ils étaient le résultat des études bibliques sérieuses de nos pionniers, après le passage du temps en 1844. Au fur et à mesure que le mouvement Adventiste du Septième Jour progressait, il est apparu un besoin d'instituer l'organisation, qui a été réalisée en 1863. En 1872, l'Église Adventiste du Septième Jour a publié le document intitulé “\textit{Déclaration des Principes Fondamentaux Enseignés et Pratiqués par les Adventistes du Septième Jour}. C'était le premier document écrit déclarant les \emcap{principes fondamentaux} comme déclarations publiques de la foi Adventiste du Septième Jour. Ce document était le synopsis public de la foi Adventiste du Septième Jour et il déclarait \others{ce qui est, et a été, avec une grande unanimité, tenu par} le peuple Adventiste du Septième Jour. Il a été écrit \others{pour répondre aux demandes} sur ce qui était cru par les Adventistes du Septième Jour, \others{pour corriger les fausses déclarations diffusées} et pour \others{supprimer les impressions erronées}[FP1872 3.1; 1872][https://egwwritings.org/read?panels=p928.8].


Today it is still debated who authored the synopsis because originally, in 1872, it was left anonymous. In 1874, James White issued it in Signs of the Times\footnote{\href{https://adventistdigitallibrary.org/adl-364148/signs-times-june-4-1874}{Signs of the Times, June 4, 1874}} and Uriah Smith in the Review and Herald\footnote{\href{http://documents.adventistarchives.org/Periodicals/RH/RH18741124-V44-22.pdf}{The Advent Review and Herald of the Sabbath, November 24, 1874}}—both signing with their own signatures. In 1889, Uriah Smith revised it by adding three points; it was issued in the Adventist Yearbook with his signature on it. Uriah Smith died in 1903 and all successive printings of the \emcap{Fundamental Principles} were printed under his name. They were printed in the Yearbooks—each year from 1905 until 1914\footnote{For more detailed timeline of Fundamental Principles, see \hyperref[appendix:timeline]{Appendix: Fundamental Principles - Timeline}}. Sister White died in 1915 and, for the next 17 years, the \emcap{fundamental principles} were not printed. Their next appearance was in the 1931 Yearbook when they received significant changes.


Aujourd'hui, on débat encore sur qui a rédigé le synopsis car à l'origine, en 1872, il était resté anonyme. En 1874, James White l'a publié dans Signs of the Times\footnote{\href{https://adventistdigitallibrary.org/adl-364148/signs-times-june-4-1874}{Signs of the Times, 4 juin 1874}} et Uriah Smith dans la Review and Herald\footnote{\href{http://documents.adventistarchives.org/Periodicals/RH/RH18741124-V44-22.pdf}{The Advent Review and Herald of the Sabbath, 24 novembre 1874}}—tous deux signant de leurs propres signatures. En 1889, Uriah Smith l'a révisé en ajoutant trois points ; il a été publié dans l'Annuaire Adventiste avec sa signature. Uriah Smith est décédé en 1903 et toutes les impressions successives des \emcap{Principes Fondamentaux} ont été imprimées sous son nom. Ils ont été imprimés dans les Annuaires—chaque année de 1905 à 1914\footnote{Pour une chronologie plus détaillée des Principes Fondamentaux, voir \hyperref[appendix:timeline]{Annexe : Principes Fondamentaux - Chronologie}}. Sœur White est décédée en 1915 et, pendant les 17 années suivantes, les \emcap{principes fondamentaux} n'ont pas été imprimés. Leur prochaine apparition a été dans l'Annuaire de 1931 où ils ont reçu des changements significatifs.


In 1971, LeRoy Froom wrote about a statement from 1872: \others{Though appearing anonymously, it was actually composed by Smith}[Edwin Froom, LeRoy. Movement of Destiny. 1971., p. 160]. Unfortunately, he didn’t provide any data to support his claim. It is unfortunate to see how pro-trinitarian scholars consider the \emcap{Fundamental Principles} to be of very little importance. Their true value is starkly diminished by attributing these beliefs to those of a small group of people, mostly to James White’s or Uriah Smith’s personal belief, rather than belief which was \others{with great unanimity, held by}[Preface of the Fundamental Principles 1872] the Seventh-day Adventist people. In 1958, Ministry Magazine described the \emcap{Fundamental Principles} as follows:


En 1971, LeRoy Froom a écrit à propos d'une déclaration de 1872 : \others{Bien qu'apparaissant anonymement, il a été en réalité composé par Smith}[Edwin Froom, LeRoy. Movement of Destiny. 1971., p. 160]. Malheureusement, il n'a fourni aucune donnée pour soutenir son affirmation. Il est regrettable de voir comment les érudits pro-trinitaires considèrent les \emcap{Principes Fondamentaux} comme étant de très peu d'importance. Leur vraie valeur est fortement diminuée en attribuant ces croyances à celles d'un petit groupe de personnes, principalement à la croyance personnelle de James White ou d'Uriah Smith, plutôt qu'à la croyance qui était \others{avec une grande unanimité, tenue par}[Préface des Principes Fondamentaux 1872] le peuple Adventiste du Septième Jour. En 1958, le Magazine Ministry a décrit les \emcap{Principes Fondamentaux} comme suit :


\others{It is true that in 1872 a ‘Declaration of the Fundamental Principles Taught and Practiced by Seventhday Adventists’ was printed, \textbf{but it was never adopted by the denomination and therefore cannot be considered official}. Evidently a small group, \textbf{perhaps even one or two, endeavored to put into words what they thought were the views of the entire church…}}[Ministry Magazine “\textit{Our Declaration of Fundamental Beliefs}”, January 1958, Roy Anderson, J. Arthur Buckwalter, Louise Kleuser, Earl Cleveland and Walter Schubert]


\others{Il est vrai qu'en 1872 une ‘Déclaration des Principes Fondamentaux Enseignés et Pratiqués par les Adventistes du Septième Jour’ a été imprimée, \textbf{mais elle n'a jamais été adoptée par la dénomination et ne peut donc pas être considérée comme officielle}. Apparemment un petit groupe, \textbf{peut-être même une ou deux personnes, s'est efforcé de mettre en mots ce qu'ils pensaient être les vues de l'église entière…}}[Magazine Ministry “\textit{Notre Déclaration des Croyances Fondamentales}”, janvier 1958, Roy Anderson, J. Arthur Buckwalter, Louise Kleuser, Earl Cleveland et Walter Schubert]


Problematically, there is no evidence to support the claim that the \emcap{Fundamental Principles} were not the representation of faith of the whole body. We certainly know that Sister White endorsed them and, from her influence alone, we know that these beliefs were indeed accepted by the denomination—this is in addition to the fact that they were printed multiple times over the course of 42 years, during the life of Ellen White.


Problématiquement, il n'y a aucune preuve pour soutenir l'affirmation que les \emcap{Principes Fondamentaux} n'étaient pas la représentation de la foi de l'ensemble du corps. Nous savons certainement que Sœur White les a approuvés et, de son influence seule, nous savons que ces croyances étaient en effet acceptées par la dénomination—ceci en plus du fait qu'ils ont été imprimés plusieurs fois au cours de 42 ans, durant la vie d'Ellen White.


But there should be no controversy over the authorship of the \emcap{Fundamental Principles}. We have a quotation from Sister White about who authored them. When speaking of Uriah Smith, Sister White wrote:


Mais il ne devrait y avoir aucune controverse sur la paternité des \emcap{Principes Fondamentaux}. Nous avons une citation de Sœur White sur qui les a rédigés. En parlant d'Uriah Smith, Sœur White a écrit :


\egw{\textbf{Brother Smith was with us in the rise of this work. He understands how \underline{we—my husband and myself}—have carried the work forward and upward step by step and have borne the hardships, the poverty, and the want of means. With us were those early workers. Elder Smith, especially, was one with my husband in his early manhood}. …}[Ms54-1890.6; 1890][https://egwwritings.org/read?panels=p7213.15]


\egw{\textbf{Frère Smith était avec nous à l'origine de cette œuvre. Il comprend comment \underline{nous—mon mari et moi-même}—avons porté l'œuvre en avant et vers le haut pas à pas et avons supporté les difficultés, la pauvreté, et le manque de moyens. Avec nous étaient ces premiers ouvriers. Ancien Smith, en particulier, était un avec mon mari dans sa jeunesse}. …}[Ms54-1890.6; 1890][https://egwwritings.org/read?panels=p7213.15]


\egwnogap{\textbf{\underline{We have stood shoulder to shoulder with Elder Smith in this work while the Lord was laying the foundation principles}}. \textbf{We had to work constantly against one-idea men}, who thought correct business relations in regard to the work which had to be done were an evidence of worldly-mindedness, and the cranky ones who would present themselves as capable of bearing responsibilities, but could not be trusted to be connected with the work lest they swing it in wrong lines. \textbf{Step after step has had to be taken, \underline{not after the wisdom of men} but after the wisdom and instruction of One who is too wise to err and too good to do us harm}. \textbf{There have been so many elements that would have to be proved and tried. I thank the Lord that Elders Smith, Amadon, and Batchellor still live. They composed the members of our family in the most trying parts of our history}.}[Ms54-1890.7; 1890][https://egwwritings.org/read?panels=p7213.16]


\egwnogap{\textbf{\underline{Nous nous sommes tenus côte à côte avec Frère Smith dans cette œuvre pendant que le Seigneur posait les principes fondamentaux}}. \textbf{Nous avons dû travailler constamment contre des hommes à idée unique}, qui pensaient que des relations d'affaires correctes concernant le travail qui devait être fait étaient une preuve de mondanité, et les excentriques qui se présentaient comme capables d'assumer des responsabilités, mais à qui on ne pouvait pas faire confiance pour être liés à l'œuvre de peur qu'ils ne l'orientent dans de mauvaises directions. \textbf{Étape après étape a dû être franchie, \underline{non pas selon la sagesse des hommes} mais selon la sagesse et l'instruction de Celui qui est trop sage pour se tromper et trop bon pour nous nuire}. \textbf{Il y a eu tant d'éléments qui auraient dû être prouvés et testés. Je remercie le Seigneur que les Frères Smith, Amadon et Batchellor vivent encore. Ils composaient les membres de notre famille dans les parties les plus difficiles de notre histoire}.}[Ms54-1890.7; 1890][https://egwwritings.org/read?panels=p7213.16]


According to this quotation, who laid down the foundation principles?


Selon cette citation, qui a établi les principes fondamentaux ?


\egwinline{\textbf{\underline{We have stood shoulder to shoulder with Elder Smith in this work while the Lord was laying the foundation principles}}.} \textbf{It was the Lord}! But who wrote them down as a declaration of our faith? It was Elder Smith with James White and Sister White; we see that where Sister White says\egwinline{\textbf{we} have stood shoulder to shoulder with Elder Smith}. This \textit{‘we’} is explained in the previous paragraph: \egwinline{He \normaltext{[Elder Smith]} understands how\textbf{ we—my husband and myself}—have carried the work forward}. With this quotation, Sister White was clearly involved when the Lord was laying the \emcap{Fundamental Principles}.


\egwinline{\textbf{\underline{Nous nous sommes tenus côte à côte avec Frère Smith dans cette œuvre pendant que le Seigneur posait les principes fondamentaux}}.} \textbf{C'était le Seigneur} ! Mais qui les a mis par écrit comme une déclaration de notre foi ? C'était Frère Smith avec James White et Sœur White ; nous le voyons où Sœur White dit \egwinline{\textbf{nous} nous sommes tenus côte à côte avec Frère Smith}. Ce \textit{‘nous’} est expliqué dans le paragraphe précédent : \egwinline{Il \normaltext{[Frère Smith]} comprend comment \textbf{nous—mon mari et moi-même}—avons fait avancer l'œuvre}. Avec cette citation, Sœur White était clairement impliquée lorsque le Seigneur posait les \emcap{Principes Fondamentaux}.


It is true that the Declaration of the \emcap{Fundamental Principles} was written by a small group of people, namely Elder Smith, James White and Ellen White, but they endeavored to put into words what was the true view of the entire church body. They accurately represented the \emcap{fundamental principles}—the truths received in the beginning of our work. If that were not the case, then this declaration is the very opposite of what it claims to be. They were written \others{to meet inquiries} as to what was believed by Seventh-day Adventists, \others{to correct false statements circulated} and to \others{remove erroneous impressions.}[FP1872 3.1; 1872][https://egwwritings.org/read?panels=p928.8] If this document misrepresented the Adventist position, why was its continual reprinting, over the course of 42 years, permitted? It was reprinted until the death of Ellen White. If this document misrepresented the church’s position, wouldn’t Ellen White have raised her voice against it? She always raised her voice against the misrepresentation of the Seventh-day Adventist position, as she did with D. M. Canright and Dr. Kellogg. If the \emcap{Fundamental Principles} were misrepresenting the Seventh-day Adventist position, then all subsequent reprinting should be attributed to a conspiracy theory. That would be the greatest conspiracy theory within the Seventh-day Adventist Church. Ever. The harmony between the writings of Ellen White, Adventist pioneers, and the claims made in the Declaration of the \emcap{Fundamental Principles}, testify of the fact that this declaration is an accurate \others{summary of the principal features of} Seventh-day Adventist \others{faith, upon which there is, so far as we know, entire unanimity throughout the body}[The preface of the Fundamental Principles 1889].


Il est vrai que la Déclaration des \emcap{Principes Fondamentaux} a été écrite par un petit groupe de personnes, à savoir Frère Smith, James White et Ellen White, mais ils se sont efforcés de mettre en mots ce qui était la véritable vision de l'ensemble du corps de l'église. Ils ont représenté avec précision les \emcap{principes fondamentaux}—les vérités reçues au début de notre œuvre. Si ce n'était pas le cas, alors cette déclaration est tout le contraire de ce qu'elle prétend être. Ils ont été écrits \others{pour répondre aux questions} sur ce que croyaient les Adventistes du Septième Jour, \others{pour corriger les fausses déclarations diffusées} et pour \others{éliminer les impressions erronées.}[FP1872 3.1; 1872][https://egwwritings.org/read?panels=p928.8] Si ce document déformait la position adventiste, pourquoi sa réimpression continue, sur une période de 42 ans, a-t-elle été autorisée ? Il a été réimprimé jusqu'à la mort d'Ellen White. Si ce document déformait la position de l'église, Ellen White n'aurait-elle pas élevé la voix contre lui ? Elle a toujours élevé la voix contre la déformation de la position adventiste du septième jour, comme elle l'a fait avec D. M. Canright et le Dr Kellogg. Si les \emcap{Principes Fondamentaux} déformaient la position adventiste du septième jour, alors toute réimpression ultérieure devrait être attribuée à une théorie du complot. Ce serait la plus grande théorie du complot au sein de l'Église Adventiste du Septième Jour. Jamais. L'harmonie entre les écrits d'Ellen White, les pionniers adventistes et les affirmations faites dans la Déclaration des \emcap{Principes Fondamentaux}, témoigne du fait que cette déclaration est un \others{résumé précis des caractéristiques principales de} la foi adventiste du septième jour, \others{sur lesquelles il y a, pour autant que nous sachions, une unanimité totale dans tout le corps}[La préface des Principes Fondamentaux 1889].


With the death of Sister White in 1915, printing of the \emcap{Fundamental Principles} ceased. From 1915 onward, the Yearbook did not print any statement of belief until 1931. At this time, the \emcap{Fundamental Principles} received substantial changes. For the first time, the Trinity was introduced to the \emcap{fundamental principles}. In points’ 2 and 3 we read:


Avec la mort de Sœur White en 1915, l'impression des \emcap{Principes Fondamentaux} a cessé. À partir de 1915, l'Annuaire n'a imprimé aucune déclaration de croyance jusqu'en 1931. À cette époque, les \emcap{Principes Fondamentaux} ont reçu des changements substantiels. Pour la première fois, la Trinité a été introduite dans les \emcap{principes fondamentaux}. Aux points 2 et 3, nous lisons :


\others{2. \textbf{That the Godhead, or Trinity, consists of the Eternal Father, a \underline{personal, spiritual Being}}, omnipotent, \textbf{\underline{omnipresent}}, omniscient, infinite in wisdom and love; \textbf{the Lord Jesus Christ, the Son of the Eternal Father}, \textbf{through whom all things were created} and through whom the salvation of the redeemed hosts will be accomplished; \textbf{the Holy Spirit, the third person of the Godhead}, the great regenerating power in the work of redemption. Matt. 28:19}


\others{2. \textbf{Que la Divinité, ou Trinité, se compose du Père Éternel, un \underline{Être personnel et spirituel}}, omnipotent, \textbf{\underline{omniprésent}}, omniscient, infini en sagesse et en amour ; \textbf{le Seigneur Jésus-Christ, le Fils du Père Éternel}, \textbf{par qui toutes choses ont été créées} et par qui le salut des rachetés sera accompli ; \textbf{le Saint-Esprit, la troisième personne de la Divinité}, la grande puissance régénératrice dans l'œuvre de la rédemption. Matt. 28:19}


\others{3. \textbf{That Jesus Christ is very God, being of the same nature and essence as the Eternal Father}…}[Yearbook of the Seventh-day Adventist Denomination, 1931, page. 377][https://static1.squarespace.com/static/554c4998e4b04e89ea0c4073/t/59d17eec12abd9c6194cd26d/1506901758727/SDA-YB1931-22+\%28P.+377-380\%29.pdf]


\others{3. \textbf{Que Jésus-Christ est vraiment Dieu, étant de la même nature et essence que le Père Éternel}…}[Annuaire de la Dénomination Adventiste du Septième Jour, 1931, page. 377][https://static1.squarespace.com/static/554c4998e4b04e89ea0c4073/t/59d17eec12abd9c6194cd26d/1506901758727/SDA-YB1931-22+\%28P.+377-380\%29.pdf]


This change, in favor of the Trinity, appeared sixteen years after the death of Sister White. A comparison of this statement with the original \emcap{Fundamental Principles} presents several striking differences. The Father is still a personal, spiritual Being, the creator of all things, but is not addressed as “\textit{one God}” any longer. Jesus Christ is still the Son of the Eternal Father, through whom the Father created all things; Jesus is, also, of the very same nature and essence of the Father. Although these were the same terms to describe the doctrine on the \emcap{personality of God} in the original \emcap{Fundamental Principles}, we ask about the meaning of the term “\textit{personal, spiritual being}” applied to the Father, if He is, by new statement, omnipresent by Himself? The Holy Spirit is not an instrument, or means of the Father’s omnipresence anymore. Although this statement uses similar rhetoric of the original \emcap{Fundamental Principles}, it steps away from the original doctrine on the presence and the \emcap{personality of God}.


Ce changement, en faveur de la Trinité, est apparu seize ans après la mort de Sœur White. Une comparaison de cette déclaration avec les \emcap{Principes Fondamentaux} originaux présente plusieurs différences frappantes. Le Père est toujours un Être personnel et spirituel, le créateur de toutes choses, mais n'est plus désigné comme “\textit{un seul Dieu}”. Jésus-Christ est toujours le Fils du Père Éternel, par qui le Père a créé toutes choses ; Jésus est, également, de la même nature et essence que le Père. Bien que ce soient les mêmes termes pour décrire la doctrine sur la \emcap{personnalité de Dieu} dans les \emcap{Principes Fondamentaux} originaux, nous nous interrogeons sur la signification du terme “\textit{être personnel et spirituel}” appliqué au Père, s'Il est, selon la nouvelle déclaration, omniprésent par Lui-même ? Le Saint-Esprit n'est plus un instrument ou un moyen de l'omniprésence du Père. Bien que cette déclaration utilise une rhétorique similaire à celle des \emcap{Principes Fondamentaux} originaux, elle s'éloigne de la doctrine originale sur la présence et la \emcap{personnalité de Dieu}.


According to LeRoy Froom, this statement was written entirely by Francis Wilcox, with the approval of three other brothers (C.H. Watson, M.E. Kern and E.R. Palmer).\footnote{Edwin Froom, LeRoy. Movement of Destiny. 1971., p. 411, 413, 414} In the unpublished paper of \textit{The Seventh-day Adventist Church in Mission: 1919-1979}, we read how Elder Wilcox made this statement contrary to the belief of the church body and published it without their approval.


Selon LeRoy Froom, cette déclaration a été écrite entièrement par Francis Wilcox, avec l'approbation de trois autres frères (C.H. Watson, M.E. Kern et E.R. Palmer).\footnote{Edwin Froom, LeRoy. Movement of Destiny. 1971., p. 411, 413, 414} Dans le document non publié de \textit{L'Église Adventiste du Septième Jour en Mission : 1919-1979}, nous lisons comment Frère Wilcox a fait cette déclaration contraire à la croyance du corps de l'église et l'a publiée sans leur approbation.


\others{\textbf{Realizing that the General Conference Committee or any other church body would never accept the document in the form in which it was written}, Elder Wilcox, with full knowledge of the group \normaltext{[C.H. Watson, M.E. Kern and E.R. Palmer]}, handed the Statement directly to Edson Rogers, the General Conference statistician, who published it in the 1931 edition of the Yearbook, where it has appeared ever since. It was without the official approval of the General Conference Committee, therefore, and without any formal denominational adoption, that Elder Wilcox's statement became the accepted declaration of our faith.}[Dwyer, Bonnie. “A New Statement of Fundamental Beliefs (1980) - Spectrum Magazine.” \textit{Spectrum Magazine}, 7 June 2009, \href{https://spectrummagazine.org/news/new-statement-fundamental-beliefs-1980/}{spectrummagazine.org/news/new-statement-fundamental-beliefs-1980/}. Accessed 30 Jan. 2025.]


\others{\textbf{Réalisant que le Comité de la Conférence Générale ou tout autre corps ecclésiastique n'accepterait jamais le document sous la forme dans laquelle il était écrit}, Frère Wilcox, avec la pleine connaissance du groupe \normaltext{[C.H. Watson, M.E. Kern et E.R. Palmer]}, a remis la Déclaration directement à Edson Rogers, le statisticien de la Conférence Générale, qui l'a publiée dans l'édition 1931 de l'Annuaire, où elle est apparue depuis. C'est donc sans l'approbation officielle du Comité de la Conférence Générale et sans aucune adoption formelle confessionnelle que la déclaration de Frère Wilcox est devenue la déclaration acceptée de notre foi.}[Dwyer, Bonnie. “A New Statement of Fundamental Beliefs (1980) - Spectrum Magazine.” \textit{Spectrum Magazine}, 7 juin 2009, \href{https://spectrummagazine.org/news/new-statement-fundamental-beliefs-1980/}{spectrummagazine.org/news/new-statement-fundamental-beliefs-1980/}. Consulté le 30 janv. 2025.]


In 1980, the final change to the public synopsis of the Seventh-day Adventist faith was made. The General Conference voted to adopt today’s official statement:


En 1980, le changement final au synopsis public de la foi Adventiste du Septième Jour a été effectué. La Conférence Générale a voté pour adopter la déclaration officielle d'aujourd'hui:


\others{\textbf{There is one God: Father, Son and Holy Spirit, a unity of three coeternal Persons}. God is immortal, all-powerful, all-knowing, above all, and \textbf{ever present}. He is infinite and beyond human comprehension, yet known through His self-revelation. He is forever worthy of worship, adoration, and service by the whole creation.}[Seventh-day Adventists Believe: A Biblical Exposition of 27 Fundamental Doctrines, p. 16]


\others{\textbf{Il y a un seul Dieu: Père, Fils et Saint-Esprit, une unité de trois Personnes coéternelles}. Dieu est immortel, tout-puissant, omniscient, au-dessus de tout, et \textbf{toujours présent}. Il est infini et au-delà de la compréhension humaine, mais se fait connaître par Sa révélation de Lui-même. Il est à jamais digne d'adoration, de vénération et de service par toute la création.}[Seventh-day Adventists Believe: A Biblical Exposition of 27 Fundamental Doctrines, p. 16]


In this brief historical overview we see that the 1931 statement is a “middle step” between the original Adventist belief to the full trinitarian belief.


Dans ce bref aperçu historique, nous voyons que la déclaration de 1931 est une “étape intermédiaire” entre la croyance adventiste originale et la croyance trinitaire complète.


The change in our beliefs has occurred over time with many discussions. Our Adventist history has left a trace of these changes. If we are honest truth seekers we should study this matter in detail. Can we see, in our Adventist history, why we have left the first point of the \emcap{Fundamental Principles} in favor of the Trinity doctrine? Most certainly! In the following studies we will look at some of the historical documents that show why we have moved from the first point of the \emcap{Fundamental Principles}, held in the early years, to accept the Trinity doctrine. During these studies, we bid you to prayerfully evaluate the changes with your own beliefs.


Le changement dans nos croyances s'est produit au fil du temps avec de nombreuses discussions. Notre histoire adventiste a laissé une trace de ces changements. Si nous sommes d'honnêtes chercheurs de vérité, nous devrions étudier cette question en détail. Pouvons-nous voir, dans notre histoire adventiste, pourquoi nous avons abandonné le premier point des \emcap{Principes Fondamentaux} en faveur de la doctrine de la Trinité? Très certainement! Dans les études suivantes, nous examinerons certains des documents historiques qui montrent pourquoi nous sommes passés du premier point des \emcap{Principes Fondamentaux}, tenus dans les premières années, à l'acceptation de la doctrine de la Trinité. Pendant ces études, nous vous invitons à évaluer avec prière les changements par rapport à vos propres croyances.


% The authority of the Fundamental Principles

\begin{titledpoem}

    \stanza{
        Our principles stand firm and true, \\
        Established by God’s chosen few. \\
        A platform built on sturdy might, \\
        As guiding waymarks in the night.
    }

    \stanza{
        The truth was sought with earnest prayer, \\
        Point after point laid down with care. \\
        Yet modern minds the truth exchanged, \\
        For pleasing myths the doctrines changed.
    }

    \stanza{
        Return, O church, to truths ordained, \\
        Not to beliefs that men have claimed. \\
        Stand firm! God’s truth cannot be moved, \\
        Those Fundamental’s God approved.
    }

    \stanza{
        Let not new scholars lead astray, \\
        From paths our founders led the way. \\
        The Lord laid down these truths of old, \\
        Embrace these truths with courage bold.
    }
    
\end{titledpoem}