\qrchapter{https://forgottenpillar.com/rsc/en-fp-chapter15}{Dr. Kellogg and the Trinity doctrine}


\qrchapter{https://forgottenpillar.com/rsc/en-fp-chapter15}{Dr. Kellogg et la doctrine de la Trinité}


The key problem with the Kellogg controversy was the sentiments over the \emcap{personality of God}, which were departing from the foundation of our faith, that God established at the beginning of our work. We have been told that \egwinline{Many things of like character will in the future arise}[Ms137-1903.10; 1903][https://egwwritings.org/read?panels=p9939.17]. In the book, the Living Temple, we see the sentiments regarding the \emcap{personality of God} and where His presence is, which were stepping off of the \emcap{Fundamental Principles}. This step was never supposed to be made! But we raise the question, where was this step heading? We will see the evidence that this step was heading toward the Trinity doctrine. Sister White prophesied that Kellogg’s step would lead toward the Omega heresy. Can we see the connection between Kellogg’s controversy and the Trinity doctrine?


Le problème clé de la controverse Kellogg était le raisonnement concernant la \emcap{personnalité de Dieu}, qui s'écartait du fondement de notre foi, que Dieu a établi au commencement de notre œuvre. On nous a dit que \egwinline{Beaucoup de choses de caractère semblable surgiront dans le futur}[Ms137-1903.10; 1903][https://egwwritings.org/read?panels=p9939.17]. Dans le livre, Le Temple Vivant, nous voyons le raisonnement concernant la \emcap{personnalité de Dieu} et où se trouve Sa présence, qui s'écartait des \emcap{Principes Fondamentaux}. Ce pas n'aurait jamais dû être fait ! Mais nous soulevons la question, où ce pas menait-il ? Nous verrons les preuves que ce pas se dirigeait vers la doctrine de la Trinité. Sœur White a prophétisé que le pas de Kellogg mènerait vers l'hérésie Oméga. Pouvons-nous voir le lien entre la controverse de Kellogg et la doctrine de la Trinité ?


In the following section, we want to present you with the connection between Kellogg’s controversy and the doctrine of Trinity. It is important to emphasize that the Living Temple does not contain this doctrine as it is believed today. The main problem with Kellogg’s teaching was the \textit{stepping off} of the \emcap{Fundamental Principles}, which were the foundation of our faith. The information we will present to you reveals that Dr. Kellogg justified his actions in stepping off of the foundation through his belief in the doctrine of Trinity. This is not difficult to see when we recognize that the \emcap{Fundamental Principles} were a non-Trinitarian. Our main focus should not be in recognizing the Trinity doctrine in Kellogg's arguments, but rather in understanding the differences between Kellogg’s teachings and the teachings of the \emcap{Fundamental Principles} regarding \egwinline{the personality of God and where His presence is}[SpTB02 51.3; 1903][https://egwwritings.org/read?panels=p417.262]. In other words, what were the steps Kellogg made in stepping off of the foundation of our faith? This approach is advocated by the Spirit of Prophecy and it will help us to avoid speculations regarding Kellogg’s motives—it will help us to focus upon the truth. Ellen White tells us that there are many good things written in the Living Temple, but they are mingled with specious, deceptive theories regarding the \emcap{personality of God} and \emcap{of Christ}.


Dans la section suivante, nous voulons vous présenter le lien entre la controverse de Kellogg et la doctrine de la Trinité. Il est important de souligner que Le Temple Vivant ne contient pas cette doctrine telle qu'elle est crue aujourd'hui. Le problème principal avec l'enseignement de Kellogg était le \textit{fait de s'écarter} des \emcap{Principes Fondamentaux}, qui étaient le fondement de notre foi. Les informations que nous vous présenterons révèlent que le Dr. Kellogg justifiait ses actions de s'écarter du fondement par sa croyance en la doctrine de la Trinité. Ce n'est pas difficile à voir quand nous reconnaissons que les \emcap{Principes Fondamentaux} étaient non-trinitaires. Notre objectif principal ne devrait pas être de reconnaître la doctrine de la Trinité dans les arguments de Kellogg, mais plutôt de comprendre les différences entre les enseignements de Kellogg et les enseignements des \emcap{Principes Fondamentaux} concernant \egwinline{la personnalité de Dieu et où se trouve Sa présence}[SpTB02 51.3; 1903][https://egwwritings.org/read?panels=p417.262]. En d'autres termes, quels étaient les pas que Kellogg a faits en s'écartant du fondement de notre foi ? Cette approche est préconisée par l'Esprit de Prophétie et elle nous aidera à éviter les spéculations concernant les motifs de Kellogg—elle nous aidera à nous concentrer sur la vérité. Ellen White nous dit qu'il y a beaucoup de bonnes choses écrites dans Le Temple Vivant, mais elles sont mélangées avec des théories spécieuses et trompeuses concernant la \emcap{personnalité de Dieu} et \emcap{du Christ}.


\begin{figure}[hp]
    \centering
    \includegraphics[width=1\linewidth]{images/john-h-kellogg.jpg}
    \caption*{John Harvey Kellogg (1852-1943)}
    \label{fig:john-h-kellogg}
\end{figure}


\begin{figure}[hp]
    \centering
    \includegraphics[width=1\linewidth]{images/john-h-kellogg.jpg}
    \caption*{John Harvey Kellogg (1852-1943)}
    \label{fig:john-h-kellogg}
\end{figure}


\egw{\textbf{The book Living Temple contains specious, \underline{deceptive sentiments regarding the personality of God and of Christ}}. The Lord opened before me the true meaning of these sentiments, showing me that unless they were steadfastly repudiated, they would deceive the very elect. \textbf{Precious truth and beautiful sentiments were woven in with false, misleading theories. Thus truth was used to substantiate the \underline{most dangerous errors}. The precious representations of God are so misconstrued as to appear to uphold falsehoods \underline{originated by the great apostate}. Sentiments that belong to the revealings of God are mingled with specious, deceptive theories of satanic agencies}.}[Lt146-1905.2; 1905][https://egwwritings.org/read?panels=p9430.8]


\egw{\textbf{Le livre Le Temple Vivant contient un raisonnement spécieux et \underline{trompeur concernant la personnalité de Dieu et du Christ}}. Le Seigneur m'a ouvert la vraie signification de ce raisonnement, me montrant qu'à moins qu'il ne soit fermement répudié, il tromperait les élus eux-mêmes. \textbf{Des vérités précieuses et de beaux sentiments étaient tissés avec de fausses théories trompeuses. Ainsi la vérité était utilisée pour justifier les \underline{erreurs les plus dangereuses}. Les représentations précieuses de Dieu sont si mal interprétées qu'elles semblent soutenir les mensonges \underline{originés par le grand apostat}. Les sentiments qui appartiennent aux révélations de Dieu sont mélangés avec des théories spécieuses et trompeuses des agences sataniques}.}[Lt146-1905.2; 1905][https://egwwritings.org/read?panels=p9430.8]


\egwnogap{In the controversy over these theories \textbf{it has been asserted that I believed and taught the same things} that I have been instructed to condemn in the book Living Temple. \textbf{This I deny}. In the name of Jesus Christ of Nazareth, \textbf{I say that this is not so}.}[Lt146-1905.3; 1905][https://egwwritings.org/read?panels=p9430.9]


\egwnogap{Dans la controverse sur ces théories, \textbf{il a été affirmé que je croyais et enseignais les mêmes choses} que j'ai été instruite de condamner dans le livre Le Temple Vivant. \textbf{Ceci, je le nie}. Au nom de Jésus-Christ de Nazareth, \textbf{je dis que ce n'est pas le cas}.}[Lt146-1905.3; 1905][https://egwwritings.org/read?panels=p9430.9]


This mixture of truth and error makes the matter difficult. In the eyes of pro-trinitarian scholars, the problem is solely attributed to pantheism, and the evidence of Kellogg's belief in the Trinity doctrine is interpreted as belief in a false Trinity\footnote{Whidden, Woodrow W, et al. \textit{The Trinity : Understanding God's Love, His Plan of Salvation, and Christian Relationships}. Hagerstown, Md, Review And Herald Pub. Association, 2002., p. 217}. Sister White's rebuke is attributed to the defense of the “correct” Trinity, which she supposedly believed. Unfortunately, such interpretation does not acknowledge Sister White's defense of the \emcap{Fundamental Principles} regarding the \emcap{personality of God} and of Christ, thus it is a misinterpretation of her work. In the following sections, we will examine historical data on Dr. Kellogg's connection with the doctrine of Trinity from the perspective of the Adventist truth on the \emcap{personality of God}, which constituted the foundation of our faith. With this perspective, we believe that the historical data will shine in a new light and spark honest and constructive dialogue in our church.


Ce mélange de vérité et d'erreur rend la question difficile. Aux yeux des érudits pro-trinitaires, le problème est uniquement attribué au panthéisme, et la preuve de la croyance de Kellogg en la doctrine de la Trinité est interprétée comme une croyance en une fausse Trinité\footnote{Whidden, Woodrow W, et al. \textit{The Trinity : Understanding God's Love, His Plan of Salvation, and Christian Relationships}. Hagerstown, Md, Review And Herald Pub. Association, 2002., p. 217}. La réprimande de Sœur White est attribuée à la défense de la “correcte” Trinité, qu'elle aurait supposément crue. Malheureusement, une telle interprétation ne reconnaît pas la défense de Sœur White des \emcap{Principes Fondamentaux} concernant la \emcap{personnalité de Dieu} et du Christ, c'est donc une mauvaise interprétation de son œuvre. Dans les sections suivantes, nous examinerons les données historiques sur le lien du Dr. Kellogg avec la doctrine de la Trinité du point de vue de la vérité adventiste sur la \emcap{personnalité de Dieu}, qui constituait le fondement de notre foi. Avec cette perspective, nous croyons que les données historiques brilleront d'une nouvelle lumière et susciteront un dialogue honnête et constructif dans notre église.


\section*{Correspondence of Dr. Kellogg and Brother Butler}


\section*{Correspondance du Dr. Kellogg et de Frère Butler}


In the following section we briefly present you with the well-known correspondence between Dr. Kellogg and G. I. Butler over the book, the Living Temple. Here, we see Dr. Kellogg’s objections regarding the controversy. He wrote to Brother Butler:


Dans la section suivante, nous vous présentons brièvement la correspondance bien connue entre le Dr. Kellogg et G. I. Butler concernant le livre, Le Temple Vivant. Ici, nous voyons les objections du Dr. Kellogg concernant la controverse. Il a écrit à Frère Butler :


\others{As far as I can fathom, the \textbf{difficulty }which is found \textbf{in ‘The Living Temple’,} \textbf{the whole thing may be simmered down to the question}: \textbf{\underline{Is the Holy Ghost a person}?} You say no. I had supposed the Bible said this for the reason that the personal pronoun ‘he’ is used in speaking of the Holy Ghost. \textbf{Sister White uses the pronoun ‘he’ and has said in so many words that the Holy Ghost is \underline{the third person of the Godhead}}. \textbf{How the Holy Ghost can be the third person and not be a person at all is difficult for me to see}.}[Letter: J. H. Kellogg to G. I. Butler. Oct 28. 1903][https://static1.squarespace.com/static/554c4998e4b04e89ea0c4073/t/5db9fbc96defed1e45b497a4/1572469707862/1903-10-28-Kellog-to-Butler.pdf]


\others{Pour autant que je puisse comprendre, la \textbf{difficulté} qui se trouve \textbf{dans ‘Le Temple Vivant’,} \textbf{toute la chose peut être réduite à la question} : \textbf{\underline{Le Saint-Esprit est-il une personne}} ? Vous dites non. J'avais supposé que la Bible disait cela pour la raison que le pronom personnel ‘il’ est utilisé en parlant du Saint-Esprit. \textbf{Sœur White utilise le pronom ‘il’ et a dit en autant de mots que le Saint-Esprit est \underline{la troisième personne de la Divinité}}. \textbf{Comment le Saint-Esprit peut être la troisième personne et ne pas être une personne du tout est difficile pour moi à voir}.}[Lettre : J. H. Kellogg à G. I. Butler. 28 oct. 1903][https://static1.squarespace.com/static/554c4998e4b04e89ea0c4073/t/5db9fbc96defed1e45b497a4/1572469707862/1903-10-28-Kellog-to-Butler.pdf]


\begin{figure}[hp]
    \centering
    \includegraphics[width=1\linewidth]{images/george-ide-butler.jpg}
    \caption*{George Ide Butler (1834-1918)}
    \label{fig:g-i-butler}
\end{figure}


\begin{figure}[hp]
    \centering
    \includegraphics[width=1\linewidth]{images/george-ide-butler.jpg}
    \caption*{George Ide Butler (1834-1918)}
    \label{fig:g-i-butler}
\end{figure}


According to Dr. Kellogg’s perspective, the whole problem with the book ‘The Living Temple’ comes down to the question “\textit{Is the Holy Ghost a person?}”. Obviously, he does not advocate an impersonal God, as he is often accused of\footnote{Whidden, Woodrow W, et al. \textit{The Trinity : Understanding God's Love, His Plan of Salvation, and Christian Relationships}. Hagerstown, Md, Review And Herald Pub. Association, 2002., p. 217}. Moreover, he even believes that the Holy Ghost is a \textit{third person of the Godhead}. Also, he claims that Brother Butler does not believe that the Holy Ghost is a person. The problem obviously lies in the definition of the word \textit{‘person’}. On this point, Kellogg continues:


Selon la perspective du Dr Kellogg, tout le problème avec le livre « Le Temple Vivant » se résume à la question « \textit{Le Saint-Esprit est-il une personne ?} ». Évidemment, il ne préconise pas un Dieu impersonnel, comme on l'en accuse souvent\footnote{Whidden, Woodrow W, et al. \textit{The Trinity : Understanding God's Love, His Plan of Salvation, and Christian Relationships}. Hagerstown, Md, Review And Herald Pub. Association, 2002., p. 217}. De plus, il croit même que le Saint-Esprit est une \textit{troisième personne de la Divinité}. De plus, il affirme que Frère Butler ne croit pas que le Saint-Esprit est une personne. Le problème réside évidemment dans la définition du mot \textit{« personne »}. Sur ce point, Kellogg continue :


\others{I believe this Spirit of God to be a personality you don’t. But this is purely a question of definition. \textbf{I believe the Spirit of God is a personality}; you say, No, it is not a personality. Now the only reason why we differ is because we \textbf{differ in our ideas as to \underline{what a personality is}}. \textbf{Your idea of personality is perhaps that of \underline{semblance to a person} or a human being}.}[Letter: J. H. Kellogg to G. I. Butler. Oct 28. 1903][https://static1.squarespace.com/static/554c4998e4b04e89ea0c4073/t/5db9fbc96defed1e45b497a4/1572469707862/1903-10-28-Kellog-to-Butler.pdf]


\others{Je crois que cet Esprit de Dieu est une personnalité, vous non. Mais c'est purement une question de définition. \textbf{Je crois que l'Esprit de Dieu est une personnalité} ; vous dites, Non, ce n'est pas une personnalité. Maintenant, la seule raison pour laquelle nous différons est parce que nous \textbf{différons dans nos idées quant à \underline{ce qu'est une personnalité}}. \textbf{Votre idée de la personnalité est peut-être celle de \underline{ressemblance à une personne} ou à un être humain}.}[Lettre : J. H. Kellogg à G. I. Butler. 28 oct. 1903][https://static1.squarespace.com/static/554c4998e4b04e89ea0c4073/t/5db9fbc96defed1e45b497a4/1572469707862/1903-10-28-Kellog-to-Butler.pdf]


Brother Butler replied:


Frère Butler a répondu :


\others{\textbf{So far as Sister White and you being in perfect agreement, I shall have to leave that entirely between you and Sister White. \underline{Sister White says there is not perfect agreement; you claim there is}. \underline{I know some of her remarks seem to give you strong ground for claiming that she does}. I am candid enough to say that, but I must give her the credit until she disowns it of saying there is a difference too, and I do not believe you can fully tell just what she means. \underline{God dwells in us by His Holy Spirit}, as a Comforter, as a Reprover, especially the former. When we come to Him we partake of Him in that sense, because the Spirit comes forth from Him; \underline{it comes forth from the Father and the Son}. It is not a person walking around on foot, or flying \underline{as a literal being}, \underline{in any such sense as Christ and the Father are} – at least, if it is, it is utterly beyond my comprehension of the meaning of language or words}.}[Letter: G. I. Butler to J. H. Kellogg. April 5. 1904]


\others{\textbf{En ce qui concerne le fait que Sœur White et vous soyez en parfait accord, je devrai laisser cela entièrement entre vous et Sœur White. \underline{Sœur White dit qu'il n'y a pas d'accord parfait ; vous prétendez qu'il y en a un}. \underline{Je sais que certaines de ses remarques semblent vous donner de solides raisons de prétendre qu'elle le fait}. Je suis assez honnête pour dire cela, mais je dois lui accorder le crédit jusqu'à ce qu'elle le désavoue de dire qu'il y a aussi une différence, et je ne crois pas que vous puissiez dire exactement ce qu'elle veut dire. \underline{Dieu habite en nous par Son Saint-Esprit}, comme un Consolateur, comme un Réprobateur, surtout le premier. Quand nous venons à Lui, nous participons de Lui dans ce sens, parce que l'Esprit vient de Lui ; \underline{il vient du Père et du Fils}. Ce n'est pas une personne qui marche à pied, ou qui vole \underline{comme un être littéral}, \underline{dans le même sens que Christ et le Père le sont} – du moins, si c'est le cas, c'est totalement au-delà de ma compréhension du sens du langage ou des mots}.}[Lettre : G. I. Butler à J. H. Kellogg. 5 avril 1904]


The given correspondence is crucial for understanding the Kellogg controversy. Kellogg himself stated, \others{the whole thing may be simmered down to the question: \textbf{Is the Holy Ghost a person?}} Similarly Dr. Kellogg wrote to William White: \others{I have been studying very carefully to see what is \textbf{the real root of the difficulty with the Living Temple}, and as far as I can see \textbf{\underline{the whole question} resolves itself into this: \underline{Is the Holy Ghost, a person}?}}[Letter J. H. Kellogg to William White, October 28, 1903][https://drive.google.com/file/d/1\_S4S-Hc0K7Ka8gda9oRhPuAb9XzBTwmb/view] How does Kellogg's conclusion compare to the review and instruction of heavenly origin, which clearly told us that the reasoning in the Living Temple is \egwinline{naught but speculation in regard to \textbf{the personality of God and where His presence is}}[SpTB02 51.3; 1904][https://egwwritings.org/read?panels=p417.262]? In the writings of Ellen White and the pioneers, the term ‘\textit{personality of God}’ refers specifically to the personality of the Father. So, why does Kellogg claim that the real issue is the personality of the Holy Spirit, when God indicated that the issue concerns the personality of the Father?


La correspondance donnée est cruciale pour comprendre la controverse Kellogg. Kellogg lui-même a déclaré, \others{toute la chose peut être réduite à la question : \textbf{Le Saint-Esprit est-il une personne ?}} De même, le Dr Kellogg a écrit à William White : \others{J'ai étudié très attentivement pour voir quelle est \textbf{la vraie racine de la difficulté avec Le Temple Vivant}, et autant que je puisse voir \textbf{\underline{toute la question} se résout à ceci : \underline{Le Saint-Esprit est-il une personne} ?}}[Lettre J. H. Kellogg à William White, 28 octobre 1903][https://drive.google.com/file/d/1\_S4S-Hc0K7Ka8gda9oRhPuAb9XzBTwmb/view] Comment la conclusion de Kellogg se compare-t-elle à la révision et à l'instruction d'origine céleste, qui nous a clairement dit que le raisonnement dans Le Temple Vivant n'est \egwinline{rien d'autre que de la spéculation concernant \textbf{la personnalité de Dieu et où Sa présence est}}[SpTB02 51.3; 1904][https://egwwritings.org/read?panels=p417.262] ? Dans les écrits d'Ellen White et des pionniers, le terme « \textit{personnalité de Dieu} » se réfère spécifiquement à la personnalité du Père. Alors, pourquoi Kellogg prétend-il que le vrai problème est la personnalité du Saint-Esprit, quand Dieu a indiqué que le problème concerne la personnalité du Père ?


Many assume that Dr. Kellogg is being manipulative, evading the core issue. However, under a particular premise, his arguments concerning the personality of the Holy Spirit logically support his controversial views on the \emcap{personality of God}. This premise becomes evident within the data itself when we closely follow his reasoning.


Beaucoup supposent que le Dr Kellogg est manipulateur, évitant le problème central. Cependant, sous une prémisse particulière, ses arguments concernant la personnalité du Saint-Esprit soutiennent logiquement ses vues controversées sur la \emcap{personnalité de Dieu}. Cette prémisse devient évidente dans les données elles-mêmes lorsque nous suivons de près son raisonnement.


As we have seen earlier, the doctrine on the \emcap{personality of God} teaches that God, the Father, possesses a form—a tangible, material body. Dr. Kellogg concurred that this assertion holds true within the bounds of our finite conception of God\footnote{\href{https://archive.org/details/J.H.Kellogg.TheLivingTemple1903/page/n33/}{Dr. John H. Kellogg, The Living Temple, p.31.}}. However, he argued that, in reality, God transcends our conceptions regarding His form, as He is beyond the constraints of space\footnote{\href{https://archive.org/details/J.H.Kellogg.TheLivingTemple1903/page/n33/}{Dr. John H. Kellogg, The Living Temple, p.33.}}. In this sense, Kellogg effectively does away with the reality of God’s physical, material body. The premise that would validate Dr. Kellogg’s viewpoint is the \textit{exclusive equivalence} in understanding the \emcap{personality of God} and that of the Holy Spirit. Is the Holy Spirit constrained by space? No, He is not. Does the Holy Spirit have a physical body? No! According to Jesus, \bible{for a spirit hath not flesh and bones}[Luke 24:39]. Is the Holy Ghost a person? The answer hinges on our interpretation of what it means to be a person. What is that quality or state of the Holy Spirit being a person?\footnote{Direct application of the definition on the word ‘\textit{personality}’ from the \href{https://www.merriam-webster.com/dictionary/personality}{Merriam Webster Dictionary}} When comparing Dr. Kellogg's belief in the personality of the Holy Spirit with Brother Butler's views, it becomes evident that the quality of the Holy Spirit being a person does not align with \others{that of \textbf{semblance to a person} or a human being}. Butler explicitly stated his criteria for this determination\footnote{In his letter to Dr. Kellogg, Brother Butler further asserted that there is no distinction between the person and the bodily presence. See \href{https://c7da.us/egwdl/Butler\%20to\%20Kellogg\%20Aug121904.pdf}{Letter from Butler to Kellogg, August 12, 1904, p.6}}: \others{\textbf{It is not a person walking around on foot, or flying \underline{as a literal being}, \underline{in any such sense as Christ and the Father are} – at least, if it is, it is utterly beyond my comprehension of the meaning of language or words}}.


Comme nous l'avons vu plus tôt, la doctrine sur la \emcap{personnalité de Dieu} enseigne que Dieu, le Père, possède une forme — un corps tangible et matériel. Le Dr Kellogg a convenu que cette affirmation est vraie dans les limites de notre conception finie de Dieu\footnote{\href{https://archive.org/details/J.H.Kellogg.TheLivingTemple1903/page/n33/}{Dr. John H. Kellogg, The Living Temple, p.31.}}. Cependant, il a soutenu qu'en réalité, Dieu transcende nos conceptions concernant Sa forme, car Il est au-delà des contraintes de l'espace\footnote{\href{https://archive.org/details/J.H.Kellogg.TheLivingTemple1903/page/n33/}{Dr. John H. Kellogg, The Living Temple, p.33.}}. Dans ce sens, Kellogg élimine effectivement la réalité du corps physique et matériel de Dieu. La prémisse qui validerait le point de vue du Dr Kellogg est l’\textit{équivalence exclusive} dans la compréhension de la \emcap{personnalité de Dieu} et celle du Saint-Esprit. Le Saint-Esprit est-il contraint par l'espace ? Non, Il ne l'est pas. Le Saint-Esprit a-t-il un corps physique ? Non ! Selon Jésus, \bible{car un esprit n'a ni chair ni os}[Luc 24:39]. Le Saint-Esprit est-il une personne ? La réponse dépend de notre interprétation de ce que signifie être une personne. Quelle est cette qualité ou cet état du Saint-Esprit d'être une personne ?\footnote{Application directe de la définition du mot « \textit{personnalité} » du \href{https://www.merriam-webster.com/dictionary/personality}{Dictionnaire Merriam Webster}} En comparant la croyance du Dr Kellogg en la personnalité du Saint-Esprit avec les vues de Frère Butler, il devient évident que la qualité du Saint-Esprit d'être une personne ne s'aligne pas avec \others{celle de \textbf{ressemblance à une personne} ou à un être humain}. Butler a explicitement déclaré ses critères pour cette détermination\footnote{Dans sa lettre au Dr Kellogg, Frère Butler a en outre affirmé qu'il n'y a pas de distinction entre la personne et la présence corporelle. Voir \href{https://c7da.us/egwdl/Butler\%20to\%20Kellogg\%20Aug121904.pdf}{Lettre de Butler à Kellogg, 12 août 1904, p.6}} : \others{\textbf{Ce n'est pas une personne qui marche à pied, ou qui vole \underline{comme un être littéral}, \underline{dans le même sens que Christ et le Père le sont} – du moins, si c'est le cas, c'est totalement au-delà de ma compréhension du sens du langage ou des mots}}.


Have you noticed that Brother Butler addressed Kellogg’s unspoken premise? Butler drew a distinction between the Father and Christ, in relation to the Holy Spirit. Brother Butler is correct. There exists a contrast between the personality of the Holy Spirit and that of God and Christ. Christ and the Father possess a physical form of a person, whereas the Holy Spirit does not. To do away with the physical form of a person of the Father is to \textit{exclusively equate} the understanding of the personality of the Father with that of the Holy Spirit. Kellogg’s approach is compelling, because it was backed by valid arguments regarding the personality of the Holy Spirit.


Avez-vous remarqué que Frère Butler a abordé la prémisse non exprimée de Kellogg ? Butler a établi une distinction entre le Père et Christ, par rapport au Saint-Esprit. Frère Butler a raison. Il existe un contraste entre la personnalité du Saint-Esprit et celle de Dieu et de Christ. Christ et le Père possèdent une forme physique d'une personne, alors que le Saint-Esprit n'en possède pas. Éliminer la forme physique d'une personne du Père, c'est \textit{assimiler exclusivement} la compréhension de la personnalité du Père à celle du Saint-Esprit. L'approche de Kellogg est convaincante, car elle était soutenue par des arguments valides concernant la personnalité du Saint-Esprit.


Let us briefly examine the personality of the Holy Spirit. What is the quality or state of the Holy Spirit being a person?


Examinons brièvement la personnalité du Saint-Esprit. Quelle est la qualité ou l'état du Saint-Esprit d'être une personne ?


\egw{\textbf{The Holy Spirit has a personality}, \textbf{\underline{else} }He could not \textbf{bear witness} to our spirits and with our spirits that we are the children of God. \textbf{He must also be a \underline{divine person}}, \textbf{\underline{else}} He could not \textbf{search out the secrets} which lie hidden \textbf{in the mind of God}.}[21LtMs, Ms 20, 1906, par. 32; 1906][https://egwwritings.org/read?panels=p14071.10296041&index=0]


\egw{\textbf{Le Saint-Esprit a une personnalité}, \textbf{\underline{sinon}} Il ne pourrait pas \textbf{rendre témoignage} à nos esprits et avec nos esprits que nous sommes les enfants de Dieu. \textbf{Il doit aussi être une \underline{personne divine}}, \textbf{\underline{sinon}} Il ne pourrait pas \textbf{sonder les secrets} qui sont cachés \textbf{dans la pensée de Dieu}.}[21LtMs, Ms 20, 1906, par. 32; 1906][https://egwwritings.org/read?panels=p14071.10296041&index=0]


\egw{\textbf{The Holy Spirit is a person}; \textbf{\underline{for}} He \textbf{beareth witness} with our spirits that we are the children of God.}[21LtMs, Ms 20, 1906, par. 31; 1906][https://egwwritings.org/read?panels=p14071.10296040&index=0]


\egw{\textbf{Le Saint-Esprit est une personne} ; \textbf{\underline{car}} Il \textbf{rend témoignage} avec nos esprits que nous sommes les enfants de Dieu.}[21LtMs, Ms 20, 1906, par. 31; 1906][https://egwwritings.org/read?panels=p14071.10296040&index=0]


The qualities or states that define the Holy Spirit as a person are explicitly mentioned in the provided quotations. These include the ability to bear witness and search out the mind. Further support can be found in Scripture, which attributes actions to the Holy Spirit such as speaking (\textit{Acts 13:2}), teaching (\textit{John 14:26; 1 Corinthians 2:13}), making decisions (\textit{Acts 15:28}), and experiencing emotions (\textit{Ephesians 4:30}), among others. These \textit{qualities }collectively affirm the personality of the Holy Spirit. Can these same qualities be also applied to the Father and the Son? Most certainly. However, unlike the Father and the Son, the Holy Spirit is distinguished by the absence of a material, tangible form. When Ellen White questioned Christ about the \emcap{personality of God}, her inquiry specifically targeted the personal form as the defining quality of the Father's personality.


Les qualités ou états qui définissent le Saint-Esprit comme une personne sont explicitement mentionnés dans les citations fournies. Ceux-ci incluent la capacité de rendre témoignage et de sonder la pensée. Un soutien supplémentaire peut être trouvé dans l'Écriture, qui attribue des actions au Saint-Esprit telles que parler (\textit{Actes 13:2}), enseigner (\textit{Jean 14:26; 1 Corinthiens 2:13}), prendre des décisions (\textit{Actes 15:28}), et éprouver des émotions (\textit{Éphésiens 4:30}), entre autres. Ces \textit{qualités} affirment collectivement la personnalité du Saint-Esprit. Ces mêmes qualités peuvent-elles être également appliquées au Père et au Fils ? Très certainement. Cependant, contrairement au Père et au Fils, le Saint-Esprit se distingue par l'absence d'une forme matérielle et tangible. Quand Ellen White a interrogé Christ sur la \emcap{personnalité de Dieu}, son enquête visait spécifiquement la forme personnelle comme la qualité définissant la personnalité du Père.


\egw{I have often \textbf{seen }the lovely Jesus, that \textbf{He is a person}. \textbf{I asked Him if His Father \underline{was a person} and \underline{had a form} like Himself}. Said Jesus, ‘\textbf{I am in the express image of My Father's person}.’}[EW 77.1; 1882][https://egwwritings.org/read?panels=p28.490&index=0]


\egw{J'ai souvent \textbf{vu} le beau Jésus, qu’\textbf{Il est une personne}. \textbf{Je Lui ai demandé si Son Père \underline{était une personne} et \underline{avait une forme} comme Lui-même}. Jésus a dit : ‘\textbf{Je suis l'empreinte de la personne de Mon Père}.’}[EW 77.1; 1882][https://egwwritings.org/read?panels=p28.490&index=0]


This brings us to a profound distinction in how the personality of the Holy Spirit is understood, as opposed to that of the Father and the Son. Ellen White describes the Holy Spirit as a spiritual manifestation of Christ, drawing a clear line between the outward, visible manifestation of Christ and His spiritual manifestation. This contrast underscores the unique nature of the Holy Spirit's presence and action in the world, distinct from the physical presence of Christ and the Father. Pay attention to the contrast between the outward, visible manifestation of Christ, and His spiritual manifestation:


Cela nous amène à une distinction profonde dans la façon dont la personnalité du Saint-Esprit est comprise, par opposition à celle du Père et du Fils. Ellen White décrit le Saint-Esprit comme une manifestation spirituelle de Christ, traçant une ligne claire entre la manifestation extérieure et visible de Christ et Sa manifestation spirituelle. Ce contraste souligne la nature unique de la présence et de l'action du Saint-Esprit dans le monde, distincte de la présence physique de Christ et du Père. Prêtez attention au contraste entre la manifestation extérieure et visible de Christ, et Sa manifestation spirituelle :


\egw{That \textbf{Christ }should \textbf{manifest Himself} to them, and yet \textbf{be invisible to the world}, was a mystery to the disciples. They could not understand \textbf{the words of Christ in their \underline{spiritual sense}}. \textbf{They were thinking of \underline{the outward, visible manifestation}}. They could not take in the fact that they could have \textbf{the presence of Christ with them}, and \textbf{yet He be unseen by the world}. \textbf{They did not understand the meaning of \underline{a spiritual manifestation}}.}[ST November 18, 1897, par. 6; 1897][https://egwwritings.org/read?panels=p820.14727&index=0]


\egw{Que \textbf{Christ} devait \textbf{Se manifester} à eux, et pourtant \textbf{être invisible au monde}, était un mystère pour les disciples. Ils ne pouvaient pas comprendre \textbf{les paroles de Christ dans leur \underline{sens spirituel}}. \textbf{Ils pensaient à \underline{la manifestation extérieure et visible}}. Ils ne pouvaient pas saisir le fait qu'ils pouvaient avoir \textbf{la présence de Christ avec eux}, et \textbf{pourtant qu'Il soit invisible au monde}. \textbf{Ils ne comprenaient pas la signification d’\underline{une manifestation spirituelle}}.}[ST November 18, 1897, par. 6; 1897][https://egwwritings.org/read?panels=p820.14727&index=0]


The Holy Spirit is not a person in the physical sense but is manifested in a spiritual sense. If the exclusive understanding of the personality of the Holy Spirit is applied to the Father, then consequently His physical form of a person is done away. His personality is spiritualized. This is why Ellen White critically labeled Kellogg's perspective as spiritualism. Do you know which doctrine, in particular, has a core tenet, that the Father and the Holy Spirit are co-equal in their personalities? It is \textit{the doctrine of the trinity}. Could it be possible that Dr. Kellogg was actually raising the theological side of questions of the trinity?


Le Saint-Esprit n'est pas une personne au sens physique mais est manifesté dans un sens spirituel. Si la compréhension exclusive de la personnalité du Saint-Esprit est appliquée au Père, alors par conséquent Sa forme physique de personne est supprimée. Sa personnalité est spiritualisée. C'est pourquoi Ellen White a qualifié de manière critique la perspective de Kellogg de spiritualisme. Savez-vous quelle doctrine, en particulier, a comme principe fondamental que le Père et le Saint-Esprit sont co-égaux dans leurs personnalités ? C'est \textit{la doctrine de la Trinité}. Serait-il possible que le Dr Kellogg soulevait en fait le côté théologique des questions de la trinité ?


\section*{Kellogg’s confession about the Living Temple}


\section*{La confession de Kellogg à propos du Temple Vivant}


In his interview with G. W. Amadon and A. C. Bourdeau, one month after being disfellowshipped, he confessed that he unintentionally brought the theological side of the question of the Trinity into his book “The Living Temple”.


Dans son entretien avec G. W. Amadon et A. C. Bourdeau, un mois après avoir été excommunié, il a confessé qu'il avait involontairement introduit le côté théologique de la question de la Trinité dans son livre « Le Temple Vivant ».


\others{\textbf{Now, I thought I had cut out entirely the theological side of questions of \underline{the trinity and all that sort of things}}. \textbf{I didn't mean to \underline{put it in} at all}, and I took pains to state in the preface that I did not. I never dreamed of such a thing as \textbf{any theological question being} \textbf{\underline{brought into it}}. I only wanted to show that \textbf{the heart does not beat of its own motion but that it is the power of God that keeps it going}.}[Kellogg vs. The Brethren: His Last Interview as an Adventist, p. 58.][https://forgotten-pillar.s3.us-east-2.amazonaws.com/1990\_kellogg\_vs\_brethren\_lastInterview\_oct7\_1907\_spectrum\_v20\_n3-4.pdf]


\others{\textbf{Maintenant, je pensais avoir complètement éliminé le côté théologique des questions de \underline{la trinité et toutes ces sortes de choses}}. \textbf{Je n'avais pas l'intention de \underline{l'y mettre} du tout}, et j'ai pris soin de déclarer dans la préface que je ne l'avais pas fait. Je n'ai jamais imaginé une telle chose qu’\textbf{une question théologique soit} \textbf{\underline{introduite dedans}}. Je voulais seulement montrer que \textbf{le cœur ne bat pas de son propre mouvement mais que c'est la puissance de Dieu qui le fait continuer}.}[Kellogg vs. The Brethren: His Last Interview as an Adventist, p. 58.][https://forgotten-pillar.s3.us-east-2.amazonaws.com/1990\_kellogg\_vs\_brethren\_lastInterview\_oct7\_1907\_spectrum\_v20\_n3-4.pdf]


If we were to look in his book for trinitarian expressions, we would not find any. Would that be a proof that Kellogg is disingenuous in his confession? The only thing we find is the teaching that is stepping off of the foundation of our faith—the \emcap{fundamental principles}—regarding the \emcap{personality of God} and where His presence is. The trinitarian expressions are not there but his sentiments regarding the \emcap{personality of God} are in line with the trinitarian sentiments on God’s person. These sentiments are deceptive and Kellogg was rebuked for them. When he wanted to explicitly state the belief in the Trinity doctrine, in hopes of fixing the book, he was again rebuked by the words, \egwinline{\textbf{Patchwork theories} cannot be accepted by those who are loyal to the faith} and to the \emcap{Fundamental Principles}\footnote{\href{https://egwwritings.org/?ref=en_Lt253-1903.28&para=9980.36}{EGW, Lt253-1903.28; 1903}}. The crucial problem of the Trinity doctrine, in regard to the \emcap{personality of God}, is the underlying assumption that all Three, the Father, the Son, and the Holy Spirit, possess the same type of personality in such a way that They make one monotheistic God. In this light, we may understand Kellogg's assertions over the personality of the Holy Spirit, that the Holy Spirit is the third person of the Godhead. Dr. Kellogg quoted Ellen White when asserting his claims; although he used the same words, he had a wrong sentiment. In light of Dr. Kellogg’s confession, for including \others{\textbf{the theological side of questions of \underline{the trinity}}}, and His assertion that \others{\textbf{the whole thing may be simmered down to the question}: \textbf{\underline{Is the Holy Ghost a person}}?}, we may see the unspoken premise that the Father and the Son are in the same way persons as is the Holy Spirit. This is why Brother Butler wrote to him regarding the personality of the Holy Spirit: \others{\textbf{It is not a person walking around on foot, or flying \underline{as a literal being}, \underline{in any such sense as Christ and the Father are} – at least, if it is, it is utterly beyond my comprehension of the meaning of language or words.}}[Letter from G. I. Butler to J. H. Kellogg, April 5 1904.]


Si nous devions chercher dans son livre des expressions trinitaires, nous n'en trouverions aucune. Serait-ce une preuve que Kellogg est malhonnête dans sa confession ? La seule chose que nous trouvons est l'enseignement qui s'écarte du fondement de notre foi—les \emcap{principes fondamentaux}—concernant la \emcap{personnalité de Dieu} et où Sa présence se trouve. Les expressions trinitaires ne sont pas là mais son raisonnement concernant la \emcap{personnalité de Dieu} est en accord avec le raisonnement trinitaire sur la personne de Dieu. Ce raisonnement est trompeur et Kellogg a été réprimandé pour cela. Quand il a voulu déclarer explicitement la croyance en la doctrine de la Trinité, dans l'espoir de corriger le livre, il a été de nouveau réprimandé par les mots, \egwinline{\textbf{Les théories de rapiécement} ne peuvent être acceptées par ceux qui sont loyaux à la foi} et aux \emcap{Principes Fondamentaux}\footnote{\href{https://egwwritings.org/?ref=en_Lt253-1903.28&para=9980.36}{EGW, Lt253-1903.28; 1903}}. Le problème crucial de la doctrine de la Trinité, en ce qui concerne la \emcap{personnalité de Dieu}, est l'hypothèse sous-jacente que tous les Trois, le Père, le Fils et le Saint-Esprit, possèdent le même type de personnalité de telle manière qu'Ils forment un seul Dieu monothéiste. Dans cette lumière, nous pouvons comprendre les affirmations de Kellogg sur la personnalité du Saint-Esprit, que le Saint-Esprit est la troisième personne de la Divinité. Le Dr Kellogg a cité Ellen White en affirmant ses revendications ; bien qu'il ait utilisé les mêmes mots, il avait un raisonnement erroné. À la lumière de la confession du Dr Kellogg, pour inclure \others{\textbf{le côté théologique des questions de \underline{la trinité}}}, et son affirmation que \others{\textbf{toute la chose peut être réduite à la question} : \textbf{\underline{Le Saint-Esprit est-il une personne}} ?}, nous pouvons voir la prémisse non dite que le Père et le Fils sont de la même manière des personnes comme l'est le Saint-Esprit. C'est pourquoi Frère Butler lui a écrit concernant la personnalité du Saint-Esprit : \others{\textbf{Ce n'est pas une personne marchant à pied, ou volant \underline{comme un être littéral}, \underline{dans le même sens que Christ et le Père le sont} – du moins, si c'est le cas, c'est totalement au-delà de ma compréhension du sens du langage ou des mots.}}[Lettre de G. I. Butler à J. H. Kellogg, 5 avril 1904.]


\section*{The presence of God manifested in nature}


\section*{La présence de Dieu manifestée dans la nature}


From the works of our pioneers, we have seen that the personality of the Holy Ghost is most clearly expressed in terms of God's presence. Sister White told us that the Living Temple \egwinline{introduces that which is naught but speculation in \textbf{regard to the personality of God and where His presence is}.}[SpTB02 51.3; 1904][https://egwwritings.org/read?panels=p417.262] The \emcap{personality of God} and where His presence is are two mutually inclusive doctrines; one affirms the other. Deny one, and you deny the other. This notion is clearly seen in the book, the Living Temple. In the previous sections, we read Kellogg's arguments for the \emcap{personality of God} taken from his book. He argued that it is unprofitable to talk about God's shape or any tangible form. He raised skepticism in the reality of God as a definite, material, and tangible Being. If God is spirit, possessing no form nor body, then He is not restricted in His presence to one locality; this was the sentiment Kellogg advocated in the Living Temple.


D'après les œuvres de nos pionniers, nous avons vu que la personnalité du Saint-Esprit est le plus clairement exprimée en termes de présence de Dieu. Sœur White nous a dit que Le Temple Vivant \egwinline{introduit ce qui n'est que spéculation en ce qui concerne \textbf{la personnalité de Dieu et où Sa présence se trouve}.}[SpTB02 51.3; 1904][https://egwwritings.org/read?panels=p417.262] La \emcap{personnalité de Dieu} et où Sa présence se trouve sont deux doctrines mutuellement inclusives ; l'une affirme l'autre. Niez l'une, et vous niez l'autre. Cette notion est clairement vue dans le livre, Le Temple Vivant. Dans les sections précédentes, nous avons lu les arguments de Kellogg pour la \emcap{personnalité de Dieu} tirés de son livre. Il a soutenu qu'il est inutile de parler de la forme de Dieu ou de toute forme tangible. Il a soulevé le scepticisme dans la réalité de Dieu comme un Être défini, matériel et tangible. Si Dieu est esprit, ne possédant ni forme ni corps, alors Il n'est pas restreint dans Sa présence à une localité ; c'était le raisonnement que Kellogg défendait dans Le Temple Vivant.


\others{Says one, ‘\textbf{God may be \underline{present by his Spirit}, or by his power, but \underline{certainly God himself} cannot be present everywhere at once}.’ We answer: How can power be separated from the source of power? \textbf{Where God's Spirit is at work}, where God's power is manifested, \textbf{God \underline{himself} is actually and truly present}…}[John H. Kellogg, The Living Temple, p.28.][https://archive.org/details/J.H.Kellogg.TheLivingTemple1903/page/n29/]


\others{Dit quelqu'un, ‘\textbf{Dieu peut être \underline{présent par son Esprit}, ou par sa puissance, mais \underline{certainement Dieu lui-même} ne peut pas être présent partout à la fois}.’ Nous répondons : Comment la puissance peut-elle être séparée de la source de puissance ? \textbf{Où l'Esprit de Dieu est à l'œuvre}, où la puissance de Dieu est manifestée, \textbf{Dieu \underline{lui-même} est réellement et véritablement présent}…}[John H. Kellogg, Le Temple Vivant, p.28.][https://archive.org/details/J.H.Kellogg.TheLivingTemple1903/page/n29/]


When Dr. Kellogg wrote \others{Says one, ‘God may be present by His Spirit…’}, he referred to the sentiments of our pioneers who were loyal to the \emcap{Fundamental Principles}. This is the most obvious point where Dr. Kellogg stepped off from the \emcap{Fundamental Principles}. Is this step in harmony with the doctrine of the Trinity? Examining our current stance in the Fundamental Beliefs \#2, we see that one God, as a unity of three persons, is not everywhere present through the agency of the Holy Spirit, but rather is everywhere present by Himself.


Quand le Dr Kellogg a écrit \others{Dit quelqu'un, ‘Dieu peut être présent par Son Esprit…‘}, il faisait référence au raisonnement de nos pionniers qui étaient loyaux aux \emcap{Principes Fondamentaux}. C'est le point le plus évident où le Dr Kellogg s'est écarté des \emcap{Principes Fondamentaux}. Ce pas est-il en harmonie avec la doctrine de la Trinité ? En examinant notre position actuelle dans les Croyances Fondamentales \#2, nous voyons qu'un seul Dieu, comme une unité de trois personnes, n'est pas partout présent par l'intermédiaire du Saint-Esprit, mais plutôt est partout présent par Lui-même.


\others{There is \textbf{one God}: Father, Son, and Holy Spirit, \textbf{a unity of three} coeternal \textbf{Persons}. God is immortal, all-powerful… and \textbf{ever present}.}[Fundamental Beliefs of Seventh-day Adventist, \#2 Trinity; 2020 Edition][https://www.adventist.org/wp-content/uploads/2020/06/ADV-28Beliefs2020.pdf]


\others{Il y a \textbf{un seul Dieu} : Père, Fils et Saint-Esprit, \textbf{une unité de trois} \textbf{Personnes} coéternelles. Dieu est immortel, tout-puissant… et \textbf{toujours présent}.}[Croyances Fondamentales des Adventistes du Septième Jour, \#2 Trinité ; Édition 2020][https://www.adventist.org/wp-content/uploads/2020/06/ADV-28Beliefs2020.pdf]


\section*{Dr. Kellogg's perception of God}


\section*{La perception de Dieu du Dr Kellogg}


In examining the surrounding controversy over the Living Temple, we truly see that Dr. Kellogg raised \others{the theological side of questions of the trinity.}[Kellogg vs. The Brethren: His Last Interview as an Adventist, p. 58.][https://forgotten-pillar.s3.us-east-2.amazonaws.com/1990\_kellogg\_vs\_brethren\_lastInterview\_oct7\_1907\_spectrum\_v20\_n3-4.pdf] Another question we raise in examining Kellogg's sentiments with the \emcap{Fundamental Principles} is whom does he address in terms of “\textit{one God}”? There is no data to directly answer that question, but there is plenty of data which suggests that Dr. Kellogg's understanding of “\textit{one God}” was a Trinitarian understanding. His letter to W. W. Prescott is one piece of evidence supporting that notion:


En examinant la controverse entourant Le Temple Vivant, nous voyons vraiment que le Dr Kellogg a soulevé \others{le côté théologique des questions de la trinité.}[Kellogg vs. Les Frères : Sa Dernière Interview en tant qu'Adventiste, p. 58.][https://forgotten-pillar.s3.us-east-2.amazonaws.com/1990\_kellogg\_vs\_brethren\_lastInterview\_oct7\_1907\_spectrum\_v20\_n3-4.pdf] Une autre question que nous soulevons en examinant le raisonnement de Kellogg avec les \emcap{Principes Fondamentaux} est à qui s'adresse-t-il en termes de “\textit{un seul Dieu}” ? Il n'y a pas de données pour répondre directement à cette question, mais il y a beaucoup de données qui suggèrent que la compréhension du Dr Kellogg de “\textit{un seul Dieu}” était une compréhension Trinitaire. Sa lettre à W. W. Prescott est une preuve soutenant cette notion :


\others{The difference is this: \textbf{When we say God} is in the tree, the word ‘\textbf{God}’ \textbf{is understood in its most comprehensive sense}, and people understand the meaning to be \textbf{that the Godhead} is in the tree, \textbf{God the Father, God the Son, and God the Holy Spirit}, whereas the proper understanding in order \textbf{that wholesome conceptions} should be preserved in our minds, is that God the Father sits upon his throne in heaven where God the Son is also; \textbf{while God's life, or spirit or presence is the all-pervading power which is carrying out the will of God in all the universe}.}[Letter: Dr. Kellogg to Prof. W. W. Prescott, Oct. 25, 1903][https://forgotten-pillar.s3.us-east-2.amazonaws.com/1903-10-25-JHKellogg-to-W.W.Prescott.pdf]


\others{La différence est celle-ci : \textbf{Quand nous disons que Dieu} est dans l'arbre, le mot ‘\textbf{Dieu}’ \textbf{est compris dans son sens le plus complet}, et les gens comprennent que le sens est \textbf{que la Divinité} est dans l'arbre, \textbf{Dieu le Père, Dieu le Fils, et Dieu le Saint-Esprit}, alors que la compréhension appropriée afin \textbf{que des conceptions saines} soient préservées dans nos esprits, est que Dieu le Père siège sur son trône au ciel où Dieu le Fils est aussi ; \textbf{tandis que la vie de Dieu, ou esprit ou présence est la puissance omniprésente qui exécute la volonté de Dieu dans tout l'univers}.}[Lettre : Dr Kellogg au Prof. W. W. Prescott, 25 oct. 1903][https://forgotten-pillar.s3.us-east-2.amazonaws.com/1903-10-25-JHKellogg-to-W.W.Prescott.pdf]


In the next chapter, we will make our case: if the given \others{wholesome conception} of God advocated by Dr. Kellogg was true, then his clarification of the Holy Spirit being \others{God's life, or spirit or presence is the all-pervading power which is carrying out the will of God in all the universe} would truly solve the entire difficulty of the Living Temple. But that was not the case. Dr. Kellogg's true problem was his perception of God, and his trinitarian stance was not solving the real issue—the \emcap{personality of God}.


Dans le chapitre suivant, nous présenterons notre argument : si la \others{conception saine} donnée de Dieu défendue par le Dr Kellogg était vraie, alors sa clarification du Saint-Esprit étant \others{la vie de Dieu, ou esprit ou présence est la puissance omniprésente qui exécute la volonté de Dieu dans tout l'univers} résoudrait vraiment toute la difficulté du Temple Vivant. Mais ce n'était pas le cas. Le vrai problème du Dr Kellogg était sa perception de Dieu, et sa position trinitaire ne résolvait pas le vrai problème—la \emcap{personnalité de Dieu}.


There is another revealing letter showing us the consequences of raising \others{the theological side of questions of the trinity.} Writing to his friend Dr. Hayward, Dr. Kellogg reflected:


Il y a une autre lettre révélatrice qui nous montre les conséquences de soulever \others{le côté théologique des questions de la trinité.} Écrivant à son ami le Dr Hayward, le Dr Kellogg réfléchissait :


\others{\textbf{These theologians} have sought to darken the minds of the people and to make \textbf{this sweet and beautiful truth \underline{appear loathsome} to them, by drawing into it \underline{the old controversy about the Trinity}}.}


\others{\textbf{Ces théologiens} ont cherché à obscurcir l'esprit des gens et à faire \textbf{paraître cette vérité douce et belle \underline{répugnante} pour eux, en y introduisant \underline{l'ancienne controverse sur la Trinité}}.}


\othersnogap{I never raised the question as to \textbf{which part of God is present in a man}, whether it was \textbf{God, the Father};\textbf{ God, the Son}; or \textbf{God, the Holy Spirit}. The only point was that it is God and not man.}[Letter: Dr. J. H. Kellogg to Dr. Hayward, Aug., 15. 1905][https://forgotten-pillar.s3.us-east-2.amazonaws.com/1903-08-15-kellogg-to-hayward.pdf]


\othersnogap{Je n'ai jamais soulevé la question de savoir \textbf{quelle partie de Dieu est présente dans un homme}, si c'était \textbf{Dieu, le Père} ; \textbf{Dieu, le Fils} ; ou \textbf{Dieu, le Saint-Esprit}. Le seul point était que c'est Dieu et non l'homme.}[Lettre : Dr J. H. Kellogg au Dr Hayward, 15 août 1905][https://forgotten-pillar.s3.us-east-2.amazonaws.com/1903-08-15-kellogg-to-hayward.pdf]


Here we see the tensions between Dr. Kellogg and certain Seventh-day Adventist theologians of that time, where Dr. Kellogg's \others{sweet and beautiful truth} of God's divine immanence got entangled with \others{the old controversy about the Trinity}. This tells us that in the days of Dr. Kellogg, the doctrine of Trinity was controversial, and certainly it was not regarded as something positive, but rather as something which made Kellogg's teachings \others{loathsome}. But who were these theologians Dr. Kellogg referred to? He did not name anyone in his letter to Dr. Hayward, but we can get the idea of whom \others{these theologians} were based on his letter sent 10 days earlier to I. G. Butler\footnote{\href{https://forgotten-pillar.s3.us-east-2.amazonaws.com/1905-08-05-kellogg-butler.pdf}{Letter: J. H. Kellogg to I. G. Butler, Aug., 5. 1905}}, venting his frustration with the General Conference's bidding with him. These were A. G. Daniells, W. C. White, and W. W. Prescott. We can also include G. I. Butler himself to that group, since he also was a theologian participating in this \others{old controversy about the Trinity}. All of these people held leading positions within the Seventh-day Adventist church, and all of them were non-Trinitarians. The argument is being made that the issue with Dr. Kellogg's teaching lies somewhere other than his trinitarian sentiments, because supposedly the church was trinitarian at that time, and supposedly Ellen White was trinitarian herself. \footnote{This is currently the popular narrative promoted by laity.} If this was the case, and in this mix of truth and error, should we not have at least some defense of the trinity doctrine, dissecting it from error? We have not found any such data. Instead, all data we have is in defense of the \emcap{Fundamental Principles}, and the doctrine on the presence and the \emcap{personality of God}, which both are opposed to the doctrine of the Trinity. Ellen White said of the truth: the Trinity doctrine \egwinline{cannot be accepted by those who are \textbf{loyal to the faith and to the principles} that have withstood all the opposition of satanic influences.}[Lt253-1903.28; 1903][https://egwwritings.org/read?panels=p14068.9980036]


Ici, nous voyons les tensions entre le Dr Kellogg et certains théologiens adventistes du septième jour de cette époque, où la \others{vérité douce et belle} du Dr Kellogg de l'immanence divine de Dieu s'est enchevêtrée avec \others{l'ancienne controverse sur la Trinité}. Cela nous dit qu'à l'époque du Dr Kellogg, la doctrine de la Trinité était controversée, et certainement elle n'était pas considérée comme quelque chose de positif, mais plutôt comme quelque chose qui rendait les enseignements de Kellogg \others{répugnants}. Mais qui étaient ces théologiens auxquels le Dr Kellogg faisait référence ? Il n'a nommé personne dans sa lettre au Dr Hayward, mais nous pouvons avoir une idée de qui étaient \others{ces théologiens} sur la base de sa lettre envoyée 10 jours plus tôt à I. G. Butler\footnote{\href{https://forgotten-pillar.s3.us-east-2.amazonaws.com/1905-08-05-kellogg-butler.pdf}{Lettre : J. H. Kellogg à I. G. Butler, 5 août 1905}}, exprimant sa frustration face aux enchères de la Conférence Générale avec lui. Il s'agissait de A. G. Daniells, W. C. White et W. W. Prescott. Nous pouvons également inclure G. I. Butler lui-même dans ce groupe, puisqu'il était également un théologien participant à cette \others{ancienne controverse sur la Trinité}. Toutes ces personnes occupaient des postes de direction au sein de L'Église Adventiste du Septième Jour, et toutes étaient non-trinitaires. L'argument avancé est que le problème avec l'enseignement du Dr Kellogg se situe ailleurs que dans ses sentiments trinitaires, parce que supposément l'église était trinitaire à cette époque, et supposément Ellen White était elle-même trinitaire.\footnote{C'est actuellement le récit populaire promu par les laïcs.} Si c'était le cas, et dans ce mélange de vérité et d'erreur, ne devrions-nous pas avoir au moins une certaine défense de la doctrine de la Trinité, la disséquant de l'erreur ? Nous n'avons trouvé aucune donnée de ce type. Au lieu de cela, toutes les données que nous avons sont en défense des \emcap{Principes Fondamentaux}, et de la doctrine sur la présence et la \emcap{personnalité de Dieu}, qui sont toutes deux opposées à la doctrine de la Trinité. Ellen White a dit de la vérité : la doctrine de la Trinité \egwinline{ne peut être acceptée par ceux qui sont \textbf{fidèles à la foi et aux principes} qui ont résisté à toute l'opposition des influences sataniques.}[Lt253-1903.28; 1903][https://egwwritings.org/read?panels=p14068.9980036]


In this short reflection on differences between Dr. Kellogg's sentiments and the \emcap{Fundamental Principles} from which he stepped off, we can recognize the following characteristics which are akin to the Trinity doctrine:


Dans cette courte réflexion sur les différences entre le raisonnement du Dr Kellogg et les \emcap{Principes Fondamentaux} dont il s'est écarté, nous pouvons reconnaître les caractéristiques suivantes qui sont apparentées à la doctrine de la Trinité :


\begin{itemize}
    \item The word ‘God’ represents the wholesome conception of God as God the Father, God the Son, and God the Holy Spirit.
    \item God is everywhere present by Himself.
    \item The quality or state of the Father being a person is equalized to that of the Holy Spirit.\footnote{\href{https://www.adventist.org/wp-content/uploads/2020/06/ADV-28Beliefs2020.pdf}{Fundamental Beliefs \#5}: \others{He \normaltext{[the Holy Spirit]} \textbf{is as much a person} \underline{as} are \textbf{the Father} and the Son}; \href{https://www.adventist.org/wp-content/uploads/2020/06/ADV-28Beliefs2020.pdf}{Fundamental Beliefs \#3}: \others{\textbf{The qualities} and powers \textbf{exhibited in} the Son and \textbf{the Holy Spirit are \underline{also} those of the Father}}}
\end{itemize}


\begin{itemize}
    \item Le mot « Dieu » représente la conception complète de Dieu comme Dieu le Père, Dieu le Fils et Dieu le Saint-Esprit.
    \item Dieu est partout présent par Lui-même.
    \item La qualité ou l'état du Père d'être une personne est égalisé à celui du Saint-Esprit.\footnote{\href{https://www.adventist.org/wp-content/uploads/2020/06/ADV-28Beliefs2020.pdf}{Croyances Fondamentales \#5} : \others{Il \normaltext{[le Saint-Esprit]} \textbf{est autant une personne} \underline{que} le sont \textbf{le Père} et le Fils} ; \href{https://www.adventist.org/wp-content/uploads/2020/06/ADV-28Beliefs2020.pdf}{Croyances Fondamentales \#3} : \others{\textbf{Les qualités} et pouvoirs \textbf{manifestés dans} le Fils et \textbf{le Saint-Esprit sont \underline{aussi} ceux du Père}}}
\end{itemize}


These three characteristics of Dr. Kellogg's sentiments depart from the foundation of our faith—the \emcap{Fundamental Principles}—but are in harmony with the teachings of the Trinity. In saying this, we are not claiming that Dr. Kellogg is responsible for the acceptance of the Trinity doctrine into our ranks, but rather that the Trinity doctrine was Kellogg's justification for stepping off from the foundation of our faith, established at the beginning of our work. The true problem was \textit{stepping off} from the \emcap{fundamental principles}, and both Dr. Kellogg and we as a church have made those steps. The difference is that Dr. Kellogg landed in pantheism, while we landed on the \#2 point of the Fundamental Beliefs.


Ces trois caractéristiques du raisonnement du Dr Kellogg s'écartent du fondement de notre foi — les \emcap{Principes Fondamentaux} — mais sont en harmonie avec les enseignements de la Trinité. En disant cela, nous ne prétendons pas que le Dr Kellogg est responsable de l'acceptation de la doctrine de la Trinité dans nos rangs, mais plutôt que la doctrine de la Trinité était la justification de Kellogg pour s'écarter du fondement de notre foi, établi au début de notre œuvre. Le vrai problème était de \textit{s'écarter} des \emcap{principes fondamentaux}, et le Dr Kellogg et nous en tant qu'église avons fait ces pas. La différence est que le Dr Kellogg a atterri dans le panthéisme, tandis que nous avons atterri sur le point \#2 des Croyances Fondamentales.


In the following chapter, we will examine Dr. Kellogg's teaching that God sustains all life, and how this truth, in combination with a false perception of God and His personality, led him to become a pantheist.


Dans le chapitre suivant, nous examinerons l'enseignement du Dr Kellogg selon lequel Dieu soutient toute vie, et comment cette vérité, en combinaison avec une fausse perception de Dieu et de Sa personnalité, l'a conduit à devenir panthéiste.


% Dr. Kellogg and the Trinity doctrine

\begin{titledpoem}
    
    \stanza{
        In Kellogg’s quest, the question posed, \\
        "The Spirit – how is He composed?" \\
        The issue stirred a great debate, \\
        How does this mystery relate?
    }

    \stanza{
        The question was beyond the seen \\
        To trinity J.H. did lean \\
        The Father wasn’t bound by space? \\
        Without a body or a face?
    }

    \stanza{
        To Ellen, Jesus did inform \\
        Like Him, His Father had a form \\
        “I am His image as express, \\
        Revealing form and righteousness.”
    }

    \stanza{
        In vision was the truth revealed \\
        The inspiration, it was sealed \\
        The Father’s form upon the throne \\
        And Christ with form just like His own.
    }

    \stanza{
        The Spirit’s personality \\
        In actions and in quality \\
        A role distinct, within us dwells. \\
        The mind of Christ the Spirit tells.
    }

    \stanza{
        God’s power and His presence show \\
        Wherever God would have it go \\
        And thus, He’s present everywhere \\
        Invisible, His Spirit there.
    }

    \stanza{
        The Living Temple showed a flaw \\
        A dangerous error Ellen saw \\
        The wayward theories in his mind \\
        Blocked him from truth he could not find.
    }

    \stanza{
        He went off searching on his own \\
        And did not follow what was shown \\
        If he had stayed where God had led, \\
        His teaching would have never spread.
    }
    
\end{titledpoem}