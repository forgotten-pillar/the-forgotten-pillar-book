\qrchapter{https://forgottenpillar.com/rsc/en-fp-chapter5}{The patchwork theories - Lt253-1903}


\qrchapter{https://forgottenpillar.com/rsc/en-fp-chapter5}{Les théories de rapiècement - Lt253-1903}


\egw{Dear Brother,—}


\egw{Cher frère,—}


\egwnogap{\textbf{I must tell you that your ideas in regard to some things \underline{have been decidedly wrong}.} I would that you could see your errors. \textbf{The book Living Temple \underline{is not to be patched up}, a few changes made in it, and then advertised and praised as a valuable production}. It would be better to present the physiological parts in another book under another title. \textbf{When you wrote that book}, \textbf{you were not under the inspiration of God}. There was by your side the one who inspired Adam to look at God in a false light. Your whole heart needs to be changed, thoroughly and entirely cleansed.}[Lt253-1903.1; 1903][https://egwwritings.org/read?panels=p9980.7]


\egwnogap{\textbf{Je dois vous dire que vos idées concernant certaines choses \underline{ont été décidément erronées}.} Je souhaiterais que vous puissiez voir vos erreurs. \textbf{Le livre Le Temple Vivant \underline{ne doit pas être rapiécé}, quelques changements y être apportés, puis être annoncé et loué comme une production de valeur}. Il serait préférable de présenter les parties physiologiques dans un autre livre sous un autre titre. \textbf{Lorsque vous avez écrit ce livre}, \textbf{vous n'étiez pas sous l'inspiration de Dieu}. Il y avait à vos côtés celui qui a inspiré Adam à regarder Dieu sous un faux jour. Votre cœur entier a besoin d'être changé, nettoyé à fond et entièrement.}[Lt253-1903.1; 1903][https://egwwritings.org/read?panels=p9980.7]


\egwnogap{\textbf{My brother, do not allow yourself to be alienated from your ministering brethren who tell you of your dangers. Those who faithfully and frankly tell you of your errors are your best friends.} I am sorry, very sorry, for your medical associates. They have been unfaithful to God and untrue to you in failing to tell you kindly but firmly where you were not working righteously.}[Lt253-1903.2; 1903][https://egwwritings.org/read?panels=p9980.8]


\egwnogap{\textbf{Mon frère, ne vous laissez pas aliéner de vos frères dans le ministère qui vous signalent vos dangers. Ceux qui vous disent fidèlement et franchement vos erreurs sont vos meilleurs amis.} Je suis désolée, très désolée, pour vos associés médicaux. Ils ont été infidèles à Dieu et déloyaux envers vous en ne vous disant pas avec gentillesse mais fermeté où vous ne travailliez pas avec droiture.}[Lt253-1903.2; 1903][https://egwwritings.org/read?panels=p9980.8]


\egwnogap{There are many things that you must overcome before you can be saved. In the heart that is not led by God, there is a something that leads it to desire to be sustained in its wrong course. The men who faithfully tell you the truth, pointing out your mistakes, you have regarded as your enemies. But often they are your best friends and, in telling you wherein you were walking in strange paths, were doing a very disagreeable duty. The Lord’s servants are not to flatter your pride; they are not to stand silent, fearing to say, ‘Why do ye thus?’ They are faithfully to warn you of your danger.}[Lt253-1903.3; 1903][https://egwwritings.org/read?panels=p9980.9]


\egwnogap{Il y a beaucoup de choses que vous devez surmonter avant de pouvoir être sauvé. Dans le cœur qui n'est pas conduit par Dieu, il y a quelque chose qui le pousse à désirer être soutenu dans sa mauvaise voie. Les hommes qui vous disent fidèlement la vérité, en soulignant vos erreurs, vous les avez considérés comme vos ennemis. Mais souvent, ils sont vos meilleurs amis et, en vous disant où vous marchiez sur des chemins étranges, ils accomplissaient un devoir très désagréable. Les serviteurs du Seigneur ne doivent pas flatter votre orgueil ; ils ne doivent pas rester silencieux, craignant de dire : « Pourquoi faites-vous ainsi ? » Ils doivent fidèlement vous avertir de votre danger.}[Lt253-1903.3; 1903][https://egwwritings.org/read?panels=p9980.9]


\egwnogap{\textbf{My husband, Elder Joseph Bates, Father Pierce, Elder Edson, and many others who were keen, noble, and true were among those who, after the passing of the time in 1844, searched for truth}. \textbf{At our important meetings, these men would meet together and search for the truth as for hidden treasure}. I met with them, and we studied and prayed earnestly; for we felt that we must learn God’s truth. Often we remained together until late at night, and sometimes through the entire night, praying for light and studying the Word. As we fasted and prayed, great power came upon us. But I could not understand the reasoning of the brethren. My mind was locked, as it were, and I could not comprehend what we were studying. Then the Spirit of God would come upon me, I would be taken off in vision, and a clear explanation of the passages we had been studying would be given me with instruction as to the position we were to take regarding truth and duty. Again and again this happened. \textbf{A line of truth extending from that time to the time when we shall enter the city of God was plainly marked out before me}, and I gave my brethren and sisters the instruction that the Lord had given me. They knew that when not in vision, I could not understand these matters, and they accepted as light direct from heaven the revelations given me. \textbf{Thus the leading points of our faith as we hold them today were firmly established}. \textbf{\underline{Point after point} was clearly defined, and all the brethren came into harmony}.}[Lt253-1903.4; 1903][https://egwwritings.org/read?panels=p14068.9980010]


\egwnogap{\textbf{Mon mari, l'Ancien Joseph Bates, le Père Pierce, l'Ancien Edson, et beaucoup d'autres qui étaient perspicaces, nobles et fidèles étaient parmi ceux qui, après le passage du temps en 1844, ont cherché la vérité}. \textbf{Lors de nos importantes réunions, ces hommes se réunissaient et cherchaient la vérité comme un trésor caché}. Je me réunissais avec eux, et nous étudiions et priions avec ferveur ; car nous sentions que nous devions apprendre la vérité de Dieu. Souvent, nous restions ensemble jusqu'à tard dans la nuit, et parfois toute la nuit, priant pour la lumière et étudiant la Parole. Alors que nous jeûnions et priions, une grande puissance venait sur nous. Mais je ne pouvais pas comprendre le raisonnement des frères. Mon esprit était comme verrouillé, et je ne pouvais pas comprendre ce que nous étudiions. Puis l'Esprit de Dieu venait sur moi, j'étais emportée en vision, et une explication claire des passages que nous avions étudiés m'était donnée avec des instructions quant à la position que nous devions prendre concernant la vérité et le devoir. Cela s'est produit maintes et maintes fois. \textbf{Une ligne de vérité s'étendant de ce temps jusqu'au moment où nous entrerons dans la cité de Dieu était clairement tracée devant moi}, et j'ai donné à mes frères et sœurs l'instruction que le Seigneur m'avait donnée. Ils savaient que lorsque je n'étais pas en vision, je ne pouvais pas comprendre ces questions, et ils acceptaient comme lumière directe du ciel les révélations qui m'étaient données. \textbf{Ainsi, les points principaux de notre foi tels que nous les soutenons aujourd'hui ont été fermement établis}. \textbf{\underline{Point après point} a été clairement défini, et tous les frères sont parvenus à l'harmonie}.}[Lt253-1903.4; 1903][https://egwwritings.org/read?panels=p14068.9980010]


\egwnogap{\textbf{The whole company of believers were united in the truth}. \textbf{There were those who came in with strange doctrines, but we were never afraid to meet them. Our experience was wonderfully established by the revelations of the Holy Spirit}.}[Lt253-1903.5; 1903][https://egwwritings.org/read?panels=p9980.11]


\egwnogap{\textbf{L'ensemble des croyants était uni dans la vérité}. \textbf{Il y avait ceux qui venaient avec des doctrines étranges, mais nous n'avons jamais eu peur de les rencontrer. Notre expérience a été merveilleusement établie par les révélations du Saint-Esprit}.}[Lt253-1903.5; 1903][https://egwwritings.org/read?panels=p9980.11]


\egwnogap{For two or three years my mind continued to be locked to the Scriptures. In 1846 I was married to Elder James White. It was some time after my second son was born that we were in great perplexity regarding certain points of doctrine. I was praying to the Lord to unlock my mind, that I might understand His Word. Suddenly I seemed to be enshrouded in clear, beautiful light, and ever since, \textbf{the Scriptures have been an open book to me}.}[Lt253-1903.6; 1903][https://egwwritings.org/read?panels=p14068.9980012]


\egwnogap{Pendant deux ou trois ans, mon esprit est resté verrouillé aux Écritures. En 1846, j'ai épousé l'Ancien James White. C'était quelque temps après la naissance de mon second fils que nous étions dans une grande perplexité concernant certains points de doctrine. Je priais le Seigneur de déverrouiller mon esprit, afin que je puisse comprendre Sa Parole. Soudain, j'ai semblé être enveloppée d'une lumière claire et belle, et depuis lors, \textbf{les Écritures ont été un livre ouvert pour moi}.}[Lt253-1903.6; 1903][https://egwwritings.org/read?panels=p14068.9980012]


\egwnogap{I was at that time in Paris, Maine. Old Father Andrews was very sick. For some time he had been a great sufferer from inflammatory rheumatism. He could not move without intense pain. We prayed for him. I laid my hands on his head, and said, “Father Andrews, the Lord Jesus maketh thee whole.” He was healed instantly. He got up and walked about the room, praising God, and saying, “I never saw it on this wise before. Angels of God are in this room.” The glory of God was revealed. \textbf{Light seemed to shine all through the house, and an angel’s hand was laid upon my head. From that time to this I have been able to understand the Word of God.}}[Lt253-1903.7; 1903][https://egwwritings.org/read?panels=p9980.13]


\egwnogap{J'étais à ce moment-là à Paris, dans le Maine. Le vieux Père Andrews était très malade. Depuis quelque temps, il souffrait beaucoup de rhumatismes inflammatoires. Il ne pouvait pas bouger sans douleur intense. Nous avons prié pour lui. J'ai posé mes mains sur sa tête et j'ai dit : « Père Andrews, le Seigneur Jésus te guérit. » Il a été guéri instantanément. Il s'est levé et a marché dans la pièce, louant Dieu et disant : « Je n'ai jamais vu cela de cette manière auparavant. Des anges de Dieu sont dans cette pièce. » La gloire de Dieu a été révélée. \textbf{La lumière semblait briller dans toute la maison, et la main d'un ange a été posée sur ma tête. Depuis ce moment jusqu'à maintenant, j'ai été capable de comprendre la Parole de Dieu.}}[Lt253-1903.7; 1903][https://egwwritings.org/read?panels=p9980.13]


\egwnogap{\textbf{After the passing of the time, we were opposed and cruelly falsified. Erroneous theories were pressed in upon us by men and women who had gone into fanaticism}. I was directed to go to the places where these people were advocating these erroneous theories, and as I went, the power of the Spirit was wonderfully displayed in rebuking the errors that were creeping in. \textbf{\underline{Satan himself, in the person of a man}, was working to make of no effect my testimony regarding the position that we now know to be substantiated by Scripture.}}[Lt253-1903.8; 1903][https://egwwritings.org/read?panels=p9980.14]


\egwnogap{\textbf{Après le passage du temps, nous avons été opposés et cruellement calomniés. Des théories erronées nous ont été imposées par des hommes et des femmes qui étaient tombés dans le fanatisme}. On m'a demandé d'aller aux endroits où ces personnes défendaient ces théories erronées, et comme j'y allais, la puissance de l'Esprit se manifestait merveilleusement en réprimandant les erreurs qui s'infiltraient. \textbf{\underline{Satan lui-même, dans la personne d'un homme}, travaillait à rendre sans effet mon témoignage concernant la position que nous savons maintenant être étayée par l'Écriture.}}[Lt253-1903.8; 1903][https://egwwritings.org/read?panels=p9980.14]


\egwnogap{\textbf{Just such theories as you have presented in Living Temple were presented then}. \textbf{These subtle, deceiving sophistries have again and again sought to find place amongst us. \underline{But I have ever had the same testimony to bear which I now bear regarding the personality of God}}.}[Lt253-1903.9; 1903][https://egwwritings.org/read?panels=p9980.15]


\egwnogap{\textbf{Des théories exactement comme celles que vous avez présentées dans Le Temple Vivant ont été présentées à cette époque}. \textbf{Ces subtiles sophisteries trompeuses ont cherché maintes et maintes fois à trouver place parmi nous. \underline{Mais j'ai toujours eu le même témoignage à porter que celui que je porte maintenant concernant la personnalité de Dieu}}.}[Lt253-1903.9; 1903][https://egwwritings.org/read?panels=p9980.15]


\egwnogap{In (Early Writings, 60, 66, 67)\footnote{It appears that the pages are incorrect. The mentioned paragraphs can be found in Early Writings on pages \href{https://egwwritings.org/read?panels=p28.462&index=0}{70.2}, \href{https://egwwritings.org/read?panels=p28.490&index=0}{77}, and \href{https://egwwritings.org/read?panels=p28.390&index=0}{54.2}.}, are the following statements:}[Lt253-1903.10; 1903][https://egwwritings.org/read?panels=p9980.16]


\egwnogap{Dans (Premiers Écrits, 60, 66, 67)\footnote{Il semble que les pages soient incorrectes. Les paragraphes mentionnés peuvent être trouvés dans Premiers Écrits aux pages \href{https://egwwritings.org/read?panels=p28.462&index=0}{70.2}, \href{https://egwwritings.org/read?panels=p28.490&index=0}{77}, et \href{https://egwwritings.org/read?panels=p28.390&index=0}{54.2}.}, se trouvent les déclarations suivantes :}[Lt253-1903.10; 1903][https://egwwritings.org/read?panels=p9980.16]


\egwnogap{‘May 14, 1851, I saw the beauty and loveliness of Jesus. As I beheld His glory, the thought did not occur to me that I should ever be separated from His presence. \textbf{I saw a light coming from the glory that encircled the Father}, and as it approached near to me, my body shook and trembled like a leaf. I thought that if it should come near me, I would be struck out of existence; but the light passed me. \textbf{Then could I have some sense of the great and terrible \underline{God} with whom we have to do}.’}[Lt253-1903.11; 1903][https://egwwritings.org/read?panels=p9980.17]


\egwnogap{‘Le 14 mai 1851, j'ai vu la beauté et la splendeur de Jésus. Tandis que je contemplais Sa gloire, la pensée ne m'est pas venue que je serais un jour séparée de Sa présence. \textbf{J'ai vu une lumière venant de la gloire qui entourait le Père}, et comme elle s'approchait de moi, mon corps tremblait comme une feuille. Je pensais que si elle venait près de moi, je serais anéantie ; mais la lumière est passée à côté de moi. \textbf{Alors j'ai pu avoir une certaine idée du grand et terrible \underline{Dieu} avec qui nous avons affaire}.’}[Lt253-1903.11; 1903][https://egwwritings.org/read?panels=p9980.17]


\egwnogap{‘I have often seen \textbf{the lovely Jesus, that He is a person}. \textbf{I asked Him if His Father was a person, and had \underline{a form} like Himself}. Said Jesus, ‘\textbf{I am the express image of My Father’s person!}’ [Hebrews 1:3.]}[Lt253-1903.12; 1903][https://egwwritings.org/read?panels=p9980.18]


\egwnogap{‘J'ai souvent vu \textbf{le bien-aimé Jésus, qu'Il est une personne}. \textbf{Je Lui ai demandé si Son Père était une personne, et avait \underline{une forme} comme Lui-même}. Jésus a dit : ‘\textbf{Je suis l'empreinte de sa personne !}’ [Hébreux 1:3.]}[Lt253-1903.12; 1903][https://egwwritings.org/read?panels=p9980.18]


\egwnogap{‘\textbf{I have often seen that the spiritual view took away all the glory of heaven, and that in many minds the throne of David and the lovely person of Jesus have been burned up in the fire of spiritualism}. I have seen that some who have been deceived and led into this error, will be brought out into the light of truth, \textbf{but it will be almost impossible for them to get entirely rid of the deceptive power of spiritualism. Such should make thorough work in confessing their errors, and leaving them forever}.’}[Lt253-1903.13; 1903][https://egwwritings.org/read?panels=p9980.19]


\egwnogap{‘\textbf{J'ai souvent vu que la vue spiritualiste a enlevé toute la gloire du ciel, et que dans l'esprit de beaucoup, le trône de David et la personne bien-aimée de Jésus ont été consumés dans le feu du spiritualisme}. J'ai vu que certains qui ont été trompés et conduits dans cette erreur, seront amenés à la lumière de la vérité, \textbf{mais il leur sera presque impossible de se débarrasser entièrement du pouvoir trompeur du spiritualisme. Ces personnes devraient faire un travail approfondi en confessant leurs erreurs et en les abandonnant pour toujours}.’}[Lt253-1903.13; 1903][https://egwwritings.org/read?panels=p9980.19]


\egwnogap{\textbf{There is a strain of spiritualism \underline{coming in} among our people, and \underline{it will undermine the faith} of those who give place to it, leading them to give heed to seducing spirits and doctrines of devils}. Errors will be presented in a pleasing and flattering manner. The enemy desires to divert the minds of our brethren and sisters from the work of preparing a people to stand in these last days.}[Lt253-1903.14; 1903][https://egwwritings.org/read?panels=p9980.21]


\egwnogap{\textbf{Il y a une tendance au spiritualisme qui \underline{s'introduit} parmi notre peuple, et \underline{cela minera la foi} de ceux qui lui font place, les amenant à prêter attention aux esprits séducteurs et aux doctrines des démons}. Des erreurs seront présentées d'une manière agréable et flatteuse. L'ennemi désire détourner l'esprit de nos frères et sœurs de l'œuvre de préparation d'un peuple qui tiendra ferme en ces derniers jours.}[Lt253-1903.14; 1903][https://egwwritings.org/read?panels=p9980.21]


\egwnogap{I am instructed to warn our brethren and sisters \textbf{not to discuss the nature of our God}. Many of the curious who attempted to open the ark of the testament, to see what was inside, were punished for their presumption. \textbf{We are not to say that the Lord God of heaven is in a leaf, or in a tree; for He is not there. \underline{He sitteth upon His throne in the heavens}.}}[Lt253-1903.15; 1903][https://egwwritings.org/read?panels=p9980.22]


\egwnogap{Je suis chargée d'avertir nos frères et sœurs \textbf{de ne pas discuter de la nature de notre Dieu}. Beaucoup de curieux qui ont tenté d'ouvrir l'arche du testament, pour voir ce qu'il y avait à l'intérieur, ont été punis pour leur présomption. \textbf{Nous ne devons pas dire que le Seigneur Dieu des cieux est dans une feuille, ou dans un arbre ; car Il n'y est pas. \underline{Il siège sur Son trône dans les cieux}.}}[Lt253-1903.15; 1903][https://egwwritings.org/read?panels=p9980.22]


\egwnogap{The work of the Creator as seen in nature reveals His power. But nature is not above God, nor is God in nature as some represent Him to be. God made the world, but the world is not God; it is but the work of His hands. \textbf{Nature reveals the work of a positive, \underline{personal God}, showing that God is, and that He is a rewarder of those who diligently seek Him}.}[Lt253-1903.16, 1903][https://egwwritings.org/read?panels=p9980.23]


\egwnogap{L'œuvre du Créateur telle qu'on la voit dans la nature révèle Sa puissance. Mais la nature n'est pas au-dessus de Dieu, et Dieu n'est pas dans la nature comme certains le représentent. Dieu a créé le monde, mais le monde n'est pas Dieu ; il n'est que l'œuvre de Ses mains. \textbf{La nature révèle l'œuvre d'un Dieu positif et \underline{personnel}}, montrant que Dieu existe, et qu'Il récompense ceux qui Le cherchent avec diligence.}[Lt253-1903.16, 1903][https://egwwritings.org/read?panels=p9980.23]


\egwnogap{I could say much regarding the sanctuary; the ark containing the law of God; the cover of the ark, which is the mercy seat; the angels at either end of the ark; and other things connected with the heavenly sanctuary and with the great day of atonement. I could say much regarding the mysteries of heaven; but my lips are closed. I have no inclination to try to describe them.}[Lt253-1903.17; 1903][https://egwwritings.org/read?panels=p9980.25]


\egwnogap{Je pourrais dire beaucoup concernant le sanctuaire ; l'arche contenant la loi de Dieu ; le couvercle de l'arche, qui est le propitiatoire ; les anges à chaque extrémité de l'arche ; et d'autres choses liées au sanctuaire céleste et au grand jour de l'expiation. Je pourrais dire beaucoup concernant les mystères du ciel ; mais mes lèvres sont fermées. Je n'ai aucune envie d'essayer de les décrire.}[Lt253-1903.17; 1903][https://egwwritings.org/read?panels=p9980.25]


\egwnogap{\textbf{I would not dare to speak of God as you have spoken of Him}. He is high and lifted up, and His glory fills the heavens. “The voice of the Lord is mighty; it shaketh the cedars of Lebanon. \textbf{The Lord is in His holy temple}; let all the earth keep silence before Him.” [See Psalm 29:5; Habakkuk 2:20.]}[Lt253-1903.18; 1903][https://egwwritings.org/read?panels=p9980.26]


\egwnogap{\textbf{Je n'oserais pas parler de Dieu comme vous l'avez fait}. Il est haut et élevé, et Sa gloire remplit les cieux. “La voix de l'Éternel est puissante ; elle fait trembler les cèdres du Liban. \textbf{L'Éternel est dans son saint temple} ; que toute la terre fasse silence devant lui.” [Voir Psaume 29:5 ; Habacuc 2:20.]}[Lt253-1903.18; 1903][https://egwwritings.org/read?panels=p9980.26]


\egwnogap{\textbf{My brother, when you are tempted to speak of God, \underline{where He is, or what He is}, remember that on this point silence is eloquence}. Take off your shoes from off your feet; for the ground on which you are placing your careless, unsanctified feet is holy ground.}[Lt253-1903.19; 1903][https://egwwritings.org/read?panels=p14068.9980027]


\egwnogap{\textbf{Mon frère, lorsque vous êtes tenté de parler de Dieu, \underline{où Il est, ou ce qu'Il est}, rappelez-vous que sur ce point le silence est éloquence}. Ôtez vos souliers de vos pieds; car le sol sur lequel vous posez vos pieds insouciants et non sanctifiés est une terre sainte.}[Lt253-1903.19; 1903][https://egwwritings.org/read?panels=p14068.9980027]


\egwnogap{\textbf{I am instructed to say that there is nothing in the Word of God to substantiate your spiritualistic theories. Will you not renounce these theories at once? Upon them your mind has been dwelling for a long time, but they have had no sanctifying, refining, ennobling influence upon your life. The Lord has no use for these theories, and He would not have His people vindicate or propagate them.}}[Lt253-1903.20; 1903][https://egwwritings.org/read?panels=p9980.28]


\egwnogap{\textbf{Je suis chargée de dire qu'il n'y a rien dans la Parole de Dieu pour étayer vos théories spiritualistes. Ne renoncerez-vous pas à ces théories immédiatement? Votre esprit s'est attardé sur elles pendant longtemps, mais elles n'ont eu aucune influence sanctifiante, raffinante ou ennoblissante sur votre vie. Le Seigneur n'a aucune utilité pour ces théories, et Il ne voudrait pas que Son peuple les défende ou les propage.}}[Lt253-1903.20; 1903][https://egwwritings.org/read?panels=p9980.28]


\egwnogap{\textbf{The Father, the omniscient One, created the world \underline{through} Christ Jesus}. Christ is the light of the world, the way to eternal life. He, the anointed One, God gave to make an atonement for the sins of the world. You need to understand that unless you believe \textbf{in that atonement}, and know that you are bought with the price of the blood of \textbf{the only begotten Son of God}, you will assuredly be bound up with the wicked one. \textbf{If you continue to cherish the theories that you have been cherishing, you will be left to become the sport of Satan’s temptations}. He is playing the game of life for your soul. Remain for a little longer linked up with him, and be assured that you will lose your soul.}[Lt253-1903.21; 1903][https://egwwritings.org/read?panels=p9980.29]


\egwnogap{\textbf{Le Père, l'Omniscient, a créé le monde \underline{par} Christ Jésus}. Christ est la lumière du monde, le chemin vers la vie éternelle. Lui, l'Oint, Dieu l'a donné pour faire l'expiation des péchés du monde. Vous devez comprendre que si vous ne croyez pas \textbf{en cette expiation}, et ne savez pas que vous avez été racheté au prix du sang \textbf{du seul Fils engendré de Dieu}, vous serez assurément lié au malin. \textbf{Si vous continuez à chérir les théories que vous avez chéries, vous serez laissé pour devenir le jouet des tentations de Satan}. Il joue le jeu de la vie pour votre âme. Restez encore un peu lié à lui, et soyez assuré que vous perdrez votre âme.}[Lt253-1903.21; 1903][https://egwwritings.org/read?panels=p9980.29]


\egwnogap{By declaring that our institutions are undenominational, you have put our people and our work in a false position. You have been led over a terrible path, the dangers of which you have not known, but may sometime see. It is not yet too late for wrongs to be righted. There is hope for you. \textbf{You have followed the enemy step by step, striving to look into mysteries too high and holy for your comprehension}. \textbf{Then in your teaching the Holy One has been brought down to man’s \underline{scientific, spiritualistic ideas}}. You have been walking in crooked paths. You have lost the moral image of God. But there is hope for you. You may still turn your feet into the right path. Will you not now make straight paths for your feet, lest the lame be turned out of the way? Will you now refuse to sow one more seed of skepticism and sophistry in the minds of others? Will you now come to Christ and be healed?}[Lt253-1903.22; 1903][https://egwwritings.org/read?panels=p14068.9980030]


\egwnogap{En déclarant que nos institutions sont non confessionnelles, vous avez mis notre peuple et notre œuvre dans une position fausse. Vous avez été conduit sur un chemin terrible, dont vous n'avez pas connu les dangers, mais que vous pourriez un jour voir. Il n'est pas encore trop tard pour que les torts soient redressés. Il y a de l'espoir pour vous. \textbf{Vous avez suivi l'ennemi pas à pas, vous efforçant de regarder dans des mystères trop élevés et trop saints pour votre compréhension}. \textbf{Puis, dans votre enseignement, le Saint a été rabaissé aux \underline{idées scientifiques et spiritualistes} de l'homme}. Vous avez marché sur des chemins tortueux. Vous avez perdu l'image morale de Dieu. Mais il y a de l'espoir pour vous. Vous pouvez encore tourner vos pieds vers le bon chemin. Ne voulez-vous pas maintenant faire des sentiers droits pour vos pieds, de peur que le boiteux ne soit détourné du chemin? Ne refuserez-vous pas maintenant de semer une seule graine de plus de scepticisme et de sophisme dans l'esprit des autres? Ne viendrez-vous pas maintenant à Christ pour être guéri?}[Lt253-1903.22; 1903][https://egwwritings.org/read?panels=p14068.9980030]


\egwnogap{\textbf{I have hesitated and delayed about the sending out of that which the Spirit of the Lord has impelled me to write}. I did not want to be compelled to present the satanic influence of these sophistries. But unless there is a decided change, in yourself and your associates, I shall have to do this, to save others from following the path that you have been following. I shall have to obey the command given me of God, “\textbf{Meet it}.” This is the only thing that I can do.}[Lt253-1903.23; 1903][https://egwwritings.org/read?panels=p9980.31]


\egwnogap{\textbf{J'ai hésité et tardé à envoyer ce que l'Esprit du Seigneur m'a poussée à écrire}. Je ne voulais pas être obligée de présenter l'influence satanique de ces sophismes. Mais à moins qu'il n'y ait un changement décisif, en vous-même et chez vos associés, je devrai le faire, pour sauver d'autres de suivre le chemin que vous avez suivi. Je devrai obéir à l'ordre que Dieu m'a donné, “\textbf{Affrontez-le}.” C'est la seule chose que je puisse faire.}[Lt253-1903.23; 1903][https://egwwritings.org/read?panels=p9980.31]


\egwnogap{I present to you the things that the Lord has presented to me. There is a great work to be done. We are to take hold of the work understandingly, praying, believing, and receiving the Holy Spirit. Thus only can we do the work given us. \textbf{I am required by God to bear testimony against Living Temple}. Whatever your associates may say concerning this book,\textbf{ I take the position now and forever that it is a snare}. \textbf{No union will be formed by our people as a whole upon the \underline{theories} that you have begun to present in that book}. \textbf{You may regard this as forever decided}. \textbf{As a people we shall stand firm \underline{on the platform that has withstood test and trial}. We shall hold to the \underline{sure pillars of our faith}. \underline{The principles of truth} that God has revealed to us are our only foundation. They have made us what we are. These new, fanciful theories are fascinating and misleading. They endanger the eternal interests of the soul. The Scriptures do not sustain them}. Clothed with the Christian armor, shod with the preparation of the gospel of peace, we shall stand \textbf{firm against these misleading theories}. You may turn and wrest the Word of God to your own destruction, but I entreat you not to do this.}[Lt253-1903.24; 1903][https://egwwritings.org/read?panels=p9980.32]


\egwnogap{Je vous présente les choses que le Seigneur m'a présentées. Il y a une grande œuvre à faire. Nous devons entreprendre l'œuvre avec intelligence, priant, croyant et recevant le Saint-Esprit. C'est seulement ainsi que nous pouvons accomplir l'œuvre qui nous est confiée. \textbf{Je suis obligée par Dieu de témoigner contre Le Temple Vivant}. Quoi que vos associés puissent dire concernant ce livre, \textbf{je prends position maintenant et pour toujours qu'il est un piège}. \textbf{Aucune union ne sera formée par notre peuple dans son ensemble sur les \underline{théories} que vous avez commencé à présenter dans ce livre}. \textbf{Vous pouvez considérer cela comme définitivement décidé}. \textbf{En tant que peuple, nous resterons fermes \underline{sur la plateforme qui a résisté à l'épreuve et aux tests}. Nous nous tiendrons aux \underline{piliers sûrs de notre foi}. \underline{Les principes de vérité} que Dieu nous a révélés sont notre seul fondement. Ils ont fait de nous ce que nous sommes. Ces nouvelles théories fantaisistes sont fascinantes et trompeuses. Elles mettent en danger les intérêts éternels de l'âme. Les Écritures ne les soutiennent pas}. Revêtus de l'armure chrétienne, chaussés de la préparation de l'évangile de paix, nous tiendrons \textbf{ferme contre ces théories trompeuses}. Vous pouvez vous détourner et tordre la Parole de Dieu pour votre propre destruction, mais je vous supplie de ne pas le faire.}[Lt253-1903.24; 1903][https://egwwritings.org/read?panels=p9980.32]


\egwnogap{\textbf{Heaven is not a vapor. It is a place}. \textbf{Christ has gone to prepare mansions for those who love Him}, those who, in obedience to His commands, come out from the world and are separate. The principles of heaven must be brought into our experience, that we may be distinguished from the world. \textbf{There must be a marked contrast between us and the world; for we are God’s denominated people}.}[Lt253-1903.25; 1903][https://egwwritings.org/read?panels=p9980.33]


\egwnogap{\textbf{Le ciel n'est pas une vapeur. C'est un lieu}. \textbf{Christ est allé préparer des demeures pour ceux qui L'aiment}, ceux qui, en obéissance à Ses commandements, sortent du monde et sont séparés. Les principes du ciel doivent être introduits dans notre expérience, afin que nous puissions être distingués du monde. \textbf{Il doit y avoir un contraste marqué entre nous et le monde; car nous sommes le peuple désigné de Dieu}.}[Lt253-1903.25; 1903][https://egwwritings.org/read?panels=p9980.33]


\egwnogap{The Lord has given you an opportunity to make things right. \textbf{I rejoice that you have made a beginning. Do not think that we have no right to try to correct your errors and the results of these errors. As long as God gives me breath, and commissions me to use pen and voice in beating back this evil thing that has come in among us, I shall act my part in the warfare. Ever since I was seventeen years old, I have had to fight this battle against false theories, in defense of the truth}. \textbf{The history of our past experience is indelibly fixed in my mind, and I am determined that \underline{no theories of the order that you have been accepting} shall come into our ranks}. If you refuse to change, and labor to lead your associates after you, and they venture to follow your leading, the accountability rests with you and with them, not on my soul.}[Lt253-1903.26, 1903][https://egwwritings.org/read?panels=p9980.34]


\egwnogap{Le Seigneur vous a donné une opportunité de rectifier les choses. \textbf{Je me réjouis que vous ayez fait un début. Ne pensez pas que nous n'avons pas le droit d'essayer de corriger vos erreurs et les résultats de ces erreurs. Tant que Dieu me donnera le souffle, et me chargera d'utiliser la plume et la voix pour repousser cette chose maléfique qui s'est introduite parmi nous, je jouerai mon rôle dans cette guerre. Depuis que j'ai dix-sept ans, j'ai dû mener cette bataille contre les fausses théories, pour défendre la vérité}. \textbf{L'histoire de notre expérience passée est indélébilement fixée dans mon esprit, et je suis déterminée à ce qu’\underline{aucune théorie de l'ordre de celles que vous avez acceptées} n'entre dans nos rangs}. Si vous refusez de changer, et travaillez à conduire vos associés après vous, et qu'ils s'aventurent à suivre votre direction, la responsabilité repose sur vous et sur eux, pas sur mon âme.}[Lt253-1903.26, 1903][https://egwwritings.org/read?panels=p9980.34]


\egwnogap{\textbf{I speak decidedly, in order that you may know, that unless there is a decided change in you, there can be no hope of a union between you and those who are holding the beginning of their confidence firm unto the end.} You have made the division. \textbf{\underline{We must stand firm for the truths that the Lord has given us as the pillars of our faith}}.}[Lt253-1903.27; 1903][https://egwwritings.org/read?panels=p9980.35]


\egwnogap{\textbf{Je parle de façon décisive, afin que vous sachiez que, à moins qu'il n'y ait un changement décisif en vous, il ne peut y avoir aucun espoir d'union entre vous et ceux qui gardent fermement le commencement de leur confiance jusqu'à la fin.} Vous avez créé la division. \textbf{\underline{Nous devons rester fermes pour les vérités que le Seigneur nous a données comme les piliers de notre foi}}.}[Lt253-1903.27; 1903][https://egwwritings.org/read?panels=p9980.35]


\egwnogap{I entreat you to turn to the Lord with full purpose of heart, before it is forever too late. Separate yourself from the influences which have separated you from your brethren who are engaged in the gospel ministry and from the people whom God is leading. \textbf{\underline{Patchwork theories} cannot be accepted by those who are loyal to the faith and to \underline{the principles} that have withstood all the opposition of satanic influences}.}[Lt253-1903.28; 1903][https://egwwritings.org/read?panels=p9980.36]


\egwnogap{Je vous supplie de vous tourner vers le Seigneur avec une pleine détermination de cœur, avant qu'il ne soit à jamais trop tard. Séparez-vous des influences qui vous ont séparé de vos frères qui sont engagés dans le ministère de l'évangile et du peuple que Dieu conduit. \textbf{Les \underline{théories de rapiècement} ne peuvent pas être acceptées par ceux qui sont loyaux à la foi et aux \underline{principes} qui ont résisté à toute l'opposition des influences sataniques}.}[Lt253-1903.28; 1903][https://egwwritings.org/read?panels=p9980.36]


\egwnogap{If you will empty yourself of all that has separated you from Christ, and receive the Saviour into your heart, you will be transformed in character. Lay off responsibilities for a time, and go away somewhere with a few of your brethren, and with them search the Scriptures. Humble your heart before the Lord, and make thorough work for repentance. \textbf{The religion of Christ is the spiritual leaven that is to be introduced into the heart. This changes the life and character}. This religion is a heavenly principle, seen in the Christian’s life and conversation. It is revealed in Christian purity. The love of Christ is seen in the tenderness and grace of sanctified humanity. It is by the Word made flesh that we are saved. Our redemption was wrought out, \textbf{not by the Son of God’s remaining in heaven, but by the Son of God’s becoming incarnate—taking humanity upon Him and coming to this world}. Thus eternal life was brought to us. That which authority, commands, and promises could not do, God did by coming to this world in the likeness of sinful flesh.}[Lt253-1903.29; 1903][https://egwwritings.org/read?panels=p9980.37]


\egwnogap{Si vous vous videz de tout ce qui vous a séparé de Christ, et que vous recevez le Sauveur dans votre cœur, vous serez transformé dans votre caractère. Déchargez-vous des responsabilités pour un temps, et allez quelque part avec quelques-uns de vos frères, et avec eux sondez les Écritures. Humiliez votre cœur devant le Seigneur, et faites un travail complet de repentance. \textbf{La religion de Christ est le levain spirituel qui doit être introduit dans le cœur. Cela change la vie et le caractère}. Cette religion est un principe céleste, visible dans la vie et la conversation du chrétien. Elle se révèle dans la pureté chrétienne. L'amour de Christ se voit dans la tendresse et la grâce d'une humanité sanctifiée. C'est par la Parole faite chair que nous sommes sauvés. Notre rédemption a été accomplie, \textbf{non pas par le Fils de Dieu restant au ciel, mais par le Fils de Dieu devenant incarné—prenant l'humanité sur Lui et venant dans ce monde}. Ainsi la vie éternelle nous a été apportée. Ce que l'autorité, les commandements et les promesses ne pouvaient pas faire, Dieu l'a fait en venant dans ce monde dans la ressemblance de la chair pécheresse.}[Lt253-1903.29; 1903][https://egwwritings.org/read?panels=p9980.37]


\egwnogap{Christ came to the earth to live as a man among men, not to be spoiled by human frailty, but to place in the minds of men principles of truth that could never be obliterated, because they are eternally true. He came to bring a new life to fallen human beings—an excellence that could not be stained or deteriorated by sin.}[Lt253-1903.30; 1903][https://egwwritings.org/read?panels=p9980.38]


\egwnogap{Christ est venu sur la terre pour vivre comme un homme parmi les hommes, non pour être gâté par la fragilité humaine, mais pour placer dans l'esprit des hommes des principes de vérité qui ne pourraient jamais être effacés, car ils sont éternellement vrais. Il est venu pour apporter une nouvelle vie aux êtres humains déchus—une excellence qui ne pouvait être ni souillée ni détériorée par le péché.}[Lt253-1903.30; 1903][https://egwwritings.org/read?panels=p9980.38]


\egwnogap{\textbf{My brother, I must tell you that you have little realization of whither your feet have been tending}. You have been binding yourself up with those who belong to the army of the great apostate. \textbf{Your mind has been as dark as Egypt}. \textbf{If you will fall on the Rock and be broken}, Christ will accept you. But you have been standing on the enemy’s ground, doing his work. \textbf{The religious world is fast going over the same road that you have been following. If you continue to follow this road, you will have plenty of company. But what will the end be?}}[Lt253-1903.31; 1903][https://egwwritings.org/read?panels=p14068.9980039]


\egwnogap{\textbf{Mon frère, je dois vous dire que vous avez peu conscience de la direction vers laquelle vos pas se sont dirigés}. Vous vous êtes lié avec ceux qui appartiennent à l'armée du grand apostat. \textbf{Votre esprit a été aussi sombre que l'Égypte}. \textbf{Si vous tombez sur le Rocher et que vous êtes brisé}, Christ vous acceptera. Mais vous vous êtes tenu sur le terrain de l'ennemi, faisant son œuvre. \textbf{Le monde religieux emprunte rapidement le même chemin que vous avez suivi. Si vous continuez à suivre ce chemin, vous aurez beaucoup de compagnie. Mais quelle en sera la fin?}}[Lt253-1903.31; 1903][https://egwwritings.org/read?panels=p14068.9980039]


\egwnogap{So long have you been walking in darkness, so long have you followed your own way, that you may be strongly tempted to resist this appeal that I make. If it were not that your \textbf{eternal interests are involved}, I would not speak to you on this subject. It would seem that I have written enough, that there is no need of my urging this subject upon you further. \textbf{But I tell you in truth that I clearly understand what I am doing}. Sufficient light has been given you. But for several years you have not heeded this light. If you had wished to know what the Lord has said, you could have known; \textbf{for you have the books that have been written under the guidance of His Spirit}. You have had all the directions that could be asked for to point out the right way. Direct light has been sent you. But you have looked upon this as of less importance than your own plans and devisings. If you had heeded the testimonies sent you, Living Temple would never have been written.}[Lt253-1903.32; 1903][https://egwwritings.org/read?panels=p9980.40]


\egwnogap{Vous avez marché si longtemps dans les ténèbres, si longtemps suivi votre propre voie, que vous pourriez être fortement tenté de résister à cet appel que je vous lance. Si vos \textbf{intérêts éternels n'étaient pas en jeu}, je ne vous parlerais pas de ce sujet. Il semblerait que j'ai écrit suffisamment, qu'il n'est pas nécessaire que j'insiste davantage sur ce sujet. \textbf{Mais je vous dis en vérité que je comprends clairement ce que je fais}. Une lumière suffisante vous a été donnée. Mais depuis plusieurs années, vous n'avez pas tenu compte de cette lumière. Si vous aviez souhaité connaître ce que le Seigneur a dit, vous auriez pu le savoir; \textbf{car vous avez les livres qui ont été écrits sous la direction de Son Esprit}. Vous avez reçu toutes les directives que l'on pourrait demander pour indiquer le bon chemin. Une lumière directe vous a été envoyée. Mais vous avez considéré cela comme étant de moindre importance que vos propres plans et projets. Si vous aviez tenu compte des témoignages qui vous ont été envoyés, Le Temple Vivant (le livre) n'aurait jamais été écrit.}[Lt253-1903.32; 1903][https://egwwritings.org/read?panels=p9980.40]


\egwnogap{Will you not make a thorough, determined, Christlike effort to break the spell that Satan has cast over you? He has had great power over your mind and has swayed you in wrong lines. He thinks that he can hold you now. Will you not defeat and disappoint him?}[Lt253-1903.33; 1903][https://egwwritings.org/read?panels=p9980.41]


\egwnogap{Ne ferez-vous pas un effort approfondi, déterminé et semblable à celui de Christ pour briser le sort que Satan a jeté sur vous? Il a eu un grand pouvoir sur votre esprit et vous a entraîné dans de mauvaises directions. Il pense qu'il peut vous retenir maintenant. Ne voulez-vous pas le décevoir et le déjouer?}[Lt253-1903.33; 1903][https://egwwritings.org/read?panels=p9980.41]


\egwnogap{I write to you as I would to a son. Break away from the enemy—the accuser of the brethren. Say to him, “Get thee behind me Satan. I have committed a grievous sin in heeding your suggestions. I will no longer listen to them.” I beg of you, for your soul’s sake, to resist the tempter, that he may flee from you. Draw near to God, and He will draw near to you. \textbf{You will lose heaven unless you fall on the Rock and are broken}.}[Lt253-1903.34; 1903][https://egwwritings.org/read?panels=p9980.42]


\egwnogap{Je vous écris comme je le ferais à un fils. Éloignez-vous de l'ennemi—l'accusateur des frères. Dites-lui: “Arrière de moi, Satan. J'ai commis un péché grave en écoutant tes suggestions. Je n'y prêterai plus l'oreille.” Je vous supplie, pour le bien de votre âme, de résister au tentateur, afin qu'il fuie loin de vous. Approchez-vous de Dieu, et Il s'approchera de vous. \textbf{Vous perdrez le ciel à moins de tomber sur le Rocher et d'être brisé}.}[Lt253-1903.34; 1903][https://egwwritings.org/read?panels=p9980.42]


Many things in this letter to Dr. Kellogg go without being said, yet are explained when the context is understood. Ellen White read the letter from Brother Daniells expressing how Dr. Kellogg wanted to revise the Living Temple because he\others{had been thinking the matter over, and began to see that he had made a slight mistake in \textbf{expressing }his views}, and\others{that within a short time \textbf{he had come to believe in the trinity} and could now see pretty clearly where all the difficulty was, and believed that he could clear the matter up satisfactorily}. Kellogg confessed,\others{that he now believed \textbf{in God the Father, God the Son, and God the Holy Ghost}}. In answer to that, Sister White personally wrote to him:\egwinline{The book Living Temple \textbf{is not to be patched up}, a few changes made in it, and then advertised and praised as a valuable production}. How did Kellogg want to patch up his book? According to A. G. Daniells’ testimony, he thought to change a few expressions by explicitly stating his trinitarian sentiment. But the expression of the views was not the real problem—it was the views themselves. Sister White did not spare rebuking him for his views of God, which were \textit{trinitarian} views. She told him that she is\egwinline{\textbf{determined that \underline{no theories of the order that you have been accepting} shall come into our ranks}}. This is a very strong statement. Could it be that, since Kellogg confessed that he was accepting the Trinity doctrine, Sister White was also including it in her statement? It seems unthinkable because this doctrine is in our ranks today. But her statement actually pinpoints the Trinity when she said:\egwinline{\textbf{Patchwork theories} cannot be accepted by those who are loyal \textbf{to the faith and to the principles} that have withstood all the opposition of satanic influences}. Kellogg wanted to patch up “\textit{Living Temple}” by explicitly mentioning the Trinity doctrine. Why was Sister White determined to keep this doctrine out of our ranks, yet it is in our ranks today? It is fair to point out that the Trinity was not part of Seventh-day Adventist faith in her time and it came into our ranks later. Today, many argue that it was because of her works that the Trinity is a part of our beliefs, but Ellen White’s reaction, and her answer to Kellogg’s belief in it, showcases how she dealt with such doctrine. What can we learn from that?


Beaucoup de choses dans cette lettre au Dr Kellogg ne sont pas dites explicitement, mais s'expliquent lorsque le contexte est compris. Ellen White a lu la lettre du frère Daniells exprimant comment le Dr Kellogg voulait réviser le Temple Vivant parce qu'il\others{avait réfléchi à la question et commençait à voir qu'il avait fait une légère erreur en \textbf{exprimant} ses opinions}, et\others{qu'en peu de temps \textbf{il en était venu à croire en la trinité} et pouvait maintenant voir assez clairement où était toute la difficulté, et croyait qu'il pourrait clarifier la question de manière satisfaisante}. Kellogg a confessé,\others{qu'il croyait maintenant \textbf{en Dieu le Père, Dieu le Fils et Dieu le Saint-Esprit}}. En réponse à cela, Sœur White lui a personnellement écrit:\egwinline{Le livre Le Temple Vivant \textbf{ne doit pas être rapiécé}, quelques changements y être apportés, puis être annoncé et loué comme une production de valeur}. Comment Kellogg voulait-il rapiécer son livre? Selon le témoignage d'A. G. Daniells, il pensait changer quelques expressions en déclarant explicitement son sentiment trinitaire. Mais l'expression des opinions n'était pas le vrai problème—c'était les opinions elles-mêmes. Sœur White n'a pas épargné de le réprimander pour ses conceptions de Dieu, qui étaient des conceptions \textit{trinitaires}. Elle lui a dit qu'elle est\egwinline{\textbf{déterminée à ce qu’\underline{aucune théorie de l'ordre de celles que vous avez acceptées} n'entre dans nos rangs}}. C'est une déclaration très forte. Se pourrait-il que, puisque Kellogg a confessé qu'il acceptait la doctrine de la Trinité, Sœur White l'incluait également dans sa déclaration? Cela semble impensable parce que cette doctrine est dans nos rangs aujourd'hui. Mais sa déclaration désigne en fait la Trinité quand elle a dit:\egwinline{\textbf{Les théories de rapiècement} ne peuvent pas être acceptées par ceux qui sont fidèles \textbf{à la foi et aux principes} qui ont résisté à toute l'opposition des influences sataniques}. Kellogg voulait rapiécer “\textit{Le Temple Vivant}” en mentionnant explicitement la doctrine de la Trinité. Pourquoi Sœur White était-elle déterminée à garder cette doctrine hors de nos rangs, alors qu'elle est dans nos rangs aujourd'hui? Il est juste de souligner que la Trinité ne faisait pas partie de la foi adventiste du septième jour à son époque et qu'elle est entrée dans nos rangs plus tard. Aujourd'hui, beaucoup soutiennent que c'est grâce à ses œuvres que la Trinité fait partie de nos croyances, mais la réaction d'Ellen White et sa réponse à la croyance de Kellogg en celle-ci montrent comment elle a traité cette doctrine. Que pouvons-nous en apprendre?


Taken in its context, this letter sheds new light on Kellogg’s controversy and demonstrates how we should deal with the Trinity doctrine. The first thing we question is why Sister White never used the word “Trinity” in her writings, even when she was directly dealing with this doctrine? Elsewhere, she answers:


Prise dans son contexte, cette lettre jette une nouvelle lumière sur la controverse de Kellogg et démontre comment nous devrions traiter la doctrine de la Trinité. La première chose que nous questionnons est pourquoi Sœur White n'a jamais utilisé le mot “Trinité” dans ses écrits, même lorsqu'elle traitait directement de cette doctrine? Ailleurs, elle répond:


\egw{I was cautioned not to enter into controversy \textbf{regarding the question} that \textbf{\underline{will come up}} over \textbf{these things, because controversy \underline{might lead men to resort to subterfuges, and their minds would be led away from the truth of the Word of God to assumption and guesswork}}. \textbf{The more that fanciful theories are discussed, the \underline{less men will know of God and of the truth that sanctifies the soul}}.}[Lt232-1903.41; 1903][https://egwwritings.org/read?panels=p14068.10197050]


\egw{J'ai été mise en garde de ne pas entrer dans la controverse \textbf{concernant la question} qui \textbf{\underline{se posera}} au sujet de \textbf{ces choses, car la controverse \underline{pourrait amener les hommes à recourir à des subterfuges, et leurs esprits seraient détournés de la vérité de la Parole de Dieu vers des suppositions et des conjectures}}. \textbf{Plus les théories fantaisistes sont discutées, \underline{moins les hommes connaîtront Dieu et la vérité qui sanctifie l'âme}}.}[Lt232-1903.41; 1903][https://egwwritings.org/read?panels=p14068.10197050]


This is a very important lesson and principle that Sister White is teaching us here. When the controversy over Kellogg’s theories arose, she did not venture into the theories themselves, because this would lead the minds of men away from the truth of the Word of God to assumption and guesswork. Rather, she led the minds of men into the truth, which sanctifies the soul. She led by example, evident here in her letter to Dr. Kellogg. This truth that she led the minds of men to, was the truth on the \emcap{personality of God}. She rebuked Kellogg for his theories but, very importantly, we properly identify these theories by their context and her implicit expression of them.


C'est une leçon et un principe très importants que Sœur White nous enseigne ici. Lorsque la controverse sur les théories de Kellogg a surgi, elle ne s'est pas aventurée dans les théories elles-mêmes, car cela aurait détourné l'esprit des hommes de la vérité de la Parole de Dieu vers des suppositions et des conjectures. Au contraire, elle a dirigé l'esprit des hommes vers la vérité, qui sanctifie l'âme. Elle a dirigé par l'exemple, ce qui est évident ici dans sa lettre au Dr Kellogg. Cette vérité vers laquelle elle a dirigé l'esprit des hommes était la vérité sur la \emcap{personnalité de Dieu}. Elle a réprimandé Kellogg pour ses théories mais, très important, nous identifions correctement ces théories par leur contexte et son expression implicite de celles-ci.


We see that she made a contrast between the Trinity and the \emcap{personality of God}. She made a contrast between the old principles of our faith and the new theories. First, she drew our minds back to the beginning of our spiritual heritage,\egwinline{after the passing of the time in 1844}, when her husband James White, Joseph Bates, Father Pierce, Elder Edson, and many others who were keen, noble, and true, searched for truth. She pointed back to the wonderful and mighty experiences of how the leading points of our faith, held in 1903, were firmly established. \egwinline{\textbf{Thus \underline{the leading points of our faith}} as we hold them today were firmly established.} \egwinline{\textbf{\underline{Point after point} was clearly defined, and all the brethren came into harmony}.} \egwinline{\textbf{The whole company of believers were united in the truth}}. Obviously, from the context of chapter 10 of the Special Testimonies, we know that these experiences explain \egwinline{\textbf{how firmly the foundation of our faith has been laid}}[SpTB02 56.4; 1904][https://egwwritings.org/read?panels=p417.288]. This foundation is expressed in the \emcap{Fundamental Principles}\footnote{\href{https://static1.squarespace.com/static/554c4998e4b04e89ea0c4073/t/59d17e24c027d84167e17617/1506901547915/SDA-YB1905+\%28P.+188-192\%29.pdf}{Yearbook Of Seventh-day Adventist denomination 1905, p. 188-192}}. This foundation is the truth which,\egwinline{\textbf{\underline{point by point}}, \textbf{has been sought out by prayerful study, and testified to by the miracle-working power of the Lord}}. God \egwinline{\textbf{calls upon us to \underline{hold firmly}, with the grip of faith, to \underline{the fundamental principles} that are \underline{based upon unquestionable authority}}.}[SpTB02 59.1; 1904][https://egwwritings.org/read?panels=p417.299] In light of these experiences and the truth expressed in the \emcap{fundamental principles}, \egwinline{\textbf{\underline{Patchwork theories} cannot be accepted by those who are loyal \underline{to the faith} and \underline{to the principles} that have withstood all the opposition of satanic influences}}[Lt253-1903.28; 1903][https://egwwritings.org/read?panels=p14068.9980036]. From the historical record of these brethren who were keen, noble and true, we have evidence that they, too, have contrasted the Trinity doctrine with the truth on the \emcap{personality of God}. James White, in the Review and Herald article, listed \others{some of the popular fables of the age}, saying: \others{Here we might mention \textbf{the Trinity, which \underline{does away the personality of God, and of his Son Jesus Christ}}}[James White, Review \& Herald, December 11, 1855, p. 85.15][http://documents.adventistarchives.org/Periodicals/RH/RH18551211-V07-11.pdf]. J. N. Andrews said, \others{\textbf{The doctrine of the Trinity which was established in the church by the council of Nicea, A. D. 325}. \textbf{This doctrine \underline{destroys the personality of God, and his Son Jesus Christ our Lord}}...}[J. N. Andrews, Review \& Herald, March 6, 1855, p. 185][http://documents.adventistarchives.org/Periodicals/RH/RH18550306-V06-24.pdf] J. B. Frisbie, in his article “\textit{Seventh-day Sabbath not abolished}”, compares the Sabbath God to the Sunday god; he describes the Sabbath God in light of the \emcap{personality of God} expressed in the first point of the \emcap{Fundamental Principles}. The Sunday god is described by the \others{unity of this God-head, there are three persons of one substance, power and eternity; the Father, the Son, and the Holy Ghost}[J. B. Frisbie, Review \& Herald March 7, 1854. p. 50][http://documents.adventistarchives.org/Periodicals/RH/RH18540307-V05-07.pdf]. He explained how the doctrine on the \emcap{personality of God} stands in conflict with the doctrine of Trinity, in the same way the Holy Sabbath stands in conflict with pagan Sunday worship. Also, brother J. N. Loughborough wrote the objections to the Trinity doctrine in the Adventist Review and Sabbath Herald\footnote{\href{https://adventistdigitallibrary.org/adl-349160/advent-review-and-sabbath-herald-november-5-1861}{J. N. Loughborough, November 5, 1861, Review \& Herald, vol. 18, p. 184, par. 1-11}}. In the other publication of the Review and Herald, he published the article “\textit{Is God a person?}”, explaining the position of Seventh-day Adventist belief on the \emcap{personality of God}, expressed in the first point of the \emcap{Fundamental Principles}\footnote{\href{http://documents.adventistarchives.org/Periodicals/RH/RH18550918-V07-06.pdf}{J. N. Loughborough, September 18. 1855, Review \& Herald, vol. 7, p. 6.}}. James White was also explaining the same position in his multiple print pamphlet, “\textit{The Personality of God}”\footnote{\href{https://egwwritings.org/?ref=en_PERGO.1.1&para=1471.3}{J. White, The Personality of God, June 18. 1861.}}. These are just a few examples where the Adventist pioneers explained the position on the \emcap{personality of God} expressed by the first point of the \emcap{fundamental principles}.


Nous voyons qu'elle a fait un contraste entre la Trinité et la \emcap{personnalité de Dieu}. Elle a fait un contraste entre les anciens principes de notre foi et les nouvelles théories. D'abord, elle a ramené nos esprits au début de notre héritage spirituel,\egwinline{après le passage du temps en 1844}, quand son mari James White, Joseph Bates, Père Pierce, Ancien Edson, et beaucoup d'autres qui étaient perspicaces, nobles et fidèles, ont cherché la vérité. Elle a rappelé les expériences merveilleuses et puissantes de la façon dont les points principaux de notre foi, tenus en 1903, ont été fermement établis. \egwinline{\textbf{Ainsi \underline{les points principaux de notre foi}} tels que nous les tenons aujourd'hui ont été fermement établis.} \egwinline{\textbf{\underline{Point après point} a été clairement défini, et tous les frères sont parvenus à l'harmonie}.} \egwinline{\textbf{Toute la compagnie des croyants était unie dans la vérité}}. Évidemment, d'après le contexte du chapitre 10 des Témoignages Spéciaux, nous savons que ces expériences expliquent \egwinline{\textbf{comment le fondement de notre foi a été solidement posé}}[SpTB02 56.4; 1904][https://egwwritings.org/read?panels=p417.288]. Ce fondement est exprimé dans les \emcap{Principes Fondamentaux}\footnote{\href{https://static1.squarespace.com/static/554c4998e4b04e89ea0c4073/t/59d17e24c027d84167e17617/1506901547915/SDA-YB1905+\%28P.+188-192\%29.pdf}{Annuaire de la dénomination Adventiste du Septième Jour 1905, p. 188-192}}. Ce fondement est la vérité qui,\egwinline{\textbf{\underline{point par point}}, \textbf{a été recherchée par une étude priante, et attestée par la puissance miraculeuse du Seigneur}}. Dieu \egwinline{\textbf{nous appelle à \underline{tenir fermement}, avec la poigne de la foi, aux \underline{principes fondamentaux} qui sont \underline{basés sur une autorité incontestable}}.}[SpTB02 59.1; 1904][https://egwwritings.org/read?panels=p417.299] À la lumière de ces expériences et de la vérité exprimée dans les \emcap{principes fondamentaux}, \egwinline{\textbf{les \underline{théories de rapiècement} ne peuvent pas être acceptées par ceux qui sont loyaux \underline{à la foi} et \underline{aux principes} qui ont résisté à toute l'opposition des influences sataniques}}[Lt253-1903.28; 1903][https://egwwritings.org/read?panels=p14068.9980036]. D'après le témoignage historique de ces frères qui étaient perspicaces, nobles et fidèles, nous avons la preuve qu'ils ont, eux aussi, opposé la doctrine de la Trinité à la vérité sur la \emcap{personnalité de Dieu}. James White, dans l'article de la Review and Herald, a énuméré \others{certaines des fables populaires de l'époque}, en disant: \others{Ici, nous pourrions mentionner \textbf{la Trinité, qui \underline{supprime la personnalité de Dieu, et de son Fils Jésus-Christ}}}[James White, Review \& Herald, 11 décembre 1855, p. 85.15][http://documents.adventistarchives.org/Periodicals/RH/RH18551211-V07-11.pdf]. J. N. Andrews a dit, \others{\textbf{La doctrine de la Trinité qui a été établie dans l'église par le concile de Nicée, en 325 après J.-C.}. \textbf{Cette doctrine \underline{détruit la personnalité de Dieu, et de son Fils Jésus-Christ notre Seigneur}}...}[J. N. Andrews, Review \& Herald, 6 mars 1855, p. 185][http://documents.adventistarchives.org/Periodicals/RH/RH18550306-V06-24.pdf] J. B. Frisbie, dans son article “\textit{Le sabbat du septième jour n'est pas aboli}”, compare le Dieu du sabbat au dieu du dimanche; il décrit le Dieu du sabbat à la lumière de la \emcap{personnalité de Dieu} exprimée dans le premier point des \emcap{Principes Fondamentaux}. Le dieu du dimanche est décrit par \others{l'unité de cette divinité, il y a trois personnes d'une seule substance, puissance et éternité; le Père, le Fils et le Saint-Esprit}[J. B. Frisbie, Review \& Herald 7 mars 1854. p. 50][http://documents.adventistarchives.org/Periodicals/RH/RH18540307-V05-07.pdf]. Il a expliqué comment la doctrine sur la \emcap{personnalité de Dieu} est en conflit avec la doctrine de la Trinité, de la même manière que le Saint Sabbat est en conflit avec le culte païen du dimanche. De même, le frère J. N. Loughborough a écrit les objections à la doctrine de la Trinité dans l'Adventist Review and Sabbath Herald\footnote{\href{https://adventistdigitallibrary.org/adl-349160/advent-review-and-sabbath-herald-november-5-1861}{J. N. Loughborough, 5 novembre 1861, Review \& Herald, vol. 18, p. 184, par. 1-11}}. Dans une autre publication de la Review and Herald, il a publié l'article “\textit{Dieu est-il une personne?}”, expliquant la position de la croyance Adventiste du Septième Jour sur la \emcap{personnalité de Dieu}, exprimée dans le premier point des \emcap{Principes Fondamentaux}\footnote{\href{http://documents.adventistarchives.org/Periodicals/RH/RH18550918-V07-06.pdf}{J. N. Loughborough, 18 septembre 1855, Review \& Herald, vol. 7, p. 6.}}. James White expliquait également la même position dans sa brochure imprimée multiple, “\textit{La Personnalité de Dieu}”\footnote{\href{https://egwwritings.org/?ref=en_PERGO.1.1&para=1471.3}{J. White, La Personnalité de Dieu, 18 juin 1861.}}. Ce ne sont là que quelques exemples où les pionniers adventistes ont expliqué la position sur la \emcap{personnalité de Dieu} exprimée par le premier point des \emcap{principes fondamentaux}.


Sister White rebuked Kellogg:\egwinline{\textbf{But I tell you in truth that I clearly understand what I am doing}. \textbf{Sufficient light has been given you}. But for several years you have not heeded this light. If you had wished to know what the Lord has said, you could have known; \textbf{for \underline{you have the books} that have been written under the guidance of His Spirit}. You have had all the directions that could be asked for to point out the right way. Direct light has been sent you. But you have looked upon this as of less importance than your own plans and devisings. If you had heeded the testimonies sent you, Living Temple would never have been written.}[Lt253-1903.32; 1903][https://egwwritings.org/read?panels=p14068.9980040] The core issue of Dr. Kellogg’s controversy was \egwinline{the personality of God and where His presence is}[SpTB02 51.3; 1904][https://egwwritings.org/read?panels=p417.262]. Dr. Kellogg had access to the pioneer writings, books and the church's \emcap{Fundamental Principles} that were testified to by the miracle working power of the Holy Spirit.


Sœur White a réprimandé Kellogg:\egwinline{\textbf{Mais je vous dis en vérité que je comprends clairement ce que je fais}. \textbf{Une lumière suffisante vous a été donnée}. Mais depuis plusieurs années, vous n'avez pas tenu compte de cette lumière. Si vous aviez souhaité savoir ce que le Seigneur a dit, vous auriez pu le savoir; \textbf{car \underline{vous avez les livres} qui ont été écrits sous la direction de Son Esprit}. Vous avez reçu toutes les directives que l'on pourrait demander pour indiquer le bon chemin. Une lumière directe vous a été envoyée. Mais vous avez considéré cela comme moins important que vos propres plans et projets. Si vous aviez tenu compte des témoignages qui vous ont été envoyés, Le Temple Vivant n'aurait jamais été écrit.}[Lt253-1903.32; 1903][https://egwwritings.org/read?panels=p14068.9980040] Le problème central de la controverse du Dr Kellogg était \egwinline{la personnalité de Dieu et où est Sa présence}[SpTB02 51.3; 1904][https://egwwritings.org/read?panels=p417.262]. Le Dr Kellogg avait accès aux écrits des pionniers, aux livres et aux \emcap{Principes Fondamentaux} de l'église qui ont été attestés par la puissance miraculeuse du Saint-Esprit.


Sister White recalled the experiences of how the \textit{leading points of our faith}, as were held in former times, were firmly established.\egwinline{\textbf{\underline{Point after point} was clearly defined, and all the brethren came into harmony}}[Lt253-1903.4; 1903][https://egwwritings.org/read?panels=p14068.9980010]. These leading points were the \emcap{Fundamental Principles}, of which the \emcap{personality of God} was one. This point, and Sister White’s testimony of it, remained the same during the course of her life.  She said\egwinline{\textbf{\underline{I have ever had the same testimony to bear which I now bear regarding the personality of God}}}[Lt253-1903.9; 1903][https://egwwritings.org/read?panels=p14068.9980015]. From Early Writings, she then quoted her visions of the Heavenly reality. She recalled how she had had the privilege to be in the presence of God, how God, encircled by the light of His glory, passed by her side. She did not see God from the light He was encircled by; she was afraid of Him, thinking that if He were to approach her she\egwinline{would be struck out of existence}. Then she saw\egwinline{\textbf{the lovely Jesus, that He is a person}. \textbf{I asked Him if His Father was a person, and had \underline{a form like} Himself}. Said Jesus, ‘\textbf{I am the express image of My Father’s person!}’}[Lt253-1903.12; 1903][https://egwwritings.org/read?panels=p14068.9980018]. The question she had was: \textit{is God a person, having a form like Jesus}? The answer was affirmative—with a strong biblical foundation. Her visions were not the source of the truth on the \emcap{personality of God}; rather, they confirmed the truth the pioneers had discovered through diligent study of God’s word.


Sœur White a rappelé les expériences de la façon dont les \textit{points principaux de notre foi}, tels qu'ils étaient tenus autrefois, ont été fermement établis.\egwinline{\textbf{\underline{Point après point} a été clairement défini, et tous les frères sont parvenus à l'harmonie}}[Lt253-1903.4; 1903][https://egwwritings.org/read?panels=p14068.9980010]. Ces points principaux étaient les \emcap{Principes Fondamentaux}, dont la \emcap{personnalité de Dieu} faisait partie. Ce point, et le témoignage de Sœur White à ce sujet, sont restés les mêmes tout au long de sa vie. Elle a dit\egwinline{\textbf{J'ai toujours eu le même témoignage à porter que celui que je porte maintenant concernant la personnalité de Dieu}}[Lt253-1903.9; 1903][https://egwwritings.org/read?panels=p14068.9980015]. Dans Premiers Écrits, elle a ensuite cité ses visions de la réalité céleste. Elle a rappelé comment elle avait eu le privilège d'être en présence de Dieu, comment Dieu, entouré par la lumière de Sa gloire, est passé à côté d'elle. Elle n'a pas vu Dieu à cause de la lumière dont Il était entouré; elle avait peur de Lui, pensant que s'Il s'approchait d'elle, elle\egwinline{serait anéantie}. Puis elle a vu\egwinline{\textbf{le bien-aimé Jésus, qu'Il est une personne}. \textbf{Je Lui ai demandé si Son Père était une personne, et avait \underline{une forme comme} Lui-même}. Jésus a dit: ‘\textbf{Je suis l'empreinte de sa personne!}’}[Lt253-1903.12; 1903][https://egwwritings.org/read?panels=p14068.9980018]. La question qu'elle avait était: \textit{Dieu est-il une personne, ayant une forme comme Jésus}? La réponse était affirmative—avec un solide fondement biblique. Ses visions n'étaient pas la source de la vérité sur la \emcap{personnalité de Dieu}; elles confirmaient plutôt la vérité que les pionniers avaient découverte par une étude diligente de la parole de Dieu.


Therefore, their final conclusion on the \emcap{personality of God} was,\others{That there is \textbf{one God}, \textbf{a personal, spiritual \underline{being}}, \textbf{the creator of all things}, omnipotent, omniscient, and eternal, infinite in wisdom, holiness, justice, goodness, truth, and mercy; unchangeable, and \textbf{everywhere present by his representative, the Holy Spirit}. Ps. 139:7; That there is one Lord Jesus Christ, \textbf{the Son of the Eternal Father, the one by whom he created all things, and by whom they do consist} …and as the closing portion of his work as priest, before he takes his throne as king, he will make \textbf{the great atonement} for the sins of all such, and their sins will then be blotted out (Acts 3:19) and borne away from the sanctuary, as shown in the service of the Levitical priesthood, which foreshadowed and prefigured the ministry of our Lord in heaven. See Lev. 16; Heb. 8: 4, 5; 9: 6, 7; etc.}[The first, and part of the second, point of the Fundamental Principles, 1905.]


Par conséquent, leur conclusion finale sur la \emcap{personnalité de Dieu} était,\others{Qu'il y a \textbf{un seul Dieu}, \textbf{un \underline{être} personnel et spirituel}, \textbf{le créateur de toutes choses}, omnipotent, omniscient, et éternel, infini en sagesse, sainteté, justice, bonté, vérité et miséricorde; immuable, et \textbf{présent partout par son représentant, le Saint-Esprit}. Ps. 139:7; Qu'il y a un seul Seigneur Jésus-Christ, \textbf{le Fils du Père Éternel, celui par qui il a créé toutes choses, et par qui elles subsistent} …et comme la dernière partie de son œuvre en tant que prêtre, avant qu'il ne prenne son trône en tant que roi, il fera \textbf{la grande expiation} pour les péchés de tous ceux-là, et leurs péchés seront alors effacés (Actes 3:19) et emportés loin du sanctuaire, comme le montre le service du sacerdoce lévitique, qui préfigurait et représentait le ministère de notre Seigneur dans le ciel. Voir Lév. 16; Héb. 8: 4, 5; 9: 6, 7; etc.}[Le premier, et une partie du deuxième, point des Principes Fondamentaux, 1905.]


Ellen White reminded Dr. Kellogg on this point of the \emcap{fundamental principles} by stating:\egwinline{\textbf{The Father, the omniscient One, created the world \underline{through} Christ Jesus}. Christ is the light of the world, the way to eternal life. \textbf{He, the anointed One, God gave to make an atonement for the sins of the world}...}[Lt253-1903.21; 1903][https://egwwritings.org/read?panels=p14068.9980029]


Ellen White a rappelé au Dr Kellogg ce point des \emcap{principes fondamentaux} en déclarant:\egwinline{\textbf{Le Père, l'Omniscient, a créé le monde \underline{par} Christ Jésus}. Christ est la lumière du monde, le chemin vers la vie éternelle. \textbf{Lui, l'Oint, Dieu l'a donné pour faire une expiation pour les péchés du monde}...}[Lt253-1903.21; 1903][https://egwwritings.org/read?panels=p14068.9980029]


The question on the \emcap{personality of God} deals with the quality or state of God being a person. The Adventist pioneers gave an answer to it and God approved it through the writings of Ellen White: God is a \textit{personal spiritual Being} and He is our heavenly Father. Where is His presence?\egwinline{\textbf{We are not to say that the Lord God of heaven is in a leaf, or in a tree; for He is not there. \underline{He sitteth upon His throne in the heavens}}.}[Lt253-1903.15; 1903][https://egwwritings.org/read?panels=p14068.9980022] \\
His presence is on the throne in heaven. \\
\egwinline{\textbf{Heaven is not a vapor. It is a place}. \textbf{Christ has gone to prepare mansions for those who love Him}, those who, in obedience to His commands, come out from the world and are separate...}[EGW, Lt253-1903.25; 1903][https://egwwritings.org/read?panels=p14068.9980033]. \\
“...\egwinline{‘The voice of the Lord is mighty; it shaketh the cedars of Lebanon. \textbf{The Lord is in His holy temple}; let all the earth keep silence before Him.’ [See Psalm 29:5; Habakkuk 2:20.]}[Lt253-1903.18; 1903][https://egwwritings.org/read?panels=p14068.9980026]


La question sur la \emcap{personnalité de Dieu} traite de la qualité ou l'état de Dieu comme étant une personne. Les pionniers adventistes y ont répondu et Dieu l'a approuvé par les écrits d'Ellen White: Dieu est un \textit{Être spirituel personnel} et Il est notre Père céleste. Où est Sa présence?\egwinline{\textbf{Nous ne devons pas dire que le Seigneur Dieu du ciel est dans une feuille, ou dans un arbre; car Il n'y est pas. \underline{Il siège sur Son trône dans les cieux}}.}[Lt253-1903.15; 1903][https://egwwritings.org/read?panels=p14068.9980022] \\
Sa présence est sur le trône dans le ciel. \\
\egwinline{\textbf{Le ciel n'est pas une vapeur. C'est un lieu}. \textbf{Christ est allé préparer des demeures pour ceux qui L'aiment}, ceux qui, en obéissance à Ses commandements, sortent du monde et sont séparés...}[EGW, Lt253-1903.25; 1903][https://egwwritings.org/read?panels=p14068.9980033]. \\
“...\egwinline{‘La voix du Seigneur est puissante; elle fait trembler les cèdres du Liban. \textbf{Le Seigneur est dans Son saint temple}; que toute la terre fasse silence devant Lui.’ [Voir Psaume 29:5; Habacuc 2:20.]}[Lt253-1903.18; 1903][https://egwwritings.org/read?panels=p14068.9980026]


According to Adventist pioneers and Sister White, our heavenly Father is one God. He is a personal Spiritual Being, present in heaven, on His throne. The throne of heaven is a real, physical throne, upon which sits a real Person (Being, having a form, just like Jesus)—our heavenly Father. That place is a real place; it is not a vapor, or any other spiritual view.


Selon les pionniers adventistes et Sœur White, notre Père céleste est un seul Dieu. Il est un Être spirituel personnel, présent dans le ciel, sur Son trône. Le trône du ciel est un trône réel, physique, sur lequel est assis une Personne réelle (un Être, ayant une forme, tout comme Jésus)—notre Père céleste. Ce lieu est un lieu réel; ce n'est pas une vapeur, ou toute autre vue spiritualiste.


\egwinline{\textbf{I have often seen that the spiritual view took away all the glory of heaven, and that in many minds the throne of David and the lovely person of Jesus have been burned up in the fire of spiritualism}. I have seen that some who have been deceived and led into this error, will be brought out into the light of truth, \textbf{but it will be almost impossible for them to get entirely rid of the deceptive power of spiritualism. Such should make thorough work in confessing their errors, and leaving them forever}.}[Lt253-1903.13; 1903][https://egwwritings.org/read?panels=p14068.9980019]


\egwinline{\textbf{J'ai souvent vu que la vue spiritualiste enlevait toute la gloire du ciel, et que dans l'esprit de beaucoup, le trône de David et la personne aimable de Jésus ont été consumés dans le feu du spiritualisme}. J'ai vu que certains qui ont été trompés et conduits dans cette erreur, seront amenés à la lumière de la vérité, \textbf{mais il leur sera presque impossible de se débarrasser entièrement du pouvoir trompeur du spiritualisme. Ces personnes devraient faire un travail approfondi en confessant leurs erreurs et en les abandonnant pour toujours}.}[Lt253-1903.13; 1903][https://egwwritings.org/read?panels=p14068.9980019]


The spiritual view of God’s person is an erroneous view. In the Bible we have testimonies of heaven, the heavenly throne, and God who is sitting upon it. If we accept these testimonies in their obvious meaning, then the Trinity doctrine cannot be sustained. The Bible and Spirit of Prophecy present one God in heaven, as a personal being, having a body and form just as Jesus has. This view is not in harmony with the doctrine of the Triune God, since it requires the Holy Spirit to be a Being\footnote{Please look at \hyperref[appendix:unauthenticated-reports]{the appendix} for more quotations which exclude the Holy Spirit to be a being, possessing physical body and form.}, having a body and form—this idea would compromise the Holy Spirit to be a means of the Father and Son by which They are everywhere present. In order to sustain the Trinity doctrine, the testimonies regarding the throne of God and of God’s person, need to be understood by some spiritual view. Here we have seen that Sister White contrasted the truth of the \emcap{personality of God} with the doctrine of Trinity. She contrasted the doctrine of Trinity with the first two points of the \emcap{Fundamental Principles}, which were the results of our pioneers studying the Word of God. Referring to the pioneers and the \emcap{Fundamental Principles}, she said: \egwinline{\textbf{\underline{Patchwork theories} cannot be accepted by those who are \underline{loyal to the faith and to the principles} that have withstood all the opposition of satanic influences.}}[Lt253-1903.28; 1903][https://egwwritings.org/read?panels=p14068.9980036]


La vue spiritualiste de la personne de Dieu est une vue erronée. Dans la Bible, nous avons des témoignages du ciel, du trône céleste, et de Dieu qui y est assis. Si nous acceptons ces témoignages dans leur sens évident, alors la doctrine de la Trinité ne peut pas être soutenue. La Bible et l'Esprit de Prophétie présentent un seul Dieu dans le ciel, comme un être personnel, ayant un corps et une forme tout comme Jésus. Cette vue n'est pas en harmonie avec la doctrine du Dieu Trinitaire, puisqu'elle exigerait que le Saint-Esprit soit un Être\footnote{Veuillez consulter \hyperref[appendix:unauthenticated-reports]{l'annexe} pour plus de citations qui excluent que le Saint-Esprit soit un être, possédant un corps physique et une forme.}, ayant un corps et une forme—cette idée compromettrait le Saint-Esprit comme moyen du Père et du Fils par lequel Ils sont présents partout. Afin de soutenir la doctrine de la Trinité, les témoignages concernant le trône de Dieu et la personne de Dieu doivent être compris par une certaine vue spiritualiste. Ici, nous avons vu que Sœur White a opposé la vérité de la \emcap{personnalité de Dieu} à la doctrine de la Trinité. Elle a opposé la doctrine de la Trinité aux deux premiers points des \emcap{Principes Fondamentaux}, qui étaient les résultats de l'étude de la Parole de Dieu par nos pionniers. En se référant aux pionniers et aux \emcap{Principes Fondamentaux}, elle a dit: \egwinline{\textbf{Les \underline{théories de rapiècement} ne peuvent pas être acceptées par ceux qui sont \underline{loyaux à la foi et aux principes} qui ont résisté à toute l'opposition des influences sataniques.}}[Lt253-1903.28; 1903][https://egwwritings.org/read?panels=p14068.9980036]


The conclusion is straightforward and simple. Those who are loyal to the faith, and to the principles received in the beginning of the work, cannot accept patchwork theories. Put into context, the patchwork theory, which is the Trinity doctrine, cannot be accepted by those who are holding fast \egwinline{\textbf{to \underline{the fundamental principles} that are \underline{based upon unquestionable authority}}}[SpTB02 59.1; 1904][https://egwwritings.org/read?panels=p417.299]. This conclusion leads us back to our first proposed test of the foundation of our faith.


La conclusion est simple et directe. Ceux qui sont loyaux à la foi, et aux principes reçus au début de l'œuvre, ne peuvent pas accepter les théories de rapiècement. Mis en contexte, la théorie de rapiècement, qui est la doctrine de la Trinité, ne peut pas être acceptée par ceux qui tiennent fermement \egwinline{\textbf{aux \underline{principes fondamentaux} qui sont \underline{basés sur une autorité incontestable}}}[SpTB02 59.1; 1904][https://egwwritings.org/read?panels=p417.299]. Cette conclusion nous ramène à notre premier test proposé du fondement de notre foi.


% The Patchwork Theories

\begin{titledpoem}
    
    \stanza{
        Truth established through earnest prayer, \\
        Points of faith discovered with care. \\
        Principles tested by time and trial, \\
        Stand firm against Satan's denial.
    }

    \stanza{
        Patchwork theories seek to sway, \\
        Those from the ancient, proven way. \\
        No revisions of truth we'll accept, \\
        The faithful path must be kept.
    }

    \stanza{
        The personality of God, a sacred revelation, \\
        Not subject to human innovation. \\
        Loyal hearts stand on ground that's sure, \\
        Where foundations eternal endure.
    }
    
\end{titledpoem}