\qrchapterstar{https://forgottenpillar.com/rsc/en-fp-appendix}{Appendix} \label{chap:appendix}


\qrchapterstar{https://forgottenpillar.com/rsc/en-fp-appendix}{Annexe} \label{chap:appendix}


\addcontentsline{toc}{chapter}{Appendix}


\addcontentsline{toc}{chapter}{Annexe}


\section*{The Fundamental Principles 1889}


\section*{Les Principes Fondamentaux 1889}


As elsewhere stated, Seventh-day Adventists have no creed but the Bible; but they hold to certain well-defined points of faith for which they feel prepared to give a reason “to every man that asketh” them. The following propositions may be taken as a summary of the principal features of their religious faith, upon which there is, so far as we know, entire unanimity throughout the body. They believe,—


Comme indiqué ailleurs, les Adventistes du Septième Jour n'ont pas d'autre credo que la Bible ; mais ils adhèrent à certains points de foi bien définis pour lesquels ils se sentent prêts à donner une raison « à quiconque leur demande ». Les propositions suivantes peuvent être considérées comme un résumé des principales caractéristiques de leur foi religieuse, sur lesquelles il y a, pour autant que nous sachions, une unanimité complète dans tout le corps. Ils croient,—


\lettrine{I.} That there is one God, a personal, spiritual being, the creator of all things, omnipotent, omniscient, and eternal; infinite in wisdom, holiness, justice, goodness, truth, and mercy; unchangeable, and everywhere present by his representative, the Holy Spirit. Psalm 139:7.


\lettrine{I.} Qu'il y a un seul Dieu, un être personnel et spirituel, le créateur de toutes choses, omnipotent, omniscient et éternel ; infini en sagesse, sainteté, justice, bonté, vérité et miséricorde ; immuable, et partout présent par son représentant, le Saint-Esprit. Psaume 139:7.


\lettrine{II.} That there is one Lord Jesus Christ, the Son of the Eternal Father, the one by whom he created all things, and by whom they do consist; that he took on him the nature of the seed of Abraham for the redemption of our fallen race; that he dwelt among men, full of grace and truth, lived our example, died our sacrifice, was raised for our justification, ascended on high to be our only mediator in the sanctuary in heaven, where, through the merits of his shed blood, he secures the pardon and forgiveness of the sins of all those who penitently come to him; and as the closing portion of his work as priest, before he takes his throne as king, he will make the great atonement for the sins of all such, and their sins will then be blotted out (Acts 3:19) and borne away from the sanctuary, as shown in the service of the Levitical priesthood, which foreshadowed and prefigured the ministry of our Lord in heaven. See Leviticus 16; Hebrews 8:4, 5; 9:6, 7; etc.


\lettrine{II.} Qu'il y a un seul Seigneur Jésus-Christ, le Fils du Père Éternel, celui par qui il a créé toutes choses, et par qui elles subsistent ; qu'il a pris sur lui la nature de la postérité d'Abraham pour la rédemption de notre race déchue ; qu'il a habité parmi les hommes, plein de grâce et de vérité, a vécu notre exemple, est mort notre sacrifice, a été ressuscité pour notre justification, est monté en haut pour être notre seul médiateur dans le sanctuaire céleste, où, par les mérites de son sang versé, il assure le pardon et la rémission des péchés de tous ceux qui viennent à lui avec repentance ; et comme partie finale de son œuvre en tant que prêtre, avant qu'il ne prenne son trône comme roi, il fera la grande expiation pour les péchés de tous ceux-là, et leurs péchés seront alors effacés (Actes 3:19) et emportés loin du sanctuaire, comme le montre le service du sacerdoce lévitique, qui préfigurait et symbolisait le ministère de notre Seigneur dans le ciel. Voir Lévitique 16 ; Hébreux 8:4, 5 ; 9:6, 7 ; etc.


\lettrine{III.} That the Holy Scriptures of the Old and New Testaments were given by inspiration of God, contain a full revelation of his will to man, and are the only infallible rule of faith and practice.


\lettrine{III.} Que les Saintes Écritures de l'Ancien et du Nouveau Testament ont été données par inspiration de Dieu, contiennent une révélation complète de sa volonté à l'homme, et sont la seule règle infaillible de foi et de pratique.


\lettrine{IV.} That baptism is an ordinance of the Christian church, to follow faith and repentance,—an ordinance by which we commemorate the resurrection of Christ, as by this act we show our faith in his burial and resurrection, and through that, in the resurrection of all the saints at the last day; and that no other mode more fitly represents these facts than that which the Scriptures prescribe, namely, immersion. Romans 6:3-5; Colossians 2:12.


\lettrine{IV.} Que le baptême est une ordonnance de l'église chrétienne, qui doit suivre la foi et la repentance,—une ordonnance par laquelle nous commémorons la résurrection du Christ, car par cet acte nous montrons notre foi en son ensevelissement et sa résurrection, et par là, en la résurrection de tous les saints au dernier jour ; et qu'aucun autre mode ne représente plus convenablement ces faits que celui que les Écritures prescrivent, à savoir, l'immersion. Romains 6:3-5 ; Colossiens 2:12.


\lettrine{V.} That the new birth comprises the entire change necessary to fit us for the kingdom of God, and consists of two parts; First, a moral change wrought by conversion and a Christian life (John 3:3, 5); second, a physical change at the second coming of Christ, whereby, if dead, we are raised incorruptible, and if living, are changed to immortality in a moment, in the twinkling of an eye. Luke 20:36; 1 Corinthians 15:51, 52.


\lettrine{V.} Que la nouvelle naissance comprend le changement entier nécessaire pour nous préparer au royaume de Dieu, et consiste en deux parties ; Premièrement, un changement moral opéré par la conversion et une vie chrétienne (Jean 3:3, 5) ; deuxièmement, un changement physique à la seconde venue du Christ, par lequel, si nous sommes morts, nous sommes ressuscités incorruptibles, et si nous sommes vivants, nous sommes changés à l'immortalité en un instant, en un clin d'œil. Luc 20:36 ; 1 Corinthiens 15:51, 52.


\lettrine{VI.} That prophecy is a part of God’s revelation to man; that it is included in that Scripture which is profitable for instruction (2 Timothy 3:16); that it is designed for us and our children (Deuteronomy 29:29); that so far from being enshrouded in impenetrable mystery, it is that which especially constitutes the word of God a lamp to our feet and a light to our path (Psalm 119:105; 2 Peter 1:19); that a blessing is pronounced upon those who study it (Revelation 1:1-3); and that, consequently, it is to be understood by the people of God sufficiently to show them their position in the world’s history and the special duties required at their hands.


\lettrine{VI.} Que la prophétie est une partie de la révélation de Dieu à l'homme ; qu'elle est incluse dans cette Écriture qui est utile pour l'instruction (2 Timothée 3:16) ; qu'elle est destinée pour nous et nos enfants (Deutéronome 29:29) ; que loin d'être enveloppée dans un mystère impénétrable, c'est elle qui constitue spécialement la parole de Dieu comme une lampe à nos pieds et une lumière sur notre sentier (Psaume 119:105 ; 2 Pierre 1:19) ; qu'une bénédiction est prononcée sur ceux qui l'étudient (Apocalypse 1:1-3) ; et que, par conséquent, elle doit être comprise par le peuple de Dieu suffisamment pour leur montrer leur position dans l'histoire du monde et les devoirs spéciaux requis de leurs mains.


\lettrine{VII.} That the world’s history from specified dates in the past, the rise and fall of empires, and the chronological succession of events down to the setting up of God’s everlasting kingdom, are outlined in numerous great chains of prophecy; and that these prophecies are now all fulfilled except the closing scenes.


\lettrine{VII.} Que l'histoire du monde depuis des dates spécifiées dans le passé, l'ascension et la chute des empires, et la succession chronologique des événements jusqu'à l'établissement du royaume éternel de Dieu, sont décrits dans de nombreuses grandes chaînes de prophétie ; et que ces prophéties sont maintenant toutes accomplies sauf les scènes finales.


\lettrine{VIII.} That the doctrine of the world’s conversion and a temporal millennium is a fable of these last days, calculated to lull men into a state of carnal security, and cause them to be overtaken by the great day of the Lord as by a thief in the night (1 Thessalonians 5:3); that the second coming of Christ is to precede, not follow, the millennium; for until the Lord appears, the papal power, with all its abominations, is to continue (2 Thessalonians 2:8), the wheat and tares grow together (Matthew 13:29, 30, 39), and evil men and seducers wax worse and worse, as the word of God declares. 2 Timothy 3:1, 13.


\lettrine{VIII.} Que la doctrine de la conversion du monde et d'un millénium temporel est une fable de ces derniers jours, calculée pour bercer les hommes dans un état de sécurité charnelle, et les faire surprendre par le grand jour du Seigneur comme par un voleur dans la nuit (1 Thessaloniciens 5:3) ; que la seconde venue du Christ doit précéder, et non suivre, le millénium ; car jusqu'à ce que le Seigneur apparaisse, le pouvoir papal, avec toutes ses abominations, doit continuer (2 Thessaloniciens 2:8), le blé et l'ivraie croissent ensemble (Matthieu 13:29, 30, 39), et les hommes méchants et les séducteurs vont de mal en pis, comme la parole de Dieu le déclare. 2 Timothée 3:1, 13.


\lettrine{IX.} That the mistake of Adventists in 1844 pertained to the nature of the event then to transpire, not to the time; that no prophetic period is given to reach to the second advent, but that the longest one, the two thousand and three hundred days of Daniel 8:14, terminated in 1844, and brought us to an event called the cleansing of the sanctuary.


\lettrine{IX.} Que l'erreur des Adventistes en 1844 concernait la nature de l'événement qui devait alors se produire, non le temps ; qu'aucune période prophétique n'est donnée pour atteindre le second avènement, mais que la plus longue, les deux mille trois cents jours de Daniel 8:14, s'est terminée en 1844, et nous a amenés à un événement appelé la purification du sanctuaire.


\lettrine{X.} That the sanctuary of the new covenant is the tabernacle of God in heaven, of which Paul speaks in Hebrews 8 and onward, and of which our Lord, as great high priest, is minister; that this sanctuary is the antitype of the Mosaic tabernacle, and that the priestly work of our Lord, connected therewith, is the antitype of the work of the Jewish priests of the former dispensation (Hebrews 8:1-5, etc.); that this, and not the earth, is the sanctuary to be cleansed at the end of the two thousand and three hundred days, what is termed its cleansing being in this case, as in the type, simply the entrance of the high priest into the most holy place, to finish the round of service connected therewith, by making the atonement and removing from the sanctuary the sins which had been transferred to it by means of the ministration in the first apartment (Leviticus 16; Hebrews 9:22, 23); and that this work in the antitype, beginning in 1844, consists in actually blotting out the sins of believers (Acts 3:19), and occupies a brief but indefinite space of time, at the conclusion of which the work of mercy for the world will be finished, and the second advent of Christ will take place.


\lettrine{X.} Que le sanctuaire de la nouvelle alliance est le tabernacle de Dieu dans le ciel, dont Paul parle dans Hébreux 8 et suivants, et dont notre Seigneur, comme grand souverain sacrificateur, est ministre ; que ce sanctuaire est l'antitype du tabernacle mosaïque, et que l'œuvre sacerdotale de notre Seigneur, qui y est liée, est l'antitype de l'œuvre des prêtres juifs de l'ancienne dispensation (Hébreux 8:1-5, etc.) ; que ceci, et non la terre, est le sanctuaire à purifier à la fin des deux mille trois cents jours, ce qui est appelé sa purification étant dans ce cas, comme dans le type, simplement l'entrée du souverain sacrificateur dans le lieu très saint, pour finir le cycle de service qui y est lié, en faisant l'expiation et en enlevant du sanctuaire les péchés qui y avaient été transférés par le moyen du ministère dans le premier appartement (Lévitique 16 ; Hébreux 9:22, 23) ; et que cette œuvre dans l'antitype, commençant en 1844, consiste à effacer réellement les péchés des croyants (Actes 3:19), et occupe un espace de temps bref mais indéfini, à la conclusion duquel l'œuvre de miséricorde pour le monde sera terminée, et le second avènement du Christ aura lieu.


\lettrine{XI.} That God’s moral requirements are the same upon all men in all dispensations; that these are summarily contained in the commandments spoken by Jehovah from Sinai, engraven on the tables of stone, and deposited in the ark, which was in consequence called the “ark of the covenant,” or testament (Numbers 10:33; Hebrews 9:4, etc.); that this law is immutable and perpetual, being a transcript of the tables deposited in the ark in the true sanctuary on high, which is also, for the same reason, called the ark of God’s testament; for under the sounding of the seventh trumpet we are told that “the temple of God was opened in heaven, and there was seen in his temple the ark of his testament.” Revelation 11:19.


\lettrine{XI.} Que les exigences morales de Dieu sont les mêmes pour tous les hommes dans toutes les dispensations ; que celles-ci sont sommairement contenues dans les commandements prononcés par Jéhovah depuis le Sinaï, gravés sur les tables de pierre, et déposés dans l'arche, qui fut en conséquence appelée « l'arche de l'alliance », ou testament (Nombres 10:33 ; Hébreux 9:4, etc.) ; que cette loi est immuable et perpétuelle, étant une transcription des tables déposées dans l'arche dans le vrai sanctuaire en haut, qui est aussi, pour la même raison, appelée l'arche du testament de Dieu ; car sous le son de la septième trompette, il nous est dit que « le temple de Dieu fut ouvert dans le ciel, et l'on vit dans son temple l'arche de son testament ». Apocalypse 11:19.


\lettrine{XII.} That the fourth commandment of this law requires that we devote the seventh day of each week, commonly called Saturday, to abstinence from our own labor, and to the performance of sacred and religious duties; that this is the only weekly Sabbath known to the Bible, being the day that was set apart before Paradise was lost (Genesis 2:2, 3), and which will be observed in Paradise restored (Isaiah 66:22, 23); that the facts upon which the Sabbath institution is based confine it to the seventh day, as they are not true of any other day; and that the terms Jewish Sabbath, as applied to the seventh day, and Christian Sabbath, as applied to the first day of the week, are names of human invention, unscriptural in fact, and false in meaning.


\lettrine{XII.} Que le quatrième commandement de cette loi exige que nous consacrions le septième jour de chaque semaine, communément appelé samedi, à l'abstinence de notre propre travail, et à l'accomplissement de devoirs sacrés et religieux ; que c'est le seul sabbat hebdomadaire connu de la Bible, étant le jour qui fut mis à part avant que le Paradis ne soit perdu (Genèse 2:2, 3), et qui sera observé dans le Paradis restauré (Ésaïe 66:22, 23) ; que les faits sur lesquels l'institution du sabbat est fondée le confinent au septième jour, car ils ne sont vrais d'aucun autre jour ; et que les termes sabbat juif, appliqués au septième jour, et sabbat chrétien, appliqués au premier jour de la semaine, sont des noms d'invention humaine, non scripturaires en fait, et faux en signification.


\lettrine{XIII.} That as the man of sin, the papacy, has thought to change times and laws (the law of God, Daniel 7:25), and has misled almost all Christendom in regard to the fourth commandment, we find a prophecy of a reform in this respect to be wrought among believers just before the coming of Christ. Isaiah 56:1, 2; 1 Peter 1:5; Revelation 14:12, etc.


\lettrine{XIII.} Que comme l'homme de péché, la papauté, a pensé changer les temps et les lois (la loi de Dieu, Daniel 7:25), et a égaré presque toute la chrétienté en ce qui concerne le quatrième commandement, nous trouvons une prophétie d'une réforme à cet égard devant être accomplie parmi les croyants juste avant la venue du Christ. Ésaïe 56:1, 2 ; 1 Pierre 1:5 ; Apocalypse 14:12, etc.


\lettrine{XIV.} That the followers of Christ should be a peculiar people, not following the maxims, nor conforming to the ways, of the world; not loving its pleasures nor countenancing its follies; inasmuch as the apostle says that “whosoever therefore will be” in this sense, “a friend of the world, is the enemy of God” (James 4:4); and Christ says that we cannot have two masters, or, at the same time, serve God and mammon. Matthew 6:24.


\lettrine{XIV.} Que les disciples du Christ devraient être un peuple particulier, ne suivant pas les maximes, ni se conformant aux voies du monde ; n'aimant pas ses plaisirs ni encourageant ses folies ; d'autant que l'apôtre dit que « quiconque donc veut être » dans ce sens, « ami du monde, se rend ennemi de Dieu » (Jacques 4:4) ; et Christ dit que nous ne pouvons avoir deux maîtres, ou, en même temps, servir Dieu et Mammon. Matthieu 6:24.


\lettrine{XV.} That the Scriptures insist upon plainness and modesty of attire as a prominent mark of discipleship in those who profess to be the followers of Him who was, “meek and lowly in heart,” that the wearing of gold, pearls, and costly array, or anything designed merely to adorn the person and foster the pride of the natural heart, is to be discarded, according to such scriptures as 1 Timothy 2:9, 10; 1 Peter 3:3, 4.


\lettrine{XV.} Que les Écritures insistent sur la simplicité et la modestie de la tenue comme une marque éminente de discipulat chez ceux qui professent être les disciples de Celui qui était « doux et humble de cœur », que le port d'or, de perles et de parures coûteuses, ou de tout ce qui est conçu simplement pour orner la personne et favoriser l'orgueil du cœur naturel, doit être rejeté, selon des écritures telles que 1 Timothée 2:9, 10 ; 1 Pierre 3:3, 4.


\lettrine{XVI.} That means for the support of evangelical work among men should be contributed from love to God and love of souls, not raised by church lotteries, or occasions designed to contribute to the fun-loving, appetite-indulging propensities of the sinner, such as fairs, festivals, oyster suppers, tea, broom, donkey, and crazy socials, etc., which are a disgrace to the professed church of Christ; that the proportion of one’s income required in former dispensation can be no less under the gospel; that it is the same as Abraham (whose children we are, if we are Christ’s, Galatians 3:29) paid to Melchisedec (type of Christ) when he gave him a tenth of all (Hebrews 7:1-4); the title is the Lord’s (Leviticus 27:30); and this tenth of one’s income is also to be supplemented by offerings from those who are able, for the support of the gospel. 2 Corinthians 9:6; Malachi 3:8, 10.


\lettrine{XVI.} Que les moyens pour le soutien de l'œuvre évangélique parmi les hommes devraient être contribués par amour pour Dieu et amour des âmes, non collectés par des loteries d'église, ou des occasions conçues pour contribuer aux propensions amusantes et gourmandes du pécheur, telles que les foires, festivals, soupers aux huîtres, thés, sociales de balais, d'ânes et folles, etc., qui sont une disgrâce pour l'église professée du Christ ; que la proportion du revenu d'une personne requise dans l'ancienne dispensation ne peut être moindre sous l'évangile ; que c'est la même qu'Abraham (dont nous sommes les enfants, si nous sommes du Christ, Galates 3:29) a payée à Melchisédec (type du Christ) quand il lui a donné la dîme de tout (Hébreux 7:1-4) ; la dîme appartient au Seigneur (Lévitique 27:30) ; et ce dixième du revenu d'une personne doit aussi être complété par des offrandes de ceux qui en sont capables, pour le soutien de l'évangile. 2 Corinthiens 9:6 ; Malachie 3:8, 10.


\lettrine{XVII.} That as the natural or carnal heart is at enmity with God and his law, this enmity can be subdued only by a radical transformation of the affections, the exchange of unholy for holy principles; that this transformation follows repentance and faith, is the special work of the Holy Spirit, and constitutes regeneration, or conversion.


\lettrine{XVII.} Que comme le cœur naturel ou charnel est en inimitié avec Dieu et sa loi, cette inimitié ne peut être soumise que par une transformation radicale des affections, l'échange de principes impies pour des principes saints ; que cette transformation suit la repentance et la foi, est l'œuvre spéciale du Saint-Esprit, et constitue la régénération, ou conversion.


\lettrine{XVIII.} That as all have violated the law of God, and cannot of themselves render obedience to his just requirements, we are dependent on Christ, first, for justification from our past offenses, and, secondly, for grace whereby to render acceptable obedience to his holy law in time to come.


\lettrine{XVIII.} Que comme tous ont violé la loi de Dieu, et ne peuvent d'eux-mêmes rendre obéissance à ses justes exigences, nous sommes dépendants de Christ, premièrement, pour la justification de nos offenses passées, et, deuxièmement, pour la grâce par laquelle rendre une obéissance acceptable à sa sainte loi dans le temps à venir.


\lettrine{XIX.} That the Spirit of God was promised to manifest itself in the church through certain gifts, enumerated especially in 1 Corinthians 12 and Ephesians 4; that these gifts are not designed to supersede, or take the place of, the Bible, which is sufficient to make us wise unto salvation, any more than the Bible can take the place of the Holy Spirit; that, in specifying the various channels of its operation, that Spirit has simply made provision for its own existence and presence with the people of God to the end of time, to lead to an understanding of that word which it had inspired, to convince of sin, and to work a transformation in the heart and life; and that those who deny to the Spirit its place and operation, do plainly deny that part of the Bible which assigns to it this work and position.


\lettrine{XIX.} Que l'Esprit de Dieu a été promis pour se manifester dans l'église à travers certains dons, énumérés spécialement dans 1 Corinthiens 12 et Éphésiens 4 ; que ces dons ne sont pas conçus pour remplacer, ou prendre la place de la Bible, qui est suffisante pour nous rendre sages à salut, pas plus que la Bible ne peut prendre la place du Saint-Esprit ; qu'en spécifiant les divers canaux de son opération, cet Esprit a simplement pourvu à sa propre existence et présence avec le peuple de Dieu jusqu'à la fin des temps, pour conduire à une compréhension de cette parole qu'il avait inspirée, pour convaincre de péché, et pour opérer une transformation dans le cœur et la vie ; et que ceux qui nient à l'Esprit sa place et son opération, nient clairement cette partie de la Bible qui lui assigne ce travail et cette position.


\lettrine{XX.} That God, in accordance with his uniform dealings with the race, sends forth a proclamation of the approach of the second advent of Christ; and that this work is symbolized by the three messages of Revelation 14, the last one bringing to view the work of reform on the law of God, that his people may acquire a complete readiness for that event.


\lettrine{XX.} Que Dieu, en accord avec ses rapports uniformes avec la race, envoie une proclamation de l'approche du second avènement de Christ ; et que cette œuvre est symbolisée par les trois messages d'Apocalypse 14, le dernier présentant l'œuvre de réforme sur la loi de Dieu, afin que son peuple puisse acquérir une complète préparation pour cet événement.


\lettrine{XXI.} That the time of the cleansing of the sanctuary (See proposition X.), synchronizing with the time of the proclamation of the third message (Revelation 14:9, 10), is a time of investigative judgment, first, with reference to the dead, and secondly, at the close of probation, with reference to the living, to determine who of the myriads now sleeping in the dust of the earth are worthy of a part in the first resurrection, and who of its living multitudes are worthy of translation,—points which must be determined before the Lord appears.


\lettrine{XXI.} Que le temps de la purification du sanctuaire (Voir proposition X.), se synchronisant avec le temps de la proclamation du troisième message (Apocalypse 14:9, 10), est un temps de jugement investigatif, premièrement, en référence aux morts, et deuxièmement, à la fin du temps de grâce, en référence aux vivants, pour déterminer qui des myriades dormant maintenant dans la poussière de la terre sont dignes d'une part dans la première résurrection, et qui de ses multitudes vivantes sont dignes de translation,—points qui doivent être déterminés avant que le Seigneur apparaisse.


\lettrine{XXII.} That the grave, whether we all tend, expressed by the Hebrew word sheol and the Greek word hades, is a place, or condition, in which there is no work, device, wisdom, nor knowledge. Ecclesiastes 9:10.


\lettrine{XXII.} Que la tombe, vers laquelle nous tendons tous, exprimée par le mot hébreu sheol et le mot grec hadès, est un lieu, ou condition, dans lequel il n'y a ni œuvre, ni dessein, ni sagesse, ni connaissance. Ecclésiaste 9:10.


\lettrine{XXIII.} That the state to which we are reduced by death is one of silence, inactivity, and entire unconsciousness. Psalm 146:4; Ecclesiastes 9:5, 6; Daniel 12:2.


\lettrine{XXIII.} Que l'état auquel nous sommes réduits par la mort est un état de silence, d'inactivité, et d'entière inconscience. Psaume 146:4 ; Ecclésiaste 9:5, 6 ; Daniel 12:2.


\lettrine{XXIV.} That out of this prison-house of the grave, mankind are to be brought by a bodily resurrection; the righteous having part in the first resurrection, which takes place at the second coming of Christ; the wicked, in the second resurrection, which takes place in a thousand years thereafter. Revelation 20:4-6.


\lettrine{XXIV.} Que de cette prison de la tombe, l'humanité doit être amenée par une résurrection corporelle ; les justes ayant part à la première résurrection, qui a lieu au second avènement de Christ ; les méchants, dans la seconde résurrection, qui a lieu mille ans après. Apocalypse 20:4-6.


\lettrine{XXV.} That at the last trump, the living righteous are to be changed in a moment, in the twinkling of an eye, and with the risen righteous are to be caught up to meet the Lord in the air, so forever to be with the Lord. 1 Thessalonians 4:16, 17; 1 Corinthians 15:51, 52.


\lettrine{XXV.} Qu'à la dernière trompette, les justes vivants doivent être changés en un moment, en un clin d'œil, et avec les justes ressuscités doivent être enlevés pour rencontrer le Seigneur dans les airs, pour être ainsi pour toujours avec le Seigneur. 1 Thessaloniciens 4:16, 17 ; 1 Corinthiens 15:51, 52.


\lettrine{XXVI.} That these immortalized ones are then taken to heaven, to the New Jerusalem, the Father’s house, in which there are many mansions (John 14:1-3), where they reign with Christ a thousand years, judging the world and fallen angels, that is, apportioning the punishment to be executed upon them at the close of the one thousand years (Revelation 20:4; 1 Corinthians 6:2, 3); that during this time the earth lies in a desolate and chaotic condition (Jeremiah 4:23-27), described, as in the beginning, by the Greek term abussos— “bottom-less pit” (Septuagint of Genesis 1:2); and that here Satan is confined during the thousand years (Revelation 20:1, 2), and here finally destroyed (Revelation 20:10; Malachi 4:1); the theater of the ruin he has wrought in the universe being appropriately made, for a time, his gloomy prison-house, and then the place of his final execution.


\lettrine{XXVI.} Que ces immortalisés sont alors emmenés au ciel, à la Nouvelle Jérusalem, la maison du Père, dans laquelle il y a plusieurs demeures (Jean 14:1-3), où ils règnent avec Christ mille ans, jugeant le monde et les anges déchus, c'est-à-dire, répartissant la punition à être exécutée sur eux à la fin des mille ans (Apocalypse 20:4 ; 1 Corinthiens 6:2, 3) ; que pendant ce temps la terre gît dans une condition désolée et chaotique (Jérémie 4:23-27), décrite, comme au commencement, par le terme grec abussos—« abîme sans fond » (Septante de Genèse 1:2) ; et que c'est ici que Satan est confiné pendant les mille ans (Apocalypse 20:1, 2), et ici finalement détruit (Apocalypse 20:10 ; Malachie 4:1) ; le théâtre de la ruine qu'il a opérée dans l'univers étant appropriément fait, pour un temps, sa sombre prison, et ensuite le lieu de son exécution finale.


\lettrine{XXVII.} That at the end of the thousand years the Lord descends with his people and the New Jerusalem (Revelation 21:2), the wicked dead are raised, and come up on the surface of the yet unrenewed earth, and gather about the city, the camp of the saints (Revelation 20:9), and fire comes down from God out of heaven and devours them. They are then consumed, root and branch (Malachi 4:1), becoming as though they had not been. Obadiah 15, 16. In this everlasting destruction from the presence of the Lord (2 Thessalonians 1:9), the wicked meet the “everlasting punishment” threatened against them (Matthew 25:46), which is everlasting death. Romans 6:23; Revelation 20:14, 15. This is the perdition of ungodly men, the fire which consumes them being the fire for which “the heavens and the earth, which are now,... are kept in store.” which shall melt even the elements with its intensity, and purge the earth from the deepest stains of the curse of sin. 2 Peter 3:7-12.


\lettrine{XXVII.} Qu'à la fin des mille ans, le Seigneur descend avec son peuple et la Nouvelle Jérusalem (Apocalypse 21:2), les méchants morts sont ressuscités et montent à la surface de la terre encore non renouvelée, et se rassemblent autour de la ville, le camp des saints (Apocalypse 20:9), et le feu descend de Dieu du ciel et les dévore. Ils sont alors consumés, racine et rameau (Malachie 4:1), devenant comme s'ils n'avaient jamais été. Abdias 15, 16. Dans cette destruction éternelle loin de la présence du Seigneur (2 Thessaloniciens 1:9), les méchants rencontrent le « châtiment éternel » dont ils sont menacés (Matthieu 25:46), qui est la mort éternelle. Romains 6:23 ; Apocalypse 20:14, 15. C'est la perdition des hommes impies, le feu qui les consume étant le feu pour lequel « les cieux et la terre qui sont maintenant... sont gardés en réserve », qui fera fondre même les éléments par son intensité, et purgera la terre des taches les plus profondes de la malédiction du péché. 2 Pierre 3:7-12.


\lettrine{XXVIII.} That new heavens and a new earth shall spring by the power of God from the ashes of the old, and this renewed earth, with the New Jerusalem for its metropolis and capital, shall be the eternal inheritance of the saints, the place where the righteous shall evermore dwell. 2 Peter 3:13; Psalm 37:11, 29; Matthew 5:5.


\lettrine{XXVIII.} Que de nouveaux cieux et une nouvelle terre surgiront par la puissance de Dieu des cendres de l'ancienne, et cette terre renouvelée, avec la Nouvelle Jérusalem pour métropole et capitale, sera l'héritage éternel des saints, le lieu où les justes demeureront à jamais. 2 Pierre 3:13 ; Psaume 37:11, 29 ; Matthieu 5:5.


\section*{Fundamental Principles - Timeline} \label{appendix:timeline}


\section*{Principes Fondamentaux - Chronologie} \label{appendix:timeline}


The following is a list of some appearances of the Declaration of Fundamental Principles in our publications. All links are accessible at \href{https://notefp.link/fp-timeline}{https://notefp.link/fp-timeline}.


Ce qui suit est une liste de certaines apparitions de la Déclaration des Principes Fondamentaux dans nos publications. Tous les liens sont accessibles à \href{https://notefp.link/fp-timeline}{https://notefp.link/fp-timeline}.


\leftsubsection{1872 - The first appearance}


\leftsubsection{1872 - La première apparition}


\textit{“A Declaration of the Fundamental Principles Taught and Practiced by Seventh-day Adventists}” - printed as a pamphlet (\href{https://adventistdigitallibrary.org/islandora/object/adl:366607?link_only=true}{original scan} \href{https://forgotten-pillar.s3.us-east-2.amazonaws.com/A+declaration+of+the+fundamental+principles+taught+and+practiced+by+the+Seventh-day+Adventists++.pdf}{*}). They appeared anonymous, presented as a short public synopsis of what Seventh-day Adventists believe.


\textit{« Déclaration des Principes Fondamentaux Enseignés et Pratiqués par les Adventistes du Septième Jour »} - imprimée sous forme de brochure (\href{https://adventistdigitallibrary.org/islandora/object/adl:366607?link_only=true}{scan original} \href{https://forgotten-pillar.s3.us-east-2.amazonaws.com/A+declaration+of+the+fundamental+principles+taught+and+practiced+by+the+Seventh-day+Adventists++.pdf}{*}). Ils sont apparus de manière anonyme, présentés comme un bref synopsis public de ce que croient les Adventistes du Septième Jour.


\leftsubsection{1874 - The Signs of the Times}


\leftsubsection{1874 - The Signs of the Times}


Original scan: \href{https://adventistdigitallibrary.org/adl-364148/signs-times-june-4-1874}{ST June 4, 1874, p.3.} \href{https://forgotten-pillar.s3.us-east-2.amazonaws.com/Signs+of+the+Times+_+June+4%2C+1874++.pdf}{*} James White stood behind the declaration as a main editor of the Signs of the Times at that time.


Scan original : \href{https://adventistdigitallibrary.org/adl-364148/signs-times-june-4-1874}{ST 4 juin 1874, p.3.} \href{https://forgotten-pillar.s3.us-east-2.amazonaws.com/Signs+of+the+Times+_+June+4%2C+1874++.pdf}{*} James White soutenait la déclaration en tant que rédacteur en chef du Signs of the Times à cette époque.


\leftsubsection{1874 - The Advent Review and Herald of the Sabbath}


\leftsubsection{1874 - The Advent Review and Herald of the Sabbath}


Original scan: \href{https://documents.adventistarchives.org/Periodicals/RH/RH18741124-V44-22.pdf}{RH November 24, 1874, p.171} \href{https://forgotten-pillar.s3.us-east-2.amazonaws.com/RH18741124-V44-22.pdf}{*} Uriah Smith signed the declaration as the main editor of the Review and Herald of the Sabbath periodical at that time.


Scan original : \href{https://documents.adventistarchives.org/Periodicals/RH/RH18741124-V44-22.pdf}{RH 24 novembre 1874, p.171} \href{https://forgotten-pillar.s3.us-east-2.amazonaws.com/RH18741124-V44-22.pdf}{*} Uriah Smith a signé la déclaration en tant que rédacteur en chef du périodique Review and Herald of the Sabbath à cette époque.


\leftsubsection{1874 - Part of a booklet: The Seventh-day Adventists: A Brief Sketch of Their Origin, Progress, and Principles}


\leftsubsection{1874 - Partie d'une brochure : Les Adventistes du Septième Jour : Un bref aperçu de leur origine, progrès et principes}


Booklet was reprinted in 1876 and 1878 and later years. \\
Original scan: (\href{https://adventistdigitallibrary.org/islandora/object/adl%3A22250872?solr_nav%5Bid%5D=a09d3902c2540c98eb7f&solr_nav%5Bpage%5D=56&solr_nav%5Boffset%5D=3}{1878 copy})


La brochure a été réimprimée en 1876 et 1878 et les années suivantes. \\
Scan original : (\href{https://adventistdigitallibrary.org/islandora/object/adl%3A22250872?solr_nav%5Bid%5D=a09d3902c2540c98eb7f&solr_nav%5Bpage%5D=56&solr_nav%5Boffset%5D=3}{copie de 1878})


\leftsubsection{1875 - The Signs of the Times}


\leftsubsection{1875 - The Signs of the Times}


Original scan: \href{https://documents.adventistarchives.org/Periodicals/ST/ST18750128-V01-14.pdf#search=ST18750128}{ST January 28, 1875} \href{https://forgotten-pillar.s3.us-east-2.amazonaws.com/ST18750128-V01-14.pdf}{*} (p. 108, 109)


Scan original : \href{https://documents.adventistarchives.org/Periodicals/ST/ST18750128-V01-14.pdf#search=ST18750128}{ST 28 janvier 1875} \href{https://forgotten-pillar.s3.us-east-2.amazonaws.com/ST18750128-V01-14.pdf}{*} (p. 108, 109)


\leftsubsection{1878 - The Signs of the Times}


\leftsubsection{1878 - The Signs of the Times}


Original scan: \href{https://documents.adventistarchives.org/Periodicals/ST/ST18780221-V04-08.pdf#search=%22As%20already%20stated%2C%20S%2E%20D%2E%20Adventists%22}{ST February 21, 1878} \href{https://forgotten-pillar.s3.us-east-2.amazonaws.com/ST18780221-V04-08.pdf}{*} (p. 59)


Scan original : \href{https://documents.adventistarchives.org/Periodicals/ST/ST18780221-V04-08.pdf#search=%22As%20already%20stated%2C%20S%2E%20D%2E%20Adventists%22}{ST 21 février 1878} \href{https://forgotten-pillar.s3.us-east-2.amazonaws.com/ST18780221-V04-08.pdf}{*} (p. 59)


\leftsubsection{1888 - Gospel Sickle, April 1, 1888}


\leftsubsection{1888 - Gospel Sickle, 1er avril 1888}


Original scan: \href{https://adventistdigitallibrary.org/adl-410336/gospel-sickle-april-1-1888?view_only=true&solr_nav%5Bid%5D=ff4d7f3f77b9bdf9e9ac&solr_nav%5Bpage%5D=0&solr_nav%5Boffset%5D=6}{Gospel Sickle, April 1, 1888}


Scan original : \href{https://adventistdigitallibrary.org/adl-410336/gospel-sickle-april-1-1888?view_only=true&solr_nav%5Bid%5D=ff4d7f3f77b9bdf9e9ac&solr_nav%5Bpage%5D=0&solr_nav%5Boffset%5D=6}{Gospel Sickle, 1er avril 1888}


\leftsubsection{1888 - The Present Truth, August 16, 1888}


\leftsubsection{1888 - The Present Truth, 16 août 1888}


Original scan: \href{https://adventistdigitallibrary.org/adl-402854/present-truth-august-16-1888?view_only=true&solr_nav%5Bid%5D=ff4d7f3f77b9bdf9e9ac&solr_nav%5Bpage%5D=0&solr_nav%5Boffset%5D=13}{PT18880816} (p. 250 - 252)


Scan original : \href{https://adventistdigitallibrary.org/adl-402854/present-truth-august-16-1888?view_only=true&solr_nav%5Bid%5D=ff4d7f3f77b9bdf9e9ac&solr_nav%5Bpage%5D=0&solr_nav%5Boffset%5D=13}{PT18880816} (p. 250 - 252)


\leftsubsection{1889 - SDA Yearbook for 1889}


\leftsubsection{1889 - Annuaire ASD pour 1889}


Original scan: \href{https://documents.adventistarchives.org/Yearbooks/YB1889.pdf#search=Yearbook%201889}{YB1889} \href{https://forgotten-pillar.s3.us-east-2.amazonaws.com/YB1889.pdf}{*} (p. 145 - 151) Uriah Smith extended Fundamental Principles to 28 propositions. He added point on sanctification (point 14), dress reform (point 15) and tithing (point 16). Also he made small textual changes in some expressions, but semantics remained the same.


Scan original : \href{https://documents.adventistarchives.org/Yearbooks/YB1889.pdf#search=Yearbook%201889}{YB1889} \href{https://forgotten-pillar.s3.us-east-2.amazonaws.com/YB1889.pdf}{*} (p. 145 - 151) Uriah Smith a étendu les Principes Fondamentaux à 28 propositions. Il a ajouté un point sur la sanctification (point 14), la réforme vestimentaire (point 15) et la dîme (point 16). Il a également apporté de petits changements textuels dans certaines expressions, mais la sémantique est restée la même.


\leftsubsection{1897 - Words of Truth - no. 5}


\leftsubsection{1897 - Paroles de Vérité - no. 5}


Original scan: \href{https://adl.b2.adventistdigitallibrary.org/concern/published_works/4ffda25e-a06b-48d4-8ace-67cdcd33726f}{WoT no.5}
Word of Truth was a series of pamphlets with \href{https://adl.b2.adventistdigitallibrary.org/concern/parent/22267078_fundamental_principles_of_seventh_day_adventists/published_works/94a22141-33e8-4b9a-b397-2fe48c17bec4}{29 sections}.


Scan original : \href{https://adl.b2.adventistdigitallibrary.org/concern/published_works/4ffda25e-a06b-48d4-8ace-67cdcd33726f}{WoT no.5}
Paroles de Vérité était une série de brochures avec \href{https://adl.b2.adventistdigitallibrary.org/concern/parent/22267078_fundamental_principles_of_seventh_day_adventists/published_works/94a22141-33e8-4b9a-b397-2fe48c17bec4}{29 sections}.


\leftsubsection{1905 - SDA Yearbook for 1905}


\leftsubsection{1905 - Annuaire ASD pour 1905}


Original scan: \href{https://documents.adventistarchives.org/Yearbooks/YB1905.pdf#search=Yearbook%201905}{YB1905} \href{https://forgotten-pillar.s3.us-east-2.amazonaws.com/YB1905.pdf}{*} (p. 188 - 192)


Scan original : \href{https://documents.adventistarchives.org/Yearbooks/YB1905.pdf#search=Yearbook%201905}{YB1905} \href{https://forgotten-pillar.s3.us-east-2.amazonaws.com/YB1905.pdf}{*} (p. 188 - 192)


\leftsubsection{1907 - SDA Yearbook for 1907}


\leftsubsection{1907 - Annuaire ASD pour 1907}


Original scan: \href{https://documents.adventistarchives.org/Yearbooks/YB1907.pdf#search=Yearbook%201906}{YB1907} \href{https://forgotten-pillar.s3.us-east-2.amazonaws.com/YB1907.pdf}{*} (p. 175 - 179)


Scan original : \href{https://documents.adventistarchives.org/Yearbooks/YB1907.pdf#search=Yearbook%201906}{YB1907} \href{https://forgotten-pillar.s3.us-east-2.amazonaws.com/YB1907.pdf}{*} (p. 175 - 179)


\leftsubsection{1908 - SDA Yearbook for 1908}


\leftsubsection{1908 - Annuaire ASD pour 1908}


Original scan: \href{https://documents.adventistarchives.org/Yearbooks/YB1908.pdf#search=Yearbook%201906}{YB1908} \href{https://forgotten-pillar.s3.us-east-2.amazonaws.com/YB1908.pdf}{*} (p. 213 - 217)


Scan original : \href{https://documents.adventistarchives.org/Yearbooks/YB1908.pdf#search=Yearbook%201906}{YB1908} \href{https://forgotten-pillar.s3.us-east-2.amazonaws.com/YB1908.pdf}{*} (p. 213 - 217)


\leftsubsection{1909 - SDA Yearbook for 1909}


\leftsubsection{1909 - Annuaire AJS pour 1909}


Original scan: \href{https://documents.adventistarchives.org/Yearbooks/YB1909.pdf#search=Yearbook%201909}{YB1909} \href{https://forgotten-pillar.s3.us-east-2.amazonaws.com/YB1909.pdf}{*} (p. 220 - 224)


Scan original : \href{https://documents.adventistarchives.org/Yearbooks/YB1909.pdf#search=Yearbook%201909}{YB1909} \href{https://forgotten-pillar.s3.us-east-2.amazonaws.com/YB1909.pdf}{*} (p. 220 - 224)


\leftsubsection{1910 - SDA Yearbook for 1910}


\leftsubsection{1910 - Annuaire AJS pour 1910}


Original scan: \href{https://documents.adventistarchives.org/Yearbooks/YB1910.pdf#search=Yearbook%201910}{YB1910} \textbf{\href{https://forgotten-pillar.s3.us-east-2.amazonaws.com/YB1910.pdf}{*}} (p. 224 - 228)


Scan original : \href{https://documents.adventistarchives.org/Yearbooks/YB1910.pdf#search=Yearbook%201910}{YB1910} \textbf{\href{https://forgotten-pillar.s3.us-east-2.amazonaws.com/YB1910.pdf}{*}} (p. 224 - 228)


\leftsubsection{1911 - SDA Yearbook for 1911}


\leftsubsection{1911 - Annuaire AJS pour 1911}


Original scan: \href{https://documents.adventistarchives.org/Yearbooks/YB1911.pdf#search=Yearbook%201910}{YB1911} \href{https://forgotten-pillar.s3.us-east-2.amazonaws.com/YB1911.pdf}{*} (p. 223 - 227)


Scan original : \href{https://documents.adventistarchives.org/Yearbooks/YB1911.pdf#search=Yearbook%201910}{YB1911} \href{https://forgotten-pillar.s3.us-east-2.amazonaws.com/YB1911.pdf}{*} (p. 223 - 227)


\leftsubsection{1912 - Advent Review and Sabbath Herald, August 22, 1912}


\leftsubsection{1912 - Advent Review and Sabbath Herald, 22 août 1912}


Original scan: \href{https://adventistdigitallibrary.org/adl-351682/advent-review-and-sabbath-herald-august-22-1912?view_only=true&solr_nav%5Bid%5D=ff4d7f3f77b9bdf9e9ac&solr_nav%5Bpage%5D=0&solr_nav%5Boffset%5D=15}{RH19120822} (p. 4 - 6)


Scan original : \href{https://adventistdigitallibrary.org/adl-351682/advent-review-and-sabbath-herald-august-22-1912?view_only=true&solr_nav%5Bid%5D=ff4d7f3f77b9bdf9e9ac&solr_nav%5Bpage%5D=0&solr_nav%5Boffset%5D=15}{RH19120822} (p. 4 - 6)


\leftsubsection{1912 - SDA Yearbook for 1912}


\leftsubsection{1912 - Annuaire AJS pour 1912}


Original scan: \href{https://documents.adventistarchives.org/Yearbooks/YB1912.pdf#search=Yearbook%201910}{YB1912} \href{https://forgotten-pillar.s3.us-east-2.amazonaws.com/YB1912.pdf}{*} (p. 261 - 265)


Scan original : \href{https://documents.adventistarchives.org/Yearbooks/YB1912.pdf#search=Yearbook%201910}{YB1912} \href{https://forgotten-pillar.s3.us-east-2.amazonaws.com/YB1912.pdf}{*} (p. 261 - 265)


\leftsubsection{1913 - SDA Yearbook for 1913}


\leftsubsection{1913 - Annuaire ASD pour 1913}


Original scan: \href{https://documents.adventistarchives.org/Yearbooks/YB1913.pdf#search=Yearbook%201913}{YB1913} \href{https://forgotten-pillar.s3.us-east-2.amazonaws.com/YB1913.pdf}{*} (p. 281 -285 )


Scan original : \href{https://documents.adventistarchives.org/Yearbooks/YB1913.pdf#search=Yearbook%201913}{YB1913} \href{https://forgotten-pillar.s3.us-east-2.amazonaws.com/YB1913.pdf}{*} (p. 281 -285 )


\leftsubsection{1914 - SDA Yearbook for 1914}
Original scan: \href{https://documents.adventistarchives.org/Yearbooks/YB1914.pdf#search=Yearbook%201914}{YB1914} \href{https://forgotten-pillar.s3.us-east-2.amazonaws.com/YB1914.pdf}{*} (p. 293 - 297)


\leftsubsection{1914 - Annuaire ASD pour 1914}
Scan original : \href{https://documents.adventistarchives.org/Yearbooks/YB1914.pdf#search=Yearbook%201914}{YB1914} \href{https://forgotten-pillar.s3.us-east-2.amazonaws.com/YB1914.pdf}{*} (p. 293 - 297)


\section*{Unauthenticated reports in Ellen White writings}


\section*{Rapports non authentifiés dans les écrits d'Ellen White}


\label{appendix:unauthenticated-reports}
We would like to present to you one Ellen White quotation that challenges the conclusion on the personality of the Holy Spirit. In this study, we have seen that the Holy Spirit is a spirit and not a being. In studying the \emcap{personality of God} and where His presence is, we have seen the distinction between the terms ‘being’ and ‘spirit’. We came to the conclusion that the Father and the Son are two distinct beings, thus constrained in space, while the Holy Spirit is a spirit, a means by which the Father and Son are everywhere present.


\label{appendix:unauthenticated-reports}
Nous aimerions vous présenter une citation d'Ellen White qui remet en question la conclusion sur la personnalité du Saint-Esprit. Dans cette étude, nous avons vu que le Saint-Esprit est un esprit et non un être. En étudiant la \emcap{personnalité de Dieu} et où se trouve Sa présence, nous avons vu la distinction entre les termes « être » et « esprit ». Nous sommes arrivés à la conclusion que le Père et le Fils sont deux êtres distincts, donc contraints dans l'espace, tandis que le Saint-Esprit est un esprit, un moyen par lequel le Père et le Fils sont partout présents.


The following quotation testifies that the Holy Spirit is also a being, just as the Father and Son are:


La citation suivante témoigne que le Saint-Esprit est aussi un être, tout comme le Père et le Fils le sont :


\egw{Here is where the work of the Holy Ghost comes in, after your baptism. You are baptized in the name of \textbf{the Father, of the Son, and of the Holy Ghost}. You are raised up out of the water to live henceforth in newness of life—to live a new life. You are born unto God, and you stand under the sanction and \textbf{the power of the three holiest \underline{beings} in heaven}, who are able to keep you from falling.}[Ms95-1906.29; 1906][https://egwwritings.org/read?panels=p8872.35]


\egw{C'est ici qu'intervient l'œuvre du Saint-Esprit, après votre baptême. Vous êtes baptisés au nom \textbf{du Père, du Fils et du Saint-Esprit}. Vous êtes relevés hors de l'eau pour vivre désormais en nouveauté de vie — pour vivre une vie nouvelle. Vous êtes nés à Dieu, et vous vous tenez sous la sanction et \textbf{la puissance des trois \underline{êtres} les plus saints dans le ciel}, qui sont capables de vous garder de tomber.}[Ms95-1906.29; 1906][https://egwwritings.org/read?panels=p8872.35]


Many have come across this quotation and presented it as proof that the Holy Spirit is a being rather than a spirit. In the following, we present our concerns.


Beaucoup ont rencontré cette citation et l'ont présentée comme preuve que le Saint-Esprit est un être plutôt qu'un esprit. Dans ce qui suit, nous présentons nos préoccupations.


The source of this quotation is Manuscript 95, 1906.


La source de cette citation est le Manuscrit 95, 1906.


This quotation is actually a report from the sermon Sister White held in Oakland, California, on Sabbath afternoon, October 20, 1906. Many of Ellen White’s public sermons were stenographically reported and later rewritten for publication. When Sister White preached, she never had a written sermon. There were no tape recorders at that time that could accurately document word for word. The only reference we have from that time is the report by the stenographer. This opens the possibility for human error in reporting, or later editing, prior to publication. The plethora of evidence presented in this book makes it clear that this statement is not in harmony with the authenticated quotations. Plainly stated, it’s obvious that a mistake was made in the report of this sermon.


Cette citation est en fait un rapport du sermon que Sœur White a tenu à Oakland, en Californie, le sabbat après-midi du 20 octobre 1906. Beaucoup de sermons publics d'Ellen White ont été rapportés sténographiquement et plus tard réécrits pour publication. Quand Sœur White prêchait, elle n'avait jamais de sermon écrit. Il n'y avait pas de magnétophones à cette époque qui pouvaient documenter avec précision mot pour mot. La seule référence que nous avons de cette époque est le rapport du sténographe. Cela ouvre la possibilité d'une erreur humaine dans le rapport, ou d'une édition ultérieure, avant la publication. La pléthore de preuves présentées dans ce livre montre clairement que cette déclaration n'est pas en harmonie avec les citations authentifiées. Dit simplement, il est évident qu'une erreur a été commise dans le rapport de ce sermon.


In order to clear any such mistakes for the future generations, Sister White actually warns us when it comes to unauthenticated reports of what she may have said.


Afin d'éviter de telles erreurs pour les générations futures, Sœur White nous met en garde concernant les rapports non authentifiés de ce qu'elle aurait pu dire.


\egw{And now to all who have a desire for truth I would say: \textbf{Do not give credence to \underline{unauthenticated reports} as to what Sister White has done or said or written}. If you desire to know what the Lord has revealed through her, \textbf{read her published works}. Are there any points of interest concerning which she has not written, do not eagerly catch up and report rumors as to what she has said.}[5T 696.1; 1889][https://egwwritings.org/read?panels=p113.3386]


\egw{Et maintenant, à tous ceux qui désirent la vérité, je dirais : \textbf{Ne donnez pas crédit aux \underline{rapports non authentifiés} concernant ce que Sœur White a fait, dit ou écrit}. Si vous désirez savoir ce que le Seigneur a révélé à travers elle, \textbf{lisez ses œuvres publiées}. Y a-t-il des points d'intérêt sur lesquels elle n'a pas écrit, ne saisissez pas avidement et ne rapportez pas les rumeurs sur ce qu'elle a dit.}[5T 696.1; 1889][https://egwwritings.org/read?panels=p113.3386]


The published works of Ellen White during her life represent the accurate and authentic material from Sister White. The process of publication ensured that the final product was genuine. The weight of the evidence is that Sister White herself was involved in the process of the publishing and she would review manuscripts prior to printing.


Les œuvres publiées d'Ellen White durant sa vie représentent le matériel exact et authentique de Sœur White. Le processus de publication assurait que le produit final était authentique. Le poids de la preuve est que Sœur White elle-même était impliquée dans le processus de publication et qu'elle révisait les manuscrits avant l'impression.


\egw{I read over all that is copied, to see that everything is as it should be. I read all the book manuscript before it is sent to the printer.}[Lt133-1902.4; 1902][https://egwwritings.org/read?panels=p9791.10]


\egw{Je relis tout ce qui est copié, pour voir que tout est comme il se doit. Je lis tout le manuscrit du livre avant qu'il ne soit envoyé à l'imprimeur.}[Lt133-1902.4; 1902][https://egwwritings.org/read?panels=p9791.10]


\egw{I have all my publications closely examined. I desire that nothing shall appear in print without careful investigation.}[Lt49-1894.11; 1894][https://egwwritings.org/read?panels=p5289.20]


\egw{Je fais examiner attentivement toutes mes publications. Je désire que rien ne paraisse en impression sans une investigation minutieuse.}[Lt49-1894.11; 1894][https://egwwritings.org/read?panels=p5289.20]


The statement that the Holy Spirit is a being was not part of the process of publishing because this statement appeared after the death of Ellen White. Thus, it is not authenticated. It does not belong to her “\textit{published work}”. We do not seek any conspiracy in this; we’re simply adhering to Ellen White’s own suggestion to not give credence to these reports. In 1990, Ellen White Estate published the collection of her sermons and talks and in 2015, they included the sermons and talks into the files of her Manuscripts. We do not understand why they did that since the sermons and talks do not contain manuscripts from Ellen White, but from some stenographers. Nevertheless, above every manuscript the EGW Estate annotated its source, whether a sermon or letter. This tells us if the quotation is authenticated or not.


La déclaration que le Saint-Esprit est un être ne faisait pas partie du processus de publication car cette déclaration est apparue après la mort d'Ellen White. Ainsi, elle n'est pas authentifiée. Elle n'appartient pas à ses “\textit{œuvres publiées}”. Nous ne cherchons aucune conspiration en cela ; nous adhérons simplement à la propre suggestion d'Ellen White de ne pas donner crédit à ces rapports. En 1990, Ellen White Estate a publié la collection de ses sermons et discours et en 2015, ils ont inclus les sermons et discours dans les fichiers de ses Manuscrits. Nous ne comprenons pas pourquoi ils ont fait cela puisque les sermons et discours ne contiennent pas de manuscrits d'Ellen White, mais de certains sténographes. Néanmoins, au-dessus de chaque manuscrit, l'EGW Estate a annoté sa source, qu'il s'agisse d'un sermon ou d'une lettre. Cela nous indique si la citation est authentifiée ou non.


\begin{figure}
    \centering
    \includegraphics[width=1\linewidth]{images/sermons-and-talks.png}
    \label{fig:enter-label}
\end{figure}


\begin{figure}
    \centering
    \includegraphics[width=1\linewidth]{images/sermons-and-talks.png}
    \label{fig:enter-label}
\end{figure}


For us, personally, these quotations are unauthenticated and, especially, invalid compared to Ellen White’s authenticated works. But if someone insists on weighing her unconfirmed reports and published writings equally, we will not stand in their way but even further push the conclusion of the Holy Spirit as a being. Let’s follow together.


Pour nous, personnellement, ces citations sont non authentifiées et, surtout, invalides comparées aux œuvres authentifiées d'Ellen White. Mais si quelqu'un insiste pour peser également ses rapports non confirmés et ses écrits publiés, nous ne nous mettrons pas en travers de leur chemin mais pousserons même plus loin la conclusion du Saint-Esprit comme un être. Suivons ensemble.


Even compared with Ellen White’s authenticated works, such a Holy Spirit, a being, would not be one with God because Christ was \egwinline{\textbf{The only being who was one with God}}[Lt121-1897.7; 1897][https://egwwritings.org/read?panels=p7266.13]. This Holy Spirit, a being, could not \egwinline{\textbf{enter into all the counsels and purposes of God}}, because Christ was \egwinline{\textbf{the only being}}[PP 34.1; 1890][https://egwwritings.org/read?panels=p84.75] who could do that. This Being is not to be exalted because \egwinline{\textbf{The Father and the Son \underline{alone} are to be exalted}}[YI, July 7, 1898 par.2.; 1898][https://egwwritings.org/read?panels=p469.2964]. The Holy Spirit, as a being, would not fit in the order of heaven as the third being because Satan was \egwinline{\textbf{next to Christ the most exalted \underline{being}} in the heavenly courts}[RH August 9, 1898, par. 7; 1898][https://egwwritings.org/read?panels=p821.17145]. This Holy Spirit, a being, was not invested in the cost of salvation; neither was he in the covenant with Father and Son to save the world, nor dishonored by man’s transgression.


Même comparé aux œuvres authentifiées d'Ellen White, un tel Saint-Esprit, un être, ne serait pas un avec Dieu parce que Christ était \egwinline{\textbf{Le seul être qui était un avec Dieu}}[Lt121-1897.7; 1897][https://egwwritings.org/read?panels=p7266.13]. Ce Saint-Esprit, un être, ne pourrait pas \egwinline{\textbf{entrer dans tous les conseils et desseins de Dieu}}, parce que Christ était \egwinline{\textbf{le seul être}}[PP 34.1; 1890][https://egwwritings.org/read?panels=p84.75] qui pouvait le faire. Cet Être ne doit pas être exalté parce que \egwinline{\textbf{Le Père et le Fils \underline{seuls} doivent être exaltés}}[YI, July 7, 1898 par.2.; 1898][https://egwwritings.org/read?panels=p469.2964]. Le Saint-Esprit, comme un être, ne s'intégrerait pas dans l'ordre du ciel comme le troisième être parce que Satan était \egwinline{\textbf{après Christ l’\underline{être} le plus exalté} dans les cours célestes}[RH August 9, 1898, par. 7; 1898][https://egwwritings.org/read?panels=p821.17145]. Ce Saint-Esprit, un être, n'était pas investi dans le coût du salut ; il n'était pas non plus dans l'alliance avec le Père et le Fils pour sauver le monde, ni déshonoré par la transgression de l'homme.


\egwinline{The great gift of salvation has been placed within our reach at an \textbf{infinite cost to the Father and the Son}.}[RH November 21, 1912, par. 2; 1912][https://egwwritings.org/read?panels=p821.33329]


\egwinline{Le grand don du salut a été placé à notre portée à un \textbf{coût infini pour le Père et le Fils}.}[RH November 21, 1912, par. 2; 1912][https://egwwritings.org/read?panels=p821.33329]


\egwinline{In the plan to save a lost world, the counsel was between them \textbf{\underline{both}}; \textbf{the covenant of peace was between the Father and the Son}.}[ST December 23, 1897, par. 2; 1897][https://egwwritings.org/read?panels=p820.14803]


\egwinline{Dans le plan pour sauver un monde perdu, le conseil était entre eux \textbf{\underline{deux}} ; \textbf{l'alliance de paix était entre le Père et le Fils}.}[ST December 23, 1897, par. 2; 1897][https://egwwritings.org/read?panels=p820.14803]


\egwinline{But in the transgression of man \textbf{\underline{both} the Father and the Son were dishonored}.}[ST December 12, 1895, par. 7; 1895][https://egwwritings.org/read?panels=p820.13243]


\egwinline{Mais dans la transgression de l'homme, \textbf{\underline{tous les deux}, le Père et le Fils, furent déshonorés}.}[ST December 12, 1895, par. 7; 1895][https://egwwritings.org/read?panels=p820.13243]


Such a Holy Spirit, a being, does not fit into harmony with the authenticated reports of Ellen White, nor with the Scriptures. The Holy Spirit is called ‘\textit{spirit}’, so it is a spirit, exclusively.


Un tel Saint-Esprit, un être, ne s'harmonise pas avec les rapports authentifiés d'Ellen White, ni avec les Écritures. Le Saint-Esprit est appelé ‘\textit{esprit}’, donc c'est un esprit, exclusivement.


Many of Sister White’s quotations are sourced from sermons or talks that were published after her death. In what follows, we will present a few that are most often discussed in an effort to prove that Sister White was a trinitarian. We invite everyone to weigh these quotations with her authenticated and published work, those during her lifetime.


Beaucoup de citations de Sœur White proviennent de sermons ou de discours qui furent publiés après sa mort. Dans ce qui suit, nous présenterons quelques-unes qui sont le plus souvent discutées dans un effort pour prouver que Sœur White était trinitaire. Nous invitons chacun à peser ces citations avec son œuvre authentifiée et publiée, celles de son vivant.


“\textit{And then the golden harps are touched, and the music flows all through the heavenly host, and they fall down and worship the Father and the Son and the Holy Spirit}.”\footnote{\href{https://egwwritings.org/?ref=en_Ms139-1906.32&para=9579.38}{EGW; Ms139-1906.32; 1906}} [Sermon/Thoughts on Matthew 4. Oakland, California July 24, 1906; Previously unpublished.]


« \textit{Et alors les harpes d'or sont touchées, et la musique coule à travers toute l'armée céleste, et ils tombent et adorent le Père et le Fils et le Saint-Esprit}. »\footnote{\href{https://egwwritings.org/?ref=en_Ms139-1906.32&para=9579.38}{EGW; Ms139-1906.32; 1906}} [Sermon/Pensées sur Matthieu 4. Oakland, Californie 24 juillet 1906 ; Précédemment non publié.]


“\textit{We need to realize that the Holy Spirit, who is as much a person as God is a person, is walking through these grounds.}”\footnote{\href{https://egwwritings.org/?ref=en_Ms66-1899.11&para=6622.19}{EGW; Ms66-1899.11: 1899}} [Talk/Extracts From Talks Given by Mrs. E. G. White at the Opening of College Hall, Avondale, and in the Avondale Church]


« \textit{Nous devons réaliser que le Saint-Esprit, qui est autant une personne que Dieu est une personne, marche à travers ces lieux}. »\footnote{\href{https://egwwritings.org/?ref=en_Ms66-1899.11&para=6622.19}{EGW; Ms66-1899.11: 1899}} [Discours/Extraits de discours donnés par Mme E. G. White à l'ouverture du College Hall, Avondale, et dans l'église d'Avondale]
