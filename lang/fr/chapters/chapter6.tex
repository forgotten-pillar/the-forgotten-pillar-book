\chapter{Examining the test}


\chapter{Examen du test}


In Sister White's reply to Dr. Kellogg's belief on the Trinity doctrine and his attempts to \textit{patch up} the Living Temple, we see that she viewed the Trinity doctrine as contradicting the light given her regarding \emcap{the personality of God}. If she had actually embraced the Trinity doctrine, we would expect her to carefully separate it from pantheism and preserve its legitimate aspects. However, this is not what we see in her response. Instead, her response was to contrast the Trinity doctrine with the truth about the \emcap{personality of God}, recalling her past visions which showed that this doctrine would rob God's people of their past experiences. In her reactive recalling of how God established the \emcap{fundamental principles}, she indicated that the Trinity doctrine \textit{tears down the pillars of our faith} and \textit{leads us astray from the foundation principles}. This stark difference can be clearly seen by comparing our current Fundamental Beliefs with the \emcap{Fundamental Principles} held in the past.


Dans la réponse de Sœur White à la croyance du Dr Kellogg sur la doctrine de la Trinité et ses tentatives de \textit{rapiécer} le Temple Vivant, nous voyons qu'elle considérait la doctrine de la Trinité comme contredisant la lumière qui lui avait été donnée concernant \emcap{la personnalité de Dieu}. Si elle avait réellement adopté la doctrine de la Trinité, nous nous attendrions à ce qu'elle la sépare soigneusement du panthéisme et en préserve les aspects légitimes. Cependant, ce n'est pas ce que nous voyons dans sa réponse. Au contraire, sa réponse fut de contraster la doctrine de la Trinité avec la vérité sur \emcap{la personnalité de Dieu}, rappelant ses visions passées qui montraient que cette doctrine priverait le peuple de Dieu de ses expériences passées. Dans son rappel réactif de la façon dont Dieu a établi \emcap{les principes fondamentaux}, elle a indiqué que la doctrine de la Trinité \textit{détruit les piliers de notre foi} et \textit{nous égare des principes fondamentaux}. Cette différence marquée peut être clairement vue en comparant nos Croyances Fondamentales actuelles avec les \emcap{Principes Fondamentaux} tenus dans le passé.


Keeping in mind Sister White’s reply to Dr. Kellogg's belief on the Trinity doctrine, let us review the characteristics of the theories she described in the chapter “\textit{The Foundation of our Faith}”. When Sister White is speaking of Kellogg’s theories of God, our question should be, “do her quotations make sense if the Trinity doctrine is applied to their context?” Let’s examine each characteristic.


En gardant à l'esprit la réponse de Sœur White à la croyance du Dr Kellogg sur la doctrine de la Trinité, examinons les caractéristiques des théories qu'elle a décrites dans le chapitre “\textit{Le Fondement de notre Foi}”. Lorsque Sœur White parle des théories de Kellogg sur Dieu, notre question devrait être : “ses citations ont-elles un sens si la doctrine de la Trinité est appliquée à leur contexte ?” Examinons chaque caractéristique.


\subsection*{Does the Trinity “rob the people of God of their past experience”?}


\subsection*{La Trinité “prive-t-elle le peuple de Dieu de son expérience passée” ?}


\egw{They \normaltext{[the spiritualistic theories]} make of no effect the truth of heavenly origin, and \textbf{rob the people of God of their past experience}, giving them instead a false science.}[SpTB02 54.1; 1904][https://egwwritings.org/?ref=en\_SpTB02.54.1]


\egw{Elles \normaltext{[les théories spiritualistes]} rendent sans effet la vérité d'origine céleste et \textbf{privent le peuple de Dieu de son expérience passée}, leur donnant à la place une fausse science.}[SpTB02 54.1; 1904][https://egwwritings.org/?ref=en\_SpTB02.54.1]


\egw{This foundation was built by the Masterworker, and will stand storm and tempest. Will they permit this man \normaltext{[Kellogg]} to present \textbf{doctrines that deny the past experience of the people of God}? The time has come to take decided action.}[SpTB02 54.2; 1904][https://egwwritings.org/?ref=en\_SpTB02.54.2]


\egw{Ce fondement a été construit par le Maître d'œuvre, et résistera à la tempête et à l'orage. Permettront-ils à cet homme \normaltext{[Kellogg]} de présenter des \textbf{doctrines qui nient l'expérience passée du peuple de Dieu} ? Le temps est venu de prendre des mesures décisives.}[SpTB02 54.2; 1904][https://egwwritings.org/?ref=en\_SpTB02.54.2]


\egw{\textbf{What influence is it that would lead men at this stage of our history to work in an underhanded, powerful way to \underline{tear down the foundation of our faith},—the foundation that was laid \underline{at the beginning of our work} by prayerful study of the word and by revelation? Upon this foundation \underline{we have been building for the past fifty years}}. Do you wonder that when I see the beginning of a work \textbf{that would \underline{remove some of the pillars of our faith},} I have something to say? I must obey the command, ‘Meet it!’}[SpTB02 58.1; 1904][https://egwwritings.org/?ref=en\_SpTB02.58.1]


\egw{\textbf{Quelle influence est-ce qui conduirait des hommes à ce stade de notre histoire à travailler d'une manière sournoise et puissante pour \underline{détruire le fondement de notre foi},—le fondement qui a été posé \underline{au début de notre œuvre} par l'étude priante de la parole et par révélation ? Sur ce fondement \underline{nous avons construit pendant les cinquante dernières années}}. Vous étonnez-vous que lorsque je vois le début d'une œuvre qui \textbf{\underline{enlèverait certains des piliers de notre foi},} j'aie quelque chose à dire ? Je dois obéir au commandement, ‘Affrontez-le !’}[SpTB02 58.1; 1904][https://egwwritings.org/?ref=en\_SpTB02.58.1]


According to Sister White’s testimony, the foundation of our faith was the \emcap{Fundamental Principles}. Currently, they do not represent our beliefs. Most objectionable is the first point, concerning who God is. Instead of the belief that there is one God—the Father, a personal spiritual being, we have a new belief that there is one God—Father, Son, and Holy Spirit, a unity of three Persons. From the light and the experiences of how God established the first point of the \emcap{Fundamental Principles}, does the newly formed doctrine about who God is and what He is, robbed the people of God of their past experience?


Selon le témoignage de Sœur White, le fondement de notre foi était les \emcap{Principes Fondamentaux}. Actuellement, ils ne représentent plus nos croyances. Le plus contestable est le premier point, concernant qui est Dieu. Au lieu de la croyance qu'il y a un seul Dieu—le Père, un être spirituel personnel, nous avons une nouvelle croyance qu'il y a un seul Dieu—Père, Fils et Saint-Esprit, une unité de trois Personnes. À la lumière des expériences de la façon dont Dieu a établi le premier point des \emcap{Principes Fondamentaux}, la doctrine nouvellement formée sur qui est Dieu et ce qu'Il est, a-t-elle privé le peuple de Dieu de son expérience passée ?


\subsection*{Does the Trinity tear down the pillars of our faith, or lead astray from foundation principles?}


\subsection*{La Trinité détruit-elle les piliers de notre foi ou égare-t-elle des principes fondamentaux ?}


\egw{I have been instructed by the heavenly messenger that some of the reasoning in the book, ‘Living Temple,’ is unsound and that \textbf{this reasoning would lead astray the minds of those who are not thoroughly established on the foundation principles of present truth.}}[SpTB02 51.3; 1904][https://egwwritings.org/?ref=en\_SpTB02.51.3]


\egw{J'ai reçu l'instruction du messager céleste que certains des raisonnements dans le livre ‘Le Temple Vivant’ sont erronés et que \textbf{ce raisonnement égarerait les esprits de ceux qui ne sont pas fermement établis sur les principes fondamentaux de la vérité présente.}}[SpTB02 51.3; 1904][https://egwwritings.org/?ref=en\_SpTB02.51.3]


\egw{About the time that ‘Living Temple’ was published, there passed before me in the night season, representations indicating that some \textbf{danger was approaching}, and that I must prepare for it by writing out the things God has revealed to me \textbf{regarding the foundation principles of our faith}.}[SpTB02 52.3; 1904][https://egwwritings.org/?ref=en\_SpTB02.52.3]


\egw{À l'époque où ‘Le Temple Vivant’ a été publié, des représentations m'ont été montrées pendant la nuit, indiquant qu'un \textbf{danger approchait}, et que je devais m'y préparer en écrivant les choses que Dieu m'avait révélées \textbf{concernant les principes fondamentaux de notre foi}.}[SpTB02 52.3; 1904][https://egwwritings.org/?ref=en\_SpTB02.52.3]


\egw{\textbf{The enemy of souls has sought to bring in the supposition that a great reformation was to take place among Seventh-day Adventists, and that this reformation would consist in \underline{giving up the doctrines which stand as the pillars of our faith,} and engaging in a process of reorganization}. Were this reformation to take place, what would result? \textbf{The principles of truth} that God in His wisdom has given to the remnant church, \textbf{would be discarded}. Our religion would be changed. \textbf{The fundamental principles} that have sustained the work for the last fifty years \textbf{would be accounted as error}. A new organization would be established. Books of a new order would be written. A system of intellectual philosophy would be introduced.}[SpTB02 54.3; 1904][https://egwwritings.org/?ref=en\_SpTB02.54.3]


\egw{\textbf{L'ennemi des âmes a cherché à introduire l'idée qu'une grande réforme devait avoir lieu parmi les Adventistes du Septième Jour, et que cette réforme consisterait à \underline{abandonner les doctrines qui constituent les piliers de notre foi,} et à s'engager dans un processus de réorganisation}. Si cette réforme devait avoir lieu, quel en serait le résultat ? \textbf{Les principes de vérité} que Dieu, dans Sa sagesse, a donnés à l'église du reste, \textbf{seraient rejetés}. Notre religion serait changée. \textbf{Les principes fondamentaux} qui ont soutenu l'œuvre durant les cinquante dernières années \textbf{seraient considérés comme une erreur}. Une nouvelle organisation serait établie. Des livres d'un nouveau genre seraient écrits. Un système de philosophie intellectuelle serait introduit.}[SpTB02 54.3; 1904][https://egwwritings.org/?ref=en\_SpTB02.54.3]


Dr. Kellogg’s theories on the \emcap{personality of God}, if accepted, would ignite a reformation within the Seventh-day Adventist Church. Based on intellectual philosophy, they would cause us to renounce some of the doctrines that stand as the pillars of our faith, condemning the \emcap{Fundamental Principles} as error. Could it be that by adhering to the Trinity doctrine we entered into a new organization?


Les théories du Dr Kellogg sur la \emcap{personnalité de Dieu}, si elles étaient acceptées, déclencheraient une réforme au sein de l'Église Adventiste du Septième Jour. Basées sur la philosophie intellectuelle, elles nous amèneraient à renoncer à certaines des doctrines qui constituent les piliers de notre foi, condamnant les \emcap{Principes Fondamentaux} comme une erreur. Se pourrait-il qu'en adhérant à la doctrine de la Trinité, nous soyons entrés dans une nouvelle organisation ?


\egw{Shortly before I sent out the testimonies \textbf{regarding the efforts of the enemy to undermine the foundation of our faith through the dissemination of seductive theories}, I had read an incident about a ship in a fog meeting an iceberg…}[SpTB02 55.3; 1904][https://egwwritings.org/?ref=en\_SpTB02.55.3]


\egw{Peu avant que j'envoie les témoignages \textbf{concernant les efforts de l'ennemi pour saper le fondement de notre foi par la diffusion de théories séduisantes}, j'avais lu un incident à propos d'un navire dans le brouillard rencontrant un iceberg...}[SpTB02 55.3; 1904][https://egwwritings.org/?ref=en\_SpTB02.55.3]


\egw{Messages of every order and kind have been \textbf{urged upon Seventh-day Adventists, to take the place of the truth which, \underline{point by point}, has been sought out by prayerful study, and testified to by the miracle-working power of the Lord}. \textbf{But the way-marks which have made us what we are, are to be preserved, and they will be preserved}, as God has signified through His word and the testimony of His Spirit. \textbf{He calls upon us to hold firmly}, with the grip of faith, \textbf{to \underline{the fundamental principles} that are based upon \underline{unquestionable authority}}.}[SpTB02 59.1; 1904][https://egwwritings.org/?ref=en\_SpTB02.59.1]


\egw{Des messages de tout ordre et de toute nature ont été \textbf{imposés aux Adventistes du Septième Jour, pour prendre la place de la vérité qui, \underline{point par point}, a été recherchée par une étude dans la prière, et attestée par la puissance miraculeuse du Seigneur}. \textbf{Mais les repères qui ont fait de nous ce que nous sommes, doivent être préservés, et ils seront préservés}, comme Dieu l'a signifié par Sa parole et le témoignage de Son Esprit. \textbf{Il nous appelle à tenir fermement}, avec la force de la foi, \textbf{aux \underline{principes fondamentaux} qui sont basés sur une \underline{autorité incontestable}}.}[SpTB02 59.1; 1904][https://egwwritings.org/?ref=en\_SpTB02.59.1]


The \emcap{personality of God} was the pillar of our faith\footnote{\href{https://egwwritings.org/?ref=en_Ms62-1905.14}{EGW, Ms62-1905.14; 1905}}. The \emcap{personality of God} was expressed in the first point of the \emcap{Fundamental Principles}. Could it be that by adherence to the Trinity doctrine we have torn down this particular pillar of our faith? Is it possible that by accepting the Trinity doctrine we were led astray from this foundation principle—the \emcap{personality of God}?


La \emcap{personnalité de Dieu} était le pilier de notre foi\footnote{\href{https://egwwritings.org/?ref=en_Ms62-1905.14}{EGW, Ms62-1905.14; 1905}}. La \emcap{personnalité de Dieu} était exprimée dans le premier point des \emcap{Principes Fondamentaux}. Se pourrait-il qu'en adhérant à la doctrine de la Trinité, nous ayons démoli ce pilier particulier de notre foi ? Est-il possible qu'en acceptant la doctrine de la Trinité, nous ayons été détournés de ce principe fondamental—la \emcap{personnalité de Dieu} ?


\subsection*{Does the Trinity do away with the personality of God?}


\subsection*{La Trinité supprime-t-elle la personnalité de Dieu ?}


\egw{\textbf{It \normaltext{[The Living Temple]} introduces that which is naught but \underline{speculation} in regard to \underline{the personality of God} and where His presence is.}}[SpTB02 51.3; 1904][https://egwwritings.org/?ref=en\_SpTB02.51.3]


\egw{\textbf{Il \normaltext{[Le Temple Vivant]} introduit ce qui n'est que \underline{spéculation} concernant \underline{la personnalité de Dieu} et où se trouve Sa présence.}}[SpTB02 51.3; 1904][https://egwwritings.org/?ref=en\_SpTB02.51.3]


\egw{\textbf{The spiritualistic theories \underline{regarding the personality of God}, followed to their logical conclusion, sweep away the whole Christian economy.}}[SpTB02 54.1; 1904][https://egwwritings.org/?ref=en\_SpTB02.54.1]


\egw{\textbf{Les théories spiritualistes \underline{concernant la personnalité de Dieu}, suivies jusqu'à leur conclusion logique, balaient toute l'économie chrétienne.}}[SpTB02 54.1; 1904][https://egwwritings.org/?ref=en\_SpTB02.54.1]


\egw{‘Living Temple’ contains the alpha of these theories. I knew that the omega would follow in a little while; and I trembled for our people. I knew that \textbf{I must warn our brethren and sisters not to enter into controversy over \underline{the presence} and \underline{personality of God}. The statements made in ‘Living Temple’ \underline{in regard to this point are incorrect}. The scripture used to substantiate the doctrine there set forth, is scripture misapplied}.}[SpTB02 53.2; 1904][https://egwwritings.org/?ref=en\_SpTB02.53.2]


\egw{‘Le Temple Vivant’ contient l'alpha de ces théories. Je savais que l'oméga suivrait peu après ; et je tremblais pour notre peuple. Je savais que \textbf{je devais avertir nos frères et sœurs de ne pas entrer dans une controverse sur \underline{la présence} et \underline{la personnalité de Dieu}. Les déclarations faites dans ‘Le Temple Vivant’ \underline{concernant ce point sont incorrectes}. L'écriture utilisée pour étayer la doctrine qui y est exposée est une écriture mal appliquée}.}[SpTB02 53.2; 1904][https://egwwritings.org/?ref=en\_SpTB02.53.2]


The theories Kellogg presented in the Living Temple are speculative in regard to the \emcap{personality of God} and where His presence is. These theories deal with the question of the quality or state of God being a person\footnote{The Merriam-Webster definition of ‘\textit{personality}’ - “\textit{the quality or state of being a person}”}. God has given us definite light regarding this issue in our \emcap{Fundamental Principles}. Could it be that the Trinity doctrine is casting doubt on this definite light regarding the \emcap{personality of God}?


Les théories que Kellogg a présentées dans Le Temple Vivant sont spéculatives en ce qui concerne la \emcap{personnalité de Dieu} et où se trouve Sa présence. Ces théories traitent de la question de la qualité ou l'état de Dieu comme étant une personne\footnote{La définition du Merriam-Webster de ‘\textit{personality}’ - “\textit{la qualité ou l'état par lequel quelqu'un est défini comme une personne}”}. Dieu nous a donné une lumière précise concernant cette question dans nos \emcap{Principes Fondamentaux}. Se pourrait-il que la doctrine de la Trinité jette le doute sur cette lumière précise concernant la \emcap{personnalité de Dieu} ?


\subsection*{Is the Trinity doctrine presented as if Mrs. White supported it?}


\subsection*{La doctrine de la Trinité est-elle présentée comme si Mme White la soutenait ?}


\egw{In the controversy that arose among our brethren \textbf{regarding the teachings of this book,} those in favor of giving it a wide circulation \textbf{declared: ‘It contains the very sentiments that Sister White has been teaching.’ This assertion struck right to my heart. I felt heart-broken; for I knew that this representation of the matter was not true}.}[SpTB02 53.1; 1904][https://egwwritings.org/?ref=en\_SpTB02.53.1]


\egw{Dans la controverse qui s'est élevée parmi nos frères \textbf{concernant les enseignements de ce livre,} ceux qui étaient favorables à lui donner une large diffusion \textbf{déclaraient : ‘Il contient exactement le raisonnement qu'Ellen White a enseigné.’ Cette affirmation m'a frappée droit au cœur. J'étais bouleversée, car je savais que cette représentation des choses n'était pas vraie}.}[SpTB02 53.1; 1904][https://egwwritings.org/?ref=en\_SpTB02.53.1]


\egw{\textbf{I am compelled to speak in denial of the claim that the teachings of ‘Living Temple’ can be sustained by statements from my writings}. There may be in this book expressions and sentiments that are in harmony with my writings. And there may be in my writings many statements which, taken from their connection, and interpreted according to the mind of the writer of ‘Living Temple,’ would seem to be in harmony with the teachings of this book. This may give apparent support to the assertion that the sentiments in ‘Living Temple’ are in harmony with my writings. \textbf{But God forbid that this sentiment should prevail}.}[SpTB02 53.3; 1904][https://egwwritings.org/?ref=en\_SpTB02.53.3]


\egw{\textbf{Je suis contrainte de parler pour nier l'affirmation selon laquelle les enseignements du ‘Temple Vivant’ peuvent être soutenus par des déclarations de mes écrits}. Il peut y avoir dans ce livre des expressions et des raisonnements qui sont en harmonie avec mes écrits. Et il peut y avoir dans mes écrits de nombreuses déclarations qui, sorties de leur contexte et interprétées selon l'esprit de l'auteur du ‘Temple Vivant’, sembleraient être en harmonie avec les enseignements de ce livre. Cela peut donner un apparent soutien à l'affirmation que les raisonnements dans le ‘Temple Vivant’ sont en harmonie avec mes écrits. \textbf{Mais que Dieu nous préserve que ce raisonnement prévale}.}[SpTB02 53.3; 1904][https://egwwritings.org/?ref=en\_SpTB02.53.3]


At this point, we have many unanswered questions. But, as we continue to study the first point of the \emcap{Fundamental Principles}, we will find answers to all of these questions. So far, in light of the \emcap{Fundamental Principles}, belief in the Trinity doctrine—as a Seventh-day Adventist—becomes very questionable. In order to defend the Trinity doctrine, the authority of the \emcap{Fundamental Principles} must be compromised. In what follows, we will briefly study their authority, context in Adventist history, and God’s purpose in giving them. We will also look at the true authorship of the \emcap{Fundamental Principles} and their role in present days.


À ce stade, nous avons beaucoup de questions sans réponse. Mais, en continuant d'étudier le premier point des \emcap{Principes Fondamentaux}, nous trouverons des réponses à toutes ces questions. Jusqu'ici, à la lumière des \emcap{Principes Fondamentaux}, la croyance en la doctrine de la Trinité—en tant qu'Adventiste du Septième Jour—devient très questionnable. Pour défendre la doctrine de la Trinité, l'autorité des \emcap{Principes Fondamentaux} doit être compromise. Dans ce qui suit, nous étudierons brièvement leur autorité, leur contexte dans l'histoire adventiste et le but de Dieu en les donnant. Nous examinerons également la véritable paternité des \emcap{Principes Fondamentaux} et leur rôle à l'époque actuelle.
