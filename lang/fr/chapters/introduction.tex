\qrchapterstar{https://forgottenpillar.com/rsc/fr-fp-introduction}{Introduction}

\addcontentsline{toc}{chapter}{Introduction}

Ce livre a trois objectifs à accomplir. Le premier est de raviver l'ancien pilier de notre foi appelé « \textit{la personnalité de Dieu} ». Le deuxième objectif est de rétablir la confiance dans les écrits d'Ellen White, et le troisième est de rétablir l'identité adventiste originelle.

Avant le 22 octobre 1844, il y avait un grand nombre d'adventistes qui attendaient le retour du Christ sur les nuées du ciel. C'était un mouvement mondial de personnes attendant Son second avènement. Le 22 octobre passa sans que Christ ne descende sur les nuées et la grande majorité quitta le mouvement, méprisant celui-ci, les prophéties, la Bible et Dieu. Seuls quelques hommes et femmes fidèles et humbles, restèrent, étant indiscutablement sûrs que Dieu dirigeait ce mouvement. Ils savaient que Dieu faisait briller la lumière de la Vérité et leurs cœurs étaient impatients de la recevoir. Mais aux yeux du monde, ils n'étaient en réalité que des fanatiques et des rêveurs. Cette grande déception peut être comparée à celle que les disciples de Jésus ont eue après avoir vu leur Seigneur être mis au tombeau. Ils étaient indiscutablement sûrs que Christ « \textit{était un prophète puissant en œuvres et en paroles devant Dieu et tout le peuple} », mais comme Il mourut sur la croix, ils furent amèrement déçus, car ils « \textit{espéraient que ce serait lui qui délivrerait Israël} ». Pourtant dans leur état de désespoir, dans leur état de déception personnelle, ils étaient prêts à recevoir la puissance pour conquérir le monde entier avec l'Évangile. Ils rencontrèrent Christ et reçurent plus tard Son Esprit. La même chose arriva aux pionniers adventistes. Ils étaient un petit groupe de personnes, amèrement déçues ; ils cherchèrent le Seigneur de tout leur cœur et Le reçurent en puissance et en Vérité. Les vérités que Dieu révéla pendant ce précieux temps de crise constituent le fondement de la foi adventiste du septième jour. Ces vérités furent testées par toutes les théories séduisantes et trompeuses du monde, par ceux qui méprisaient ce petit groupe, pourtant ces grandes vérités prévalurent. Au moment où le besoin était le plus pressant, Jésus donna Son témoignage en établissant une petite fille, la plus faible des faibles, pour approuver toutes ses vérités. Ellen White ne devait pas être la source des vérités ; plutôt, soutenir les frères qui cherchaient la vérité dans la Bible. Dieu se serva d'Ellen White pour approuver leurs études et les diriger vers la Bible. Le résultat final fut l'établissement d'un fondement de foi basé sur la Bible, qui demeure ferme jusqu'à la fin du monde.

Seriez-vous surpris de savoir que le fondement de la foi adventiste du septième jour, qui fut établi au début de notre œuvre est, dans une large mesure, différent de ce qu'il est actuellement ? Aujourd'hui, plus d'un siècle et demi plus tard, nous nous émerveillons devant les récits des expériences de nos pionniers ; mais depuis lors, l'Église Adventiste du Septième Jour a été soumise à plusieurs nouveaux mouvements. Depuis lors, l'église a connu de nombreux changements, y compris des changements dans notre doctrine. Certains soutiennent que ces changements sont bons et progressifs ; d'autres affirment qu'ils sont destructeurs et trompeurs. En dirigeant les projecteurs sur l'adventisme originel, la grande controverse d’aujourd’hui fut déclenchée. Nous sommes personnellement dans cette controverse depuis plus de 6 ans maintenant et nous avons vu qu'elle ne fera que s'amplifier et se renforcer, souvent avec des résultats d'un triste bilan. De nombreuses personnes des deux côtés de cette controverse rejettent l'Esprit de Prophétie d'une manière ou d'une autre. Certains ont complètement quitté l'Église Adventiste du Septième Jour. L'identité adventiste est soit perdue, soit radicalement changée par rapport à celle d'origine.

Nous sommes actuellement témoins du criblage de l'Église Adventiste du Septième Jour, la voyant ballottée d'une vague de crise à l'autre. Beaucoup perdent leur foi et leur identité en tant qu'adventistes du septième jour. Mais nous croyons en une solution que le Seigneur, dans sa miséricorde, a déjà fournie. La solution peut être trouvée dans l'histoire du mouvement adventiste.

\egw{\textbf{En nous remémorant notre histoire}, ayant parcouru chaque étape de progrès jusqu'à notre position actuelle, je puis dire: “Loué soit le Seigneur!” Lorsque je constate tout ce que le Seigneur a accompli, je suis remplie d'étonnement, et de confiance dans le Christ, notre chef. \textbf{Nous n'avons rien à craindre pour l'avenir, \underline{si ce n'est d'oublier} la façon dont le Seigneur nous a conduits, et \underline{ses enseignements} dans notre histoire passée}.}[LS 196.2; 1915][https://egwwritings.org/?ref=en\_LS.196.2]

nous ne craindrons rien ! C'est une grande assurance et promesse — bien que conditionnelle. Nous devons nous \textit{souvenir} de la façon dont le Seigneur nous a conduits, et de \textit{ses enseignements dans notre histoire passée}. Quand nous regardons ce que le Seigneur nous a enseigné dans notre histoire passée, nous sommes surpris de voir comment les choses ont changé. Le changement a pris plusieurs années et de nombreuses crises. Pour juger ces changements de doctrine, qu'ils soient bons et progressifs ou mauvais et destructeurs, l'évaluation devrait être basée sur les expériences passées, comme le Seigneur a clairement conduit son église.

À ce moment, nous avançons une affirmation audacieuse — une qui est censée vous faire tenir ce livre jusqu'à sa dernière page. Encouragés par les conseils d'Ellen White de remémorer notre histoire passée, nous sommes arrivés à la conclusion que nous, en tant que peuple, avons oublié un pilier crucial de notre foi, qui était le sujet principal de la controverse de Kellogg — la \emcap{personnalité de Dieu}. Une des plus grandes crises que l'Église Adventiste ait jamais connue du temps de la prophétesse vivante fut la crise de Kellogg. C'est de cette crise que de nombreuses autres crises, aujourd'hui, trouvent leurs racines. À la lumière de ce qui précède, le sujet de la \emcap{personnalité de Dieu} est crucial pour notre époque actuelle.

Sœur White écrivit à Kellogg que la \emcap{personnalité de Dieu} et la \emcap{personnalité du Christ} étaient un pilier de notre foi au même rang que le message du sanctuaire :

\egw{Ceux qui cherchent à enlever \textbf{les anciens repères} ne tiennent pas ferme ; ils \textbf{\underline{ne se souviennent pas} comment ils ont reçu et entendu}. Ceux qui essaient d’\textbf{\underline{introduire} des théories qui supprimeraient \underline{les piliers de notre foi} concernant le sanctuaire, \underline{ou concernant la personnalité de Dieu ou du Christ}, travaillent comme des aveugles}. Ils cherchent à introduire des incertitudes et à mettre le peuple de Dieu à la dérive, sans ancre.}[Ms62-1905.14][https://egwwritings.org/?ref=en\_Ms62-1905.14]

La \emcap{personnalité de Dieu} reçoit très peu d'attention aujourd'hui en tant que sujet, pourtant c'est l'un des éléments cruciaux dans le traitement d'autres doctrines relatives à l'Adventisme, comme la doctrine de la Trinité, le service du Sanctuaire, 1844 et toute autre doctrine traitant de la réalité céleste.

La \emcap{personnalité de Dieu} était un pilier de notre foi. Aujourd'hui, elle est presqu'oubliée. Nous proposons une explication raisonnable pour cela. C'est dû à l'évolution de la langue anglaise. Que signifie le terme “\textit{la personnalité de Dieu}” ? La compréhension générale du mot anglais ‘\textit{personality}’ a changé au fil des ans. Aujourd'hui, ‘\textit{personality}’ est généralement considérée comme “\textit{l'ensemble caractéristique de comportements, de cognitions et de schémas émotionnels}”\footnote{Wikipedia Contributors. “\textit{Personality.}” Wikipedia, Wikimedia Foundation, 19 Apr. 2019, \href{https://en.wikipedia.org/wiki/Personality}{en.wikipedia.org/wiki/Personality}.}, mais au dix-neuvième et au début du vingtième siècle, cela signifiait “\textit{la qualité ou l'état par lequel quelqu'un est défini comme une personne}”\footnote{\href{https://www.merriam-webster.com/dictionary/personality}{Merriam-Webster Dictionary}, - ‘personality’} \footnote{\href{https://babel.hathitrust.org/cgi/pt?id=mdp.39015050663213&view=1up&seq=780}{Hunter Robert, The American encyclopaedic dictionary}, ‘\textit{personality}’ - “\textit{la qualité ou l'état d'être personnel}”; Ce dictionnaire était en possession d'Ellen White (voir \href{https://repo.adventistdigitallibrary.org/PDFs/adl-22/adl-22251050.pdf?_ga=2.116010630.1065317374.1621993520-1506151612.1617862694&fbclid=IwAR3vwmp8jxtnpPEKv0KD9mCv8dJpmRGoyIXW0CkbQAjbU0h6YaBGqhgBzbk}{EGW Private and Office Libraries})}. Nous lisons cette définition comme la définition principale du mot ‘\textit{personality}’ dans le dictionnaire Merriam-Webster\footnote{\href{https://www.merriam-webster.com/dictionary/personality\#word-history}{Merriam-Webster Dictionary} indique que la première occurrence de la définition “la qualité ou l'état d'être une personne” est enregistrée au 15e siècle.}. Quand Sœur White et nos pionniers écrivaient sur la \emcap{personnalité de Dieu}, ils faisaient référence à \textit{la qualité ou l'état de Dieu comme étant une personne}. En d'autres termes, ils traitaient la question, “\textit{Dieu est-il une personne}”, et, “\textit{qu'est-ce qui fait de Lui une personne}” ou “\textit{quelle est la qualité ou l'état de Dieu comme étant une personne}” ? Essayez de vous rappeler la dernière fois que vous avez eu une étude biblique sur la question, “\textit{Dieu est-il une personne ?}” Réfléchissez à la façon dont vous pouvez vous prouver, à partir de la Bible, que Dieu est une personne. Pensez-y. C'est une question importante. De cette question dépend votre vision de Dieu et votre relation avec Lui. La \emcap{personnalité de Dieu} est fondamentale pour une véritable spiritualité ; la vraie spiritualité est basée sur votre relation personnelle avec Dieu. Aucune relation réelle ne peut être établie avec quiconque s'il n'est pas une personne. Peut-être ne vous êtes-vous jamais posé cette question parce que vous n'avez jamais ressenti le besoin de vous demander si Dieu est une personne, et ce qui fait de Lui (la qualité ou l'état) une personne. Ou peut-être vous êtes-vous abstenu de cette question parce que vous pensiez que c'était un mystère que Dieu n'avait pas l'intention de révéler. Peut-être serez-vous surpris de savoir que Dieu a donné une réponse définitive et affirmative dans sa Parole à la question “\textit{quelle est la qualité ou l'état de Dieu comme étant une personne}”. Ce qui était encore plus surprenant pour nous, c'est que les pionniers adventistes, y compris Sœur White, avaient une lumière précise concernant ce sujet, et ils le considéraient comme un pilier de notre foi, comme faisant partie du fondement de la foi adventiste du septième jour. Lorsque la \emcap{personnalité de Dieu} est correctement comprise à la lumière de notre passé historique, les anciennes citations brillent d'une nouvelle lumière et de nouvelles preuves sont présentées, qui approfondiront la compréhension de notre histoire passée et de la crise actuelle.

Le problème fondamental de la crise de Kellogg concernait la \emcap{personnalité de Dieu}. Il est certainement important d'évaluer la crise de Kellogg concernant la \emcap{personnalité de Dieu} en utilisant le sens voulu à cette époque ; c'est-à-dire en utilisant la définition de ‘\textit{personnalité}’ comme la qualité ou l'état de Dieu comme étant une personne. Avec cette définition à l'esprit, la crise de Kellogg apparaît sous un nouveau jour et de nouvelles preuves pertinentes sont mises en avant pour nous aujourd'hui. À la lumière de ces preuves, nous voyons comment Dieu nous a guidés dans le passé ; ainsi, nous ne devrions pas craindre l'avenir. Connaître et comprendre cela, ainsi que son importance, nous aide à ne pas être ébranlés par toute vague de tromperie dans les controverses actuelles. Lorsque Sœur White attirait l'attention de Kellogg sur l'importance de ce sujet, elle attirait aussi notre attention, car c'est tout pour nous en tant que peuple.

[Écrivant à Kellogg] \egw{Vous n'êtes pas clairement défini sur \textbf{la personnalité de Dieu}, qui est \textbf{\underline{tout} pour nous en tant que peuple}.}[Lt300-1903.7][https://egwwritings.org/?ref=en\_Lt300-1903.7]

Ces études sur la \emcap{personnalité de Dieu} soulèveront beaucoup de questions nouvelles et difficiles. Nous ne promettons pas de répondre à toutes, et peut-être ne serez-vous pas satisfait des réponses fournies, mais nous prions, espérons et croyons que ce livre remplira les trois objectifs proposés au début de cette introduction. Grâce à la revitalisation de la doctrine sur la \emcap{personnalité de Dieu}, nous croyons que votre confiance dans l'Esprit de Prophétie se renforcera, et que vous vous trouverez plus profondément enraciné dans le message adventiste—où nous trouvons notre identité en tant que peuple—faisant de vous un adventiste du septième jour plus fidèle. Plus important encore, nous voulons que vous deveniez plus conscient de Dieu comme votre Dieu personnel. Cela renforcera et approfondira sûrement votre relation avec Lui.

Nous trouvons des réponses à la question de la \emcap{personnalité de Dieu} en examinant la crise de Kellogg, où Sœur White a donné la lumière la plus précise sur la \emcap{personnalité de Dieu} et sur le fondement de la foi adventiste du septième jour. Voici le dixième chapitre complet du livre \textit{Témoignages pour l'Église contenant des lettres aux médecins et aux pasteurs - Instruction aux Adventistes du Septième Jour}. Ce chapitre, \textit{Le Fondement de notre Foi}, contient un aperçu profond de l'histoire de la crise de Kellogg. Il donne un aperçu historique des vérités que Dieu a données comme fondement de notre foi et dans ces vérités nous trouvons notre identité en tant qu'Adventistes du Septième Jour— qui gardent les commandements de Dieu et ont la foi de Jésus.
