% Prefácio
\chapter*{Introduction}


% Prefácio
\chapter*{Introduction}


This book has three objectives to fulfill. The first one is to revive the old pillar of our faith called, “\textit{the personality of God}”. The second objective is to re-establish trust in the writings of Ellen White, and the third is to re-establish the original Adventist identity.


Ce livre a trois objectifs à accomplir. Le premier est de raviver l'ancien pilier de notre foi appelé « \textit{la personnalité de Dieu} ». Le deuxième objectif est de rétablir la confiance dans les écrits d'Ellen White, et le troisième est de rétablir l'identité adventiste originelle.


Prior to October 22, 1844, there was a great number of Adventists waiting for Christ to return on the clouds of heaven. It was a global movement of people awaiting His second coming. October 22 passed without Christ descending on the clouds and the great majority left the movement, scorning it, scorning the prophecies, the Bible, and God. Very few faithful, humble, men and women remained, who were unquestionably sure that God was leading this movement. They knew that God was shining the light of Truth and their hearts were eager to receive it. But in the eyes of the world, they were just demonstrated fanatics and dreamers. This great disappointment can be compared to the one that Jesus’ disciples had after they saw their Lord being laid in the grave. They were unquestionably sure that Christ “\textit{was a prophet mighty in deed and word before God and all the people}”, but as He died on the cross, they were bitterly disappointed, because they “\textit{trusted that it had been He which should have redeemed Israel}.” Yet in their state of despair, in their state of self-disappointment, they were ready to receive the power to conquer the whole world with the Gospel. They met Christ and later received His Spirit. The same happened with the Adventist pioneers. They were a small group of people, bitterly disappointed; they sought the Lord with all their hearts and received Him in power and in Truth. The truths God revealed during this precious time of crisis constitutes the foundation of Seventh-day Adventist faith. These truths were tested by all the seductive, deceptive theories of the world, by those scorning this small group, yet these grand truths prevailed. In the time of greatest need, Jesus gave His testimony by raising a little girl, the weakest of the weak, to approve all of His truths. Ellen White was not to be the source of the truths; rather, to support the brethren who were seeking the truth in the Bible. God used Ellen White to approve their studies and to point them to the Bible. The final result was the establishment of the foundation of faith based on the Bible, which standeth sure till the end of the world.


Avant le 22 octobre 1844, il y avait un grand nombre d'adventistes qui attendaient le retour du Christ sur les nuées du ciel. C'était un mouvement mondial de personnes attendant Son second avènement. Le 22 octobre passa sans que Christ ne descende sur les nuées et la grande majorité quitta le mouvement, méprisant celui-ci, les prophéties, la Bible et Dieu. Seuls quelques fidèles, hommes et femmes humbles, restèrent, étant indiscutablement sûrs que Dieu dirigeait ce mouvement. Ils savaient que Dieu faisait briller la lumière de la Vérité et leurs cœurs étaient impatients de la recevoir. Mais aux yeux du monde, ils n'étaient que des fanatiques et des rêveurs démontrés. Cette grande déception peut être comparée à celle que les disciples de Jésus ont eue après avoir vu leur Seigneur être mis au tombeau. Ils étaient indiscutablement sûrs que Christ « \textit{était un prophète puissant en œuvres et en paroles devant Dieu et tout le peuple} », mais comme Il mourut sur la croix, ils furent amèrement déçus, car ils « \textit{espéraient que ce serait lui qui délivrerait Israël} ». Pourtant dans leur état de désespoir, dans leur état de déception personnelle, ils étaient prêts à recevoir la puissance pour conquérir le monde entier avec l'Évangile. Ils rencontrèrent Christ et reçurent plus tard Son Esprit. La même chose arriva aux pionniers adventistes. Ils étaient un petit groupe de personnes, amèrement déçues ; ils cherchèrent le Seigneur de tout leur cœur et Le reçurent en puissance et en Vérité. Les vérités que Dieu révéla pendant ce précieux temps de crise constituent le fondement de la foi adventiste du septième jour. Ces vérités furent testées par toutes les théories séduisantes et trompeuses du monde, par ceux qui méprisaient ce petit groupe, pourtant ces grandes vérités prévalurent. Dans le moment du plus grand besoin, Jésus donna Son témoignage en élevant une petite fille, la plus faible des faibles, pour approuver toutes Ses vérités. Ellen White ne devait pas être la source des vérités ; plutôt, soutenir les frères qui cherchaient la vérité dans la Bible. Dieu utilisa Ellen White pour approuver leurs études et les diriger vers la Bible. Le résultat final fut l'établissement du fondement de la foi basé sur la Bible, qui demeure ferme jusqu'à la fin du monde.


Would you be surprised to know that the foundation of Seventh-day Adventist faith, which was laid at the beginning of our work, is in a fair degree different from what it is currently? Today, more than a century and a half later, we marvel in amazement over the accounts of the experiences of our pioneers; but since then, the Seventh-day Adventist Church has been subject to several new movements. Since then, the church has experienced many changes, including changes in our doctrine. Some argue that these changes are good and progressive; others argue that they are destructive and deceptive. Moving the spotlight to the original Seventh-day Adventism, it initiates the great controversy in the present days. We have personally been in this controversy for over 6 years now and we have seen that it will only get bigger and stronger, often with results of a sad record. Many people from both sides of this controversy are rejecting the Spirit of Prophecy in one way or another. Some have left the Seventh-day Adventist Church altogether. The Adventist identity is either lost or drastically changed from the initial one.


Seriez-vous surpris de savoir que le fondement de la foi adventiste du septième jour, qui fut établi au début de notre œuvre, est sensiblement différent de ce qu'il est actuellement ? Aujourd'hui, plus d'un siècle et demi plus tard, nous nous émerveillons devant les récits des expériences de nos pionniers ; mais depuis lors, l'Église Adventiste du Septième Jour a été soumise à plusieurs nouveaux mouvements. Depuis lors, l'église a connu de nombreux changements, y compris des changements dans notre doctrine. Certains soutiennent que ces changements sont bons et progressifs ; d'autres affirment qu'ils sont destructeurs et trompeurs. Diriger les projecteurs sur l'adventisme original déclenche la grande controverse dans les temps présents. Nous sommes personnellement dans cette controverse depuis plus de 6 ans maintenant et nous avons vu qu'elle ne fera que s'amplifier et se renforcer, souvent avec des résultats d'un triste bilan. De nombreuses personnes des deux côtés de cette controverse rejettent l'Esprit de Prophétie d'une manière ou d'une autre. Certains ont complètement quitté l'Église Adventiste du Septième Jour. L'identité adventiste est soit perdue, soit radicalement changée par rapport à celle d'origine.


We are currently witnessing the shaking of the Seventh-day Adventist church, seeing her tossed through one wave of crisis after another. Many are losing their faith and their identity as Seventh-day Adventists. But we believe in a solution that the Lord, in His mercy, has already provided. The solution can be found in the history of the Seventh-day Adventist movement.


Nous sommes actuellement témoins du criblage de l'Église Adventiste du Septième Jour, la voyant ballottée d'une vague de crise à l'autre. Beaucoup perdent leur foi et leur identité en tant qu'adventistes du septième jour. Mais nous croyons en une solution que le Seigneur, dans Sa miséricorde, a déjà fournie. La solution peut être trouvée dans l'histoire du mouvement adventiste.


\egw{\textbf{In reviewing our past history}, having traveled over every step of advance to our present standing, I can say, Praise God! As I see what the Lord has wrought, I am filled with astonishment, and with confidence in Christ as leader. \textbf{We have nothing to fear for the future, \underline{except as we shall forget} the way the Lord has led us, and \underline{His teaching} in our past history}.}[LS 196.2; 1915][https://egwwritings.org/?ref=en\_LS.196.2]


\egw{\textbf{En examinant notre histoire passée}, ayant parcouru chaque étape de progrès jusqu'à notre position actuelle, je peux dire, Louez Dieu ! Quand je vois ce que le Seigneur a accompli, je suis remplie d'étonnement et de confiance en Christ comme guide. \textbf{Nous n'avons rien à craindre pour l'avenir, \underline{sauf si nous oublions} la façon dont le Seigneur nous a conduits, et \underline{Son enseignement} dans notre histoire passée}.}[LS 196.2; 1915][https://egwwritings.org/?ref=en\_LS.196.2]


We shall not fear! This is a great reassurance and promise—though conditional. We must \textit{remember} how the Lord has led us, and \textit{His teaching in our past history}. When we look at what the Lord has taught us in our past history, we are surprised to see how things have changed. The change has taken several years and many crises. To judge these changes in doctrine, whether good and progressive or bad and destructive, evaluation should be based on past experiences, as the Lord clearly led His church.


Nous ne craindrons pas ! C'est une grande assurance et promesse — bien que conditionnelle. Nous devons nous \textit{souvenir} de la façon dont le Seigneur nous a conduits, et de \textit{Son enseignement dans notre histoire passée}. Quand nous regardons ce que le Seigneur nous a enseigné dans notre histoire passée, nous sommes surpris de voir comment les choses ont changé. Le changement a pris plusieurs années et de nombreuses crises. Pour juger ces changements de doctrine, qu'ils soient bons et progressifs ou mauvais et destructeurs, l'évaluation devrait être basée sur les expériences passées, comme le Seigneur a clairement conduit Son église.


At this time, we put forth a bold claim—one that is supposed to make you hold this book until the end of its cover. Encouraged by the councils of Ellen White to review our past history, we have concluded that we have forgotten one crucial pillar of our faith, which was the main subject of Kellogg’s controversy—the \emcap{personality of God}. One of the biggest crises that the SDA Church ever had in the time of the living prophet was the Kellogg crisis. It is out of this crisis that many other crises, today, find their roots. In this light, the subject of the \emcap{personality of God} is pivotal in our present time.


À ce moment, nous avançons une affirmation audacieuse — une qui est censée vous faire tenir ce livre jusqu'à sa dernière page. Encouragés par les conseils d'Ellen White de revoir notre histoire passée, nous avons conclu que nous avons oublié un pilier crucial de notre foi, qui était le sujet principal de la controverse de Kellogg — la \emcap{personnalité de Dieu}. Une des plus grandes crises que l'Église Adventiste ait jamais connue du temps de la prophétesse vivante fut la crise de Kellogg. C'est de cette crise que de nombreuses autres crises, aujourd'hui, trouvent leurs racines. Dans cette optique, le sujet de la \emcap{personnalité de Dieu} est crucial à notre époque.


Sister White wrote to Kellogg that the \emcap{personality of God} and the \emcap{personality of Christ} was a pillar of our faith in the same rank as is the sanctuary message:


Sœur White écrivit à Kellogg que la \emcap{personnalité de Dieu} et la \emcap{personnalité du Christ} était un pilier de notre foi au même rang que le message du sanctuaire :


\egw{Those who seek to remove \textbf{the old landmarks} are not holding fast; they \textbf{are \underline{not remembering} how they have received and heard}. Those who try to \textbf{\underline{bring in} theories that would remove \underline{the pillars of our faith} concerning the sanctuary, \underline{or concerning the personality of God or of Christ}, are working as blind men}. They are seeking to bring in uncertainties and to set the people of God adrift, without an anchor.}[Ms62-1905.14][https://egwwritings.org/?ref=en\_Ms62-1905.14]


\egw{Ceux qui cherchent à enlever \textbf{les anciens repères} ne tiennent pas ferme ; ils \textbf{\underline{ne se souviennent pas} comment ils ont reçu et entendu}. Ceux qui essaient d’\textbf{\underline{introduire} des théories qui supprimeraient \underline{les piliers de notre foi} concernant le sanctuaire, \underline{ou concernant la personnalité de Dieu ou du Christ}, travaillent comme des aveugles}. Ils cherchent à introduire des incertitudes et à mettre le peuple de Dieu à la dérive, sans ancre.}[Ms62-1905.14][https://egwwritings.org/?ref=en\_Ms62-1905.14]


The \emcap{personality of God} receives very little attention today as a subject, yet it is one of the crucial elements in dealing with other doctrines pertaining to Adventism, such as the doctrine of Trinity, the Sanctuary service, 1844 and any other doctrine dealing with the Heavenly reality.


La \emcap{personality of God} reçoit très peu d'attention aujourd'hui en tant que sujet, pourtant c'est l'un des éléments cruciaux dans le traitement d'autres doctrines relatives à l'adventisme, comme la doctrine de la Trinité, le service du Sanctuaire, 1844 et toute autre doctrine traitant de la réalité céleste.


The \emcap{personality of God} was a pillar of our faith. Today, it is almost forgotten. We propose a reasonable explanation for that. It is due to the evolution of the English language. What is meant by the term, “\textit{the personality of God}”? The general understanding of the English word ‘\textit{personality}’ has changed over the years. Today, ‘\textit{personality}’ is generally viewed as, “\textit{the characteristic set of behaviors, cognitions, and emotional patterns}”\footnote{Wikipedia Contributors. “\textit{Personality.}” Wikipedia, Wikimedia Foundation, 19 Apr. 2019, \href{https://en.wikipedia.org/wiki/Personality}{en.wikipedia.org/wiki/Personality}.}, but in the nineteenth, and beginning of the twentieth century, it meant “\textit{the quality or state of \textbf{being a person}}”\footnote{\href{https://www.merriam-webster.com/dictionary/personality}{Merriam-Webster Dictionary}, - ‘personality’} \footnote{\href{https://babel.hathitrust.org/cgi/pt?id=mdp.39015050663213&view=1up&seq=780}{Hunter Robert, The American encyclopaedic dictionary}, ‘\textit{personality}’ - “\textit{the quality or state of being personal}”; Mentioned dictionary was in possession of Ellen White (see \href{https://repo.adventistdigitallibrary.org/PDFs/adl-22/adl-22251050.pdf?_ga=2.116010630.1065317374.1621993520-1506151612.1617862694&fbclid=IwAR3vwmp8jxtnpPEKv0KD9mCv8dJpmRGoyIXW0CkbQAjbU0h6YaBGqhgBzbk}{EGW Private and Office Libraries})}. We read this definition as the primary definition of the word ‘\textit{personality}’ from the Merriam-Webster Dictionary\footnote{\href{https://www.merriam-webster.com/dictionary/personality\#word-history}{Merriam-Webster Dictionary} marks that the first record of the definition “the quality or state of being a person” is recorded in the 15th century.}. When Sister White and our pioneers wrote about the \emcap{personality of God}, they referred to \textit{the quality or state of God being a person}. In other words, they dealt with the question, “\textit{is God a person}”, and, “\textit{what is it that makes Him a person}” or “\textit{what is the quality or state of God being a person}”? Try to remember the last time you had a Bible study on the question, “\textit{is God a person?}” Think about how you can prove to yourself, from the Bible, that God is a person. Think about it. It is an important question. Upon this question hangs your view of God and your relationship toward Him. The \emcap{personality of God} is fundamental to true spirituality; true spirituality is based on your personal relationship with God. No real relationship of any kind can be formed with anyone unless he/she is a person. Maybe you have never asked yourself this question because you never felt a need to question if God is a person, and what is it (the quality or state) that makes Him a person. Or, maybe you were refraining from this question because you felt it might be a mystery that God did not intend to reveal. Maybe it will surprise you to know that God has given a definite and affirmative answer in His Word to the question “\textit{what is the quality or state of God being a person}”. What was even more surprising for us, was that the Adventist pioneers, including Sister White, had definite light regarding this topic, and they held it as a pillar of our faith, as part of the foundation of Seventh-day Adventist faith. When the \emcap{personality of God} is rightly understood in light of our historical past, old quotations shine in a new light and new shreds of evidence are presented, which will deepen the understanding of our past history and the present crisis.


La \emcap{personality of God} était un pilier de notre foi. Aujourd'hui, elle est presque oubliée. Nous proposons une explication raisonnable pour cela. C'est dû à l'évolution de la langue anglaise. Que signifie le terme “\textit{the personality of God}” ? La compréhension générale du mot anglais ‘\textit{personality}’ a changé au fil des ans. Aujourd'hui, ‘\textit{personality}’ est généralement considérée comme “\textit{l'ensemble caractéristique de comportements, de cognitions et de schémas émotionnels}”\footnote{Wikipedia Contributors. “\textit{Personality.}” Wikipedia, Wikimedia Foundation, 19 Apr. 2019, \href{https://en.wikipedia.org/wiki/Personality}{en.wikipedia.org/wiki/Personality}.}, mais au dix-neuvième et au début du vingtième siècle, cela signifiait “\textit{la qualité ou l'état par lequel quelqu'un est défini comme une personne}”\footnote{\href{https://www.merriam-webster.com/dictionary/personality}{Merriam-Webster Dictionary}, - ‘personality’} \footnote{\href{https://babel.hathitrust.org/cgi/pt?id=mdp.39015050663213&view=1up&seq=780}{Hunter Robert, The American encyclopaedic dictionary}, ‘\textit{personality}’ - “\textit{la qualité ou l'état d'être personnel}”; Ce dictionnaire était en possession d'Ellen White (voir \href{https://repo.adventistdigitallibrary.org/PDFs/adl-22/adl-22251050.pdf?_ga=2.116010630.1065317374.1621993520-1506151612.1617862694&fbclid=IwAR3vwmp8jxtnpPEKv0KD9mCv8dJpmRGoyIXW0CkbQAjbU0h6YaBGqhgBzbk}{EGW Private and Office Libraries})}. Nous lisons cette définition comme la définition principale du mot ‘\textit{personality}’ dans le dictionnaire Merriam-Webster\footnote{\href{https://www.merriam-webster.com/dictionary/personality\#word-history}{Merriam-Webster Dictionary} indique que la première occurrence de la définition “la qualité ou l'état d'être une personne” est enregistrée au 15e siècle.}. Quand Sœur White et nos pionniers écrivaient sur la \emcap{personality of God}, ils faisaient référence à \textit{la qualité ou l'état de Dieu comme étant une personne}. En d'autres termes, ils traitaient la question, “\textit{Dieu est-il une personne}”, et, “\textit{qu'est-ce qui fait de Lui une personne}” ou “\textit{quelle est la qualité ou l'état de Dieu comme étant une personne}” ? Essayez de vous rappeler la dernière fois que vous avez eu une étude biblique sur la question, “\textit{Dieu est-il une personne ?}” Réfléchissez à la façon dont vous pouvez vous prouver, à partir de la Bible, que Dieu est une personne. Pensez-y. C'est une question importante. De cette question dépend votre vision de Dieu et votre relation avec Lui. La \emcap{personality of God} est fondamentale pour une véritable spiritualité ; la vraie spiritualité est basée sur votre relation personnelle avec Dieu. Aucune relation réelle ne peut être établie avec quiconque s'il n'est pas une personne. Peut-être ne vous êtes-vous jamais posé cette question parce que vous n'avez jamais ressenti le besoin de vous demander si Dieu est une personne, et ce qui fait de Lui (la qualité ou l'état) une personne. Ou peut-être vous êtes-vous abstenu de cette question parce que vous pensiez que c'était un mystère que Dieu n'avait pas l'intention de révéler. Peut-être serez-vous surpris de savoir que Dieu a donné une réponse définitive et affirmative dans Sa Parole à la question “\textit{quelle est la qualité ou l'état de Dieu comme étant une personne}”. Ce qui était encore plus surprenant pour nous, c'est que les pionniers adventistes, y compris Sœur White, avaient une lumière précise concernant ce sujet, et ils le considéraient comme un pilier de notre foi, comme faisant partie du fondement de la foi adventiste du septième jour. Lorsque la \emcap{personality of God} est correctement comprise à la lumière de notre passé historique, les anciennes citations brillent d'une nouvelle lumière et de nouvelles preuves sont présentées, qui approfondiront la compréhension de notre histoire passée et de la crise actuelle.


The root problem of the Kellogg crisis was about the \emcap{personality of God}. It is certainly important to evaluate Kellogg's crisis over the \emcap{personality of God} using the meaning intended at that time; that is, using the definition of ‘\textit{personality},’ as the quality or state of God being a person. With this definition in mind, the Kellogg crisis comes into a new light and new relevant evidence is brought forth for us today. In light of this evidence, we see how God has led us in the past; thus, we should not fear for the future. Knowing and understanding this, as well as its importance, helps us to not be shaken by any wave of deception in present controversies. When Sister White was drawing Kellogg’s attention to the importance of this subject, she was drawing our attention also, as it is everything to us as a people.


Le problème fondamental de la crise de Kellogg concernait la \emcap{personality of God}. Il est certainement important d'évaluer la crise de Kellogg concernant la \emcap{personality of God} en utilisant le sens voulu à l'époque ; c'est-à-dire en utilisant la définition de ‘\textit{personality}’ comme la qualité ou l'état de Dieu comme étant une personne. Avec cette définition à l'esprit, la crise de Kellogg apparaît sous un nouveau jour et de nouvelles preuves pertinentes sont mises en avant pour nous aujourd'hui. À la lumière de ces preuves, nous voyons comment Dieu nous a guidés dans le passé ; ainsi, nous ne devrions pas craindre l'avenir. Connaître et comprendre cela, ainsi que son importance, nous aide à ne pas être ébranlés par toute vague de tromperie dans les controverses actuelles. Lorsque Sœur White attirait l'attention de Kellogg sur l'importance de ce sujet, elle attirait aussi notre attention, car c'est tout pour nous en tant que peuple.


[Writing to Kellogg] \egw{You are not definitely clear on \textbf{the personality of God}, which is \textbf{\underline{everything} to us as a people}.}[Lt300-1903.7][https://egwwritings.org/?ref=en\_Lt300-1903.7]


[Écrivant à Kellogg] \egw{Vous n'êtes pas clairement défini sur \textbf{la personnalité de Dieu}, qui est \textbf{\underline{tout} pour nous en tant que peuple}.}[Lt300-1903.7][https://egwwritings.org/?ref=en\_Lt300-1903.7]


These studies on the \emcap{personality of God} will prompt a lot of new and hard questions. We do not promise to answer all of them, and perhaps you won’t be satisfied with the answers provided, but we pray, hope and believe that this book will fulfill the three objectives proposed in the beginning of this introduction. Through the reviving of the doctrine on the \emcap{personality of God}, we believe that your confidence in the Spirit of Prophecy will strengthen, and that you’ll find yourself rooted deeper in the Adventist message—where we find our identity as people—making you a more faithful Seventh-day Adventist. Most importantly, we want you to become more aware of God as your personal God. This will surely strengthen and deepen your relationship with Him.


Ces études sur la \emcap{personality of God} soulèveront beaucoup de questions nouvelles et difficiles. Nous ne promettons pas de répondre à toutes, et peut-être ne serez-vous pas satisfait des réponses fournies, mais nous prions, espérons et croyons que ce livre remplira les trois objectifs proposés au début de cette introduction. À travers la renaissance de la doctrine sur la \emcap{personality of God}, nous croyons que votre confiance dans l'Esprit de Prophétie se renforcera, et que vous vous trouverez plus profondément enraciné dans le message adventiste—où nous trouvons notre identité en tant que peuple—faisant de vous un adventiste du septième jour plus fidèle. Plus important encore, nous voulons que vous deveniez plus conscient de Dieu comme votre Dieu personnel. Cela renforcera et approfondira sûrement votre relation avec Lui.


We find answers to the issue on the \emcap{personality of God} in examining the Kellogg crisis, where Sister White gave the most definite light on the \emcap{personality of God} and on the foundation of Seventh-day Adventist faith. The following is the complete tenth chapter from the book, \textit{Testimonies for the Church Containing Letters to Physicians and Ministers Instruction to Seventh-Day Adventists}. This chapter, \textit{The Foundation of our Faith}, contains deep insight into the history of Kellogg’s crisis. It gives a historical overview of the truths God gave as the foundation of our faith and in these truths we find our identity as Seventh-day Adventists— keeping the commandments of God and having the faith of Jesus.


Nous trouvons des réponses à la question de la \emcap{personality of God} en examinant la crise de Kellogg, où Sœur White a donné la lumière la plus précise sur la \emcap{personality of God} et sur le fondement de la foi adventiste du septième jour. Voici le dixième chapitre complet du livre \textit{Témoignages pour l'Église contenant des lettres aux médecins et aux pasteurs - Instruction aux Adventistes du Septième Jour}. Ce chapitre, \textit{Le Fondement de notre Foi}, contient un aperçu profond de l'histoire de la crise de Kellogg. Il donne un aperçu historique des vérités que Dieu a données comme fondement de notre foi et dans ces vérités nous trouvons notre identité en tant qu'Adventistes du Septième Jour—gardant les commandements de Dieu et ayant la foi de Jésus.
