\qrchapter{https://forgottenpillar.com/rsc/en-fp-chapter18}{The Heavenly Trio}


\qrchapter{https://forgottenpillar.com/rsc/en-fp-chapter18}{Le Trio Céleste}


So far we have seen the evidence that Ellen White knew about Dr. Kellogg's trinitarian sentiments, and we have seen how she responded to it. She always uplifted the truth on the presence and the \emcap{personality of God}, and called to come back to the foundation of our faith—\emcap{Fundamental Principles}. However, when Adventist scholars discuss the doctrine of the Trinity and Ellen White, they do not approach it in the same manner as Ellen White did. The \emcap{Fundamental Principles} together with the doctrine on the \emcap{personality of God} is downplayed, and the twisted story is presented that Ellen White was trinitarian and responsible for the church's acceptance of the Trinity doctrine into our ranks. We want to challenge this twisted story by looking at the evidence that is often used to support this false narrative.


Jusqu'à présent, nous avons vu les preuves qu'Ellen White connaissait les raisonnements trinitaires du Dr Kellogg, et nous avons vu comment elle y a répondu. Elle a toujours exalté la vérité sur la présence et la \emcap{personnalité de Dieu}, et a appelé à revenir au fondement de notre foi—les \emcap{Principes Fondamentaux}. Cependant, lorsque les érudits adventistes discutent de la doctrine de la Trinité et d'Ellen White, ils ne l'abordent pas de la même manière qu'Ellen White l'a fait. Les \emcap{Principes Fondamentaux} ainsi que la doctrine sur la \emcap{personnalité de Dieu} sont minimisés, et l'histoire déformée est présentée qu'Ellen White était trinitaire et responsable de l'acceptation de la doctrine de la Trinité dans nos rangs. Nous voulons contester cette histoire déformée en examinant les preuves qui sont souvent utilisées pour soutenir ce faux récit.


One of the most prominent quotations to support the claim that Sister White was responsible for accepting the Trinity doctrine into our ranks is her writings and comments on Matthew 28:19\footnote{\bible{Go ye therefore, and teach all nations, baptizing them in the name of the Father, and of the Son, and of the Holy Ghost}[Matthew 28:19]}. The most prominent quotation to stand out in defense of the Trinity doctrine is “\textit{the Heavenly Trio}” quotation:


L'une des citations les plus importantes pour soutenir l'affirmation que Sœur White était responsable de l'acceptation de la doctrine de la Trinité dans nos rangs est ses écrits et commentaires sur Matthieu 28:19\footnote{\bible{Allez donc et enseignez toutes les nations, les baptisant au nom du Père, du Fils et du Saint-Esprit}[Matthieu 28:19]}. La citation la plus importante qui ressort en défense de la doctrine de la Trinité est la citation « \textit{le Trio Céleste} » :


\egw{\textbf{There are \underline{three living persons} of the \underline{heavenly trio}}; in the name of these three great powers—\textbf{the Father, the Son, and the Holy Spirit}—those who receive Christ by living faith are baptized, and these powers will co-operate with the obedient subjects of heaven in their efforts to live the new life in Christ...}[Ev 615.1; 1946][https://egwwritings.org/read?panels=p30.3407]


\egw{\textbf{Il y a \underline{trois personnes vivantes} du \underline{trio céleste}} ; au nom de ces trois grandes puissances—\textbf{le Père, le Fils et le Saint-Esprit}—ceux qui reçoivent Christ par la foi vivante sont baptisés, et ces puissances coopéreront avec les sujets obéissants du ciel dans leurs efforts pour vivre la nouvelle vie en Christ...}[Ev 615.1; 1946][https://egwwritings.org/read?panels=p30.3407]


To reiterate, this quotation is often cited to argue that Sister White defended and advocated the Trinity doctrine. But, if we take a look at this quotation in its literary context, we see that within the quotation itself she actually \textit{refuted} this doctrine and exalted the truth on the \emcap{personality of God}. To some this is a ludicrous claim, but we invite you to make your judgment based on presented data. Let us examine the context of this quotation.


Pour réitérer, cette citation est souvent citée pour argumenter que Sœur White défendait et préconisait la doctrine de la Trinité. Mais, si nous examinons cette citation dans son contexte littéraire, nous voyons que dans la citation elle-même, elle a en fait \textit{réfuté} cette doctrine et exalté la vérité sur la \emcap{personnalité de Dieu}. Pour certains, c'est une affirmation ridicule, mais nous vous invitons à faire votre jugement basé sur les données présentées. Examinons le contexte de cette citation.


\egw{I am instructed to say, \textbf{The sentiments} of those who are searching for advanced scientific ideas \textbf{\underline{are not to be trusted}}. Such representations as the following are made: ‘\textbf{The Father is as the light invisible; the Son is as the light embodied; the Spirit as the light shed abroad.}’ ‘\textbf{The Father is like the dew, invisible vapor; the Son is like the dew gathered in beauteous form; the Spirit is like the dew fallen to the seat of life.}’ Another representation: ‘\textbf{The Father is like the invisible vapor. The Son is like the leaden cloud. The Spirit is rain fallen and working in refreshing power.}’}[Ms21-1906.8; 1906][https://egwwritings.org/read?panels=p9754.15]


\egw{Je suis instruite de dire : \textbf{Les raisonnements} de ceux qui recherchent des idées scientifiques avancées \textbf{\underline{ne doivent pas être dignes de confiance}}. Des représentations telles que les suivantes sont faites : ‘\textbf{Le Père est comme la lumière invisible ; le Fils est comme la lumière incarnée ; l'Esprit comme la lumière répandue.}’ ‘\textbf{Le Père est comme la rosée, vapeur invisible ; le Fils est comme la rosée rassemblée en forme magnifique ; l'Esprit est comme la rosée tombée au siège de la vie.}’ Une autre représentation : ‘\textbf{Le Père est comme la vapeur invisible. Le Fils est comme le nuage de plomb. L'Esprit est la pluie tombée et œuvrant avec une puissance rafraîchissante.}’}[Ms21-1906.8; 1906][https://egwwritings.org/read?panels=p9754.15]


What sentiments are not to be trusted? The data suggest that those sentiments are trinitarian ideas of \textit{one God in three persons}. How do we know that? We see in the literary context of the representations Sister White was quoting. Contrary to the popular belief that she was referencing the “\textit{false}” trinity expressed by Dr. Kellogg,\footnote{Whidden, Woodrow W, et al. \textit{The Trinity : Understanding God’s Love, His Plan of Salvation, and Christian Relationships}. Hagerstown, Md, Review And Herald Pub. Association, 2002, p. 216.} she was actually referencing trinitarian idea of \textit{three living persons of one living God}, advocated by William Boardman, in his book “Higher Christian Life”, which she quoted. The context matters. The context of the quotations she quoted, shows that the representations of the Father, the Son, and the Holy Spirit are serving to illustrate the sentiment of three living persons of one God. That is the sentiment we have been clearly instructed by God, not to trust. Let the data be its own interpreter.


Quels raisonnements ne doivent pas être dignes de confiance ? Les données suggèrent que ces raisonnements sont des idées trinitaires d’\textit{un seul Dieu en trois personnes}. Comment le savons-nous ? Nous le voyons dans le contexte littéraire des représentations que Sœur White citait. Contrairement à la croyance populaire qu'elle faisait référence à la « \textit{fausse} » trinité exprimée par le Dr Kellogg,\footnote{Whidden, Woodrow W, et al. \textit{The Trinity : Understanding God's Love, His Plan of Salvation, and Christian Relationships}. Hagerstown, Md, Review And Herald Pub. Association, 2002, p. 216.} elle faisait en fait référence à l'idée trinitaire de \textit{trois personnes vivantes d'un seul Dieu vivant}, préconisée par William Boardman, dans son livre « Higher Christian Life », qu'elle a cité. Le contexte est important. Le contexte des citations qu'elle a citées montre que les représentations du Père, du Fils et du Saint-Esprit servent à illustrer le raisonnement de trois personnes vivantes d'un seul Dieu. C'est le raisonnement que Dieu nous a clairement instruit de ne pas faire confiance. Laissons les données être leur propre interprète.


\section*{The Higher Christian Life, William Boardman}


\section*{The Higher Christian Life, William Boardman}


Ellen White owned William Boardman's book “Higher Christian Life.” It was a good book about Christian sanctification, but in it there was trinitarian sentiment, which Sister White was particularly instructed by God to call out. This is another instance of evidence where we see that Ellen White was familiar with the trinitarian stance, and she was addressing it directly. Let's get familiar with the trinitarian sentiments promoted by William Boardman.


Ellen White possédait le livre de William Boardman « Higher Christian Life ». C'était un bon livre sur la sanctification chrétienne, mais il contenait un raisonnement trinitaire, que Sœur White a été particulièrement instruite par Dieu d'identifier. C'est une autre preuve où nous voyons qu'Ellen White était familière avec la position trinitaire, et elle l'abordait directement. Familiarisons-nous avec les raisonnements trinitaires promus par William Boardman.


Speaking of Triune God, William Boardman writes:


En parlant du Dieu Trinitaire, William Boardman écrit :


\othersQuote{And then, again, the Father is the author and planner of salvation through faith in his Son; and when we trust in his Son we honor the Father, because we accept of his plan of salvation for us, justify his wisdom, and act in accordance with his will in the matter. \textbf{A glance at the official and essential relations of the persons of the Holy Trinity to each other and to us, may throw additional light upon our pathway}. Upon this subject flippancy would border upon blasphemy. It is holy ground. He who ventures upon it may well tread with unshod foot, and uncovered head bowed low.}[William Boardman, The Higher Christian Life, p. 99; 1858][https://archive.org/details/higherchristian02boargoog/page/n106/]


\othersQuote{Et puis, encore, le Père est l'auteur et le planificateur du salut par la foi en son Fils ; et quand nous faisons confiance à son Fils, nous honorons le Père, parce que nous acceptons son plan de salut pour nous, justifions sa sagesse, et agissons en accord avec sa volonté dans cette affaire. \textbf{Un regard sur les relations officielles et essentielles des personnes de la Sainte Trinité entre elles et envers nous, peut jeter une lumière supplémentaire sur notre chemin}. Sur ce sujet, la légèreté frôlerait le blasphème. C'est un terrain sacré. Celui qui s'y aventure peut bien marcher pieds nus, et la tête découverte inclinée bien bas.}[William Boardman, The Higher Christian Life, p. 99; 1858][https://archive.org/details/higherchristian02boargoog/page/n106/]


Brother Boardman wants us to take \others{a glance at the official and essential relations} of the three persons of the Holy Trinity. He asserts that \textit{God is one but also three}–\textit{Triune}–by presenting official and essential relations of the persons of the Holy Trinity. His fundamental statement and outline for his thesis is as follows:


Frère Boardman veut que nous jetions \others{un regard sur les relations officielles et essentielles} des trois personnes de la Sainte Trinité. Il affirme que \textit{Dieu est un mais aussi trois}–\textit{Trinitaire}–en présentant les relations officielles et essentielles des personnes de la Sainte Trinité. Sa déclaration fondamentale et le plan de sa thèse sont les suivants :


\othersQuote{\textbf{The Father is fullness of the Godhead \underline{invisibly}, without form, whom no creature hath seen or can see}. \\
\textbf{The Son is the fullness of the Godhead \underline{embodied}, that his creatures may see him, and know him, and trust him}. \\
\textbf{The Spirit is the fullness of the Godhead \underline{in all active workings}, whether of creation, providence, revelation, or salvation, by which God manifests himself to and through the universe}.}[William Boardman, The Higher Christian Life, p. 100][https://archive.org/details/higherchristian02boargoog/page/n108/]


\othersQuote{\textbf{Le Père est la plénitude de la Divinité \underline{invisiblement}, sans forme, qu'aucune créature n'a vue ou ne peut voir}. \\
\textbf{Le Fils est la plénitude de la Divinité \underline{incarnée}, afin que ses créatures puissent le voir, et le connaître, et lui faire confiance}. \\
\textbf{L'Esprit est la plénitude de la Divinité \underline{dans toutes les œuvres actives}, que ce soit de création, providence, révélation, ou salut, par lesquelles Dieu se manifeste à et à travers l'univers}.}[William Boardman, The Higher Christian Life, p. 100][https://archive.org/details/higherchristian02boargoog/page/n108/]


This statement is foundational to his following statements and illustrations. In the following paragraphs, William Boardman gives the biblical motives to illustrate \others{the official and essential relations of the Holy Trinity}—\textit{that is, God being one, but yet three}. He writes:


Cette déclaration est fondamentale pour ses déclarations et illustrations suivantes. Dans les paragraphes suivants, William Boardman donne les motifs bibliques pour illustrer \others{les relations officielles et essentielles de la Sainte Trinité}—\textit{c'est-à-dire, Dieu étant un, mais pourtant trois}. Il écrit :


\othersQuote{Another of the names of Jesus will give the same analogies in a light not less striking - \textbf{The Sun of Righteousness}. \\
All the light of the sun in the heavens was once hidden in the invisibility of primal darkness; and after this, the light now blazing in the orb of day was, when first the command when forth, Let light be! and light was, at most only the diffused haze of the gray dawn of the morn of creation out of the darkness of chaotic night, without form, or body, or centre, or radiance, or glory. But when separated from the darkness and centered in the sun, then in its glorious glitter it became so resplendent that none but the eagle eye could bear to look it in the face. \\
But then again its rays falling aslant through earth’s atmosphere and vapors, gladdens all the world with the same light, dispelling the winter, and the cold, and the darkness; starting Spring forth in floral beauty, and Summer in vernal luxuriance, and Autumn laden with golden treasures for the garner.
\textbf{The Father is as the Light invisible}. \\
\textbf{The Son is as the Light embodied}. \\
\textbf{The Spirit is as the Light shed down}.}[William Boardman, The Higher Christian Life, p. 101,102][https://archive.org/details/higherchristian02boargoog/page/n108/]


\othersQuote{Un autre des noms de Jésus donnera les mêmes analogies dans une lumière non moins frappante - \textbf{Le Soleil de Justice}. \\
Toute la lumière du soleil dans les cieux était autrefois cachée dans l'invisibilité des ténèbres primordiales ; et après cela, la lumière qui brille maintenant dans l'orbe du jour était, quand le commandement sortit pour la première fois, Que la lumière soit ! et la lumière fut, tout au plus seulement la brume diffuse de l'aube grise du matin de la création sortant des ténèbres de la nuit chaotique, sans forme, ni corps, ni centre, ni éclat, ni gloire. Mais quand elle fut séparée des ténèbres et centrée dans le soleil, alors dans son scintillement glorieux elle devint si resplendissante que seul l'œil de l'aigle pouvait supporter de la regarder en face. \\
Mais alors encore ses rayons tombant obliquement à travers l'atmosphère et les vapeurs de la terre, réjouissent le monde entier avec la même lumière, dissipant l'hiver, et le froid, et les ténèbres ; faisant jaillir le Printemps dans sa beauté florale, et l'Été dans sa luxuriance verdoyante, et l'Automne chargé de trésors dorés pour le grenier.
\textbf{Le Père est comme la Lumière invisible}. \\
\textbf{Le Fils est comme la Lumière incarnée}. \\
\textbf{L'Esprit est comme la Lumière répandue}.}[William Boardman, The Higher Christian Life, p. 101,102][https://archive.org/details/higherchristian02boargoog/page/n108/]


This illustration of the Sun of Righteousness shows that God the Father, who is \textit{the fullness of the Godhead invisible,} can be symbolically illustrated as a Light that \others{was once hidden in the invisibility of primal darkness}. The Son, who is \textit{the fullness of the Godhead embodied}, is like a Light that is embodied in \others{the morn of creation}. The Holy Spirit, who is \textit{the fullness of the Godhead in all active workings}, is like a \others{Light shed down}. William Boardman gives us another similar illustration to clarify the \others{official relations of the persons of the Godhead}:


Cette illustration du Soleil de Justice montre que Dieu le Père, qui est \textit{la plénitude de la Divinité invisible}, peut être symboliquement illustré comme une Lumière qui \others{était autrefois cachée dans l'invisibilité des ténèbres primordiales}. Le Fils, qui est \textit{la plénitude de la Divinité incarnée}, est comme une Lumière qui est incarnée dans \others{le matin de la création}. Le Saint-Esprit, qui est \textit{la plénitude de la Divinité dans toutes les œuvres actives}, est comme une \others{Lumière répandue}. William Boardman nous donne une autre illustration similaire pour clarifier les \others{relations officielles des personnes de la Divinité} :


\othersQuote{One of the similies for blessed influences of the Spirit, \textbf{while giving the self-same official relations of the persons of the Godhead, to each other and to us}, may illustrate them still further,—\textbf{The Dew},—\textbf{The dew of Hermon} - the dew on the mown meadow. Before the dew gathers at all in drops, it hangs over all the landscape in visible vapor, omnipresent but unseen. By and by as the light wanes into morning, and as the temperature sinks and touches the dew point the invisible becomes the visible, the embodied; and, as the sun rises, it stands in diamond drops trembling and glittering in the sun’s young beams in pearly beauty upon leaf and flower, over all the face of nature. \\
But now again, a breeze springs up, the breath of heaven is wafted gently along, shaking leaf and flower, and in a moment the pearly drops are invisible angina. But where now? Fallen at the root of herb and flower to impart new life, freshness, vigor to all it touches. \\
\textbf{The Father is like the dew in invisible vapor}. \\
\textbf{The Son is like the dew gathered in beauteous form}. \\
\textbf{The Spirit is like the dew fallen to the seat of life}.}[William Boardman, The Higher Christian Life, p. 102,103][https://archive.org/details/higherchristian02boargoog/page/n110/]


\othersQuote{Une des similitudes pour les influences bénies de l'Esprit, \textbf{tout en donnant les mêmes relations officielles des personnes de la Divinité, les unes envers les autres et envers nous}, peut les illustrer encore davantage,—\textbf{La Rosée},—\textbf{La rosée d'Hermon} - la rosée sur le pré fauché. Avant que la rosée ne se rassemble du tout en gouttes, elle plane sur tout le paysage en vapeur visible, omniprésente mais invisible. Peu à peu, alors que la lumière décline vers le matin, et que la température baisse et touche le point de rosée, l'invisible devient le visible, l'incarné ; et, alors que le soleil se lève, elle se tient en gouttes de diamant tremblant et scintillant dans les jeunes rayons du soleil en beauté nacrée sur la feuille et la fleur, sur toute la face de la nature. \\
Mais maintenant encore, une brise se lève, le souffle du ciel est doucement porté, secouant feuille et fleur, et en un instant les gouttes nacrées sont invisibles à nouveau. Mais où maintenant ? Tombées à la racine de l'herbe et de la fleur pour impartir une nouvelle vie, fraîcheur, vigueur à tout ce qu'elle touche. \\
\textbf{Le Père est comme la rosée en vapeur invisible}. \\
\textbf{Le Fils est comme la rosée rassemblée en forme magnifique}. \\
\textbf{L'Esprit est comme la rosée tombée au siège de la vie}.}[William Boardman, The Higher Christian Life, p. 102,103][https://archive.org/details/higherchristian02boargoog/page/n110/]


The Father, who is \textit{the fullness of the Godhead invisible,} is illustrated by the \others{dew in invisible vapor}. The Son, who is \textit{the fullness of the Godhead embodied}, is illustrated by \others{the dew gathered in beauteous form}. The Spirit, who is \textit{the fullness of the Godhead in all active works}, is illustrated by \others{the dew fallen to the seat of life}. The next illustration that exemplifies the official relations of the three personalities of one God is by another Bible likening—the Rain.


Le Père, qui est \textit{la plénitude de la Divinité invisible}, est illustré par la \others{rosée en vapeur invisible}. Le Fils, qui est \textit{la plénitude de la Divinité incarnée}, est illustré par \others{la rosée rassemblée en forme magnifique}. L'Esprit, qui est \textit{la plénitude de la Divinité dans toutes les œuvres actives}, est illustré par \others{la rosée tombée au siège de la vie}. La prochaine illustration qui exemplifie les relations officielles des trois personnalités d'un seul Dieu est par une autre comparaison biblique—la Pluie.


\othersQuote{\textbf{Yet one more of these Bible likenings} – by no means exhausting them – will not be unwelcome, or useless, - \textbf{the Rain}. \\
Rain, like the dew, floats in invisibility, and omnipresence at the first, over all, around all. Seen by none. While it remains in its invisibility, the earth parches, clods cleave together, the ground cracks open, the sun pours down his burning heat, the winds lift up the dust in circling whirls, and rolling clouds, and famine gaunt and greedy stalks through the land, followed by pestilence and death. By and by, the eager watcher sees the little hand-like cloud rising far out over the sea. It gathers, gathers, gathers; comes and spreads as it comes, in majesty over the whole heavens: - But all is parched and dry and dead yet, upon earth. \\
But now comes a drop, and drop after drop, quicker, faster – the shower, the rain – sweeping on, and giving to earth all the treasures of the clouds – clods open, furrows soften, springs, rivulets, rivers, swell and fill, and all the land is gladdened again with restored abundance. \\
\textbf{The Father is like to the invisible vapor}. \\
\textbf{The Son is as the laden cloud and falling rain}. \\
\textbf{The Spirit is the Rain – fallen and working in refreshing power}.}[William Boardman, The Higher Christian Life, p. 103,104][https://archive.org/details/higherchristian02boargoog/page/n110/]


\othersQuote{\textbf{Encore une de ces comparaisons bibliques} – sans pour autant les épuiser – ne sera pas malvenue, ou inutile, - \textbf{la Pluie}. \\
La pluie, comme la rosée, flotte dans l'invisibilité, et l'omniprésence au début, sur tout, autour de tout. Vue par personne. Tant qu'elle reste dans son invisibilité, la terre se dessèche, les mottes se collent ensemble, le sol se fissure, le soleil déverse sa chaleur brûlante, les vents soulèvent la poussière en tourbillons circulaires, et en nuages roulants, et la famine maigre et avide parcourt le pays, suivie par la pestilence et la mort. Peu à peu, l'observateur impatient voit le petit nuage semblable à une main s'élever loin au-dessus de la mer. Il se rassemble, se rassemble, se rassemble ; vient et s'étend en venant, en majesté sur tous les cieux : - Mais tout est encore desséché et sec et mort, sur terre. \\
Mais maintenant vient une goutte, et goutte après goutte, plus vite, plus rapidement – l'averse, la pluie – balayant, et donnant à la terre tous les trésors des nuages – les mottes s'ouvrent, les sillons s'adoucissent, les sources, les ruisselets, les rivières, gonflent et se remplissent, et tout le pays est à nouveau réjoui avec l'abondance restaurée. \\
\textbf{Le Père est comme la vapeur invisible}. \\
\textbf{Le Fils est comme le nuage chargé et la pluie qui tombe}. \\
\textbf{L'Esprit est la Pluie – tombée et œuvrant en puissance rafraîchissante}.}[William Boardman, The Higher Christian Life, p. 103,104][https://archive.org/details/higherchristian02boargoog/page/n110/]


Let's give William Boardman a fair hearing. He is not saying that the Father is \others{invisible vapor}; rather, he uses a metaphor of rain and \others{invisible vapor} to illustrate his main point that the Father is the invisible fullness of the Godhead. So it is with the Son, who, just like rain manifested in leaden clouds, is all the fullness of the Godhead manifested. To ensure his sentiments are not potentially misrepresented, William Boardman clarified his sentiment. This was the very sentiment that Ellen White was instructed by God not to trust:


Donnons à William Boardman une écoute équitable. Il ne dit pas que le Père est \others{vapeur invisible} ; plutôt, il utilise une métaphore de pluie et de \others{vapeur invisible} pour illustrer son point principal que le Père est la plénitude invisible de la Divinité. Il en est de même avec le Fils, qui, tout comme la pluie manifestée dans les nuages chargés, est toute la plénitude de la Divinité manifestée. Pour s'assurer que son raisonnement ne soit pas potentiellement mal représenté, William Boardman a clarifié son raisonnement. C'était le raisonnement même qu'Ellen White avait reçu l'instruction de Dieu de ne pas faire confiance :


\othersQuote{\textbf{These likenings are all imperfect. They rather hide than illustrate \underline{the tri-personality of the one God}, for they are not persons but things, poor and earthly at best, to represent the living personalities of the living God. So much they may do, however, as to illustrate the official relations of each to the others and of each and all to us. And more. They may also illustrate the truth that all the fulness of Him who filleth all in all, dwells in each person of \underline{the Triune God}}. \\
\textbf{The Father is all the fulness of the Godhead INVISIBLE}. \\
\textbf{The Son is all the fulness of the Godhead MANIFESTED}. \\
\textbf{The Spirit is all the fulness of the Godhead MAKING MANIFEST}. \\
\textbf{The persons are not mere offices, or modes of revelation, but living persons of the living God}.}[William Boardman, The Higher Christian Life, p. 104,105][https://archive.org/details/higherchristian02boargoog/page/n112/]


\othersQuote{\textbf{Ces comparaisons sont toutes imparfaites. Elles cachent plutôt qu'elles n'illustrent \underline{la tri-personnalité du Dieu unique}, car ce ne sont pas des personnes mais des choses, pauvres et terrestres au mieux, pour représenter les personnalités vivantes du Dieu vivant. Elles peuvent cependant faire ceci, illustrer les relations officielles de chacun avec les autres et de chacun et de tous avec nous. Et plus encore. Elles peuvent aussi illustrer la vérité que toute la plénitude de Celui qui remplit tout en tous, habite en chaque personne du \underline{Dieu Trinitaire}}. \\
\textbf{Le Père est toute la plénitude de la Divinité INVISIBLE}. \\
\textbf{Le Fils est toute la plénitude de la Divinité MANIFESTÉE}. \\
\textbf{L'Esprit est toute la plénitude de la Divinité RENDANT MANIFESTE}. \\
\textbf{Les personnes ne sont pas de simples offices, ou modes de révélation, mais des personnes vivantes du Dieu vivant}.}[William Boardman, The Higher Christian Life, p. 104,105][https://archive.org/details/higherchristian02boargoog/page/n112/]


It is crucial to emphasize that when Boardman uses these Bible likenings from nature, he speaks of the illustrations, and not reality. These representations are illustrating his sentiments. In his own admission, that was the sentiment of three \others{living personalities of the living God.} Though these illustrations are imperfect, they may \others{illustrate the official relations} of \others{the tri-personality of the one God} and \others{the truth that all the fullness of Him who filleth all in all dwells in each person of the Triune God.} One God in three persons is the sentiment in question, and that sentiment is common to all types and versions of the trinity doctrine—including our current trinitarian stance in the second point of the Fundamental Beliefs.\footnote{\others{There is \textbf{one God}: Father, Son, and Holy Spirit, \textbf{a unity of three} coeternal \textbf{Persons}…} 2nd point of the Fundamental Beliefs}


Il est crucial de souligner que lorsque Boardman utilise ces comparaisons bibliques tirées de la nature, il parle des illustrations, et non de la réalité. Ces représentations illustrent son raisonnement. De son propre aveu, c'était le raisonnement de trois \others{personnalités vivantes du Dieu vivant.} Bien que ces illustrations soient imparfaites, elles peuvent \others{illustrer les relations officielles} de \others{la tri-personnalité du Dieu unique} et \others{la vérité que toute la plénitude de Celui qui remplit tout en tous habite en chaque personne du Dieu Trinitaire.} Un Dieu en trois personnes est le raisonnement en question, et ce raisonnement est commun à tous les types et versions de la doctrine de la Trinité—y compris notre position trinitaire actuelle dans le deuxième point des croyances fondamentales.\footnote{\others{Il y a \textbf{un seul Dieu} : Père, Fils et Saint-Esprit, \textbf{une unité de trois} \textbf{Personnes} coéternelles…} 2ème point des croyances fondamentales}


In this brief look at William Boardman's sentiments, it is clear that the sentiments in question which Ellen White was instructed by God to call out, were the sentiments of the Triune God, or \textit{three living persons in the Trinity}. With that data in mind, let's examine Ellen White's response.


Dans ce bref aperçu du raisonnement de William Boardman, il est clair que le raisonnement en question qu'Ellen White a été instruite par Dieu de dénoncer, était le raisonnement du Dieu Trinitaire, ou \textit{trois personnes vivantes dans la Trinité}. Avec ces données à l'esprit, examinons la réponse d'Ellen White.


\section*{Ellen White on William Boardman’s sentiment}


\section*{Ellen White sur le raisonnement de William Boardman}


With the Heavenly Trio quotation, it has been asserted that Ellen White was trinitarian. This is done by ignorantly or sometimes purposely ignoring the context of this valuable quotation. When reading Ellen White’s response, in which she defends our perceptions of God, try to recognize whom she is addressing when she speaks of God. Was the God she defended the Trinity or the Father? Referencing William Boardmans illustrations she said:


Avec la citation du Trio Céleste, il a été affirmé qu'Ellen White était trinitaire. Cela se fait en ignorant de manière ignorante ou parfois volontaire le contexte de cette précieuse citation. En lisant la réponse d'Ellen White, dans laquelle elle défend nos perceptions de Dieu, essayez de reconnaître à qui elle s'adresse lorsqu'elle parle de Dieu. Le Dieu qu'elle défendait était-il la Trinité ou le Père ? En référence aux illustrations de William Boardman, elle a dit :


\egw{\textbf{All these \underline{spiritualistic} representations are simply nothingness}. They are imperfect, untrue. They weaken and diminish the Majesty which no earthly likeness can be compared to. \textbf{God cannot be compared with the things His hands have made}. These are mere earthly things, suffering under the curse of God because of the sins of man. \textbf{The Father cannot be described by the things of earth}. \textbf{The Father is all the fulness of the Godhead \underline{bodily} and is \underline{invisible to mortal sight}}.}[Ms21-1906.9; 1906][https://egwwritings.org/read?panels=p9754.15]


\egw{\textbf{Toutes ces représentations \underline{spiritualistes} ne sont simplement rien}. Elles sont imparfaites, fausses. Elles affaiblissent et diminuent la Majesté à laquelle aucune ressemblance terrestre ne peut être comparée. \textbf{Dieu ne peut être comparé aux choses que Ses mains ont faites}. Ce ne sont que de simples choses terrestres, souffrant sous la malédiction de Dieu à cause des péchés de l'homme. \textbf{Le Père ne peut être décrit par les choses de la terre}. \textbf{Le Père est toute la plénitude de la Divinité \underline{corporellement} et est \underline{invisible aux yeux mortels}}.}[Ms21-1906.9; 1906][https://egwwritings.org/read?panels=p9754.15]


By observing the context, it is obvious that Sister White follows Boardman’s line of reasoning and corrects the mistakes. For better comparison, let us look at their writings side by side:


En observant le contexte, il est évident que Sœur White suit le raisonnement de Boardman et corrige les erreurs. Pour une meilleure comparaison, regardons leurs écrits côte à côte :


\begin{table}[H]
\centering
\renewcommand{\arraystretch}{1.5}
\setlength{\tabcolsep}{15pt}
\resizebox{\textwidth}{!}{
\begin{tabular}{|p{0.4\textwidth}|p{0.4\textwidth}|}
\hline
\multicolumn{1}{|c|}{\textbf{William Boardman}} & \multicolumn{1}{c|}{\textbf{Ellen G. White}} \\ \hline
\othersQuote{These likenings are all imperfect. They rather hide than \textbf{illustrate the tri-personality of the \underline{one God}}, for they are not persons but things, poor and earthly at best, to represent \textbf{the living personalities of the living God}. \textbf{So much they may do, however, as to illustrate the official relations of each to the other and of each and all to us. And more. They may also illustrate the truth that all the fulness of Him who filleth all in all, dwells in \underline{each person of Triune God}}.}[p. 104,105][https://archive.org/details/higherchristian02boargoog/page/n112] & 
\egw{\textbf{All these \underline{spiritualistic} representations are simply nothingness}. They are imperfect, untrue. They weaken and diminish the Majesty which no earthly likeness can be compared to. \textbf{God cannot be compared with the things His hands have made}. These are mere earthly things, suffering under the curse of God because of the sins of man. \textbf{The Father cannot be described by the things of earth}.}[Ms21-1906.9; 1906][https://egwwritings.org/read?panels=p9754.15] \\ \hline
\end{tabular}
}
\end{table}


\begin{table}[H]
\centering
\renewcommand{\arraystretch}{1.5}
\setlength{\tabcolsep}{15pt}
\resizebox{\textwidth}{!}{
\begin{tabular}{|p{0.4\textwidth}|p{0.4\textwidth}|}
\hline
\multicolumn{1}{|c|}{\textbf{William Boardman}} & \multicolumn{1}{c|}{\textbf{Ellen G. White}} \\ \hline
\othersQuote{Ces comparaisons sont toutes imparfaites. Elles cachent plutôt qu'elles n’\textbf{illustrent la tri-personnalité du \underline{Dieu unique}}, car ce ne sont pas des personnes mais des choses, pauvres et terrestres au mieux, pour représenter \textbf{les personnalités vivantes du Dieu vivant}. \textbf{Elles peuvent cependant faire ceci, illustrer les relations officielles de chacun avec l'autre et de chacun et de tous avec nous. Et plus encore. Elles peuvent aussi illustrer la vérité que toute la plénitude de Celui qui remplit tout en tous, habite en \underline{chaque personne du Dieu Trinitaire}}.}[p. 104,105][https://archive.org/details/higherchristian02boargoog/page/n112] & 
\egw{\textbf{Toutes ces représentations \underline{spiritualistes} ne sont simplement rien}. Elles sont imparfaites, fausses. Elles affaiblissent et diminuent la Majesté à laquelle aucune ressemblance terrestre ne peut être comparée. \textbf{Dieu ne peut être comparé aux choses que Ses mains ont faites}. Ce ne sont que de simples choses terrestres, souffrant sous la malédiction de Dieu à cause des péchés de l'homme. \textbf{Le Père ne peut être décrit par les choses de la terre}.}[Ms21-1906.9; 1906][https://egwwritings.org/read?panels=p9754.15] \\ \hline
\end{tabular}
}
\end{table}


In this comparison, it is clear who God is for William Boardman, and who He is for Sister White. For Boardman, God is the Triune God, a tri-personality of the one God. For Sister White, God is the Father. For Boardman, these representations are imperfect because they \others{rather hide than illustrate the tri-personality of the one God}, and for Sister White these representations are imperfect because \egw{The Father cannot be described by the things of earth}. For Boardman, God is the \textit{Triune God}; for Sister White, God is \textit{the Father}.


Dans cette comparaison, il est clair qui est Dieu pour William Boardman, et qui Il est pour Sœur White. Pour Boardman, Dieu est le Dieu Trinitaire, une tri-personnalité du Dieu unique. Pour Sœur White, Dieu est le Père. Pour Boardman, ces représentations sont imparfaites parce qu'elles \others{cachent plutôt qu'elles n'illustrent la tri-personnalité du Dieu unique}, et pour Sœur White ces représentations sont imparfaites parce que \egw{Le Père ne peut être décrit par les choses de la terre}. Pour Boardman, Dieu est le \textit{Dieu Trinitaire} ; pour Sœur White, Dieu est \textit{le Père}.


Boardman’s only point that Ellen White affirms is that these representations are imperfect. Surely, William Boardman would not agree with Ellen White that these representations are \textit{spiritualistic} and \textit{untrue}. On the contrary, he believes that these illustrations \others{illustrate the truth that all the fulness of Him who filleth all in all, dwells in each person of Triune God}. To say that Ellen White agreed with such sentiment is gross misrepresentation.


Le seul point de Boardman qu'Ellen White affirme est que ces représentations sont imparfaites. Assurément, William Boardman ne serait pas d'accord avec Ellen White que ces représentations sont \textit{spiritualistes} et \textit{fausses}. Au contraire, il croit que ces illustrations \others{illustrent la vérité que toute la plénitude de Celui qui remplit tout en tous, habite en chaque personne du Dieu Trinitaire}. Dire qu'Ellen White était d'accord avec un tel raisonnement est une déformation grossière.


The context of this important quotation prompts important questions. Why does the prophet of God refer to the representations that illustrate the \others{tri-personality of the one God} as \egwinline{spiritualistic representations}, which illustrate the sentiment that \egwinline{is not to be trusted}? Or why does the prophet of God refer to the representations that \others{represent the living personalities of the living God} as \egwinline{spiritualistic representations}? Or why does the prophet of God, when referring to the representations that \others{illustrate the truth that all the fullness of Him who filleth all in all, dwells in each person of Triune God}, refer to them as \egwinline{spiritualistic representations}? All of these spiritualistic representations illustrate the sentiment that \egwinline{is not to be trusted}. This sentiment is clearly the trinitarian sentiment.


Le contexte de cette citation importante suscite des questions importantes. Pourquoi la prophétesse de Dieu se réfère-t-elle aux représentations qui illustrent la \others{tri-personnalité du Dieu unique} comme des \egwinline{représentations spiritualistes}, qui illustrent le raisonnement qui \egwinline{ne doit pas être digne de confiance} ? Ou pourquoi la prophétesse de Dieu se réfère-t-elle aux représentations qui \others{représentent les personnalités vivantes du Dieu vivant} comme des \egwinline{représentations spiritualistes} ? Ou pourquoi la prophétesse de Dieu, en se référant aux représentations qui \others{illustrent la vérité que toute la plénitude de Celui qui remplit tout en tous, habite en chaque personne du Dieu Trinitaire}, s'y réfère-t-elle comme des \egwinline{représentations spiritualistes} ? Toutes ces représentations spiritualistes illustrent le raisonnement qui \egwinline{ne doit pas être digne de confiance}. Ce raisonnement est clairement le raisonnement trinitaire.


Sister White continues to follow Boardman’s line of reasoning and corrects the error.


Sœur White continue de suivre le raisonnement de Boardman et corrige l'erreur.


\begin{table}[H]
\centering
\renewcommand{\arraystretch}{1.5}
\setlength{\tabcolsep}{15pt}
\resizebox{\textwidth}{!}{
\begin{tabular}{|p{0.4\textwidth}|p{0.4\textwidth}|}
\hline
\multicolumn{1}{|c|}{\textbf{William Boardman}} & \multicolumn{1}{c|}{\textbf{Ellen G. White}} \\ \hline
\othersQuote{The Father is fullness of the Godhead \textbf{invisibly}, \textbf{\underline{without form}}, whom \textbf{no creature hath seen \underline{or can see}}.}[p.100][https://archive.org/details/higherchristian02boargoog/page/n108/]


\begin{table}[H]
\centering
\renewcommand{\arraystretch}{1.5}
\setlength{\tabcolsep}{15pt}
\resizebox{\textwidth}{!}{
\begin{tabular}{|p{0.4\textwidth}|p{0.4\textwidth}|}
\hline
\multicolumn{1}{|c|}{\textbf{William Boardman}} & \multicolumn{1}{c|}{\textbf{Ellen G. White}} \\ \hline
\othersQuote{Le Père est la plénitude de la Divinité \textbf{invisiblement}, \textbf{\underline{sans forme}}, qu’\textbf{aucune créature n'a vue \underline{ou ne peut voir}}.}[p.100][https://archive.org/details/higherchristian02boargoog/page/n108/]


\othersQuote{The Father is all the fullness of the Godhead \textbf{INVISIBLE}.}[p.105][https://archive.org/details/higherchristian02boargoog/page/n112/] & 
\egw{The Father is all the fulness of the Godhead \textbf{\underline{bodily}}, and is \textbf{invisible to mortal sight}.}[Ms21-1906.9; 1906][https://egwwritings.org/read?panels=p9754.15] \\ \hline
\end{tabular}
}
\end{table}


\othersQuote{Le Père est toute la plénitude de la Divinité \textbf{INVISIBLE}.}[p.105][https://archive.org/details/higherchristian02boargoog/page/n112/] & 
\egw{Le Père est toute la plénitude de la Divinité \textbf{\underline{corporellement}}, et est \textbf{invisible à la vue mortelle}.}[Ms21-1906.9; 1906][https://egwwritings.org/read?panels=p9754.15] \\ \hline
\end{tabular}
}
\end{table}


For Boardman, the Father does not have a form nor body and is invisible to all creatures. For Sister White, the Father has a form and body and is invisible only to mortal human beings.\footnote{When Sister White talks about mortals, she talks about sin polluted humanity. After the restoration of humanity, at the resurrection, Christ will give His immortal life to His children. For more information read \href{https://egwwritings.org/?ref=en_RH.July.5.1887.par.5}{EGW, RH July 5, 1887, par. 5; 1887}.}


Pour Boardman, le Père n'a ni forme ni corps et est invisible à toutes les créatures. Pour Sœur White, le Père a une forme et un corps et est invisible seulement aux êtres humains mortels.\footnote{Quand Sœur White parle des mortels, elle parle de l'humanité polluée par le péché. Après la restauration de l'humanité, à la résurrection, Christ donnera Sa vie immortelle à Ses enfants. Pour plus d'informations, lisez \href{https://egwwritings.org/?ref=en_RH.July.5.1887.par.5}{EGW, RH July 5, 1887, par. 5; 1887}.}


This quotation is one of the most direct quotations regarding the \emcap{personality of God}. \egwinline{The Father is all the fullness of the Godhead \textbf{bodily}}[Ms21-1906.9; 1906][https://egwwritings.org/read?panels=p9754.16].


Cette citation est l'une des citations les plus directes concernant la \emcap{personnalité de Dieu}. \egwinline{Le Père est toute la plénitude de la Divinité \textbf{corporellement}}[Ms21-1906.9; 1906][https://egwwritings.org/read?panels=p9754.16].


It might be confusing to someone that the Father is all the fullness of the Godhead bodily because in \textit{Colossians 2:9}, when referring to Jesus, it is written that \bible{in him dwelleth all the fulness of the Godhead bodily.} Scripture does not contradict itself. \textit{Colossians 2:9} does not exclude the Father to be all the fulness of the Godhead bodily. Various places in the Bible describe the Father having a body (\textit{a form: Daniel 7:9,10; Revelation 4:2,3; 1 Kings 22:19-22; a shape: John 5:37}). He has the appearance of a man (\textit{Ezekiel 1:26-28}). He has a face (\textit{Exodus 33:20; Matthew 18:10; Revelation 22:3, 4}). However, the Bible is completely silent about the nature of its substance. The Bible teaches us that \bible{\textbf{The secret things belong unto the LORD our God}: \textbf{but those things which \underline{are revealed} belong unto us and to our children for ever}, that we may do all the words of this law}[Deuteronomy 29:29]. It is revealed to us that the Father has body, He is all the fulness of the Godhead bodily. Also, it is revealed that in Jesus also dwells all the fulness of the Godhead bodily, because \bible{it pleased the Father that in him should all fulness dwell}[Colossians 1:19]. This is not a contradiction whatsoever because the Son is \bible{the \textbf{express image of \underline{His person}}}[Hebrews 1:3].


Il pourrait être déroutant pour quelqu'un que le Père soit toute la plénitude de la Divinité corporellement parce que dans \textit{Colossiens 2:9}, en se référant à Jésus, il est écrit qu’\bible{en lui habite corporellement toute la plénitude de la divinité.} L'Écriture ne se contredit pas. \textit{Colossiens 2:9} n'exclut pas que le Père soit toute la plénitude de la Divinité corporellement. Divers endroits dans la Bible décrivent le Père ayant un corps (\textit{une forme : Daniel 7:9,10 ; Apocalypse 4:2,3 ; 1 Rois 22:19-22 ; une apparence : Jean 5:37}). Il a l'apparence d'un homme (\textit{Ézéchiel 1:26-28}). Il a un visage (\textit{Exode 33:20 ; Matthieu 18:10 ; Apocalypse 22:3, 4}). Cependant, la Bible est complètement silencieuse sur la nature de sa substance. La Bible nous enseigne que \bible{\textbf{Les choses cachées sont à l'Éternel, notre Dieu} ; \textbf{les choses \underline{révélées} sont à nous et à nos enfants, à perpétuité}, afin que nous mettions en pratique toutes les paroles de cette loi}[Deutéronome 29:29]. Il nous est révélé que le Père a un corps, Il est toute la plénitude de la Divinité corporellement. De plus, il est révélé qu'en Jésus habite aussi toute la plénitude de la Divinité corporellement, parce qu’\bible{il a plu au Père qu'en lui habitât toute plénitude}[Colossiens 1:19]. Ce n'est aucunement une contradiction parce que le Fils est \bible{\textbf{l'empreinte de \underline{sa personne}}}[Hébreux 1:3].


\begin{table}[H]
\centering
\renewcommand{\arraystretch}{1.5}
\setlength{\tabcolsep}{15pt}
\resizebox{\textwidth}{!}{
\begin{tabular}{|p{0.4\textwidth}|p{0.4\textwidth}|}
\hline
\multicolumn{1}{|c|}{\textbf{William Boardman}} & \multicolumn{1}{c|}{\textbf{Ellen G. White}} \\ \hline
\othersQuote{The Son is the fullness of the Godhead \textbf{embodied, that his creatures may see him, and know him, and trust him}.}[p.100][https://archive.org/details/higherchristian02boargoog/page/n108/]


\begin{table}[H]
\centering
\renewcommand{\arraystretch}{1.5}
\setlength{\tabcolsep}{15pt}
\resizebox{\textwidth}{!}{
\begin{tabular}{|p{0.4\textwidth}|p{0.4\textwidth}|}
\hline
\multicolumn{1}{|c|}{\textbf{William Boardman}} & \multicolumn{1}{c|}{\textbf{Ellen G. White}} \\ \hline
\othersQuote{Le Fils est la plénitude de la Divinité \textbf{incarnée, afin que ses créatures puissent le voir, et le connaître, et lui faire confiance}.}[p.100][https://archive.org/details/higherchristian02boargoog/page/n108/]


\othersQuote{The Son is all the fulness of the Godhead \textbf{MANIFESTED}.}[p.105][https://archive.org/details/higherchristian02boargoog/page/n112/] & 
\egw{The Son is all the fulness of the Godhead \textbf{manifested}. The Word of God declares Him to be ‘\textbf{the express image of His person}’. ‘God so loved the world that He gave \textbf{His only begotten Son}, that whosoever believeth in Him should not perish, but have everlasting life’. \textbf{Here is shown \underline{the personality of the Father}}.}[Ms21-1906.10; 1906][https://egwwritings.org/read?panels=p9754.17] \\ \hline
\end{tabular}
}
\end{table}


\othersQuote{Le Fils est toute la plénitude de la Divinité \textbf{MANIFESTÉE}.}[p.105][https://archive.org/details/higherchristian02boargoog/page/n112/] & 
\egw{Le Fils est toute la plénitude de la Divinité \textbf{manifestée}. La Parole de Dieu Le déclare être ‘\textbf{l'empreinte de sa personne}’. ‘Car Dieu a tant aimé le monde qu'il a donné \textbf{son seul Fils engendré}, afin que quiconque croit en lui ne périsse point, mais qu'il ait la vie éternelle’. \textbf{Ici est montrée \underline{la personnalité du Père}}.}[Ms21-1906.10; 1906][https://egwwritings.org/read?panels=p9754.17] \\ \hline
\end{tabular}
}
\end{table}


Sister White focused on the \emcap{personality of God}, which is the personality of the Father. In Christ, who is \egwinline{begotten in the express image of the Father’s person}[ST May 30, 1895, par. 3; 1895][https://egwwritings.org/read?panels=p820.12891], is shown the personality of the Father. In the same way that Jesus is a person, so is the Father. The quality or state of Christ being a person is the same quality or state of the Father being a person. As Christ is a personal being, so is the Father. Just as all the fullness of the Godhead bodily dwells in Christ, so it does in the Father, because Christ is begotten in the express image of the Father’s person. In Him is shown the personality of the Father. These simple conclusions have been asserted by Scripture in John 3:16 and Hebrews 1:3.


Sœur White s'est concentrée sur la \emcap{personnalité de Dieu}, qui est la personnalité du Père. En Christ, qui est \egwinline{engendré à l'empreinte expresse de la personne du Père}[ST May 30, 1895, par. 3; 1895][https://egwwritings.org/read?panels=p820.12891], est montrée la personnalité du Père. De la même manière que Jésus est une personne, ainsi l'est le Père. La qualité ou l'état de Christ d'être une personne est la même qualité ou l'état du Père d'être une personne. Comme Christ est un être personnel, ainsi l'est le Père. Tout comme toute la plénitude de la Divinité corporellement habite en Christ, ainsi en est-il dans le Père, parce que Christ est engendré à l'empreinte expresse de la personne du Père. En Lui est montrée la personnalité du Père. Ces conclusions simples ont été affirmées par l'Écriture dans Jean 3:16 et Hébreux 1:3.


Does the same reasoning, of the personality of the Father and Son, apply to the Holy Spirit? Speaking of the Holy Spirit, Sister White continues:


Le même raisonnement, de la personnalité du Père et du Fils, s'applique-t-il au Saint-Esprit ? Parlant du Saint-Esprit, Sœur White continue :


\egw{\textbf{The Comforter that Christ} promised to send after He ascended to heaven, \textbf{is the Spirit \underline{in} all the fulness of the Godhead}, making manifest the power of divine grace to all who receive and believe in Christ as a personal Saviour.}[Ms21-1906.11; 1906][https://egwwritings.org/read?panels=p9754.18]


\egw{\textbf{Le Consolateur que Christ} a promis d'envoyer après qu'Il soit monté au ciel, \textbf{est l'Esprit \underline{dans} toute la plénitude de la Divinité}, manifestant la puissance de la grâce divine à tous ceux qui reçoivent et croient en Christ comme un Sauveur personnel.}[Ms21-1906.11; 1906][https://egwwritings.org/read?panels=p9754.18]


Sister White draws a distinction between Father and Son who \textbf{are}, individually, \textbf{all} the fullness of the Godhead, and the Spirit that is \textbf{in all} the fullness of the Godhead. This is a marked contrast to William Boardman’s reasoning, where all three are the fullness of the Godhead. Sister White does not follow this trinitarian fashion. The explanation is simple in light of the \emcap{personality of God} and of Christ. The Holy Spirit is a spirit, and the spirit dwells \textbf{in} the flesh/body. The Holy Spirit is \textbf{in all} the fullness of the Godhead\footnote{Take a look at the quotation from \href{https://egwwritings.org/?ref=en_Ms128-1897.13&para=5426.19}{{EGW, Ms128-1897.13; 1897}}, where Sister White states that the Father and the Son are the absolute Godhead.}.


Sœur White établit une distinction entre le Père et le Fils qui \textbf{sont}, individuellement, \textbf{toute} la plénitude de la Divinité, et l'Esprit qui est \textbf{dans toute} la plénitude de la Divinité. C'est un contraste marqué avec le raisonnement de William Boardman, où tous les trois sont la plénitude de la Divinité. Sœur White ne suit pas cette mode trinitaire. L'explication est simple à la lumière de la \emcap{personnalité de Dieu} et de Christ. Le Saint-Esprit est un esprit, et l'esprit habite \textbf{dans} la chair/le corps. Le Saint-Esprit est \textbf{dans toute} la plénitude de la Divinité\footnote{Regardez la citation de \href{https://egwwritings.org/?ref=en_Ms128-1897.13&para=5426.19}{{EGW, Ms128-1897.13; 1897}}, où Sœur White déclare que le Père et le Fils sont la Divinité absolue.}.


Finally, the quotation continues to its most renowned part:


Finalement, la citation continue vers sa partie la plus renommée :


\begin{table}[H]
    \centering
    \renewcommand{\arraystretch}{1.5}
    \setlength{\tabcolsep}{15pt}
    \resizebox{\textwidth}{!}{
    \begin{tabular}{|p{0.4\textwidth}|p{0.4\textwidth}|}
    \hline
    \multicolumn{1}{|c|}{\textbf{William Boardman}} & \multicolumn{1}{c|}{\textbf{Ellen G. White}} \\ \hline
    \othersQuote{\textbf{The Father} is all the fulness of the Godhead INVISIBLE.}


\begin{table}[H]
    \centering
    \renewcommand{\arraystretch}{1.5}
    \setlength{\tabcolsep}{15pt}
    \resizebox{\textwidth}{!}{
    \begin{tabular}{|p{0.4\textwidth}|p{0.4\textwidth}|}
    \hline
    \multicolumn{1}{|c|}{\textbf{William Boardman}} & \multicolumn{1}{c|}{\textbf{Ellen G. White}} \\ \hline
    \othersQuote{\textbf{Le Père} est toute la plénitude de la Divinité INVISIBLE.}


\othersQuote{\textbf{The Son} is all the fulness of the Godhead MANIFESTED.}


\othersQuote{\textbf{Le Fils} est toute la plénitude de la Divinité MANIFESTÉE.}


\othersQuote{\textbf{The Spirit} is all the fulness of the Godhead MAKING MANIFEST.}


\othersQuote{\textbf{L'Esprit} est toute la plénitude de la Divinité MANIFESTANT.}


\othersQuote{\textbf{The persons} are not mere offices, or modes of revelation, \textbf{but living persons of the living God}.}[p.105][https://archive.org/details/higherchristian02boargoog/page/n112/] & 
    \egw{There are \textbf{three living persons of the heavenly trio}; in the name of these three great powers—\textbf{the Father, the Son, and the Holy Spirit}—those who receive Christ by living faith are baptized, and these powers will co-operate with the obedient subjects of heaven in their efforts to live the new life in Christ.}[Ms21-1906.11; 1906][https://egwwritings.org/read?panels=p9754.18] \\ \hline
    \end{tabular}
    }
    \end{table}


\othersQuote{\textbf{Les personnes} ne sont pas de simples offices, ou modes de révélation, \textbf{mais des personnes vivantes du Dieu vivant}.}[p.105][https://archive.org/details/higherchristian02boargoog/page/n112/] & 
    \egw{Il y a \textbf{trois personnes vivantes du trio céleste} ; au nom de ces trois grandes puissances—\textbf{le Père, le Fils, et le Saint-Esprit}—ceux qui reçoivent Christ par la foi vivante sont baptisés, et ces puissances coopéreront avec les sujets obéissants du ciel dans leurs efforts pour vivre la nouvelle vie en Christ.}[Ms21-1906.11; 1906][https://egwwritings.org/read?panels=p9754.18] \\ \hline
    \end{tabular}
    }
    \end{table}


In light of the context of William Boardman’s book, we see a marked contrast between \others{three living persons of \textbf{one living God}}, which is the trinitarian sentiment, and \egwinline{the three living persons of \textbf{the heavenly trio}}, which is in accordance with the truth on the \emcap{personality of God}.


À la lumière du contexte du livre de William Boardman, nous voyons un contraste marqué entre \others{trois personnes vivantes d’\textbf{un seul Dieu vivant}}, qui est le raisonnement trinitaire, et \egwinline{les trois personnes vivantes du \textbf{trio céleste}}, qui est en accord avec la vérité sur la \emcap{personnalité de Dieu}.


The word ‘\textit{trio}’ simply indicates the group of three. The \textit{“heavenly trio}” is represented by the Father, the Son, and the Holy Spirit. But, contrary to popular assumption, they do not make one living God. Three-in-one and one-in-three are concepts that do away with the \emcap{personality of God}. This is why Sister White referred to trinitarian sentiments as sentiments that \egwinline{are not to be trusted}[Ms21-1906.8; 1906][https://egwwritings.org/read?panels=p9754.15].


Le mot ‘\textit{trio}’ indique simplement le groupe de trois. Le \textit{« trio céleste »} est représenté par le Père, le Fils, et le Saint-Esprit. Mais, contrairement à l'hypothèse populaire, ils ne forment pas un seul Dieu vivant. Trois-en-un et un-en-trois sont des concepts qui abolissent la \emcap{personnalité de Dieu}. C'est pourquoi Sœur White a fait référence aux raisonnements trinitaires comme des raisonnements qui \egwinline{ne doivent pas être dignes de confiance}[Ms21-1906.8; 1906][https://egwwritings.org/read?panels=p9754.15].


Sister White never followed any trinitarian fashion—neither in words and expressions, nor in sentiments. There is an almost effortless research endeavor we encourage you to take: in the writings of Ellen White, search for standard trinitarian terms like “\textit{three are one},” “\textit{one are three},” “\textit{one in three},” “\textit{three in one},” or any of the permutations possible. In her impressive oeuvre you will not find a single occurrence of any of these, let alone the word ‘\textit{trinity}’ describing our God\footnote{There is but one occurrence, in the writings of Ellen White, of the word ‘\textit{trinity}’ referring to \egw{the lust of the flesh, the lust of the eyes and the pride of life}[Lt43-1898.25; 1898][https://egwwritings.org/read?panels=p4806.31]}. She never used these phrases that are necessary to explain the trinitarian sentiment. Examining the following quote, we can see why she never said that God is trinity.


Sœur White n'a jamais suivi aucune mode trinitaire—ni dans les mots et expressions, ni dans le raisonnement. Il y a un effort de recherche presque sans effort que nous vous encourageons à entreprendre : dans les écrits d'Ellen White, recherchez des termes trinitaires standards comme « \textit{trois sont un} », « \textit{un sont trois} », « \textit{un en trois} », « \textit{trois en un} », ou toute permutation possible. Dans son œuvre impressionnante, vous ne trouverez pas une seule occurrence de l'un de ces termes, encore moins le mot « \textit{trinité} » décrivant notre Dieu\footnote{Il n'y a qu'une seule occurrence, dans les écrits d'Ellen White, du mot « \textit{trinité} » se référant à \egw{la convoitise de la chair, la convoitise des yeux et l'orgueil de la vie}[Lt43-1898.25; 1898][https://egwwritings.org/read?panels=p4806.31]}. Elle n'a jamais utilisé ces phrases qui sont nécessaires pour expliquer le raisonnement trinitaire. En examinant la citation suivante, nous pouvons voir pourquoi elle n'a jamais dit que Dieu est trinité.


\egw{The subject of \textbf{\underline{speculation} regarding \underline{God’s personality} \underline{we will not venture} to express}, \textbf{\underline{except in the language of the Word which represents His personality}}. There is to be no discussion over this question \textbf{lest God would give unmistakable revelation of \underline{what He is}} that would extinguish the one who dares venture on the holy ground in \textbf{his speculative theories}, as some ventured to do in opening the ark to see what was in it as its power and how God was manifested. The men were slain for their curiosity science.}[17LtMs, Ms 223, 1902, par. 16][https://egwwritings.org/read?panels=p14067.9124037&index=0]


\egw{Le sujet de \textbf{\underline{spéculation} concernant \underline{la personnalité de Dieu} \underline{nous ne nous aventurerons pas} à exprimer}, \textbf{\underline{sauf dans le langage de la Parole qui représente Sa personnalité}}. Il ne doit y avoir aucune discussion sur cette question \textbf{de peur que Dieu ne donne une révélation indubitable de \underline{ce qu'Il est}} qui éteindrait celui qui ose s'aventurer sur le terrain sacré dans \textbf{ses théories spéculatives}, comme certains se sont aventurés à le faire en ouvrant l'arche pour voir ce qu'il y avait dedans comme sa puissance et comment Dieu était manifesté. Les hommes furent tués pour leur science de curiosité.}[17LtMs, Ms 223, 1902, par. 16][https://egwwritings.org/read?panels=p14067.9124037&index=0]


Did you catch that? There is to be no discussion over the question of what God is, \egwinline{lest God would give unmistakable revelation} of \egwinline{what He is}. To say “God is \_\_\_\_\_\_\_”, the blank must be filled with \egwinline{the language of the Word which represents His personality.} The Bible clearly teaches that God is a personal, spiritual being—a truth confirmed by Christ Himself in His revelations to Ellen White. This fits within the biblical language that describes God’s personality. However, according to above statement, can we say “\textit{God is trinity}?” No! That is not \egwinline{the language of the Word which represents His personality.} Therefore, within explored context, we can safely conclude that, the Trinitarian view of God is part of \egwinline{speculative theories} of \egwinline{what He is}.


Avez-vous saisi cela ? Il ne doit y avoir aucune discussion sur la question de ce qu'est Dieu, \egwinline{de peur que Dieu ne donne une révélation indubitable} de \egwinline{ce qu'Il est}. Pour dire « Dieu est \_\_\_\_\_\_\_ », le blanc doit être rempli avec \egwinline{le langage de la Parole qui représente Sa personnalité.} La Bible enseigne clairement que Dieu est un être personnel et spirituel—une vérité confirmée par Christ Lui-même dans Ses révélations à Ellen White. Cela s'inscrit dans le langage biblique qui décrit la personnalité de Dieu. Cependant, selon la déclaration ci-dessus, pouvons-nous dire « \textit{Dieu est trinité} ? » Non ! Ce n'est pas \egwinline{le langage de la Parole qui représente Sa personnalité.} Par conséquent, dans le contexte exploré, nous pouvons conclure en toute sécurité que la vue trinitaire de Dieu fait partie des \egwinline{théories spéculatives} de \egwinline{ce qu'Il est}.


This being said, the phrase \egwinline{Heavenly Trio} is not a definition of what God is. Our God is the Father—not \egwinline{the Heavenly Trio.} The term Heavenly Trio does not serve as a replacement for the Trinitarian idea of \textit{three living persons of one God}. This becomes obvious, when we examine the context. Ellen White was instructed to warn us against Trinitarian sentiments, not to trust them. She was not endorsing them.


Cela étant dit, l'expression \egwinline{Trio Céleste} n'est pas une définition de ce qu'est Dieu. Notre Dieu est le Père—pas \egwinline{le Trio Céleste.} Le terme Trio Céleste ne sert pas de remplacement pour l'idée trinitaire de \textit{trois personnes vivantes d'un seul Dieu}. Cela devient évident lorsque nous examinons le contexte. Ellen White a été instruite de nous avertir contre le raisonnement trinitaire, de ne pas leur faire confiance. Elle ne les approuvait pas.


Although the illustrations Ellen White quoted were not from Dr. Kellogg, it seems that Kellogg's proponents, if not Kellogg himself, were defending him with William Boardman's sentiments. We do not have direct data to confirm this, but we do know that Dr. Kellogg raised \others{the theological side of questions of \textbf{the trinity and all that sort of things}.}[Interview, J. H. Kellogg, G. W. Amadon and A. C. Bourdeau, October 7th 1907 held at Kellogg’s residence][https://archive.org/details/KelloggVs.TheBrethrenHisLastInterviewAsAnAdventistoct71907/page/n37] The last three paragraphs in the heavenly trio manuscript \href{https://egwwritings.org/?ref=en_Ms21-1906&para=9754.1}{(Ms21-1906; 1906)} reveal the connection with Dr. Kellogg, which is another “smoking gun” of Dr. Kellogg's trinitarian stance.


Bien que les illustrations qu'Ellen White a citées ne provenaient pas du Dr Kellogg, il semble que les partisans de Kellogg, sinon Kellogg lui-même, le défendaient avec le raisonnement de William Boardman. Nous n'avons pas de données directes pour confirmer cela, mais nous savons que le Dr Kellogg a soulevé \others{le côté théologique des questions de \textbf{la trinité et toutes ces sortes de choses}.}[Interview, J. H. Kellogg, G. W. Amadon et A. C. Bourdeau, 7 octobre 1907 tenue à la résidence de Kellogg][https://archive.org/details/KelloggVs.TheBrethrenHisLastInterviewAsAnAdventistoct71907/page/n37] Les trois derniers paragraphes du manuscrit du trio céleste \href{https://egwwritings.org/?ref=en_Ms21-1906&para=9754.1}{(Ms21-1906; 1906)} révèlent le lien avec le Dr Kellogg, ce qui est une autre « preuve accablante » de la position trinitaire du Dr Kellogg.


\egw{I write this because any moment my life may be ended. \textbf{Unless there is a breaking away from the influence that Satan has prepared, and a \underline{reviving of the testimonies that God has given, souls will perish in their delusion}. They will accept fallacy after fallacy and will thus keep up a disunion that will always exist until those who have been deceived take \underline{their stand on the right platform}}. All this higher education that is being planned will be extinguished; for it is spurious. The more simple the education of our workers, the less connection they have with the men whom God is not leading, the more will be accomplished. \textbf{Work will be done in the \underline{simplicity} of true godliness, and the old, old times will be back when, under the Holy Spirit’s guidance, thousands were converted in a day. When the truth in its simplicity is lived in every place, then God will work through His angels as He worked on the day of Pentecost, and hearts will be changed so decidedly that there will be a manifestation of the influence of genuine truth, as is represented in the descent of the Holy Spirit}.}[Ms21-1906.18; 1906][https://egwwritings.org/read?panels=p9754.25]


\egw{J'écris ceci parce qu'à tout moment ma vie peut prendre fin. \textbf{À moins qu'il n'y ait une rupture avec l'influence que Satan a préparée, et un \underline{réveil des témoignages que Dieu a donnés, les âmes périront dans leur illusion}. Ils accepteront erreur après erreur et maintiendront ainsi une désunion qui existera toujours jusqu'à ce que ceux qui ont été trompés prennent \underline{position sur la bonne plateforme}}. Toute cette éducation supérieure qui est planifiée sera éteinte ; car elle est fallacieuse. Plus l'éducation de nos ouvriers est simple, moins ils ont de connexion avec les hommes que Dieu ne dirige pas, plus sera accompli. \textbf{Le travail sera fait dans la \underline{simplicité} de la vraie piété, et les temps anciens reviendront quand, sous la direction du Saint-Esprit, des milliers étaient convertis en un jour. Quand la vérité dans sa simplicité est vécue en tout lieu, alors Dieu travaillera à travers Ses anges comme Il a travaillé le jour de la Pentecôte, et les cœurs seront changés si décidément qu'il y aura une manifestation de l'influence de la vérité authentique, comme elle est représentée dans la descente du Saint-Esprit}.}[Ms21-1906.18; 1906][https://egwwritings.org/read?panels=p9754.25]


\egwnogap{The Holy Spirit never has and never will in the future divorce the medical missionary work from the gospel ministry. They cannot be divorced. Bound up with Jesus Christ, the ministry of the Word and the healing of the sick are one.}[Ms21-1906.19; 1906][https://egwwritings.org/read?panels=p9754.26]


\egwnogap{Le Saint-Esprit n'a jamais et ne divorcera jamais à l'avenir le travail missionnaire médical du ministère de l'évangile. Ils ne peuvent pas être divorcés. Liés à Jésus-Christ, le ministère de la Parole et la guérison des malades sont un.}[Ms21-1906.19; 1906][https://egwwritings.org/read?panels=p9754.26]


\egwnogap{The fifty-eighth chapter of Isaiah contains instruction for today. \textbf{‘Cry aloud, spare not, lift up thy voice like a trumpet, and show My people their transgression, and the house of Jacob their sin.’ God does not accept \underline{Dr. Kellogg as His laborer}, unless he will now break with Satan}. The work would not have been hindered, as it has been for the past several years, \textbf{if Dr. Kellogg were a converted man. ‘Come,’ I call, ‘come ye out and be separate from him and his associates whom he has leavened.’ I am now giving the message God has given me, to give to all who claim to believe the truth, \underline{‘Come ye out from among them, and be separate},’ else their sin in justifying wrongs and framing deceits will continue to be the ruin of souls. We cannot afford to be on the wrong side. We cannot afford to cover the truth with scientific problems. We urge that decided changes be made and no more stumbling blocks be placed before the feet of the people of God}. Let every soul put on the gospel shoes. \textbf{Let every soul pray and work, placing their feet upon \underline{the foundation Christ laid} in giving His life for the life of the world}.}[Ms21-1906.20; 1906][https://egwwritings.org/read?panels=p9754.27]


\egwnogap{Le cinquante-huitième chapitre d'Ésaïe contient des instructions pour aujourd'hui. \textbf{« Crie à plein gosier, ne te retiens pas, élève ta voix comme une trompette, et montre à Mon peuple sa transgression, et à la maison de Jacob son péché. » Dieu n'accepte pas \underline{le Dr Kellogg comme Son ouvrier}, à moins qu'il ne rompe maintenant avec Satan}. Le travail n'aurait pas été entravé, comme il l'a été ces dernières années, \textbf{si le Dr Kellogg était un homme converti. « Venez », j'appelle, « sortez et séparez-vous de lui et de ses associés qu'il a fait lever. » Je donne maintenant le message que Dieu m'a donné, à donner à tous ceux qui prétendent croire la vérité, \underline{« Sortez du milieu d'eux, et séparez-vous »}, sinon leur péché en justifiant les torts et en élaborant des tromperies continuera d'être la ruine des âmes. Nous ne pouvons pas nous permettre d'être du mauvais côté. Nous ne pouvons pas nous permettre de couvrir la vérité avec des problèmes scientifiques. Nous exhortons à ce que des changements décidés soient faits et qu'aucune pierre d'achoppement ne soit plus placée devant les pieds du peuple de Dieu}. Que chaque âme mette les chaussures de l'évangile. \textbf{Que chaque âme prie et travaille, plaçant ses pieds sur \underline{le fondement que Christ a posé} en donnant Sa vie pour la vie du monde}.}[Ms21-1906.20; 1906][https://egwwritings.org/read?panels=p9754.27]


The heavenly trio quotation was part of Kellogg's controversy. This is evidence that Kellogg’s controversy included the Trinity doctrine. We are told to break \egwinline{away from the influence of Satan} and to revive the \egw{testimony that God has given} us, or else our souls will perish in delusions. These influences and delusions come from trinitarians such as \textit{William Boardman} and \textit{Dr. John H. Kellogg}. She is pointing us back to place our feet upon the foundation that was built by the Masterworker.\footnote{\href{https://egwwritings.org/?ref=en_SpTB02.54.2&para=417.276}{EGW, SpTB02 54.2; 1904}}


La citation du trio céleste faisait partie de la controverse de Kellogg. C'est la preuve que la controverse de Kellogg incluait la doctrine de la Trinité. On nous dit de rompre \egwinline{avec l'influence de Satan} et de raviver le \egw{témoignage que Dieu nous a donné}, sinon nos âmes périront dans les illusions. Ces influences et illusions viennent de trinitaires tels que \textit{William Boardman} et \textit{Dr John H. Kellogg}. Elle nous oriente vers le retour pour placer nos pieds sur le fondement qui a été construit par le Maître ouvrier.\footnote{\href{https://egwwritings.org/?ref=en_SpTB02.54.2&para=417.276}{EGW, SpTB02 54.2; 1904}}


We hope that this context exposes the false narrative of Ellen White's endorsement of the Trinity doctrine, propagated by our Adventist scholars. Dr. Kellogg was in apostasy for stepping off from the foundation of our faith, and the Trinity doctrine was his justification. With such data in mind, one must ask: If the Trinity was true, and Ellen White endorsed it, and this “true” Trinity was mixed with Dr. Kellogg's error, we should expect her to separate the Trinity from error. But this is not what she did. Instead, she consistently pointed us back to the foundation of our faith, where we had a clear teaching on the presence and the \emcap{personality of God}. But for the case of Trinity, she faithfully bore the message from Heaven: “\textit{\textbf{I am instructed to say}, the sentiments of those who are searching for \textbf{trinitarian ideas are not to be trusted}}.”


Nous espérons que ce contexte expose le faux récit de l'approbation d'Ellen White de la doctrine de la Trinité, propagé par nos érudits adventistes. Le Dr Kellogg était en apostasie pour s'être écarté du fondement de notre foi, et la doctrine de la Trinité était sa justification. Avec de telles données à l'esprit, on doit se demander : Si la Trinité était vraie, et qu'Ellen White l'approuvait, et que cette « vraie » Trinité était mélangée avec l'erreur du Dr Kellogg, nous devrions nous attendre à ce qu'elle sépare la Trinité de l'erreur. Mais ce n'est pas ce qu'elle a fait. Au lieu de cela, elle nous a constamment orientés vers le fondement de notre foi, où nous avions un enseignement clair sur la présence et la \emcap{personnalité de Dieu}. Mais dans le cas de la Trinité, elle a fidèlement porté le message du Ciel : « \textit{\textbf{Je suis instruite de dire}, le raisonnement de ceux qui recherchent des \textbf{idées trinitaires ne doit pas être digne de confiance}}. »


\begin{titledpoem}
\stanza{
    In a realm divine, where truths unfold, \\
    A message of clarity, brave and bold. \\
    Ellen spoke, her voice clear and bright, \\
    Revealing the depths of heavenly light.
}

\stanza{
    Misunderstood by many who read, \\
    Her words hold a truth that all must heed. \\
    Not a triune God, but a trio divine, \\
    Three living persons, distinct in line.
}

\stanza{
    The Father, not formless, but full and bright, \\
    Invisible to mortal, yet real in might. \\
    He is the fullness, a presence complete, \\
    Invisible to sight, yet real and concrete.
}

\stanza{
    The Son, God's fullness, manifest and near, \\
    In Him, the divine becomes crystal clear. \\
    The personality of God, seen in His face, \\
    In Christ, we witness God's grace.
}

\stanza{
    The Holy Spirit, in fullness resides, \\
    Within the Godhead, where mystery abides. \\
    Father and Son, with bodies they stand, \\
    Holy Spirit, as spirit, spreads through the land.
}

\stanza{
    Distinct and clear, their roles unfold, \\
    Father and Son, in form behold. \\
    Yet everywhere present, the Spirit we find, \\
    Their representative, in heart and mind.
}

\stanza{
    Spiritual and bodily, presence defined, \\
    Understanding this truth, enlightenment kind. \\
    Ellen's message, profound and bright, \\
    Guides us through the heavenly light.
}

\stanza{
    Ellen's words, in context found, \\
    Show a truth profound and sound. \\
    Not the trinity she did embrace, \\
    But a trio divine, each in their place.
}

\stanza{
    In "heavenly Trio," a contrast is drawn, \\
    Between trinity's doctrine and faith's true dawn. \\
    The pillar stands firm, the personality of God, \\
    Distinct from trinity, where truth is awed.
}
\end{titledpoem}