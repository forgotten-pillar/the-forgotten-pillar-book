\qrchapter{https://forgottenpillar.com/rsc/en-fp-chapter10}{Is God a person? - by John N. Loughborough}


\qrchapter{https://forgottenpillar.com/rsc/en-fp-chapter10}{Dieu est-il une personne ? - par John N. Loughborough}


One of the earliest articles on the \emcap{personality of God} is Loughborough’s article “\textit{Is God a person?}” where he discusses the \emcap{personality of God} and His presence. It is important to remember the definition of ‘personality’ according to the Merriam-Webster dictionary: “\textit{the quality or state of being a person}”\footnote{\href{https://www.merriam-webster.com/dictionary/personality}{Merriam-Webster Dictionary - ‘\textit{personality}’}}. We will look carefully at how Loughborough sees the quality or state of God being a person.


L'un des premiers articles sur la \emcap{personnalité de Dieu} est l'article de Loughborough « \textit{Dieu est-il une personne ?} » où il discute de la \emcap{personnalité de Dieu} et de Sa présence. Il est important de se rappeler la définition de « personnalité » selon le dictionnaire Merriam-Webster : « \textit{la qualité ou l'état d'être une personne} »\footnote{\href{https://www.merriam-webster.com/dictionary/personality}{Dictionnaire Merriam-Webster - « \textit{personnalité} »}}. Nous examinerons attentivement comment Loughborough voit la qualité ou l'état de Dieu d'être une personne.


\begin{figure}[hp]
    \centering
    \includegraphics[width=1\linewidth]{images/john-n-loughborough.jpg}
    \caption*{John Norton Loughborough (1832-1924)}
    \label{fig:john-n-loughborough}
\end{figure}


\begin{figure}[hp]
    \centering
    \includegraphics[width=1\linewidth]{images/john-n-loughborough.jpg}
    \caption*{John Norton Loughborough (1832-1924)}
    \label{fig:john-n-loughborough}
\end{figure}


\others{Whatever may be the truth in this matter, it certainly cannot be wrong for us to examine what the Word says respecting it. \textbf{Many there are that would refrain from the investigation of unpopular truths because the cry of heresy is raised against them}. We shall not consider ourselves subjects of the appellation, \textbf{neither are we prying into the secrets of the Almighty, as we pursue the investigation of this matter}. The Bible certainly contains testimony upon this point, and we again repeat, ‘\textbf{Things which are revealed belong to us}.’ We inquire then, What saith the Scripture?}


\others{Quelle que soit la vérité dans cette affaire, il ne peut certainement pas être mal pour nous d'examiner ce que la Parole dit à ce sujet. \textbf{Il y en a beaucoup qui s'abstiendraient de l'investigation de vérités impopulaires parce que le cri d'hérésie est élevé contre eux}. Nous ne nous considérerons pas sujets de l'appellation, \textbf{et nous ne fouillons pas non plus dans les secrets du Tout-Puissant, alors que nous poursuivons l'investigation de cette affaire}. La Bible contient certainement un témoignage sur ce point, et nous répétons encore : « \textbf{Les choses qui sont révélées nous appartiennent} ». Nous demandons alors : Que dit l'Écriture ?}


\othersnogap{\textbf{The very testimony we have been examining in regard to man’s being formed of the dust in \underline{the image of God}, proves conclusively that \underline{God has a form}, although the sentiment is contrary to what we have been taught, while children, from the catechism}:}


\othersnogap{\textbf{Le témoignage même que nous avons examiné concernant le fait que l'homme soit formé de la poussière à \underline{l'image de Dieu}, prouve de manière concluante que \underline{Dieu a une forme}, bien que le raisonnement soit contraire à ce qu'on nous a enseigné, alors que nous étions enfants, à partir du catéchisme} :}


\othersnogap{Question. ‘What is God?’}


\othersnogap{Question. « Qu'est-ce que Dieu ? »}


\othersnogap{Answer. ‘An infinite and eternal spirit; one that always was and always will be.’}


\othersnogap{Réponse. « Un esprit infini et éternel ; un qui a toujours été et sera toujours. »}


\othersnogap{Q. ‘Where is God?’}


\othersnogap{Q. « Où est Dieu ? »}


\othersnogap{A. ‘Everywhere.’}


\othersnogap{R. « Partout. »}


\othersnogap{\textbf{But we inquire, \underline{Is not God in one place more than another}?} Oh no, say you: \textbf{the Bible says \underline{he is a spirit}, and if so he must be \underline{everywhere alike}}. Well, if when man dies his spirit goes to God, it must go everywhere. \textbf{But the Bible certainly represents God as located in heaven. ‘For he hath looked down from the height of his sanctuary: from heaven did the Lord behold the earth.’ Psalm 102:19}. \textbf{Then certainly heaven cannot be everywhere, for God is represented as looking down from it. ‘\underline{Elijah went up} by a whirlwind \underline{into heaven}.’ 2 Kings 2:11}. \textbf{But, says one, does not the Bible represent God \underline{as everywhere present}?} Psalm 139:8, 9, 10. ‘If I ascend up into heaven, \textbf{thou art there}: if I make my bed in hell, \textbf{behold, thou art there}; if I take the wings of the morning, and dwell in the uttermost parts of the sea,\textbf{ even there shall thy hand lead me}, and thy right hand shall hold me.’}


\othersnogap{\textbf{Mais nous demandons, \underline{Dieu n'est-il pas dans un endroit plus qu'un autre} ?} Oh non, dites-vous : \textbf{la Bible dit qu’\underline{il est un esprit}, et si c'est le cas, il doit être \underline{partout de la même manière}}. Eh bien, si quand l'homme meurt, son esprit va à Dieu, il doit aller partout. \textbf{Mais la Bible représente certainement Dieu comme étant situé au ciel. « Car il a regardé du haut de son sanctuaire ; des cieux, l'Éternel a regardé sur la terre. » Psaume 102:20}. \textbf{Alors certainement le ciel ne peut pas être partout, car Dieu est représenté comme regardant depuis celui-ci. « \underline{Élie monta} au ciel \underline{dans un tourbillon}. » 2 Rois 2:11}. \textbf{Mais, dit quelqu'un, la Bible ne représente-t-elle pas Dieu \underline{comme étant partout présent} ?} Psaume 139:8, 9, 10. « Si je monte aux cieux, \textbf{tu y es} ; si je me couche au Sépulcre, \textbf{voici, tu y es}. Si je prends les ailes de l'aube du jour, et que j'aille habiter au bout de la mer, \textbf{là aussi ta main me conduira}, et ta droite me saisira. »}


\othersnogap{We reply, \textbf{the subject is introduced in verse 7, as follows}: ‘\textbf{\underline{Whither shall I go from thy Spirit}?} \textbf{or whither shall I flee from \underline{thy presence}?}’ \textbf{The Spirit is \underline{God’s representative}}. \textbf{His power is manifested wherever he listeth, through the agency of his Spirit}. Christ, when giving the commission to the disciples, says, ‘Go ye into all the world, and preach the gospel to every creature, and lo! \textbf{I am with you alway, even unto the end of the world}.’ Now, no one would contend that Christ had been on the earth personally ever since the disciples commenced to fulfill this commission. \textbf{But his Spirit has been on the earth; the Comforter that he promised to send.} \textbf{So in the same manner God manifests himself \underline{by his Spirit} which is also the power through which he works}. ‘But if \textbf{the Spirit of him} that raised up Jesus from the dead dwell in you, \textbf{he that raised up Christ} from the dead shall also quicken your mortal bodies \textbf{\underline{by his Spirit} that dwelleth in you}.’ Romans 8:11. \textbf{\underline{Here is a plain distinction made between the Spirit, and God that raises the dead by that Spirit}}. \textbf{If the living God is a Spirit in the strictest sense of the term, and at the same time is in possession of a Spirit, then we have at once the novel idea of the Spirit of a Spirit, something it will take at least a Spiritualist to explain}.}[The Adventist Review and Sabbath Herald, September 18, 1855][https://documents.adventistarchives.org/Periodicals/RH/RH18550918-V07-06.pdf]


\othersnogap{Nous répondons, \textbf{le sujet est introduit au verset 7, comme suit} : ‘\textbf{\underline{Où irais-je loin de ton Esprit}} ? \textbf{ou où fuirais-je loin de \underline{ta présence}} ?’ \textbf{L'Esprit est \underline{le représentant de Dieu}}. \textbf{Sa puissance se manifeste partout où il veut, par l'intermédiaire de son Esprit}. Christ, en donnant la commission aux disciples, dit : ‘Allez par tout le monde, et prêchez l'évangile à toute créature, et voici ! \textbf{je suis avec vous tous les jours, jusqu'à la fin du monde}.’ Maintenant, personne ne soutiendrait que Christ a été sur la terre personnellement depuis que les disciples ont commencé à accomplir cette commission. \textbf{Mais son Esprit a été sur la terre ; le Consolateur qu'il a promis d'envoyer.} \textbf{Ainsi, de la même manière, Dieu se manifeste \underline{par son Esprit} qui est aussi la puissance par laquelle il œuvre}. ‘Mais si \textbf{l'Esprit de celui} qui a ressuscité Jésus d'entre les morts habite en vous, \textbf{celui qui a ressuscité Christ} d'entre les morts rendra aussi la vie à vos corps mortels \textbf{\underline{par son Esprit} qui habite en vous}.’ Romains 8:11. \textbf{\underline{Ici, une distinction claire est faite entre l'Esprit et Dieu qui ressuscite les morts par cet Esprit}}. \textbf{Si le Dieu vivant est un Esprit au sens strict du terme, et en même temps est en possession d'un Esprit, alors nous avons immédiatement l'idée nouvelle de l'Esprit d'un Esprit, quelque chose qu'il faudra au moins un spiritualiste pour expliquer}.}[The Adventist Review and Sabbath Herald, 18 septembre 1855][https://documents.adventistarchives.org/Periodicals/RH/RH18550918-V07-06.pdf]


Allow us to make a short comment. We hope you recognize the specific topic being discussed here. The subject is the first point of the \emcap{Fundamental Principles} and the assertion is that God does have a form, for man is made in the image of God. Such understanding of God’s personality precludes the idea that God is everywhere present. Brother Loughborough gave the biblical reasons for God's omnipresence, together with the sentiment that “\textit{God is in one place more than another}”. God is everywhere present by His representative, the Holy Spirit, just as it is written in the first point of the \emcap{Fundamental Principles}. Further in this discussion, we will read that God is a spiritual being and possesses a tangible, material body, in contrast to the idea that He is purely a spirit.


Permettez-nous de faire un court commentaire. Nous espérons que vous reconnaissez le sujet spécifique dont il est question ici. Le sujet est le premier point des \emcap{Principes Fondamentaux} et l'affirmation est que Dieu a effectivement une forme, car l'homme est fait à l'image de Dieu. Une telle compréhension de la personnalité de Dieu exclut l'idée que Dieu est partout présent. Frère Loughborough a donné les raisons bibliques de l'omniprésence de Dieu, ainsi que le raisonnement que « \textit{Dieu est dans un endroit plus qu'un autre} ». Dieu est partout présent par Son représentant, le Saint-Esprit, exactement comme il est écrit dans le premier point des \emcap{Principes Fondamentaux}. Plus loin dans cette discussion, nous lirons que Dieu est un être spirituel et possède un corps tangible et matériel, contrairement à l'idée qu'Il est purement un esprit.


\others{There is at least one impassable difficulty in the way of \textbf{those who believe \underline{God is immaterial}, and heaven is not a literal, \underline{located place}: they are obliged to admit that \underline{Jesus is there bodily, a literal person}}; the same Jesus that was crucified, dead, and buried, was raised from the dead, \textbf{ascended up to heaven}, and is now \textbf{at the right hand of God}. \textbf{Jesus was possessed of flesh and bones after his resurrection}. Luke 24:39. ‘\textbf{Behold my hands and my feet, that it is I, myself; handle me, and see; \underline{for a spirit hath not flesh and bones as ye see me have}}.’ \textbf{If Jesus is there in heaven with a literal body of flesh and bones, may not heaven after all be a literal place, a habitation for a literal God, a literal Saviour, literal angels, and resurrected immortal saints?} \textbf{\underline{Oh no, says one, ‘God is a Spirit.’}} So Christ said to the woman of Samaria at the well. \textbf{It does not necessarily follow because God is a Spirit, \underline{that he has no body}}. In John 3:6, Christ says to Nicodemus, ‘\textbf{That which is born of the Spirit is spirit}.’ \textbf{If that which is born of the Spirit is spirit, then on the same principle, that which has a spiritual nature is spirit. God is \underline{a spirit being}, his nature is spirit, he is not of a mortal nature; }\textbf{\underline{but this does not exclude the idea of his having a body}}. David says, [Psalm 114:4,] ‘Who maketh \textbf{his angels spirits};’ yet \textbf{\underline{angels have bodies}}. Angels appeared to both Abraham and Lot, and ate with them. \textbf{We see the idea that angels are spirits, does not prove that they are not literal beings}.}


\others{Il y a au moins une difficulté insurmontable sur le chemin de \textbf{ceux qui croient que \underline{Dieu est immatériel}, et que le ciel n'est pas un \underline{endroit littéral et localisé}} : ils sont obligés d'admettre que \underline{Jésus y est corporellement, une personne littérale}} ; le même Jésus qui a été crucifié, mort et enterré, a été ressuscité d'entre les morts, \textbf{est monté au ciel}, et est maintenant \textbf{à la droite de Dieu}. \textbf{Jésus possédait de la chair et des os après sa résurrection}. Luc 24:39. ‘\textbf{Voyez mes mains et mes pieds, c'est bien moi ; touchez-moi et voyez ; \underline{car un esprit n'a ni chair ni os, comme vous voyez que j'ai}}.’ \textbf{Si Jésus est là au ciel avec un corps littéral de chair et d'os, le ciel ne pourrait-il pas après tout être un lieu littéral, une habitation pour un Dieu littéral, un Sauveur littéral, des anges littéraux et des saints immortels ressuscités ?} \textbf{\underline{Oh non, dit quelqu'un, ‘Dieu est un Esprit.’}} C'est ce que Christ a dit à la femme de Samarie au puits. \textbf{Il ne s'ensuit pas nécessairement que parce que Dieu est un Esprit, \underline{il n'a pas de corps}}. Dans Jean 3:6, Christ dit à Nicodème : ‘\textbf{Ce qui est né de l'Esprit est esprit}.’ \textbf{Si ce qui est né de l'Esprit est esprit, alors sur le même principe, ce qui a une nature spirituelle est esprit. Dieu est \underline{un être spirituel}, sa nature est esprit, il n'est pas de nature mortelle ;} \textbf{\underline{mais cela n'exclut pas l'idée qu'il ait un corps}}. David dit, [Psaume 104:4,] ‘Qui fait \textbf{de ses anges des esprits}’ ; pourtant \textbf{\underline{les anges ont des corps}}. Des anges sont apparus à Abraham et à Lot, et ont mangé avec eux. \textbf{Nous voyons que l'idée que les anges sont des esprits ne prouve pas qu'ils ne sont pas des êtres littéraux}.}


\othersnogap{It is inferred because the Bible says that God is a Spirit, that he is not a person. An inference should not be made the basis for an argument. Great Scripture truths are plainly stated, and it will not do for us to found a doctrine on inferences, \textbf{contrary to positive statements in the word of God}. If the Scripture states in positive \textbf{terms that God is a person, it will not answer for us to draw an inference from the text which says ‘God is a Spirit,’ \underline{that he has no body}}.}


\othersnogap{On en déduit que parce que la Bible dit que Dieu est un Esprit, il n'est pas une personne. Une déduction ne devrait pas être la base d'un argument. Les grandes vérités scripturaires sont clairement énoncées, et il ne convient pas que nous fondions une doctrine sur des déductions, \textbf{contraires aux déclarations positives dans la parole de Dieu}. Si l'Écriture déclare en \textbf{termes positifs que Dieu est une personne, il ne conviendra pas que nous tirions une déduction du texte qui dit ‘Dieu est un Esprit’, \underline{qu'il n'a pas de corps}}.}


\othersnogap{We will now present a few texts \textbf{which prove that God is a person}. Exodus 33:18, 23. ‘And he (Moses) said, I beseech thee shew me thy glory.’ Verse 20. ‘And he said, \textbf{Thou canst not see \underline{my face}, for there shall no man see me and live}.’ Verses 21-23. ‘And the Lord said, Behold there is a place by me, and thou shalt stand upon a rock: and it shall come to pass while my glory passeth by, that I will put thee in a cleft of the rock; and \textbf{will cover thee with \underline{my hand} while I pass by}; and I will take away \textbf{mine hand}, and thou shalt \textbf{see my \underline{back parts}}; but \textbf{\underline{my face} shall not be seen.’} \textbf{If God is \underline{an immaterial Spirit}, then Moses could not see him; for we are told a spirit cannot be seen by natural eyes}. \textbf{There would then be no propriety for God to say he would put his hand over Moses’ face while he passed by, (seemingly to prevent him from seeing his face,) for he could not see him}. Neither do we conceive how an immaterial hand could obstruct the rays of light from passing to Moses’ eyes. \textbf{But if the position be true \underline{that God is immaterial}, and cannot be seen by the natural eye, the text above is all superfluous}. \textbf{What sense is there in saying God put his hand over Moses’ face, to prevent him from seeing that which could not be seen}.}


\othersnogap{Nous allons maintenant présenter quelques textes \textbf{qui prouvent que Dieu est une personne}. Exode 33:18, 23. ‘Et il (Moïse) dit : Je te prie, fais-moi voir ta gloire.’ Verset 20. ‘Et il dit : \textbf{Tu ne pourras pas voir \underline{ma face}, car nul homme ne peut me voir et vivre}.’ Versets 21-23. ‘Et l'Éternel dit : Voici, il y a un lieu près de moi, et tu te tiendras sur le rocher : et il arrivera, pendant que ma gloire passera, que je te mettrai dans une fente du rocher ; et \textbf{je te couvrirai de \underline{ma main} pendant que je passerai} ; et je retirerai \textbf{ma main}, et tu \textbf{verras mon \underline{dos}} ; mais \textbf{\underline{ma face} ne sera point vue}.’ \textbf{Si Dieu est \underline{un Esprit immatériel}, alors Moïse ne pouvait pas le voir ; car on nous dit qu'un esprit ne peut pas être vu par des yeux naturels}. \textbf{Il n'y aurait alors aucune convenance pour Dieu de dire qu'il mettrait sa main sur le visage de Moïse pendant qu'il passerait, (apparemment pour l'empêcher de voir sa face,) car il ne pouvait pas le voir}. Nous ne concevons pas non plus comment une main immatérielle pourrait empêcher les rayons de lumière de passer aux yeux de Moïse. \textbf{Mais si la position est vraie \underline{que Dieu est immatériel}, et ne peut pas être vu par l'œil naturel, le texte ci-dessus est tout superflu}. \textbf{Quel sens y a-t-il à dire que Dieu a mis sa main sur le visage de Moïse, pour l'empêcher de voir ce qui ne pouvait pas être vu}.}


\othersnogap{Says one, I see we cannot harmonize the matter any other way, than that there was a literal body seen by Moses; but that was not God’s own body, \textbf{it was a body he took that he might show himself to Moses}. \textbf{Moses could form no just conceptions of God unless he assumed a form.} \textbf{So God took a body}. This throws a worse coloring on the matter than the first position; \textbf{for it charges God with deception; telling Moses he should see him, when in fact Moses according to this testimony did not see God, but another body}. A person must be given to doubt almost beyond recovery, that would attempt thus to mystify, and do away the force of this testimony.}[Ibid.][https://documents.adventistarchives.org/Periodicals/RH/RH18550918-V07-06.pdf]


\othersnogap{Dit quelqu'un, je vois que nous ne pouvons harmoniser la question d'aucune autre manière, que qu'il y avait un corps littéral vu par Moïse ; mais ce n'était pas le propre corps de Dieu, \textbf{c'était un corps qu'il a pris pour pouvoir se montrer à Moïse}. \textbf{Moïse ne pouvait former aucune conception juste de Dieu à moins qu'il n'assume une forme.} \textbf{Alors Dieu a pris un corps}. Cela jette une coloration pire sur la question que la première position ; \textbf{car cela accuse Dieu de tromperie ; disant à Moïse qu'il le verrait, alors qu'en fait Moïse selon ce témoignage n'a pas vu Dieu, mais un autre corps}. Une personne doit être encline au doute presque au-delà du rétablissement, qui tenterait ainsi de mystifier et d'annuler la force de ce témoignage.}[Ibid.][https://documents.adventistarchives.org/Periodicals/RH/RH18550918-V07-06.pdf]


Do you recognize that Brother Loughborough is tackling the sentiment that Dr. Kellogg would present in the Living Temple 48 years later? Dr. Kellogg said that it is true that God presented Himself in a\others{\textbf{\underline{particular form or place}}}[Dr. John H. Kellogg, The Living Temple, p.31.][https://archive.org/details/J.H.Kellogg.TheLivingTemple1903/page/n31/] because \others{there must be something more \textbf{tangible}, more \textbf{\underline{restricted}}, upon which to center the mind in worship}[bid, p.30][https://archive.org/details/J.H.Kellogg.TheLivingTemple1903/page/n30/], but that He is, in reality,\others{\textbf{far beyond our comprehension \underline{as are the bounds of space and time}}}[Ibid, p.33][https://archive.org/details/J.H.Kellogg.TheLivingTemple1903/page/n33/]. Brother Loughborough reasonably objected to the idea that God is only manifesting Himself to man as a definite Being, but in reality, is not what He presents Himself to be. Such a claim\others{charges God with deception}. Brother Loughborough continues with the affirmative, Biblical testimony that God is a material being.


Reconnaissez-vous que Frère Loughborough s'attaque au raisonnement que le Dr Kellogg présenterait dans Le Temple Vivant 48 ans plus tard ? Le Dr Kellogg a dit qu'il est vrai que Dieu s'est présenté dans une \others{\textbf{\underline{forme ou un lieu particulier}}}[Dr. John H. Kellogg, Le Temple Vivant, p.31.][https://archive.org/details/J.H.Kellogg.TheLivingTemple1903/page/n31/] parce qu’\others{il doit y avoir quelque chose de plus \textbf{tangible}, de plus \textbf{\underline{restreint}}, sur lequel centrer l'esprit dans l'adoration}[Ibid, p.30][https://archive.org/details/J.H.Kellogg.TheLivingTemple1903/page/n30/], mais qu'Il est, en réalité, \others{\textbf{bien au-delà de notre compréhension \underline{comme le sont les limites de l'espace et du temps}}}[Ibid, p.33][https://archive.org/details/J.H.Kellogg.TheLivingTemple1903/page/n33/]. Frère Loughborough s'est raisonnablement opposé à l'idée que Dieu ne fait que se manifester à l'homme comme un Être défini, mais en réalité, n'est pas ce qu'Il se présente être. Une telle affirmation \others{accuse Dieu de tromperie}. Frère Loughborough continue avec le témoignage biblique affirmatif que Dieu est un être matériel.


\others{Exodus 24:9. ‘Then went up Moses and Aaron, Nadab and Abihu, and seventy of the elders of Israel: \textbf{and they saw the God of Israel}: and there was under \textbf{his feet} as it were a paved work of a sapphire stone, and as it were the body of heaven in its clearness.’ They were permitted to \textbf{see his feet}, but no \textbf{man can see his face and live}. \textbf{No \underline{mortal eye} can bear the dazzling brightness of that glory of the face of God}. It far exceeds the light of the sun. For the prophet says, ‘The light of the moon shall be as the light of the sun, and the light of the sun shall be \textbf{seven fold}, as the light of seven days, in the day that the Lord bindeth up the breach of his people, and healeth the stroke of their wound.’ Isaiah 30:26. Notwithstanding this seven-fold light that is then to shine, the prophet speaking of the scene says, ‘Then the moon shall be confounded, and the sun ashamed, when the Lord of hosts shall reign in mount Zion, and in Jerusalem, and before his ancients gloriously.’ Isaiah 24:23. The testimony of John is, [Revelation 21:23,] ‘And the city had no need of the sun, neither of the moon, to shine in it: for \textbf{the glory of God did lighten it,} and the Lamb is the light thereof.’}


\others{Exode 24:9. ‘Alors montèrent Moïse et Aaron, Nadab et Abihu, et soixante-dix des anciens d'Israël : \textbf{et ils virent le Dieu d'Israël} : et il y avait sous \textbf{ses pieds} comme un ouvrage de pavé de pierre de saphir, et comme le corps du ciel dans sa clarté.’ Il leur fut permis de \textbf{voir ses pieds}, mais \textbf{nul homme ne peut voir sa face et vivre}. \textbf{Aucun \underline{œil mortel} ne peut supporter l'éclat éblouissant de cette gloire de la face de Dieu}. Elle dépasse de loin la lumière du soleil. Car le prophète dit : ‘La lumière de la lune sera comme la lumière du soleil, et la lumière du soleil sera \textbf{sept fois plus grande}, comme la lumière de sept jours, au jour où l'Éternel bandera la plaie de son peuple, et guérira la blessure de sa plaie.’ Ésaïe 30:26. Malgré cette lumière septuple qui doit alors briller, le prophète parlant de la scène dit : ‘Alors la lune sera confuse, et le soleil aura honte, quand l'Éternel des armées régnera sur la montagne de Sion, et à Jérusalem, et devant ses anciens glorieusement.’ Ésaïe 24:23. Le témoignage de Jean est, [Apocalypse 21:23,] ‘Et la ville n'avait pas besoin du soleil, ni de la lune, pour l'éclairer : car \textbf{la gloire de Dieu l'éclairait}, et l'Agneau est sa lumière.’}


\othersnogap{\textbf{Infidels claim that there is a contradiction in the testimony of Moses, because he said, he talked with God face to face}. \textbf{We reply, there was a cloud between them}, but God told Moses, ‘\textbf{No man shall see me and live}.’ The Testimony of the New Testament is in harmony with that of the Old upon this subject. ‘Follow peace with all men, and holiness without which \textbf{no man shall see the Lord}.’ Hebrews 12:14. \textbf{Who with \underline{mortal eyes} could behold a light that far outshines seven fold the brightness of the sun?} Surely none but the holy can behold him, \textbf{none but immortal eyes} could bear that radiant glory. Although the Word says we cannot see God now and live, the promise is, that the \textbf{pure in heart shall see him}. Matthew 5:3. ‘Blessed are the pure in heart, \textbf{for they shall see God}.’ Revelation 22:4. ‘And \textbf{they shall see his face}, and his name shall be in their foreheads.’}


\othersnogap{\textbf{Les infidèles prétendent qu'il y a une contradiction dans le témoignage de Moïse, parce qu'il a dit qu'il a parlé avec Dieu face à face}. \textbf{Nous répondons, il y avait un nuage entre eux}, mais Dieu a dit à Moïse : ‘\textbf{Nul homme ne me verra et vivra}.’ Le témoignage du Nouveau Testament est en harmonie avec celui de l'Ancien sur ce sujet. ‘Recherchez la paix avec tous, et la sanctification sans laquelle \textbf{personne ne verra le Seigneur}.’ Hébreux 12:14. \textbf{Qui avec des \underline{yeux mortels} pourrait contempler une lumière qui surpasse de loin sept fois l'éclat du soleil ?} Sûrement personne d'autre que les saints ne peuvent le contempler, \textbf{seuls des yeux immortels} pourraient supporter cette gloire rayonnante. Bien que la Parole dise que nous ne pouvons pas voir Dieu maintenant et vivre, la promesse est que ceux qui ont \textbf{le cœur pur le verront}. Matthieu 5:8. ‘Heureux ceux qui ont le cœur pur, \textbf{car ils verront Dieu}.’ Apocalypse 22:4. ‘Et \textbf{ils verront sa face}, et son nom sera sur leurs fronts.’}


\othersnogap{Paul, [Colossians 1:15,] speaking of Christ, says, ‘Who is the image of \textbf{the invisible God}, the first born of every creature.’ Here Christ is said to be ‘\textbf{the image of the invisible God}.’ We have already shown, that\textbf{ Christ has a body composed of substance, flesh and bones; and he is said to be}, ‘\textbf{the image of the invisible God}.’ Well, says one, we admit his divine nature is in the image of God. If by his divine nature you mean the part that existed in glory with the Father before the world was, we reply, that which was in the beginning with God, (the Word,) \textbf{was made flesh, not came into flesh}, or as some state, \textbf{clothed upon with a human nature, but made flesh}. But says another, \textbf{God is said to be invisible}. \textbf{Because he is invisible now, it does not prove that he never will be seen}. The Word says, ‘The pure in heart \textbf{shall see him}’. Willing faith says, Amen.}


\othersnogap{Paul, [Colossiens 1:15,] parlant de Christ, dit : ‘Qui est l'image du \textbf{Dieu invisible}, le premier-né de toute créature.’ Ici, Christ est dit être ‘\textbf{l'image du Dieu invisible}.’ Nous avons déjà montré que \textbf{Christ a un corps composé de substance, de chair et d'os ; et il est dit être}, ‘\textbf{l'image du Dieu invisible}.’ Eh bien, dit quelqu'un, nous admettons que sa nature divine est à l'image de Dieu. Si par sa nature divine vous entendez la partie qui existait en gloire avec le Père avant que le monde fût, nous répondons, ce qui était au commencement avec Dieu, (la Parole,) \textbf{a été fait chair, n'est pas venu dans la chair}, ou comme certains le déclarent, \textbf{revêtu d'une nature humaine, mais fait chair}. Mais dit un autre, \textbf{Dieu est dit être invisible}. \textbf{Parce qu'il est invisible maintenant, cela ne prouve pas qu'il ne sera jamais vu}. La Parole dit : ‘Ceux qui ont le cœur pur \textbf{le verront}’. La foi consentante dit : Amen.}


\othersnogap{Paul’s testimony in Philippians 2:5, 6, shows plainly what may be understood by the statement, that Christ is the image of God. ‘Let this mind be in you which was in Christ Jesus: who \textbf{being in the form of God}, thought it not robbery to \textbf{be equal with God}.’ \textbf{How can Christ be said to be in the form of God, if God has no form?} Romans 8:3. ‘God sending his own Son in the likeness of sinful flesh.’ \textbf{Christ is in the form of God, and in the form of men. This at once reveals to us the form of God}.}


\othersnogap{Le témoignage de Paul dans Philippiens 2:5, 6, montre clairement ce qui peut être compris par la déclaration que Christ est l'image de Dieu. ‘Ayez en vous les sentiments qui étaient en Jésus-Christ : lequel \textbf{existant en forme de Dieu}, n'a point regardé comme une proie à arracher \textbf{d'être égal avec Dieu}.’ \textbf{Comment peut-on dire que Christ est dans la forme de Dieu, si Dieu n'a pas de forme ?} Romains 8:3. ‘Dieu ayant envoyé son propre Fils en ressemblance de chair de péché.’ \textbf{Christ est dans la forme de Dieu, et dans la forme des hommes. Cela nous révèle immédiatement la forme de Dieu}.}


\othersnogap{\textbf{\underline{Daniel speaking of God, calls him the Ancient of days}}. Daniel 7:9. ‘And the Ancient of days did sit, \textbf{whose garment was white as snow}, and \textbf{the hair of his head} like the pure wool.’ \textbf{This personage is said to have a head, and hair; this certainly could not be said of him} \textbf{\underline{if he was immaterial and had no form}}. \textbf{But Paul’s testimony in \underline{Hebrews 1:3}, ought to settle every candid mind in \underline{regard to the personality of God}}. Speaking of Christ, he says, ‘Who being the brightness of his glory, \textbf{and the express image of his (the \underline{Father’s person})}.’ \textbf{Here then it is plainly stated \underline{God has a person}. Christ is the express image of it.} Then we can understand Christ where he says, ‘\textbf{He that hath seen me, hath seen the Father}.’ John 14:19. \textbf{He could not have meant, that he was his own father; for when he prayed he addressed his Father as another person who had sent him into the world}. He styled himself \textbf{the Son of God}. \textbf{Then he could not be the Father of which he was the son}. When he says, ‘He that hath seen me hath seen the Father,’ he must mean, that as \textbf{he was the express image of the Father’s person, those who saw him saw the likeness of the Father in him}.}[The Adventist Review and Sabbath Herald, September 18, 1855][https://documents.adventistarchives.org/Periodicals/RH/RH18550918-V07-06.pdf]


\othersnogap{\textbf{\underline{Daniel parlant de Dieu, l'appelle l'Ancien des jours}}. Daniel 7:9. ‘Et l'Ancien des jours s'assit, \textbf{dont le vêtement était blanc comme la neige}, et \textbf{les cheveux de sa tête} comme de la laine pure.’ \textbf{Il est dit que ce personnage a une tête et des cheveux ; cela ne pourrait certainement pas être dit de lui} \textbf{\underline{s'il était immatériel et n'avait pas de forme}}. \textbf{Mais le témoignage de Paul dans \underline{Hébreux 1:3}, devrait convaincre tout esprit sincère en ce qui \underline{concerne la personnalité de Dieu}}. Parlant de Christ, il dit : ‘Qui étant la splendeur de sa gloire, \textbf{et l'empreinte de sa (la \underline{personne du Père})}.’ \textbf{Ici donc il est clairement déclaré que \underline{Dieu a une personne}. Christ en est l'empreinte.} Alors nous pouvons comprendre Christ quand il dit : ‘\textbf{Celui qui m'a vu a vu le Père}.’ Jean 14:19. \textbf{Il ne pouvait pas vouloir dire qu'il était son propre père ; car quand il priait, il s'adressait à son Père comme à une autre personne qui l'avait envoyé dans le monde}. Il se qualifiait lui-même de \textbf{Fils de Dieu}. \textbf{Alors il ne pouvait pas être le Père dont il était le fils}. Quand il dit : ‘Celui qui m'a vu a vu le Père’, il doit vouloir dire que, comme \textbf{il était l'empreinte de la personne du Père, ceux qui le voyaient voyaient la ressemblance du Père en lui}.}[The Adventist Review and Sabbath Herald, September 18, 1855][https://documents.adventistarchives.org/Periodicals/RH/RH18550918-V07-06.pdf]


It is important to pay attention to the biblical evidence that brother Loughborough points out in the testimony that God has a body. Brother Loughborough reviews several Bible passages proving that God does have a material body, but it is invisible to our mortal eyes. Sister White wrote the same when she said\egwinline{\textbf{The Father is all the fulness of the Godhead \underline{bodily}} and \textbf{is invisible to mortal sight}}[Ms21-1906.9; 1906][https://egwwritings.org/read?panels=p9754.16]. No mortal eye can see the Father, but that does not prove that God can never be seen. Jesus said: \bible{\textbf{He that hath seen me, hath seen the Father}}[John 14:19]. Jesus explained these words two chapters prior: \bible{Jesus cried and said, He that believeth on me, believeth not on me, \textbf{but on him that sent me}. And \textbf{he that seeth me seeth him that sent me}}[John 12:44-45]. Jesus did not send Himself, neither is Jesus the Father, one and the same person; but we see the Father in Christ because He is the \textit{express image of the Father's person}. (Hebrews 1:3). As Jesus is a person, possessing a body, so is the Father. Brother Loughborough continues to prove his point that God is a person, possessing form and shape, because man was created in the image of God.


Il est important de prêter attention aux preuves bibliques que frère Loughborough souligne dans le témoignage que Dieu a un corps. Frère Loughborough passe en revue plusieurs passages bibliques prouvant que Dieu a effectivement un corps matériel, mais qu'il est invisible à nos yeux mortels. Sœur White a écrit la même chose quand elle a dit\egwinline{\textbf{Le Père est toute la plénitude de la Divinité \underline{corporellement}} et \textbf{est invisible à la vue mortelle}}[Ms21-1906.9; 1906][https://egwwritings.org/read?panels=p9754.16]. Aucun œil mortel ne peut voir le Père, mais cela ne prouve pas que Dieu ne peut jamais être vu. Jésus a dit : \bible{\textbf{Celui qui m'a vu a vu le Père}}[Jean 14:19]. Jésus a expliqué ces paroles deux chapitres plus tôt : \bible{Jésus s'écria et dit : Celui qui croit en moi, ne croit pas en moi, \textbf{mais en celui qui m'a envoyé}. Et \textbf{celui qui me voit, voit celui qui m'a envoyé}}[Jean 12:44-45]. Jésus ne s'est pas envoyé lui-même, et Jésus n'est pas non plus le Père, une seule et même personne ; mais nous voyons le Père en Christ parce qu'Il est \textit{l'empreinte de la personne du Père}. (Hébreux 1:3). Comme Jésus est une personne, possédant un corps, ainsi en est-il du Père. Frère Loughborough continue à prouver son point que Dieu est une personne, possédant forme et figure, parce que l'homme a été créé à l'image de Dieu.


\others{But we will now return to the subject of The creation of man. \textbf{We have seen already that man’s being made in the image of God, could not refer to a moral image, for it would involve the absurdity that the lifeless clay of which man was formed, had a character like God}. \textbf{We now see the Scriptures clearly teach, that \underline{God is a person with a body and form}}. Then Genesis 1:26, may be understood to teach the fact, \textbf{that man was made in the form of God}. Other scriptures agree with this testimony. See Genesis 9:6. ‘Whoso sheddeth man’s blood, by man shall his blood be shed: \textbf{for in the image of God made he man}.’ \textbf{\underline{This testimony cannot apply to a spirit, or immaterial part of man: that which is in the image of God has blood}}. 1 Corinthians 11:7. ‘For a man indeed ought not to cover his head, \textbf{forasmuch as he is the image and glory of God}.’ James [Chap 3:9] speaking of the tongue says, ‘Therewith bless we God, even the Father; and therewith curse we men, \textbf{which are made after the similitude (likeness, resemblance – Webster) of God}.’ \textbf{The foregoing testimony settles the point, \underline{that the image of God does not refer to character but to form}}.}


\others{Mais nous allons maintenant revenir au sujet de la création de l'homme. \textbf{Nous avons déjà vu que le fait que l'homme soit fait à l'image de Dieu ne pouvait pas se référer à une image morale, car cela impliquerait l'absurdité que l'argile sans vie dont l'homme a été formé avait un caractère comme Dieu}. \textbf{Nous voyons maintenant que les Écritures enseignent clairement que \underline{Dieu est une personne avec un corps et une forme}}. Alors Genèse 1:26 peut être compris comme enseignant le fait que \textbf{l'homme a été fait dans la forme de Dieu}. D'autres écritures s'accordent avec ce témoignage. Voir Genèse 9:6. ‘Si quelqu'un verse le sang de l'homme, par l'homme son sang sera versé ; \textbf{car Dieu a fait l'homme à son image}.’ \textbf{\underline{Ce témoignage ne peut pas s'appliquer à un esprit, ou à une partie immatérielle de l'homme : ce qui est à l'image de Dieu a du sang}}. 1 Corinthiens 11:7. ‘L'homme ne doit pas se couvrir la tête, \textbf{puisqu'il est l'image et la gloire de Dieu}.’ Jacques [Chap 3:9] parlant de la langue dit : ‘Par elle nous bénissons Dieu notre Père, et par elle nous maudissons les hommes \textbf{faits à la ressemblance de Dieu}.’ \textbf{Le témoignage précédent règle le point, \underline{que l'image de Dieu ne se réfère pas au caractère mais à la forme}}.}


\othersnogap{Genesis 2:7. ‘\textbf{And the Lord God formed man of the dust of the ground, and breathed into his nostrils the breath of life; and man became a living soul}.’}[The Adventist Review and Sabbath Herald, September 18, 1855][https://documents.adventistarchives.org/Periodicals/RH/RH18550918-V07-06.pdf]


\othersnogap{Genèse 2:7. ‘\textbf{L'Éternel Dieu forma l'homme de la poussière de la terre, il souffla dans ses narines un souffle de vie et l'homme devint une âme vivante}.’}[The Adventist Review and Sabbath Herald, September 18, 1855][https://documents.adventistarchives.org/Periodicals/RH/RH18550918-V07-06.pdf]


God formed man in His own image. God is a person, having a body, shape and form, and He formed man into His own image. From this reasoning we derive the obvious meaning of the Scriptures’ testimony about the \emcap{personality of God}. If we make false conceptions regarding God’s person, we are in danger of misunderstanding the other truths which are connected with man’s nature (mortality of the soul, the state of the dead, etc.). In his article, Brother Loughborough continues on to explain the connection between false doctrine on the immortality of the soul and wrong conceptions regarding the \emcap{personality of God}. His article in the Review and Herald from September 18, was taken from his book “\textit{An Examination of the Scripture Testimony}\footnote{\href{https://egwwritings.org/?ref=en_MPC.2&para=961.2}{John Norton Loughborough, An Examination of the Scripture Testimony, 1855}}.


Dieu a formé l'homme à Sa propre image. Dieu est une personne, ayant un corps, une forme et une figure, et Il a formé l'homme à Sa propre image. De ce raisonnement, nous dérivons la signification évidente du témoignage des Écritures sur la \emcap{personnalité de Dieu}. Si nous nous faisons de fausses conceptions concernant la personne de Dieu, nous risquons de mal comprendre les autres vérités qui sont liées à la nature de l'homme (mortalité de l'âme, l'état des morts, etc.). Dans son article, frère Loughborough continue à expliquer le lien entre la fausse doctrine sur l'immortalité de l'âme et les conceptions erronées concernant la \emcap{personnalité de Dieu}. Son article dans la Review and Herald du 18 septembre a été tiré de son livre “\textit{An Examination of the Scripture Testimony}\footnote{\href{https://egwwritings.org/?ref=en_MPC.2&para=961.2}{John Norton Loughborough, An Examination of the Scripture Testimony, 1855}}.


% Is God a person? - by John N. Loughborough

\begin{titledpoem}
    \stanza{
        In heaven's realm, upon His throne, \\
        God dwells in form, not spirit alone. \\
        A tangible being with shape and face, \\
        Beyond our sight in that holy place.
    }

    \stanza{    
        His glory shines too bright to see, \\
        No mortal eyes bear such majesty. \\
        Yet through His Spirit, everywhere present, \\
        His power extends, divine and pleasant.
    }

    \stanza{
        In Christ we glimpse the Father's form, \\
        The express image, perfect and warm. \\
        For we are made in God's own shape, \\
        Not just in virtue, soul, or trait.
    }

    \stanza{
        The dust was fashioned by His hand, \\
        In His own image, as He planned. \\
        A person true with body real, \\
        Not formless mist, as some appeal.
    }

    \stanza{
        The Father bodily, yet unseen by eye, \\
        Waits for the pure in heart to draw nigh.
    }
\end{titledpoem}