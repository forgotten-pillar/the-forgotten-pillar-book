\qrchapter{https://forgottenpillar.com/rsc/en-fp-chapter12}{Heaven's reality}


\qrchapter{https://forgottenpillar.com/rsc/en-fp-chapter12}{La réalité du ciel}


The \emcap{personality of God} deals with the quality or state of God being a person. Whenever we look at the pioneer's work on the \emcap{personality of God}, we see that they were all in harmony with the view that God is a tangible \textit{being}, possessing both body and parts. We always see the same underlying reasoning, which differentiates the term ‘\textit{spirit}’ and the term ‘\textit{being}’. By differentiating these terms, they explain the quality or state of God being a person\footnote{\href{https://www.merriam-webster.com/dictionary/personality}{Merriam-Webster Dictionary} defines the word ‘\textit{personality}’ as “\textit{quality or state of being a person}”.}—a \emcap{personality of God}. All their conclusions are summed up in the first point of the \emcap{Fundamental Principles}. \others{There is \textbf{one God}, a \textbf{personal}, \textbf{spiritual being}, the Creator of all things, omnipotent, omniscient, … and \textbf{every-where present by his representative, the Holy Spirit}. Psalm 139:7.}[FPSDA 1.2][https://egwwritings.org/read?panels=p1299.6]


La \emcap{personnalité de Dieu} traite de la qualité ou l'état de Dieu d'être une personne. Chaque fois que nous examinons le travail des pionniers sur la \emcap{personnalité de Dieu}, nous voyons qu'ils étaient tous en harmonie avec l'idée que Dieu est un \textit{être} tangible, possédant à la fois un corps et des parties. Nous voyons toujours le même raisonnement sous-jacent, qui différencie le terme ‘\textit{esprit}’ et le terme ‘\textit{être}’. En différenciant ces termes, ils expliquent la qualité ou l'état de Dieu d'être une personne\footnote{\href{https://www.merriam-webster.com/dictionary/personality}{Le dictionnaire Merriam-Webster} définit le mot ‘\textit{personnalité}’ comme “\textit{la qualité ou l'état d'être une personne}”.}—une \emcap{personnalité de Dieu}. Toutes leurs conclusions sont résumées dans le premier point des \emcap{Principes Fondamentaux}. \others{Il y a \textbf{un seul Dieu}, un \textbf{être spirituel}, \textbf{personnel}, le Créateur de toutes choses, omnipotent, omniscient, … et \textbf{partout présent par son représentant, le Saint-Esprit}. Psaume 139:7.}[FPSDA 1.2][https://egwwritings.org/read?panels=p1299.6]


So far, in the pioneers’ work, we have seen that the \emcap{personality of God} is tightly connected to the reality of God’s presence. God is a personal spiritual being, having a body and shape; as such, His presence is cumbered to one locality—as the Bible says, in His temple, at His throne where He is surrounded with unapproachable glory. But He is everywhere present by His representative, the Holy Spirit. Obviously, the Holy Spirit is a spirit, and not a being, \bible{for a spirit hath not flesh and bones as ye see me have}, said Jesus (Luke 24:39). Christ is also a Being, like His Father. He is an express image of the Father’s person; therefore, He bears the same personality, or the quality or state of being a person, as His Father does.


Jusqu'à présent, dans le travail des pionniers, nous avons vu que la \emcap{personnalité de Dieu} est étroitement liée à la réalité de la présence de Dieu. Dieu est un être spirituel personnel, ayant un corps et une forme ; en tant que tel, Sa présence est limitée à un seul endroit—comme le dit la Bible, dans Son temple, à Son trône où Il est entouré d'une gloire inapprochable. Mais Il est partout présent par Son représentant, le Saint-Esprit. Évidemment, le Saint-Esprit est un esprit, et non un être, \bible{car un esprit n'a ni chair ni os, comme vous voyez que j'ai}, dit Jésus (Luc 24:39). Christ est aussi un Être, comme Son Père. Il est l'empreinte de la personne du Père ; par conséquent, Il porte la même personnalité, ou la qualité ou l'état d'être une personne, comme Son Père.


In our experience, when we present the original Seventh-day Adventist beliefs on the \emcap{personality of God} to our trinitarian brothers, as expressed in the first two points of the \emcap{Fundamental Principles}, they often claim that the statements in the \emcap{Fundamental Principles} are correct in some way, but the understanding attributed to the terms “\textit{personal spiritual being}” are false. They usually attempt to harmonize the \emcap{Fundamental Principles} with the Trinity doctrine by twisting the words “\textit{spiritual being}”, as if the word ‘\textit{spiritual}’ means something mysterious, suitable to equalize the \emcap{personality of God} and of Christ with the personality of the Holy Ghost\footnote{The quality or state of the Holy Spirit being a person is bearing witness, not having the form of a person. \egw{\textbf{The Holy Spirit has a personality}, \textbf{\underline{else} He could not \underline{bear witness} to our spirits} and with our spirits that we are the children of God. \textbf{He must also be a divine person}, \textbf{\underline{else} He could not \underline{search out} the secrets which lie hidden in the mind of God}. ‘For what man knoweth the things of a man save the spirit of man, which is in him; even so the things of God knoweth no man, but the Spirit of God.’ [1 Corinthians 2:11.]}[21LtMs, Ms 20, 1906, par. 32][https://egwwritings.org/read?panels=p14071.10296041&index=0]. It is crystal clear that the Holy Spirit is a person, yet not in the same way as the Father and the Son, as the Holy Spirit does not possess the quality of an outward physical personage like the Father and the Son do.}. The underlying problem comes down to the understanding of the heavenly realities. The Bible is not silent about heaven, and its realities, and our pioneers understood it well. Below we read about the explanation of the terms “\textit{spiritual being}” from James White and Uriah Smith in their book, “\textit{The Biblical Institute}”. The Bible explains these terms using the example of angels, which are “\textit{spiritual beings}”.


Dans notre expérience, lorsque nous présentons les croyances originales des Adventistes du Septième Jour sur la \emcap{personnalité de Dieu} à nos frères trinitaires, telles qu'exprimées dans les deux premiers points des \emcap{Principes Fondamentaux}, ils prétendent souvent que les déclarations dans les \emcap{Principes Fondamentaux} sont correctes d'une certaine manière, mais que la compréhension attribuée aux termes “\textit{être spirituel personnel}” est fausse. Ils tentent généralement d'harmoniser les \emcap{Principes Fondamentaux} avec la doctrine de la Trinité en déformant les mots “\textit{être spirituel}”, comme si le mot ‘\textit{spirituel}’ signifiait quelque chose de mystérieux, approprié pour égaliser la \emcap{personnalité de Dieu} et de Christ avec la personnalité du Saint-Esprit\footnote{La qualité ou l'état du Saint-Esprit d'être une personne est de rendre témoignage, non d'avoir la forme d'une personne. \egw{\textbf{Le Saint-Esprit a une personnalité}, \textbf{\underline{sinon} Il ne pourrait pas \underline{rendre témoignage} à nos esprits} et avec nos esprits que nous sommes les enfants de Dieu. \textbf{Il doit aussi être une personne divine}, \textbf{\underline{sinon} Il ne pourrait pas \underline{sonder} les secrets qui sont cachés dans l'esprit de Dieu}. ‘Car qui est-ce qui connaît ce qui est en l'homme, si ce n'est l'esprit de l'homme qui est en lui ? De même aussi, personne ne connaît ce qui est en Dieu, si ce n'est l'Esprit de Dieu.’ [1 Corinthiens 2:11.]}[21LtMs, Ms 20, 1906, par. 32][https://egwwritings.org/read?panels=p14071.10296041&index=0]. Il est parfaitement clair que le Saint-Esprit est une personne, mais pas de la même manière que le Père et le Fils, car le Saint-Esprit ne possède pas la qualité d'un personnage physique extérieur comme le Père et le Fils.}. Le problème sous-jacent se résume à la compréhension des réalités célestes. La Bible n'est pas silencieuse sur le ciel et ses réalités, et nos pionniers l'ont bien compris. Ci-dessous, nous lisons l'explication des termes “\textit{être spirituel}” de James White et Uriah Smith dans leur livre, “\textit{The Biblical Institute}”. La Bible explique ces termes en utilisant l'exemple des anges, qui sont des “\textit{êtres spirituels}”.


\begin{figure}[hp]
    \centering
    \includegraphics[width=1\linewidth]{images/uriah-smith.jpg}
    \caption*{Uriah Smith (1832-1903)}
    \label{fig:uriah-smith}
\end{figure}


\begin{figure}[hp]
    \centering
    \includegraphics[width=1\linewidth]{images/uriah-smith.jpg}
    \caption*{Uriah Smith (1832-1903)}
    \label{fig:uriah-smith}
\end{figure}


\others{\textbf{Angels are real beings}. They are described in the Bible as \textbf{possessing face, feet, wings} \&x. Ezekiel says of the cherubim, ‘\textbf{Their whole \underline{body} and their backs and their hands and their wings},’ \&c. Eze. 10:12. Angels \textbf{appeared }unto Abraham. Gen. 18:1-8. They talked and ate with him. They went on to Sodom and communed with Lot, who, entering into his house baked unleavened bread for them and they did eat. \textbf{These person were called angels}. David speaks of the manna as the corn of Heaven and angel’s food. Ps. 78:23-25.}


\others{\textbf{Les anges sont des êtres réels}. Ils sont décrits dans la Bible comme \textbf{possédant un visage, des pieds, des ailes} \&x. Ézéchiel dit des chérubins, ‘\textbf{Tout leur \underline{corps}, et leur dos, et leurs mains, et leurs ailes},’ \&c. Éz. 10:12. Les anges \textbf{apparurent} à Abraham. Gen. 18:1-8. Ils parlèrent et mangèrent avec lui. Ils allèrent à Sodome et communiquèrent avec Lot, qui, entrant dans sa maison, fit cuire pour eux des pains sans levain et ils mangèrent. \textbf{Ces personnes étaient appelées anges}. David parle de la manne comme du blé du Ciel et de la nourriture des anges. Ps. 78:23-25.}


\othersnogap{The case of Balaam, Num. 22:22-31, is an interesting incident. The angel \textbf{appeared }to Balaam with a sword \textbf{drawn in his hand}. The question is sometimes asked \textbf{how angels can be \underline{material beings since we cannot see them}. This case illustrates it}. The record says the \textbf{Lord opened the eyes of Balaam and he saw the angel}. \textbf{The angel did not create a body for that occasion}.\textbf{ He was just the same as he was before Balaam saw him; \underline{but the change took place in Balaam}. His eyes were opened, then he beheld the angel}. It was the same with the servant of Elisha when he and his master were brought into a straight place, surrounded by the army of the king of Syria. 2 Kings 6:17. Elisha prayed that \textbf{the eyes of his servant might be opened}; and he immediately saw the whole mountain full of horses and chariots round about Elisha.}


\othersnogap{Le cas de Balaam, Nom. 22:22-31, est un incident intéressant. L'ange \textbf{apparut} à Balaam avec une épée \textbf{nue dans sa main}. La question est parfois posée \textbf{comment les anges peuvent être des \underline{êtres matériels puisque nous ne pouvons pas les voir}}. Ce cas l'illustre. Le récit dit que le \textbf{Seigneur ouvrit les yeux de Balaam et il vit l'ange}. \textbf{L'ange n'a pas créé un corps pour cette occasion}.\textbf{ Il était exactement le même qu'avant que Balaam le voie ; \underline{mais le changement s'est produit en Balaam}. Ses yeux furent ouverts, alors il vit l'ange}. Il en fut de même avec le serviteur d'Élisée quand lui et son maître furent mis dans une situation difficile, entourés par l'armée du roi de Syrie. 2 Rois 6:17. Élisée pria pour que \textbf{les yeux de son serviteur soient ouverts} ; et il vit immédiatement toute la montagne pleine de chevaux et de chars autour d'Élisée.}


\othersnogap{\textbf{This may be further illustrated referring to things which we know are material and yet which we cannot see}. Air is material, light is material, even thought itself is only the result of material organizations — matter acting upon matter — and yet we can see none of these things. \textbf{Just so with the angels}.}


\othersnogap{\textbf{Ceci peut être davantage illustré en se référant à des choses que nous savons être matérielles et que pourtant nous ne pouvons pas voir}. L'air est matériel, la lumière est matérielle, même la pensée elle-même n'est que le résultat d'organisations matérielles — la matière agissant sur la matière — et pourtant nous ne pouvons voir aucune de ces choses. \textbf{Il en est de même avec les anges}.}


\othersnogap{\textbf{It is further objected to the materiality of the angels that they are called spirits. }Heb. 1:13, 14.\textbf{\underline{But this is no objection to their being literal beings}}. \textbf{They are simply spiritual beings organized differently from these earthly bodies which we possess}. Paul says, 1 Cor. 15:44, ‘\textbf{There is a natural body and there is \underline{a spiritual body}}.’ \textbf{The natural body we now have; the spiritual body we shall have in the resurrection}. ‘\textbf{It is raised a spiritual body}.’ Verse 44. \textbf{But then we are equal unto the angels}, Luke 20:36; \textbf{then we have bodies like unto Christ’s most glorious body}. Phil. 3:4\footnote{Typo: It should be Philippians 3:21} \textbf{and Christ is no less a spirit than the angels}. \textbf{We read that God is a spirit, that is, simply \underline{a spiritual being}}.}[James White and Uriah Smith, The Biblical Institute (Kindle Locations 2537-2553). Kindle Edition.]


\othersnogap{\textbf{On objecte en outre à la matérialité des anges qu'ils sont appelés esprits. }Héb. 1:13, 14.\textbf{\underline{Mais ce n'est pas une objection à ce qu'ils soient des êtres littéraux}}. \textbf{Ils sont simplement des êtres spirituels organisés différemment de ces corps terrestres que nous possédons}. Paul dit, 1 Cor. 15:44, ‘\textbf{Il y a un corps naturel et il y a \underline{un corps spirituel}}.’ \textbf{Le corps naturel, nous l'avons maintenant ; le corps spirituel, nous l'aurons à la résurrection}. ‘\textbf{Il ressuscite corps spirituel}.’ Verset 44. \textbf{Mais alors nous sommes égaux aux anges}, Luc 20:36 ; \textbf{alors nous avons des corps semblables au corps très glorieux de Christ}. Phil. 3:4\footnote{Erreur typographique : Il devrait être Philippiens 3:21} \textbf{et Christ n'est pas moins un esprit que les anges}. \textbf{Nous lisons que Dieu est un esprit, c'est-à-dire, simplement \underline{un être spirituel}}.}[James White et Uriah Smith, The Biblical Institute (Kindle Locations 2537-2553). Kindle Edition.]


The Bible gives us the insight that angels are spiritual beings that possess material bodies, but are still unseen to us, unless the Lord opens our eyes to see them. When the righteous will rise up in their new glorified bodies, they will rise in a spiritual body, an incorruptible one. This body will be tangible and material just as the new Earth will be tangible and material. And with our spiritual bodies we will possess the renewed Earth, we will replenish it \bible{and subdue it: and have dominion over the fish of the sea, and over the fowl of the air, and over every living thing that moveth upon the earth}[Genesis 1:28].


La Bible nous donne l'aperçu que les anges sont des êtres spirituels qui possèdent des corps matériels, mais qui nous sont encore invisibles, à moins que le Seigneur n'ouvre nos yeux pour les voir. Quand les justes ressusciteront dans leurs nouveaux corps glorifiés, ils ressusciteront dans un corps spirituel, un corps incorruptible. Ce corps sera tangible et matériel tout comme la nouvelle Terre sera tangible et matérielle. Et avec nos corps spirituels nous posséderons la Terre renouvelée, nous la remplirons \bible{et l'assujettissons ; et dominerons sur les poissons de la mer, et sur les oiseaux des cieux, et sur toute bête qui se meut sur la terre}[Genèse 1:28].


% Heaven's reality

\begin{titledpoem}
    
    \stanza{
        God is not a vapor unknown, \\
        Not as a mystery on His throne. \\
        In just one place as beings are, \\
        Yet by His Spirit spread afar.
    }

    \stanza{
        Christ bears God’s image as His Son, \\
        Two beings divine, not joined as one. \\
        Angels have bodies, yet unseen, \\
        Physical forms with heav’nly sheen.
    }

    \stanza{
        We cannot see their spirit frame, \\
        Until immortal life we claim. \\
        We’ll resurrect like them to be, \\
        And dwell with them eternally.
    }

    \stanza{
        God is not three in mystic blend, \\
        But Father, Son, Their Spirit send. \\
        The Father, Son are not obscure, \\
        They’re personal, we know for sure.
    }
    
\end{titledpoem}