\qrchapter{https://forgottenpillar.com/rsc/en-fp-chapter16}{Dr. Kellogg and pantheism}


\qrchapter{https://forgottenpillar.com/rsc/en-fp-chapter16}{Dr. Kellogg et le panthéisme}


From her personal diary, on January 5, 1902, Sister White wrote that Kellogg’s \egwinline{science of God in nature is \textbf{true}}.


Dans son journal personnel, le 5 janvier 1902, Sœur White a écrit que la \egwinline{science de Dieu dans la nature de Kellogg est \textbf{vraie}}.


\egw{I am having things presented to me that worry my mind. Dr. Kellogg is traveling the same road that he did soon after taking up his responsibilities in the Sanitarium. \textbf{Human science is a lie in regard to God not having a personality}. I know this is a falsehood, and yet if we can in any way help the doctor we must try to do this. What can be said? There is such an exaltation given him that he is about to topple over the precipice. What can any of us do? The Lord alone can save Dr. Kellogg. \textbf{\underline{His science of God in nature is true}}, but he has placed nature where God should be. Nature is not God, but God created nature. \textbf{\underline{This science of God in nature is correct in one sense}}. \textbf{God gives to nature its life, its living properties, its beauty}. [He] is the author of all nature’s loveliness, and while He gives us this evidence of mighty power, \textbf{He is a personal God and Christ is a personal Saviour}.}[Ms236-1902.1; 1902][https://egwwritings.org/read?panels=p12779.6]


\egw{On me présente des choses qui préoccupent mon esprit. Le Dr Kellogg emprunte le même chemin qu'il a pris peu après avoir assumé ses responsabilités au Sanatorium. \textbf{La science humaine est un mensonge en ce qui concerne le fait que Dieu n'ait pas de personnalité}. Je sais que c'est une fausseté, et pourtant si nous pouvons d'une manière ou d'une autre aider le docteur, nous devons essayer de le faire. Que peut-on dire ? Il reçoit une telle exaltation qu'il est sur le point de basculer par-dessus le précipice. Que pouvons-nous faire ? Le Seigneur seul peut sauver le Dr Kellogg. \textbf{\underline{Sa science de Dieu dans la nature est vraie}}, mais il a placé la nature là où Dieu devrait être. La nature n'est pas Dieu, mais Dieu a créé la nature. \textbf{\underline{Cette science de Dieu dans la nature est correcte dans un sens}}. \textbf{Dieu donne à la nature sa vie, ses propriétés vivantes, sa beauté}. [Il] est l'auteur de toute la beauté de la nature, et tandis qu'Il nous donne cette preuve de puissance immense, \textbf{Il est un Dieu personnel et Christ est un Sauveur personnel}.}[Ms236-1902.1; 1902][https://egwwritings.org/read?panels=p12779.6]


\egwnogap{\textbf{We take not the fallacies of man but the Word of God that man was created after the image of God and Christ}, for the Word declares ‘God, who at sundry times and in divers manners spake in time past unto the fathers by the prophets, hath in these last days spoken unto us by his son, whom he hath appointed heir of all things, \textbf{by whom also he made the worlds; who being the brightness of his glory, and \underline{the express image of his person}}, and upholding all things by the word of his power, when he had by himself purged our sins, \textbf{sat down on the right hand of the Majesty of heaven}.’ Hebrews 1:1-3.}[Ms236-1902.4; 1902][https://egwwritings.org/read?panels=p12779.9]


\egwnogap{\textbf{Nous ne prenons pas les erreurs de l'homme mais la Parole de Dieu que l'homme a été créé à l'image de Dieu et de Christ}, car la Parole déclare « Dieu ayant autrefois parlé à nos pères, à plusieurs reprises et en plusieurs manières, par les prophètes, nous a parlé en ces derniers jours par son Fils, qu'il a établi héritier de toutes choses ; \textbf{par lequel aussi il a fait les mondes ; lequel étant la splendeur de sa gloire, et \underline{l'empreinte de sa personne}}, et soutenant toutes choses par la parole de sa puissance, après avoir fait par lui-même la purification de nos péchés, \textbf{s'est assis à la droite de la Majesté divine dans les lieux très hauts}. » Hébreux 1:1-3.}[Ms236-1902.4; 1902][https://egwwritings.org/read?panels=p12779.9]


Interestingly, Sister White also claimed that God is in nature, and He is giving life and the living properties. Kellogg is correct on this point and his claim is definitely supported by her writings. Based on this point, Kellogg defended himself, saying that The Living Temple is in harmony with Sister White’s writings. He wrote to brother G. I. Butler precisely where Sister White advocated the same sentiment as he did.


Il est intéressant de noter que Sœur White a également affirmé que Dieu est dans la nature, et qu'Il donne la vie et les propriétés vivantes. Kellogg a raison sur ce point et son affirmation est définitivement soutenue par ses écrits. Sur la base de ce point, Kellogg s'est défendu, disant que Le Temple Vivant est en harmonie avec les écrits de Sœur White. Il a écrit au frère G. I. Butler précisément où Sœur White défendait le même raisonnement que lui.


\others{Sister White has clearly taken the same position with reference to this matter which I have taken. You will find it, in her little work on \textbf{Education }in the chapters ‘\textbf{God in Nature}’ and ‘\textbf{Science and the Bible.}’ You will find it all through ‘\textbf{Desire of Ages,}’ and ‘\textbf{Patriarchs and Prophets.}’}[Letter from Dr. Kellogg to Eld. Butler, February 21, 1904]


\others{Sœur White a clairement pris la même position concernant cette question que j'ai prise. Vous la trouverez dans son petit ouvrage sur l’\textbf{Éducation} dans les chapitres « \textbf{Dieu dans la Nature} » et « \textbf{La Science et la Bible}. » Vous la trouverez tout au long de « \textbf{Jésus-Christ}, » et « \textbf{Patriarches et Prophètes}. »}[Lettre du Dr Kellogg à Eld. Butler, 21 février 1904]


Let’s take a look at “\textit{God in Nature}”, in the book Education, where we can find the same sentiment regarding God in Nature that Kellogg promoted.


Examinons « \textit{Dieu dans la Nature} », dans le livre Éducation, où nous pouvons trouver le même raisonnement concernant Dieu dans la Nature que Kellogg a promu.


\egw{\textbf{Upon all created things is seen the impress of the Deity}. Nature testifies of God. The susceptible mind, brought in contact with the miracle and mystery of the universe, cannot but recognize \textbf{the working of infinite power}. \textbf{\underline{Not by its own inherent energy} does the earth produce its bounties}, and year by year continue its motion around the sun. \textbf{An unseen hand guides the planets in their circuit of the heavens}. \textbf{\underline{A mysterious life pervades all nature—a life that sustains the unnumbered worlds throughout immensity}}, \textbf{that lives in the insect atom which floats in the summer breeze, that wings the flight of the swallow and feeds the young ravens which cry, that brings the bud to blossom and the flower to fruit}.}[Ed 99.1; 1903][https://egwwritings.org/read?panels=p29.470]


\egw{\textbf{Sur toutes les choses créées se voit l'empreinte de la Divinité}. La nature témoigne de Dieu. L'esprit réceptif, mis en contact avec le miracle et le mystère de l'univers, ne peut que reconnaître \textbf{l'œuvre de la puissance infinie}. \textbf{\underline{Ce n'est pas par sa propre énergie inhérente} que la terre produit ses bienfaits}, et continue année après année son mouvement autour du soleil. \textbf{Une main invisible guide les planètes dans leur circuit des cieux}. \textbf{\underline{Une vie mystérieuse imprègne toute la nature—une vie qui soutient les mondes innombrables à travers l'immensité}}, \textbf{qui vit dans l'atome insecte qui flotte dans la brise d'été, qui donne des ailes au vol de l'hirondelle et nourrit les jeunes corbeaux qui crient, qui amène le bourgeon à fleurir et la fleur à porter du fruit}.}[Ed 99.1; 1903][https://egwwritings.org/read?panels=p29.470]


\egwnogap{\textbf{The same \underline{power} that upholds nature, is working also in man}. \textbf{The same great laws that guide alike the star and the atom control human life}. \textbf{The laws that govern the heart’s action, regulating the flow of the current of life to the body, are the laws of the mighty Intelligence that has the jurisdiction of the soul}. \textbf{\underline{From Him all life proceeds}}. Only in harmony with Him can be found its true sphere of action. For all the objects of His creation the condition is the same—\textbf{a life sustained by receiving the life of God}, a life exercised in harmony with the Creator’s will...}[Ed 99.2; 1903][https://egwwritings.org/read?panels=p29.471]


\egwnogap{\textbf{La même \underline{puissance} qui soutient la nature, travaille aussi dans l'homme}. \textbf{Les mêmes grandes lois qui guident à la fois l'étoile et l'atome contrôlent la vie humaine}. \textbf{Les lois qui gouvernent l'action du cœur, régulant le flux du courant de vie vers le corps, sont les lois de la puissante Intelligence qui a la juridiction de l'âme}. \textbf{\underline{De Lui procède toute vie}}. Ce n'est qu'en harmonie avec Lui que peut être trouvée sa vraie sphère d'action. Pour tous les objets de Sa création, la condition est la même—\textbf{une vie soutenue en recevant la vie de Dieu}, une vie exercée en harmonie avec la volonté du Créateur...}[Ed 99.2; 1903][https://egwwritings.org/read?panels=p29.471]


\egw{…The heart not yet hardened by contact with evil is quick to \textbf{recognize the \underline{Presence} that pervades all created things}…}[Ed 100.2; 1903][https://egwwritings.org/read?panels=p29.475]


\egw{…Le cœur qui n'est pas encore endurci par le contact avec le mal est prompt à \textbf{reconnaître la \underline{Présence} qui imprègne toutes les choses créées}…}[Ed 100.2; 1903][https://egwwritings.org/read?panels=p29.475]


In his defense, Kellogg was also referring to the Patriarchs and Prophets. There we read the following:


Dans sa défense, Kellogg faisait également référence aux Patriarches et Prophètes. Nous y lisons ce qui suit :


\egw{Many teach that matter possesses vital power,—that certain properties are imparted to matter, and it is then left to act through its own inherent energy; and that the operations of nature are conducted in harmony with fixed laws, with which God himself cannot interfere. \textbf{This is false science, and is not sustained by the word of God}. Nature is the servant of her Creator. God does not annul his laws, or work contrary to them; \textbf{but he is continually using them as his instruments. Nature testifies of an intelligence, \underline{a presence}, \underline{an active energy}, that works in and through her laws. There is in nature the continual working of \underline{the Father and the Son}.} Christ says, ‘My Father worketh hitherto, and I work.’ John 5:17.}[PP 114.4; 1980][https://egwwritings.org/read?panels=p84.445]


\egw{Beaucoup enseignent que la matière possède une puissance vitale, — que certaines propriétés sont communiquées à la matière, et qu'elle est ensuite laissée à agir par sa propre énergie inhérente ; et que les opérations de la nature sont conduites en harmonie avec des lois fixes, avec lesquelles Dieu lui-même ne peut interférer. \textbf{C'est une fausse science, et elle n'est pas soutenue par la parole de Dieu}. La nature est la servante de son Créateur. Dieu n'annule pas ses lois, ni n'agit contrairement à elles ; \textbf{mais il les utilise continuellement comme ses instruments. La nature témoigne d'une intelligence, \underline{d'une présence}, \underline{d'une énergie active}, qui agit dans et à travers ses lois. Il y a dans la nature l'œuvre continuelle du \underline{Père et du Fils}.} Christ dit : « Mon Père agit jusqu'à présent, et j'agis aussi. » Jean 5:17.}[PP 114.4; 1980][https://egwwritings.org/read?panels=p84.445]


These quotations are in harmony with the quotations from The Living Temple.


Ces citations sont en harmonie avec les citations du Temple Vivant.


\others{The manifestations of life are as varied as the different individual animals and plants, and parts of animated things. Every leaf, every blade of grass, every flower, every bird, even every insect, as well as every beast or every tree, bears witness to the infinite versatility and inexhaustible resources of \textbf{the one all-pervading, all-creating, all-sustaining Life}.}[John H. Kellogg, The Living Temple p. 16][https://archive.org/details/J.H.Kellogg.TheLivingTemple1903/page/n15/]


\others{Les manifestations de la vie sont aussi variées que les différents animaux et plantes individuels, et les parties des choses animées. Chaque feuille, chaque brin d'herbe, chaque fleur, chaque oiseau, même chaque insecte, ainsi que chaque bête ou chaque arbre, témoigne de la polyvalence infinie et des ressources inépuisables de \textbf{la Vie unique omniprésente, toute créatrice, toute soutenante}.}[John H. Kellogg, Le Temple Vivant p. 16][https://archive.org/details/J.H.Kellogg.TheLivingTemple1903/page/n15/]


\others{Intelligence is one of the forces of the universe, one of the manifestations of the \textbf{\underline{all-pervading life which} created and creates, \underline{animates and sustains}}.}[John H. Kellogg, The Living Temple p. 396][https://archive.org/details/J.H.Kellogg.TheLivingTemple1903/page/n425/]


\others{L'intelligence est l'une des forces de l'univers, l'une des manifestations de la \textbf{\underline{vie omniprésente qui} a créé et crée, \underline{anime et soutient}}.}[John H. Kellogg, Le Temple Vivant p. 396][https://archive.org/details/J.H.Kellogg.TheLivingTemple1903/page/n425/]


If Kellogg’s understanding of God as the source that sustains and animates nature is correct, then where is his error? Why is he called a pantheist? Is it fair to call him a pantheist? He definitely doesn’t think so. Take a look at what he wrote to Elder Butler:


Si la compréhension de Kellogg de Dieu comme la source qui soutient et anime la nature est correcte, alors où est son erreur ? Pourquoi est-il appelé panthéiste ? Est-il juste de l'appeler panthéiste ? Il ne le pense certainement pas. Regardez ce qu'il a écrit à l'Ancien Butler :


\others{\textbf{I abhor pantheism} as much as you do. \textbf{I have endeavored in my book to simply teach the fact that man is dependent upon God for everything, and that without the divine power working in him the Spirit of God operating upon the elements which compose his body, he would be dust}.}[Letter from Dr. Kellogg to Eld. Butler, February 21, 1904]


\others{\textbf{J'abhorre le panthéisme} autant que vous. \textbf{J'ai essayé dans mon livre d'enseigner simplement le fait que l'homme dépend de Dieu pour tout, et que sans la puissance divine agissant en lui, l'Esprit de Dieu opérant sur les éléments qui composent son corps, il serait poussière}.}[Lettre du Dr Kellogg à l'Ancien Butler, 21 février 1904]


\others{I am willing to renounce all the awful doctrines you and others attribute to me. I am willing to confess that \textbf{I am not a pantheist} nor a spiritualist, and that I believe none of the doctrines taught by these people or \textbf{by pantheistic or spiritualistic writings}. I never read a pantheistic book in my life. I never read a book on ‘New Thought,’ or anything of that kind. Anybody who will read carefully the ‘Living Temple’ from the first page right straight through to the last, and will give the matter fair and consistent consideration, ought to see very clearly that \textbf{I have no accord whatever with these pantheistic and spiritualistic theories}.}[Ibid.]


\others{Je suis prêt à renoncer à toutes les doctrines terribles que vous et d'autres m'attribuez. Je suis prêt à confesser que \textbf{je ne suis pas panthéiste} ni spiritualiste, et que je ne crois à aucune des doctrines enseignées par ces gens ou \textbf{par des écrits panthéistes ou spiritualistes}. Je n'ai jamais lu un livre panthéiste de ma vie. Je n'ai jamais lu un livre sur la « Nouvelle Pensée », ou quoi que ce soit de ce genre. Quiconque lira attentivement le « Temple Vivant » de la première page jusqu'à la dernière, et donnera à la question une considération juste et cohérente, devrait voir très clairement que \textbf{je n'ai aucun accord avec ces théories panthéistes et spiritualistes}.}[Ibid.]


This is a very hard puzzle to solve unless you encounter the truth on the \emcap{personality of God}, which we covered in the beginning of this book. Yes, God sustains life in nature. In nature, we \egwinline{\textbf{recognize \underline{the Presence} that pervades all created things}}[Ed 100.2; 1903][https://egwwritings.org/read?panels=p29.475]. But God \textit{Himself}—in His personality—is not in nature, nor is nature God. God is a \textit{personal being}, and He is in His holy temple, sitting on His throne. God is everywhere present by His \textit{representative}, the Holy Spirit.


C'est un puzzle très difficile à résoudre à moins de rencontrer la vérité sur la \emcap{personnalité de Dieu}, que nous avons couverte au début de ce livre. Oui, Dieu soutient la vie dans la nature. Dans la nature, nous \egwinline{\textbf{reconnaissons \underline{la Présence} qui imprègne toutes les choses créées}}[Ed 100.2; 1903][https://egwwritings.org/read?panels=p29.475]. Mais Dieu \textit{Lui-même} — dans Sa personnalité — n'est pas dans la nature, et la nature n'est pas Dieu. Dieu est un \textit{être personnel}, et Il est dans Son temple saint, assis sur Son trône. Dieu est partout présent par Son \textit{représentant}, le Saint-Esprit.


When Sister White said \egwinline{Human science is a lie in regard to God \textbf{not having a personality},}[Ms236-1902; 1902][https://egwwritings.org/read?panels=p12779.6] she was particularly referencing God having a physical form of a person, as could be seen in the context of that quotation. But when Dr. Kellogg was addressing ‘\textit{personality},’ he was not addressing the form or shape of a person. In 1936 in his lecture, he expressed the same sentiments he held in the Living Temple, only more vividly:


Quand Sœur White a dit \egwinline{La science humaine est un mensonge en ce qui concerne le fait que Dieu \textbf{n'a pas de personnalité},}[Ms236-1902; 1902][https://egwwritings.org/read?panels=p12779.6] elle faisait particulièrement référence au fait que Dieu a une forme physique de personne, comme on pouvait le voir dans le contexte de cette citation. Mais quand le Dr Kellogg abordait la « \textit{personnalité} », il n'abordait pas la forme ou l'apparence d'une personne. En 1936, dans sa conférence, il a exprimé le même raisonnement qu'il avait dans le Temple Vivant, seulement de manière plus vivante :


\others{So you see it is impossible to conceive of infinite things. They are beyond us. They are \textbf{outside of comprehension} and the same thing is true of \textbf{the \underline{infinite personality}}. \textbf{We can not form any conception of its shape or its size or any limitations of any sort because it is infinite}. Now, perhaps that is a difficult idea for you to take in and \textbf{the difficulty of accepting this idea is the fact that \underline{we have not a clear idea of personality}}. \textbf{We think of personality \underline{as connected with form}}.}


\others{Vous voyez donc qu'il est impossible de concevoir des choses infinies. Elles sont au-delà de nous. Elles sont \textbf{hors de compréhension} et la même chose est vraie de \textbf{la \underline{personnalité infinie}}. \textbf{Nous ne pouvons former aucune conception de sa forme ou de sa taille ou de quelque limitation que ce soit parce qu'elle est infinie}. Maintenant, c'est peut-être une idée difficile à saisir pour vous et \textbf{la difficulté d'accepter cette idée est le fait que \underline{nous n'avons pas une idée claire de la personnalité}}. \textbf{Nous pensons à la personnalité \underline{comme étant liée à la forme}}.}


\others{…\textbf{It gave me a new conception of personality}. \textbf{\underline{Personality does not mean a person, a man or a woman}}. It does not mean that sort of thing at all. \textbf{It means the possession of the power to will and to do and to think and to plan}.}[\href{https://forgotten-pillar.s3.us-east-2.amazonaws.com/Sanitarium+Lecture+1936.pdf}{Dr. Kellogg Sanitarium Lectures, 1936}; For transcript see \href{https://notefp.link/1938-kellogg-lecture}{https://notefp.link/1938-kellogg-lecture}]


\others{…\textbf{Cela m'a donné une nouvelle conception de la personnalité}. \textbf{\underline{La personnalité ne signifie pas une personne, un homme ou une femme}}. Cela ne signifie pas du tout ce genre de chose. \textbf{Cela signifie la possession du pouvoir de vouloir et de faire et de penser et de planifier}.}[\href{https://forgotten-pillar.s3.us-east-2.amazonaws.com/Sanitarium+Lecture+1936.pdf}{Dr. Kellogg Sanitarium Lectures, 1936}; Pour la transcription voir \href{https://notefp.link/1938-kellogg-lecture}{https://notefp.link/1938-kellogg-lecture}]


Such a view of personality applied to God led Dr. Kellogg into pantheism. The doctrine of the \emcap{personality of God} deals with the correct perception of God. Dr. Kellogg's perception of God was a trinitarian perception.


Une telle vision de la personnalité appliquée à Dieu a conduit le Dr Kellogg au panthéisme. La doctrine de la \emcap{personnalité de Dieu} traite de la perception correcte de Dieu. La perception de Dieu du Dr Kellogg était une perception trinitaire.


\others{All I wanted to explain in Living Temple was that this work that is going on in the man here \textbf{is not going on by itself \underline{like a clock wound up}; but it is the power of God and \underline{the Spirit of God that is carrying it on}}. \textbf{Now, I thought I had cut out entirely the theological side of questions of \underline{the trinity and all that sort of things}}. \textbf{I didn't mean to put it in at all}, and I took pains to state in the preface that I did not. I never dreamed \textbf{of such a thing} as any theological question being \textbf{brought into it}. I only wanted to show that \textbf{\underline{the heart does not beat of its own motion} but that it is \underline{the power of God that keeps it going}}.}[Interview, J. H. Kellogg, G. W. Amadon and A. C. Bourdeau, October 7th 1907 held at Kellogg’s residence][https://archive.org/details/KelloggVs.TheBrethrenHisLastInterviewAsAnAdventistoct71907/page/n37]


\others{Tout ce que je voulais expliquer dans Le Temple Vivant était que ce travail qui se passe dans l'homme ici \textbf{ne se passe pas par lui-même \underline{comme une horloge remontée} ; mais c'est la puissance de Dieu et \underline{l'Esprit de Dieu qui le poursuit}}. \textbf{Maintenant, je pensais avoir complètement éliminé le côté théologique des questions de \underline{la trinité et toutes ces sortes de choses}}. \textbf{Je n'avais pas l'intention de l'inclure du tout}, et j'ai pris soin de déclarer dans la préface que je ne l'avais pas fait. Je n'ai jamais rêvé \textbf{d'une telle chose} qu'une question théologique soit \textbf{soulevée}. Je voulais seulement montrer que \textbf{\underline{le cœur ne bat pas de son propre mouvement} mais que c'est \underline{la puissance de Dieu qui le fait continuer}}.}[Interview, J. H. Kellogg, G. W. Amadon et A. C. Bourdeau, 7 octobre 1907 tenue à la résidence de Kellogg][https://archive.org/details/KelloggVs.TheBrethrenHisLastInterviewAsAnAdventistoct71907/page/n37]


The heart does not beat of its own motion; it is the power of God that keeps it going. In this, Kellogg was absolutely right.


Le cœur ne bat pas de son propre mouvement ; c'est la puissance de Dieu qui le fait continuer. En cela, Kellogg avait absolument raison.


\egw{\textbf{The physical organism of man is under the supervision of God, but \underline{it is not like a clock which is set in operation and must go of itself}}. \textbf{The heart beats, pulse succeeds pulse, breath succeeds breath, but bear in mind that the being is under the supervision of God}. Ye are God's husbandry, ye are God's building. \textbf{In God we live and move and have our being}. \textbf{Each heartbeat, each breath is the inspiration of that God who breathed into the nostrils of Adam the breath of life}, the inspiration of the ever present God, the great I AM.}[13LtMs, Ms 92, 1898, par. 7][https://egwwritings.org/read?panels=p14063.7342012&index=0]


\egw{\textbf{L'organisme physique de l'homme est sous la supervision de Dieu, mais \underline{il n'est pas comme une horloge qui est mise en marche et doit fonctionner d'elle-même}}. \textbf{Le cœur bat, le pouls succède au pouls, le souffle succède au souffle, mais gardez à l'esprit que l'être est sous la supervision de Dieu}. Vous êtes le champ de Dieu, vous êtes l'édifice de Dieu. \textbf{En Dieu nous vivons et nous nous mouvons et nous avons notre être}. \textbf{Chaque battement de cœur, chaque souffle est l'inspiration de ce Dieu qui a soufflé dans les narines d'Adam le souffle de vie}, l'inspiration du Dieu toujours présent, le grand JE SUIS.}[13LtMs, Ms 92, 1898, par. 7][https://egwwritings.org/read?panels=p14063.7342012&index=0]


Dr. Kellogg's \egwinline{science of God in nature is true.}[Ms236-1902; 1902][https://egwwritings.org/read?panels=p12779.6] The Scriptures clearly teach it: \bible{If he \normaltext{[God]} set his heart upon man, \textbf{if he gather unto himself \underline{his spirit} and his breath}; \textbf{\underline{All flesh shall perish together}, and man shall turn again unto dust}.}[Job 34:14-15] \bible{…thy judgments are a great deep: \textbf{O Lord, thou \underline{preservest} man and beast}… \textbf{For with thee is the fountain of life}: in thy light shall we see light.}[Psalm 36:6b,9]


La \egwinline{science de Dieu dans la nature du Dr Kellogg est vraie.}[Ms236-1902; 1902][https://egwwritings.org/read?panels=p12779.6] Les Écritures l'enseignent clairement : \bible{S'il \normaltext{[Dieu]} ne pensait qu'à lui-même, \textbf{s'il retirait à lui \underline{son esprit} et son souffle}, \textbf{\underline{Toute chair expirerait à l'instant}, et l'homme retournerait dans la poussière.}[Job 34:14-15] \bible{…tes jugements sont comme le grand abîme. \textbf{Éternel, tu \underline{soutiens} les hommes et les bêtes}… \textbf{Car la source de la vie est auprès de toi} ; c'est par ta lumière que nous voyons la lumière.}[Psaume 36:6b,9]


This evidence testifies that Dr. Kellogg's science of God in nature is true, but his problems were erroneous views on the personality of God, which were trinitarian views. Even when he clarified that \others{God the Father sits upon his throne in heaven where God the Son is also; while God's life, or spirit or presence is the all-pervading power which is carrying out the will of God in all the universe,}[Letter: Dr. Kellogg to W. W. Prescott, October 25, 1903][https://forgotten-pillar.s3.us-east-2.amazonaws.com/1903-10-25-JHKellogg-to-W.W.Prescott.pdf] still he held erroneous views on the personality of God—God in \others{comprehensive sense} as \others{the Godhead… God the Father, God the Son, and God the Holy Spirit}[Ibid.][https://forgotten-pillar.s3.us-east-2.amazonaws.com/1903-10-25-JHKellogg-to-W.W.Prescott.pdf]. His Trinitarian view could \textit{not} \others{clear the matter up satisfactorily.}[Letter: A. G. Daniells to W. C. White, October 29, 1903][https://forgotten-pillar.s3.us-east-2.amazonaws.com/Letter-A-G-Daniells-to-W-C-White-October-29-1903.pdf]


Cette preuve témoigne que la science de Dieu dans la nature du Dr Kellogg est vraie, mais ses problèmes étaient des vues erronées sur la personnalité de Dieu, qui étaient des vues trinitaires. Même quand il a clarifié que \others{Dieu le Père siège sur son trône au ciel où Dieu le Fils est aussi ; tandis que la vie de Dieu, ou l'esprit ou la présence est la puissance omniprésente qui exécute la volonté de Dieu dans tout l'univers,}[Lettre : Dr. Kellogg à W. W. Prescott, 25 octobre 1903][https://forgotten-pillar.s3.us-east-2.amazonaws.com/1903-10-25-JHKellogg-to-W.W.Prescott.pdf] il maintenait toujours des vues erronées sur la personnalité de Dieu—Dieu dans un \others{sens complet} comme \others{la Divinité… Dieu le Père, Dieu le Fils, et Dieu le Saint-Esprit}[Ibid.][https://forgotten-pillar.s3.us-east-2.amazonaws.com/1903-10-25-JHKellogg-to-W.W.Prescott.pdf]. Sa vue trinitaire ne pouvait \textit{pas} \others{éclaircir la question de manière satisfaisante.}[Lettre : A. G. Daniells à W. C. White, 29 octobre 1903][https://forgotten-pillar.s3.us-east-2.amazonaws.com/Letter-A-G-Daniells-to-W-C-White-October-29-1903.pdf]


The conclusion is frightening. If you believe that the heart does not beat of its own motion but that it is the power of God that keeps it going, and you combine it with the belief that God Himself is not a tangible being but a spirit present everywhere, then in the eyes of the Spirit of Prophecy, you are a pantheist. The perception of the quality or state of God being a person makes the difference between the true believer and the pantheist.


La conclusion est effrayante. Si vous croyez que le cœur ne bat pas de son propre mouvement mais que c'est la puissance de Dieu qui le fait continuer, et que vous combinez cela avec la croyance que Dieu Lui-même n'est pas un être tangible mais un esprit présent partout, alors aux yeux de l'Esprit de Prophétie, vous êtes un panthéiste. La perception de la qualité ou l'état de Dieu d'être une personne fait la différence entre le vrai croyant et le panthéiste.


% Dr. Kellogg and pantheism

\begin{titledpoem}
    
    \stanza{
        In nature’s vast, a truth untold, \\
        He said God was in every fold. \\
        The trees, the breeze, the soil, the sea, \\
        God’s presence there, for all to see.
    }

    \stanza{
        Yet, in this truth where we concur, \\
        A deeper error did occur. \\
        The Trinity, unsacred bond, \\
        As pantheism and beyond.
    }

    \stanza{
        God’s personality is clear, \\
        Beyond those frontiers, we revere. \\
        For God, who’s more than nature’s face, \\
        Is personal, in sacred space.
    }

    \stanza{
        The doctor’s path did lead astray, \\
        On trinity, we cannot sway. \\
        His view of God, misunderstood, \\
        A misstep from the path of good.
    }

    \stanza{
        In nature, power does reside, \\
        It’s not God’s body that presides. \\
        Beside Him, Christ stands as our guide, \\
        And by His Spirit, life abides.
    }

    \stanza{
        In nature’s charm, God’s hand we see, \\
        Beyond the vastness, He must be. \\
        A precious God, with love so wide, \\
        In whom, in peace, we can confide.  
    }
    
\end{titledpoem}