\qrchapter{https://forgottenpillar.com/rsc/en-fp-chapter26}{The steps to Omega}


\qrchapter{https://forgottenpillar.com/rsc/en-fp-chapter26}{Hatua za Omega}


In our study so far, we have seen evidence that Kellogg’s controversy was connected to the Trinity doctrine and the \emcap{personality of God} expressed in the first point of the \emcap{Fundamental Principles}. Unfortunately, today we do not stand on that foundation regarding the \emcap{personality of God}; we have built another foundation that has changed the truth on the \emcap{personality of God} to a mysterious Triune God. Sister White was clearly against this reorganization and she prophesied that in the closing of His work, God will rehearse the history of the Advent movement and re-establish every pillar of our faith that was held in the beginning.


Katika utafiti wetu kufikia hadi sasa, tumeona ushahidi kwamba utata wa Kellogg uliunganishwa na Fundisho la Utatu na \emcap{Umbile la Mungu} unaoonyeshwa katika jambo la kwanza la \emcap{Kanuni za Kimsingi}. Kwa bahati mbaya, leo hatusimami juu ya msingi huo kuhusu \emcap{Umbile la Mungu}; tumejenga msingi mwingine ambao umebadilisha ukweli juu ya \emcap{Umbile la Mungu} kwa Mungu wa Utatu wa ajabu. Dada White alikuwa wazi dhidi ya upangaji upya huu na alitabiri kwamba katika kuhitimisha kazi yake, Mungu atarudia historia ya harakati ya Majilio na kusimamisha tena kila nguzo ya imani yetu iliyoshikiliwa hapo mwanzo.


\egw{\textbf{\underline{The Lord has declared that the history of the past shall be rehearsed as we enter upon the closing work}. \underline{Every truth} that He has given for these last days is to be proclaimed to the world. \underline{Every pillar} that He has established \underline{is to be strengthened}. We cannot now step off the foundation that God has established. We cannot now enter into any new organization; for this would mean apostasy from the truth}.}[Ms129-1905.6; 1905][https://egwwritings.org/read?panels=p9797.13]


\egw{\textbf{\underline{Bwana ametangaza kwamba historia ya zamani itafanyiwa ufafanuzi tunapoingia katika kazi ya kuhitimisha}. \underline{Kila ukweli} ambao ametoa kwa siku hizi za mwisho unapaswa kutangazwa kwa ulimwengu. \underline{Kila nguzo} aliyoisimamisha \underline{inapaswa kuimarishwa}. Hatuwezi sasa kuacha msingi ambao Mungu ameweka. Hatuwezi sasa kuingia katika shirika lolote jipya; kwa maana hilo lingemaanisha kuasi ukweli}.}[Ms129-1905.6; 1905][https://egwwritings.org/read?panels=p9797.13]


Comparing the \emcap{Fundamental Principles} with the current Fundamental Beliefs of Seventh-day Adventists, we see that we have entered into a new organization. God’s warning, given through Sister White, to re-establish all pillars of our faith in these last days, is becoming imperative. As we traced the Trinity doctrine from Kellogg's controversy, we came across Ellen White’s warnings against alpha and omega apostasy, which will enter into our church.


Kulinganisha \emcap{Kanuni za Kimsingi} na Imani za Sasa za Kimsingi za Waadventista Wasabato, tunaona kwamba tumeingia katika shirika jipya. Onyo la Mungu, limetolewa kupitia kwa Dada White, kusimamisha tena nguzo zote za imani yetu katika siku hizi za mwisho, inakuwa lazima. Tulipofuatilia fundisho la Utatu kutoka kwa mabishano ya Kellogg, tulikutana na maonyo ya Ellen White dhidi ya uasi wa alfa na omega, ambayo yataingia katika kanisa letu.


\egw{\textbf{‘Living Temple’ contains the alpha of these theories. I knew that \underline{the omega would follow in a little while}; and I trembled for our people}. I knew that \textbf{I must warn our brethren and sisters not to enter into controversy \underline{over the presence and personality of God}. The statements made in ‘Living Temple’ \underline{in regard to this point are incorrect}. }The scripture used to substantiate the doctrine there set forth, is scripture misapplied.}[SpTB02 53.2; 1904][https://egwwritings.org/read?panels=p417.271]


\egw{\textbf{‘The Living Temple’ lina alfa ya nadharia hizi. Nilijua kwamba \underline{omega ingefuata baada ya muda mfupi}; na nikatetemeka kwa ajili ya watu wetu}. Nilijua kwamba \textbf{lazima niwaonye kaka na dada zetu wasiingie katika mabishano \underline{juu ya uwepo na Umbile la Mungu}. Kauli zilizotolewa katika ‘Living Temple’ \underline{kuhusiana na jambo hili si sahihi}. }Andiko linalotumiwa kuthibitisha fundisho lililowekwa hapo, limetumiwa vibaya.}[SpTB02 53.2; 1904][https://egwwritings.org/read?panels=p417.271]


In the context of Seventh-day Adventist reorganization, we identify several steps that were necessary to accomplish this reorganization and are necessary to uphold it.


Katika muktadha wa upangaji upya wa Waadventista Wasabato, tunatambua hatua kadhaa ambazo zilikuwa muhimu kukamilisha upangaji upya huu na ni muhimu kuudumisha.


\subsection*{Step 1: Deny the Fundamental Principles to be the foundation of our faith and the official, and accurate, representation of Seventh-day Adventist beliefs}


\subsection*{Hatua ya 1: Kataa Kanuni za Kimsingi kama msingi ya imani yetu na uwakilishi rasmi, na sahihi, wa Imani za Waadventista Wasabato}


The first step necessary is to hide the original foundation of our faith by unlinking it with the \emcap{Fundamental Principles}.


Hatua ya kwanza muhimu ni kuficha msingi asilia wa imani yetu kwa kuutenganisha na \emcap{Kanuni za Kimsingi}.


\egw{\textbf{As a people, we are to \underline{stand firm on the platform of eternal truth} that has withstood test and trial. We are to \underline{hold to the sure pillars of our faith}. \underline{The principles of truth} that God has revealed to us \underline{are our only true foundation}. They have made us what we are. The lapse of time has not lessened their value. \underline{It is the constant effort of the enemy to remove these truths from their setting}, and to put in their place \underline{spurious theories}. He \underline{will bring in} everything that he possibly can to carry out his deceptive designs.}}[SpTB02 51.2; 1904][https://egwwritings.org/read?panels=p417.261]


\egw{\textbf{Kama watu, tunapaswa \underline{kusimama kidete kwenye jukwaa la ukweli wa milele} ambao umestahimili mtihani na majaribio. Tunapaswa \underline{kushikilia nguzo za uhakika za imani yetu}. \underline{Kanuni za ukweli} kwamba Mungu ametufunulia \underline{ndio msingi wetu wa kweli}. Zimetufanya tulivyo. Upungufu wa wakati haujapunguza thamani yao. \underline{Ni juhudi za mara kwa mara za adui kuondoa ukweli huu kutoka kwa mpangilio yao}, na kuweka mahali pao \underline{nadharia potofu}. Yeye \underline{ataleta} kila kitu awezacho ili kutekeleza mipango yake ya udanganyifu.}}[SpTB02 51.2; 1904][https://egwwritings.org/read?panels=p417.261]


\egw{\textbf{Messages of every order and kind have been urged upon Seventh-day Adventists, to take the place of the truth which, \underline{point by point}, has been sought out by prayerful study, and testified to by the miracle-working power of the Lord}. \textbf{But \underline{the way-marks} \underline{which have made us what we are}, \underline{are to be preserved}, and they \underline{will be preserved}, as God has signified through His word and the testimony of His Spirit}. \textbf{He calls upon us to \underline{hold firmly}, with the grip of faith, to \underline{the fundamental principles} that are \underline{based upon unquestionable authority}}.}[SpTB02 59.1; 1904][https://egwwritings.org/read?panels=p417.299]


\egw{\textbf{Ujumbe wa kila utaratibu na aina umehimizwa kwa Waadventista Wasabato, ili kuchukua nafasi ya ukweli ambao, \underline{pointi baada ya pointi}, umetafutwa kwa maombi kujifunza, na kushuhudia kwa uwezo wa kutenda miujiza wa Bwana}. \textbf{Lakini \underline{alama za njia} \underline{ambazo zimetufanya tulivyo}, \underline{zinapaswa kuhifadhiwa}, na \underline{zitahifadhiwa}, kama Mungu ameonyesha kupitia neno lake na ushuhuda wa Roho wake}. \textbf{Anatuita \underline{kushika kwa uthabiti}, kwa mshiko wa imani, \underline{kanuni za msingi} ambazo \underline{msingi wake ni mamlaka isiyo na shaka}}.}[SpTB02 59.1; 1904][https://egwwritings.org/read?panels=p417.299]


The \emcap{Fundamental Principles} were the truths God revealed to the pioneers after the passing of time in 1844. We have seen the testimonies of our pioneers, including Ellen White, regarding the first point of the \emcap{Fundamental Principles}. All of them were in harmony regarding these particular points of our faith. In 1863, Seventh-day Adventists organized themselves into a church, as an organized body. Since then, many were misrepresenting the position of the Seventh-day Adventist Church and the pioneers found it necessary to meet inquiries, \others{and sometimes to correct false statements circulated against} the church’s beliefs and practices. Consequently, in 1872, the pioneers issued the document called “\textit{A Declaration of the Fundamental Principles, Taught and Practiced by the Seventh-Day Adventists}”\footnote{“A Declaration of the Fundamental Principles, Taught and Practiced by the Seventh-Day Adventists (1872) : MVT : Free Download, Borrow, and Streaming : Internet Archive.” Internet Archive, 2025, \href{https://archive.org/details/ADeclarationOfTheFundamentalPrinciplesTaughtAndPracticedByThe}{archive.org/details/ADeclarationOfTheFundamentalPrinciplesTaughtAndPracticedByThe}. Accessed 3 Feb. 2025.}. This declaration presented the public with \others{a brief statement of what is, and has been, with great unanimity, held by}[The preface of the Fundamental Principles in 1872.] Seventh-day Adventists.


Kanuni za Msingi zilikuwa kweli ambazo Mungu alifunua kwa waanzilishi baada ya kupita kwa wakati mnamo 1844. Tumeona ushuhuda wa waanzilishi wetu, pamoja na Ellen White, kuhusu hoja ya kwanza ya Kanuni za Msingi. Wote walikuwa katika maelewano kuhusu mambo haya mahususi ya imani yetu. Mnamo 1863, Waadventista Wasabato walipanga wao wenyewe wakawa kanisa, kama mwili wa madhehebu. Tangu wakati huo, wengi walikuwa wakiwakilisha vibaya nafasi ya Kanisa la Waadventista Wasabato na waanzilishi waliona ni muhimu kuyakabili maswali, \others{na nyakati nyingine kusahihisha taarifa za uwongo zinazosambazwa dhidi ya} imani za kanisa na mazoea. Kwa hivyo, mnamo 1872, waanzilishi walitoa hati inayoitwa “\textit{A Declaration of the Fundamental Principles, Taught and Practiced by the Seventh-Day Adventists}”\footnote{“A Declaration of the Fundamental Principles, Taught and Practiced by the Seventh-Day Adventists (1872) : MVT : Free Download, Borrow, and Streaming : Internet Archive.” Internet Archive, 2025, \href{https://archive.org/details/ADeclarationOfTheFundamentalPrinciplesTaughtAndPracticedByThe}{archive.org/details/ADeclarationOfTheFundamentalPrinciplesTaughtAndPracticedByThe}. Accessed 3 Feb. 2025.}. Tamko hili liliwasilisha umma na \others{taarifa fupi ya nini, na imekuwa, kwa umoja mkuu, ikishikiliwa na}[The preface of the Fundamental Principles in 1872.] Waadventista Wasabato.


In the chapter “\hyperref[chap:authority]{The Authority of the Fundamental Principles}”, we discussed how pro-Trinitarian scholars have been compromising the authority of the \emcap{Fundamental Principles}, denying their true value in our Adventist history.


Katika sura ya “\hyperref[chap:authority]{Mamlaka ya Kanuni za Msingi}”, tulijadili jinsi gani wataalamu wanaounga mkono Utatu wamekuwa wakihatarisha mamlaka ya Kanuni za Msingi, kukana thamani yao ya kweli katika historia yetu ya Waadventista.


Pro-trinitarian scholars argue that this declaration was not what it claims to be—a declaration of the \emcap{fundamental principles}, taught and practiced by the Seventh-day Adventists. This declaration was a summary of the principal features of Adventist’s faith, and no point is really as problematic or objectionable as the first point, dealing with the \emcap{personality of God} and where His presence is. But the evidence in favor of the \emcap{Fundamental Principles}, especially to the first point, is overwhelming.


Wataalamu wanaounga mkono utatu wanabisha kwamba tamko hili halikuwa kama linavyodai kuwa—tamko la Kanuni za Msingi, zinazofundishwa na kutekelezwa na Waadventista Wasabato. Tamko hili lilikuwa muhtasari wa sifa kuu za imani ya Waadventista, na kwa kweli hakuna pointi yenye tatizo au ya kuchukiza kama pointi ya kwanza, linalohusu Umbile la Mungu na uwepo wake ulipo. Lakini ushahidi unaounga mkono Kanuni za Msingi, hasa kwa pointi ya kwanza, ni mingi mno.


All of these claims are easily refuted by the fact that the \emcap{Fundamental Principles} have been regularly issued and reprinted over the course of the entire life of Sister White, until 1914. If they were mere private opinions of a few individuals, as claimed by scholars\footnote{Ministry Magazine “Our Declaration of Fundamental Beliefs”: January 1958, Roy Anderson, J. Arthur Buckwalter, Louise Kleuser, Earl Cleveland and Walter Schubert}, would they have been consistently reprinted over the course of 42 years\footnote{For a detailed list of publications throughout these years, see the Appendix.}, publicly claiming to represent the synopsis of Seventh-day Adventist faith? If they had been issued only once, we could deem it a conspiracy by some individuals to purposely misrepresent Seventh-day Adventist faith. On the contrary, the \emcap{Fundamental Principles} were regularly reprinted, and they truly represented the official Seventh-day Adventist faith and practice.


Madai haya yote yanakanushwa kwa urahisi na ukweli kwamba Kanuni za Msingi zimekuwa zilizotolewa mara kwa mara na kuchapishwa tena katika kipindi chote cha maisha ya Dada White, hadi 1914. Kama yalikuwa ni maoni ya faragha tu ya watu wachache, kama wanavyodai wasomi\footnote{Ministry Magazine “Our Declaration of Fundamental Beliefs”: January 1958, Roy Anderson, J. Arthur Buckwalter, Louise Kleuser, Earl Cleveland and Walter Schubert}, je! yamechapishwa tena mara kwa mara katika kipindi cha miaka 42\footnote{For a detailed list of publications throughout these years, see the Appendix.}, yakidai hadharani kuwakilisha muhtasari wa imani ya Waadventista Wasabato? Ikiwa zingetolewa mara moja tu, tungeweza kuona kuwa ni njama ya baadhi ya watu kuwawakilisha vibaya kimakusudi Imani ya Waadventista wa Sabato. Kinyume chake, Kanuni za Msingi zilichapishwa tena mara kwa mara, na kwa kweli iliwakilisha imani na utendaji rasmi wa Waadventista Wasabato.


Another argument is that Sister White approved the \emcap{Fundamental Principles} in her writings by explicitly referring to them, and also by teaching the same truths taught in the \emcap{Fundamental Principles}. The works of our pioneers are also in harmony with the statements in this Declaration of the \emcap{Fundamental Principles}. Considering all of these facts, it is inevitable that this declaration was truthful in its claims. This document was indeed a declaration of the \emcap{fundamental principles}, taught and practiced by the Seventh-day Adventist Church, representing a public \others{synopsis of our faith}, \others{a brief statement of what is, and has been, with great unanimity, held by} Seventh-day Adventists.\footnote{The preface of the Fundamental Principles in 1872.} As such, it accurately represents the Seventh-day Adventist belief and practice, and represents the foundation of Seventh-day Adventist faith in the time of Ellen White.


Hoja nyingine ni kwamba Dada White aliidhinisha Kanuni za Msingi katika maandishi yake kwa kuzirejelea kwa uwazi, na pia kwa kufundisha kweli zile zile zinazofundishwa katika Kanuni za Msingi. Kazi za waanzilishi wetu pia zinapatana na kauli katika Tamko hili la Kanuni za Msingi. Kwa kuzingatia ukweli huu wote, ni bila kuepukika kwamba tamko hili lilikuwa la ukweli katika madai yake. Hati hii kwa kweli ilikuwa tamko la Kanuni za Msingi, zinazofundishwa na kutekelezwa na Kanisa la Waadventista Wasabato, likiwakilisha \others{muhtasari wa imani yetu}, \others{taarifa fupi ya kile ambacho ni hadharani, na kimekuwa, kwa umoja mkuu, kikishikiliwa na} Waadventista Wasabato.\footnote{The preface of the Fundamental Principles in 1872.} Kwa hivyo, inawakilisha kwa usahihi imani na utendaji wa Waadventista wa Sabato, na inawakilisha msingi wa imani ya Waadventista Wasabato wakati wa Ellen White.


Today, in defense of the Trinity doctrine, Adventist historians boldly claim that when our pioneers were studying Adventist truths such as the sanctuary, investigative judgment, the Sabbath and other doctrines, they \others{did not study the subject of the doctrine of God}. These Adventist historians falsely claim that the doctrine of God \others{was not the question that they dealt at that time}[Denis Kaiser. “From Antitrinitarianism to Trinitarianism: The Adventist story” and Panelist. The God We Worship: A Godhead Symposium. Central California Conference, Dinuba, CA. March 23-24, 2018.]. Following this false claim, they present historical data on how Adventist doctrine gradually moved toward Trinitarian understanding. The truth is, there are some instances early on\footnote{The earliest mention of the Trinity doctrine, in a positive sense, was when M.C. Wilcox reprinted a non-Adventist article by Samuel Spear in Signs of the Times, December 7th, 1891 and December 14th, 1891} when the Trinity doctrine is mentioned in a positive light in our literature. But when you consider the fact that the Adventist church did have a positive position on the subject of the doctrine of God, as it was expressed in the \emcap{Fundamental Principles}, these instances cannot be interpreted as progressiveness in understanding, but rather an intrusion of the Trinity doctrine into the Seventh-day Adventist Church.


Leo, katika kutetea fundisho la Utatu, wanahistoria Waadventista wanadai kwa ujasiri kwamba wakati waanzilishi walikuwa wakisoma kweli za Waadventista kama vile patakatifu, hukumu ya uchunguzi, Sabato na mafundisho mengine, \others{hawakujifunza somo la mafundisho ya Mungu}. Hawa Wanahistoria Waadventista hudai kwa uwongo kwamba fundisho la Mungu \others{halikuwa swali ambalo wao walishughulikiwa wakati huo}[Denis Kaiser. “From Antitrinitarianism to Trinitarianism: The Adventist story” and Panelist. The God We Worship: A Godhead Symposium. Central California Conference, Dinuba, CA. March 23-24, 2018.]. Kufuatia dai hili la uwongo, wanawasilisha data ya kihistoria juu ya jinsi fundisho la Waadventista lilihamia hatua kwa hatua kuelekea uelewaji wa Utatu. Ukweli ni kwamba, kuna baadhi ya matukio ya hapo awali\footnote{The earliest mention of the Trinity doctrine, in a positive sense, was when M.C. Wilcox reprinted a non-Adventist article by Samuel Spear in Signs of the Times, December 7th, 1891 and December 14th, 1891} wakati fundisho la Utatu linatajwa kwa njia chanya katika fasihi zetu. Lakini unapozingatia ukweli kwamba kanisa la Waadventista lilikuwa na msimamo wake chanya juu ya somo la fundisho la Mungu, kama lilivyoelezwa katika Kanuni za Msingi, haya matukio hayawezi kutafsiriwa kama maendeleo katika kuelewa, lakini badala yake ni kuingia kwa fundisho la Utatu ndani ya Kanisa la Waadventista Wasabato.


It is easy to refute the claim that Adventist pioneers did not understand the doctrine of God. If they did not understand it, they would have failed to proclaim the first angel’s message. We discussed this point in detail in the chapter “\hyperref[chap:remembering-the-beginning]{Remembering the beginning}”. The Seventh-day Adventist movement was not a failure, but a God-led, prophetic movement.


Ni rahisi kukanusha madai kwamba waanzilishi wa Kiadventista hawakuelewa fundisho kuhusu Mungu. Ikiwa hawangeielewa, wangekosa kutangaza ujumbe wa malaika wa kwanza. Tulijadili jambo hili kwa undani katika sura ya “\hyperref[chap:remembering-the-beginning]{Kukumbuka mwanzo}”. Harakati ya Waadventista Wasabato haikuwa ya kushindwa, bali harakati ya kinabii inayoongozwa na Mungu.


\subsection*{Step 2: Ignore the warnings of building a new foundation}


\subsection*{Hatua ya 2: Puuza maonyo ya kujenga msingi mpya}


When the \emcap{Fundamental Principles} are removed from the equation, many of Ellen White’s warnings fail to shine in their true light and their true meaning does not resonate with the reader.


Wakati Kanuni za Msingi zinapoondolewa kwenye mlingano, maonyo mengi ya Ellen White yanashindwa kung'aa katika nuru yao ya kweli na maana yake ya kweli haipatani kwa msomaji.


We have cited many quotations where Sister White warned the church not to step off the \emcap{Fundamental Principles}. We dealt with them in the chapter “\hyperref[chap:apostasy]{The great apostasy is soon to be realized}”, but we will mention one of the most prominent quotations again.


Tumetaja nukuu nyingi ambapo Dada White alionya kanisa lisiondoke kwa Kanuni za Msingi. Tuliyashughulikia katika sura “\hyperref[chap:apostasy]{Uasi mkuu utakuja kitambuliwa hivi karibuni}”, lakini tutataja moja ya nukuu maarufu tena.


\egw{\textbf{The enemy of souls has sought to bring in the supposition that a great reformation was to take place among Seventh-day Adventists, and that this reformation would \underline{consist in giving up the doctrines which stand as the pillars of our faith} and engaging in a process of reorganization}. Were this reformation to take place, what would result? \textbf{The principles of truth that God in His wisdom has given to the remnant church would be discarded. Our religion would be changed. \underline{The fundamental principles that have sustained the work for the last fifty years would be accounted as error}}. \textbf{A new organization would be established. Books of a new order would be written. A system of intellectual philosophy would be introduced}...}[Lt242-1903.13; 1903][https://egwwritings.org/read?panels=p7767.20]


\egw{\textbf{Adui wa roho ametaka kuleta dhana kwamba matengenezo makubwa yangetukia kati ya Waadventista Wasabato, na kwamba matengenezo haya yangefanyika \underline{ingejumuisha kuacha mafundisho ambayo yanasimama kama nguzo za imani yetu} na kuhusika katika mchakato wa kujipanga upya}. Je, matengenezo haya yangefanyika, matokeo yangekuwa nini? \textbf{Kanuni za ukweli ambazo Mungu katika hekima yake ametoa kwa kanisa la masalio zingetupwa. Dini yetu ingebadilishwa. \underline{Kanuni za msingi ambazo zimeendeleza kazi kwa miaka hamsini iliyopita ingehesabiwa kama makosa}}. \textbf{Shirika jipya lingeanzishwa. Vitabu vya namna mpya vingeandikwa. Mfumo wa falsafa ya kiakili ingeanzishwa}...}[Lt242-1903.13; 1903][https://egwwritings.org/read?panels=p7767.20]


\egwnogap{Who has authority to begin such a movement? \textbf{We have our Bibles. We have our experience, attested to by the miraculous working of the Holy Spirit}. \textbf{We have a truth that admits of no compromise.} \textbf{\underline{Shall we not repudiate everything that is not in harmony with this truth}?}}[Lt242-1903.14; 1903][https://egwwritings.org/read?panels=p7767.21]


\egwnogap{Nani mwenye mamlaka ya kuanzisha harakati hiyo? \textbf{Tuna Biblia zetu. Tuna uzoefu wetu, unaothibitishwa na utendaji wa kimiujiza wa Roho Mtakatifu}. \textbf{Tuna ukweli ambao hatupaswi kushusha makali yake.} \textbf{\underline{Je, hatutakataa kila kitu ambacho hakipatani na ukweli huu}?}}[Lt242-1903.14; 1903][https://egwwritings.org/read?panels=p7767.21]


\subsection*{Step 3: Deny that the personality of God was the pillar of our faith and a part of the foundation of our faith}


\subsection*{Hatua ya 3: Kataa kwamba ubinafsi wa Mungu ulikuwa nguzo ya imani yetu na sehemu ya msingi wa imani yetu}


There is one Ellen White statement that apparently supports the claim that the \emcap{personality of God} was not a pillar of our faith. Another expression for “\textit{pillars of our faith}” is “\textit{landmarks}”. In the following quotations, Sister White lists several landmarks: the cleansing of the sanctuary, the three angels’ messages, the temple of God, the Sabbath and the non-immortality of the wicked.


Kuna kauli moja ya Ellen White ambayo inaonekana inaunga mkono madai kwamba \emcap{ubinafsi wa Mungu} haikuwa nguzo ya imani yetu. Usemi mwingine wa “\textit{nguzo za imani yetu}” ni “\textit{alama}”. Katika nukuu zifuatazo, Dada White anaorodhesha alama kadhaa: utakaso wa patakatifu, jumbe za malaika watatu, hekalu la Mungu, Sabato na kutokufa kwa waovu.


\egw{The passing of the time in 1844 was a period of great events, opening to our astonished eyes \textbf{the cleansing of the sanctuary transpiring in heaven}, and having decided relation to God’s people upon the earth, [also] \textbf{the first and second angels’ messages and the third}, unfurling the banner on which was inscribed, ‘The commandments of God and the faith of Jesus.’ [Revelation 14:12.] One of the landmarks under this message was \textbf{the temple of God}, seen by His truth-loving people in heaven, and the ark containing the law of God. The light of \textbf{the Sabbath} of the fourth commandment flashed its strong rays in the pathway of the transgressors of God’s law. The \textbf{non-immortality of the wicked} is an old landmark. \textbf{I can call to mind nothing more that can come under the head of the old landmarks}. All this cry about changing the old landmarks is all imaginary.}[Ms13-1889.9; 1889][https://egwwritings.org/read?panels=p4179.14]


\egw{Kupita kwa wakati huo katika 1844 kilikuwa kipindi cha matukio makubwa, kilichofungua kwa mshangao macho yetu \textbf{utakaso wa patakatifu upitapo mbinguni}, na kuwa na uhusiano wa dhati kwa watu wa Mungu duniani, [pia] \textbf{ujumbe wa malaika wa kwanza na wa pili na wa tatu}, akiifunua ile bendera ambayo juu yake ilikuwa imeandikwa, ‘Amri za Mungu na imani ya Yesu.’ [Ufunuo 14:12.] Moja ya alama muhimu chini ya ujumbe huu ilikuwa \textbf{hekalu la Mungu}, lililoonekana mbinguni na watu wake wapendao ukweli, na sanduku lenye sheria ya Mungu. Nuru ya \textbf{Sabato} ya amri ya nne ilimulika miale yake mikali katika njia ya waasi wa sheria za Mungu. \textbf{Kufariki na kutokuwa na uhai baada ya kifo kwa waovu} ni alama ya zamani ya kihistoria. \textbf{Siwezi kukumbuka chochote zaidi ambacho kinaweza kuja chini ya mada ya alama za zamani}. Kilio hiki chote cha kubadilisha alama za zamani ni cha kudhaniwa tu.}[Ms13-1889.9; 1889][https://egwwritings.org/read?panels=p4179.14]


At the end of this list of landmarks, or pillars of our faith, she states that she can recall nothing else that would fall under the category of the old landmarks. For many, this quotation serves as proof that the \emcap{personality of God} was neither an old landmark nor a pillar. It is true that in this quotation, Sister White did not explicitly mention the \emcap{personality of God}, but it would be implicitly included under the first angel’s message, as well as being an underlying doctrine of the Sanctuary message. Furthermore, there are other quotations from Sister White that explicitly include the \emcap{personality of God} as an old landmark or pillar of our faith.


Mwishoni mwa orodha hii ya alama muhimu, au nguzo za imani yetu, anasema kwamba anaweza kukumbuka hakuna kitu zaidi ambacho kinaweza kuja chini ya mada ya alama za zamani. Kwa wengi, nukuu hii ni uthibitisho kwamba \emcap{ubinafsi wa Mungu} haukuwa alama ya zamani wala nguzo. Ni kweli kwamba katika nukuu hili, Dada White hakutaja kwa uwazi \emcap{ubinafsi wa Mungu}, lakini ingekuwa imejumuishwa kwa uwazi chini ya ujumbe wa malaika wa kwanza, na pia kuwa mafundisho ya msingi ya ujumbe wa patakatifu. Zaidi ya hayo, kuna nukuu nyingine kutoka kwa Dada White ambayo yanajumuisha kwa uwazi \emcap{ubinafsi wa Mungu} kama alama ya zamani, au nguzo ya imani yetu.


\egw{Those who seek to remove the \textbf{old landmarks} are not holding fast; they \textbf{are not remembering how they have received and heard}. Those who try to \textbf{\underline{bring in} theories that would remove \underline{the pillars of our faith}} \textbf{concerning the sanctuary}, \textbf{\underline{or concerning the personality of God or of Christ}, are working as blind men}. They are seeking to bring in uncertainties and to set the people of God \textbf{adrift}, without an anchor.}[Ms62-1905.14; 1905][https://egwwritings.org/read?panels=p10026.20]


\egw{Wale wanaotafuta kuondoa \textbf{alama za zamani} hawashikilii; \textbf{hawakumbuki jinsi walivyopokea na kusikia}. Wale ambao wanajaribu \textbf{\underline{kuleta} nadharia ambayo ingeondoa \underline{nguzo za imani yetu}} \textbf{kuhusu patakatifu}, \textbf{\underline{au kwa habari ya ubinafsi wa Mungu au wa Kristo}, wanafanya kazi kama vipofu}. Wanatafuta kuingiza mashaka na kuwafanya watu wa Mungu \textbf{wapeperushwe bila nanga}.}[Ms62-1905.14; 1905][https://egwwritings.org/read?panels=p10026.20]


Sister White also teaches us that the pillars of our faith constitute the foundation of our faith.


Dada White pia anatufundisha kwamba nguzo za imani yetu ni msingi wa imani yetu.


\egw{\textbf{What influence is it that would lead men at this stage of our history to work in an underhanded, powerful way \underline{to tear down the foundation of our faith},—the foundation that was laid at the beginning of our work by prayerful study of the word and by revelation? Upon \underline{this foundation} we have been building for \underline{the past fifty years}. Do you wonder that when I see the beginning of a work that would \underline{remove some of the pillars of our faith}, I have something to say? I must obey the command, ‘Meet it!’}}[SpTB02 58.1; 1904][https://egwwritings.org/read?panels=p417.295]


\egw{\textbf{Ni ushawishi gani ambao unaweza kusababisha watu katika hatua hii ya historia yetu kufanya kazi katika njia ya kichinichini, yenye nguvu ya \underline{kubomoa msingi wa imani yetu},—msingi ambayo iliwekwa mwanzoni mwa kazi yetu kwa kujifunza kwa maombi neno na kwa ufunuo? Juu ya \underline{msingi huu} tumekuwa tukijenga kwa \underline{miaka hamsini iliyopita}. Unashangaa kwamba ninapoona mwanzo wa kazi ambayo ingeondoa \underline{baadhi ya nguzo za imani yetu}, nina la kusema? Ni lazima nitii amri, ‘Kutana nayo!’}}[SpTB02 58.1; 1904][https://egwwritings.org/read?panels=p417.295]


Removing some of the pillars of our faith means tearing down the foundation of our faith. Elsewhere, Sister White said that tearing down or undermining the foundation of our faith is done by indoctrination of the sentiments regarding the \emcap{personality of God}.


Kuondoa baadhi ya nguzo za imani yetu kunamaanisha kubomoa msingi wa imani yetu. Mahali pengine, Dada White alisema kwamba kubomoa au kudhoofisha msingi wa imani yetu inayofanywa kwa kufundishwa maoni kuhusu \emcap{ubinafsi wa Mungu}.


\egw{The college was taken out of Battle Creek; yet students are still called there, and there they \textbf{become indoctrinated with the very sentiments regarding the personality of God and Christ that would undermine the foundation of our faith}.}[Lt72-1906.5; 1906][https://egwwritings.org/read?panels=p10013.11]


\egw{Chuo kilitolewa Battle Creek; bado wanafunzi wanaitwa huko, na huko \textbf{wanafundishwa hisia zenyewe kuhusu ubinafsi wa Mungu na Kristo, mafundisho haya yanadhoofisha msingi wa imani yetu}.}[Lt72-1906.5; 1906][https://egwwritings.org/read?panels=p10013.11]


In light of these quotations we see positive testimony that the \emcap{personality of God} was part of the foundation of our faith. Furthermore, in chapter 10 of the special testimonies, entitled “\textit{The foundation of our faith}”, Sister White mentioned “\textit{Fundamental Principles}” using the synonyms “\textit{pillars of our faith}”, “\textit{waymarks}”, and “\textit{landmarks}”, when addressing the foundation of our faith.


Kwa kuzingatia nukuu hizi tunaona ushuhuda chanya ambao Umbile la Mungu ulikuwa sehemu ya msingi wa imani yetu. Zaidi ya hayo, katika sura ya kumi ya shuhuda maalum, yenye kichwa “\textit{Msingi wa imani yetu}”, Dada White alitaja “\textit{Kanuni za Msingi}” kwa kutumia visawe “\textit{nguzo za imani yetu}”, “\textit{viashiria njia}”, na “\textit{alama}”, wakati wa kushughulikia msingi wa imani yetu.


\subsection*{Step 4: Alter the meaning of the term “the personality of God”}


\subsection*{Hatua ya 4: Badilisha maana ya neno “Umbile la Mungu”}


The term ‘\textit{personality}’ has two different applications and the most common definition in everyday use is in the area of psychology. ‘\textit{Personality}’ is defined as “\textit{the characteristic sets of behaviors, cognitions, and emotional patterns that evolve from biological and environmental factors}”\footnote{Wikipedia Contributors. “Personality.” Wikipedia, Wikimedia Foundation, 19 Apr. 2019, \href{https://en.wikipedia.org/wiki/Personality}{en.wikipedia.org/wiki/Personality}.}. It is of utmost importance to recognize that when we are dealing with the pillar of our faith—“\textit{the personality of God}”—we are not in the realms of psychology. The accurate application of the word ‘\textit{personality}’ within the doctrine on the \emcap{personality of God} is found in the Merriam-Webster Dictionary: “\textit{the quality or state of being a person}”\footnote{\href{https://www.merriam-webster.com/dictionary/personality}{Merriam-Webster Dictionary} - ‘\textit{personality}’}. According to the Merriam-Webster Dictionary, this definition has been in use since the 15th century\footnote{See “\href{https://www.merriam-webster.com/dictionary/personality\#word-history}{First known use}” of the word ‘personality’ in Merriam Webster Dictionary}. In the 1828 edition of the Merriam Webster Dictionary we read definition of the word ‘\textit{personality}’ as: “\textit{that which constitutes an individual a distinct person}”\footnote{\href{https://archive.org/details/americandictiona02websrich/page/272/mode/2up}{Merriam-Webster Dictionary, 1828 edition} - ‘\textit{personality}’} \footnote{\href{https://archive.org/details/websterscomplete00webs/page/974/mode/2up}{The 1886 edition of Merriam-Webster Dictionary} defines the word ‘\textit{personality}’ as: “\textit{that which constitutes, or pertains to, a person}”}. Both of the definitions are found in The Encyclopaedic Dictionary, by Hunter Robert\footnote{\href{https://babel.hathitrust.org/cgi/pt?id=mdp.39015050663213&view=1up&seq=780}{Hunter Robert, The Encyclopaedic Dictionary} - ‘\textit{personality}’}—dictionary owned by Ellen White. The use of these definitions can be seen from the articles written on the \emcap{personality of God}.


Neno ‘\textit{personality}’ lina matumizi mawili tofauti na fasili ya kawaida zaidi katika matumizi ya kila siku ni katika eneo la saikolojia. ‘\textit{Personality}’ hufafanuliwa kuwa “\textit{seti za sifa za tabia, utambuzi, na mifumo ya kihisia ambayo hubadilika kutoka kwa kibayolojia na mambo ya mazingira}”\footnote{Wikipedia Contributors. “Personality.” Wikipedia, Wikimedia Foundation, 19 Apr. 2019, \href{https://en.wikipedia.org/wiki/Personality}{en.wikipedia.org/wiki/Personality}.}. Ni muhimu sana kutambua kwamba tunaposhughulika na nguzo ya imani yetu—“\textit{Umbile la Mungu}”—hatuko katika nyanja za saikolojia. Ufafanuzi sahihi wa neno ‘\textit{personality}’ ndani ya fundisho la \emcap{Umbile la Mungu} unapatikana katika Kamusi ya Merriam-Webster: “\textit{Ubora au hali ya kuwa Nafsi}”\footnote{\href{https://www.merriam-webster.com/dictionary/personality}{Merriam-Webster Dictionary} - ‘\textit{personality}’}. Kulingana na Kamusi ya Merriam-Webster, ufafanuzi huu umekuwa ukitumika tangu karne ya 15\footnote{Angalia “\href{https://www.merriam-webster.com/dictionary/personality\#word-history}{First known use}” ya neno ‘personality’ katika Kamusi ya Merriam Webster}. Katika toleo la 1828 la Kamusi ya Merriam Webster tunasoma ufafanuzi wa neno ‘\textit{personality}’ kama: “\textit{kile ambacho kinajumuisha Nafsi binafsi kuwa Nafsi tofauti}”\footnote{\href{https://archive.org/details/americandictiona02websrich/page/272/mode/2up}{Merriam-Webster Dictionary, 1828 edition} - ‘\textit{personality}’} \footnote{\href{https://archive.org/details/websterscomplete00webs/page/974/mode/2up}{Toleo la 1886 la Kamusi ya Merriam-Webster} linafafanua neno ‘\textit{personality}’ kama: “\textit{kile ambacho kinajumuisha, au kinahusu, Nafsi}”}. Fasili zote mbili zinapatikana katika The Encyclopaedic Dictionary, ya Hunter Robert\footnote{\href{https://babel.hathitrust.org/cgi/pt?id=mdp.39015050663213&view=1up&seq=780}{Hunter Robert, The Encyclopaedic Dictionary} - ‘\textit{personality}’}—kamusi inayomilikiwa na Ellen White. Matumizi ya fafanuzi hizi yanaweza kuonekana kutoka kwa makala zilizoandikwa kuhusu \emcap{Umbile la Mungu}.


In 1903, when Sister White wrote to Dr. Kellogg, \egwinline{I have \textbf{ever }had the same testimony to bear which I now bear \textbf{regarding the personality of God}}[Lt253-1903.9; 1903][https://egwwritings.org/read?panels=p9980.15], she recalled her vision when she saw the Father and the Son.


Katika mwaka wa 1903, Dada White alipomwandikia Dk. Kellogg, \egwinline{Nimekuwa \textbf{daima} na ushuhuda sawa ambao sasa ninao \textbf{kuhusu Umbile la Mungu}}[Lt253-1903.9; 1903][https://egwwritings.org/read?panels=p9980.15], alikumbuka maono yake wakati alimwona Baba na Mwana.


\egw{‘I have often seen the lovely Jesus, that\textbf{ He is a person}.\textbf{ I asked Him if His Father was a person, }and \textbf{had \underline{a form} like Himself}. Said Jesus, ‘\textbf{I am the express image of My Father’s person!}’ [Hebrews 1:3.]}[Lt253-1903.12; 1903][https://egwwritings.org/read?panels=p9980.18]


\egw{‘Mara nyingi nimemwona Yesu mpendwa, kwamba\textbf{ Yeye ni Nafsi}.\textbf{ Nilimuuliza kama Baba Yake alikuwa Nafsi, }na \textbf{alikuwa na \underline{umbo} kama Yeye Mwenyewe}. Yesu alisema, ‘\textbf{Mimi ni chapa kamili ya Umbile Wake!}’ [Waebrania 1:3.]}[Lt253-1903.12; 1903][https://egwwritings.org/read?panels=p9980.18]


The quality or state that Sister White defines God to be a person is to have \textit{a form}—\textit{a physical appearance}. Dr. Kellogg follows the same application of the word \textit{‘personality’}, although through speculation.


Ubora au hali ambayo Dada White anafafanua Mungu kuwa Nafsi ni kuwa ana \textit{umbo}—\textit{mwonekano wa kimwili}. Dk. Kellogg anafuata fasili sawa ya neno \textit{‘personality’}, ingawa kwa uvumi.


\others{The fact that God is so great that we cannot form a clear mental picture of \textbf{his physical appearance} need not lessen in our minds the reality of \textbf{His personality}...}[John H. Kellogg, The Living Temple, p. 31][https://archive.org/details/J.H.Kellogg.TheLivingTemple1903/page/n31/mode/2up]


\others{Ukweli kwamba Mungu ni mkuu sana hivi kwamba hatuwezi kutokeza kiakili picha iliyo wazi ya \textbf{mwonekano wake wa kimwili} hauhitaji kupunguza katika akili zetu ukweli wa \textbf{Umbile Lake}...}[John H. Kellogg, The Living Temple, p. 31][https://archive.org/details/J.H.Kellogg.TheLivingTemple1903/page/n31/mode/2up]


As we have previously seen, our Adventist pioneers also pinpointed the physical appearance as a quality that makes God a person. James White wrote, \others{Those who deny \textbf{the personality of God}, say that ‘image’ here does not mean \textbf{physical form}, but moral image...}[James S. White, PERGO 1.1; 1861][https://egwwritings.org/read?panels=p1471.3]. J. B. Frisbie wrote, \others{Some seem to suppose it argues against \textbf{the personality of God}, because he is a Spirit, and say that he is without \textbf{body, or parts}...}[\href{https://documents.adventistarchives.org/Periodicals/RH/RH18540307-V05-07.pdf}{Adventist Review and Sabbath Herald, March 7, 1854}, J. B. Frisbie, “The Seventh-Day Sabbath Not Abolished”, p. 50]


Kama tulivyoona hapo awali, waanzilishi wetu Waadventista pia walibainisha mwonekano wa kimwili kama sifa inayomfanya Mungu kuwa Nafsi. James White aliandika, \others{Wale wanaokataa \textbf{Umbile la Mungu}, wanasema kwamba ‘mfano’ hapa haimaanishi \textbf{umbo la kimwili}, bali picha ya maadili...}[James S. White, PERGO 1.1; 1861][https://egwwritings.org/read?panels=p1471.3]. J. B. Frisbie aliandika, \others{Wengine wanaonekana kudhani kuwa inapingana na \textbf{Umbile la Mungu}, kwa sababu yeye ni Roho, na husema kwamba yeye hana \textbf{mwili, wala viungo}...}[\href{https://documents.adventistarchives.org/Periodicals/RH/RH18540307-V05-07.pdf}{Adventist Review and Sabbath Herald, March 7, 1854}, J. B. Frisbie, “The Seventh-Day Sabbath Not Abolished”, p. 50]


In light of the facts, we recognize the application of the word ‘\textit{personality}’. When the subject on the \emcap{personality of God} is presented in its connection to the Trinity doctrine, there is often a tendency to alter the meaning of the word ‘\textit{personality}’. It is also important to mention that the subject on the \emcap{personality of God} deals with the personality of the Father. This is clearly seen from the presented data.


Kwa kuzingatia ukweli, tunatambua matumizi ya neno ‘\textit{personality}’. Wakati mada juu ya \emcap{Umbile la Mungu} inapowasilishwa katika uhusiano wake na fundisho la Utatu, mara nyingi kuna tabia ya kubadilisha maana ya neno ‘\textit{personality}’. Pia ni muhimu kutaja kwamba somo la \emcap{Umbile la Mungu} linahusu Umbile la Baba. Hii inaonekana wazi kutoka kwa data iliyotolewa.


\subsection*{Step 5: In examining the Kellogg crisis, shifting the main focus from the personality of God to pantheism}


\subsection*{Hatua ya 5: Katika kuchunguza mgogoro wa Kellogg, kuhamisha lengo kuu kutoka kwa Umbile la Mungu hadi pantheism}


The data on the Kellogg crisis, in connection with the Trinity doctrine, is overwhelming if the \emcap{personality of God} is accounted for in the equation. The only way to not connect the dots is to ignore the \emcap{personality of God} and shift focus to pantheism exclusively. We do not deny the pantheistic nature of Kellogg's controversy. We believe that the pantheistic nature of Kellogg's controversy cannot be rightly understood if it is not examined in the true light of the \emcap{personality of God}. But, unfortunately, in examination of the Kellogg crisis, the attention that pantheism receives supersedes the examination of the truth on the \emcap{personality of God}.


Data juu ya mgogoro wa Kellogg, kuhusiana na fundisho la Utatu, ni nyingi sana ikiwa \emcap{Umbile la Mungu} linahesabiwa katika mlinganyo huo. Njia pekee ya kutounganisha pointi hizi ni kupuuza \emcap{Umbile la Mungu} na kuelekeza umakini kwenye pantheism pekee. Hatukatai asili ya pantheistic ya utata wa Kellogg. Tunaamini kwamba asili ya pantheistic ya Mzozo wa Kellogg hauwezi kueleweka ipasavyo ikiwa hautachunguzwa katika mwanga wa kweli wa \emcap{Umbile la Mungu}. Lakini, kwa bahati mbaya, katika uchunguzi wa mgogoro wa Kellogg, tahadhari hiyo ya pantheism hupokea umakini zaidi badala ya uchunguzi wa ukweli juu ya \emcap{Umbile la Mungu}.


You can do a search of Ellen White’s compilations to see just how much more attention pantheism received than the \emcap{personality of God}. If you were to search her writings for ‘pantheism’ or ‘pantheistic’, excluding the compilations after her death, you would find 36 occurrences. Among them are several repetitive quotations that Sister White copied from one letter to another, or to the special testimonies for the church. If you were to count the distinct occurrences you would only find 12 distinct quotations containing words like ‘\textit{pantheism}’ or ‘\textit{pantheistic}’\footnote{On the \href{https://egwwritings.org/}{https://egwwritings.org/} search bar, input the word “\textit{pantheis*} ”; this will include all words beginning with the ‘\textit{pantheis...}’, (including ‘\textit{pantheism}’ and ‘\textit{pantheistic}’). The results can be compared in subsetting the corpus of Ellen White writings by including or excluding compilations after her death. This option is available in the dropdown menu under the search bar.}. If you conducted the same search, but only in the compilations issued after her death, you would find 140 occurrences! All of these fall into one of the twelve distinct instances Sister White wrote on the subject of pantheism.


Unaweza kutafuta mkusanyo wa Ellen White ili kuona umakini zaidi uliopokelewa na pantheism kuliko \emcap{Umbile la Mungu}. Ikiwa ungetafuta maandishi yake ya ‘\textit{pantheism}’ au ‘\textit{pantheistic}’, ukiondoa mkusanyiko baada ya kifo chake, utapata matukio 36. Miongoni mwao kuna nukuu kadhaa zinazojirudiarudia ambazo Dada White alinakili kutoka barua moja kwa nyingine, au kwa ushuhuda maalum kwa ajili ya kanisa. Ikiwa ungehesabu matukio tofauti utapata nukuu 12 tu tofauti zenye maneno kama ‘\textit{pantheism}’ au ‘\textit{pantheistic}’\footnote{Kwenye \href{https://egwwritings.org/}{https://egwwritings.org/} upau wa utafutaji, ingiza neno “\textit{pantheis*} “; hii itajumuisha maneno yote yanayoanza na ‘\textit{pantheis...}’, (ikiwa ni pamoja na ‘\textit{pantheism}’ na ‘\textit{pantheistic}’). Matokeo yanaweza kulinganishwa katika kuweka kikundi cha maandishi ya Ellen White kwa kujumuisha au kutojumuisha makusanyo baada ya kifo chake. Chaguo hili linapatikana katika menyu inayoanguka chini ya upau wa utafutaji.}. Ikiwa ulifanya utafutaji sawa, lakini tu katika mkusanyiko iliyotolewa baada ya kifo chake, ungepata matukio 140! Yote haya yanaangukia katika moja ya matukio kumi na mawili tofauti Dada White aliandika juu ya somo la pantheism.


In a search of Ellen White writings on the phrase “\textit{personality of God}”, excluding the compilations after her death, you would find 58 occurrences. Among them are also several repetitive quotations that Sister White copied to several different letters and to the testimonies for the church. Yet, if you were to search this phrase within the compilations that were issued after her death you would only find 52 occurrences.


Katika utafutaji wa maandishi ya Ellen White juu ya maneno “\textit{Umbile la Mungu}”, ukiondoa makusanyo baada ya kifo chake, utapata matukio 58. Miongoni mwao pia ni nukuu kadhaa zinazojirudiarudia ambazo Dada White alinakili kwa herufi kadhaa tofauti na kwa shuhuda kwa kanisa. Walakini, ikiwa ungetafuta kifungu hiki ndani ya mkusanyiko huo zilitolewa baada ya kifo chake utapata tu matukio 52.


These simple statistics demonstrate the focus of the compilators after the death of Sister White. Such emphasis on pantheism changed our public opinion regarding Kellogg’s crisis. Forty-three, out of fifty-eight, quotations on the phrase “\textit{personality of God}” are found in letters and manuscripts, available to the public from 2015 onwards. This means that three quarters (\textit{74 percent}) of the quotation regarding the \emcap{personality of God}, prior to 2015, was not available to the public. Prior to 2015 we did not have much available data to study Kellogg's crisis in light of the \emcap{personality of God} and in its context.


Takwimu hizi rahisi zinaonyesha lengo la watunzi baada ya kifo cha Dada White. Msisitizo kama huo juu ya pantheism ulibadilisha maoni yetu ya umma kuhusu shida ya Kellogg. Nukuu Arobaini na tatu, kati ya hamsini na nane, ya maneno “\textit{Umbile la Mungu}” zinapatikana katika barua na maandishi, yanayopatikana kwa umma kuanzia 2015 na kuendelea. Hii ina maana kwamba theluthi tatu (\textit{asilimia 74}) ya nukuu kuhusu \emcap{Umbile la Mungu}, kabla ya 2015, haikuwa inapatikana kwa umma. Kabla ya 2015 hatukuwa na data nyingi za kusoma mgogoro wa Kellogg katika mwanga wa \emcap{Umbile la Mungu} na katika mazingira yake.


% Steps to Omega

\begin{titledpoem}
    
    \stanza{
        On pillars now, the shadows cast— \\
        A truth forsaken, from the past. \\
        In steps they chart the silent drift, \\
        Five marks of change, through sacred rift.
    }

    \stanza{
        Denial blooms when once truth stood, \\
        Foundations are not understood, \\
        The fundamentals, once held dear \\
        Obscured, as new creeds appear.
    }

    \stanza{
        Prophetic warnings have been dimmed, \\
        Pioneers are shunned, old hymns are trimmed. \\
        The testimonies once rang out \\
        But now they’re often tinged with doubt.
    }

    \stanza{
        “God is a person” cast aside, \\
        And now His essence they deride. \\
        Forgotten pillar once was strong \\
        Now a new pillar, which is wrong!
    }

    \stanza{
        Scholars now twist the sacred term, \\
        Words redefined, they now affirm. \\
        Gone is the quest to see God’s face, \\
        Dim the desire for His embrace.
    }

    \stanza{
        The Kellogg crisis point is missed, \\
        The alpha given untrue twist \\
        And thus, the lessons are not learned \\
        The church toward omega turned.
    }

    \stanza{
        Confusion reigns, we can’t perceive \\
        It is not clear what we believe \\
        Our history has been revised \\
        We wanted truth, but then they lied.
    }
    
\end{titledpoem}


% Steps to Omega

\begin{titledpoem}
    
    \stanza{
        On pillars now, the shadows cast— \\
        A truth forsaken, from the past. \\
        In steps they chart the silent drift, \\
        Five marks of change, through sacred rift.
    }

    \stanza{
        Denial blooms when once truth stood, \\
        Foundations are not understood, \\
        The fundamentals, once held dear \\
        Obscured, as new creeds appear.
    }

    \stanza{
        Prophetic warnings have been dimmed, \\
        Pioneers are shunned, old hymns are trimmed. \\
        The testimonies once rang out \\
        But now they’re often tinged with doubt.
    }

    \stanza{
        “God is a person” cast aside, \\
        And now His essence they deride. \\
        Forgotten pillar once was strong \\
        Now a new pillar, which is wrong!
    }

    \stanza{
        Scholars now twist the sacred term, \\
        Words redefined, they now affirm. \\
        Gone is the quest to see God’s face, \\
        Dim the desire for His embrace.
    }

    \stanza{
        The Kellogg crisis point is missed, \\
        The alpha given untrue twist \\
        And thus, the lessons are not learned \\
        The church toward omega turned.
    }

    \stanza{
        Confusion reigns, we can’t perceive \\
        It is not clear what we believe \\
        Our history has been revised \\
        We wanted truth, but then they lied.
    }
    
\end{titledpoem}
