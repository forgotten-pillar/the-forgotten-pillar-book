
\qrchapter{https://forgottenpillar.com/rsc/en-fp-chapter26}{Hatua za Omega}

Katika utafiti wetu kufikia hadi sasa, tumeona ushahidi kwamba utata wa Kellogg uliunganishwa na Fundisho la Utatu na \emcap{Umbile la Mungu} unaoonyeshwa katika pointi ya kwanza la \emcap{Kanuni za Kimsingi}. Kwa bahati mbaya, leo hatusimami juu ya msingi huo kuhusu \emcap{Umbile la Mungu}; tumejenga msingi mwingine ambao umebadilisha ukweli juu ya \emcap{Umbile la Mungu} kwa Mungu wa Utatu wa ajabu. Dada White alikuwa wazi dhidi ya upangaji huu upya na alitabiri kwamba katika kuhitimisha kazi yake, Mungu atarudia historia ya harakati ya Majilio na kusimamisha tena kila nguzo ya imani yetu iliyoshikiliwa hapo mwanzo.

\egw{\textbf{\underline{Bwana ametangaza kwamba historia ya zamani itafanyiwa ufafanuzi tunapoingia katika kazi ya kuhitimisha}. \underline{Kila ukweli} ambao ametoa kwa siku hizi za mwisho unapaswa kutangazwa kwa ulimwengu. \underline{Kila nguzo} aliyoisimamisha \underline{inapaswa kuimarishwa}. Hatuwezi sasa kuacha msingi ambao Mungu ameweka. Hatuwezi sasa kuingia katika shirika lolote jipya; kwa maana hilo lingemaanisha kuasi ukweli}.}[Ms129-1905.6; 1905][https://egwwritings.org/read?panels=p9797.13]

Kulinganisha \emcap{Kanuni za Kimsingi} na Imani za Sasa za Kimsingi za Waadventista Wasabato, tunaona kwamba tumeingia katika shirika jipya. Onyo la Mungu, limetolewa kupitia kwa Dada White, kusimamisha tena nguzo zote za imani yetu katika siku hizi za mwisho, inakuwa lazima. Tulipofuatilia fundisho la Utatu kutoka kwa mabishano ya Kellogg, tulikutana na maonyo ya Ellen White dhidi ya uasi wa alfa na omega, ambayo yataingia katika kanisa letu.

\egw{\textbf{‘The Living Temple’ lina alfa ya nadharia hizi. Nilijua kwamba \underline{omega ingefuata baada ya muda mfupi}; na nikatetemeka kwa ajili ya watu wetu}. Nilijua kwamba \textbf{lazima niwaonye kaka na dada zetu wasiingie katika mabishano \underline{juu ya uwepo na Umbile la Mungu}. Kauli zilizotolewa katika ‘Living Temple’ \underline{kuhusiana na jambo hili si sahihi}. }Andiko linalotumiwa kuthibitisha fundisho lililowekwa hapo, limetumiwa vibaya.}[SpTB02 53.2; 1904][https://egwwritings.org/read?panels=p417.271]

Katika muktadha wa upangaji upya wa Waadventista Wasabato, tunatambua hatua kadhaa ambazo zilikuwa muhimu kukamilisha upangaji upya huu na ni muhimu kuudumisha.

\subsection*{Hatua ya 1: Kataa Kanuni za Kimsingi kama msingi ya imani yetu na uwakilishi rasmi, na sahihi, wa Imani za Waadventista Wasabato}

Hatua ya kwanza muhimu ni kuficha msingi asilia wa imani yetu kwa kuutenganisha na \emcap{Kanuni za Kimsingi}.

\egw{\textbf{Kama watu, tunapaswa \underline{kusimama kidete kwenye jukwaa la ukweli wa milele} ambao umestahimili mtihani na majaribio. Tunapaswa \underline{kushikilia nguzo za uhakika za imani yetu}. \underline{Kanuni za ukweli} ambazo Mungu ametufunulia \underline{ndio msingi wetu wa kweli}. Zimetufanya tulivyo. Mpwito wa wakati haujapunguza thamani yao. \underline{Ni juhudi za mara kwa mara za adui kuondoa ukweli huu kutoka kwa mpangilio yao}, na kuweka mahali pao \underline{nadharia potofu}. Yeye \underline{ataleta} kila kitu awezacho ili kutekeleza mipango yake ya udanganyifu.}}[SpTB02 51.2; 1904][https://egwwritings.org/read?panels=p417.261]

\egw{\textbf{Ujumbe wa kila utaratibu na aina umehimizwa kwa Waadventista Wasabato, ili kuchukua nafasi ya ukweli ambao, \underline{pointi baada ya pointi}, umetafutwa kwa maombi kujifunza, na kushuhudia kwa uwezo wa kutenda miujiza wa Bwana}. \textbf{Lakini \underline{alama za njia} \underline{ambazo zimetufanya tulivyo}, \underline{zinapaswa kuhifadhiwa}, na \underline{zitahifadhiwa}, kama Mungu ameonyesha kupitia neno lake na ushuhuda wa Roho wake}. \textbf{Anatuita \underline{kushika kwa uthabiti}, kwa mshiko wa imani, \underline{kanuni za msingi} ambazo \underline{msingi wake ni mamlaka isiyo na shaka}}.}[SpTB02 59.1; 1904][https://egwwritings.org/read?panels=p417.299]

Kanuni za Msingi zilikuwa kweli ambazo Mungu alifunua kwa waanzilishi baada ya kupita kwa wakati mnamo 1844. Tumeona ushuhuda wa waanzilishi wetu, pamoja na Ellen White, kuhusu hoja ya kwanza ya Kanuni za Msingi. Wote walikuwa katika maelewano kuhusu mambo haya mahususi ya imani yetu. Mnamo 1863, Waadventista Wasabato walipanga wao wenyewe wakawa kanisa, kama mwili wa madhehebu. Tangu wakati huo, wengi walikuwa wakiwakilisha vibaya nafasi ya Kanisa la Waadventista Wasabato na waanzilishi waliona ni muhimu kuyakabili maswali, \others{na nyakati nyingine kusahihisha taarifa za uwongo zinazosambazwa dhidi ya} imani za kanisa na mazoea. Kwa hivyo, mnamo 1872, waanzilishi walitoa hati inayoitwa “\textit{A Declaration of the Fundamental Principles, Taught and Practiced by the Seventh-Day Adventists}”\footnote{“A Declaration of the Fundamental Principles, Taught and Practiced by the Seventh-Day Adventists (1872) : MVT : Free Download, Borrow, and Streaming : Internet Archive.” Internet Archive, 2025, \href{https://archive.org/details/ADeclarationOfTheFundamentalPrinciplesTaughtAndPracticedByThe}{archive.org/details/ADeclarationOfTheFundamentalPrinciplesTaughtAndPracticedByThe}. Accessed 3 Feb. 2025.}. Tamko hili liliwasilisha kwa umma na \others{taarifa fupi ya nini, na imekuwa, kwa umoja mkuu, ikishikiliwa na}[The preface of the Fundamental Principles in 1872.] Waadventista Wasabato.

Katika sura ya “\hyperref[chap:authority]{Mamlaka ya Kanuni za Msingi}”, tulijadili jinsi gani wataalamu wanaounga mkono Utatu wamekuwa wakihatarisha mamlaka ya Kanuni za Msingi, kukana thamani yao ya kweli katika historia yetu ya Waadventista.

Wataalamu wanaounga mkono utatu wanabisha kwamba tamko hili halikuwa kama linavyodai kuwa—tamko la Kanuni za Msingi, zinazofundishwa na kutekelezwa na Waadventista Wasabato. Tamko hili lilikuwa muhtasari wa sifa kuu za imani ya Waadventista, na kwa kweli hakuna pointi yenye tatizo au ya kuchukiza kama pointi ya kwanza, linalohusu Umbile la Mungu na uwepo wake ulipo. Lakini ushahidi unaounga mkono Kanuni za Msingi, hasa kwa pointi ya kwanza, ni mingi mno.

Madai haya yote yanakanushwa kwa urahisi na ukweli kwamba Kanuni za Msingi zimekuwa zilizotolewa mara kwa mara na kuchapishwa tena katika kipindi chote cha maisha ya Dada White, hadi 1914. Kama yalikuwa ni maoni ya faragha tu ya watu wachache, kama wanavyodai wasomi\footnote{Ministry Magazine “Our Declaration of Fundamental Beliefs”: January 1958, Roy Anderson, J. Arthur Buckwalter, Louise Kleuser, Earl Cleveland and Walter Schubert}, je! yamechapishwa tena mara kwa mara katika kipindi cha miaka 42\footnote{For a detailed list of publications throughout these years, see the Appendix.}, yakidai hadharani kuwakilisha muhtasari wa imani ya Waadventista Wasabato? Ikiwa zingetolewa mara moja tu, tungeweza kuona kuwa ni njama ya baadhi ya watu kuwawakilisha vibaya kimakusudi Imani ya Waadventista wa Sabato. Kinyume chake, Kanuni za Msingi zilichapishwa tena mara kwa mara, na kwa kweli iliwakilisha imani na utendaji rasmi wa Waadventista Wasabato.

Hoja nyingine ni kwamba Dada White aliidhinisha Kanuni za Msingi katika maandishi yake kwa kuzirejelea kwa uwazi, na pia kwa kufundisha kweli zile zile zinazofundishwa katika Kanuni za Msingi. Kazi za waanzilishi wetu pia zinapatana na kauli katika Tamko hili la Kanuni za Msingi. Kwa kuzingatia ukweli huu wote, ni bila kuepukika kwamba tamko hili lilikuwa la ukweli katika madai yake. Hati hii kwa kweli ilikuwa tamko la Kanuni za Msingi, zinazofundishwa na kutekelezwa na Kanisa la Waadventista Wasabato, likiwakilisha \others{muhtasari wa imani yetu}, \others{taarifa fupi ya kile ambacho ni hadharani, na kimekuwa, kwa umoja mkuu, kikishikiliwa na} Waadventista Wasabato.\footnote{The preface of the Fundamental Principles in 1872.} Kwa hivyo, inawakilisha kwa usahihi imani na utendaji wa Waadventista wa Sabato, na inawakilisha msingi wa imani ya Waadventista Wasabato wakati wa Ellen White.

Leo, katika kutetea fundisho la Utatu, wanahistoria Waadventista wanadai kwa ujasiri kwamba wakati waanzilishi walikuwa wakisoma kweli za Waadventista kama vile patakatifu, hukumu ya uchunguzi, Sabato na mafundisho mengine, \others{hawakujifunza somo la mafundisho ya Mungu}. Hawa Wanahistoria Waadventista hudai kwa uwongo kwamba fundisho la Mungu \others{halikuwa swali ambalo wao walishughulikiwa wakati huo}[Denis Kaiser. “From Antitrinitarianism to Trinitarianism: The Adventist story” and Panelist. The God We Worship: A Godhead Symposium. Central California Conference, Dinuba, CA. March 23-24, 2018.]. Kufuatia dai hili la uwongo, wanawasilisha data ya kihistoria juu ya jinsi fundisho la Waadventista lilihamia hatua kwa hatua kuelekea uelewaji wa Utatu. Ukweli ni kwamba, kuna baadhi ya matukio ya hapo awali\footnote{The earliest mention of the Trinity doctrine, in a positive sense, was when M.C. Wilcox reprinted a non-Adventist article by Samuel Spear in Signs of the Times, December 7th, 1891 and December 14th, 1891} wakati fundisho la Utatu linatajwa kwa njia chanya katika fasihi zetu. Lakini unapozingatia ukweli kwamba kanisa la Waadventista lilikuwa na msimamo wake chanya juu ya somo la fundisho la Mungu, kama lilivyoelezwa katika Kanuni za Msingi, haya matukio hayawezi kutafsiriwa kama maendeleo katika kuelewa, lakini badala yake ni kuingia kwa fundisho la Utatu ndani ya Kanisa la Waadventista Wasabato.

Ni rahisi kukanusha madai kwamba waanzilishi wa Kiadventista hawakuelewa fundisho kuhusu Mungu. Ikiwa hawangeielewa, wangekosa kutangaza ujumbe wa malaika wa kwanza. Tulijadili jambo hili kwa undani katika sura ya “\hyperref[chap:remembering-the-beginning]{Kukumbuka mwanzo}”. Harakati ya Waadventista Wasabato haikuwa ya kushindwa, bali harakati ya kinabii inayoongozwa na Mungu.

\subsection*{Hatua ya 2: Puuza maonyo ya kujenga msingi mpya}

Wakati Kanuni za Msingi zinapoondolewa kwenye mlingano, maonyo mengi ya Ellen White yanashindwa kung'aa katika nuru yao ya kweli na maana yake ya kweli haipatani kwa msomaji.

Tumetaja nukuu nyingi ambapo Dada White alionya kanisa lisiondoke kwa Kanuni za Msingi. Tuliyashughulikia katika sura “\hyperref[chap:apostasy]{Uasi mkuu utakuja kutambuliwa hivi karibuni}”, lakini tutataja moja ya nukuu maarufu tena.

\egw{\textbf{Adui wa roho ametaka kuleta dhana kwamba matengenezo makubwa yangetukia kati ya Waadventista Wasabato, na kwamba matengenezo haya yangefanyika \underline{ingejumuisha kuacha mafundisho ambayo yanasimama kama nguzo za imani yetu} na kuhusika katika mchakato wa kujipanga upya}. Je, matengenezo haya yangefanyika, matokeo yangekuwa nini? \textbf{Kanuni za ukweli ambazo Mungu katika hekima yake ametoa kwa kanisa la masalio zingetupwa. Dini yetu ingebadilishwa. \underline{Kanuni za msingi ambazo zimeendeleza kazi kwa miaka hamsini iliyopita ingehesabiwa kama makosa}}. \textbf{Shirika jipya lingeanzishwa. Vitabu vya namna mpya vingeandikwa. Mfumo wa falsafa ya kiakili ingeanzishwa}...}[Lt242-1903.13; 1903][https://egwwritings.org/read?panels=p7767.20]

\egwnogap{Nani mwenye mamlaka ya kuanzisha harakati hiyo? \textbf{Tuna Biblia zetu. Tuna uzoefu wetu, unaothibitishwa na utendaji wa kimiujiza wa Roho Mtakatifu}. \textbf{Tuna ukweli ambao hatupaswi kushusha makali yake.} \textbf{\underline{Je, hatutakataa kila kitu ambacho hakipatani na ukweli huu}?}}[Lt242-1903.14; 1903][https://egwwritings.org/read?panels=p7767.21]

\subsection*{Hatua ya 3: Kataa kwamba ubinafsi wa Mungu ulikuwa nguzo ya imani yetu na sehemu ya msingi wa imani yetu}

Kuna kauli moja ya Ellen White ambayo inaonekana inaunga mkono madai kwamba \emcap{ubinafsi wa Mungu} haikuwa nguzo ya imani yetu. Usemi mwingine wa “\textit{nguzo za imani yetu}” ni “\textit{alama}”. Katika nukuu zifuatazo, Dada White anaorodhesha alama kadhaa: utakaso wa patakatifu, jumbe za malaika watatu, hekalu la Mungu, Sabato na kutokufa kwa waovu.

\egw{Kupita kwa wakati huo katika 1844 kilikuwa kipindi cha matukio makubwa, kilichofungua kwa mshangao machoni  mwetu \textbf{utakaso wa patakatifu upitapo mbinguni}, na kuwa na uhusiano wa dhati kwa watu wa Mungu duniani, [pia] \textbf{ujumbe wa malaika wa kwanza na wa pili na wa tatu}, akiifunua ile bendera ambayo juu yake ilikuwa imeandikwa, ‘Amri za Mungu na imani ya Yesu.’ [Ufunuo 14:12.] Moja ya alama muhimu chini ya ujumbe huu ilikuwa \textbf{hekalu la Mungu}, lililoonekana mbinguni na watu wake wapendao ukweli, na sanduku lenye sheria ya Mungu. Nuru ya \textbf{Sabato} ya amri ya nne ilimulika miale yake mikali katika njia ya waasi wa sheria za Mungu. \textbf{Kufariki na kutokuwa na uhai baada ya kifo kwa waovu} ni alama ya zamani ya kihistoria. \textbf{Siwezi kukumbuka chochote zaidi ambacho kinaweza kuja chini ya mada ya alama za zamani}. Kilio hiki chote cha kubadilisha alama za zamani ni cha kudhaniwa tu.}[Ms13-1889.9; 1889][https://egwwritings.org/read?panels=p4179.14]

Mwishoni mwa orodha hii ya alama muhimu, au nguzo za imani yetu, anasema kwamba anaweza kukumbuka hakuna kitu zaidi ambacho kinaweza kuja chini ya mada ya alama za zamani. Kwa wengi, nukuu hii ni uthibitisho kwamba \emcap{ubinafsi wa Mungu} haukuwa alama ya zamani wala nguzo. Ni kweli kwamba katika nukuu hili, Dada White hakutaja kwa uwazi \emcap{ubinafsi wa Mungu}, lakini ingekuwa imejumuishwa kwa uwazi chini ya ujumbe wa malaika wa kwanza, na pia kuwa mafundisho ya msingi ya ujumbe wa patakatifu. Zaidi ya hayo, kuna nukuu nyingine kutoka kwa Dada White ambayo yanajumuisha kwa uwazi \emcap{ubinafsi wa Mungu} kama alama ya zamani, au nguzo ya imani yetu.

\egw{Wale wanaotafuta kuondoa \textbf{alama za zamani} hawashikilii; \textbf{hawakumbuki jinsi walivyopokea na kusikia}. Wale ambao wanajaribu \textbf{\underline{kuleta} nadharia ambayo ingeondoa \underline{nguzo za imani yetu}} \textbf{kuhusu patakatifu}, \textbf{\underline{au kwa habari ya ubinafsi wa Mungu au wa Kristo}, wanafanya kazi kama vipofu}. Wanatafuta kuingiza mashaka na kuwafanya watu wa Mungu \textbf{wapeperushwe bila nanga}.}[Ms62-1905.14; 1905][https://egwwritings.org/read?panels=p10026.20]

Dada White pia anatufundisha kwamba nguzo za imani yetu ni msingi wa imani yetu.

\egw{\textbf{Ni ushawishi gani ambao unaweza kusababisha watu katika hatua hii ya historia yetu kufanya kazi katika njia ya kichinichini, yenye nguvu ya \underline{kubomoa msingi wa imani yetu},—msingi ambayo iliwekwa mwanzoni mwa kazi yetu kwa kujifunza kwa maombi neno na kwa ufunuo? Juu ya \underline{msingi huu} tumekuwa tukijenga kwa \underline{miaka hamsini iliyopita}. Unashangaa kwamba ninapoona mwanzo wa kazi ambayo ingeondoa \underline{baadhi ya nguzo za imani yetu}, nina la kusema? Ni lazima nitii amri, ‘Kutana nayo!’}}[SpTB02 58.1; 1904][https://egwwritings.org/read?panels=p417.295]

Kuondoa baadhi ya nguzo za imani yetu kunamaanisha kubomoa msingi wa imani yetu. Mahali pengine, Dada White alisema kwamba kubomoa au kudhoofisha msingi wa imani yetu inayofanywa kwa kufundishwa maoni kuhusu \emcap{ubinafsi wa Mungu}.

\egw{Chuo kilitolewa Battle Creek; bado wanafunzi wanaitwa huko, na huko \textbf{wanafundishwa hisia zenyewe kuhusu ubinafsi wa Mungu na Kristo, mafundisho haya yanadhoofisha msingi wa imani yetu}.}[Lt72-1906.5; 1906][https://egwwritings.org/read?panels=p10013.11]

Kwa kuzingatia nukuu hizi tunaona ushuhuda chanya ambao Umbile la Mungu ulikuwa sehemu ya msingi wa imani yetu. Zaidi ya hayo, katika sura ya kumi ya shuhuda maalum, yenye kichwa “\textit{Msingi wa imani yetu}”, Dada White alitaja “\textit{Kanuni za Msingi}” kwa kutumia visawe “\textit{nguzo za imani yetu}”, “\textit{viashiria njia}”, na “\textit{alama}”, wakati wa kushughulikia msingi wa imani yetu.

\subsection*{Hatua ya 4: Badilisha maana ya neno “Umbile la Mungu”}

Neno ‘\textit{personality}’ lina matumizi mawili tofauti na fasili ya kawaida zaidi katika matumizi ya kila siku ni katika eneo la saikolojia. ‘\textit{Personality}’ hufafanuliwa kuwa “\textit{seti za sifa za tabia, utambuzi, na mifumo ya kihisia ambayo hubadilika kutoka kwa kibayolojia na mambo ya mazingira}”\footnote{Wikipedia Contributors. “Personality.” Wikipedia, Wikimedia Foundation, 19 Apr. 2019, \href{https://en.wikipedia.org/wiki/Personality}{en.wikipedia.org/wiki/Personality}.}. Ni muhimu sana kutambua kwamba tunaposhughulika na nguzo ya imani yetu—“\textit{Umbile la Mungu}”—hatuko katika nyanja za saikolojia. Ufafanuzi sahihi wa neno ‘\textit{personality}’ ndani ya fundisho la \emcap{Umbile la Mungu} unapatikana katika Kamusi ya Merriam-Webster: “\textit{Ubora au hali ya kuwa Nafsi}”\footnote{\href{https://www.merriam-webster.com/dictionary/personality}{Merriam-Webster Dictionary} - ‘\textit{personality}’}. Kulingana na Kamusi ya Merriam-Webster, ufafanuzi huu umekuwa ukitumika tangu karne ya 15\footnote{Angalia “\href{https://www.merriam-webster.com/dictionary/personality\#word-history}{First known use}” ya neno ‘personality’ katika Kamusi ya Merriam Webster}. Katika toleo la 1828 la Kamusi ya Merriam Webster tunasoma ufafanuzi wa neno ‘\textit{personality}’ kama: “\textit{kile ambacho kinajumuisha Nafsi binafsi kuwa Nafsi tofauti}”\footnote{\href{https://archive.org/details/americandictiona02websrich/page/272/mode/2up}{Merriam-Webster Dictionary, 1828 edition} - ‘\textit{personality}’} \footnote{\href{https://archive.org/details/websterscomplete00webs/page/974/mode/2up}{Toleo la 1886 la Kamusi ya Merriam-Webster} linafafanua neno ‘\textit{personality}’ kama: “\textit{kile ambacho kinajumuisha, au kinahusu, Nafsi}”}. Fasili zote mbili zinapatikana katika The Encyclopaedic Dictionary, ya Hunter Robert\footnote{\href{https://babel.hathitrust.org/cgi/pt?id=mdp.39015050663213&view=1up&seq=780}{Hunter Robert, The Encyclopaedic Dictionary} - ‘\textit{personality}’}—kamusi inayomilikiwa na Ellen White. Matumizi ya fafanuzi hizi yanaweza kuonekana kutoka kwa makala zilizoandikwa kuhusu \emcap{Umbile la Mungu}.

Katika mwaka wa 1903, Dada White alipomwandikia Dk. Kellogg, \egwinline{Nimekuwa \textbf{daima} na ushuhuda sawa ambao sasa ninao \textbf{kuhusu Umbile la Mungu}}[Lt253-1903.9; 1903][https://egwwritings.org/read?panels=p9980.15], alikumbuka maono yake wakati alimwona Baba na Mwana.

\egw{‘Mara nyingi nimemwona Yesu mpendwa, kwamba\textbf{ Yeye ni Nafsi}.\textbf{ Nilimuuliza kama Baba Yake alikuwa Nafsi, }na \textbf{alikuwa na \underline{umbo} kama Yeye Mwenyewe}. Yesu alisema, ‘\textbf{Mimi ni chapa kamili ya Umbile Wake!}’ [Waebrania 1:3.]}[Lt253-1903.12; 1903][https://egwwritings.org/read?panels=p9980.18]

Ubora au hali ambayo Dada White anafafanua Mungu kuwa Nafsi ni kuwa ana \textit{umbo}—\textit{mwonekano wa kimwili}. Dk. Kellogg anafuata fasili sawa ya neno \textit{‘personality’}, ingawa kwa uvumi.

\others{Ukweli kwamba Mungu ni mkuu sana hivi kwamba hatuwezi kutokeza kiakili picha iliyo wazi ya \textbf{mwonekano wake wa kimwili} hauhitaji kupunguza katika akili zetu ukweli wa \textbf{Umbile Lake}...}[John H. Kellogg, The Living Temple, p. 31][https://archive.org/details/J.H.Kellogg.TheLivingTemple1903/page/n31/mode/2up]

Kama tulivyoona hapo awali, waanzilishi wetu Waadventista pia walibainisha mwonekano wa kimwili kama sifa inayomfanya Mungu kuwa Nafsi. James White aliandika, \others{Wale wanaokataa \textbf{Umbile la Mungu}, wanasema kwamba ‘mfano’ hapa haimaanishi \textbf{umbo la kimwili}, bali picha ya maadili...}[James S. White, PERGO 1.1; 1861][https://egwwritings.org/read?panels=p1471.3]. J. B. Frisbie aliandika, \others{Wengine wanaonekana kudhani kuwa inapingana na \textbf{Umbile la Mungu}, kwa sababu yeye ni Roho, na husema kwamba yeye hana \textbf{mwili, wala viungo}...}[\href{https://documents.adventistarchives.org/Periodicals/RH/RH18540307-V05-07.pdf}{Adventist Review and Sabbath Herald, March 7, 1854}, J. B. Frisbie, “The Seventh-Day Sabbath Not Abolished”, p. 50]

Kwa kuzingatia ukweli, tunatambua matumizi ya neno ‘\textit{personality}’. Wakati mada juu ya \emcap{Umbile la Mungu} inapowasilishwa katika uhusiano wake na fundisho la Utatu, mara nyingi kuna tabia ya kubadilisha maana ya neno ‘\textit{personality}’. Pia ni muhimu kutaja kwamba somo la \emcap{Umbile la Mungu} linahusu Umbile la Baba. Hii inaonekana wazi kutoka kwa data iliyotolewa.

\subsection*{Hatua ya 5: Katika kuchunguza mgogoro wa Kellogg, kuhamisha lengo kuu kutoka kwa Umbile la Mungu hadi pantheism}

Data juu ya mgogoro wa Kellogg, kuhusiana na fundisho la Utatu, ni nyingi sana ikiwa \emcap{Umbile la Mungu} linahesabiwa katika mlinganyo huo. Njia pekee ya kutounganisha pointi hizi ni kupuuza \emcap{Umbile la Mungu} na kuelekeza umakini kwenye pantheism pekee. Hatukatai asili ya pantheistic ya utata wa Kellogg. Tunaamini kwamba asili ya pantheistic ya Mzozo wa Kellogg hauwezi kueleweka ipasavyo ikiwa hautachunguzwa katika mwanga wa kweli wa \emcap{Umbile la Mungu}. Lakini, kwa bahati mbaya, katika uchunguzi wa mgogoro wa Kellogg, tahadhari hiyo ya pantheism hupokea umakini zaidi badala ya uchunguzi wa ukweli juu ya \emcap{Umbile la Mungu}.

Unaweza kutafuta mkusanyo wa Ellen White ili kuona umakini zaidi uliopokelewa na pantheism kuliko \emcap{Umbile la Mungu}. Ikiwa ungetafuta maandishi yake ya ‘\textit{pantheism}’ au ‘\textit{pantheistic}’, ukiondoa mkusanyiko baada ya kifo chake, utapata matukio 36. Miongoni mwao kuna nukuu kadhaa zinazojirudiarudia ambazo Dada White alinakili kutoka barua moja kwa nyingine, au kwa ushuhuda maalum kwa ajili ya kanisa. Ikiwa ungehesabu matukio tofauti utapata nukuu 12 tu tofauti zenye maneno kama ‘\textit{pantheism}’ au ‘\textit{pantheistic}’\footnote{Kwenye \href{https://egwwritings.org/}{https://egwwritings.org/} upau wa utafutaji, ingiza neno “\textit{pantheis*} “; hii itajumuisha maneno yote yanayoanza na ‘\textit{pantheis...}’, (ikiwa ni pamoja na ‘\textit{pantheism}’ na ‘\textit{pantheistic}’). Matokeo yanaweza kulinganishwa katika kuweka kikundi cha maandishi ya Ellen White kwa kujumuisha au kutojumuisha makusanyo baada ya kifo chake. Chaguo hili linapatikana katika menyu inayoanguka chini ya upau wa utafutaji.}. Ikiwa ulifanya utafutaji sawa, lakini tu katika mkusanyiko iliyotolewa baada ya kifo chake, ungepata matukio 140! Yote haya yanaangukia katika moja ya matukio kumi na mawili tofauti Dada White aliandika juu ya somo la pantheism.

Katika utafutaji wa maandishi ya Ellen White juu ya maneno “\textit{Umbile la Mungu}”, ukiondoa makusanyo baada ya kifo chake, utapata matukio 58. Miongoni mwao pia ni nukuu kadhaa zinazojirudiarudia ambazo Dada White alinakili kwa herufi kadhaa tofauti na kwa shuhuda kwa kanisa. Walakini, ikiwa ungetafuta kifungu hiki ndani ya mkusanyiko huo zilitolewa baada ya kifo chake utapata tu matukio 52.

Takwimu hizi rahisi zinaonyesha lengo la watunzi baada ya kifo cha Dada White. Msisitizo kama huo juu ya pantheism ulibadilisha maoni yetu ya umma kuhusu shida ya Kellogg. Nukuu Arobaini na tatu, kati ya hamsini na nane, ya maneno “\textit{Umbile la Mungu}” zinapatikana katika barua na maandishi, yanayopatikana kwa umma kuanzia 2015 na kuendelea. Hii ina maana kwamba theluthi tatu (\textit{asilimia 74}) ya nukuu kuhusu \emcap{Umbile la Mungu}, kabla ya 2015, haikuwa inapatikana kwa umma. Kabla ya 2015 hatukuwa na data nyingi za kusoma mgogoro wa Kellogg katika mwanga wa \emcap{Umbile la Mungu} na katika mazingira yake.

% Steps to Omega

\begin{titledpoem}
    
    \stanza{
        On pillars now, the shadows cast— \\
        A truth forsaken, from the past. \\
        In steps they chart the silent drift, \\
        Five marks of change, through sacred rift.
    }

    \stanza{
        Denial blooms when once truth stood, \\
        Foundations are not understood, \\
        The fundamentals, once held dear \\
        Obscured, as new creeds appear.
    }

    \stanza{
        Prophetic warnings have been dimmed, \\
        Pioneers are shunned, old hymns are trimmed. \\
        The testimonies once rang out \\
        But now they’re often tinged with doubt.
    }

    \stanza{
        “God is a person” cast aside, \\
        And now His essence they deride. \\
        Forgotten pillar once was strong \\
        Now a new pillar, which is wrong!
    }

    \stanza{
        Scholars now twist the sacred term, \\
        Words redefined, they now affirm. \\
        Gone is the quest to see God’s face, \\
        Dim the desire for His embrace.
    }

    \stanza{
        The Kellogg crisis point is missed, \\
        The alpha given untrue twist \\
        And thus, the lessons are not learned \\
        The church toward omega turned.
    }

    \stanza{
        Confusion reigns, we can’t perceive \\
        It is not clear what we believe \\
        Our history has been revised \\
        We wanted truth, but then they lied.
    }
    
\end{titledpoem}
