

\qrchapter{https://forgottenpillar.com/rsc/en-fp-chapter7}{Mamlaka ya Kanuni za Msingi} \label{chap:authority}


Katika sura ya 10 ya Shuhuda Maalum, tunasoma jinsi Mungu alivyoweka msingi wa imani yetu. Dada White alitumia semi kadhaa tofauti kurejelea msingi wa imani yetu. Marejeleo yake yalijumuisha: “\textit{jukwaa la ukweli wa milele}, \textit{“nguzo za imani yetu”}, \textit{“kanuni za ukweli”}, \textit{“alama kuu”}, \textit{“viashiria njia”}, na “\textit{kanuni za msingi}— hizi zote zinarejelea \emcap{Kanuni za Msingi}. Mwishoni mwa sura, alithibitisha mapenzi ya Mungu kwamba \egwinline{Anatoa wito kwetu sisi kushikilia kwa uthabiti, kwa mshiko wa imani, \textbf{kanuni za msingi} ambazo \textbf{msingi wake ni mamlaka isiyo na \underline{shaka}}.}[SpTB02 59.1; 1904][https://egwwritings.org/read?panels=p417.299]


Mamlaka ya \emcap{kanuni za msingi} hayana shaka. Yalikuwa matokeo ya kujifunza kwa kina na kwa bidii katika wakati wa kukatishwa tamaa sana, wakati \egwinline{\textbf{\underline{pointi baada ya pointi}, ilitafutwa kwa kujifunza kwa maombi, na kushuhudiwa kwa \underline{uwezo wa kutenda miujiza wa Bwana}}}\footnote{Ibid.}. \egwinline{\textbf{Hivyo \underline{pointi kuu za imani yetu} tulivyozishikilia leo ziliimarishwa kwa uthabiti}. \textbf{\underline{Pointi baada ya pointi} ilifafanuliwa waziwazi, na ndugu wote wakapatana}.}[Lt253-1903.4; 1903][https://egwwritings.org/read?panels=p14068.9980010]


Yalikuwa tokeo la waanzilishi wetu kujifunza Biblia kwa bidii, baada ya hitimisho ya wakati wa 1844. Wakati vuguvugu ya Waadventista Wasabato ikiendelea, kulizuka haja ya kuanzishwa kwa shirika, ambalo lilianzishwa mnamo 1863. Mnamo 1872, Kanisa la Waadventista Wasabato ilitoa hati inayoitwa “\textit{A Declaration of the Fundamental Principles, Taught and Practiced by the Seventh-Day Adventists}. Hii ilikuwa hati ya kwanza iliyoandikwa kutangaza \emcap{kanuni za kimsingi} kama taarifa za umma za imani ya Waadventista Wasabato. Hati hii ilikuwa muhtasari wa imani ya Waadventista Wasabato kwa umma na ilitangaza \others{ni nini, imekuwa kwa umoja mkuu, inashikiliwa na} Waadventista Wasabato. Iliandikwa \others{kujibu maswali} kuhusu kile kilichoaminiwa na Waadventista Wasabato, \others{kusahihisha taarifa za uwongo zilizosambazwa} na \others{kuondoa hisia potofu}[FP1872 3.1; 1872][https://egwwritings.org/read?panels=p928.8].


Leo mjadala bado upo kuhusu ni nani aliyeandika muhtasari hiyo kwa sababu hapo awali, mnamo 1872, iliachwa bila kutambulishwa. Mnamo mwaka wa 1874, James White aliichapisha katika Signs of the Times\footnote{\href{https://adventistdigitallibrary.org/adl-364148/signs-times-june-4-1874}{Signs of the Times, June 4, 1874}} na Uriah Smith katika the Review and Herald\footnote{\href{http://documents.adventistarchives.org/Periodicals/RH/RH18741124-V44-22.pdf}{The Advent Review and Herald of the Sabbath, November 24, 1874}}—wote wakitia kwa sahihi zao wenyewe. Mnamo 1889, Uriah Smith aliirekebisha kwa kuongeza pointi tatu; ilitolewa katika Kitabu cha Mwaka cha Waadventista na kwa sahihi yake juu yake. Uriah Smith alikufa mwaka wa 1903 na uchapishaji wote uliofuatana wa \emcap{Kanuni za Msingi} ulichapishwa chini ya jina lake. Vilichapishwa katika Vitabu vya Mwaka—kila mwaka kuanzia 1905 hadi 1914\footnote{For more detailed timeline of Fundamental Principles, see \hyperref[appendix:timeline]{Appendix: Fundamental Principles - Timeline}}. Dada White alikufa mwaka wa 1915 na, kwa miaka 17 iliyofuata, \emcap{kanuni za msingi} hazikuchapishwa. Kutokea kwao tena kulikuwa katika Kitabu cha Mwaka cha 1931 zilipopokea mabadiliko makubwa.


Mnamo 1971, LeRoy Froom aliandika kuhusu taarifa fulani kutoka 1872: \others{Ingawa ilichapishwa bila utambulishi wa wazi wa mwandishi, ilitungwa na Smith}[Edwin Froom, LeRoy. Movement of Destiny. 1971., p. 160]. Kwa bahati mbaya, hakutoa data yoyote kuunga mkono madai yake. Inasikitisha kuona jinsi wataalamu wanaounga mkono utatu wanazingatia \emcap{Kanuni za Msingi} kuwa na umuhimu mdogo sana. Thamani yao ya kweli imepungua sana kwa kuhusisha imani hizi na zile za kikundi kidogo cha watu, hasa kwa James White au Imani binafsi ya Uriah Smith, badala ya imani ambayo ilikuwa \others{kwa umoja mkubwa, ikishikiliwa na}[Preface of the Fundamental Principles 1872] Waadventista Wasabato. Mwaka 1958, Makala ya Utumishi ilielezea \emcap{Kanuni za Msingi} kama ifuatavyo:


\others{Ni kweli kwamba mnamo 1872 ‘Declaration of the Fundamental Principles Taught and Practiced by Seventhday Adventists’ ilichapishwa, \textbf{lakini haikukubaliwa kamwe na dhehebu na kwa hiyo haiwezi kuchukuliwa kuwa rasmi}. Ni dhahiri kuwa kikundi kidogo, \textbf{labda hata moja au wawili, walijaribu kuweka kwa maneno yale waliyofikiri ni maoni ya kanisa yote…}}[Ministry Magazine “\textit{Our Declaration of Fundamental Beliefs}”, January 1958, Roy Anderson, J. Arthur Buckwalter, Louise Kleuser, Earl Cleveland and Walter Schubert]


Shida kuu kwayo, hakuna ushahidi wa kuunga mkono dai kwamba \emcap{Kanuni za Msingi} hazikuwa kielelezo cha imani ya mwili mzima. Tunajua kwa uhakika kwamba Dada White aliziidhinisha na, kutokana na ushawishi wake pekee, tunajua kwamba imani hizi zilikuwa kwa kweli imekubaliwa na dhehebu—hii ni pamoja na ukweli kwamba zilichapishwa mara nyingi kwa kipindi cha miaka 42, wakati wa maisha ya Ellen White.


Lakini kusiwe na ubishi juu ya uandishi wa \emcap{Kanuni za Msingi}. Sisi tunayo nukuu kutoka kwa Dada White kuhusu ni nani aliyeziandika. Akimzungumzia Uriah Smith, Dada White aliandika:


\egw{\textbf{Ndugu Smith alikuwa nasi tangia kuchipuka kwa kazi hii. Anaelewa jinsi \underline{sisi—mume wangu pamoja nami}—tumeipeleka kazi mbele na juu hatua kwa hatua na tumevumilia dhiki, umaskini, na ukosefu wa mali. Pamoja nasi kulikuwa na wale wafanyakazi wa awali. Mzee Smith, hasa, alikuwa pamoja na mume wangu katika ujana wake}. …}[Ms54-1890.6; 1890][https://egwwritings.org/read?panels=p7213.15]


\egwnogap{\textbf{\underline{Tumesimama bega kwa bega na Mzee Smith katika kazi hii wakati Bwana alipokuwa akiweka kanuni za msingi}}. \textbf{Ilitubidi kufanya kazi mara kwa mara dhidi ya watu wenye wazo moja pekee}, ambao walifikiri mahusiano sahihi ya kibiashara kuhusiana na kazi ambayo ilipaswa kufanywa ulikuwa ushahidi wa mawazo ya kidunia, na wale wapumbavu ambao wangejionyesha kama wanao uwezo wa kubeba majukumu, lakini hawakuweza kuaminiwa kuunganishwa na kazi wasije wakaizungusha katika mistari isiyo sahihi. \textbf{Hatua baada ya hatua imebidi ichukuliwe, \underline{sio kwa kufuata hekima ya watu} ila kwa kufuata hekima na mawaidha ya Mwenye hekima kupita kiasi hata kwamba asiweze kukosea na mwenye uzuri sana hata asiweze kutudhuru}. \textbf{Kunavyo vipengele vingi ambavyo vingepaswa kuthibitishwa na kujaribiwa. Ninamshukuru Bwana kwamba Wazee Smith, Amadon, na Batchellor bado wanaishi. Walijumuisha washiriki wa familia yetu katika sehemu ngumu zaidi za historia yetu}.}[Ms54-1890.7; 1890][https://egwwritings.org/read?panels=p7213.16]



Kulingana na nukuu hii, ni nani aliweka kanuni za msingi?


\egwinline{\textbf{\underline{Tumesimama bega kwa bega na Mzee Smith katika kazi hii wakati Bwana alipokuwa akiweka kanuni za msingi}}.} \textbf{Ilikuwa ni Bwana}! Lakini ni nani aliyeziandika kama tamko la imani yetu? Ilikuwa Mzee Smith pamoja na James White na Dada White; tunaona kwamba pale Dada White anaposema\egwinline{\textbf{sisi} tulisimama bega kwa bega na Mzee Smith}. Hili \textit{‘sisi’} limefafanuliwa katika aya iliyotangulia: \egwinline{Yeye \normaltext{[Mzee Smith]} anaelewa jinsi\textbf{ sisi—mume wangu pamoja nami}—tulivyoendeleza kazi}. Kwa nukuu hii, Dada White alihusika waziwazi wakati ambapo Bwana alikuwa anaweka \emcap{Kanuni za Msingi}.


Ni kweli kwamba Tamko la \emcap{Kanuni za Msingi} liliandikwa na kikundi kidogo cha watu, yaani Mzee Smith, James White na Ellen White, lakini walijitahidi kuweka kwa maneno yale yaliyokuwa maoni ya kweli ya shirika zima la kanisa. Waliwakilisha kwa usahihi \emcap{kanuni za msingi}—kweli zilizopokelewa mwanzoni mwa kazi yetu. Kama hiyo si kweli, basi tamko hili ni kinyume kabisa na kile kinachodai kuwa. Ziliandikwa \others{ili kujibu maswali} kuhusu kile kilichoaminiwa na Waadventista Wasabato, \others{kusahihisha taarifa za uwongo zilizosambazwa} na \others{kuondoa hisia zenye makosa.}[FP1872 3.1; 1872][https://egwwritings.org/read?panels=p928.8] Iwapo hati hii iliwakilisha vibaya msimamo wa Waadventista, kwa nini uchapishaji wake wa mara kwa mara, katika kipindi cha miaka 42, uliruhusiwa? Ilichapishwa tena hadi kifo cha Ellen White. Ikiwa hati hii iliwakilisha vibaya msimamo wa kanisa, je Ellen White hangepaza sauti yake dhidi yake? Kila mara alipaza sauti yake dhidi ya upotoshaji wa nafasi ya Waadventista Wasabato, kama alivyofanya na D. M. Canright na Dk. Kellogg. Ikiwa \emcap{Kanuni za Msingi} zilikuwa zinapotosha msimamo wa Waadventista Wasabato, basi uchapishaji wote uliofuata unapaswa kuhusishwa na nadharia ya njama. Hiyo itakuwa nadharia kuu ya njama ndani ya Kanisa la Waadventista Wasabato. Kilichokuweko. Maelewano kati ya maandishi ya Ellen White, waanzilishi wa Kiadventista, na madai yaliyotolewa katika Tamko la \emcap{Kanuni za Msingi}, yanashuhudia ukweli kwamba tamko hili ni \others{muhtasari sahihi wa sifa kuu za} Waadventista Wasabato \others{imani, ambayo kuhusu kwayo kunao, kama tujuavyo, umoja kamili mwilini mzima}[The preface of the Fundamental Principles 1889].


Wakati Dada White alipofariki mnamo 1915, uchapishaji wa \emcap{Kanuni za Msingi} ulikoma. Kutoka 1915 na kuendelea, Kitabu cha Mwaka hakikuchapisha taarifa yoyote ya imani hadi 1931. Kwa wakati huu, \emcap{Kanuni za Msingi} zilipokea mabadiliko makubwa. Kwa mara ya kwanza, Utatu uliingizwa katika \emcap{kanuni za msingi}. Katika pointi 2 na 3 tunasoma:


\others{2. \textbf{Kwamba Uungu, au Utatu, unajumuisha Baba wa Milele, \underline{huluki wa kibinafsi, wa kiroho}}, muweza wa yote, \textbf{\underline{aliye kila mahali}}, mjuzi wa yote, asiye na kikomo katika hekima na upendo; \textbf{Bwana Yesu Kristo, Mwana wa Baba wa Milele}, \textbf{ambaye kupitia kwake vitu vyote viliumbwa} na ambaye kwayo wokovu wa majeshi waliokombolewa utatimizwa; \textbf{Roho Mtakatifu, nafsi ya tatu ya Uungu}, ambako ni nguvu kuu ya kuzaliwa upya katika kazi ya ukombozi. Mt. 28:19}


\others{3. \textbf{Kwamba Yesu Kristo ni Mungu hakika, kuwa wa asili na kiini sawa na Baba wa Milele}…}[Yearbook of the Seventh-day Adventist Denomination, 1931, page. 377][https://static1.squarespace.com/static/554c4998e4b04e89ea0c4073/t/59d17eec12abd9c6194cd26d/1506901758727/SDA-YB1931-22+\%28P.+377-380\%29.pdf]


Mabadiliko haya, kwa ajili ya Utatu, yalionekana miaka kumi na sita baada ya kifo cha Dada White. Ulinganisho wa kauli hii na \emcap{Kanuni za Msingi} za awali unawasilisha tofauti kadhaa za kushangaza. Baba bado ni Mtu wa kibinafsi, wa kiroho, Muumba wa vitu vyote, lakini hazungumzwi tena kama “\textit{Mungu mmoja}”. Yesu Kristo bado ni Mwana wa Baba wa Milele, ambaye kwa yeye Baba aliumba vitu vyote; Yesu, pia, ni wa asili ile ile na asili ya Baba. Ingawa haya yalikuwa maneno yale yale ya kuelezea mafundisho juu ya \emcap{Umbile la Mungu} katika \emcap{Kanuni za Msingi} za awali, tunajiuliza maana ya neno “\textit{huluki wa kibinafsi, wa kiroho}” linalotumika kwa Baba, ikiwa Yeye, kwa kauli mpya, yuko kila mahali kwa nafsi yake? Roho Mtakatifu si chombo, au njia ya uwepo wa Baba kila mahali tena. Ingawa kauli hii inatumia maneno yanayofanana na yale ya \emcap{Kanuni za Msingi} za awali, inajitenga na mafundisho ya awali juu ya uwepo na \emcap{Umbile la Mungu}.


Kulingana na LeRoy Froom, taarifa hii iliandikwa kabisa na Francis Wilcox, pamoja na idhini ya ndugu wengine watatu (C.H. Watson, M.E. Kern na E.R. Palmer).\footnote{Edwin Froom, LeRoy. Movement of Destiny. 1971., p. 411, 413, 414} Katika karatasi ambayo haijachapishwa ya \textit{The Seventh-day Adventist Church in Mission: 1919-1979}, tunasoma jinsi Mzee Wilcox alivyotoa kauli hii kinyume na imani ya shirika la kanisa na kuichapisha bila idhini yao.


\others{\textbf{Kwa kutambua kwamba Kamati ya Konferensi Kuu au baraza lingine lolote la kanisa halingeikubali hati hiyo katika namna ambayo iliandikwa}, Mzee Wilcox, kwa ujuzi kamilifu wa kikundi \normaltext{[C.H. Watson, M.E. Kern na E.R. Palmer]}, walimkabidhi Taarifa moja kwa moja kwa Edson Rogers, mwanatakwimu wa Mkutano Mkuu, ambaye aliichapisha katika chapa ya 1931 ya Yearbook, ambapo imeonekana tangu wakati huo. Ilikuwa bila idhini rasmi ya Kamati ya Konferensi Kuu, kwa hiyo, bila upokezi wowote wa kirasmi wa kidhehebu, kwamba kauli ya Mzee Wilcox ikawa tangazo lililokubaliwa la imani yetu.}[Dwyer, Bonnie. “A New Statement of Fundamental Beliefs (1980) - Spectrum Magazine.” \textit{Spectrum Magazine}, 7 June 2009, \href{https://spectrummagazine.org/news/new-statement-fundamental-beliefs-1980/}{spectrummagazine.org/news/new-statement-fundamental-beliefs-1980/}. Accessed 30 Jan. 2025.]


Mnamo 1980, mabadiliko ya mwisho ya muhtasari wa umma wa imani ya Waadventista Wasabato yalifanywa. Mkutano Mkuu ulipiga kura kupitisha taarifa rasmi ya leo:


\others{\textbf{Kuna Mungu mmoja: Baba, Mwana na Roho Mtakatifu, umoja wa Nafsi tatu za milele}. Mungu ni asiyeweza kufa, mwenye uwezo wote, anayejua yote, juu ya yote, na \textbf{aliyepo mahali popote}. Asiye na kikomo vilevile Yeye ni zaidi ya ufahamu wa kibinadamu, ilhali Anajulikana kupitia ufunuo Wake binafsi. Anastahili milele kuabudiwa, kuhimidiwa, na utumishu wa viumbe vyote.}[Seventh-day Adventists Believe: A Biblical Exposition of 27 Fundamental Doctrines, p. 16]


Katika muhtasari huu mfupi wa kihistoria tunaona kwamba taarifa ya 1931 ni “hatua ya kati” kati ya imani ya awali ya Waadventista hadi imani kamili ya utatu.


Mabadiliko katika imani yetu yametokea baada ya muda kwa mazungumzo mengi. Historia yetu ya Waadventista imeacha alama ya mabadiliko haya. Ikiwa sisi ni watafuta ukweli waaminifu tunapaswa kujifunza jambo hili kwa undani. Je, tunaweza kuona, katika historia yetu ya Waadventista, kwa nini tumeiacha pointi ya kwanza la \emcap{Kanuni za Msingi} na kuliunga mkono fundisho la Utatu? Hakika! Ndani ya utafiti zifuatazo tutaangalia baadhi ya nyaraka za kihistoria zinazoonyesha kwa nini tumejihamisha kutoka kipengele cha kwanza ya \emcap{Kanuni za Msingi}, iliyoshikiliwa katika miaka ya awali, kukubali fundisho la Utatu. Wakati wa masomo haya, tunakuagiza kutathmini kwa maombi mabadiliko haya na imani yako mwenyewe.



% The authority of the Fundamental Principles

\begin{titledpoem}
\stanza{
    Our principles of faith stand firm and true, \\
    Established by the Lord through chosen few. \\
    A platform built on unquestionable might, \\
    Waymarks that guide us through the darkest night.
}

\stanza{
    The pioneers sought truth with earnest prayer, \\
    Point after point laid down with godly care. \\
    Yet modern minds have altered what was clear, \\
    Changing foundations held for many a year.
}

\stanza{
    Return, O church, to truths that God ordained, \\
    Not to revised beliefs that men have claimed. \\
    Stand firm upon the rock that cannot move, \\
    In Fundamental Principles approved.
}

\stanza{
    Let not new scholars lead your faith astray, \\
    From paths our founders walked in heaven's way. \\
    The Lord Himself laid down these truths of old, \\
    Embrace their power with faith both strong and bold.
}
\end{titledpoem}
