\qrchapter{https://forgottenpillar.com/rsc/en-fp-chapter5}{The patchwork theories - Lt253-1903}


\qrchapter{https://forgottenpillar.com/rsc/en-fp-chapter5}{Nadharia za kiraka - Lt253-1903}


\egw{Dear Brother,—}


\egw{Ndugu Mpendwa,—}


\egwnogap{\textbf{I must tell you that your ideas in regard to some things \underline{have been decidedly wrong}.} I would that you could see your errors. \textbf{The book Living Temple \underline{is not to be patched up}, a few changes made in it, and then advertised and praised as a valuable production}. It would be better to present the physiological parts in another book under another title. \textbf{When you wrote that book}, \textbf{you were not under the inspiration of God}. There was by your side the one who inspired Adam to look at God in a false light. Your whole heart needs to be changed, thoroughly and entirely cleansed.}[Lt253-1903.1; 1903][https://egwwritings.org/read?panels=p9980.7]


\egwnogap{\textbf{Lazima nikuambie kwamba mawazo yako kuhusu baadhi ya mambo \underline{yamepotoka pakubwa kwa hakika}.} Ningependa kwamba upate kuona dosari zako. \textbf{Kitabu Living Temple \underline{hakipaswi kuwekwa viraka}, mabadiliko machache yafanywe ndani yake, na kisha kutangazwa na kusifiwa kama uzalishaji wa thamani}. Ingekuwa bora kuwasilisha sehemu za kifiziolojia katika kitabu kingine chini ya kichwa kingine. \textbf{Wakati ulipoandika kitabu hicho}, \textbf{hukuwa chini ya mwongozo wa Mungu}. Kando yako palikuwepo yule aliyemwongoza Adamu kuwa na mtazamo kuhusu Mungu kwa nuru iliyopotosha. Moyo wako wote unahitaji kubadilishwa, kusafishwa kabisa na kwa ukamilifu.}[Lt253-1903.1; 1903][https://egwwritings.org/read?panels=p9980.7]


\egwnogap{\textbf{My brother, do not allow yourself to be alienated from your ministering brethren who tell you of your dangers. Those who faithfully and frankly tell you of your errors are your best friends.} I am sorry, very sorry, for your medical associates. They have been unfaithful to God and untrue to you in failing to tell you kindly but firmly where you were not working righteously.}[Lt253-1903.2; 1903][https://egwwritings.org/read?panels=p9980.8]


\egwnogap{\textbf{Ndugu yangu, usikubali kutengwa na ndugu zako wahudumu ambao hukuonyesha hatari zako. Wale wanaokuambia kwa uaminifu na kwa uwazi makosa yako ndio marafiki zako wa dhati.} Ninasikitika, nimehuzunishwa kwa kina, kuhusu washirika wako wa matibabu. Wamekuwa wasio waaminifu kwa Mungu na wasio wa kweli kwako kwa kushindwa kukuambia kwa fadhili lakini kwa uthabiti mahali ambapo haukuwa ukifanya kazi kwa uadilifu.}[Lt253-1903.2; 1903][https://egwwritings.org/read?panels=p9980.8]


\egwnogap{There are many things that you must overcome before you can be saved. In the heart that is not led by God, there is a something that leads it to desire to be sustained in its wrong course. The men who faithfully tell you the truth, pointing out your mistakes, you have regarded as your enemies. But often they are your best friends and, in telling you wherein you were walking in strange paths, were doing a very disagreeable duty. The Lord’s servants are not to flatter your pride; they are not to stand silent, fearing to say, ‘Why do ye thus?’ They are faithfully to warn you of your danger.}[Lt253-1903.3; 1903][https://egwwritings.org/read?panels=p9980.9]


\egwnogap{Kuna mambo mengi ambayo lazima uyashinde kabla ya kuokoka. Katika moyo ambao hauongozwi na Mungu, kuna kitu kinachoufanya utamani kudumishwa katika mwenendo wake usio sahihi. Watu wanaokuambia ukweli kwa uaminifu, wakionyesha makosa yako, umewachukulia kama adui zako. Lakini mara nyingi wao ni marafiki zako wa dhati na, katika kukuambia mahali ulikuwa ukitembea katika njia za ajabu, walikuwa wanafanya wajibu usiofurahisha. Watumishi wa Bwana hawapaswi kukuza kiburi chako; hawapaswi kusimama kimya, wakiogopa kusema, ‘Kwa nini unafanya hivi?’ Wanapaswa kwa uaminifu kukuonya kuhusu hatari yako.}[Lt253-1903.3; 1903][https://egwwritings.org/read?panels=p9980.9]


\egwnogap{\textbf{My husband, Elder Joseph Bates, Father Pierce, Elder Edson, and many others who were keen, noble, and true were among those who, after the passing of the time in 1844, searched for truth}. \textbf{At our important meetings, these men would meet together and search for the truth as for hidden treasure}. I met with them, and we studied and prayed earnestly; for we felt that we must learn God’s truth. Often we remained together until late at night, and sometimes through the entire night, praying for light and studying the Word. As we fasted and prayed, great power came upon us. But I could not understand the reasoning of the brethren. My mind was locked, as it were, and I could not comprehend what we were studying. Then the Spirit of God would come upon me, I would be taken off in vision, and a clear explanation of the passages we had been studying would be given me with instruction as to the position we were to take regarding truth and duty. Again and again this happened. \textbf{A line of truth extending from that time to the time when we shall enter the city of God was plainly marked out before me}, and I gave my brethren and sisters the instruction that the Lord had given me. They knew that when not in vision, I could not understand these matters, and they accepted as light direct from heaven the revelations given me. \textbf{Thus the leading points of our faith as we hold them today were firmly established}. \textbf{\underline{Point after point} was clearly defined, and all the brethren came into harmony}.}[Lt253-1903.4; 1903][https://egwwritings.org/read?panels=p14068.9980010]


\egwnogap{\textbf{Mume wangu, Mzee Joseph Bates, Baba Pierce, Mzee Edson, na wengine wengi ambao walikuwa makini, waungwana, na wa kweli walikuwa miongoni mwa wale ambao, baada ya kupita kwa wakati mnamo 1844, walitafuta ukweli}. \textbf{Katika mikutano yetu muhimu, wanaume hawa walikutana pamoja na kutafuta ukweli kama hazina iliyofichwa}. Nilikutana nao, tukasoma na kuomba kwa bidii; kwa maana tulihisi kwamba ni lazima tujifunze ukweli wa Mungu. Mara nyingi tulibaki pamoja hadi usiku sana, na wakati mwingine usiku kucha, tukiomba kwa ajili ya nuru na kujifunza Neno. Tulipofunga na kuomba, nguvu kuu ilitujia. Lakini sikuweza kuelewa hoja za ndugu. Akili yangu ilikuwa imefungwa, kama ilivyokuwa, na sikuweza kuelewa tulichokuwa tunajifunza. Ndipo Roho wa Mungu angenijia, ningechukuliwa katika maono, na maelezo wazi ya vifungu tulivyokuwa tukijifunza ningepewa pamoja na maelekezo kuhusu msimamo tunaopaswa kuchukua kuhusu ukweli na wajibu. Tena na tena hili lilitokea. \textbf{Mstari wa ukweli unaoenea kutoka wakati huo hadi wakati ambapo tutaingia mji wa Mungu uliwekwa wazi mbele yangu}, na nikawapa ndugu na dada zangu maagizo ambayo Bwana alinipa. Walijua kwamba wakati siko katika maono, sikuweza kuelewa mambo haya, na walikubali kama nuru ya moja kwa moja kutoka mbinguni mafunuo niliyopewa. \textbf{Hivyo pointi kuu za imani yetu tunavyozishikilia leo zilidhibitishwa imara}. \textbf{\underline{Hoja baada ya hoja} ilifafanuliwa wazi, na ndugu wote wakaingia katika maelewano}.}[Lt253-1903.4; 1903][https://egwwritings.org/read?panels=p14068.9980010]


\egwnogap{\textbf{The whole company of believers were united in the truth}. \textbf{There were those who came in with strange doctrines, but we were never afraid to meet them. Our experience was wonderfully established by the revelations of the Holy Spirit}.}[Lt253-1903.5; 1903][https://egwwritings.org/read?panels=p9980.11]


\egwnogap{\textbf{Kundi zima la waumini liliunganishwa katika ukweli}. \textbf{Kulikuwa na wale waliokuja na mafundisho ya ajabu, lakini hatukuogopa kukutana nao. Uzoefu wetu ulidhibitishwa kwa ajabu na mafunuo ya Roho Mtakatifu}.}[Lt253-1903.5; 1903][https://egwwritings.org/read?panels=p9980.11]


\egwnogap{For two or three years my mind continued to be locked to the Scriptures. In 1846 I was married to Elder James White. It was some time after my second son was born that we were in great perplexity regarding certain points of doctrine. I was praying to the Lord to unlock my mind, that I might understand His Word. Suddenly I seemed to be enshrouded in clear, beautiful light, and ever since, \textbf{the Scriptures have been an open book to me}.}[Lt253-1903.6; 1903][https://egwwritings.org/read?panels=p14068.9980012]


\egwnogap{Kwa miaka miwili au mitatu akili yangu iliendelea kufungwa kwa Maandiko. Mnamo 1846 niliolewa na Mzee James White. Ilikuwa muda fulani baada ya mtoto wangu wa pili kuzaliwa tulipokuwa katika mkanganyiko mkubwa kuhusu mambo fulani ya mafundisho. Nilikuwa nikimwomba Bwana afungue akili yangu, ili niweze kuelewa Neno Lake. Ghafla nilionekana kufunikwa na nuru nzuri, ya wazi, na tangu wakati huo, \textbf{Maandiko yamekuwa kitabu wazi kwangu}.}[Lt253-1903.6; 1903][https://egwwritings.org/read?panels=p14068.9980012]


\egwnogap{I was at that time in Paris, Maine. Old Father Andrews was very sick. For some time he had been a great sufferer from inflammatory rheumatism. He could not move without intense pain. We prayed for him. I laid my hands on his head, and said, “Father Andrews, the Lord Jesus maketh thee whole.” He was healed instantly. He got up and walked about the room, praising God, and saying, “I never saw it on this wise before. Angels of God are in this room.” The glory of God was revealed. \textbf{Light seemed to shine all through the house, and an angel’s hand was laid upon my head. From that time to this I have been able to understand the Word of God.}}[Lt253-1903.7; 1903][https://egwwritings.org/read?panels=p9980.13]


\egwnogap{Wakati huo nilikuwa Paris, Maine. Baba Andrews alikuwa mgonjwa sana. Kwa muda alikuwa akiteseka sana kutokana na ugonjwa wa “inflammatory rheumatism”. Hakuweza kusogea bila maumivu makali. Tulimwombea. Niliweka mikono yangu juu ya kichwa chake, na kusema, “Baba Andrews, Bwana Yesu anakuponya.” Aliponywa papo hapo. Akainuka na kutembea chumbani, akimsifu Mungu, na kusema, “Sijawahi kuiona kwa namna hii hapo awali. Malaika wa Mungu wako katika chumba hili.” Utukufu wa Mungu ulifunuliwa. \textbf{Nuru ilionekana kuangaza nyumba nzima, na mkono wa malaika uliwekwa juu ya kichwa changu. Tangu wakati huo hadi sasa nimeweza kuelewa Neno la Mungu.}}[Lt253-1903.7; 1903][https://egwwritings.org/read?panels=p9980.13]


\egwnogap{\textbf{After the passing of the time, we were opposed and cruelly falsified. Erroneous theories were pressed in upon us by men and women who had gone into fanaticism}. I was directed to go to the places where these people were advocating these erroneous theories, and as I went, the power of the Spirit was wonderfully displayed in rebuking the errors that were creeping in. \textbf{\underline{Satan himself, in the person of a man}, was working to make of no effect my testimony regarding the position that we now know to be substantiated by Scripture.}}[Lt253-1903.8; 1903][https://egwwritings.org/read?panels=p9980.14]


\egwnogap{\textbf{Baada ya wakati kupita, tulipingwa na kutafsiriwa vibaya kwa ukatili. Nadharia za uongo zilisongwa ndani yetu na wanaume na wanawake ambao walikuwa wameingia katika ushupavu}. Nilielekezwa kwenda mahali ambapo watu hawa walikuwa wakitetea nadharia hizi za uongo, na nilipokwenda, nguvu za Roho zilionyeshwa kwa ajabu katika kukemea makosa yaliyokuwa yakianza kuingia. \textbf{\underline{Shetani mwenyewe, katika utu wa mwanadamu}, alikuwa akifanya kazi ili kufanya ushuhuda wangu usiwe na athari kuhusu msimamo ambao sasa tunajua unathibitishwa na Maandiko.}}[Lt253-1903.8; 1903][https://egwwritings.org/read?panels=p9980.14]


\egwnogap{\textbf{Just such theories as you have presented in Living Temple were presented then}. \textbf{These subtle, deceiving sophistries have again and again sought to find place amongst us. \underline{But I have ever had the same testimony to bear which I now bear regarding the personality of God}}.}[Lt253-1903.9; 1903][https://egwwritings.org/read?panels=p9980.15]


\egwnogap{\textbf{Nadharia kama hizo kama ulizowasilisha katika Living Temple ziliwasilishwa wakati huo. Mafundisho haya yenye hila na yenye kudanganya yametafuta tena na tena kupata mahali kati yetu. \underline{Lakini ninao ushuhuda uleule wa kutoa ambao sasa ninautoa kuhusu Umbile la Mungu}}.}[Lt253-1903.9; 1903][https://egwwritings.org/read?panels=p9980.15]


\egwnogap{In (Early Writings, 60, 66, 67)\footnote{It appears that the pages are incorrect. The mentioned paragraphs can be found in Early Writings on pages \href{https://egwwritings.org/read?panels=p28.462&index=0}{70.2}, \href{https://egwwritings.org/read?panels=p28.490&index=0}{77}, and \href{https://egwwritings.org/read?panels=p28.390&index=0}{54.2}.}, are the following statements:}[Lt253-1903.10; 1903][https://egwwritings.org/read?panels=p9980.16]


\egwnogap{Katika (Maandiko ya Awali, 60, 66, 67)\footnote{Inaonekana kwamba kurasa hizo si sahihi. Aya zilizotajwa zinaweza kupatikana katika Maandiko ya Awali kwenye kurasa \href{https://egwwritings.org/read?panels=p28.462&index=0}{70.2}, \href{https://egwwritings.org/read?panels=p28.490&index=0}{77}, na \href{https://egwwritings.org/read?panels=p28.390&index=0}{54.2}.}, kuna taarifa zifuatazo:}[Lt253-1903.10; 1903][https://egwwritings.org/read?panels=p9980.16]


\egwnogap{‘May 14, 1851, I saw the beauty and loveliness of Jesus. As I beheld His glory, the thought did not occur to me that I should ever be separated from His presence. \textbf{I saw a light coming from the glory that encircled the Father}, and as it approached near to me, my body shook and trembled like a leaf. I thought that if it should come near me, I would be struck out of existence; but the light passed me. \textbf{Then could I have some sense of the great and terrible \underline{God} with whom we have to do}.’}[Lt253-1903.11; 1903][https://egwwritings.org/read?panels=p9980.17]


\egwnogap{‘Mei 14, 1851, niliona uzuri na raghba ya Yesu. Nilipouona utukufu Wake, wazo haikunijia kwamba ningetengwa na uwepo wake. \textbf{Niliona mwanga unanurishwa kutoka kwa utukufu uliomzunguka Baba}, na ulipokaribia nami, mwili wangu ulitetemeka na kuteterekeka kama jani. Nilifikiri kwamba ikiwa ingekuja karibu nami, ningetolewa uhai; lakini nuru ilinipita. \textbf{Basi ndipo niliweza kuwa na hisia ilhali kwa kadri ya Ukuu na Utishi wa \underline{Mungu} ambaye tunapaswa kufanya naye}.’}[Lt253-1903.11; 1903][https://egwwritings.org/read?panels=p9980.17]


\egwnogap{‘I have often seen \textbf{the lovely Jesus, that He is a person}. \textbf{I asked Him if His Father was a person, and had \underline{a form} like Himself}. Said Jesus, ‘\textbf{I am the express image of My Father’s person!}’ [Hebrews 1:3.]}[Lt253-1903.12; 1903][https://egwwritings.org/read?panels=p9980.18]


\egwnogap{‘\textbf{Mara nyingi nimemwona Yesu mpendwa, kwamba Yeye ni Nafsi}. \textbf{Nilimuuliza kama Baba Yake alikuwa Nafsi, na alikuwa na \underline{umbo} kama Yeye Mwenyewe}. Yesu alisema, ‘\textbf{Mimi ni chapa kamili ya Umbile Wake}!’ [Waebrania 1:3.]}[Lt253-1903.12; 1903][https://egwwritings.org/read?panels=p9980.18]


\egwnogap{‘\textbf{I have often seen that the spiritual view took away all the glory of heaven, and that in many minds the throne of David and the lovely person of Jesus have been burned up in the fire of spiritualism}. I have seen that some who have been deceived and led into this error, will be brought out into the light of truth, \textbf{but it will be almost impossible for them to get entirely rid of the deceptive power of spiritualism. Such should make thorough work in confessing their errors, and leaving them forever}.’}[Lt253-1903.13; 1903][https://egwwritings.org/read?panels=p9980.19]


\egwnogap{‘\textbf{Mara nyingi nimeona kwamba mtazamo wa umizimu uliondoa utukufu wote wa mbinguni, na katika fikira za waja ainati kiti cha enzi cha Daudi na nafsi ya kuvutia ya Yesu vimeteketezwa katika moto wa umizimu}. Nimeona kwamba baadhi wamedanganywa na kuongozwa katika hili makosa, watadumishwa katika nuru ya ukweli, \textbf{lakini itakuwa muhali hata kuwe kutokuwezekana kwao kuondoa kabisa nguvu za udanganyifu za umizimu. Wao wanapaswa kwa juhudi za kina kuungama makosa yao, na kuyaacha milele}.’}[Lt253-1903.13; 1903][https://egwwritings.org/read?panels=p9980.19]


\egwnogap{\textbf{There is a strain of spiritualism \underline{coming in} among our people, and \underline{it will undermine the faith} of those who give place to it, leading them to give heed to seducing spirits and doctrines of devils}. Errors will be presented in a pleasing and flattering manner. The enemy desires to divert the minds of our brethren and sisters from the work of preparing a people to stand in these last days.}[Lt253-1903.14; 1903][https://egwwritings.org/read?panels=p9980.21]


\egwnogap{\textbf{Kuna aina ya umizimu \underline{inayoingia} miongoni mwa watu wetu, na \underline{itadhoofisha imani} ya wale wanaoipa nafasi, inawaongoza kuzitii roho zidanganyazo pamoja na mafundisho ya mashetani}. Makosa yatawasilishwa kwa njia ya kupendeza na ya kufurahisha. Adui anatamani kugeuza mawazo ya kaka na dada zetu kutoka kwa kazi ya kuandaa watu kusimama katika siku hizi za mwisho.}[Lt253-1903.14; 1903][https://egwwritings.org/read?panels=p9980.21]


\egwnogap{I am instructed to warn our brethren and sisters \textbf{not to discuss the nature of our God}. Many of the curious who attempted to open the ark of the testament, to see what was inside, were punished for their presumption. \textbf{We are not to say that the Lord God of heaven is in a leaf, or in a tree; for He is not there. \underline{He sitteth upon His throne in the heavens}.}}[Lt253-1903.15; 1903][https://egwwritings.org/read?panels=p9980.22]


\egwnogap{Nimeagizwa kuwaonya ndugu na dada zetu \textbf{wasijadili asili ya Mungu wetu}. Wengi wa wale wenye tamaa na mwasho walipojaribu kulifungua sanduku la agano, waone kilichokuwa ndani, waliadhibiwa kulingana na ghururi lao. \textbf{Hatupaswi kusema kwamba Bwana Mungu wa mbinguni yuko ndani ya jani, au kwenye mti; kwa maana hayupo papo. \underline{Ameketi juu ya kiti chake cha enzi mbinguni}}.}[Lt253-1903.15; 1903][https://egwwritings.org/read?panels=p9980.22]


\egwnogap{The work of the Creator as seen in nature reveals His power. But nature is not above God, nor is God in nature as some represent Him to be. God made the world, but the world is not God; it is but the work of His hands. \textbf{Nature reveals the work of a positive, \underline{personal God}, showing that God is, and that He is a rewarder of those who diligently seek Him}.}[Lt253-1903.16, 1903][https://egwwritings.org/read?panels=p9980.23]


\egwnogap{Kazi ya Muumba inavyoonekana katika maumbile hufichua uwezo Wake. Lakini asili haichukui nafasi ya Mungu, wala Mungu hayupo katika asili jinsi wengine wanavyomwakilisha. Mungu aliumba ulimwengu, lakini hata hivyo ulimwengu sio Mungu; ila ni kazi ya mikono yake. \textbf{Asili hufichua kazi ya Mungu aliye chanya na \underline{wa kibinafsi}, kuonyesha kwamba Mungu yuko, na kwamba Yeye ni mthawabishaji wa wale wanaomtafuta kwa bidii}.}[Lt253-1903.16, 1903][https://egwwritings.org/read?panels=p9980.23]


\egwnogap{I could say much regarding the sanctuary; the ark containing the law of God; the cover of the ark, which is the mercy seat; the angels at either end of the ark; and other things connected with the heavenly sanctuary and with the great day of atonement. I could say much regarding the mysteries of heaven; but my lips are closed. I have no inclination to try to describe them.}[Lt253-1903.17; 1903][https://egwwritings.org/read?panels=p9980.25]


\egwnogap{Ningeweza kusema mengi kuhusu patakatifu; sanduku lenye sheria ya Mungu; kifuniko cha sanduku, ambalo ni kiti cha rehema; malaika katika pande za safina; na mambo mengine kuunganishwa na patakatifu pa mbinguni na siku kuu ya upatanisho. Ningeweza kusema mengi kuhusu mafumbo ya mbinguni; lakini midomo yangu imefungwa. Sina mapendekezo kujaribu kueleza kuhusu kwayo.}[Lt253-1903.17; 1903][https://egwwritings.org/read?panels=p9980.25]


\egwnogap{\textbf{I would not dare to speak of God as you have spoken of Him}. He is high and lifted up, and His glory fills the heavens. “The voice of the Lord is mighty; it shaketh the cedars of Lebanon. \textbf{The Lord is in His holy temple}; let all the earth keep silence before Him.” [See Psalm 29:5; Habakkuk 2:20.]}[Lt253-1903.18; 1903][https://egwwritings.org/read?panels=p9980.26]


\egwnogap{\textbf{Singethubutu kusema kuhusu Mungu kama ulivyonena juu Yake}. Yuko juu na ameinuliwa, na utukufu wake unazijaza mbingu. “Sauti ya Bwana ina nguvu; inatikisa mierezi ya Lebanon. \textbf{Bwana yu katika hekalu lake takatifu}; dunia yote na ikae kimya mbele zake.” [Ona Zaburi 29:5; Habakuki 2:20.]}[Lt253-1903.18; 1903][https://egwwritings.org/read?panels=p9980.26]


\egwnogap{\textbf{My brother, when you are tempted to speak of God, \underline{where He is, or what He is}, remember that on this point silence is eloquence}. Take off your shoes from off your feet; for the ground on which you are placing your careless, unsanctified feet is holy ground.}[Lt253-1903.19; 1903][https://egwwritings.org/read?panels=p14068.9980027]


\egwnogap{\textbf{Ndugu yangu, unapojaribiwa kusema kuhusu Mungu, \underline{mahali alipo, au kile Alicho}, kumbuka kwamba katika hatua hii ukimya ni ufasaha. Vua viatu vyako kutoka kwa miguu yako; kwa maana ardhi hiyo unayoiweka miguu yako ya kutokumakinika na isiyotakaswa, ni ardhi palipo takatifu.}}[Lt253-1903.19; 1903][https://egwwritings.org/read?panels=p14068.9980027]


\egwnogap{\textbf{I am instructed to say that there is nothing in the Word of God to substantiate your spiritualistic theories. Will you not renounce these theories at once? Upon them your mind has been dwelling for a long time, but they have had no sanctifying, refining, ennobling influence upon your life. The Lord has no use for these theories, and He would not have His people vindicate or propagate them.}}[Lt253-1903.20; 1903][https://egwwritings.org/read?panels=p9980.28]


\egwnogap{\textbf{Nimeagizwa kusema kwamba hakuna kitu katika Neno la Mungu kuthibitisha nadharia zako za umizimu. Je, hutazikana nadharia hizi mara moja? Nadharia hizi zimekaa mawazoni mwako kwa muda mrefu, lakini hazijakuwa zenye utakaso, usafishaji, zenye ushawishi wa kuboresha maishani mwako. Bwana hana matumizi kwa nadharia hizi, na Yeye asingetaka watu wake wazitetee au kuzieneza.}}[Lt253-1903.20; 1903][https://egwwritings.org/read?panels=p9980.28]


\egwnogap{\textbf{The Father, the omniscient One, created the world \underline{through} Christ Jesus}. Christ is the light of the world, the way to eternal life. He, the anointed One, God gave to make an atonement for the sins of the world. You need to understand that unless you believe \textbf{in that atonement}, and know that you are bought with the price of the blood of \textbf{the only begotten Son of God}, you will assuredly be bound up with the wicked one. \textbf{If you continue to cherish the theories that you have been cherishing, you will be left to become the sport of Satan’s temptations}. He is playing the game of life for your soul. Remain for a little longer linked up with him, and be assured that you will lose your soul.}[Lt253-1903.21; 1903][https://egwwritings.org/read?panels=p9980.29]


\egwnogap{\textbf{Baba, Mjuzi wa yote, aliumba ulimwengu \underline{kupitia} Kristo Yesu}. Kristo ndiye nuru ya ulimwengu, njia ya uzima wa milele. Yeye, aliyetiwa mafuta, Mungu alimtoa kufanya upatanisho kwa ajili ya dhambi za ulimwengu. Unahitaji kuelewa kwamba usipokuwa \textbf{mwaamini katika upatanisho huo}, na ujue ya kuwa umenunuliwa kwa bei ya damu ya \textbf{Mwana wa Mungu mzaliwa wa kipekee}, hakika utajumuishwa na yule mwovu. \textbf{Ukiendelea kuthamini nadharia ambazo umekuwa ukithamini, utaachwa kuchezeshwa kwa Majaribu ya Shetani}. Anacheza mchezo wa maisha kwa nafsi yako. Zidi katika utangamano naye kwa muda kidogo tu, na uwe na hakika kwamba utaipoteza nafsi yako.}[Lt253-1903.21; 1903][https://egwwritings.org/read?panels=p9980.29]


\egwnogap{By declaring that our institutions are undenominational, you have put our people and our work in a false position. You have been led over a terrible path, the dangers of which you have not known, but may sometime see. It is not yet too late for wrongs to be righted. There is hope for you. \textbf{You have followed the enemy step by step, striving to look into mysteries too high and holy for your comprehension}. \textbf{Then in your teaching the Holy One has been brought down to man’s \underline{scientific, spiritualistic ideas}}. You have been walking in crooked paths. You have lost the moral image of God. But there is hope for you. You may still turn your feet into the right path. Will you not now make straight paths for your feet, lest the lame be turned out of the way? Will you now refuse to sow one more seed of skepticism and sophistry in the minds of others? Will you now come to Christ and be healed?}[Lt253-1903.22; 1903][https://egwwritings.org/read?panels=p14068.9980030]


\egwnogap{Kwa kutangaza kwamba taasisi zetu zimetenganishwa na madhehebu zetu, umeweka watu wetu na kazi yetu katika nafasi ya uongo. Umeongozwa kwenye njia mbaya, hatari ambayo kwayo wewe hukuyang'amua, lakini wakati mwingine unaweza kuona. Bado hatujachelewa kwa makosa kusahihishwa. Panapo matumaini kwako. \textbf{Umemfuata adui hatua kwa hatua, ukijitahidi kuangalia ndani siri zilizo juu sana na takatifu wa kupindukia kupiku ufahamu wako}. \textbf{Kisha katika mafundisho yako Nafsi Yule Mtakatifu ameshushwa hadi hadhi ya mawazo ya mwanadamu ya \underline{kisayansi, ya kimizimu}}. Umekuwa katika kutembea kwa njia potofu. Umepoteza sura ya maadili ya Mungu. Lakini kuna tumaini kwako. Bado unaweza kugeuza miguu yako kuwa njia sahihi. Je! hutafanya sasa njia ziwe zilizonyooka kwa ajili yako? viwete wasije wakapotoshwa? Je, utakataa sasa kupanda mbegu moja zaidi ya mashaka na ujanja katika akili za wengine? Je, utakuja kwa Kristo sasa hivi na kupata uponyaji?}[Lt253-1903.22; 1903][https://egwwritings.org/read?panels=p14068.9980030]


\egwnogap{\textbf{I have hesitated and delayed about the sending out of that which the Spirit of the Lord has impelled me to write}. I did not want to be compelled to present the satanic influence of these sophistries. But unless there is a decided change, in yourself and your associates, I shall have to do this, to save others from following the path that you have been following. I shall have to obey the command given me of God, “\textbf{Meet it}.” This is the only thing that I can do.}[Lt253-1903.23; 1903][https://egwwritings.org/read?panels=p9980.31]


\egwnogap{\textbf{Nimesitasita na kuchelewesha kupeleka yale ambayo Roho wa Mwenyezi Mungu Bwana amenisukuma kuandika}. Sikutaka kulazimishwa kuwasilisha ushawishi wa kishetani wa hizi nadharia za kupotosha. Lakini isipokuwepo mabadiliko ya dhati, ndani yako na washirika wenza, itabidi nifanye hivi, ili kuwaokoa wengine wasifuate njia ambayo umekuwa ukifuata. Itanibidi kutii amri niliyopewa na Mungu, “\textbf{Kutana nayo}.” Hili ndilo jambo pekee ambalo ninaweza kulifanya.}[Lt253-1903.23; 1903][https://egwwritings.org/read?panels=p9980.31]


\egwnogap{I present to you the things that the Lord has presented to me. There is a great work to be done. We are to take hold of the work understandingly, praying, believing, and receiving the Holy Spirit. Thus only can we do the work given us. \textbf{I am required by God to bear testimony against Living Temple}. Whatever your associates may say concerning this book,\textbf{ I take the position now and forever that it is a snare}. \textbf{No union will be formed by our people as a whole upon the \underline{theories} that you have begun to present in that book}. \textbf{You may regard this as forever decided}. \textbf{As a people we shall stand firm \underline{on the platform that has withstood test and trial}. We shall hold to the \underline{sure pillars of our faith}. \underline{The principles of truth} that God has revealed to us are our only foundation. They have made us what we are. These new, fanciful theories are fascinating and misleading. They endanger the eternal interests of the soul. The Scriptures do not sustain them}. Clothed with the Christian armor, shod with the preparation of the gospel of peace, we shall stand \textbf{firm against these misleading theories}. You may turn and wrest the Word of God to your own destruction, but I entreat you not to do this.}[Lt253-1903.24; 1903][https://egwwritings.org/read?panels=p9980.32]


\egwnogap{Ninakuwasilishia mambo ambayo Bwana amenionyesha. Kuna kazi kubwa ambayo yanapaswa kutendeka. Tunapaswa kushika kazi kwa ufahamu, kwa kuomba, kwa kuamini, na kwa kupokea Roho takatifu. Hivyo tu ndivyo tunaweza kufanya kazi tuliyopewa. \textbf{Nimetiwa jukumu na Mungu kuwasilisha ushuhuda dhidi ya Living Temple}. Chochote ambacho washirika wako wanaweza kusema kuhusu hiki kitabu,\textbf{ ninachukua msimamo sasa na milele kwamba ni mtego}. \textbf{Hakuna muungano utakaoundwa na watu wetu kwa ujumla kujiunga na \underline{nadharia} ambazo umeanza kuwasilisha katika kitabu hicho}. \textbf{Unaweza kuzingatia hili kama lililoamuliwa milele}. \textbf{Kama watu tutasimama imara \underline{juu ya jukwaa ambalo limehimili mtihani na majaribio}. Tutashikilia \underline{nguzo za imani yetu}. \underline{Kanuni za ukweli} ambazo Mungu ametufunulia ndio msingi wetu pekee. Zimetukuza kwa jinsi tulivyo sasa. Nadharia hizi mpya, za kudhaniwa tu zinavutia na kupotosha. Zinahatarisha maslahi ya milele ya nafsi. Maandiko hayazitegemezi}. Tukiwa tumevaa silaha za Kikristo, tukiwa tumevaa matayarisho ya Injili ya amani tutasimama kidete \textbf{dhidi ya nadharia hizi potofu}. Unaweza kugeuka na kulipotosha Neno la Mungu kwa uharibifu wako mwenyewe, lakini nakusihi usifanye hivi.}[Lt253-1903.24; 1903][https://egwwritings.org/read?panels=p9980.32]


\egwnogap{\textbf{Heaven is not a vapor. It is a place}. \textbf{Christ has gone to prepare mansions for those who love Him}, those who, in obedience to His commands, come out from the world and are separate. The principles of heaven must be brought into our experience, that we may be distinguished from the world. \textbf{There must be a marked contrast between us and the world; for we are God’s denominated people}.}[Lt253-1903.25; 1903][https://egwwritings.org/read?panels=p9980.33]


\egwnogap{\textbf{Mbingu si mvuke. Ni mahali}. \textbf{Kristo amekwenda kuandaa makao kwa ajili ya hao wampendao}, wale ambao, kwa kutii amri zake, wanajitenga na ulimwengu na wako watengwa. Kanuni za mbinguni lazima ziletwe katika uzoefu wetu, ili tutofautishwe na ulimwengu. \textbf{Lazima kuwe na tofauti kubwa kati yetu sisi na ulimwengu; kwa maana sisi tu watu wa Mungu walioteuliwa kwenye dhehebu lililo la kipekee}.}[Lt253-1903.25; 1903][https://egwwritings.org/read?panels=p9980.33]


\egwnogap{The Lord has given you an opportunity to make things right. \textbf{I rejoice that you have made a beginning. Do not think that we have no right to try to correct your errors and the results of these errors. As long as God gives me breath, and commissions me to use pen and voice in beating back this evil thing that has come in among us, I shall act my part in the warfare. Ever since I was seventeen years old, I have had to fight this battle against false theories, in defense of the truth}. \textbf{The history of our past experience is indelibly fixed in my mind, and I am determined that \underline{no theories of the order that you have been accepting} shall come into our ranks}. If you refuse to change, and labor to lead your associates after you, and they venture to follow your leading, the accountability rests with you and with them, not on my soul.}[Lt253-1903.26, 1903][https://egwwritings.org/read?panels=p9980.34]


\egwnogap{Bwana amekupa nafasi ya kurekebisha mambo. \textbf{Ninafurahi kwamba umefanya mwanzo. Usifikirie kuwa hatuna haki ya kujaribu kurekebisha makosa yako na matokeo ya makosa haya. Maadamu Mungu ananipa pumzi, na kuniagiza kutumia kalamu na sauti katika kulishambulia uovu hili lililoingia kati yetu, nitatenda majukumu yangu katika vita hivi. Tangu nilipokuwa na umri wa miaka kumi na saba, imenibidi kupigana vita hivi dhidi ya nadharia za uwongo, katika kutetea ukweli}. \textbf{Historia ya uzoefu wetu wa zamani umekita mizizi akilini mwangu, na nimedhamiria kuwa \underline{hakuna nadharia zozote zile sawia na ambazo wewe umekuwa ukizikubali} zitaingia katika safu zetu}. Ikiwa unakataa kubadilika, na kufanya kazi ya kuongoza washirika wako baada yako, nao wakathubutu kufuata uongozi wako, uwajibikaji unabakia pamoja nawe na pamoja nao, si juu ya nafsi yangu.}[Lt253-1903.26, 1903][https://egwwritings.org/read?panels=p9980.34]


\egwnogap{\textbf{I speak decidedly, in order that you may know, that unless there is a decided change in you, there can be no hope of a union between you and those who are holding the beginning of their confidence firm unto the end.} You have made the division. \textbf{\underline{We must stand firm for the truths that the Lord has given us as the pillars of our faith}}.}[Lt253-1903.27; 1903][https://egwwritings.org/read?panels=p9980.35]


\egwnogap{\textbf{Nazungumza kwa dhati, ili kwamba mpate kujua, kama hakuna mabadiliko yaliyo ya dhati kwako, hakuwezi kuwa na tumaini la muungano kati yako na wale wanaoshikilia mwanzo wa imani yao imara hadi mwisho.} Umefanya mgawanyiko. \textbf{\underline{Hatuna budi kusimama imara kwa ajili ya kweli ambazo Bwana ametupa kama nguzo za imani yetu}}.}[Lt253-1903.27; 1903][https://egwwritings.org/read?panels=p9980.35]


\egwnogap{I entreat you to turn to the Lord with full purpose of heart, before it is forever too late. Separate yourself from the influences which have separated you from your brethren who are engaged in the gospel ministry and from the people whom God is leading. \textbf{\underline{Patchwork theories} cannot be accepted by those who are loyal to the faith and to \underline{the principles} that have withstood all the opposition of satanic influences}.}[Lt253-1903.28; 1903][https://egwwritings.org/read?panels=p9980.36]


\egwnogap{Nakusihi umrudie Bwana kwa kusudi kamili la moyo, kabla haijachelewa milele. Jitenge na mvuto ambao umekutenganisha na ndugu zako ambao wanashiriku katika huduma ya injili na kutoka kwa watu ambao Mungu anawaongoza. \textbf{\underline{Nadharia za viraka} haziwezi kukubaliwa na wale ambao ni waaminifu kwa imani na kwa \underline{kanuni} ambayo yamestahimili upinzani wote wa uvutano wa kishetani}.}[Lt253-1903.28; 1903][https://egwwritings.org/read?panels=p9980.36]


\egwnogap{If you will empty yourself of all that has separated you from Christ, and receive the Saviour into your heart, you will be transformed in character. Lay off responsibilities for a time, and go away somewhere with a few of your brethren, and with them search the Scriptures. Humble your heart before the Lord, and make thorough work for repentance. \textbf{The religion of Christ is the spiritual leaven that is to be introduced into the heart. This changes the life and character}. This religion is a heavenly principle, seen in the Christian’s life and conversation. It is revealed in Christian purity. The love of Christ is seen in the tenderness and grace of sanctified humanity. It is by the Word made flesh that we are saved. Our redemption was wrought out, \textbf{not by the Son of God’s remaining in heaven, but by the Son of God’s becoming incarnate—taking humanity upon Him and coming to this world}. Thus eternal life was brought to us. That which authority, commands, and promises could not do, God did by coming to this world in the likeness of sinful flesh.}[Lt253-1903.29; 1903][https://egwwritings.org/read?panels=p9980.37]


\egwnogap{Ikiwa utayaondoa nafsini mwako yale yote ambayo yamekutenganisha na Kristo, na kumpokea Mwokozi ndani ya moyo wako, utabadilishwa katika tabia. Acha majukumu kwa muda, na nenda zako mahali fulani pamoja na wachache wa ndugu zako, na pamoja nao uyachunguze Maandiko. Nyenyekeza moyo wako mbele za Bwana, na ufanye kazi kamili kwa ajili ya toba. \textbf{Dini ya Kristo ndiye chachu ya kiroho ambayo inapaswa kuingizwa ndani ya moyo. Hii inabadilisha maisha na tabia}. Dini hii ni kanuni ya mbinguni, inayoonekana katika maisha ya Mkristo na mazungumzo yake. Inafunuliwa katika usafi wa Kikristo. Upendo wa Kristo unaonekana katika upole na neema ya wanadamu waliotakaswa. Ni kwa Neno lililofanyika mwili tunaokolewa. Ukombozi wetu ulifanywa, \textbf{si kwa Mwana wa Mungu kubaki mbinguni, bali kwa Mwana wa Mungu kufanyika mwili—kuchukua ubinadamu juu Yake na kuja kwa hii dunia}. Hivyo uzima wa milele uliletwa kwetu. Yale ambayo mamlaka, amri, na ahadi haingeweza kufanya, Mungu alifanya kwa kuja katika ulimwengu huu katika mfano wa mwili wenye dhambi.}[Lt253-1903.29; 1903][https://egwwritings.org/read?panels=p9980.37]


\egwnogap{Christ came to the earth to live as a man among men, not to be spoiled by human frailty, but to place in the minds of men principles of truth that could never be obliterated, because they are eternally true. He came to bring a new life to fallen human beings—an excellence that could not be stained or deteriorated by sin.}[Lt253-1903.30; 1903][https://egwwritings.org/read?panels=p9980.38]


\egwnogap{Kristo alikuja duniani kuishi kama mwanadamu miongoni mwa wanadamu, si kuharibiwa na udhaifu wa kibinadamu, bali kuweka katika mawazo ya wanadamu kanuni za ukweli ambazo hazingeweza kufutika, kwa sababu ni zile zilizoko kweli milele. Alikuja kuleta maisha mapya kwa wanadamu walioanguka—ubora huo hauwezi kuchafuliwa au kuharibiwa na dhambi.}[Lt253-1903.30; 1903][https://egwwritings.org/read?panels=p9980.38]


\egwnogap{\textbf{My brother, I must tell you that you have little realization of whither your feet have been tending}. You have been binding yourself up with those who belong to the army of the great apostate. \textbf{Your mind has been as dark as Egypt}. \textbf{If you will fall on the Rock and be broken}, Christ will accept you. But you have been standing on the enemy’s ground, doing his work. \textbf{The religious world is fast going over the same road that you have been following. If you continue to follow this road, you will have plenty of company. But what will the end be?}}[Lt253-1903.31; 1903][https://egwwritings.org/read?panels=p14068.9980039]


\egwnogap{\textbf{Ndugu yangu, lazima nikuambie kwamba huna utambuzi toshelezi wa mahali miguu yako imekuwa ikielekea}. Umekuwa ukijifunga kwa utangamano na wale walio wa jeshi la muasi mkuu. \textbf{Akili yako imekuwa giza kama Misri}. \textbf{Ikiwa utaanguka kwenye Mwamba na kuvunjika}, Kristo atakukubali. Lakini umekuwa ukisimama kwenye ardhi ya adui, ukifanya kazi yake. \textbf{Ulimwengu wa kidini unaelekea kwa kasi katika njia ile ile ambayo umekuwa ukifuata. Ukiendelea kufuata barabara hii, utakuwa na watu wengi. Lakini mwishowe utakuwa nini?}}[Lt253-1903.31; 1903][https://egwwritings.org/read?panels=p14068.9980039]


\egwnogap{So long have you been walking in darkness, so long have you followed your own way, that you may be strongly tempted to resist this appeal that I make. If it were not that your \textbf{eternal interests are involved}, I would not speak to you on this subject. It would seem that I have written enough, that there is no need of my urging this subject upon you further. \textbf{But I tell you in truth that I clearly understand what I am doing}. Sufficient light has been given you. But for several years you have not heeded this light. If you had wished to know what the Lord has said, you could have known; \textbf{for you have the books that have been written under the guidance of His Spirit}. You have had all the directions that could be asked for to point out the right way. Direct light has been sent you. But you have looked upon this as of less importance than your own plans and devisings. If you had heeded the testimonies sent you, Living Temple would never have been written.}[Lt253-1903.32; 1903][https://egwwritings.org/read?panels=p9980.40]


\egwnogap{Kwa muda mrefu sana umetembea gizani, umefuata njia yako mwenyewe kwa muda mrefu unaweza kujaribiwa kwa nguvu kupinga ombi hili ninalotoa. Isingekuwepo maslahi yako ya \textbf{milele yanahusika}, nisingezungumza nawe juu ya mada hii. Inaweza kuonekana kuwa nimeandika vya kutosha, kwamba hakuna haja ya mimi kuhimiza somo hili kwako zaidi. \textbf{Lakini nakuambia kwa ukweli kwamba ninaelewa wazi kile ninachofanya}. Nuru ya kutosha umepokezwa. Lakini kwa miaka kadhaa haujazingatia nuru hii. Ikiwa ungetaka kujua ni nini Bwana amesema, ungalijua; \textbf{kwa kuwa una vitabu vilivyoandikwa chini ya uongozi wa Roho wake}. Umekuwa na maelekezo yote ambayo kwayo yanaweza kuhitajika ili kuonyesha njia sahihi. Nuru ya moja kwa moja imetumwa kwako. Lakini umetazama hii kama yenye umuhimu wa kadiri ndogo kuliko mipango na mawazo yako. Ikiwa ungetii shuhuda zilizotumwa kwako, Living Temple halingeandikwa kamwe.}[Lt253-1903.32; 1903][https://egwwritings.org/read?panels=p9980.40]


\egwnogap{Will you not make a thorough, determined, Christlike effort to break the spell that Satan has cast over you? He has had great power over your mind and has swayed you in wrong lines. He thinks that he can hold you now. Will you not defeat and disappoint him?}[Lt253-1903.33; 1903][https://egwwritings.org/read?panels=p9980.41]


\egwnogap{Je, hutafanya jitihada kamili, iliyotanda, kama ya Kikristo ili kuvunja nguvu za uchawi ambazo Shetani ameweka juu yako? Amekuwa na uwezo mkubwa juu ya akili yako na amekuyumbisha katika njia zisizo sahihi. Anadhani kwamba anaweza kukushikilia sasa. Je, hutamshinda na kumkatisha tamaa?}[Lt253-1903.33; 1903][https://egwwritings.org/read?panels=p9980.41]


\egwnogap{I write to you as I would to a son. Break away from the enemy—the accuser of the brethren. Say to him, “Get thee behind me Satan. I have committed a grievous sin in heeding your suggestions. I will no longer listen to them.” I beg of you, for your soul’s sake, to resist the tempter, that he may flee from you. Draw near to God, and He will draw near to you. \textbf{You will lose heaven unless you fall on the Rock and are broken}.}[Lt253-1903.34; 1903][https://egwwritings.org/read?panels=p9980.42]


\egwnogap{Nakuandikia wewe kama vile ningemwandikia mwana yeyote yule. Achana na adui—mshitaki wa ndugu. Mwambie, “Nenda nyuma yangu Shetani. Nimefanya dhambi kubwa kwa kusikiliza mapendekezo zako. Hakika sitazisikiliza tena.” Ninakuomba, kwa ajili ya nafsi yako, kupinga mjaribu, ili aepukane nawe. Umkaribie Mungu, naye atakukaribia wewe. \textbf{Wewe utapoteza mbingu usipoanguka juu ya Mwamba na kuvunjika}.}[Lt253-1903.34; 1903][https://egwwritings.org/read?panels=p9980.42]


Many things in this letter to Dr. Kellogg go without being said, yet are explained when the context is understood. Ellen White read the letter from Brother Daniells expressing how Dr. Kellogg wanted to revise the Living Temple because he\others{had been thinking the matter over, and began to see that he had made a slight mistake in \textbf{expressing }his views}, and\others{that within a short time \textbf{he had come to believe in the trinity} and could now see pretty clearly where all the difficulty was, and believed that he could clear the matter up satisfactorily}. Kellogg confessed,\others{that he now believed \textbf{in God the Father, God the Son, and God the Holy Ghost}}. In answer to that, Sister White personally wrote to him:\egwinline{The book Living Temple \textbf{is not to be patched up}, a few changes made in it, and then advertised and praised as a valuable production}. How did Kellogg want to patch up his book? According to A. G. Daniells’ testimony, he thought to change a few expressions by explicitly stating his trinitarian sentiment. But the expression of the views was not the real problem—it was the views themselves. Sister White did not spare rebuking him for his views of God, which were \textit{trinitarian} views. She told him that she is\egwinline{\textbf{determined that \underline{no theories of the order that you have been accepting} shall come into our ranks}}. This is a very strong statement. Could it be that, since Kellogg confessed that he was accepting the Trinity doctrine, Sister White was also including it in her statement? It seems unthinkable because this doctrine is in our ranks today. But her statement actually pinpoints the Trinity when she said:\egwinline{\textbf{Patchwork theories} cannot be accepted by those who are loyal \textbf{to the faith and to the principles} that have withstood all the opposition of satanic influences}. Kellogg wanted to patch up “\textit{Living Temple}” by explicitly mentioning the Trinity doctrine. Why was Sister White determined to keep this doctrine out of our ranks, yet it is in our ranks today? It is fair to point out that the Trinity was not part of Seventh-day Adventist faith in her time and it came into our ranks later. Today, many argue that it was because of her works that the Trinity is a part of our beliefs, but Ellen White’s reaction, and her answer to Kellogg’s belief in it, showcases how she dealt with such doctrine. What can we learn from that?


Mambo mengi katika barua hii kwa Dk. Kellogg huenda bila kuelezwa, lakini yanawekwa wazi ikiwa muktadha umebainishwa. Ellen White alisoma barua kutoka kwa Ndugu Daniells akieleza jinsi Dk. Kellogg alitaka kurekebisha Living Temple kwa sababu \others{alikuwa akifikiria jambo hilo tena, na akaanza kuona kwamba alikuwa amefanya makosa kidogo katika \textbf{kutoa} maoni yake}, na\others{hivyo kwa muda mfupi \textbf{alikuwa amesadiki imani katika utatu} na sasa aliweza kuona vizuri sana pale ambapo ugumu wote ulikuwepo, na aliamini kwamba angeweza kuliondoa jambo hilo tatanishi kwa njia ya kuridhisha}. Kellogg alikiri,\others{kwamba sasa aliamini \textbf{katika Mungu Baba, Mungu Mwana, na Mungu Roho Mtakatifu}}. Ili kujibu hilo, Dada White alimwandikia hivi kibinafsi:\egwinline{Kitabu Living Temple \textbf{hakipaswi kuwekwa viraka}, mabadiliko machache kufanyiwa ndani yake, na kisha kutangazwa na kusifiwa kama uzalishaji wa thamani}. Je, Kellogg alitaka kuweka viraka kitabu chake vipi? Kulingana na ushuhuda wa A. G. Daniells, alifikiria kubadili maneno machache kwa kusema waziwazi hisia zake za utatu. Lakini usemi wa maoni haukuwa shida halisi—ilikuwa maoni yenyewe. Dada White hakuacha kumkemea kwa maoni yake juu ya Mungu, ambayo yalikuwa maoni ya utatu. Alimwambia kwamba \egwinline{\textbf{amedhamiria kwamba \underline{hakuna nadharia sawia na zile ambazo amekuwa akikubali} zitapata nafasi katika safu zetu}}. Hii ni kauli kali sana. Je, inaweza kuwa hivyo, kwa kuwa Kellogg alikiri kwamba alikuwa akikubali fundisho la Utatu, Dada White pia alikuwa akiijumuisha kwenye taarifa yake? Inaonekana kuwa haiwezekani kwa sababu fundisho hili liko ndani safu zetu leo. Lakini kauli yake kwa kweli inakazia Utatu aliposema:\egwinline{\textbf{Nadharia za viraka} haziwezi kukubaliwa na wale ambao ni waaminifu \textbf{kwa imani na kwa kanuni} ambazo zimestahimili upinzani wote wa mivuto ya kishetani}. Kellogg alitaka kurekebisha “Living Temple” kwa kutaja waziwazi fundisho la Utatu. Kwa nini Dada White alidhamiria kuliweka fundisho hili nje ya safu zetu, ilhali liko katika safu zetu leo? Ni sawa kusema kwamba Utatu haukuwa sehemu ya imani ya Waadventista Wasabato wakati wake na ulikuja katika safu zetu baadae. Leo, wengi wanasema kwamba ni kwa sababu ya kazi zake ndiposa Utatu ni sehemu ya imani yetu, lakini majibu ya Ellen White, na jibu lake kwa imani ya Kellogg juu yake, inaonyesha jinsi alishughulika na fundisho kama hilo. Tunaweza kujifunza nini kutokana na hilo?


Taken in its context, this letter sheds new light on Kellogg’s controversy and demonstrates how we should deal with the Trinity doctrine. The first thing we question is why Sister White never used the word “Trinity” in her writings, even when she was directly dealing with this doctrine? Elsewhere, she answers:


Ikichukuliwa katika muktadha wake, barua hii inatoa mwanga mpya juu ya utata wa Kellogg na kuonyesha jinsi tunavyopaswa kushughulika na fundisho la Utatu. Jambo la kwanza tunalohoji ni kwa nini Dada White hakuwahi kutumia neno “Utatu” katika maandishi yake, hata alipokuwa akishughulikia fundisho hili moja kwa moja? Mahali pengine anajibu:


\egw{I was cautioned not to enter into controversy \textbf{regarding the question} that \textbf{\underline{will come up}} over \textbf{these things, because controversy \underline{might lead men to resort to subterfuges, and their minds would be led away from the truth of the Word of God to assumption and guesswork}}. \textbf{The more that fanciful theories are discussed, the \underline{less men will know of God and of the truth that sanctifies the soul}}.}[Lt232-1903.41; 1903][https://egwwritings.org/read?panels=p14068.10197050]


\egw{Nilitahadharishwa nisiingie kwenye mabishano \textbf{kuhusiana na swali} ambalo \textbf{\underline{litajitokeza}} juu ya \textbf{mambo haya, kwa sababu mabishano \underline{yanaweza kusababisha watu kutumia hila, na akili zao zingeongozwa mbali na kweli ya Neno la Mungu hadi kwenye dhana na kazi ya kubahatisha}}. \textbf{Kadiri nadharia potofu zinavyojadiliwa, ndivyo \underline{wanadamu watakavyojua kiasi kuhusu Mungu na wa ile kweli inayotakasa nafsi}}.}[Lt232-1903.41; 1903][https://egwwritings.org/read?panels=p14068.10197050]


This is a very important lesson and principle that Sister White is teaching us here. When the controversy over Kellogg’s theories arose, she did not venture into the theories themselves, because this would lead the minds of men away from the truth of the Word of God to assumption and guesswork. Rather, she led the minds of men into the truth, which sanctifies the soul. She led by example, evident here in her letter to Dr. Kellogg. This truth that she led the minds of men to, was the truth on the \emcap{personality of God}. She rebuked Kellogg for his theories but, very importantly, we properly identify these theories by their context and her implicit expression of them.


Hili ni somo muhimu sana na kanuni ambayo Dada White anatufundisha hapa. Wakati mzozo juu ya nadharia za Kellogg uliibuka, hakujitosa kwenye nadharia zenyewe, kwa sababu hii ingeongoza mawazo ya watu kutoka kwenye kweli ya Neno la Mungu hadi dhana na kazi ya kubahatisha. Badala yake, aliongoza mawazo ya wanadamu katika kweli, ambayo hutakasa nafsi. Aliongoza kwa mfano, jinsi umedhihirishwa hapa katika barua yake kwa Dk. Kellogg. Ukweli huu ambao kwayo aliouongoza akili za watu, ulikuwa ukweli juu ya \emcap{Umbile la Mungu}. Alimkemea Kellogg kwa ajili ya nadharia zake lakini, cha muhimu sana, tunatambua nadharia hizi ipasavyo kwa muktadha wao na kwa maelezo yake yaliyofiche kwao.


We see that she made a contrast between the Trinity and the \emcap{personality of God}. She made a contrast between the old principles of our faith and the new theories. First, she drew our minds back to the beginning of our spiritual heritage,\egwinline{after the passing of the time in 1844}, when her husband James White, Joseph Bates, Father Pierce, Elder Edson, and many others who were keen, noble, and true, searched for truth. She pointed back to the wonderful and mighty experiences of how the leading points of our faith, held in 1903, were firmly established. \egwinline{\textbf{Thus \underline{the leading points of our faith}} as we hold them today were firmly established.} \egwinline{\textbf{\underline{Point after point} was clearly defined, and all the brethren came into harmony}.} \egwinline{\textbf{The whole company of believers were united in the truth}}. Obviously, from the context of chapter 10 of the Special Testimonies, we know that these experiences explain \egwinline{\textbf{how firmly the foundation of our faith has been laid}}[SpTB02 56.4; 1904][https://egwwritings.org/read?panels=p417.288]. This foundation is expressed in the \emcap{Fundamental Principles}\footnote{\href{https://static1.squarespace.com/static/554c4998e4b04e89ea0c4073/t/59d17e24c027d84167e17617/1506901547915/SDA-YB1905+\%28P.+188-192\%29.pdf}{Yearbook Of Seventh-day Adventist denomination 1905, p. 188-192}}. This foundation is the truth which,\egwinline{\textbf{\underline{point by point}}, \textbf{has been sought out by prayerful study, and testified to by the miracle-working power of the Lord}}. God \egwinline{\textbf{calls upon us to \underline{hold firmly}, with the grip of faith, to \underline{the fundamental principles} that are \underline{based upon unquestionable authority}}.}[SpTB02 59.1; 1904][https://egwwritings.org/read?panels=p417.299] In light of these experiences and the truth expressed in the \emcap{fundamental principles}, \egwinline{\textbf{\underline{Patchwork theories} cannot be accepted by those who are loyal \underline{to the faith} and \underline{to the principles} that have withstood all the opposition of satanic influences}}[Lt253-1903.28; 1903][https://egwwritings.org/read?panels=p14068.9980036]. From the historical record of these brethren who were keen, noble and true, we have evidence that they, too, have contrasted the Trinity doctrine with the truth on the \emcap{personality of God}. James White, in the Review and Herald article, listed \others{some of the popular fables of the age}, saying: \others{Here we might mention \textbf{the Trinity, which \underline{does away the personality of God, and of his Son Jesus Christ}}}[James White, Review \& Herald, December 11, 1855, p. 85.15][http://documents.adventistarchives.org/Periodicals/RH/RH18551211-V07-11.pdf]. J. N. Andrews said, \others{\textbf{The doctrine of the Trinity which was established in the church by the council of Nicea, A. D. 325}. \textbf{This doctrine \underline{destroys the personality of God, and his Son Jesus Christ our Lord}}...}[J. N. Andrews, Review \& Herald, March 6, 1855, p. 185][http://documents.adventistarchives.org/Periodicals/RH/RH18550306-V06-24.pdf] J. B. Frisbie, in his article “\textit{Seventh-day Sabbath not abolished}”, compares the Sabbath God to the Sunday god; he describes the Sabbath God in light of the \emcap{personality of God} expressed in the first point of the \emcap{Fundamental Principles}. The Sunday god is described by the \others{unity of this God-head, there are three persons of one substance, power and eternity; the Father, the Son, and the Holy Ghost}[J. B. Frisbie, Review \& Herald March 7, 1854. p. 50][http://documents.adventistarchives.org/Periodicals/RH/RH18540307-V05-07.pdf]. He explained how the doctrine on the \emcap{personality of God} stands in conflict with the doctrine of Trinity, in the same way the Holy Sabbath stands in conflict with pagan Sunday worship. Also, brother J. N. Loughborough wrote the objections to the Trinity doctrine in the Adventist Review and Sabbath Herald\footnote{\href{https://adventistdigitallibrary.org/adl-349160/advent-review-and-sabbath-herald-november-5-1861}{J. N. Loughborough, November 5, 1861, Review \& Herald, vol. 18, p. 184, par. 1-11}}. In the other publication of the Review and Herald, he published the article “\textit{Is God a person?}”, explaining the position of Seventh-day Adventist belief on the \emcap{personality of God}, expressed in the first point of the \emcap{Fundamental Principles}\footnote{\href{http://documents.adventistarchives.org/Periodicals/RH/RH18550918-V07-06.pdf}{J. N. Loughborough, September 18. 1855, Review \& Herald, vol. 7, p. 6.}}. James White was also explaining the same position in his multiple print pamphlet, “\textit{The Personality of God}”\footnote{\href{https://egwwritings.org/?ref=en_PERGO.1.1&para=1471.3}{J. White, The Personality of God, June 18. 1861.}}. These are just a few examples where the Adventist pioneers explained the position on the \emcap{personality of God} expressed by the first point of the \emcap{fundamental principles}.


Tunaona kwamba alifanya kutofautisha kati ya Utatu na Umbile la Mungu. Alifanya kutofautisha kati ya kanuni za zamani za imani yetu na nadharia mpya. Kwanza, alivuta mawazo yetu nyuma kwenye mwanzo wa urithi wetu wa kiroho,\egwinline{baada ya kupita kwa wakati mnamo 1844}, wakati mumewe James White, Joseph Bates, Baba Pierce, Mzee Edson, na wengi wengine waliokuwa makini, waungwana, na wa kweli, walitafuta ukweli. Yeye aliashiria hapo awali kwa uzoefu wa ajabu na wenye nguvu jinsi pointi kuu za imani yetu, zilizoshikiliwa mwaka wa 1903, zilivyowekwa msingi imara. \egwinline{\textbf{Hivyo \underline{pointi kuu za imani yetu}} tunavyozishikilia leo ziliwekwa msingi imara.} \egwinline{\textbf{\underline{Hoja baada ya hoja} ilifafanuliwa wazi, na ndugu wote wakaingia maelewano}.} \egwinline{\textbf{Kundi zima la waumini liliunganishwa katika ukweli}}. Ni wazi, kutoka kwa muktadha wa sura ya 10 ya Shuhuda Maalum, tunajua kwamba uzoefu huu unaeleza \egwinline{\textbf{jinsi msingi wa imani yetu ulivyowekwa imara}}[SpTB02 56.4; 1904][https://egwwritings.org/read?panels=p417.288]. Msingi huu unaonyeshwa wazi ndani ya Kanuni za Msingi\footnote{\href{https://static1.squarespace.com/static/554c4998e4b04e89ea0c4073/t/59d17e24c027d84167e17617/1506901547915/SDA-YB1905+\%28P.+188-192\%29.pdf}{Yearbook Of Seventh-day Adventist denomination 1905, p. 188-192}}. Msingi huu ni ukweli ambao,\egwinline{\textbf{\underline{hatua kwa hatua}}, \textbf{umetafutwa kwa kujifunza kwa maombi, na kushuhudiwa kwa uwezo wa utendaji wa miujiza wa Bwana}}. Mungu \egwinline{\textbf{anatuita kushika kwa uthabiti, kwa mshiko wa imani, \underline{kanuni za msingi} ambazo \underline{zimetegemezwa kwa mamlaka isiyotiliwa shaka}}.}[SpTB02 59.1; 1904][https://egwwritings.org/read?panels=p417.299] Kwa kuzingatia uzoefu huu na ukweli unaoonyeshwa katika Kanuni za Kimsingi, \egwinline{\textbf{\underline{Nadharia za viraka} haziwezi kukubaliwa na wale ambao ni waaminifu \underline{kwa imani} na \underline{kwa kanuni} ambazo zimepinga upinzani wote wa ushawishi wa kishetani}}[Lt253-1903.28; 1903][https://egwwritings.org/read?panels=p14068.9980036]. Kutoka kwa kumbukumbu ya kihistoria ya ndugu hawa ambao walikuwa makini, waungwana na wa kweli, tunao uthibitisho kwamba wao pia wametofautisha kati ya fundisho la Utatu na ukweli juu ya Umbile la Mungu. James White, katika makala ya Review and Herald, aliorodhesha \others{baadhi ya hekaya maarufu za enzi hizi}, akisema: \others{Hapa tunaweza kutaja \textbf{Utatu, ambao \underline{unaondoa Umbile la Mungu, na la Mwanawe Yesu Kristo}}}[James White, Review \& Herald, December 11, 1855, p. 85.15][http://documents.adventistarchives.org/Periodicals/RH/RH18551211-V07-11.pdf]. J. N. Andrews alisema, \others{\textbf{Fundisho la Utatu ambalo lilianzishwa katika kanisa na Baraza la Nicea, A. D. 325}. \textbf{Fundisho hili \underline{linaharibu Umbile la Mungu, na Mwana wake Yesu Kristo Bwana wetu}}...}[J. N. Andrews, Review \& Herald, March 6, 1855, p. 185][http://documents.adventistarchives.org/Periodicals/RH/RH18550306-V06-24.pdf] J. B. Frisbie, katika makala yake “\textit{Seventh-day Sabbath not abolished}”, analinganisha Mungu wa Sabato na mungu wa Jumapili; anaelezea Mungu wa Sabato katika mwanga wa Umbile la Mungu unaoonyeshwa katika hoja ya kwanza ya Kanuni za Msingi. Mungu wa jumapili anaelezewa na \others{umoja wa huyu Mungu mkuu, kuna nafsi tatu za nafsi moja, nguvu na umilele; Baba, Mwana, na Roho Mtakatifu}[J. B. Frisbie, Review \& Herald March 7, 1854. p. 50][http://documents.adventistarchives.org/Periodicals/RH/RH18540307-V05-07.pdf]. Alieleza jinsi fundisho la Umbile la Mungu linapingana na fundisho la Utatu, katika hali hiyo hiyo jinsi Sabato Takatifu inavyosimama kinyume na ibada ya kipagani ya Jumapili. Pia, ndugu J. N. Loughborough aliandika pingamizi kwa fundisho la Utatu katika Adventist Review and Sabbath Herald\footnote{\href{https://adventistdigitallibrary.org/adl-349160/advent-review-and-sabbath-herald-november-5-1861}{J. N. Loughborough, November 5, 1861, Review \& Herald, vol. 18, p. 184, par. 1-11}}. Katika uchapishaji mwingine wa Review and Herald, alichapisha makala hiyo “\textit{Is God a person?}”, akieleza msimamo wa imani ya Waadventista Wasabato kuhusu Umbile la Mungu, iliyoelezwa katika hoja ya kwanza ya Kanuni za Msingi\footnote{\href{http://documents.adventistarchives.org/Periodicals/RH/RH18550918-V07-06.pdf}{J. N. Loughborough, September 18. 1855, Review \& Herald, vol. 7, p. 6.}}. James White pia alikuwa akifafanua msimamo huo katika kijitabu chake kilichopokea uchapishaji nyingi, “\textit{The Personality of God}”\footnote{\href{https://egwwritings.org/?ref=en_PERGO.1.1&para=1471.3}{J. White, The Personality of God, June 18. 1861.}}. Haya ni mifano tu ambapo waanzilishi wa Kiadventista walieleza msimamo juu ya Umbile la Mungu unaoonyeshwa na kipengele cha kwanza ya Kanuni za Msingi.


Sister White rebuked Kellogg:\egwinline{\textbf{But I tell you in truth that I clearly understand what I am doing}. \textbf{Sufficient light has been given you}. But for several years you have not heeded this light. If you had wished to know what the Lord has said, you could have known; \textbf{for \underline{you have the books} that have been written under the guidance of His Spirit}. You have had all the directions that could be asked for to point out the right way. Direct light has been sent you. But you have looked upon this as of less importance than your own plans and devisings. If you had heeded the testimonies sent you, Living Temple would never have been written.}[Lt253-1903.32; 1903][https://egwwritings.org/read?panels=p14068.9980040] The core issue of Dr. Kellogg’s controversy was \egwinline{the personality of God and where His presence is}[SpTB02 51.3; 1904][https://egwwritings.org/read?panels=p417.262]. Dr. Kellogg had access to the pioneer writings, books and the church's \emcap{Fundamental Principles} that were testified to by the miracle working power of the Holy Spirit.


Dada White alimkemea Kellogg:\egwinline{\textbf{Lakini ninakuambia kwa ukweli kwamba ninaelewa waziwazi kile ninachofanya}. \textbf{Nuru ya kutosha umepewa wewe}. Lakini kwa miaka kadhaa haujazingatia nuru hii. Kama ungetaka kujua kile ambacho Bwana amesema, ungalijua; \textbf{kwa kuwa unazo \underline{vitabu} vilivyoandikwa chini ya uongozi wa Roho wake}. Umekuwa na maelekezo yote ambayo yanaweza kuhitajika ili kuonyesha njia sahihi. Nuru ya moja kwa moja umepokezwa wewe. Lakini umeona hili kama lisilo na umuhimu kuliko mipango yako mwenyewe na miundo yako. Kama ungetii shuhuda zilizotumiwa wewe, Living Temple haingeandikwa kamwe.}[Lt253-1903.32; 1903][https://egwwritings.org/read?panels=p14068.9980040] Suala la msingi la mgogoro wa Dk. Kellogg lilikuwa \egwinline{Umbile la Mungu na uwepo wake ulipo}[SpTB02 51.3; 1904][https://egwwritings.org/read?panels=p417.262]. Dk. Kellogg alipata kupokea maandishi ya waanzilishi, vitabu na Kanuni za Msingi za kanisa ambazo zilishuhudiwa na nguvu ya utendaji wa miujiza ya Roho Mtakatifu.


Sister White recalled the experiences of how the \textit{leading points of our faith}, as were held in former times, were firmly established.\egwinline{\textbf{\underline{Point after point} was clearly defined, and all the brethren came into harmony}}[Lt253-1903.4; 1903][https://egwwritings.org/read?panels=p14068.9980010]. These leading points were the \emcap{Fundamental Principles}, of which the \emcap{personality of God} was one. This point, and Sister White’s testimony of it, remained the same during the course of her life.  She said\egwinline{\textbf{\underline{I have ever had the same testimony to bear which I now bear regarding the personality of God}}}[Lt253-1903.9; 1903][https://egwwritings.org/read?panels=p14068.9980015]. From Early Writings, she then quoted her visions of the Heavenly reality. She recalled how she had had the privilege to be in the presence of God, how God, encircled by the light of His glory, passed by her side. She did not see God from the light He was encircled by; she was afraid of Him, thinking that if He were to approach her she\egwinline{would be struck out of existence}. Then she saw\egwinline{\textbf{the lovely Jesus, that He is a person}. \textbf{I asked Him if His Father was a person, and had \underline{a form like} Himself}. Said Jesus, ‘\textbf{I am the express image of My Father’s person!}’}[Lt253-1903.12; 1903][https://egwwritings.org/read?panels=p14068.9980018]. The question she had was: \textit{is God a person, having a form like Jesus}? The answer was affirmative—with a strong biblical foundation. Her visions were not the source of the truth on the \emcap{personality of God}; rather, they confirmed the truth the pioneers had discovered through diligent study of God’s word.


Dada White alikumbuka uzoefu wa jinsi \textit{pointi kuu za imani yetu}, kama zilivyoshikiliwa nyakati za zamani, ziliwekwa imara.\egwinline{\textbf{\underline{Hoja baada ya hoja} ilifafanuliwa wazi, na ndugu wote wakaingia maelewano}}[Lt253-1903.4; 1903][https://egwwritings.org/read?panels=p14068.9980010]. Pointi hizi kuu zilikuwa Kanuni za Msingi, ambayo Umbile la Mungu lilikuwa mojawapo. Jambo hili, na ushuhuda wa Dada White juu yake, ulibaki vile vile wakati wa maisha yake. Alisema\egwinline{\textbf{\underline{Ninao ushuhuda uleule wa kutoa ninaotoa sasa kuhusu Umbile la Mungu}}}[Lt253-1903.9; 1903][https://egwwritings.org/read?panels=p14068.9980015]. Kutoka Maandiko ya Awali, kisha alinukuu maono yake ya ukweli wa Mbinguni. Alikumbuka jinsi ilivyokuwa upendeleo kuwa mbele za Mungu, jinsi Mungu, akizungukwa na nuru ya utukufu wake, akapita pembeni mwake. Hakumwona Mungu kutoka kwenye nuru Aliyokuwa amezingirwa; alimwogopa Yeye, akifikiri kwamba kama Angemkaribia\egwinline{angetolewa uhai}. Kisha alimwona\egwinline{\textbf{Yesu mpendwa, kwamba Yeye ni Nafsi}. \textbf{Nilimuuliza kama Baba yake alikuwa Nafsi, na alikuwa na \underline{umbo kama} Yeye}. Yesu alisema, ‘\textbf{Mimi ni chapa kamili ya Umbile la Baba Yangu!}’}[Lt253-1903.12; 1903][https://egwwritings.org/read?panels=p14068.9980018]. Swali alilokuwa nalo lilikuwa: \textit{je, Mungu ni Nafsi, mwenye umbo kama Yesu?} Jibu lilikuwa la uthibitisho—likiwa na msingi thabiti wa kibiblia. Maono yake hayakuwa chanzo cha ukweli juu ya Umbile la Mungu; badala yake, walithibitisha ukweli ambao waanzilishi walikuwa wamegundua kwa kujifunza neno la Mungu kwa bidii.


Therefore, their final conclusion on the \emcap{personality of God} was,\others{That there is \textbf{one God}, \textbf{a personal, spiritual \underline{being}}, \textbf{the creator of all things}, omnipotent, omniscient, and eternal, infinite in wisdom, holiness, justice, goodness, truth, and mercy; unchangeable, and \textbf{everywhere present by his representative, the Holy Spirit}. Ps. 139:7; That there is one Lord Jesus Christ, \textbf{the Son of the Eternal Father, the one by whom he created all things, and by whom they do consist} …and as the closing portion of his work as priest, before he takes his throne as king, he will make \textbf{the great atonement} for the sins of all such, and their sins will then be blotted out (Acts 3:19) and borne away from the sanctuary, as shown in the service of the Levitical priesthood, which foreshadowed and prefigured the ministry of our Lord in heaven. See Lev. 16; Heb. 8: 4, 5; 9: 6, 7; etc.}[The first, and part of the second, point of the Fundamental Principles, 1905.]


Kwa hiyo, hitimisho lao la mwisho kuhusu Umbile la Mungu lilikuwa,\others{Kwamba kuna \textbf{Mungu mmoja}, \textbf{huluki binafsi wa kiroho}, \textbf{muumba wa vitu vyote}, muweza wa yote, mjuzi wa yote, na wa milele, asiye na kikomo katika hekima, utakatifu, haki, wema, ukweli, na rehema; asiyebadilika, na \textbf{uwepo wake uliopo kila mahali kupitia mwakilishaji wake, Roho Mtakatifu}. Zab. 139:7; Kwamba kuna Bwana Mmoja Yesu Kristo, \textbf{Mwana wa Baba wa Milele, ambaye kwa yeye aliumba vitu vyote, na vitu vyote hushikana katika yeye} …na kama sehemu ya mwisho ya kazi yake ya ukuhani, kabla hajatwaa kiti chake cha enzi kama mfalme, atafanya \textbf{upatanisho mkuu} kwa dhambi za hao wote, na dhambi zao zitafutwa (Matendo 3:19) na kuchukuliwa mbali na patakatifu, kama inavyoonyeshwa katika utumishi wa ukuhani wa Walawi, ambao ulionyesha kimbele na kufananisha huduma ya Mola wetu aliye mbinguni. Angalia Law. 16; Ebr. 8: 4, 5; 9: 6, 7; na kadhalika.}[Kipengele cha kwanza, na sehemu ya pili, ya Kanuni za Msingi, 1905.]


Ellen White reminded Dr. Kellogg on this point of the \emcap{fundamental principles} by stating:\egwinline{\textbf{The Father, the omniscient One, created the world \underline{through} Christ Jesus}. Christ is the light of the world, the way to eternal life. \textbf{He, the anointed One, God gave to make an atonement for the sins of the world}...}[Lt253-1903.21; 1903][https://egwwritings.org/read?panels=p14068.9980029]


Ellen White alimkumbusha Dk. Kellogg juu ya hatua hii ya Kanuni za Msingi kwa kusema:\egwinline{\textbf{Baba, Mjuzi wa yote, aliumba ulimwengu \underline{kupitia} Kristo Yesu}. Kristo ndiye nuru ya ulimwengu, njia ya uzima wa milele. \textbf{Yeye, aliyetiwa mafuta, Mungu alitoa kufanya upatanisho kwa dhambi za ulimwengu}...}[Lt253-1903.21; 1903][https://egwwritings.org/read?panels=p14068.9980029]


The question on the \emcap{personality of God} deals with the quality or state of God being a person. The Adventist pioneers gave an answer to it and God approved it through the writings of Ellen White: God is a \textit{personal spiritual Being} and He is our heavenly Father. Where is His presence?\egwinline{\textbf{We are not to say that the Lord God of heaven is in a leaf, or in a tree; for He is not there. \underline{He sitteth upon His throne in the heavens}}.}[Lt253-1903.15; 1903][https://egwwritings.org/read?panels=p14068.9980022] \\
His presence is on the throne in heaven. \\
\egwinline{\textbf{Heaven is not a vapor. It is a place}. \textbf{Christ has gone to prepare mansions for those who love Him}, those who, in obedience to His commands, come out from the world and are separate...}[EGW, Lt253-1903.25; 1903][https://egwwritings.org/read?panels=p14068.9980033]. \\
“...\egwinline{‘The voice of the Lord is mighty; it shaketh the cedars of Lebanon. \textbf{The Lord is in His holy temple}; let all the earth keep silence before Him.’ [See Psalm 29:5; Habakkuk 2:20.]}[Lt253-1903.18; 1903][https://egwwritings.org/read?panels=p14068.9980026]


Swali la Umbile la Mungu linahusika na ubora au hali ya Mungu kuwa Nafsi. Waanzilishi wa Kiadventista walitoa jibu kwalo na Mungu akaidhinisha kupitia maandishi ya Ellen White: Mungu ni \textit{huluki binafsi wa kiroho} na ni Baba yetu wa mbinguni. Uwepo Wake lipo wapi?\egwinline{\textbf{Hatupaswi kusema kwamba Bwana, Mungu wa mbinguni, yu ndani ya jani, au ndani ya mti; kwa kuwa Hayupo papo. \underline{Ameketi juu ya kiti chake cha enzi mbinguni}}.}[Lt253-1903.15; 1903][https://egwwritings.org/read?panels=p14068.9980022] \\
Uwepo wake uko kwenye kiti cha enzi mbinguni. \\
\egwinline{\textbf{Mbingu si mvuke. Ni mahali}. \textbf{Kristo amekwenda kuandaa makao kwa ajili ya hao wampendao}, wale ambao, kwa kutii amri zake, hujitenga na ulimwengu na wako watengwa...}[EGW, Lt253-1903.25; 1903][https://egwwritings.org/read?panels=p14068.9980033]. \\
“...\egwinline{‘Sauti ya Bwana ina nguvu; inaitikisa mierezi ya Lebanoni. \textbf{Bwana yu ndani ya hekalu Lake takatifu}; dunia yote na ikae kimya mbele zake.’ [Ona Zaburi 29:5; Habakuki 2:20.]}[Lt253-1903.18; 1903][https://egwwritings.org/read?panels=p14068.9980026]


According to Adventist pioneers and Sister White, our heavenly Father is one God. He is a personal Spiritual Being, present in heaven, on His throne. The throne of heaven is a real, physical throne, upon which sits a real Person (Being, having a form, just like Jesus)—our heavenly Father. That place is a real place; it is not a vapor, or any other spiritual view.


Kulingana na waanzilishi wa Kiadventista na Dada White, Baba yetu wa mbinguni ni Mungu mmoja. Yeye ni Huluki binafsi wa Kiroho, aliyepo mbinguni, kwenye kiti Chake cha enzi. Kiti cha enzi cha mbinguni ni halisi, kiti cha enzi cha kimwili, ambacho juu yake ameketi Mtu halisi (Kuwa, mwenye umbo, kama Yesu)—Baba yetu wa mbinguni. Mahali hapo ni mahali halisi; sio mvuke, au mtazamo mwingine wowote wa kiroho.


\egwinline{\textbf{I have often seen that the spiritual view took away all the glory of heaven, and that in many minds the throne of David and the lovely person of Jesus have been burned up in the fire of spiritualism}. I have seen that some who have been deceived and led into this error, will be brought out into the light of truth, \textbf{but it will be almost impossible for them to get entirely rid of the deceptive power of spiritualism. Such should make thorough work in confessing their errors, and leaving them forever}.}[Lt253-1903.13; 1903][https://egwwritings.org/read?panels=p14068.9980019]


\egwinline{\textbf{Mara nyingi nimeona kwamba mtazamo wa kimizimu uliondoa utukufu wote wa mbinguni, na kwenye mawazo ya waja wengi kiti cha enzi cha Daudi na Nafsi nzuri ya Yesu imeteketezwa katika moto wa umizimu}. Nimeona kwamba baadhi ya wale ambao wamedanganywa na kuongozwa katika makosa hili, watafunuliwa nuru iliyo ya ukweli, \textbf{lakini itakuwa karibu kutowezekana kwao kuondoa kabisa nguvu za udanganyifu wa umizimu. Hao wanapaswa kufanya kwa uangalifu kazi ya kuungama makosa yao, na kuyaacha milele}.}[Lt253-1903.13; 1903][https://egwwritings.org/read?panels=p14068.9980019]


The spiritual view of God’s person is an erroneous view. In the Bible we have testimonies of heaven, the heavenly throne, and God who is sitting upon it. If we accept these testimonies in their obvious meaning, then the Trinity doctrine cannot be sustained. The Bible and Spirit of Prophecy present one God in heaven, as a personal being, having a body and form just as Jesus has. This view is not in harmony with the doctrine of the Triune God, since it requires the Holy Spirit to be a Being\footnote{Please look at \hyperref[appendix:unauthenticated-reports]{the appendix} for more quotations which exclude the Holy Spirit to be a being, possessing physical body and form.}, having a body and form—this idea would compromise the Holy Spirit to be a means of the Father and Son by which They are everywhere present. In order to sustain the Trinity doctrine, the testimonies regarding the throne of God and of God’s person, need to be understood by some spiritual view. Here we have seen that Sister White contrasted the truth of the \emcap{personality of God} with the doctrine of Trinity. She contrasted the doctrine of Trinity with the first two points of the \emcap{Fundamental Principles}, which were the results of our pioneers studying the Word of God. Referring to the pioneers and the \emcap{Fundamental Principles}, she said: \egwinline{\textbf{\underline{Patchwork theories} cannot be accepted by those who are \underline{loyal to the faith and to the principles} that have withstood all the opposition of satanic influences.}}[Lt253-1903.28; 1903][https://egwwritings.org/read?panels=p14068.9980036]


Mtazamo wa kimizimu wa nafsi ya Mungu ni mtazamo potovu. Katika Biblia tuna ushuhuda wa mbinguni, kiti cha enzi cha mbinguni, na Mungu aketiye juu yake. Tukikubali shuhuda hizi katika maana yao iliyo wazi, basi fundisho la Utatu haliwezi kutegemezwa. Biblia na Roho ya Unabii inawasilisha Mungu mmoja mbinguni, kama huluki binafsi, mwenye mwili na umbo kama Yesu alivyo navyo. Mtazamo huu haupatani na fundisho la Mungu wa Utatu, kwani hilo linahitaji Roho Mtakatifu kuwa Huluki\footnote{Tafadhali angalia \hyperref[appendix:unauthenticated-reports]{kiambatisho} kwa nukuu zaidi ambazo zinaondoa Roho Mtakatifu kuwa huluki, mwenye mwili na umbo.}, mwenye mwili na umbo—wazo hili lingekuwa linahujumu Roho Mtakatifu kuwa njia ya Baba na Mwana ambayo kwayo wao wapo kila mahali. Ili kudumisha fundisho la Utatu, ushuhuda kuhusu kiti cha enzi cha Mungu na cha nafsi ya Mungu, kinahitaji kueleweka kwa mtazamo fulani wa kiroho. Hapa sisi tumeona kwamba Dada White alitofautisha ukweli wa Umbile la Mungu na fundisho la Utatu. Alilinganisha fundisho la Utatu na mambo mawili ya kwanza ya Kanuni za Msingi, ambazo zilikuwa matokeo ya waanzilishi wetu kujifunza Neno la Mungu. Akirejelea waanzilishi pamoja na Kanuni za Msingi, alisema: \egwinline{\textbf{\underline{Nadharia za viraka} haziwezi kukubaliwa na wale ambao ni \underline{waaminifu kwa imani na kwa kanuni} ambazo zimepinga upinzani wote wa ushawishi wa kishetani.}}[Lt253-1903.28; 1903][https://egwwritings.org/read?panels=p14068.9980036]


The conclusion is straightforward and simple. Those who are loyal to the faith, and to the principles received in the beginning of the work, cannot accept patchwork theories. Put into context, the patchwork theory, which is the Trinity doctrine, cannot be accepted by those who are holding fast \egwinline{\textbf{to \underline{the fundamental principles} that are \underline{based upon unquestionable authority}}}[SpTB02 59.1; 1904][https://egwwritings.org/read?panels=p417.299]. This conclusion leads us back to our first proposed test of the foundation of our faith.


Hitimisho ni moja kwa moja na rahisi. Wale ambao ni waaminifu kwa imani, na kwa kanuni zilizopokelewa mwanzoni mwa kazi, hawawezi kukubali nadharia za viraka. Ikiwekwa katika muktadha, nadharia za viraka, ambayo ni fundisho la Utatu, haliwezi kukubaliwa na wale ambao wanashikilia sana \egwinline{\textbf{\underline{kanuni za msingi} ambazo \underline{zimetegemezwa kwa mamlaka isiyotiliwa shaka}}}[SpTB02 59.1; 1904][https://egwwritings.org/read?panels=p417.299]. Hitimisho hili huturudisha kwenye jaribio letu la kwanza lililopendekezwa la msingi wa imani yetu.


% The Patchwork Theories

\begin{titledpoem}
    
    \stanza{
        Truth established through earnest prayer, \\
        Points of faith discovered with care. \\
        Principles tested by time and trial, \\
        Stand firm against Satan's denial.
    }

    \stanza{
        Patchwork theories seek to sway, \\
        Those from the ancient, proven way. \\
        No revisions of truth we'll accept, \\
        The faithful path must be kept.
    }

    \stanza{
        The personality of God, a sacred revelation, \\
        Not subject to human innovation. \\
        Loyal hearts stand on ground that's sure, \\
        Where foundations eternal endure.
    }
    
\end{titledpoem}


Nadharia za viraka - Lt253-1903
“Ndugu Mpendwa,—“
“Lazima nikuambie kwamba mawazo yako kuhusu baadhi ya mambo yamepotoka pakubwa kwa hakika. Ni pendekezo langu hivi kwamba upate kuona dosari zako. Kitabu cha Hekalu Hai hakipaswi kuwekwa viraka, mabadiliko machache yafanywe ndanimo, na kisha kutangazwa na kusifiwa kama uzalishaji wa thamani. Na ingekuwa bora kuwasilisha sehemu za kifiziolojia katika kitabu kingine chini ya kichwa kingine. Wakati ule ulipoandika kitabu hicho, hukuwa chini ya mwongozo wa Mungu. Kando yako palikuwepo yule ali-yemwongoza Adamu kuwa Na mtazamo kuhusu Mungu kwa nuru uli-opotosha. Moyo wako wote upaswa kubadilishwa, iliyosafishwa kabisa na kwa ukamilifu.”
