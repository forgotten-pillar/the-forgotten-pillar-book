
\qrchapter{https://forgottenpillar.com/rsc/en-fp-chapter5}{Nadharia za kiraka - Lt253-1903}


\egw{Ndugu Mpendwa,—}



\egwnogap{\textbf{Lazima nikuambie kwamba mawazo yako kuhusu baadhi ya mambo \underline{yamepotoka pakubwa kwa hakika}.} Ningependa kwamba upate kuona dosari zako. \textbf{Kitabu Living Temple \underline{hakipaswi kuwekwa viraka}, mabadiliko machache yafanywe ndani yake, na kisha kutangazwa na kusifiwa kama uzalishaji wa thamani}. Ingekuwa bora kuwasilisha sehemu za kifiziolojia katika kitabu kingine chini ya kichwa kingine. \textbf{Wakati ulipoandika kitabu hicho}, \textbf{hukuwa chini ya mwongozo wa Mungu}. Kando yako palikuwepo yule aliyemwongoza Adamu kuwa na mtazamo kuhusu Mungu kwa nuru iliyopotosha. Moyo wako wote unahitaji kubadilishwa, kusafishwa kabisa na kwa ukamilifu.}[Lt253-1903.1; 1903][https://egwwritings.org/read?panels=p9980.7]


\egwnogap{\textbf{Ndugu yangu, usikubali kutengwa na ndugu zako wahudumu ambao hukuonyesha hatari zako. Wale wanaokuambia kwa uaminifu na kwa uwazi makosa yako ndio marafiki zako wa dhati.} Ninasikitika, nimehuzunishwa kwa kina, kuhusu washirika wako wa matibabu. Wamekuwa wasio waaminifu kwa Mungu na wasio wa kweli kwako kwa kushindwa kukuambia kwa fadhili lakini kwa uthabiti mahali ambapo haukuwa ukifanya kazi kwa uadilifu.}[Lt253-1903.2; 1903][https://egwwritings.org/read?panels=p9980.8]


\egwnogap{Kuna mambo mengi ambayo lazima uyashinde kabla ya kuokolewa. Katika moyo ambao hauongozwi na Mungu, kuna kitu kinachoufanya utamani kudumishwa katika mwenendo wake usio sahihi. Watu wanaokuambia ukweli kwa uaminifu, wakionyesha makosa yako, umewachukulia kama adui zako. Lakini mara nyingi wao ni marafiki zako wa dhati na, katika kukuambia mahali ulikuwa ukitembea katika njia za ajabu, walikuwa wanafanya wajibu usiofurahisha. Watumishi wa Bwana hawapaswi kukuza kiburi chako; hawapaswi kusimama kimya, wakiogopa kusema, ‘Kwa nini unafanya hivi?’ Wanapaswa kwa uaminifu kukuonya kuhusu hatari yako.}[Lt253-1903.3; 1903][https://egwwritings.org/read?panels=p9980.9]


\egwnogap{\textbf{Mume wangu, Mzee Joseph Bates, Baba Pierce, Mzee Edson, na wengine wengi ambao walikuwa makini, waungwana, na wa kweli walikuwa miongoni mwa wale ambao, baada ya kupita kwa wakati mnamo 1844, walitafuta ukweli}. \textbf{Katika mikutano yetu muhimu, wanaume hawa walikutana pamoja na kutafuta ukweli kama hazina iliyofichwa}. Nilikutana nao, tukasoma na kuomba kwa bidii; kwa maana tulihisi kwamba ni lazima tujifunze ukweli wa Mungu. Mara nyingi tulibaki pamoja hadi usiku sana, na wakati mwingine usiku kucha, tukiomba kwa ajili ya nuru na kujifunza Neno. Tulipofunga na kuomba, nguvu kuu ilitujia. Lakini sikuweza kuelewa hoja za ndugu wenzangu. Akili yangu ilikuwa imefungwa, kama ilivyokuwa, na sikuweza kuelewa tulichokuwa tunajifunza. Ndipo Roho wa Mungu angenijia, ningechukuliwa katika maono, na maelezo wazi ya vifungu tulivyokuwa tukijifunza ningepewa pamoja na maelekezo kuhusu msimamo tunaopaswa kuchukua kuhusu ukweli na wajibu. Tena na tena hili lilitokea. \textbf{Mstari wa ukweli unaoenea kutoka wakati huo hadi wakati ambapo tutaingia mji wa Mungu uliwekwa wazi mbele yangu}, na nikawapa ndugu na dada zangu maagizo ambayo Bwana alinipa. Walijua kwamba wakati siko katika maono, sikuweza kuelewa mambo haya, na walikubali kama nuru ya moja kwa moja kutoka mbinguni mafunuo niliyopewa. \textbf{Hivyo pointi kuu za imani yetu tunavyozishikilia leo zilidhibitishwa imara}. \textbf{\underline{Hoja baada ya hoja} ilifafanuliwa wazi, na ndugu wote wakaingia katika maelewano}.}[Lt253-1903.4; 1903][https://egwwritings.org/read?panels=p14068.9980010]


\egwnogap{\textbf{Kundi zima la waumini liliunganishwa katika ukweli}. \textbf{Kulikuwa na wale waliokuja na mafundisho ya ajabu, lakini hatukuogopa kukutana nao. Uzoefu wetu ulidhibitishwa kwa ajabu na mafunuo ya Roho Mtakatifu}.}[Lt253-1903.5; 1903][https://egwwritings.org/read?panels=p9980.11]


\egwnogap{Kwa miaka miwili au mitatu akili yangu iliendelea kufungwa kwa Maandiko. Mnamo 1846 niliolewa na Mzee James White. Ilikuwa muda fulani baada ya mtoto wangu wa pili kuzaliwa tulipokuwa katika mkanganyiko mkubwa kuhusu mambo fulani ya mafundisho. Nilikuwa nikimwomba Bwana afungue akili yangu, ili niweze kuelewa Neno Lake. Ghafla nilionekana kufunikwa na nuru nzuri, ya wazi, na tangu wakati huo, \textbf{Maandiko yamekuwa kitabu wazi kwangu}.}[Lt253-1903.6; 1903][https://egwwritings.org/read?panels=p14068.9980012]


\egwnogap{Wakati huo nilikuwa Paris, Maine. Baba Andrews alikuwa mgonjwa sana. Kwa muda alikuwa akiteseka sana kutokana na ugonjwa wa “inflammatory rheumatism”. Hakuweza kusogea bila maumivu makali. Tulimwombea. Niliweka mikono yangu juu ya kichwa chake, na kusema, “Baba Andrews, Bwana Yesu anakuponya.” Aliponywa papo hapo. Akainuka na kutembea chumbani, akimsifu Mungu, na kusema, “Sijawahi kuiona kwa namna hii hapo awali. Malaika wa Mungu wako katika chumba hili.” Utukufu wa Mungu ulifunuliwa. \textbf{Nuru ilionekana kuangaza nyumba nzima, na mkono wa malaika uliwekwa juu ya kichwa changu. Tangu wakati huo hadi sasa nimeweza kuelewa Neno la Mungu.}}[Lt253-1903.7; 1903][https://egwwritings.org/read?panels=p9980.13]

\egwnogap{\textbf{Baada ya wakati kupita mnamo 1844, tulipingwa na kutafsiriwa vibaya kwa ukatili. Nadharia za uongo zilisongwa ndani yetu na wanaume na wanawake ambao walikuwa wameingia katika ushupavu}. Nilielekezwa kwenda mahali ambapo watu hawa walikuwa wakitetea nadharia hizi za uongo, na nilipokwenda, nguvu za Roho zilionyeshwa kwa ajabu katika kukemea makosa yaliyokuwa yakianza kuingia. \textbf{\underline{Shetani mwenyewe, katika utu wa mwanadamu}, alikuwa akifanya kazi ili kufanya ushuhuda wangu usiwe na athari kuhusu msimamo ambao sasa tunajua unathibitishwa na Maandiko.}}[Lt253-1903.8; 1903][https://egwwritings.org/read?panels=p9980.14]

\egwnogap{\textbf{Nadharia kama hizo kama ulizowasilisha katika Living Temple ziliwasilishwa wakati huo. Mafundisho haya yenye hila na yenye kudanganya yametafuta tena na tena kupata mahali kati yetu. \underline{Lakini ninao ushuhuda uleule wa kutoa ambao sasa ninautoa kuhusu Umbile la Mungu}}.}[Lt253-1903.9; 1903][https://egwwritings.org/read?panels=p9980.15]

\egwnogap{Katika (Maandiko ya Awali, 60, 66, 67)\footnote{Inaonekana kwamba kurasa hizo si sahihi. Aya zilizotajwa zinaweza kupatikana katika Maandiko ya Awali kwenye kurasa \href{https://egwwritings.org/read?panels=p28.462&index=0}{70.2}, \href{https://egwwritings.org/read?panels=p28.490&index=0}{77}, na \href{https://egwwritings.org/read?panels=p28.390&index=0}{54.2}.}, kuna taarifa zifuatazo:}[Lt253-1903.10; 1903][https://egwwritings.org/read?panels=p9980.16]

\egwnogap{‘Mei 14, 1851, niliona uzuri na raghba ya Yesu. Nilipouona utukufu Wake, wazo haikunijia kwamba ningetengwa na uwepo wake. \textbf{Niliona mwanga unanurishwa kutoka kwa utukufu uliomzunguka Baba}, na ulipokaribia nami, mwili wangu ulitetemeka na kuteterekeka kama jani. Nilifikiri kwamba ikiwa ingekuja karibu nami, ningetolewa uhai; lakini nuru ilinipita. \textbf{Basi ndipo niliweza kuwa na hisia ilhali kwa kadri ya Ukuu na Utishi wa \underline{Mungu} ambaye tunapaswa kufanya naye}.’}[Lt253-1903.11; 1903][https://egwwritings.org/read?panels=p9980.17]

\egwnogap{‘\textbf{Mara nyingi nimemwona Yesu mpendwa, kwamba Yeye ni Nafsi}. \textbf{Nilimuuliza kama Baba Yake alikuwa Nafsi, na alikuwa na \underline{umbo} kama Yeye Mwenyewe}. Yesu alisema, ‘\textbf{Mimi ni chapa kamili ya Umbile Wake}!’ [Waebrania 1:3.]}[Lt253-1903.12; 1903][https://egwwritings.org/read?panels=p9980.18]

\egwnogap{‘\textbf{Mara nyingi nimeona kwamba mtazamo wa umizimu uliondoa utukufu wote wa mbinguni, na katika fikira za waja ainati kiti cha enzi cha Daudi na nafsi ya kuvutia ya Yesu vimeteketezwa katika moto wa umizimu}. Nimeona kwamba baadhi wamedanganywa na kuongozwa katika hili makosa, watadumishwa katika nuru ya ukweli, \textbf{lakini itakuwa muhali hata kuwe kutokuwezekana kwao kuondoa kabisa nguvu za udanganyifu za umizimu. Wao wanapaswa kwa juhudi za kina kuungama makosa yao, na kuyaacha milele}.’}[Lt253-1903.13; 1903][https://egwwritings.org/read?panels=p9980.19]

\egwnogap{\textbf{Kuna aina ya umizimu \underline{inayoingia} miongoni mwa watu wetu, na \underline{itadhoofisha imani} ya wale wanaoipa nafasi, inawaongoza kuzitii roho zidanganyazo pamoja na mafundisho ya mashetani}. Makosa yatawasilishwa kwa njia ya kupendeza na ya kufurahisha. Adui anatamani kugeuza mawazo ya kaka na dada zetu kutoka kwa kazi ya kuandaa watu kusimama katika siku hizi za mwisho.}[Lt253-1903.14; 1903][https://egwwritings.org/read?panels=p9980.21]

\egwnogap{Nimeagizwa kuwaonya ndugu na dada zetu \textbf{wasijadili asili ya Mungu wetu}. Wengi wa wale wenye tamaa na mwasho walipojaribu kulifungua sanduku la agano, waone kilichokuwa ndani, waliadhibiwa kulingana na ghururi lao. \textbf{Hatupaswi kusema kwamba Bwana Mungu wa mbinguni yuko ndani ya jani, au kwenye mti; kwa maana hayupo papo. \underline{Ameketi juu ya kiti chake cha enzi mbinguni}}.}[Lt253-1903.15; 1903][https://egwwritings.org/read?panels=p9980.22]

\egwnogap{Kazi ya Muumba inavyoonekana katika maumbile hufichua uwezo Wake. Lakini asili haichukui nafasi ya Mungu, wala Mungu hayupo katika asili jinsi wengine wanavyomwakilisha. Mungu aliumba ulimwengu, lakini hata hivyo ulimwengu sio Mungu; ila ni kazi ya mikono yake. \textbf{Asili hufichua kazi ya Mungu aliye chanya na \underline{wa kibinafsi}, kuonyesha kwamba Mungu yuko, na kwamba Yeye ni mthawabishaji wa wale wanaomtafuta kwa bidii}.}[Lt253-1903.16, 1903][https://egwwritings.org/read?panels=p9980.23]

\egwnogap{Ningeweza kusema mengi kuhusu patakatifu; sanduku lenye sheria ya Mungu; kifuniko cha sanduku, ambalo ni kiti cha rehema; malaika katika pande za safina; na mambo mengine kuunganishwa na patakatifu pa mbinguni na siku kuu ya upatanisho. Ningeweza kusema mengi kuhusu mafumbo ya mbinguni; lakini midomo yangu imefungwa. Sina mapendekezo kujaribu kueleza kuhusu kwayo.}[Lt253-1903.17; 1903][https://egwwritings.org/read?panels=p9980.25]

\egwnogap{\textbf{Singethubutu kusema kuhusu Mungu kama ulivyonena juu Yake}. Yuko juu na ameinuliwa, na utukufu wake unazijaza mbingu. “Sauti ya Bwana ina nguvu; inatikisa mierezi ya Lebanon. \textbf{Bwana yu katika hekalu lake takatifu}; dunia yote na ikae kimya mbele zake.” [Ona Zaburi 29:5; Habakuki 2:20.]}[Lt253-1903.18; 1903][https://egwwritings.org/read?panels=p9980.26]

\egwnogap{\textbf{Ndugu yangu, unapojaribiwa kusema kuhusu Mungu, \underline{mahali alipo, au kile Alicho}, kumbuka kwamba katika hatua hii ukimya ni ufasaha. Vua viatu vyako kutoka kwa miguu yako; kwa maana ardhi hiyo unayoiweka miguu yako ya kutokumakinika na isiyotakaswa, ni ardhi palipo takatifu.}}[Lt253-1903.19; 1903][https://egwwritings.org/read?panels=p14068.9980027]

\egwnogap{\textbf{Nimeagizwa kusema kwamba hakuna kitu katika Neno la Mungu kuthibitisha nadharia zako za umizimu. Je, hutazikana nadharia hizi mara moja? Nadharia hizi zimekaa mawazoni mwako kwa muda mrefu, lakini hazijakuwa zenye utakaso, usafishaji, zenye ushawishi wa kuboresha maishani mwako. Bwana hana matumizi kwa nadharia hizi, na Yeye asingetaka watu wake wazitetee au kuzieneza.}}[Lt253-1903.20; 1903][https://egwwritings.org/read?panels=p9980.28]

\egwnogap{\textbf{Baba, Mjuzi wa yote, aliumba ulimwengu \underline{kupitia} Kristo Yesu}. Kristo ndiye nuru ya ulimwengu, njia ya uzima wa milele. Yeye, aliyetiwa mafuta, Mungu alimtoa kufanya upatanisho kwa ajili ya dhambi za ulimwengu. Unahitaji kuelewa kwamba usipokuwa \textbf{mwaamini katika upatanisho huo}, na ujue ya kuwa umenunuliwa kwa bei ya damu ya \textbf{Mwana wa Mungu mzaliwa wa kipekee}, hakika utajumuishwa na yule mwovu. \textbf{Ukiendelea kuthamini nadharia ambazo umekuwa ukithamini, utaachwa kuchezeshwa kwa Majaribu ya Shetani}. Anacheza mchezo wa maisha kwa nafsi yako. Zidi katika utangamano naye kwa muda kidogo tu, na uwe na hakika kwamba utaipoteza nafsi yako.}[Lt253-1903.21; 1903][https://egwwritings.org/read?panels=p9980.29]

\egwnogap{Kwa kutangaza kwamba taasisi zetu zimetenganishwa na madhehebu zetu, umeweka watu wetu na kazi yetu katika nafasi ya uongo. Umeongozwa kwenye njia mbaya, hatari ambayo kwayo wewe hukuyang'amua, lakini wakati mwingine unaweza kuona. Bado hakujachelewa kwa makosa kusahihishwa. Panapo matumaini kwako. \textbf{Umemfuata adui hatua kwa hatua, ukijitahidi kuangalia ndanimo siri zilizo juu sana na takatifu wa kupindukia kupiku ufahamu wako}. \textbf{Kisha katika mafundisho yako Nafsi Yule Mtakatifu ameshushwa hadi hadhi ya mawazo ya mwanadamu ya \underline{kisayansi, ya kimizimu}}. Umekuwa katika kutembea kwa njia potofu. Umepoteza sura ya maadili ya Mungu. Lakini kuna tumaini kwako. Bado unaweza kugeuza miguu yako kuwa njia sahihi. Je! hutafanya sasa njia ziwe zilizonyooka kwa ajili yako? viwete wasije wakapotoshwa? Je, utakataa sasa kupanda mbegu moja zaidi ya mashaka na ujanja katika akili za wengine? Je, utakuja kwa Kristo sasa hivi na kupata uponyaji?}[Lt253-1903.22; 1903][https://egwwritings.org/read?panels=p14068.9980030]

\egwnogap{\textbf{Nimesitasita na kuchelewesha kupeleka yale ambayo Roho wa Mwenyezi Mungu Bwana amenisukuma kuandika}. Sikutaka kulazimishwa kuwasilisha ushawishi wa kishetani wa hizi nadharia za kupotosha. Lakini isipokuwepo mabadiliko ya dhati, ndani yako na washirika wenza, itabidi nifanye hivi, ili kuwaokoa wengine wasifuate njia ambayo umekuwa ukifuata. Itanibidi kutii amri niliyopewa na Mungu, “\textbf{Kutana nayo}.” Hili ndilo jambo pekee ambalo ninaweza kulifanya.}[Lt253-1903.23; 1903][https://egwwritings.org/read?panels=p9980.31]

\egwnogap{Ninakuwasilishia mambo ambayo Bwana amenionyesha. Kuna kazi kubwa ambayo yanapaswa kutendeka. Tunapaswa kushika kazi kwa ufahamu, kwa kuomba, kwa kuamini, na kwa kupokea Roho takatifu. Hivyo tu ndivyo tunaweza kufanya kazi tuliyopewa. \textbf{Nimetiwa jukumu na Mungu kuwasilisha ushuhuda dhidi ya Living Temple}. Chochote ambacho washirika wako wanaweza kusema kuhusu hiki kitabu,\textbf{ ninachukua msimamo sasa na milele kwamba ni mtego}. \textbf{Hakuna muungano utakaoundwa na watu wetu kwa ujumla kujiunga na \underline{nadharia} ambazo umeanza kuwasilisha katika kitabu hicho}. \textbf{Unaweza kuzingatia hili kama lililoamuliwa milele}. \textbf{Kama watu tutasimama imara \underline{juu ya jukwaa ambalo limehimili mtihani na majaribio}. Tutashikilia \underline{nguzo za imani yetu}. \underline{Kanuni za ukweli} ambazo Mungu ametufunulia ndio msingi wetu pekee. Zimetukuza kwa jinsi tulivyo sasa. Nadharia hizi mpya, za kudhaniwa tu zinavutia na kupotosha. Zinahatarisha maslahi ya milele ya nafsi. Maandiko hayazitegemezi}. Tukiwa tumevaa silaha za Kikristo, tukiwa tumevaa matayarisho ya Injili ya amani tutasimama kidete \textbf{dhidi ya nadharia hizi potofu}. Unaweza kugeuka na kulipotosha Neno la Mungu kwa uharibifu wako mwenyewe, lakini nakusihi usifanye hivi.}[Lt253-1903.24; 1903][https://egwwritings.org/read?panels=p9980.32]

\egwnogap{\textbf{Mbingu si mvuke. Ni mahali}. \textbf{Kristo amekwenda kuandaa makao kwa ajili ya hao wampendao}, wale ambao, kwa kutii amri zake, wanajitenga na ulimwengu na wako watengwa. Kanuni za mbinguni lazima ziletwe katika uzoefu wetu, ili tutofautishwe na ulimwengu. \textbf{Lazima kuwe na tofauti kubwa kati yetu sisi na ulimwengu; kwa maana sisi tu watu wa Mungu walioteuliwa kwenye dhehebu lililo la kipekee}.}[Lt253-1903.25; 1903][https://egwwritings.org/read?panels=p9980.33]

\egwnogap{Bwana amekupa nafasi ya kurekebisha mambo. \textbf{Ninafurahi kwamba umefanya mwanzo. Usifikirie kuwa hatuna haki ya kujaribu kurekebisha makosa yako na matokeo ya makosa haya. Maadamu Mungu ananipa pumzi, na kuniagiza kutumia kalamu na sauti katika kulishambulia uovu hili lililoingia kati yetu, nitatenda majukumu yangu katika vita hivi. Tangu nilipokuwa na umri wa miaka kumi na saba, imenibidi kupigana vita hivi dhidi ya nadharia za uwongo, katika kutetea ukweli}. \textbf{Historia ya uzoefu wetu wa zamani umekita mizizi akilini mwangu, na nimedhamiria kuwa \underline{hakuna nadharia zozote zile sawia na ambazo wewe umekuwa ukizikubali} zitaingia katika safu zetu}. Ikiwa unakataa kubadilika, na kufanya kazi ya kuongoza washirika wako baada yako, nao wakathubutu kufuata uongozi wako, uwajibikaji unabakia pamoja nawe na pamoja nao, si juu ya nafsi yangu.}[Lt253-1903.26, 1903][https://egwwritings.org/read?panels=p9980.34]

\egwnogap{\textbf{Nazungumza kwa dhati, ili kwamba mpate kujua, kama hakuna mabadiliko yaliyo ya dhati kwako, hakuwezi kuwa na tumaini la muungano kati yako na wale wanaoshikilia mwanzo wa imani yao imara hadi mwisho.} Umefanya mgawanyiko. \textbf{\underline{Hatuna budi kusimama imara kwa ajili ya kweli ambazo Bwana ametupa kama nguzo za imani yetu}}.}[Lt253-1903.27; 1903][https://egwwritings.org/read?panels=p9980.35]

\egwnogap{Nakusihi umrudie Bwana kwa kusudi kamili la moyo, kabla haijachelewa milele. Jitenge na mvuto ambao umekutenganisha na ndugu zako ambao wanashiriku katika huduma ya injili na kutoka kwa watu ambao Mungu anawaongoza. \textbf{\underline{Nadharia za viraka} haziwezi kukubaliwa na wale ambao ni waaminifu kwa imani na kwa \underline{kanuni} ambayo yamestahimili upinzani wote wa uvutano wa kishetani}.}[Lt253-1903.28; 1903][https://egwwritings.org/read?panels=p9980.36]

\egwnogap{Ikiwa utayaondoa nafsini mwako yale yote ambayo yamekutenganisha na Kristo, na kumpokea Mwokozi ndani ya moyo wako, utabadilishwa katika tabia. Acha majukumu kwa muda, na nenda zako mahali fulani pamoja na wachache wa ndugu zako, na pamoja nao uyachunguze Maandiko. Nyenyekeza moyo wako mbele za Bwana, na ufanye kazi kamili kwa ajili ya toba. \textbf{Dini ya Kristo ndiye chachu ya kiroho ambayo inapaswa kuingizwa ndani ya moyo. Hii inabadilisha maisha na tabia}. Dini hii ni kanuni ya mbinguni, inayoonekana katika maisha ya Mkristo na mazungumzo yake. Inafunuliwa katika usafi wa Kikristo. Upendo wa Kristo unaonekana katika upole na neema ya wanadamu waliotakaswa. Ni kwa Neno lililofanyika mwili tunaokolewa. Ukombozi wetu ulifanywa, \textbf{si kwa Mwana wa Mungu kubaki mbinguni, bali kwa Mwana wa Mungu kufanyika mwili—kuchukua ubinadamu juu Yake na kuja kwa hii dunia}. Hivyo uzima wa milele uliletwa kwetu. Yale ambayo mamlaka, amri, na ahadi haingeweza kufanya, Mungu alifanya kwa kuja katika ulimwengu huu katika mfano wa mwili wenye dhambi.}[Lt253-1903.29; 1903][https://egwwritings.org/read?panels=p9980.37]

\egwnogap{Kristo alikuja duniani kuishi kama mwanadamu miongoni mwa wanadamu, si kuharibiwa na udhaifu wa kibinadamu, bali kuweka katika mawazo ya wanadamu kanuni za ukweli ambazo hazingeweza kufutika, kwa sababu ni zile zilizoko kweli milele. Alikuja kuleta maisha mapya kwa wanadamu walioanguka—ubora huo hauwezi kuchafuliwa au kuharibiwa na dhambi.}[Lt253-1903.30; 1903][https://egwwritings.org/read?panels=p9980.38]

\egwnogap{\textbf{Ndugu yangu, lazima nikuambie kwamba huna utambuzi toshelezi wa mahali miguu yako imekuwa ikielekea}. Umekuwa ukijifunga kwa utangamano na wale walio wa jeshi la muasi mkuu. \textbf{Akili yako imekuwa giza kama Misri}. \textbf{Ikiwa utaanguka kwenye Mwamba na kuvunjika}, Kristo atakukubali. Lakini umekuwa ukisimama kwenye ardhi ya adui, ukifanya kazi yake. \textbf{Ulimwengu wa kidini unaelekea kwa kasi katika njia ile ile ambayo umekuwa ukifuata. Ukiendelea kufuata barabara hii, utakuwa na watu wengi. Lakini mwishowe utakuwa nini?}}[Lt253-1903.31; 1903][https://egwwritings.org/read?panels=p14068.9980039]

\egwnogap{Kwa muda mrefu sana umetembea gizani, umefuata njia yako mwenyewe kwa muda mrefu unaweza kujaribiwa kwa nguvu kupinga ombi hili ninalotoa. Isingekuwepo maslahi yako ya \textbf{milele yanahusika}, nisingezungumza nawe juu ya mada hii. Inaweza kuonekana kuwa nimeandika vya kutosha, kwamba hakuna haja ya mimi kuhimiza somo hili kwako zaidi. \textbf{Lakini nakuambia kwa ukweli kwamba ninaelewa wazi kile ninachofanya}. Nuru ya kutosha umepokezwa. Lakini kwa miaka kadhaa haujazingatia nuru hii. Ikiwa ungetaka kujua ni nini Bwana amesema, ungalijua; \textbf{kwa kuwa una vitabu vilivyoandikwa chini ya uongozi wa Roho wake}. Umekuwa na maelekezo yote ambayo kwayo yanaweza kuhitajika ili kuonyesha njia sahihi. Nuru ya moja kwa moja imetumwa kwako. Lakini umetazama hii kama yenye umuhimu wa kadiri ndogo kuliko mipango na mawazo yako. Ikiwa ungetii shuhuda zilizotumwa kwako, Living Temple halingeandikwa kamwe.}[Lt253-1903.32; 1903][https://egwwritings.org/read?panels=p9980.40]

\egwnogap{Je, hutafanya jitihada kamili, iliyotanda, kama ya Kikristo ili kuvunja nguvu za uchawi ambazo Shetani ameweka juu yako? Amekuwa na uwezo mkubwa juu ya akili yako na amekuyumbisha katika njia zisizo sahihi. Anadhani kwamba anaweza kukushikilia sasa. Je, hutamshinda na kumkatisha tamaa?}[Lt253-1903.33; 1903][https://egwwritings.org/read?panels=p9980.41]

\egwnogap{Nakuandikia wewe kama vile ningemwandikia mwana yeyote yule. Achana na adui—mshitaki wa ndugu. Mwambie, “Nenda nyuma yangu Shetani. Nimefanya dhambi kubwa kwa kusikiliza mapendekezo zako. Hakika sitazisikiliza tena.” Ninakuomba, kwa ajili ya nafsi yako, kupinga mjaribu, ili aepukane nawe. Umkaribie Mungu, naye atakukaribia wewe. \textbf{Wewe utapoteza mbingu usipoanguka juu ya Mwamba na kuvunjika}.}[Lt253-1903.34; 1903][https://egwwritings.org/read?panels=p9980.42]


Mambo mengi katika barua hii kwa Dk. Kellogg huenda bila kuelezwa, lakini yanawekwa wazi ikiwa muktadha umebainishwa. Ellen White alisoma barua kutoka kwa Ndugu Daniells akieleza jinsi Dk. Kellogg alitaka kurekebisha Living Temple kwa sababu \others{alikuwa akifikiria jambo hilo tena, na akaanza kuona kwamba alikuwa amefanya makosa kidogo katika \textbf{kutoa} maoni yake}, na\others{hivyo kwa muda mfupi \textbf{alikuwa amesadiki imani katika utatu} na sasa aliweza kuona vizuri sana pale ambapo ugumu wote ulikuwepo, na aliamini kwamba angeweza kuliondoa jambo hilo tatanishi kwa njia ya kuridhisha}. Kellogg alikiri,\others{kwamba sasa aliamini \textbf{katika Mungu Baba, Mungu Mwana, na Mungu Roho Mtakatifu}}. Ili kujibu hilo, Dada White alimwandikia hivi kibinafsi:\egwinline{Kitabu Living Temple \textbf{hakipaswi kuwekwa viraka}, mabadiliko machache kufanyiwa ndani yake, na kisha kutangazwa na kusifiwa kama uzalishaji wa thamani}. Je, Kellogg alitaka kuweka viraka kitabu chake vipi? Kulingana na ushuhuda wa A. G. Daniells, alifikiria kubadili maneno machache kwa kusema waziwazi hisia zake za utatu. Lakini usemi wa maoni haukuwa shida halisi—ilikuwa maoni yenyewe. Dada White hakuacha kumkemea kwa maoni yake juu ya Mungu, ambayo yalikuwa maoni ya utatu. Alimwambia kwamba \egwinline{\textbf{amedhamiria kwamba \underline{hakuna nadharia sawia na zile ambazo amekuwa akikubali} zitapata nafasi katika safu zetu}}. Hii ni kauli kali sana. Je, inaweza kuwa hivyo, kwa kuwa Kellogg alikiri kwamba alikuwa akikubali fundisho la Utatu, Dada White pia alikuwa akiijumuisha kwenye taarifa yake? Inaonekana kuwa haiwezekani kwa sababu fundisho hili liko ndani safu zetu leo. Lakini kauli yake kwa kweli inakazia Utatu aliposema:\egwinline{\textbf{Nadharia za viraka} haziwezi kukubaliwa na wale ambao ni waaminifu \textbf{kwa imani na kwa kanuni} ambazo zimestahimili upinzani wote wa mivuto ya kishetani}. Kellogg alitaka kurekebisha “Living Temple” kwa kutaja waziwazi fundisho la Utatu. Kwa nini Dada White alidhamiria kuliweka fundisho hili nje ya safu zetu, ilhali liko katika safu zetu leo? Ni sawa kusema kwamba Utatu haukuwa sehemu ya imani ya Waadventista Wasabato wakati wake na ulikuja katika safu zetu baadae. Leo, wengi wanasema kwamba ni kwa sababu ya kazi zake ndiposa Utatu ni sehemu ya imani yetu, lakini majibu ya Ellen White, na jibu lake kwa imani ya Kellogg juu yake, inaonyesha jinsi alishughulika na fundisho kama hilo. Tunaweza kujifunza nini kutokana na hilo?


Ikichukuliwa katika muktadha wake, barua hii inatoa mwanga mpya juu ya utata wa Kellogg na kuonyesha jinsi tunavyopaswa kushughulika na fundisho la Utatu. Jambo la kwanza tunalohoji ni kwa nini Dada White hakuwahi kutumia neno “Utatu” katika maandishi yake, hata alipokuwa akishughulikia fundisho hili moja kwa moja? Mahali pengine anajibu:


\egw{Nilitahadharishwa nisiingie kwenye mabishano \textbf{kuhusiana na swali} ambalo \textbf{\underline{litajitokeza}} juu ya \textbf{mambo haya, kwa sababu mabishano \underline{yanaweza kusababisha watu kutumia hila, na akili zao zingeongozwa mbali na kweli ya Neno la Mungu hadi kwenye dhana na kazi ya kubahatisha}}. \textbf{Kadiri nadharia potofu zinavyojadiliwa, ndivyo \underline{wanadamu watakavyojua kiasi kuhusu Mungu na wa ile kweli inayotakasa nafsi}}.}[Lt232-1903.41; 1903][https://egwwritings.org/read?panels=p14068.10197050]


Hili ni somo muhimu sana na kanuni ambayo Dada White anatufundisha hapa. Wakati mzozo juu ya nadharia za Kellogg uliibuka, hakujitosa kwenye nadharia zenyewe, kwa sababu hii ingeongoza mawazo ya watu kutoka kwenye kweli ya Neno la Mungu hadi dhana na kazi ya kubahatisha. Badala yake, aliongoza mawazo ya wanadamu katika kweli, ambayo hutakasa nafsi. Aliongoza kwa mfano, jinsi umedhihirishwa hapa katika barua yake kwa Dk. Kellogg. Ukweli huu ambao kwayo aliouongoza akili za watu, ulikuwa ukweli juu ya \emcap{Umbile la Mungu}. Alimkemea Kellogg kwa ajili ya nadharia zake lakini, cha muhimu sana, tunatambua nadharia hizi ipasavyo kwa muktadha wao na kwa maelezo yake yaliyofiche kwao.


Tunaona kwamba alifanya kutofautisha kati ya Utatu na Umbile la Mungu. Alifanya kutofautisha kati ya kanuni za zamani za imani yetu na nadharia mpya. Kwanza, alivuta mawazo yetu nyuma kwenye mwanzo wa urithi wetu wa kiroho,\egwinline{baada ya kupita kwa wakati mnamo 1844}, wakati mumewe James White, Joseph Bates, Baba Pierce, Mzee Edson, na wengi wengine waliokuwa makini, waungwana, na wa kweli, walitafuta ukweli. Yeye aliashiria hapo awali kwa uzoefu wa ajabu na wenye nguvu jinsi pointi kuu za imani yetu, zilizoshikiliwa mwaka wa 1903, zilivyowekwa msingi imara. \egwinline{\textbf{Hivyo \underline{pointi kuu za imani yetu}} tunavyozishikilia leo ziliwekwa msingi imara.} \egwinline{\textbf{\underline{Hoja baada ya hoja} ilifafanuliwa wazi, na ndugu wote wakaingia maelewano}.} \egwinline{\textbf{Kundi zima la waumini liliunganishwa katika ukweli}}. Ni wazi, kutoka kwa muktadha wa sura ya 10 ya Shuhuda Maalum, tunajua kwamba uzoefu huu unaeleza \egwinline{\textbf{jinsi msingi wa imani yetu ulivyowekwa imara}}[SpTB02 56.4; 1904][https://egwwritings.org/read?panels=p417.288]. Msingi huu unaonyeshwa wazi ndani ya Kanuni za Msingi\footnote{\href{https://static1.squarespace.com/static/554c4998e4b04e89ea0c4073/t/59d17e24c027d84167e17617/1506901547915/SDA-YB1905+\%28P.+188-192\%29.pdf}{Yearbook Of Seventh-day Adventist denomination 1905, p. 188-192}}. Msingi huu ni ukweli ambao,\egwinline{\textbf{\underline{hatua kwa hatua}}, \textbf{umetafutwa kwa kujifunza kwa maombi, na kushuhudiwa kwa uwezo wa utendaji wa miujiza wa Bwana}}. Mungu \egwinline{\textbf{anatuita kushika kwa uthabiti, kwa mshiko wa imani, \underline{kanuni za msingi} ambazo \underline{zimetegemezwa kwa mamlaka isiyotiliwa shaka}}.}[SpTB02 59.1; 1904][https://egwwritings.org/read?panels=p417.299] Kwa kuzingatia uzoefu huu na ukweli unaoonyeshwa katika Kanuni za Kimsingi, \egwinline{\textbf{\underline{Nadharia za viraka} haziwezi kukubaliwa na wale ambao ni waaminifu \underline{kwa imani} na \underline{kwa kanuni} ambazo zimepinga upinzani wote wa ushawishi wa kishetani}}[Lt253-1903.28; 1903][https://egwwritings.org/read?panels=p14068.9980036]. Kutoka kwa kumbukumbu ya kihistoria ya ndugu hawa ambao walikuwa makini, waungwana na wa kweli, tunao uthibitisho kwamba wao pia wametofautisha kati ya fundisho la Utatu na ukweli juu ya Ubinafsi wa Mungu. James White, katika makala ya Review and Herald, aliorodhesha \others{baadhi ya hekaya maarufu za enzi hizi}, akisema: \others{Hapa tunaweza kutaja \textbf{Utatu, ambao \underline{unaondoa Umbile la Mungu, na la Mwanawe Yesu Kristo}}}[James White, Review \& Herald, December 11, 1855, p. 85.15][http://documents.adventistarchives.org/Periodicals/RH/RH18551211-V07-11.pdf]. J. N. Andrews alisema, \others{\textbf{Fundisho la Utatu ambalo lilianzishwa katika kanisa na Baraza la Nicea, A. D. 325}. \textbf{Fundisho hili \underline{linaharibu Umbile la Mungu, na Mwana wake Yesu Kristo Bwana wetu}}...}[J. N. Andrews, Review \& Herald, March 6, 1855, p. 185][http://documents.adventistarchives.org/Periodicals/RH/RH18550306-V06-24.pdf] J. B. Frisbie, katika makala yake “\textit{Seventh-day Sabbath not abolished}”, analinganisha Mungu wa Sabato na mungu wa Jumapili; anaelezea Mungu wa Sabato katika mwanga wa Umbile la Mungu unaoonyeshwa katika hoja ya kwanza ya Kanuni za Msingi. Mungu wa jumapili anaelezewa na \others{umoja wa huyu Mungu mkuu, kuna nafsi tatu za nafsi asili moja, nguvu na umilele; Baba, Mwana, na Roho Mtakatifu}[J. B. Frisbie, Review \& Herald March 7, 1854. p. 50][http://documents.adventistarchives.org/Periodicals/RH/RH18540307-V05-07.pdf]. Alieleza jinsi fundisho la Umbile la Mungu linapingana na fundisho la Utatu, katika hali hiyo hiyo jinsi Sabato Takatifu inavyosimama kinyume na ibada ya kipagani ya Jumapili. Pia, ndugu J. N. Loughborough aliandika pingamizi kwa fundisho la Utatu katika Adventist Review and Sabbath Herald\footnote{\href{https://adventistdigitallibrary.org/adl-349160/advent-review-and-sabbath-herald-november-5-1861}{J. N. Loughborough, November 5, 1861, Review \& Herald, vol. 18, p. 184, par. 1-11}}. Katika uchapishaji mwingine wa Review and Herald, alichapisha makala hiyo “\textit{Is God a person?}”, akieleza msimamo wa imani ya Waadventista Wasabato kuhusu Umbile la Mungu, iliyoelezwa katika hoja ya kwanza ya Kanuni za Msingi\footnote{\href{http://documents.adventistarchives.org/Periodicals/RH/RH18550918-V07-06.pdf}{J. N. Loughborough, September 18. 1855, Review \& Herald, vol. 7, p. 6.}}. James White pia alikuwa akifafanua msimamo huo katika kijitabu chake kilichopokea uchapishaji nyingi, “\textit{The Personality of God}”\footnote{\href{https://egwwritings.org/?ref=en_PERGO.1.1&para=1471.3}{J. White, The Personality of God, June 18. 1861.}}. Haya ni mifano tu ambapo waanzilishi wa Kiadventista walieleza msimamo juu ya Umbile la Mungu unaoonyeshwa na kipengele cha kwanza ya Kanuni za Msingi.

Dada White alimkemea Kellogg:\egwinline{\textbf{Lakini ninakuambia kwa ukweli kwamba ninaelewa waziwazi kile ninachofanya}. \textbf{Nuru ya kutosha umepewa wewe}. Lakini kwa miaka kadhaa haujazingatia nuru hii. Kama ungetaka kujua kile ambacho Bwana amesema, ungalijua; \textbf{kwa kuwa unazo \underline{vitabu} vilivyoandikwa chini ya uongozi wa Roho wake}. Umekuwa na maelekezo yote ambayo yanaweza kuhitajika ili kuonyesha njia sahihi. Nuru ya moja kwa moja umepokezwa wewe. Lakini umeona hili kama lisilo na umuhimu kuliko mipango yako mwenyewe na miundo yako. Kama ungetii shuhuda zilizotumiwa wewe, Living Temple haingeandikwa kamwe.}[Lt253-1903.32; 1903][https://egwwritings.org/read?panels=p14068.9980040] Suala la msingi la mgogoro wa Dk. Kellogg lilikuwa \egwinline{Umbile la Mungu na uwepo wake ulipo}[SpTB02 51.3; 1904][https://egwwritings.org/read?panels=p417.262]. Dk. Kellogg alipata kupokea maandishi ya waanzilishi, vitabu na Kanuni za Msingi za kanisa ambazo zilishuhudiwa na nguvu ya utendaji wa miujiza ya Roho Mtakatifu.


Dada White alikumbuka uzoefu wa jinsi \textit{pointi kuu za imani yetu}, kama zilivyoshikiliwa nyakati za zamani, ziliwekwa imara.\egwinline{\textbf{\underline{Hoja baada ya hoja} ilifafanuliwa wazi, na ndugu wote wakaingia maelewano}}[Lt253-1903.4; 1903][https://egwwritings.org/read?panels=p14068.9980010]. Pointi hizi kuu zilikuwa Kanuni za Msingi, ambayo Umbile la Mungu lilikuwa mojawapo. Jambo hili, na ushuhuda wa Dada White juu yake, ulibaki vile vile wakati wa maisha yake. Alisema\egwinline{\textbf{\underline{Ninao ushuhuda uleule wa kutoa ninaotoa sasa kuhusu Umbile la Mungu}}}[Lt253-1903.9; 1903][https://egwwritings.org/read?panels=p14068.9980015]. Kutoka Maandiko ya Awali, kisha alinukuu maono yake ya ukweli wa Mbinguni. Alikumbuka jinsi ilivyokuwa upendeleo kuwa mbele za Mungu, jinsi Mungu, akizungukwa na nuru ya utukufu wake, akapita pembeni mwake. Hakumwona Mungu kutoka kwenye nuru Aliyokuwa amezingirwa; alimwogopa Yeye, akifikiri kwamba kama Angemkaribia\egwinline{angetolewa uhai}. Kisha alimwona\egwinline{\textbf{Yesu mpendwa, kwamba Yeye ni Nafsi}. \textbf{Nilimuuliza kama Baba yake alikuwa Nafsi, na alikuwa na \underline{umbo kama} Yeye}. Yesu alisema, ‘\textbf{Mimi ni chapa kamili ya Umbile la Baba Yangu!}’}[Lt253-1903.12; 1903][https://egwwritings.org/read?panels=p14068.9980018]. Swali alilokuwa nalo lilikuwa: \textit{je, Mungu ni Nafsi, mwenye umbo kama Yesu?} Jibu lilikuwa la uthibitisho—likiwa na msingi thabiti wa kibiblia. Maono yake hayakuwa chanzo cha ukweli juu ya Umbile la Mungu; badala yake, walithibitisha ukweli ambao waanzilishi walikuwa wamegundua kwa kujifunza neno la Mungu kwa bidii.


Kwa hiyo, hitimisho lao la mwisho kuhusu Umbile la Mungu lilikuwa,\others{Kwamba kuna \textbf{Mungu mmoja}, \textbf{huluki binafsi wa kiroho}, \textbf{muumba wa vitu vyote}, muweza wa yote, mjuzi wa yote, na wa milele, asiye na kikomo katika hekima, utakatifu, haki, wema, ukweli, na rehema; asiyebadilika, na \textbf{uwepo wake uliopo kila mahali kupitia mwakilishaji wake, Roho Mtakatifu}. Zab. 139:7; Kwamba kuna Bwana Mmoja Yesu Kristo, \textbf{Mwana wa Baba wa Milele, ambaye kwa yeye aliumba vitu vyote, na vitu vyote hushikana katika yeye} …na kama sehemu ya mwisho ya kazi yake ya ukuhani, kabla hajatwaa kiti chake cha enzi kama mfalme, atafanya \textbf{upatanisho mkuu} kwa dhambi za hao wote, na dhambi zao zitafutwa (Matendo 3:19) na kuchukuliwa mbali na patakatifu, kama inavyoonyeshwa katika utumishi wa ukuhani wa Walawi, ambao ulionyesha kimbele na kufananisha huduma ya Mola wetu aliye mbinguni. Angalia Law. 16; Ebr. 8: 4, 5; 9: 6, 7; na kadhalika.}[Kipengele cha kwanza, na sehemu ya pili, ya Kanuni za Msingi, 1905.]


Ellen White alimkumbusha Dk. Kellogg juu ya hatua hii ya Kanuni za Msingi kwa kusema:\egwinline{\textbf{Baba, Mjuzi wa yote, aliumba ulimwengu \underline{kupitia} Kristo Yesu}. Kristo ndiye nuru ya ulimwengu, njia ya uzima wa milele. \textbf{Yeye, aliyetiwa mafuta, Mungu alitoa kufanya upatanisho kwa dhambi za ulimwengu}...}[Lt253-1903.21; 1903][https://egwwritings.org/read?panels=p14068.9980029]


Swali la Umbile la Mungu linahusika na ubora au hali ya Mungu kuwa Nafsi. Waanzilishi wa Kiadventista walitoa jibu kwalo na Mungu akaidhinisha kupitia maandishi ya Ellen White: Mungu ni \textit{huluki binafsi wa kiroho} na ni Baba yetu wa mbinguni. Uwepo Wake lipo wapi?\egwinline{\textbf{Hatupaswi kusema kwamba Bwana, Mungu wa mbinguni, yu ndani ya jani, au ndani ya mti; kwa kuwa Hayupo papo. \underline{Ameketi juu ya kiti chake cha enzi mbinguni}}.}[Lt253-1903.15; 1903][https://egwwritings.org/read?panels=p14068.9980022] \\
Uwepo wake uko kwenye kiti cha enzi mbinguni. \\
\egwinline{\textbf{Mbingu si mvuke. Ni mahali}. \textbf{Kristo amekwenda kuandaa makao kwa ajili ya hao wampendao}, wale ambao, kwa kutii amri zake, hujitenga na ulimwengu na wako watengwa...}[EGW, Lt253-1903.25; 1903][https://egwwritings.org/read?panels=p14068.9980033]. \\
“...\egwinline{‘Sauti ya Bwana ina nguvu; inaitikisa mierezi ya Lebanoni. \textbf{Bwana yu ndani ya hekalu Lake takatifu}; dunia yote na ikae kimya mbele zake.’ [Ona Zaburi 29:5; Habakuki 2:20.]}[Lt253-1903.18; 1903][https://egwwritings.org/read?panels=p14068.9980026]


Kulingana na waanzilishi wa Kiadventista na Dada White, Baba yetu wa mbinguni ni Mungu mmoja. Yeye ni Huluki binafsi wa Kiroho, aliyepo mbinguni, kwenye kiti Chake cha enzi. Kiti cha enzi cha mbinguni ni halisi, kiti cha enzi cha kimwili, ambacho juu yake ameketi Mtu halisi (Kuwa, mwenye umbo, kama Yesu)—Baba yetu wa mbinguni. Mahali hapo ni mahali halisi; sio mvuke, au mtazamo mwingine wowote wa kiroho.


\egwinline{\textbf{Mara nyingi nimeona kwamba mtazamo wa kimizimu uliondoa utukufu wote wa mbinguni, na kwenye mawazo ya waja wengi kiti cha enzi cha Daudi na Nafsi nzuri ya Yesu imeteketezwa katika moto wa umizimu}. Nimeona kwamba baadhi ya wale ambao wamedanganywa na kuongozwa katika makosa hili, watafunuliwa nuru iliyo ya ukweli, \textbf{lakini itakuwa karibu kutowezekana kwao kuondoa kabisa nguvu za udanganyifu wa umizimu. Hao wanapaswa kufanya kwa uangalifu kazi ya kuungama makosa yao, na kuyaacha milele}.}[Lt253-1903.13; 1903][https://egwwritings.org/read?panels=p14068.9980019]


Mtazamo wa kimizimu wa nafsi ya Mungu ni mtazamo potovu. Katika Biblia tuna ushuhuda wa mbinguni, kiti cha enzi cha mbinguni, na Mungu aketiye juu yake. Tukikubali shuhuda hizi katika maana yao iliyo wazi, basi fundisho la Utatu haliwezi kutegemezwa. Biblia na Roho ya Unabii inawasilisha Mungu mmoja mbinguni, kama huluki binafsi, mwenye mwili na umbo kama Yesu alivyo navyo. Mtazamo huu haupatani na fundisho la Mungu wa Utatu, kwani hilo linahitaji Roho Mtakatifu kuwa Huluki\footnote{Tafadhali angalia \hyperref[appendix:unauthenticated-reports]{kiambatisho} kwa nukuu zaidi ambazo zinaondoa Roho Mtakatifu kuwa huluki, mwenye mwili na umbo.}, mwenye mwili na umbo—wazo hili lingekuwa linahujumu Roho Mtakatifu kuwa njia ya Baba na Mwana ambayo kwayo wao wapo kila mahali. Ili kudumisha fundisho la Utatu, ushuhuda kuhusu kiti cha enzi cha Mungu na cha nafsi ya Mungu, kinahitaji kueleweka kwa mtazamo fulani wa kiroho. Hapa sisi tumeona kwamba Dada White alitofautisha ukweli wa Umbile la Mungu na fundisho la Utatu. Alilinganisha fundisho la Utatu na mambo mawili ya kwanza ya Kanuni za Msingi, ambazo zilikuwa matokeo ya waanzilishi wetu kujifunza Neno la Mungu. Akirejelea waanzilishi pamoja na Kanuni za Msingi, alisema: \egwinline{\textbf{\underline{Nadharia za viraka} haziwezi kukubaliwa na wale ambao ni \underline{waaminifu kwa imani na kwa kanuni} ambazo zimepinga upinzani wote wa ushawishi wa kishetani.}}[Lt253-1903.28; 1903][https://egwwritings.org/read?panels=p14068.9980036]


Hitimisho ni moja kwa moja na rahisi. Wale ambao ni waaminifu kwa imani, na kwa kanuni zilizopokelewa mwanzoni mwa kazi, hawawezi kukubali nadharia za viraka. Ikiwekwa katika muktadha, nadharia za viraka, ambayo ni fundisho la Utatu, haliwezi kukubaliwa na wale ambao wanashikilia sana \egwinline{\textbf{\underline{kanuni za msingi} ambazo \underline{zimetegemezwa kwa mamlaka isiyotiliwa shaka}}}[SpTB02 59.1; 1904][https://egwwritings.org/read?panels=p417.299]. Hitimisho hili huturudisha kwenye jaribio letu la kwanza lililopendekezwa la msingi wa imani yetu.


% The Patchwork Theories

\begin{titledpoem}
    
    \stanza{
        Truth established through earnest prayer, \\
        Points of faith discovered with care. \\
        Principles tested by time and trial, \\
        Stand firm against Satan's denial.
    }

    \stanza{
        Patchwork theories seek to sway, \\
        Those from the ancient, proven way. \\
        No revisions of truth we'll accept, \\
        The faithful path must be kept.
    }

    \stanza{
        The personality of God, a sacred revelation, \\
        Not subject to human innovation. \\
        Loyal hearts stand on ground that's sure, \\
        Where foundations eternal endure.
    }
    
\end{titledpoem}


Nadharia za viraka - Lt253-1903
“Ndugu Mpendwa,—“
“Lazima nikuambie kwamba mawazo yako kuhusu baadhi ya mambo yamepotoka pakubwa kwa hakika. Ni pendekezo langu hivi kwamba upate kuona dosari zako. Kitabu cha Hekalu Hai hakipaswi kuwekwa viraka, mabadiliko machache yafanywe ndanimo, na kisha kutangazwa na kusifiwa kama uzalishaji wa thamani. Na ingekuwa bora kuwasilisha sehemu za kifiziolojia katika kitabu kingine chini ya kichwa kingine. Wakati ule ulipoandika kitabu hicho, hukuwa chini ya mwongozo wa Mungu. Kando yako palikuwepo yule ali-yemwongoza Adamu kuwa Na mtazamo kuhusu Mungu kwa nuru uli-opotosha. Moyo wako wote upaswa kubadilishwa, iliyosafishwa kabisa na kwa ukamilifu.”
