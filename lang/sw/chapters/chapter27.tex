\qrchapter{https://forgottenpillar.com/rsc/en-fp-chapter27}{Steps to apostasy}


\qrchapter{https://forgottenpillar.com/rsc/en-fp-chapter27}{Hatua za uasi}


In the following quotation, brother J. N. Loughborough, who was one of the pioneers of the Seventh-day Adventist Church, warned us about the five steps to apostasy.


Katika nukuu ifuatayo, ndugu J. N. Loughborough, ambaye alikuwa mmoja wa waanzilishi wa Kanisa la Waadventista Wasabato, alituonya kuhusu hatua tano za uasi.


\others{\textbf{The} \textbf{first step} of apostasy is to \textbf{get up a creed}, telling us what we shall believe. \textbf{The second} is to \textbf{make that creed a test of fellowship}. \textbf{The third} is to \textbf{try members by that creed}. \textbf{The fourth} is to \textbf{denounce as heretics those who do not believe that creed}. And \textbf{fifth}, to \textbf{commence persecution against such}. I plead that we are not patterning after the churches in any unwarrantable sense in the step proposed.}[John N. Loughborough, Review and Herald, Oct. 8, 1861.][https://egwwritings.org/read?panels=p1685.5326]


\others{\textbf{Hatua} \textbf{ya kwanza} ya uasi ni \textbf{kusimamisha kanuni ya imani}, ukituambia kile tutakachokiamini. \textbf{Ya pili} ni \textbf{kufanya imani hiyo kuwa mtihani wa ushirika}. \textbf{Ya tatu} ni \textbf{kujaribu wanachama kwa hiyo imani}. \textbf{Ya nne} ni \textbf{kuwashutumu kama wazushi wale wasioamini imani hiyo}. Na \textbf{ya tano}, \textbf{kuanza mateso dhidi ya watu kama hao}. Ninawasihi kwamba hatufanyi muundo baada ya makanisa kwa maana yoyote isiyokubalika katika hatua iliyopendekezwa.}[John N. Loughborough, Review and Herald, Oct. 8, 1861.][https://egwwritings.org/read?panels=p1685.5326]


These principles are important to have in mind, and we ought to ask ourselves if we, today, are patterning after the churches in any unwarrantable sense in the step proposed. What would happen to a Seventh-day Adventist who would reject the Trinity doctrine in favor of the \emcap{Fundamental Principles}? Do we have a creed set up in our church? Do we test our membership by it?


Ni muhimu kuzingatia Kanuni hizi, na tunapaswa kujiuliza kama sisi, leo, tunaiga makanisa kwa maana yoyote isiyokubalika katika hatua iliyopendekezwa. Nini ingetokea kwa Mwaadventista wa Sabato ambaye angekataa fundisho la Utatu kwa kupendelea \emcap{Kanuni za Msingi}? Je! tunayo kanuni ya imani iliyoanzishwa katika kanisa letu? Je, tunaangazia uanachama wetu nayo?


The \emcap{Fundamental Principles} had a different nature and role in the Seventh-day Adventist Church contrary to that of the pattern held by other churches. The \emcap{Fundamental Principles} were not designed as a creed. In the preface of the 1872 statement, we read about their nature:


\emcap{Kanuni za Msingi} zilikuwa na asili na jukumu tofauti katika Kanisa la Waadventista Wasabato kinyume na utaratibu unaoshikiliwa na makanisa mengine. \emcap{Kanuni za Msingi} hazikuundwa kama imani. Katika utangulizi wa taarifa ya 1872, tunasoma juu ya asili yao:


\others{In presenting to the \textbf{public} this \textbf{synopsis of our faith}, we wish to have it distinctly understood that \textbf{\underline{we have no articles of faith, creed}, or discipline, \underline{aside from the Bible}}. We \textbf{do not} put forth this \textbf{\underline{as having any authority with our people}}, \textbf{nor is it designed to secure uniformity among them}, \textbf{as a system of faith}, \textbf{but is a brief statement of what is, and has been, with great unanimity, held by them}.}[A Declaration of the Fundamental Principles, Taught and Practiced by the Seventh-Day Adventists, 1872]


\others{Katika kuwasilisha kwa \textbf{umma} \textbf{muhtasari huu wa imani yetu}, tungependa ieleweke kwa uwazi kwamba \textbf{\underline{hatuna vifungu vya imani, kanuni za imani}, au nidhamu, \underline{kando na Biblia}}. \textbf{Hatusemi} haya \textbf{\underline{kuwa na mamlaka yoyote kwa watu wetu}}, \textbf{wala haikusudiwa kufanya hivyo kupata umoja kati yao}, \textbf{kama mfumo wa imani}, \textbf{lakini ni maelezo mafupi ya ni nini, na imekuwa, kwa umoja mkubwa, unaoshikiliwa nao}.}[A Declaration of the Fundamental Principles, Taught and Practiced by the Seventh-Day Adventists, 1872]


In the preface of the 1889 statement, we read similar sentiments:


Katika utangulizi wa taarifa ya 1889, tunasoma maoni yanayofanana:


\others{As elsewhere stated, Seventh-day Adventists \textbf{have no creed but the Bible}; but they hold to \textbf{certain well-defined points of faith}, for which they \textbf{feel prepared to give a reason ‘to every man that asketh’ them}. The following propositions may be taken as a summary of \textbf{the principal features of their religious faith}, upon which there is, so far as we know, \textbf{entire unanimity throughout the body}.}[Seventh-day Adventist Year Book of statistics for 1889, pg. 147, The Fundamental Principles of Seventh-day Adventists]


\others{Kama mahali penginepo inavyosema, Waadventista Wasabato \textbf{hawana imani ila Biblia}; lakini wanashikilia kwa \textbf{mambo fulani ya imani yaliyofafanuliwa vizuri}, ambayo \textbf{wanahisi kuwa tayari kutoa sababu ‘ya kila mtu anayewauliza’}. Mapendekezo yafuatayo yanaweza kuchukuliwa kama muhtasari wa \textbf{sifa kuu za imani yao ya kidini}, ambayo iko juu yake, kama tunavyojua, \textbf{umoja mzima katika mwili wote}.}[Seventh-day Adventist Year Book of statistics for 1889, pg. 147, The Fundamental Principles of Seventh-day Adventists]


The \emcap{Fundamental Principles} were not designed to dictate someone’s faith. The believers, led by the Holy Spirit, freely rendered their consciences to the Word of God; under the influence of the Holy Spirit, they came to the same conclusions. There was entire unanimity throughout the body. All believers felt “\textit{prepared to give a reason to every man that asketh them}” regarding their faith.


\emcap{Kanuni za Msingi} hazikuundwa ili kulazimisha imani ya mtu fulani. Waumini waliongozwa na Roho Mtakatifu, kwa hiari walitoa dhamiri zao kwa Neno la Mungu; chini ya ushawishi wa Roho Mtakatifu, walifikia maamuzi sawa. Kulikuwa na umoja mzima wa mwili mzima. Waumini wote waliona “\textit{wamejitayarisha kutoa sababu kwa kila mtu anayeuliza wao}” kuhusu imani yao.


Today we see a striking difference in the principles and practice of Adventist beliefs compared to our pioneers. We are keeping the spirit of unity by disciplining our members for the denial of the Fundamental Beliefs. In our church manual, under the section “\textit{Reason for Disciplines}”, we read the first point which states the discipline for denial of faith in the Seventh-day Adventist Fundamental Beliefs.


Leo tunaona tofauti kubwa katika kanuni na utendaji wa imani za Waadventista ikilinganishwa na waanzilishi wetu. Tunatunza roho ya umoja kwa kuwaadhibu washiriki wetu kukanusha Imani za Msingi. Katika mwongozo wetu wa kanisa, chini ya sehemu “\textit{Sababu ya Nidhamu}”, tunasoma nukta ya kwanza inayosema nidhamu ya kukataa imani katika Imani za Msingi za Waadventista Wasabato.


\others{Reasons for Discipline}


\others{Sababu za Nidhamu}


\others{1. \textbf{Denial of faith} in the fundamentals of the gospel and \textbf{in the Fundamental Beliefs of the Church} or \textbf{teaching doctrines contrary to the same}.}[SDA Church Manual, 20th edition, Revised 2022, p. 67][https://www.adventist.org/wp-content/uploads/2023/07/2022-Seventh-day-Adventist-Church-Manual.pdf]


\others{1. \textbf{Kukanusha imani} katika misingi ya injili na \textbf{katika Imani za Msingi za Kanisa} au \textbf{kufundisha mafundisho kinyume na hayo}.}[SDA Church Manual, 20th edition, Revised 2022, p. 67][https://www.adventist.org/wp-content/uploads/2023/07/2022-Seventh-day-Adventist-Church-Manual.pdf]


To discipline someone over their faith is nothing else than coercion of conscience. We are to render our conscience to the Bible alone—not to any man, councils or church creed(s). Disciplining members for their denial of the Fundamental Beliefs is clear evidence that we, indeed, have a creed besides the Bible. We cannot exercise the freedom of our conscience in subjection to the Word of God while confined to a set of beliefs that, if questioned with the authority of the Bible, will be disciplined. In our practice we have forgotten the foundation of protestantism and reformation. All reformers have had their conscience coerced to the extent of their lives. Martin Luther had famously put this principle in action in his defense before the Diet of Worms.


Kumtia mtu adabu juu ya imani yake si kitu kingine isipokuwa upokonyaji wa dhamiri. Tunapaswa kutoa dhamiri zetu kwa Biblia pekee—si kwa mtu yeyote, mabaraza au kanuni za imani za kanisa. Kuwaadhibu washiriki kwa kukataa Imani za Msingi ni ushahidi wa wazi kwamba sisi, hakika, tuna imani badala ya Biblia. Hatuwezi kutumia uhuru wa dhamiri yetu kwa kutii Neno la Mungu huku tukifungiwa kwenye seti ya imani ambazo, zikihojiwa na mamlaka ya Biblia, zitaadhibiwa. Katika mazoezi yetu tumesahau msingi wa uprotestanti na matengenezo. Wanamatengenezo wote walipokonywa dhamiri zao kiwango cha maisha yao. Martin Luther alikuwa ameiweka kanuni hii kwa vitendo katika utetezi wake kabla ya Diet of Worms.


\others{Unless I am \textbf{convicted by Scripture} and plain reason—I do not accept the authority of popes and councils, for they have contradicted each other—\textbf{\underline{my conscience is captive to the Word of God}}. I cannot and I will not recant anything, for \textbf{to go against conscience is neither right nor safe}. Here I stand, I cannot do otherwise. God help me. Amen.}[Bainton, 182]


\others{Isipokuwa \textbf{nimehukumiwa kwa Maandiko} na sababu zilizo wazi—sikubali mamlaka ya mapapa na mabaraza, kwa maana yamepingana—\textbf{\underline{dhamiri yangu imetekwa na Neno la Mungu}}. Siwezi na sitakataa chochote, kwa kuwa \textbf{kwenda kinyume na dhamiri si sahihi wala salama}. Hapa nimesimama, siwezi kufanya vinginevyo. Mungu nisaidie. Amina.}[Bainton, 182]


If one Seventh-day Adventist member has his conscience captive to the Word of God and is not in harmony with the Seventh-day Adventist Fundamental Beliefs, his conscience should not be coerced by church discipline. We know that in the end of time, the whole Seventh-day Adventist Church will be coerced over the issue of the Sabbath. We have been fighting for religious freedom, yet we’re allowing ourselves to coerce the conscience of those who are not in harmony with the Fundamental Beliefs. If today we discipline our members for not subjecting their consciences to men, councils and creeds, how shall we act tomorrow when the government will discipline their citizens for not subjecting their conscience to its power, when they will force obedience to legislation contrary to the Scriptures?


Ikiwa mshiriki mmoja wa Kiadventista wa Sabato ana dhamiri yake imefungwa kwa Neno la Mungu na hapatanani na Imani za Msingi za Waadventista Wasabato, dhamiri yake haipaswi kuchukuliwa nidhamu kwa kanisa. Tunajua kwamba katika mwisho wa wakati, Kanisa la Waadventista litanyang'anywa kwa ajili ya suala la Sabato. Tumekuwa tukipigania uhuru wa kidini, lakini tunajiruhusu kunyang'anya dhamiri za wale ambao hawakubaliani na Imani za Msingi. Ikiwa leo tunawaadhibu washiriki wetu kwa kutofanya hivyo wakiweka dhamiri zao kwa wanadamu, mabaraza na kanuni za imani, tutafanyaje wakati serikali itawatia adabu raia wao kwa kutotii dhamiri zao chini ya mamlaka yake, watakapolazimisha utii wa sheria kinyume na Maandiko?


Adventist pioneers were very much aware of the dangers of extorting church members’ consciences. The expression of their beliefs was not designed to form unity. They were ready to justify their faith, from the Bible, when asked. The Bible was their only creed and article of faith.


Waanzilishi wa Kiadventista walifahamu sana hatari za kuwanyang'anya washiriki wa kanisa dhamiri. Usemi wa imani yao haukuundwa kuunda umoja. Walikuwa tayari kuhalalisha imani yao, kutoka katika Biblia, wanapoulizwa. Biblia ndiyo ilikuwa imani yao pekee na makala ya imani.


In 1883, there was a suggestion to introduce the church manual into the Seventh-day Adventist Church. This proposal was rejected after close investigation of the committee appointed by the General Conference. In the article “\textit{No Church Manual}”, we read their reasons for not accepting the proposed church manual.


Mnamo 1883, kulikuwa na pendekezo la kuanzisha mwongozo wa kanisa katika Kanisa la Waadventista Wasabato. Pendekezo hili lilikataliwa baada ya uchunguzi wa kina wa kamati walioteuliwa na Mkutano Mkuu. Katika makala “\textit{Hakuna Mwongozo wa Kanisa}”, tunasoma sababu zao za kutokubali mwongozo wa kanisa uliopendekezwa.


\others{\textbf{While brethren who have favored a manual have ever contended that such a work was not to be anything like a creed or a discipline, or to have authority to settle disputed points}, but was only to be considered as a book containing hints for the help of those of little experience, \textbf{yet it must be evident that such a work, issued under the auspices of the General Conference, would at once carry with it much weight of authority, and would be consulted by most of our younger ministers}. \textbf{\underline{It would gradually shape and mold the whole body}}; \textbf{and those who did not follow it would be considered out of harmony with established principles of church order}. \textbf{And, really, is this not the object of the manual?} And what would be the use of one if not to accomplish such a result? But would this result, on the whole, be a benefit? Would our ministers be broader, more original, more self-reliant men? Could they be better depended on in great emergencies? Would their spiritual experiences likely be deeper and their judgment more reliable? \textbf{We think the tendency all the other way}.}[No Church Manual, The Review and Herald, November 27, 1883, pg. 745][https://documents.adventistarchives.org/Periodicals/RH/RH18831127-V60-47.pdf]


\others{\textbf{Wakati ndugu ambao wamependelea mwongozo wamewahi kubishana kuwa kazi kama hiyo haikupaswa kuwa kitu kama imani au nidhamu, au kuwa na mamlaka ya kusuluhisha pointi zilizobishaniwa}, lakini ilizingatiwa tu kama kitabu chenye vidokezo kwa usaidizi wa wale wenye uzoefu mdogo, \textbf{lakini lazima iwe dhahiri kwamba kazi kama hiyo, iliyotolewa chini ya mwavuli wa Kongamano Kuu, mara moja ungebeba uzito mkubwa wa mamlaka, na ingeshauriwa na wahudumu wetu wengi wachanga}. \textbf{\underline{Ingekuwa hatua kwa hatua kuunda na kuufinyanga mwili mzima}}; \textbf{na wale ambao hawakuifuata wangekuwa wakizingatiwa kuwa wameenda kinyume na kanuni zilizowekwa za utaratibu wa kanisa}. \textbf{Na, kwa kweli, hio sio lengo la mwongozo?} Na nini itakuwa matumizi ya moja kama si kukamilisha matokeo kama hayo? Lakini je, matokeo haya, kwa ujumla, yangekuwa ya faida? Je, mawaziri wetu wangekuwa wenye upana, asili zaidi, watu wanaojitegemea zaidi? Je, wanaweza kutegemewa zaidi katika dharura kubwa? Je, uzoefu wao wa kiroho ungekuwa wa kina zaidi na uamuzi wao zaidi kuaminika? \textbf{Sisi tunaifikiria uvutaji huo kwa njia mkabala}.}[No Church Manual, The Review and Herald, November 27, 1883, pg. 745][https://documents.adventistarchives.org/Periodicals/RH/RH18831127-V60-47.pdf]


\others{\textbf{The Bible contains our creed and discipline. It \underline{thoroughly} furnishes the man of God unto all good works}. What it has not revealed relative to church organization and management, the duties of officers and ministers, and kindred subjects, should not be strictly defined and drawn out into minute specifications for the sake of uniformity, \textbf{but rather be left to individual judgment under the guidance of the Holy Spirit}. \textbf{Had it been best to have a book of directions of this sort, the Spirit would doubtless have gone further, and left one on record with the stamp of inspiration upon it}.}[Ibid.][https://documents.adventistarchives.org/Periodicals/RH/RH18831127-V60-47.pdf]


\others{\textbf{Biblia ina kanuni zetu za imani na nidhamu. Inamvisha \underline{kikamilifu} mtu wa Mungu kwa matendo yote mema}. Kile ambacho haijafunua kuhusiana na shirika la kanisa na usimamizi, majukumu ya maafisa na mawaziri, na masomo ya jamaa, haipaswi kuwa imefafanuliwa madhubuti na kutolewa katika maelezo madogo kwa ajili ya usawa, \textbf{lakini badala yake iachwe kwa hukumu ya mtu binafsi chini ya uongozi wa Roho Mtakatifu}. \textbf{Ingekuwa bora kuwa na kitabu cha maelekezo ya namna hii, Roho bila shaka angeenda mbali zaidi, na kuacha moja kwenye kumbukumbu lenye chapa ya wahyi juu yake}.}[Ibid.][https://documents.adventistarchives.org/Periodicals/RH/RH18831127-V60-47.pdf]


Since 1883, the Seventh-day Adventist Church had grown considerably; so, for the sake of convenience, in 1931, the General Conference Committee voted to publish a church manual.\footnote{Maratas, Prince. “Church Manual.” General Conference of Seventh-Day Adventists, 20 Aug. 2023, \href{https://gc.adventist.org/church-manual/}{gc.adventist.org/church-manual/}. Accessed 3 Feb. 2025.} The church, as an organized body, should exercise order and discipline, in the matters of organization and plans of the prosperity of the Church's mission. But no committee should exercise authority over someone’s conscience and someone’s belief. Only God holds the right to this authority. This is why the Bible is our only creed. We render our conscience to the Word of God, not a man, nor a group of men or committee. Contrary to this, many believe that God vested this authority to the general assembly of the General Conference. But such an idea is based on misrepresentation of one particular quotation. Let us read this quotation carefully.


Tangu 1883, Kanisa la Waadventista Wasabato lilikuwa limekua kwa kiasi kikubwa; hivyo, kwa ajili ya urahisi, katika 1931, Kamati ya Konferensi Kuu ilipiga kura kuchapisha mwongozo wa kanisa.\footnote{Maratas, Prince. “Church Manual.” General Conference of Seventh-Day Adventists, 20 Aug. 2023, \href{https://gc.adventist.org/church-manual/}{gc.adventist.org/church-manual/}. Accessed 3 Feb. 2025.} Kanisa, kama kundi la madhehebu, linapaswa kutekeleza utaratibu na nidhamu, katika mambo ya mpangilio na mipango ya ustawi wa utume wa Kanisa. Lakini hakuna kamati inapaswa kutumia mamlaka juu ya dhamiri ya mtu na imani ya mtu. Mungu pekee ndiye anayeshikilia haki ya mamlaka hii. Hii ndiyo sababu Biblia ndiyo imani yetu pekee. Tunatoa dhamiri zetu kwa Neno la Mungu, si mwanadamu, wala kundi la watu au kamati. Kinyume na hili, wengi wanaamini kwamba Mungu alitoa mamlaka haya kwa mkutano mkuu wa Konferensi Kuu. Lakini wazo kama hilo linategemea upotoshaji wa nukuu moja fulani. Hebu tusome hii nukuu kwa uangalifu.


\egw{At times, when a small group of men entrusted with \textbf{the general management of the work} have, in the name of the General Conference, sought to carry out unwise plans and to restrict God’s work, I have said that I could no longer regard the voice of the General Conference, represented by these few men, as the voice of God. \textbf{But this is not saying that the decisions of a General Conference composed of an assembly of duly appointed, representative men from all parts of the field should not be respected}. \textbf{God has ordained that the representatives of His church from all parts of the earth, when assembled in a General Conference, \underline{shall have authority}}. The error that some are in danger of committing is in giving to the mind and judgment of one man, or of a small group of men, \textbf{the full measure of authority and influence that God has vested in His church in the judgment and voice of the General Conference assembled \underline{to plan for the prosperity and advancement of His work}}.}[9T 260.2; 1909][https://egwwritings.org/read?panels=p115.1474]


\egw{Wakati fulani, wakati kikundi kidogo cha watu kilikabidhiwa \textbf{usimamizi mkuu wa kazi} kwa jina la Mkutano Mkuu, wametafuta kutekeleza mipango isiyo ya busara na kufanya kuzuia kazi ya Mungu, nimesema kwamba singeweza tena kuzingatia sauti ya Mkutano Mkuu, unaowakilishwa na watu hawa wachache, kama sauti ya Mungu. \textbf{Lakini hii haisemi kwamba maamuzi ya Mkutano Mkuu unaojumuisha mkutano wa kisheria kuteuliwa, watu wawakilishi kutoka sehemu zote za uwanja hawapaswi kuheshimiwa}. \textbf{Mungu ameagiza kwamba wawakilishi wa kanisa lake kutoka sehemu zote za dunia, wanapokusanyika katika Kongamano Kuu, \underline{watakuwa na mamlaka}}. Hitilafu ambayo wengine wanayo ni hatari ya kufanya ni katika kutoa kwa akili na hukumu ya mtu mmoja, au ya kikundi kidogo ya wanadamu, \textbf{kipimo kamili cha mamlaka na ushawishi ambao Mungu ameweka katika kanisa lake katika hukumu na sauti ya Kongamano Kuu lililokusanyika \underline{kupanga kwa ajili ya ustawi na maendeleo ya kazi Yake}}.}[9T 260.2; 1909][https://egwwritings.org/read?panels=p115.1474]


Sister White pointed out that the world wide assembly of the General Conference meeting does have authority as the voice of God, yet she is very particular over what matters it has this authority. The authority God vested in the assembly of the General Conference is \egwinline{to plan for the prosperity and advancement of His work}. It is about making mission plans, not about managing beliefs or the conscience. God’s church does have His voice regarding beliefs; the voice of God pertaining to the faith is the Bible. The Bible is fully sufficient for us and we are free to render our conscience to it. No synopsis of any denominational faith has authority to dictate someone's faith; neither do \emcap{Fundamental Principles}, or current Fundamental Beliefs.\footnote{Although the Fundamental Principles were not designed to have authority over the people, nor were they designed to secure uniformity among them, as a system of faith, there is some evidence to the contrary. In his article, “\textit{Seventh-day Adventists and the Doctrine of the Trinity}”, of the “\textit{Christian Workers Magazine}”, 1915, D.M. Caright gave evidence that a Conference president used the \emcap{Fundamental Principles} as a test of fellowship in 1911. Such practice is not constructive to the Truth, neither is it beneficial for believers.} Sister White was very clear about the Bible being the only rule of faith, and every doctrine should be questioned with Scripture. In the Great Controversy, we read the following:


Dada White alidokeza kwamba mkutano wa dunia nzima wa Mkutano Mkuu ina mamlaka kama sauti ya Mungu, lakini yeye ni mahususi sana juu ya yale mambo yaliyo nayo katika mamlaka hii. Mamlaka ambayo Mungu ameweka katika mkutano wa Konferensi Kuu ni \egwinline{kupanga kwa ajili ya ustawi na maendeleo ya kazi Yake}. Ni juu ya kufanya mipango ya misheni, sio kuhusu kusimamia imani au dhamiri. Kanisa la Mungu lina sauti yake kuhusu imani; sauti ya Mungu inayohusu imani ni Biblia. Biblia inatutosha kabisa na tuko huru kuitolea dhamiri yetu. Hakuna muhtasari wa imani ya madhehebu yoyote iliyo na mamlaka ya kuamuru imani ya mtu; wala \emcap{Kanuni za Msingi}, au Imani za Msingi za sasa.\footnote{Ingawa Kanuni za Msingi hazikuundwa ili kuwa na mamlaka juu ya watu, wala hazikuundwa ili kuhakikisha umoja kati yao, kama mfumo wa imani, kuna ushahidi fulani kinyume na hilo. Katika makala yake, “\textit{Seventh-day Adventists and the Doctrine of the Trinity}”, ya “\textit{Christian Workers Magazine}”, 1915, D.M. Caright alitoa ushahidi kwamba Rais wa Konferensi alitumia \emcap{Kanuni za Msingi} kama mtihani wa ushirika mnamo 1911. Mazoea kama hayo si ya kujenga Ukweli, wala si ya manufaa kwa waumini.} Dada White alikuwa wazi sana kuhusu Biblia kuwa kanuni pekee ya imani, na kila fundisho linapaswa kutiliwa shaka kwa Maandiko. Katika Pambano Kuu, tunasoma yafuatayo:


\egw{But God will have a people upon the earth \textbf{to maintain the Bible, and \underline{the Bible only}}, \textbf{as the standard of all doctrines and the basis of all reforms}. \textbf{The opinions of learned men, the deductions of science, \underline{the creeds or decisions of ecclesiastical councils}, as numerous and discordant as are the churches which they represent, the voice of the majority - not one nor all of these should be regarded as evidence for or against any point of religious faith.} \textbf{Before accepting any doctrine or precept, we should demand a plain ‘Thus saith the Lord’ in its support.}}[GC 595.1; 1888][https://egwwritings.org/read?panels=p132.2689]


\egw{Lakini Mungu atakuwa na watu duniani \textbf{wa kutunza Biblia, na \underline{Biblia pekee}}, \textbf{kama wastani wa mafundisho yote na msingi wa marekebisho yote}. \textbf{Maoni ya watu waliojifunza, makato ya sayansi, \underline{kanuni za imani au maamuzi ya mabaraza ya kikanisa}, kama mengi na yasiokubaliana kama yalivyo makanisa yanayowakilisha, sauti ya Kanisa mengi - sio moja au yote haya yanapaswa kuzingatiwa kama ushahidi wa au dhidi ya uhakika wowote wa imani ya kidini.} \textbf{Kabla ya kukubali fundisho au agizo lolote, tunapaswa kudai wazi ‘Bwana asema hivi’ katika kuunga mkono.}}[GC 595.1; 1888][https://egwwritings.org/read?panels=p132.2689]


The liberty of conscience is the basics of protestantism and reformation. We hope and believe that every Seventh-day Adventist can exercise freedom to render his conscience to the Bible without being coerced by discipline, or any other means. The issue of the church's creed and discipline becomes more relevant today, when we have the promise that God will re-establish the original foundation of our faith. We hope and pray that the evidence brought up here will bring light to the church leadership and encourage them to eradicate the false practices in our midst. As the religious leaders in Christ’s time were entrusted with the duty to preserve the Truth and to recognize the time of God’s visitation, so it is today with the leaders of the Seventh-day Adventist Church. In what follows, we will present the prophecies God specifically gave to the Seventh-day Adventist Church. In our time, the end-time, all the pillars of our faith that were held in the beginning will be re-established. May every member of the Seventh-day Adventist Church recognize the importance of the revival that God is about to establish.


Uhuru wa dhamiri ndio msingi wa uprotestanti na matengenezo. Tunatumai na tunaamini kwamba kila Muadventista wa Sabato anaweza kutumia uhuru wa kutoa dhamiri yake kwa Biblia bila kunyang'anywa kwa nidhamu, au njia nyingine yoyote. Suala la imani ya kanisa na nidhamu inakuwa muhimu zaidi leo, tunapokuwa na ahadi ambayo Mungu atafanya kuanzisha upya msingi asilia wa imani yetu. Tunatumai na kuomba kwamba ushahidi ukiletwa hapa juu italeta nuru kwa uongozi wa kanisa na kuwatia moyo kutokomeza uwongo mazoea katikati yetu. Kama vile viongozi wa kidini katika wakati wa Kristo walikabidhiwa jukumu hilo kuhifadhi Ukweli na kutambua wakati wa kujiliwa na Mungu, ndivyo ilivyo leo na viongozi wa Kanisa la Waadventista Wasabato. Katika kile kinachofuata, tutawasilisha unabii Mungu alitoa hasa kwa Kanisa la Waadventista Wasabato. Katika wakati wetu, wakati wa mwisho, nguzo zote za imani yetu ambazo zilishikiliwa hapo mwanzo zitaimarishwa tena. Na kila mshiriki wa Kanisa la Waadventista Wasabato anatambua umuhimu wa uamsho ambao Mungu anaenda kuanzisha.


% Steps to Apostasy

\begin{titledpoem}
    
    \stanza{
        A creed established past God’s Word, \\
        The voice of conscience was not heard. \\
        Test fellowship by men’s decree, \\
        From Bible rules we now are free.
    }

    \stanza{
        Those who dissent are labeled lost, \\
        Their faith they held at terrible cost. \\
        As “heretics” they are cast out, \\
        Bringing great sorrow, there’s no doubt.
    }

    \stanza{
        God’s Word alone should be our guide, \\
        Walking with Jesus, by our side, \\
        From strong convictions, do not turn. \\
        Faithful to truth, this lesson learn.
    }

    \stanza{
        The pioneers knew this freedom well, \\
        Against men’s creeds they did rebel. \\
        Truth only dwells with conscience free, \\
        As God intends His church to be.
    }
    
\end{titledpoem}


% Steps to Apostasy

\begin{titledpoem}
    
    \stanza{
        A creed established past God’s Word, \\
        The voice of conscience was not heard. \\
        Test fellowship by men’s decree, \\
        From Bible rules we now are free.
    }

    \stanza{
        Those who dissent are labeled lost, \\
        Their faith they held at terrible cost. \\
        As “heretics” they are cast out, \\
        Bringing great sorrow, there’s no doubt.
    }

    \stanza{
        God’s Word alone should be our guide, \\
        Walking with Jesus, by our side, \\
        From strong convictions, do not turn. \\
        Faithful to truth, this lesson learn.
    }

    \stanza{
        The pioneers knew this freedom well, \\
        Against men’s creeds they did rebel. \\
        Truth only dwells with conscience free, \\
        As God intends His church to be.
    }
    
\end{titledpoem}
