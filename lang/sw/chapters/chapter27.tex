\qrchapter{https://forgottenpillar.com/rsc/sw-fp-chapter27}{Hatua za uasi}

Katika nukuu ifuatayo, ndugu J. N. Loughborough, ambaye alikuwa mmoja wa waanzilishi wa Kanisa la Waadventista Wasabato, alituonya kuhusu hatua tano za uasi.

\others{\textbf{Hatua} \textbf{ya kwanza} ya uasi ni \textbf{kusimamisha kanuni ya imani}, ukituambia kile tutakachokiamini. \textbf{Ya pili} ni \textbf{kufanya imani hiyo kuwa mtihani wa ushirika}. \textbf{Ya tatu} ni \textbf{kujaribu wanachama kwa hiyo imani}. \textbf{Ya nne} ni \textbf{kuwashutumu kama wazushi wale wasioamini imani hiyo}. Na \textbf{ya tano}, \textbf{kuanza mateso dhidi ya watu kama hao}. Ninawasihi kwamba hatufanyi muundo baada ya makanisa kwa maana yoyote isiyokubalika katika hatua iliyopendekezwa.}[John N. Loughborough, Review and Herald, Oct. 8, 1861.][https://egwwritings.org/read?panels=p1685.5326]

Ni muhimu kuzingatia Kanuni hizi, na tunapaswa kujiuliza kama sisi, leo, tunaiga makanisa kwa maana yoyote isiyokubalika katika hatua iliyopendekezwa. Nini ingetokea kwa Mwaadventista wa Sabato ambaye angekataa fundisho la Utatu kwa kupendelea \emcap{Kanuni za Msingi}? Je! tunayo kanuni ya imani iliyoanzishwa katika kanisa letu? Je, tunaangazia uanachama wetu nayo?

\emcap{Kanuni za Msingi} zilikuwa na asili na jukumu tofauti katika Kanisa la Waadventista Wasabato kinyume na utaratibu unaoshikiliwa na makanisa mengine. \emcap{Kanuni za Msingi} hazikuundwa kama imani. Katika utangulizi wa taarifa ya 1872, tunasoma juu ya asili yao:

\others{Katika kuwasilisha kwa \textbf{umma} \textbf{muhtasari huu wa imani yetu}, tungependa ieleweke kwa uwazi kwamba \textbf{\underline{hatuna vifungu vya imani, kanuni za imani}, au nidhamu, \underline{kando na Biblia}}. \textbf{Hatusemi} haya \textbf{\underline{kuwa na mamlaka yoyote kwa watu wetu}}, \textbf{wala haikusudiwa kufanya hivyo kupata umoja kati yao}, \textbf{kama mfumo wa imani}, \textbf{lakini ni maelezo mafupi ya ni nini, na imekuwa, kwa umoja mkubwa, unaoshikiliwa nao}.}[A Declaration of the Fundamental Principles, Taught and Practiced by the Seventh-Day Adventists, 1872]

Katika utangulizi wa taarifa ya 1889, tunasoma maoni yanayofanana:

\others{Kama mahali penginepo inavyosema, Waadventista Wasabato \textbf{hawana imani ila Biblia}; lakini wanashikilia kwa \textbf{mambo fulani ya imani yaliyofafanuliwa vizuri}, ambayo \textbf{wanahisi kuwa tayari kutoa sababu ‘kwa kila mtu anayewauliza’}. Mapendekezo yafuatayo yanaweza kuchukuliwa kama muhtasari wa \textbf{sifa kuu za imani yao ya kidini}, ambayo iko juu yake, kama tunavyojua, \textbf{umoja mzima katika mwili wote}.}[Seventh-day Adventist Year Book of statistics for 1889, pg. 147, The Fundamental Principles of Seventh-day Adventists]

\emcap{Kanuni za Msingi} hazikuundwa ili kulazimisha imani ya mtu fulani. Waumini waliongozwa na Roho Mtakatifu, kwa hiari walitoa dhamiri zao kwa Neno la Mungu; chini ya ushawishi wa Roho Mtakatifu, walifikia maamuzi sawa. Kulikuwa na umoja mzima wa mwili mzima. Waumini wote waliona “\textit{wamejitayarisha kutoa sababu kwa kila mtu anayeuliza wao}” kuhusu imani yao.

Leo tunaona tofauti kubwa katika kanuni na utendaji wa imani za Waadventista ikilinganishwa na waanzilishi wetu. Tunatunza roho ya umoja kwa kuwaadhibu washiriki wetu kukanusha Imani za Msingi. Katika mwongozo wetu wa kanisa, chini ya sehemu “\textit{Sababu ya Nidhamu}”, tunasoma nukta ya kwanza inayosema nidhamu ya kukataa imani katika Imani za Msingi za Waadventista Wasabato.

\others{Sababu za Nidhamu}

\others{1. \textbf{Kukanusha imani} katika misingi ya injili na \textbf{katika Imani za Msingi za Kanisa} au \textbf{kufundisha mafundisho kinyume na hayo}.}[SDA Church Manual, 20th edition, Revised 2022, p. 67][https://www.adventist.org/wp-content/uploads/2023/07/2022-Seventh-day-Adventist-Church-Manual.pdf]

Kumtia mtu adabu juu ya imani yake si kitu kingine isipokuwa upokonyaji wa dhamiri. Tunapaswa kutoa dhamiri zetu kwa Biblia pekee—si kwa mtu yeyote, mabaraza au kanuni za imani za kanisa. Kuwaadhibu washiriki kwa kukataa Imani za Msingi ni ushahidi wa wazi kwamba sisi, hakika, tuna imani badala ya Biblia. Hatuwezi kutumia uhuru wa dhamiri yetu kwa kutii Neno la Mungu huku tukifungiwa kwenye seti ya imani ambazo, zikihojiwa na mamlaka ya Biblia, zitaadhibiwa. Katika mazoezi yetu tumesahau msingi wa uprotestanti na matengenezo. Wanamatengenezo wote walipokonywa dhamiri zao kiwango cha maisha yao. Martin Luther alikuwa ameiweka kanuni hii kwa vitendo katika utetezi wake kabla ya Diet of Worms.

\others{Isipokuwa \textbf{nimehukumiwa kwa Maandiko} na sababu zilizo wazi—sikubali mamlaka ya mapapa na mabaraza, kwa maana yamepingana—\textbf{\underline{dhamiri yangu imetekwa na Neno la Mungu}}. Siwezi na sitakataa chochote, kwa kuwa \textbf{kwenda kinyume na dhamiri si sahihi wala salama}. Hapa nimesimama, siwezi kufanya vinginevyo. Mungu nisaidie. Amina.}[Bainton, 182]

Ikiwa mshiriki mmoja wa Kiadventista wa Sabato ana dhamiri yake imefungwa kwa Neno la Mungu na hapatanani na Imani za Msingi za Waadventista Wasabato, dhamiri yake haipaswi kuchukuliwa nidhamu kwa kanisa. Tunajua kwamba katika mwisho wa wakati, Kanisa la Waadventista litanyang'anywa kwa ajili ya suala la Sabato. Tumekuwa tukipigania uhuru wa kidini, lakini tunajiruhusu kunyang'anya dhamiri za wale ambao hawakubaliani na Imani za Msingi. Ikiwa leo tunawaadhibu washiriki wetu kwa kutofanya hivyo wakiweka dhamiri zao kwa wanadamu, mabaraza na kanuni za imani, tutafanyaje wakati serikali itawatia adabu raia wao kwa kutotii dhamiri zao chini ya mamlaka yake, watakapolazimisha utii wa sheria kinyume na Maandiko?

Waanzilishi wa Kiadventista walifahamu sana hatari za kuwanyang'anya washiriki wa kanisa dhamiri. Usemi wa imani yao haukuundwa kuunda umoja. Walikuwa tayari kuhalalisha imani yao, kutoka katika Biblia, wanapoulizwa. Biblia ndiyo ilikuwa imani yao pekee na makala ya imani.

Mnamo 1883, kulikuwa na pendekezo la kuanzisha mwongozo wa kanisa katika Kanisa la Waadventista Wasabato. Pendekezo hili lilikataliwa baada ya uchunguzi wa kina wa kamati walioteuliwa na Mkutano Mkuu. Katika makala “\textit{Hakuna Mwongozo wa Kanisa}”, tunasoma sababu zao za kutokubali mwongozo wa kanisa uliopendekezwa.

\others{\textbf{Wakati ndugu ambao wamependelea mwongozo wamewahi kubishana kuwa kazi kama hiyo haikupaswa kuwa kitu kama imani au nidhamu, au kuwa na mamlaka ya kusuluhisha pointi zilizobishaniwa}, lakini ilizingatiwa tu kama kitabu chenye vidokezo kwa usaidizi wa wale wenye uzoefu mdogo, \textbf{lakini lazima iwe dhahiri kwamba kazi kama hiyo, iliyotolewa chini ya mwavuli wa Kongamano Kuu, mara moja ungebeba uzito mkubwa wa mamlaka, na ingeshauriwa na wahudumu wetu wengi wachanga}. \textbf{\underline{Ingekuwa hatua kwa hatua kuunda na kuufinyanga mwili mzima}}; \textbf{na wale ambao hawakuifuata wangekuwa wakizingatiwa kuwa wameenda kinyume na kanuni zilizowekwa za utaratibu wa kanisa}. \textbf{Na, kwa kweli, hio sio lengo la mwongozo?} Na nini itakuwa matumizi ya moja kama si kukamilisha matokeo kama hayo? Lakini je, matokeo haya, kwa ujumla, yangekuwa ya faida? Je, mawaziri wetu wangekuwa wenye upana, asili zaidi, watu wanaojitegemea zaidi? Je, wanaweza kutegemewa zaidi katika dharura kubwa? Je, uzoefu wao wa kiroho ungekuwa wa kina zaidi na uamuzi wao zaidi kuaminika? \textbf{Sisi tunaifikiria uvutaji huo kwa njia mkabala}.}[No Church Manual, The Review and Herald, November 27, 1883, pg. 745][https://documents.adventistarchives.org/Periodicals/RH/RH18831127-V60-47.pdf]

\others{\textbf{Biblia ina kanuni zetu za imani na nidhamu. Inamvisha \underline{kikamilifu} mtu wa Mungu kwa matendo yote mema}. Kile ambacho haijafunua kuhusiana na shirika la kanisa na usimamizi, majukumu ya maafisa na mawaziri, na masomo ya jamaa, haipaswi kuwa imefafanuliwa madhubuti na kutolewa katika maelezo madogo kwa ajili ya usawa, \textbf{lakini badala yake iachwe kwa hukumu ya mtu binafsi chini ya uongozi wa Roho Mtakatifu}. \textbf{Ingekuwa bora kuwa na kitabu cha maelekezo ya namna hii, Roho bila shaka angeenda mbali zaidi, na kuacha moja kwenye kumbukumbu lenye chapa ya wahyi juu yake}.}[Ibid.][https://documents.adventistarchives.org/Periodicals/RH/RH18831127-V60-47.pdf]

Tangu 1883, Kanisa la Waadventista Wasabato lilikuwa limekua kwa kiasi kikubwa; hivyo, kwa ajili ya urahisi, katika 1931, Kamati ya Konferensi Kuu ilipiga kura kuchapisha mwongozo wa kanisa.\footnote{Maratas, Prince. “Church Manual.” General Conference of Seventh-Day Adventists, 20 Aug. 2023, \href{https://gc.adventist.org/church-manual/}{gc.adventist.org/church-manual/}. Accessed 3 Feb. 2025.} Kanisa, kama kundi la madhehebu, linapaswa kutekeleza utaratibu na nidhamu, katika mambo ya mpangilio na mipango ya ustawi wa utume wa Kanisa. Lakini hakuna kamati inapaswa kutumia mamlaka juu ya dhamiri ya mtu na imani ya mtu. Mungu pekee ndiye anayeshikilia haki ya mamlaka hii. Hii ndiyo sababu Biblia ndiyo imani yetu pekee. Tunatoa dhamiri zetu kwa Neno la Mungu, si mwanadamu, wala kundi la watu au kamati. Kinyume na hili, wengi wanaamini kwamba Mungu alitoa mamlaka haya kwa mkutano mkuu wa Konferensi Kuu. Lakini wazo kama hilo linategemea upotoshaji wa nukuu moja fulani. Hebu tusome hii nukuu kwa uangalifu.

\egw{Wakati fulani, wakati kikundi kidogo cha watu kilikabidhiwa \textbf{usimamizi mkuu wa kazi} kwa jina la Mkutano Mkuu, wametafuta kutekeleza mipango isiyo ya busara na kufanya kuzuia kazi ya Mungu, nimesema kwamba singeweza tena kuzingatia sauti ya Mkutano Mkuu, unaowakilishwa na watu hawa wachache, kama sauti ya Mungu. \textbf{Lakini hii haisemi kwamba maamuzi ya Mkutano Mkuu unaojumuisha mkutano wa kisheria kuteuliwa, watu wawakilishi kutoka sehemu zote za uwanja hawapaswi kuheshimiwa}. \textbf{Mungu ameagiza kwamba wawakilishi wa kanisa lake kutoka sehemu zote za dunia, wanapokusanyika katika Kongamano Kuu, \underline{watakuwa na mamlaka}}. Hitilafu ambayo wengine wanayo ni hatari ya kufanya ni katika kutoa kwa akili na hukumu ya mtu mmoja, au ya kikundi kidogo ya wanadamu, \textbf{kipimo kamili cha mamlaka na ushawishi ambao Mungu ameweka katika kanisa lake katika hukumu na sauti ya Kongamano Kuu lililokusanyika \underline{kupanga kwa ajili ya ustawi na maendeleo ya kazi Yake}}.}[9T 260.2; 1909][https://egwwritings.org/read?panels=p115.1474]

Dada White alidokeza kwamba mkutano wa dunia nzima wa Mkutano Mkuu ina mamlaka kama sauti ya Mungu, lakini yeye ni mahususi sana juu ya yale mambo yaliyo nayo katika mamlaka hii. Mamlaka ambayo Mungu ameweka katika mkutano wa Konferensi Kuu ni \egwinline{kupanga kwa ajili ya ustawi na maendeleo ya kazi Yake}. Ni juu ya kufanya mipango ya misheni, sio kuhusu kusimamia imani au dhamiri. Kanisa la Mungu lina sauti yake kuhusu imani; sauti ya Mungu inayohusu imani ni Biblia. Biblia inatutosha kabisa na tuko huru kuitolea dhamiri yetu. Hakuna muhtasari wa imani ya madhehebu yoyote iliyo na mamlaka ya kuamuru imani ya mtu; wala \emcap{Kanuni za Msingi}, au Imani za Msingi za sasa.\footnote{Ingawa Kanuni za Msingi hazikuundwa ili kuwa na mamlaka juu ya watu, wala hazikuundwa ili kuhakikisha umoja kati yao, kama mfumo wa imani, kuna ushahidi fulani kinyume na hilo. Katika makala yake, “\textit{Seventh-day Adventists and the Doctrine of the Trinity}”, ya “\textit{Christian Workers Magazine}”, 1915, D.M. Caright alitoa ushahidi kwamba Rais wa Konferensi alitumia \emcap{Kanuni za Msingi} kama mtihani wa ushirika mnamo 1911. Mazoea kama hayo si ya kujenga Ukweli, wala si ya manufaa kwa waumini.} Dada White alikuwa wazi sana kuhusu Biblia kuwa kanuni pekee ya imani, na kila fundisho linapaswa kutiliwa shaka kwa Maandiko. Katika Pambano Kuu, tunasoma yafuatayo:

\egw{Lakini Mungu atakuwa na watu duniani \textbf{wa kutunza Biblia, na \underline{Biblia pekee}}, \textbf{kama wastani wa mafundisho yote na msingi wa marekebisho yote}. \textbf{Maoni ya watu waliojifunza, makato ya sayansi, \underline{kanuni za imani au maamuzi ya mabaraza ya kikanisa}, kama mengi na yasiokubaliana kama yalivyo makanisa yanayowakilisha, sauti ya Kanisa mengi - sio moja au yote haya yanapaswa kuzingatiwa kama ushahidi wa au dhidi ya uhakika wowote wa imani ya kidini.} \textbf{Kabla ya kukubali fundisho au agizo lolote, tunapaswa kudai wazi ‘Bwana asema hivi’ katika kuunga mkono.}}[GC 595.1; 1888][https://egwwritings.org/read?panels=p132.2689]

Uhuru wa dhamiri ndio msingi wa uprotestanti na matengenezo. Tunatumai na tunaamini kwamba kila Muadventista wa Sabato anaweza kutumia uhuru wa kutoa dhamiri yake kwa Biblia bila kunyang'anywa kwa nidhamu, au njia nyingine yoyote. Suala la imani ya kanisa na nidhamu inakuwa muhimu zaidi leo, tunapokuwa na ahadi ambayo Mungu atafanya kuanzisha upya msingi asilia wa imani yetu. Tunatumai na kuomba kwamba ushahidi ukiletwa hapa juu italeta nuru kwa uongozi wa kanisa na kuwatia moyo kutokomeza uwongo mazoea katikati yetu. Kama vile viongozi wa kidini katika wakati wa Kristo walikabidhiwa jukumu hilo kuhifadhi Ukweli na kutambua wakati wa kujiliwa na Mungu, ndivyo ilivyo leo na viongozi wa Kanisa la Waadventista Wasabato. Katika kile kinachofuata, tutawasilisha unabii Mungu alitoa hasa kwa Kanisa la Waadventista Wasabato. Katika wakati wetu, wakati wa mwisho, nguzo zote za imani yetu ambazo zilishikiliwa hapo mwanzo zitaimarishwa tena. Na kila mshiriki wa Kanisa la Waadventista Wasabato anatambua umuhimu wa uamsho ambao Mungu anaenda kuanzisha.

% Steps to Apostasy

\begin{titledpoem}
    
    \stanza{
        A creed established past God’s Word, \\
        The voice of conscience was not heard. \\
        Test fellowship by men’s decree, \\
        From Bible rules we now are free.
    }

    \stanza{
        Those who dissent are labeled lost, \\
        Their faith they held at terrible cost. \\
        As “heretics” they are cast out, \\
        Bringing great sorrow, there’s no doubt.
    }

    \stanza{
        God’s Word alone should be our guide, \\
        Walking with Jesus, by our side, \\
        From strong convictions, do not turn. \\
        Faithful to truth, this lesson learn.
    }

    \stanza{
        The pioneers knew this freedom well, \\
        Against men’s creeds they did rebel. \\
        Truth only dwells with conscience free, \\
        As God intends His church to be.
    }
    
\end{titledpoem}
