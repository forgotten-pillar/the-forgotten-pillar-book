\qrchapter{https://forgottenpillar.com/rsc/sw-fp-chapter17}{Jibu kwa hisia za utatu za Kellogg}

Ikiwa tunatazama MGOGORO wa Kellogg kupitia mtazamo wa \emcap{ubinafsi wa Mungu} na \emcap{Kanuni za Msingi}, manukuu ya Dada White bila shaka yanaangaza kwa Mwanga mpya. Katika mwanga huu tunaona mgongano kati ya ukweli tuliopokea hapo mwanzo, juu ya \emcap{ubinafsi wa Mungu}, na fundisho la Utatu. Ili kuepusha hitilafu, kwa maslahi ya kutetea Fundisho la Utatu, wasomi daima husisitiza zaidi upande wa pantheism wa tatizo hili.

Tungependa kutoa changamoto kwa tabia hii ya kusisitiza zaidi upande wa pantheism wa mgogoro wa Kellogg. Dada White kwa ujumla aliandika ukweli; alikaripia kosa kwa kuinua ukweli. Hii ndiyo sababu aliandika sana kuhusu \emcap{ubinafsi wa Mungu}. Katika zaidi ya nukuu zake juu ya somo hili, tunamwona akiondoa kosa la Utatu, badala ya kosa la pantheism. Tunasoma mfano mmoja kama huo ambapo anathibitisha ukweli juu ya \emcap{ubinafsi wa Mungu} akiirejelea sura ya kumi na saba ya Yohana.

\egw{\textbf{Ubinafsi wa Baba na Mwana, pia umoja uliopo kati Yao, yametolewa katika sura ya kumi na saba ya Yohana, katika maombi ya Kristo kwa ajili Ya wanafunzi wake:}}[MH 421.7; 1905][https://egwwritings.org/read?panels=p135.2173]

Kuna matukio mengi ambapo Dada White ananukuu Yohana 17 kuhusiana na mgogoro wa Kellogg. Wale wanaodai kwamba mzozo wa Kellogg ulihusu pantheism tu wanapaswa kuuliza jinsi John 17 inazungumzia Mungu katika asili. Na si Yohana 17 tu, bali pia sura za 13-16. Katika barua yake kwa Kellogg, aliandika:

\egw{\textbf{\underline{...soma sura ya kumi na tatu, ya kumi na nne, ya kumi na tano, ya kumi na sita, na ya kumi na saba ya Yohana}. Maneno ya sura hizi yanajieleza yenyewe. ‘Huu ndio uzima wa milele,’ Kristo alitangaza, ‘wapate kukujua \underline{wewe, Mungu wa pekee wa kweli}, na Yesu Kristo ambaye Wewe amemtuma.’ \underline{Katika maneno haya ubinafsi wa Mungu na wa Mwanawe unasemwa waziwazi.} \underline{Ubinafsi wa Mmoja hauondoi umuhimu wa ubinafsi wa mwingine}.}}[Lt232-1903.48, 1903][https://egwwritings.org/read?panels=p10197.57]

Katika sura zilizotajwa hapo juu za Yohana, Yohana hakurejelea chochote kinachomhusu Mungu katika Asili. Maudhui ya sura hizo yanahusu ni nani aliye Mungu wa pekee wa kweli, jinsi Baba na Mwana ni mmoja, uhusiano wao wa kweli, na jinsi Yesu anavyoweza kuwepo kila mahali na bado kupaa kwa Baba.

\egw{Yesu aliwaambia Wayahudi: ‘Baba yangu anafanya kazi hata sasa, nami ninafanya kazi.... Mwana hawezi kufanya lolote bali lile analomuona Baba analifanya; kwa kuwa yote ayatendayo, hayo pia naye Mwana vivyo hivyo. Kwa maana Baba anampenda Mwana, na humwonyesha mambo yote ambayo Mwenyewe anafanya.’ Yohana 5:17-20.}[8T 268.4, 1904][https://egwwritings.org/read?panels=p112.1557]

\egwnogap{\textbf{Hapa tena inaletwa machoni petu, \underline{ubinafsi wa Baba na Mwana}, ikionyesha umoja uliopo kati yao}.}[8T 269.1; 1904][https://egwwritings.org/read?panels=p112.1560]

\egwnogap{\textbf{Umoja huu unaonyeshwa pia katika \underline{sura ya kumi na saba ya Yohana}, katika maombi ya Kristo kwa wanafunzi Wake:}}[8T 269.2; 1904][https://egwwritings.org/read?panels=p112.1561]

\egwnogap{‘Wala siwaombei hawa peke yao, bali na wale watakaoniamini kwa njia ya neno lao; ili wote wawe kitu kimoja; \textbf{kama wewe, Baba, ulivyo ndani yangu, nami ndani yako, hao nao pia wawe wamoja ndani Yetu}: ili ulimwengu upate kusadiki ya kwamba ndiwe uliyenituma. Na \textbf{utukufu ambao Ulinipa Mimi} nimewapa wao; \textbf{ili wawe na umoja kama sisi tulivyo na umoja: mimi ndani yao, nawe ndani yangu, ili wawe wamekamilika katika umoja}; na ili ulimwengu upate kujua kwamba Umenituma mimi, nawe umewapenda wao kama vile ulivyonipenda mimi.’ Yohana 17:20-23.}[8T 269.3; 1904][https://egwwritings.org/read?panels=p112.1562]

\egwnogap{Kauli ya ajabu! \textbf{Umoja uliopo kati ya Kristo na wanafunzi wake \underline{hauharibu ubinafsi wa mwingine}. Wao ni wamoja katika kusudi, akilini, katika tabia, lakini \underline{si kwa nafsi}. Hivyo ndivyo Mungu na Kristo Wana umoja}.}[8T 269.4; 1904][https://egwwritings.org/read?panels=p112.1563]

\egwnogap{\textbf{Uhusiano kati ya Baba na Mwana, na ubinafsi wa wote wawili, unafanywa wazi katika andiko pia}:}[8T 269.5; 1904][https://egwwritings.org/read?panels=p112.1564]

\egwnogap{Asema hivi \textbf{BWANA wa majeshi},} \\
\egw{Tazama, \textbf{mtu} ambaye jina lake ni \textbf{Tawi}:} \\
\egw{Naye atakua kutoka mahali pake;} \\
\egw{\textbf{Naye atalijenga hekalu la BWANA;... }} \\
\egw{\textbf{Naye atabeba utukufu,}} \\
\egw{\textbf{Naye ataketi na kutawala juu ya kiti chake cha enzi;}} \\
\egw{\textbf{Naye atakuwa kuhani katika kiti chake cha enzi;}} \\
\egw{\textbf{Na \underline{shauri la amani litakuwa kati ya hao wawili}}.’}[8T 269.6; 1904][https://egwwritings.org/read?panels=p112.1565]

Sura zilizotajwa hapo juu za Injili ya Yohana zinahusu \emcap{ubinafsi wa Mungu}, ambao umeelezwa katika hoja mbili za kwanza za \emcap{Kanuni za Msingi}. Ni kosa lipi Dada White alipigana aliporejelea mistari kuhusu jinsi Baba ni Mungu wa pekee wa kweli, na jinsi Baba na Mwana si wamoja katika nafsi? Pantheism? Hakika sivyo; lakini wengi pengine hisia za utatu, au imani katika Mungu mmoja-katika-tatu, au watatu-katika-mmoja.

Ndugu J. N. Loughborough, mmoja wa ndugu wa kwanza walioandika juu ya \emcap{ubinafsi wa Mungu}, aliandika maelezo yafuatayo juu ya Yohana sura ya 17:

\others{\textbf{\underline{Sura ya kumi na saba ya Yohana pekee inatosha kukanusha fundisho la Utatu}}. \textbf{...Soma sura ya kumi na saba ya Yohana, na uone kama haifanyi hivyo kabisa kuvuruga fundisho la Utatu}.}[John N. Loughborough, The Adventist Review, and Sabbath Herald, November 5, 1861, p. 184.10][https://egwwritings.org/read?panels=p1685.6615]

Uandishi wa umakini wa Dada White katika kuunga mkono ukweli juu ya \emcap{ubinafsi wa Mungu} na uwepo wake ni sawa na waanzilishi wengine wa Kiadventista. Ikiwa waanzilishi wa Kiadventista walikuwa wanakanusha Fundisho la Utatu kwa kuinua ukweli juu ya \emcap{Ubinafsi wa Mungu} na uwepo wa Mungu, nini inatufanya tufikiri Ellen White hakuwa anafanya hivyo, wakati upande wa kitheolojia wa swali la Utatu lilizushwa? Kwa kusema hili, hatukatai upande wa pantheism wa Mzozo wa Kellogg, lakini kwa kuusisitiza kupita kiasi, unashindwa kuelezea kwa usahihi ukweli wa suala Hilo. Uelewa sahihi wa utata wa Kellogg unaweza tu kukamilika kwa kulenga hasa ukweli ulioinuliwa Dada White, badala ya kuzingatia makosa, kama pantheism au Utatu. Ukweli huu ambao Dada White aliinua ulikuwa ukweli juu ya \emcap{ubinafsi wa Mungu} na ulipo uwepo wake. Hii inaonyeshwa katika hoja ya kwanza ya \emcap{Kanuni za Msingi}, ambazo zilikuwa muhtasari rasmi na uwakilishi wa Imani ya Waadventista Wasabato katika wakati wa Ellen White; ukweli ambao sisi kama kanisa \egwinline{tumeupokea na kuusikia na kutetea}[Ms124-1905.12; 1905][https://egwwritings.org/read?panels=p9099.18] hapo mwanzo.

\egw{\textbf{Ninasihi kila mmoja awe wazi na thabiti kuhusu kweli fulani tulizopokea na kusikia na zilizotetewa. Kauli za Neno la Mungu ziko wazi. Panda miguu yako imara kwenye \underline{jukwaa la ukweli wa milele}. \underline{Kataa kila awamu ya makosa}, hata \underline{ingawa imefunikwa na mwonekano wa ukweli, ambao unakana ubinafsi wa Mungu au ya Kristo}}.}[Ms124-1905.12; 1905][https://egwwritings.org/read?panels=p9099.18]

Onyo kutoka kwa nukuu zilizopita hazikupungua baada ya muda. Leo ni muhimu zaidi. Tunapaswa \egwinline{kukataa kila awamu ya kosa, ingawa imefunikwa kwa mfano wa ukweli, unaokana ubinafsi wa Mungu au wa Kristo}. Katika sura inayofuata tunataka kuonyesha awamu maalum ya kosa ambalo limefunikwa na mwonekano wa ukweli, ambalo linakana ubinafsi wa Mungu na wa Kristo—nafsi tatu hai za \textit MUNGU {mmoja}, kinyume na \egwinline{nafsi tatu hai za utatu wa mbinguni.}[Ms21-1906.11; 1906][https://egwwritings.org/read?panels=p9754.18]

% Reply to Kellogg’s trinitarian sentiments

\begin{titledpoem}
    
    \stanza{
        The light of truth, so clear and bold, \\
        A crisis came, a story told. \\
        Not pantheism, dim and wide, \\
        But God’s persona, we confide.
    }

    \stanza{
        But God, through Ellen, did uphold \\
        God’s personality was told. \\
        Against the Trinity, she leaned, \\
        A unity, by John unseen.
    }

    \stanza{
        "The Father and the Son," she wrote, \\
        Are one in purpose was her quote. \\
        John seventeen, her chosen guide, \\
        Where God’s true nature cannot hide.
    }

    \stanza{
        The pioneers, with her agreed, \\
        Of God’s true person, they did plead. \\
        Loughborough echoed, his words clear, \\
        The Trinity dismissed, no fear.
    }

    \stanza{
        The Fundamental Points, so dear, \\
        They make it plain, we must revere. \\
        Not in the trinity’s wrong creed, \\
        But in His presence, faith is freed.
    }

    \stanza{
        So let us stand on truth so bright, \\
        Rejecting wrong, with all our might. \\
        God’s person, where we find our plea, \\
        Truth’s platform for eternity.
    }
    
\end{titledpoem}

Jibu kwa hisia za utatu za Kellogg

Ikiwa tunatazama shida ya Kellogg kupitia mtazamo wa ubinafsi wa Mungu na Kanuni za Msingi, manukuu ya Dada White bila shaka yanaangaza kwa Mwanga mpya. Katika mwanga huu tunaona mgongano kati ya ukweli tuliopokea hapo mwanzo, juu ya ubinafsi wa Mungu, na fundisho la Utatu. Ili kuepusha hitilafu, kwa maslahi ya kutetea Fundisho la Utatu, wasomi daima husisitiza zaidi upande wa pantheism wa tatizo.
