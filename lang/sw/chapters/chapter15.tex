\qrchapter{https://forgottenpillar.com/rsc/en-fp-chapter15}{Dr. Kellogg and the Trinity doctrine}


\qrchapter{https://forgottenpillar.com/rsc/en-fp-chapter15}{Dk. Kellogg na fundisho la Utatu}


The key problem with the Kellogg controversy was the sentiments over the \emcap{personality of God}, which were departing from the foundation of our faith, that God established at the beginning of our work. We have been told that \egwinline{Many things of like character will in the future arise}[Ms137-1903.10; 1903][https://egwwritings.org/read?panels=p9939.17]. In the book, the Living Temple, we see the sentiments regarding the \emcap{personality of God} and where His presence is, which were stepping off of the \emcap{Fundamental Principles}. This step was never supposed to be made! But we raise the question, where was this step heading? We will see the evidence that this step was heading toward the Trinity doctrine. Sister White prophesied that Kellogg’s step would lead toward the Omega heresy. Can we see the connection between Kellogg’s controversy and the Trinity doctrine?


Shida kuu ya mzozo wa Kellogg ilikuwa hisia juu ya \emcap{ubinafsi wa Mungu}, ambazo zilikuwa zinaondoka kutoka kwenye msingi wa imani yetu, ambao Mungu alianzisha katika mwanzo wa kazi yetu. Tumeambiwa kwamba \egwinline{Mambo mengi ya tabia kama hayo yatatokea katika siku zijazo}[Ms137-1903.10; 1903][https://egwwritings.org/read?panels=p9939.17]. Katika kitabu, the Living Temple, tunaona hisia kuhusu \emcap{ubinafsi wa Mungu} na mahali uwepo wake ulipo, ambazo zilikuwa zinaondoka kwenye \emcap{Kanuni za Msingi}. Hatua hii haikupaswa kufanywa kamwe! Lakini tunauliza swali, hatua hii ilikuwa inaelekea wapi? Tutaona uthibitisho kwamba hatua hiyo ilikuwa ikielekea kwenye fundisho la Utatu. Dada White alitabiri kwamba hatua ya Kellogg ingeongoza kwenye Omega wa uzushi. Tunaweza kuona uhusiano kati ya mabishano ya Kellogg na fundisho la Utatu?


In the following section, we want to present you with the connection between Kellogg’s controversy and the doctrine of Trinity. It is important to emphasize that the Living Temple does not contain this doctrine as it is believed today. The main problem with Kellogg’s teaching was the \textit{stepping off} of the \emcap{Fundamental Principles}, which were the foundation of our faith. The information we will present to you reveals that Dr. Kellogg justified his actions in stepping off of the foundation through his belief in the doctrine of Trinity. This is not difficult to see when we recognize that the \emcap{Fundamental Principles} were a non-Trinitarian. Our main focus should not be in recognizing the Trinity doctrine in Kellogg's arguments, but rather in understanding the differences between Kellogg’s teachings and the teachings of the \emcap{Fundamental Principles} regarding \egwinline{the personality of God and where His presence is}[SpTB02 51.3; 1903][https://egwwritings.org/read?panels=p417.262]. In other words, what were the steps Kellogg made in stepping off of the foundation of our faith? This approach is advocated by the Spirit of Prophecy and it will help us to avoid speculations regarding Kellogg’s motives—it will help us to focus upon the truth. Ellen White tells us that there are many good things written in the Living Temple, but they are mingled with specious, deceptive theories regarding the \emcap{personality of God} and \emcap{of Christ}.


Katika sehemu ifuatayo, tunataka kukuonyesha uhusiano kati ya mzozo wa Kellogg na fundisho la Utatu. Ni muhimu kusisitiza kwamba Hekalu Hai haina fundisho hili kama inavyoaminika leo. Tatizo kuu la ufundishaji wa Kellogg ulikuwa ni \textit{kuondoka} kutoka kwa \emcap{Kanuni za Msingi}, ambazo zilikuwa msingi wa imani yetu. Taarifa tutakazowasilisha kwako zinafichua kuwa Dk. Kellogg alihalalisha matendo yake ya kuondoka kutoka kwenye msingi kupitia imani yake katika fundisho la Utatu. Hii si vigumu kuona tunapotambua kwamba \emcap{Kanuni za Msingi} zilikuwa zisizo za Utatu. Lengo letu kuu halitakuwa katika kutambua fundisho la Utatu katika hoja za Kellogg, bali kuelewa tofauti kati ya mafundisho ya Kellogg na mafundisho ya \emcap{Kanuni za Msingi} kuhusu \egwinline{ubinafsi wa Mungu na mahali uwepo Wake ulipo}[SpTB02 51.3; 1903][https://egwwritings.org/read?panels=p417.262]. Katika maneno mengine, ni hatua gani Kellogg alichukua katika kuondoka kwenye msingi wa imani yetu? Mbinu hii inatetewa na Roho ya Unabii na itatusaidia kuepukana na mawazo kuhusu nia za Kellogg—itatusaidia kuzingatia ukweli. Ellen White anatuambia kwamba kuna mambo mengi mazuri yaliyoandikwa katika Hekalu Hai, lakini yamechanganyikana na nadharia za udanganyifu na za uongo kuhusu \emcap{ubinafsi wa Mungu} na \emcap{wa Kristo}.


\begin{figure}[hp]
    \centering
    \includegraphics[width=1\linewidth]{images/john-h-kellogg.jpg}
    \caption*{John Harvey Kellogg (1852-1943)}
    \label{fig:john-h-kellogg}
\end{figure}


\begin{figure}[hp]
    \centering
    \includegraphics[width=1\linewidth]{images/john-h-kellogg.jpg}
    \caption*{John Harvey Kellogg (1852-1943)}
    \label{fig:john-h-kellogg}
\end{figure}


\egw{\textbf{The book Living Temple contains specious, \underline{deceptive sentiments regarding the personality of God and of Christ}}. The Lord opened before me the true meaning of these sentiments, showing me that unless they were steadfastly repudiated, they would deceive the very elect. \textbf{Precious truth and beautiful sentiments were woven in with false, misleading theories. Thus truth was used to substantiate the \underline{most dangerous errors}. The precious representations of God are so misconstrued as to appear to uphold falsehoods \underline{originated by the great apostate}. Sentiments that belong to the revealings of God are mingled with specious, deceptive theories of satanic agencies}.}[Lt146-1905.2; 1905][https://egwwritings.org/read?panels=p9430.8]


\egw{\textbf{Kitabu Hekalu Hai kina maoni ya uwongo, \underline{ya udanganyifu kuhusu ubinafsi wa Mungu na wa Kristo}}. Bwana alifungua mbele yangu maana halisi ya hisia hizi, akinionyesha kwamba ikiwa mafundisho haya hayatakataliwa kwa uthabiti, yanaweza kuwapoteza, kama yamkini, hata walio wateule. \textbf{Ukweli wa thamani na hisia nzuri ziliunganishwa na uwongo, nadharia potofu. Hivyo ukweli ulitumiwa kuthibitisha \underline{makosa ya hatari zaidi}. Vielelezo vya thamani vya Mungu vimefafanuliwa vibaya sana hivi kwamba vinaonekana kuunga mkono udanganyifu \underline{ulioasisiwa na yule muasi mkuu}. Hisia ambazo ni za ufunuo ya Mungu zimechanganyika na nadharia zenye udanganyifu za mashirika ya kishetani}.}[Lt146-1905.2; 1905][https://egwwritings.org/read?panels=p9430.8]


\egwnogap{In the controversy over these theories \textbf{it has been asserted that I believed and taught the same things} that I have been instructed to condemn in the book Living Temple. \textbf{This I deny}. In the name of Jesus Christ of Nazareth, \textbf{I say that this is not so}.}[Lt146-1905.3; 1905][https://egwwritings.org/read?panels=p9430.9]


\egwnogap{Katika mabishano ya nadharia hizi \textbf{imedaiwa kwamba niliamini na kufundisha mambo yale yale} ambayo nimeagizwa kukemea katika kitabu Hekalu Hai. \textbf{Hili ninakataa}. Kwa jina la Yesu Kristo wa Nazareti, \textbf{nasema sivyo hivyo}.}[Lt146-1905.3; 1905][https://egwwritings.org/read?panels=p9430.9]


This mixture of truth and error makes the matter difficult. In the eyes of pro-trinitarian scholars, the problem is solely attributed to pantheism, and the evidence of Kellogg's belief in the Trinity doctrine is interpreted as belief in a false Trinity\footnote{Whidden, Woodrow W, et al. \textit{The Trinity : Understanding God's Love, His Plan of Salvation, and Christian Relationships}. Hagerstown, Md, Review And Herald Pub. Association, 2002., p. 217}. Sister White's rebuke is attributed to the defense of the “correct” Trinity, which she supposedly believed. Unfortunately, such interpretation does not acknowledge Sister White's defense of the \emcap{Fundamental Principles} regarding the \emcap{personality of God} and of Christ, thus it is a misinterpretation of her work. In the following sections, we will examine historical data on Dr. Kellogg's connection with the doctrine of Trinity from the perspective of the Adventist truth on the \emcap{personality of God}, which constituted the foundation of our faith. With this perspective, we believe that the historical data will shine in a new light and spark honest and constructive dialogue in our church.


Mchanganyiko huu wa ukweli na makosa hufanya jambo kuwa gumu. Katika macho ya wataalamu wanaounga mkono utatu, tatizo linahusishwa tu na pantheism, na ushahidi wa imani ya Kellogg katika fundisho la Utatu linafasiriwa kuwa imani katika Utatu wa uwongo\footnote{Whidden, Woodrow W, et al. \textit{The Trinity : Understanding God's Love, His Plan of Salvation, and Christian Relationships}. Hagerstown, Md, Review And Herald Pub. Association, 2002., p. 217}. Karipio la Dada White linahusishwa na utetezi wa Utatu “sahihi,” ambao inasemekana aliamini. Kwa bahati mbaya, tafsiri kama hiyo haihusishi utetezi wa Dada White wa \emcap{Kanuni za Msingi} kuhusu \emcap{ubinafsi wa Mungu} na wa Kristo, hivyo ni tafsiri mbaya ya kazi yake. Katika sehemu zifuatazo, tutaangalia data ya kihistoria kuhusu uhusiano wa Dk. Kellogg na fundisho la Utatu kutoka kwa mtazamo wa ukweli wa Waadventista juu ya \emcap{ubinafsi wa Mungu}, ambao ulikuwa msingi wa imani yetu. Tukiwa na mtazamo huu, tunaamini kwamba data ya kihistoria itaangaza katika mwanga mpya na kuibua mazungumzo ya uaminifu na yenye kujenga katika kanisa letu.


\section*{Correspondence of Dr. Kellogg and Brother Butler}


\section*{Mawasiliano ya Dk. Kellogg na Ndugu Butler}


In the following section we briefly present you with the well-known correspondence between Dr. Kellogg and G. I. Butler over the book, the Living Temple. Here, we see Dr. Kellogg’s objections regarding the controversy. He wrote to Brother Butler:


Katika sehemu ifuatayo tunawasilisha kwa ufupi mawasiliano yanayojulikana kati ya Dk. Kellogg na G. I. Butler juu ya kitabu, the Living Temple. Hapa, tunaona pingamizi ya Dk. Kellogg kuhusu mzozo huo. Alimwandikia Ndugu Butler:


\others{As far as I can fathom, the \textbf{difficulty }which is found \textbf{in ‘The Living Temple’,} \textbf{the whole thing may be simmered down to the question}: \textbf{\underline{Is the Holy Ghost a person}?} You say no. I had supposed the Bible said this for the reason that the personal pronoun ‘he’ is used in speaking of the Holy Ghost. \textbf{Sister White uses the pronoun ‘he’ and has said in so many words that the Holy Ghost is \underline{the third person of the Godhead}}. \textbf{How the Holy Ghost can be the third person and not be a person at all is difficult for me to see}.}[Letter: J. H. Kellogg to G. I. Butler. Oct 28. 1903][https://static1.squarespace.com/static/554c4998e4b04e89ea0c4073/t/5db9fbc96defed1e45b497a4/1572469707862/1903-10-28-Kellog-to-Butler.pdf]


\others{Kadiri ninavyoweza kufahamu, \textbf{utata} unaopatikana \textbf{katika ‘The Living Temple’,} \textbf{kwa ujumla jambo linaweza kuchemshwa hadi kwenye swali}: \textbf{\underline{Je, Roho Mtakatifu ni nafsi}?} Unasema hapana. Nilidhani kwamba Biblia ilisema hivi kwa sababu kiwakilishi cha kibinafsi ‘yeye’ kinatumika akizungumza juu ya Roho Mtakatifu. \textbf{Dada White anatumia kiwakilishi ‘yeye’ na amesema katika maneno mengi sana kwamba Roho Mtakatifu ni \underline{nafsi ya tatu ya Uungu}}. \textbf{Jinsi Roho Mtakatifu anaweza kuwa nafsi ya tatu na asiwe nafsi hata kidogo ni vigumu kwangu kuona}.}[Letter: J. H. Kellogg to G. I. Butler. Oct 28. 1903][https://static1.squarespace.com/static/554c4998e4b04e89ea0c4073/t/5db9fbc96defed1e45b497a4/1572469707862/1903-10-28-Kellog-to-Butler.pdf]


\begin{figure}[hp]
    \centering
    \includegraphics[width=1\linewidth]{images/george-ide-butler.jpg}
    \caption*{George Ide Butler (1834-1918)}
    \label{fig:g-i-butler}
\end{figure}


\begin{figure}[hp]
    \centering
    \includegraphics[width=1\linewidth]{images/george-ide-butler.jpg}
    \caption*{George Ide Butler (1834-1918)}
    \label{fig:g-i-butler}
\end{figure}


According to Dr. Kellogg’s perspective, the whole problem with the book ‘The Living Temple’ comes down to the question “\textit{Is the Holy Ghost a person?}”. Obviously, he does not advocate an impersonal God, as he is often accused of\footnote{Whidden, Woodrow W, et al. \textit{The Trinity : Understanding God's Love, His Plan of Salvation, and Christian Relationships}. Hagerstown, Md, Review And Herald Pub. Association, 2002., p. 217}. Moreover, he even believes that the Holy Ghost is a \textit{third person of the Godhead}. Also, he claims that Brother Butler does not believe that the Holy Ghost is a person. The problem obviously lies in the definition of the word \textit{‘person’}. On this point, Kellogg continues:


Kulingana na mtazamo wa Dk. Kellogg, tatizo zima la kitabu ‘The Living Temple’ inakuja kwa swali “\textit{Je! Roho Mtakatifu ni nafsi?}”. Ni wazi, yeye hatetei Mungu asiye na nafsi, kama anavyoshutumiwa mara nyingi\footnote{Whidden, Woodrow W, et al. \textit{The Trinity : Understanding God's Love, His Plan of Salvation, and Christian Relationships}. Hagerstown, Md, Review And Herald Pub. Association, 2002., p. 217}. Zaidi ya hayo, hata anaamini kwamba Roho Mtakatifu ni \textit{nafsi ya tatu ya Uungu}. Pia, anadai kwamba Ndugu Butler haamini kwamba Roho Mtakatifu ni nafsi. Tatizo ni wazi liko katika ufafanuzi wa neno \textit{‘nafsi’}. Katika hatua hii, Kellogg anaendelea:


\others{I believe this Spirit of God to be a personality you don’t. But this is purely a question of definition. \textbf{I believe the Spirit of God is a personality}; you say, No, it is not a personality. Now the only reason why we differ is because we \textbf{differ in our ideas as to \underline{what a personality is}}. \textbf{Your idea of personality is perhaps that of \underline{semblance to a person} or a human being}.}[Letter: J. H. Kellogg to G. I. Butler. Oct 28. 1903][https://static1.squarespace.com/static/554c4998e4b04e89ea0c4073/t/5db9fbc96defed1e45b497a4/1572469707862/1903-10-28-Kellog-to-Butler.pdf]


\others{Ninaamini huyu Roho wa Mungu kuwa nafsi wewe huamini. Lakini hili ni swali tu la ufafanuzi. \textbf{Ninaamini Roho wa Mungu ni nafsi}; unasema, Hapana, sio nafsi. Sasa sababu pekee kwa nini tunatofautiana ni kwa sababu \textbf{tunatofautiana katika mawazo yetu kuhusu \underline{nini ni nafsi}}. \textbf{Wazo lako la nafsi labda ni la \underline{kufanana na mtu} au na binadamu}.}[Letter: J. H. Kellogg to G. I. Butler. Oct 28. 1903][https://static1.squarespace.com/static/554c4998e4b04e89ea0c4073/t/5db9fbc96defed1e45b497a4/1572469707862/1903-10-28-Kellog-to-Butler.pdf]


Brother Butler replied:


Ndugu Butler alijibu:


\others{\textbf{So far as Sister White and you being in perfect agreement, I shall have to leave that entirely between you and Sister White. \underline{Sister White says there is not perfect agreement; you claim there is}. \underline{I know some of her remarks seem to give you strong ground for claiming that she does}. I am candid enough to say that, but I must give her the credit until she disowns it of saying there is a difference too, and I do not believe you can fully tell just what she means. \underline{God dwells in us by His Holy Spirit}, as a Comforter, as a Reprover, especially the former. When we come to Him we partake of Him in that sense, because the Spirit comes forth from Him; \underline{it comes forth from the Father and the Son}. It is not a person walking around on foot, or flying \underline{as a literal being}, \underline{in any such sense as Christ and the Father are} – at least, if it is, it is utterly beyond my comprehension of the meaning of language or words}.}[Letter: G. I. Butler to J. H. Kellogg. April 5. 1904]


\others{\textbf{Ikiwa Dada White na wewe mko katika makubaliano kamili, itabidi niachane na hilo swala kabisa liwe kati yako na Dada White. \underline{Dada White anasema hakuna makubaliano kamilifu; unadai ipo}. \underline{Najua baadhi ya maandishi yake yanaonekana kukupa nguvu katika kudai kuwa anafanya hivyo}. Niko wazi vya kutosha kusema hivyo, lakini lazima nimpe nafasi yake mpaka akanushe kwa kusema kuna tofauti pia, na siamini unaweza kusema kikamilifu kile anachomaanisha. \underline{Mungu anakaa ndani yetu kwa Roho wake Mtakatifu}, kama Mfariji, kama Mkemeaji, hasa yule wa kwanza. Tunapokuja Kwake tunamshiriki yeye kwa sababu Roho hutoka Kwake; \underline{inatoka kwa Baba na Mwana}. Si nafsi anayetembea kwa miguu, au kuruka \underline{kama huluki halisi}, \underline{kwa maana yoyote kama Kristo na Baba walivyo} - angalau, ikiwa ni hivyo, ni zaidi ya ufahamu wangu wa maana ya lugha au maneno}.}[Letter: G. I. Butler to J. H. Kellogg. April 5. 1904]


The given correspondence is crucial for understanding the Kellogg controversy. Kellogg himself stated, \others{the whole thing may be simmered down to the question: \textbf{Is the Holy Ghost a person?}} Similarly Dr. Kellogg wrote to William White: \others{I have been studying very carefully to see what is \textbf{the real root of the difficulty with the Living Temple}, and as far as I can see \textbf{\underline{the whole question} resolves itself into this: \underline{Is the Holy Ghost, a person}?}}[Letter J. H. Kellogg to William White, October 28, 1903][https://drive.google.com/file/d/1\_S4S-Hc0K7Ka8gda9oRhPuAb9XzBTwmb/view] How does Kellogg's conclusion compare to the review and instruction of heavenly origin, which clearly told us that the reasoning in the Living Temple is \egwinline{naught but speculation in regard to \textbf{the personality of God and where His presence is}}[SpTB02 51.3; 1904][https://egwwritings.org/read?panels=p417.262]? In the writings of Ellen White and the pioneers, the term ‘\textit{personality of God}’ refers specifically to the personality of the Father. So, why does Kellogg claim that the real issue is the personality of the Holy Spirit, when God indicated that the issue concerns the personality of the Father?


Mawasiliano yaliyotolewa ni muhimu kwa kuelewa mgogoro wa Kellogg. Kellogg mwenyewe alisema, \others{jambo zima linaweza kuchambuliwa hadi kukifikia kiini chake kwa swali: \textbf{Je, Roho Mtakatifu ni nafsi?}} Vivyo hivyo Dk. Kellogg alimwandikia William White: \others{Nimekuwa nikisoma kwa makini sana kuona \textbf{ni nini kiini cha tatizo la Hekalu Hai}, na kadiri ninavyoweza kuona \textbf{\underline{swali zima} linajikita katika hili: \underline{Je, Roho Mtakatifu ni nafsi}?}}[Letter J. H. Kellogg to William White, October 28, 1903][https://drive.google.com/file/d/1\_S4S-Hc0K7Ka8gda9oRhPuAb9XzBTwmb/view] Je! hitimisho la Kellogg linaendeleaje kulinganishwa na mapitio na maelekezo ya asili ya mbinguni, ambayo yalituambia wazi kwamba mantiki katika Hekalu Hai ni \egwinline{si chochote ila ni dhana tu kuhusiana na \textbf{Umbile la Mungu na mahali uwepo wake ulipo}}[SpTB02 51.3; 1904][https://egwwritings.org/read?panels=p417.262]? Katika maandiko ya Ellen White na waanzilishi, neno ‘\textit{Umbile la Mungu}’ linahusu hasa Umbile la Baba. Kwa hiyo, kwa nini Kellogg anadai kwamba suala halisi ni Umbile la Roho Mtakatifu, wakati Mungu alionyesha kwamba suala linahusu Umbile la Baba?


Many assume that Dr. Kellogg is being manipulative, evading the core issue. However, under a particular premise, his arguments concerning the personality of the Holy Spirit logically support his controversial views on the \emcap{personality of God}. This premise becomes evident within the data itself when we closely follow his reasoning.


Wengi wanadhani kwamba Dk. Kellogg anajaribu kudanganya, akikwepa suala la msingi. Hata hivyo, chini ya dhana fulani, hoja zake kuhusu Umbile la Roho Mtakatifu kimatiki zinaunga mkono maoni yake ya utata kuhusu \emcap{Umbile la Mungu}. Dhana hii inakuwa dhahiri ndani ya data yenyewe tunapofuata kwa makini mantiki yake.


As we have seen earlier, the doctrine on the \emcap{personality of God} teaches that God, the Father, possesses a form—a tangible, material body. Dr. Kellogg concurred that this assertion holds true within the bounds of our finite conception of God\footnote{\href{https://archive.org/details/J.H.Kellogg.TheLivingTemple1903/page/n33/}{Dr. John H. Kellogg, The Living Temple, p.31.}}. However, he argued that, in reality, God transcends our conceptions regarding His form, as He is beyond the constraints of space\footnote{\href{https://archive.org/details/J.H.Kellogg.TheLivingTemple1903/page/n33/}{Dr. John H. Kellogg, The Living Temple, p.33.}}. In this sense, Kellogg effectively does away with the reality of God’s physical, material body. The premise that would validate Dr. Kellogg’s viewpoint is the \textit{exclusive equivalence} in understanding the \emcap{personality of God} and that of the Holy Spirit. Is the Holy Spirit constrained by space? No, He is not. Does the Holy Spirit have a physical body? No! According to Jesus, \bible{for a spirit hath not flesh and bones}[Luke 24:39]. Is the Holy Ghost a person? The answer hinges on our interpretation of what it means to be a person. What is that quality or state of the Holy Spirit being a person?\footnote{Direct application of the definition on the word ‘\textit{personality}’ from the \href{https://www.merriam-webster.com/dictionary/personality}{Merriam Webster Dictionary}} When comparing Dr. Kellogg's belief in the personality of the Holy Spirit with Brother Butler's views, it becomes evident that the quality of the Holy Spirit being a person does not align with \others{that of \textbf{semblance to a person} or a human being}. Butler explicitly stated his criteria for this determination\footnote{In his letter to Dr. Kellogg, Brother Butler further asserted that there is no distinction between the person and the bodily presence. See \href{https://c7da.us/egwdl/Butler\%20to\%20Kellogg\%20Aug121904.pdf}{Letter from Butler to Kellogg, August 12, 1904, p.6}}: \others{\textbf{It is not a person walking around on foot, or flying \underline{as a literal being}, \underline{in any such sense as Christ and the Father are} – at least, if it is, it is utterly beyond my comprehension of the meaning of language or words}}.


Kama tulivyoona hapo awali, fundisho juu ya \emcap{Umbile la Mungu} linafundisha kwamba Mungu, Baba, ana umbo—mwili unaoonekana, wa nyenzo. Dk. Kellogg alikubali kwamba dai hili ni kweli ndani ya mipaka ya uelewa wetu finyu wa Mungu\footnote{\href{https://archive.org/details/J.H.Kellogg.TheLivingTemple1903/page/n33/}{Dr. John H. Kellogg, The Living Temple, p.31.}}. Hata hivyo, alihoji kwamba, kwa kweli, Mungu anazidi dhana zetu kuhusu umbo lake, kwani yuko nje ya vikwazo vya nafasi\footnote{\href{https://archive.org/details/J.H.Kellogg.TheLivingTemple1903/page/n33/}{Dr. John H. Kellogg, The Living Temple, p.33.}}. Kwa maana hii, Kellogg kwa kweli anaondoa uhalisia wa mwili wa kimwili, wa nyenzo wa Mungu. Dhana ambayo ingethibitisha mtazamo wa Dk. Kellogg ni \textit{ulinganifu wa kipekee} katika kuelewa \emcap{Umbile la Mungu} na lile la Roho Mtakatifu. Je, Roho Mtakatifu anazuiliwa na nafasi? Hapana, hazuiliwi. Je, Roho Mtakatifu ana mwili wa kimwili? Hapana! Kulingana na Yesu, \bible{kwa maana roho haina mwili wala mifupa}[Luka 24:39]. Je, Roho Mtakatifu ni nafsi? Jibu linategemea tafsiri yetu ya maana ya kuwa nafsi. Ni nini ubora au hali ya Roho Mtakatifu kuwa nafsi?\footnote{Matumizi ya moja kwa moja ya ufafanuzi wa neno ‘\textit{Umbile}’ kutoka \href{https://www.merriam-webster.com/dictionary/personality}{Kamusi ya Merriam Webster}} Tunapolinganisha imani ya Dk. Kellogg katika Umbile la Roho Mtakatifu na maoni ya Ndugu Butler, inakuwa dhahiri kwamba ubora wa Roho Mtakatifu kuwa nafsi hauendani na \others{ule wa \textbf{kufanana na mtu} au binadamu}. Butler alieleza wazi vigezo vyake kwa uamuzi huu\footnote{Katika barua yake kwa Dk. Kellogg, Ndugu Butler pia alidai kwamba hakuna tofauti kati ya nafsi na uwepo wa kimwili. Tazama \href{https://c7da.us/egwdl/Butler\%20to\%20Kellogg\%20Aug121904.pdf}{Barua kutoka Butler kwa Kellogg, Agosti 12, 1904, uk.6}}: \others{\textbf{Si nafsi anayetembea kwa miguu, au kuruka \underline{kama huluki halisi}, \underline{kwa maana yoyote kama Kristo na Baba walivyo} - angalau, ikiwa ni hivyo, ni zaidi ya ufahamu wangu wa maana ya lugha au maneno}}.


Have you noticed that Brother Butler addressed Kellogg’s unspoken premise? Butler drew a distinction between the Father and Christ, in relation to the Holy Spirit. Brother Butler is correct. There exists a contrast between the personality of the Holy Spirit and that of God and Christ. Christ and the Father possess a physical form of a person, whereas the Holy Spirit does not. To do away with the physical form of a person of the Father is to \textit{exclusively equate} the understanding of the personality of the Father with that of the Holy Spirit. Kellogg’s approach is compelling, because it was backed by valid arguments regarding the personality of the Holy Spirit.


Je, umeona kwamba Ndugu Butler alishughulikia dhana ya Kellogg ambayo hakusema? Butler aliweka tofauti kati ya Baba na Kristo, kuhusiana na Roho Mtakatifu. Ndugu Butler yuko sahihi. Kuna tofauti kati ya Umbile la Roho Mtakatifu na lile la Mungu na Kristo. Kristo na Baba wana umbo la kimwili la mtu, wakati Roho Mtakatifu hana. Kuondoa umbo la kimwili la nafsi ya Baba ni \textit{kulinganisha kwa kipekee} uelewa wa Umbile la Baba na lile la Roho Mtakatifu. Mtazamo wa Kellogg ni wa kushawishi, kwa sababu uliungwa mkono na hoja halali kuhusu Umbile la Roho Mtakatifu.


Let us briefly examine the personality of the Holy Spirit. What is the quality or state of the Holy Spirit being a person?


Hebu tuchunguze kwa ufupi Umbile la Roho Mtakatifu. Ni nini ubora au hali ya Roho Mtakatifu kuwa nafsi?


\egw{\textbf{The Holy Spirit has a personality}, \textbf{\underline{else} }He could not \textbf{bear witness} to our spirits and with our spirits that we are the children of God. \textbf{He must also be a \underline{divine person}}, \textbf{\underline{else}} He could not \textbf{search out the secrets} which lie hidden \textbf{in the mind of God}.}[21LtMs, Ms 20, 1906, par. 32; 1906][https://egwwritings.org/read?panels=p14071.10296041&index=0]


\egw{\textbf{Roho Mtakatifu ana Umbile}, \textbf{\underline{vinginevyo} }Asingeweza \textbf{kushuhudia} kwa roho zetu na pamoja na roho zetu kwamba sisi ni watoto wa Mungu. \textbf{Lazima pia awe \underline{Nafsi ya kimungu}}, \textbf{\underline{vinginevyo}} Asingeweza \textbf{kuchunguza siri} zilizofichika \textbf{katika akili ya Mungu}.}[21LtMs, Ms 20, 1906, par. 32; 1906][https://egwwritings.org/read?panels=p14071.10296041&index=0]


\egw{\textbf{The Holy Spirit is a person}; \textbf{\underline{for}} He \textbf{beareth witness} with our spirits that we are the children of God.}[21LtMs, Ms 20, 1906, par. 31; 1906][https://egwwritings.org/read?panels=p14071.10296040&index=0]


\egw{\textbf{Roho Mtakatifu ni nafsi}; \textbf{\underline{kwa}} Yeye \textbf{hushuhudia} pamoja na roho zetu kwamba sisi ni watoto wa Mungu.}[21LtMs, Ms 20, 1906, par. 31; 1906][https://egwwritings.org/read?panels=p14071.10296040&index=0]


The qualities or states that define the Holy Spirit as a person are explicitly mentioned in the provided quotations. These include the ability to bear witness and search out the mind. Further support can be found in Scripture, which attributes actions to the Holy Spirit such as speaking (\textit{Acts 13:2}), teaching (\textit{John 14:26; 1 Corinthians 2:13}), making decisions (\textit{Acts 15:28}), and experiencing emotions (\textit{Ephesians 4:30}), among others. These \textit{qualities }collectively affirm the personality of the Holy Spirit. Can these same qualities be also applied to the Father and the Son? Most certainly. However, unlike the Father and the Son, the Holy Spirit is distinguished by the absence of a material, tangible form. When Ellen White questioned Christ about the \emcap{personality of God}, her inquiry specifically targeted the personal form as the defining quality of the Father's personality.


Sifa au hali zinazomfafanua Roho Mtakatifu kama nafsi zimetajwa wazi katika nukuu zilizotolewa. Hizi ni pamoja na uwezo wa kushuhudia na kuchunguza akili. Ushahidi zaidi unaweza kupatikana katika Maandiko, ambayo yanampa Roho Mtakatifu matendo kama vile kuzungumza (\textit{Matendo 13:2}), kufundisha (\textit{Yohana 14:26; 1 Wakorintho 2:13}), kufanya maamuzi (\textit{Matendo 15:28}), na kuwa na hisia (\textit{Waefeso 4:30}), miongoni mwa nyingine. Hizi \textit{sifa} kwa pamoja zinathibitisha Umbile la Roho Mtakatifu. Je, sifa hizi zinaweza kutumika pia kwa Baba na Mwana? Bila shaka. Hata hivyo, tofauti na Baba na Mwana, Roho Mtakatifu anatofautishwa kwa kukosa umbo la kimwili, linaloonekana. Wakati Ellen White alipomwuliza Kristo kuhusu \emcap{Umbile la Mungu}, swali lake hasa lililenga umbo la kibinafsi kama sifa inayofafanua Umbile la Baba.


\egw{I have often \textbf{seen }the lovely Jesus, that \textbf{He is a person}. \textbf{I asked Him if His Father \underline{was a person} and \underline{had a form} like Himself}. Said Jesus, ‘\textbf{I am in the express image of My Father's person}.’}[EW 77.1; 1882][https://egwwritings.org/read?panels=p28.490&index=0]


\egw{Mara nyingi \textbf{nimemwona} Yesu mzuri, kwamba \textbf{Yeye ni nafsi}. \textbf{Nilimwuliza kama Baba Yake \underline{alikuwa nafsi} na \underline{alikuwa na umbo} kama Yeye}. Yesu alisema, ‘\textbf{Mimi ni chapa kamili ya Umbile la Baba yangu}.’}[EW 77.1; 1882][https://egwwritings.org/read?panels=p28.490&index=0]


This brings us to a profound distinction in how the personality of the Holy Spirit is understood, as opposed to that of the Father and the Son. Ellen White describes the Holy Spirit as a spiritual manifestation of Christ, drawing a clear line between the outward, visible manifestation of Christ and His spiritual manifestation. This contrast underscores the unique nature of the Holy Spirit's presence and action in the world, distinct from the physical presence of Christ and the Father. Pay attention to the contrast between the outward, visible manifestation of Christ, and His spiritual manifestation:


Hii inatuleta kwenye tofauti kubwa katika jinsi Umbile la Roho Mtakatifu linavyoeleweka, tofauti na lile la Baba na Mwana. Ellen White anaelezea Roho Mtakatifu kama dhihirisho la kiroho la Kristo, akiweka mstari wazi kati ya dhihirisho la nje, linaloonekana la Kristo na dhihirisho lake la kiroho. Tofauti hii inasisitiza asili ya kipekee ya uwepo na utendaji wa Roho Mtakatifu duniani, tofauti na uwepo wa kimwili wa Kristo na Baba. Zingatia tofauti kati ya dhihirisho la nje, linaloonekana la Kristo, na dhihirisho lake la kiroho:


\egw{That \textbf{Christ }should \textbf{manifest Himself} to them, and yet \textbf{be invisible to the world}, was a mystery to the disciples. They could not understand \textbf{the words of Christ in their \underline{spiritual sense}}. \textbf{They were thinking of \underline{the outward, visible manifestation}}. They could not take in the fact that they could have \textbf{the presence of Christ with them}, and \textbf{yet He be unseen by the world}. \textbf{They did not understand the meaning of \underline{a spiritual manifestation}}.}[ST November 18, 1897, par. 6; 1897][https://egwwritings.org/read?panels=p820.14727&index=0]


\egw{Kwamba \textbf{Kristo} angeweza \textbf{kujidhihirisha} kwao, na bado \textbf{asiwe anaonekana kwa ulimwengu}, ilikuwa siri kwa wanafunzi. Hawakuweza kuelewa \textbf{maneno ya Kristo katika \underline{maana yake ya kiroho}}. \textbf{Walikuwa wanafikiria \underline{dhihirisho la nje, linaloonekana}}. Hawakuweza kuelewa ukweli kwamba wangeweza kuwa na \textbf{uwepo wa Kristo pamoja nao}, na \textbf{bado Yeye asionekane na ulimwengu}. \textbf{Hawakuelewa maana ya \underline{dhihirisho la kiroho}}.}[ST November 18, 1897, par. 6; 1897][https://egwwritings.org/read?panels=p820.14727&index=0]


The Holy Spirit is not a person in the physical sense but is manifested in a spiritual sense. If the exclusive understanding of the personality of the Holy Spirit is applied to the Father, then consequently His physical form of a person is done away. His personality is spiritualized. This is why Ellen White critically labeled Kellogg's perspective as spiritualism. Do you know which doctrine, in particular, has a core tenet, that the Father and the Holy Spirit are co-equal in their personalities? It is \textit{the doctrine of the trinity}. Could it be possible that Dr. Kellogg was actually raising the theological side of questions of the trinity?


Roho Mtakatifu si nafsi kwa maana ya kimwili bali anadhihirishwa kwa maana ya kiroho. Ikiwa uelewa wa kipekee wa Umbile la Roho Mtakatifu utatumika kwa Baba, basi kwa matokeo yake umbo lake la kimwili la nafsi linaondolewa. Umbile lake linafanywa kuwa la kiroho. Hii ndiyo sababu Ellen White alitaja mtazamo wa Kellogg kama uchawi. Je, unajua fundisho gani, hasa, lina kanuni kuu, kwamba Baba na Roho Mtakatifu ni sawa katika Umbile lao? Ni \textit{fundisho la utatu}. Je, inawezekana kwamba Dk. Kellogg alikuwa kwa kweli anainua upande wa kitheolojia wa maswali ya utatu?


\section*{Kellogg’s confession about the Living Temple}


\section*{Ungamo la Kellogg kuhusu Hekalu Hai}


In his interview with G. W. Amadon and A. C. Bourdeau, one month after being disfellowshipped, he confessed that he unintentionally brought the theological side of the question of the Trinity into his book “The Living Temple”.


Katika mahojiano yake na G. W. Amadon na A. C. Bourdeau, mwezi mmoja baada ya kutengwa na ushirika, alikiri kwamba bila kukusudia alileta upande wa kitheolojia wa swali la Utatu katika kitabu chake “The Living Temple”.


\others{\textbf{Now, I thought I had cut out entirely the theological side of questions of \underline{the trinity and all that sort of things}}. \textbf{I didn't mean to \underline{put it in} at all}, and I took pains to state in the preface that I did not. I never dreamed of such a thing as \textbf{any theological question being} \textbf{\underline{brought into it}}. I only wanted to show that \textbf{the heart does not beat of its own motion but that it is the power of God that keeps it going}.}[Kellogg vs. The Brethren: His Last Interview as an Adventist, p. 58.][https://forgotten-pillar.s3.us-east-2.amazonaws.com/1990\_kellogg\_vs\_brethren\_lastInterview\_oct7\_1907\_spectrum\_v20\_n3-4.pdf]


\others{\textbf{Sasa, nilifikiri nilikuwa nimeondoa kabisa upande wa kitheolojia wa maswali ya \underline{utatu na aina hiyo ya mambo}}. \textbf{Sikukusudia \underline{kuiweka ndani} hata kidogo}, na nilichukua tahadhari kueleza katika dibaji kwamba sikufanya hivyo. Sikuwahi kufikiri chochote kuhusu \textbf{swali lolote la kitheolojia} \textbf{\underline{kuletwa ndani yake}}. Nilitaka tu kuonyesha kwamba \textbf{moyo haupigi kwa mwendo wake bali kwamba ni uweza wa Mungu unaoufanya uendelee}.}[Kellogg vs. The Brethren: His Last Interview as an Adventist, p. 58.][https://forgotten-pillar.s3.us-east-2.amazonaws.com/1990\_kellogg\_vs\_brethren\_lastInterview\_oct7\_1907\_spectrum\_v20\_n3-4.pdf]


If we were to look in his book for trinitarian expressions, we would not find any. Would that be a proof that Kellogg is disingenuous in his confession? The only thing we find is the teaching that is stepping off of the foundation of our faith—the \emcap{fundamental principles}—regarding the \emcap{personality of God} and where His presence is. The trinitarian expressions are not there but his sentiments regarding the \emcap{personality of God} are in line with the trinitarian sentiments on God’s person. These sentiments are deceptive and Kellogg was rebuked for them. When he wanted to explicitly state the belief in the Trinity doctrine, in hopes of fixing the book, he was again rebuked by the words, \egwinline{\textbf{Patchwork theories} cannot be accepted by those who are loyal to the faith} and to the \emcap{Fundamental Principles}\footnote{\href{https://egwwritings.org/?ref=en_Lt253-1903.28&para=9980.36}{EGW, Lt253-1903.28; 1903}}. The crucial problem of the Trinity doctrine, in regard to the \emcap{personality of God}, is the underlying assumption that all Three, the Father, the Son, and the Holy Spirit, possess the same type of personality in such a way that They make one monotheistic God. In this light, we may understand Kellogg's assertions over the personality of the Holy Spirit, that the Holy Spirit is the third person of the Godhead. Dr. Kellogg quoted Ellen White when asserting his claims; although he used the same words, he had a wrong sentiment. In light of Dr. Kellogg’s confession, for including \others{\textbf{the theological side of questions of \underline{the trinity}}}, and His assertion that \others{\textbf{the whole thing may be simmered down to the question}: \textbf{\underline{Is the Holy Ghost a person}}?}, we may see the unspoken premise that the Father and the Son are in the same way persons as is the Holy Spirit. This is why Brother Butler wrote to him regarding the personality of the Holy Spirit: \others{\textbf{It is not a person walking around on foot, or flying \underline{as a literal being}, \underline{in any such sense as Christ and the Father are} – at least, if it is, it is utterly beyond my comprehension of the meaning of language or words.}}[Letter from G. I. Butler to J. H. Kellogg, April 5 1904.]


Kama tungetafuta katika kitabu chake maneno ya utatu, hatungepata yoyote. Je! hilo lingekuwa ushahidi kwamba Kellogg hana ufahamu katika kukiri kwake? Kitu pekee tunachopata ni mafundisho ambayo yanaondoka kutoka kwenye msingi wa imani yetu—\emcap{kanuni za kimsingi}—kuhusu \emcap{ubinafsi wa Mungu} na mahali uwepo wake ulipo. Semi za utatu hazipo hapo bali hisia zake kuhusu \emcap{ubinafsi wa Mungu} zinapatana na hisia za utatu kuhusu ubinafsi wa Mungu. Hisia hizi ni za udanganyifu na Kellogg alikemewa kwayo. Alipotaka kueleza kwa uwazi imani katika fundisho la Utatu, kwa matumaini ya kurekebisha kitabu, alikemewa tena kwa maneno, \egwinline{\textbf{Nadharia za kiraka} haziwezi kukubaliwa na wale walio waaminifu kwa imani} na kwa \emcap{Kanuni za Msingi}\footnote{\href{https://egwwritings.org/?ref=en_Lt253-1903.28&para=9980.36}{EGW, Lt253-1903.28; 1903}}. Tatizo la muhimu la fundisho la Utatu, kuhusiana na \emcap{ubinafsi wa Mungu}, ndilo dhana ya msingi ambayo wote Tatu, Baba, Mwana, na Roho Mtakatifu, wana aina moja ya ubinafsi hivi kwamba Hao hujumuisha Mungu mmoja. Katika mwanga huu, tunaweza kuelewa madai ya Kellogg juu ya ubinafsi wa Roho Mtakatifu, kwamba Roho Mtakatifu ni nafsi ya tatu ya Uungu. Dk. Kellogg alimnukuu Ellen White wakati akisisitiza madai yake; ingawa alitumia maneno sawa, alikuwa na hisia mbaya. Kwa kuzingatia kukiri kwa Dk. Kellogg, kwa kujumuisha \others{\textbf{upande wa kitheolojia wa maswali ya \underline{utatu}}}, na madai yake kwamba \others{\textbf{jambo zima inaweza kuchambuliwa hadi swali}: \textbf{\underline{Je, Roho Mtakatifu ni nafsi}}?}, tunaweza kuona dhana fiche kwamba Baba na Mwana ni Nafsi kwa njia sawa na Roho Mtakatifu. Hii ndiyo sababu Ndugu Butler alimwandikia kuhusu ubinafsi wa Roho Mtakatifu: \others{\textbf{Siyo Nafsi anayetembea kwa miguu, au kuruka \underline{kama huluki halisi}, \underline{sawia na jinsi Kristo na Baba walivyo} – angalau, ikiwa ni, ni zaidi ya ufahamu wangu ya maana ya lugha au maneno.}}[Letter from G. I. Butler to J. H. Kellogg, April 5 1904.]


\section*{The presence of God manifested in nature}


\section*{Uwepo wa Mungu unadhihirika katika asili}


From the works of our pioneers, we have seen that the personality of the Holy Ghost is most clearly expressed in terms of God's presence. Sister White told us that the Living Temple \egwinline{introduces that which is naught but speculation in \textbf{regard to the personality of God and where His presence is}.}[SpTB02 51.3; 1904][https://egwwritings.org/read?panels=p417.262] The \emcap{personality of God} and where His presence is are two mutually inclusive doctrines; one affirms the other. Deny one, and you deny the other. This notion is clearly seen in the book, the Living Temple. In the previous sections, we read Kellogg's arguments for the \emcap{personality of God} taken from his book. He argued that it is unprofitable to talk about God's shape or any tangible form. He raised skepticism in the reality of God as a definite, material, and tangible Being. If God is spirit, possessing no form nor body, then He is not restricted in His presence to one locality; this was the sentiment Kellogg advocated in the Living Temple.


Kutoka kwa kazi za waanzilishi wetu, tumeona kwamba ubinafsi wa Roho Mtakatifu ni zaidi imeonyeshwa wazi katika suala la uwepo wa Mungu. Dada White alituambia kwamba Hekalu Hai \egwinline{hutanguliza yale ambayo si chochote ila ni dhana tu kuhusiana na \textbf{ubinafsi wa Mungu na mahali ambapo uwepo wake upo}.}[SpTB02 51.3; 1904][https://egwwritings.org/read?panels=p417.262] \emcap{Ubinafsi wa Mungu} na mahali palipo na uwepo wake ni mafundisho mawili yanayojumuisha pande zote; mmoja inathibitisha nyingine. Kataa moja, unakana lingine. Dhana hii inaonekana wazi katika kitabu, Hekalu Hai. Katika sehemu zilizopita, tunasoma hoja za Kellogg za \emcap{ubinafsi wa Mungu} zilizochukuliwa kutoka katika kitabu chake. Alipinga kuwa haina maana kuzungumza kuhusu umbo la Mungu au namna yoyote inayoonekana. Alikana ukweli wa Mungu kama nyenzo iliyo dhahiri na huluki anayeonekana. Ikiwa Mungu ni roho, hana umbo wala mwili, basi hasitiriwi katika uwepo wake kwa eneo moja; Haya ndiyo maoni ambayo Kellogg aliyatetea katika Hekalu Hai.


\others{Says one, ‘\textbf{God may be \underline{present by his Spirit}, or by his power, but \underline{certainly God himself} cannot be present everywhere at once}.’ We answer: How can power be separated from the source of power? \textbf{Where God's Spirit is at work}, where God's power is manifested, \textbf{God \underline{himself} is actually and truly present}…}[John H. Kellogg, The Living Temple, p.28.][https://archive.org/details/J.H.Kellogg.TheLivingTemple1903/page/n29/]


\others{Mmoja husema, ‘\textbf{Mungu anaweza kuwapo kwa Roho wake, au kwa nguvu zake, lakini \underline{hakika Mungu mwenyewe} hawezi kuwapo kila mahali mara moja kwa wakati mmoja}.’ Tunajibu: Nguvu inaweza kuwaje kutengwa na chanzo cha nguvu? \textbf{Ambapo Roho wa Mungu anafanya kazi}, ambapo kuna nguvu za Mungu iliyodhihirika, \textbf{Mwenyezi Mungu \underline{yumo} na hakika yuko}…}[John H. Kellogg, The Living Temple, p.28.][https://archive.org/details/J.H.Kellogg.TheLivingTemple1903/page/n29/]


When Dr. Kellogg wrote \others{Says one, ‘God may be present by His Spirit…’}, he referred to the sentiments of our pioneers who were loyal to the \emcap{Fundamental Principles}. This is the most obvious point where Dr. Kellogg stepped off from the \emcap{Fundamental Principles}. Is this step in harmony with the doctrine of the Trinity? Examining our current stance in the Fundamental Beliefs \#2, we see that one God, as a unity of three persons, is not everywhere present through the agency of the Holy Spirit, but rather is everywhere present by Himself.


Wakati Dk. Kellogg aliandika \others{Anasema mmoja, ‘Mungu anaweza kuwa kwa Roho Wake…‘}, alirejelea hisia za mapainia wetu ambao walikuwa waaminifu kwa \emcap{Kanuni za Msingi}. Hapa ndipo mahali dhahiri ambapo Dk. Kellogg aliondoka kutoka kwenye \emcap{Kanuni za Msingi}. Je, hatua hii inapatana na fundisho la Utatu? Tukichunguza msimamo wetu wa sasa katika Mafundisho za Kimsingi \#2, tunaona kwamba Mungu mmoja, kama umoja wa nafsi tatu, hayupo kila mahali kupitia wakala wa Roho Mtakatifu, bali yupo kila mahali kwa Nafsi yake.


\others{There is \textbf{one God}: Father, Son, and Holy Spirit, \textbf{a unity of three} coeternal \textbf{Persons}. God is immortal, all-powerful… and \textbf{ever present}.}[Fundamental Beliefs of Seventh-day Adventist, \#2 Trinity; 2020 Edition][https://www.adventist.org/wp-content/uploads/2020/06/ADV-28Beliefs2020.pdf]


\others{Kuna \textbf{Mungu mmoja}: Baba, Mwana, na Roho Mtakatifu, \textbf{umoja wa} \textbf{Nafsi} tatu zenye umilele sawa. Mungu ni asiyekufa, mwenye nguvu zote… na \textbf{yupo kila mahali}.}[Fundamental Beliefs of Seventh-day Adventist, \#2 Trinity; 2020 Edition][https://www.adventist.org/wp-content/uploads/2020/06/ADV-28Beliefs2020.pdf]


\section*{Dr. Kellogg's perception of God}


\section*{Mtazamo wa Dk. Kellogg kuhusu Mungu}


In examining the surrounding controversy over the Living Temple, we truly see that Dr. Kellogg raised \others{the theological side of questions of the trinity.}[Kellogg vs. The Brethren: His Last Interview as an Adventist, p. 58.][https://forgotten-pillar.s3.us-east-2.amazonaws.com/1990\_kellogg\_vs\_brethren\_lastInterview\_oct7\_1907\_spectrum\_v20\_n3-4.pdf] Another question we raise in examining Kellogg's sentiments with the \emcap{Fundamental Principles} is whom does he address in terms of “\textit{one God}”? There is no data to directly answer that question, but there is plenty of data which suggests that Dr. Kellogg's understanding of “\textit{one God}” was a Trinitarian understanding. His letter to W. W. Prescott is one piece of evidence supporting that notion:


Katika kuchunguza mzozo uliozunguka Hekalu Hai, tunaona kweli kwamba Dk. Kellogg aliibua \others{upande wa kitheolojia wa maswali ya utatu.}[Kellogg vs. The Brethren: His Last Interview as an Adventist, p. 58.][https://forgotten-pillar.s3.us-east-2.amazonaws.com/1990\_kellogg\_vs\_brethren\_lastInterview\_oct7\_1907\_spectrum\_v20\_n3-4.pdf] Swali lingine tunaloibua katika kuchunguza hisia za Kellogg na \emcap{Kanuni za Msingi} ni nani anayerejelea kwa maneno “\textit{Mungu mmoja}”? Hakuna data ya kujibu swali hilo moja kwa moja, lakini kuna data nyingi zinazoonyesha kwamba uelewa wa Dk. Kellogg wa “\textit{Mungu mmoja}” ulikuwa uelewa wa Utatu. Barua yake kwa W. W. Prescott ni ushahidi mmoja unaoonyesha dhana hiyo:


\others{The difference is this: \textbf{When we say God} is in the tree, the word ‘\textbf{God}’ \textbf{is understood in its most comprehensive sense}, and people understand the meaning to be \textbf{that the Godhead} is in the tree, \textbf{God the Father, God the Son, and God the Holy Spirit}, whereas the proper understanding in order \textbf{that wholesome conceptions} should be preserved in our minds, is that God the Father sits upon his throne in heaven where God the Son is also; \textbf{while God's life, or spirit or presence is the all-pervading power which is carrying out the will of God in all the universe}.}[Letter: Dr. Kellogg to Prof. W. W. Prescott, Oct. 25, 1903][https://forgotten-pillar.s3.us-east-2.amazonaws.com/1903-10-25-JHKellogg-to-W.W.Prescott.pdf]


\others{Tofauti ni hii: \textbf{Tunaposema Mungu} yumo kwenye mti, neno ‘\textbf{Mungu}’ \textbf{linafahamika katika maana yake pana zaidi}, na watu wanaelewa maana kuwa \textbf{Uungu} umo kwenye mti, \textbf{Mungu Baba, Mungu Mwana, na Mungu Roho Mtakatifu}, ambapo uelewa sahihi ili \textbf{dhana nzuri} zihifadhiwe katika akili zetu, ni kwamba Mungu Baba anaketi kwenye kiti chake cha enzi mbinguni ambapo Mungu Mwana pia yupo; \textbf{wakati uhai wa Mungu, au roho au uwepo ni nguvu inayoenea kila mahali ambayo inatekeleza mapenzi ya Mungu katika ulimwengu wote}.}[Letter: Dr. Kellogg to Prof. W. W. Prescott, Oct. 25, 1903][https://forgotten-pillar.s3.us-east-2.amazonaws.com/1903-10-25-JHKellogg-to-W.W.Prescott.pdf]


In the next chapter, we will make our case: if the given \others{wholesome conception} of God advocated by Dr. Kellogg was true, then his clarification of the Holy Spirit being \others{God's life, or spirit or presence is the all-pervading power which is carrying out the will of God in all the universe} would truly solve the entire difficulty of the Living Temple. But that was not the case. Dr. Kellogg's true problem was his perception of God, and his trinitarian stance was not solving the real issue—the \emcap{personality of God}.


Katika sura inayofuata, tutawasilisha hoja yetu: ikiwa \others{dhana nzuri} ya Mungu iliyotetewa na Dk. Kellogg ilikuwa kweli, basi ufafanuzi wake wa Roho Mtakatifu kuwa \others{uhai wa Mungu, au roho au uwepo ni nguvu inayoenea kila mahali ambayo inatekeleza mapenzi ya Mungu katika ulimwengu wote} ungetatua kweli ugumu wote wa Hekalu Hai. Lakini haikuwa hivyo. Tatizo la kweli la Dk. Kellogg lilikuwa mtazamo wake wa Mungu, na msimamo wake wa utatu haukutatua tatizo halisi—\emcap{ubinafsi wa Mungu}.


There is another revealing letter showing us the consequences of raising \others{the theological side of questions of the trinity.} Writing to his friend Dr. Hayward, Dr. Kellogg reflected:


Kuna barua nyingine inayofichua inayotuonyesha matokeo ya kuinua \others{upande wa kitheolojia wa maswali ya utatu.} Akiandika kwa rafiki yake Dk. Hayward, Dk. Kellogg alitafakari:


\others{\textbf{These theologians} have sought to darken the minds of the people and to make \textbf{this sweet and beautiful truth \underline{appear loathsome} to them, by drawing into it \underline{the old controversy about the Trinity}}.}


\others{\textbf{Wanatheolojia hawa} wametafuta kufifisha akili za watu na kufanya \textbf{ukweli huu mtamu na mzuri \underline{uonekane wa kuchukiza} kwao, kwa kuingiza ndani yake \underline{mgogoro wa zamani kuhusu Utatu}}.}


\othersnogap{I never raised the question as to \textbf{which part of God is present in a man}, whether it was \textbf{God, the Father};\textbf{ God, the Son}; or \textbf{God, the Holy Spirit}. The only point was that it is God and not man.}[Letter: Dr. J. H. Kellogg to Dr. Hayward, Aug., 15. 1905][https://forgotten-pillar.s3.us-east-2.amazonaws.com/1903-08-15-kellogg-to-hayward.pdf]


\othersnogap{Sijawahi kuuliza swali kuhusu \textbf{sehemu gani ya Mungu ipo ndani ya mwanadamu}, kama ni \textbf{Mungu, Baba}; \textbf{Mungu, Mwana}; au \textbf{Mungu, Roho Mtakatifu}. Jambo pekee lilikuwa kwamba ni Mungu na sio mwanadamu.}[Letter: Dr. J. H. Kellogg to Dr. Hayward, Aug., 15. 1905][https://forgotten-pillar.s3.us-east-2.amazonaws.com/1903-08-15-kellogg-to-hayward.pdf]


Here we see the tensions between Dr. Kellogg and certain Seventh-day Adventist theologians of that time, where Dr. Kellogg's \others{sweet and beautiful truth} of God's divine immanence got entangled with \others{the old controversy about the Trinity}. This tells us that in the days of Dr. Kellogg, the doctrine of Trinity was controversial, and certainly it was not regarded as something positive, but rather as something which made Kellogg's teachings \others{loathsome}. But who were these theologians Dr. Kellogg referred to? He did not name anyone in his letter to Dr. Hayward, but we can get the idea of whom \others{these theologians} were based on his letter sent 10 days earlier to I. G. Butler\footnote{\href{https://forgotten-pillar.s3.us-east-2.amazonaws.com/1905-08-05-kellogg-butler.pdf}{Letter: J. H. Kellogg to I. G. Butler, Aug., 5. 1905}}, venting his frustration with the General Conference's bidding with him. These were A. G. Daniells, W. C. White, and W. W. Prescott. We can also include G. I. Butler himself to that group, since he also was a theologian participating in this \others{old controversy about the Trinity}. All of these people held leading positions within the Seventh-day Adventist church, and all of them were non-Trinitarians. The argument is being made that the issue with Dr. Kellogg's teaching lies somewhere other than his trinitarian sentiments, because supposedly the church was trinitarian at that time, and supposedly Ellen White was trinitarian herself. \footnote{This is currently the popular narrative promoted by laity.} If this was the case, and in this mix of truth and error, should we not have at least some defense of the trinity doctrine, dissecting it from error? We have not found any such data. Instead, all data we have is in defense of the \emcap{Fundamental Principles}, and the doctrine on the presence and the \emcap{personality of God}, which both are opposed to the doctrine of the Trinity. Ellen White said of the truth: the Trinity doctrine \egwinline{cannot be accepted by those who are \textbf{loyal to the faith and to the principles} that have withstood all the opposition of satanic influences.}[Lt253-1903.28; 1903][https://egwwritings.org/read?panels=p14068.9980036]


Hapa tunaona mvutano kati ya Dk. Kellogg na baadhi ya wanatheolojia wa Waadventista Wasabato wa wakati huo, ambapo \others{ukweli mtamu na mzuri} wa Dk. Kellogg kuhusu uwepo wa kimungu wa Mungu ulisukwa na \others{mgogoro wa zamani kuhusu Utatu}. Hii inatuambia kwamba katika siku za Dk. Kellogg, fundisho la Utatu lilikuwa la utata, na kwa hakika halikuchukuliwa kama kitu chema, bali kama kitu ambacho kilifanya mafundisho ya Kellogg \others{ya kuchukiza}. Lakini ni akina nani hawa wanatheolojia ambao Dk. Kellogg aliwataja? Hakutaja mtu yeyote katika barua yake kwa Dk. Hayward, lakini tunaweza kupata wazo la \others{wanatheolojia hawa} kulingana na barua yake iliyotumwa siku 10 kabla kwa I. G. Butler\footnote{\href{https://forgotten-pillar.s3.us-east-2.amazonaws.com/1905-08-05-kellogg-butler.pdf}{Letter: J. H. Kellogg to I. G. Butler, Aug., 5. 1905}}, akitoa masikitiko yake kuhusu ushindani wa Kongamano Kuu naye. Hawa walikuwa A. G. Daniells, W. C. White, na W. W. Prescott. Tunaweza pia kumjumuisha G. I. Butler mwenyewe katika kundi hilo, kwani yeye pia alikuwa mwanatheolojia aliyeshiriki katika \others{mgogoro wa zamani kuhusu Utatu}. Wote hawa walikuwa na nafasi za uongozi ndani ya kanisa la Waadventista Wasabato, na wote walikuwa wasio-Watatu. Hoja inatolewa kwamba tatizo la mafundisho ya Dk. Kellogg liko mahali pengine kuliko maoni yake ya utatu, kwa sababu inasemekana kanisa lilikuwa la utatu wakati huo, na inasemekana Ellen White alikuwa mwamini utatu mwenyewe. \footnote{Hii kwa sasa ni hadithi maarufu inayoendelezwa na walei.} Kama hii ilikuwa hivyo, na katika mchanganyiko huu wa ukweli na makosa, je, hatupaswi kuwa na angalau utetezi wa fundisho la utatu, kulichambua kutoka kwenye makosa? Hatujapata data yoyote kama hiyo. Badala yake, data zote tulizonazo ni katika utetezi wa \emcap{Kanuni za Msingi}, na fundisho juu ya uwepo na \emcap{Umbile la Mungu}, ambayo yote yanapinga fundisho la Utatu. Ellen White alisema kuhusu ukweli: fundisho la Utatu \egwinline{haliwezi kukubaliwa na wale ambao ni \textbf{waaminifu kwa imani na kwa kanuni} ambazo zimestahimili upinzani wote wa nguvu za kishetani.}[Lt253-1903.28; 1903][https://egwwritings.org/read?panels=p14068.9980036]


In this short reflection on differences between Dr. Kellogg's sentiments and the \emcap{Fundamental Principles} from which he stepped off, we can recognize the following characteristics which are akin to the Trinity doctrine:


Katika tafakari hii fupi juu ya tofauti kati ya maoni ya Dk. Kellogg na \emcap{Kanuni za Msingi} ambayo aliondoka kwayo, tunaweza kutambua sifa zifuatazo ambazo zinafanana na fundisho la Utatu:


\begin{itemize}
    \item The word ‘God’ represents the wholesome conception of God as God the Father, God the Son, and God the Holy Spirit.
    \item God is everywhere present by Himself.
    \item The quality or state of the Father being a person is equalized to that of the Holy Spirit.\footnote{\href{https://www.adventist.org/wp-content/uploads/2020/06/ADV-28Beliefs2020.pdf}{Fundamental Beliefs \#5}: \others{He \normaltext{[the Holy Spirit]} \textbf{is as much a person} \underline{as} are \textbf{the Father} and the Son}; \href{https://www.adventist.org/wp-content/uploads/2020/06/ADV-28Beliefs2020.pdf}{Fundamental Beliefs \#3}: \others{\textbf{The qualities} and powers \textbf{exhibited in} the Son and \textbf{the Holy Spirit are \underline{also} those of the Father}}}
\end{itemize}


\begin{itemize}
    \item Neno ‘Mungu’ linawakilisha dhana kamili ya Mungu kama Mungu Baba, Mungu Mwana, na Mungu Roho Mtakatifu.
    \item Mungu yupo kila mahali kwa Nafsi yake.
    \item Ubora au hali ya Baba kuwa nafsi inasawazishwa na ile ya Roho Mtakatifu.\footnote{\href{https://www.adventist.org/wp-content/uploads/2020/06/ADV-28Beliefs2020.pdf}{Mafundisho za Kimsingi \#5}: \others{Yeye \normaltext{[Roho Mtakatifu]} \textbf{ni nafsi} \underline{kama} walivyo \textbf{Baba} na Mwana}; \href{https://www.adventist.org/wp-content/uploads/2020/06/ADV-28Beliefs2020.pdf}{Mafundisho za Kimsingi \#3}: \others{\textbf{Sifa} na nguvu \textbf{zinazoonyeshwa katika} Mwana na \textbf{Roho Mtakatifu ni \underline{pia} zile za Baba}}}
\end{itemize}


These three characteristics of Dr. Kellogg's sentiments depart from the foundation of our faith—the \emcap{Fundamental Principles}—but are in harmony with the teachings of the Trinity. In saying this, we are not claiming that Dr. Kellogg is responsible for the acceptance of the Trinity doctrine into our ranks, but rather that the Trinity doctrine was Kellogg's justification for stepping off from the foundation of our faith, established at the beginning of our work. The true problem was \textit{stepping off} from the \emcap{fundamental principles}, and both Dr. Kellogg and we as a church have made those steps. The difference is that Dr. Kellogg landed in pantheism, while we landed on the \#2 point of the Fundamental Beliefs.


Sifa hizi tatu za maoni ya Dk. Kellogg zinaondoka kutoka kwenye msingi wa imani yetu—\emcap{Kanuni za Msingi}—lakini zinapatana na mafundisho ya Utatu. Kwa kusema hivi, hatudai kwamba Dk. Kellogg anawajibika kwa kukubaliwa kwa fundisho la Utatu katika safu zetu, bali kwamba fundisho la Utatu lilikuwa haki ya Kellogg ya kuondoka kutoka kwenye msingi wa imani yetu, ulioanzishwa mwanzoni mwa kazi yetu. Tatizo la kweli lilikuwa \textit{kuondoka} kutoka kwenye \emcap{kanuni za msingi}, na wote Dk. Kellogg na sisi kama kanisa tumefanya hatua hizo. Tofauti ni kwamba Dk. Kellogg aliishia kwenye pantheism, wakati sisi tuliishia kwenye nukta \#2 ya Mafundisho za Kimsingi.


In the following chapter, we will examine Dr. Kellogg's teaching that God sustains all life, and how this truth, in combination with a false perception of God and His personality, led him to become a pantheist.


Katika sura ifuatayo, tutachunguza mafundisho ya Dk. Kellogg kwamba Mungu hutegemeza uhai wote, na jinsi ukweli huu, ukichanganywa na mtazamo wa uwongo wa Mungu na Umbile lake, ulimfanya awe mpantheisti.


% Dr. Kellogg and the Trinity doctrine

\begin{titledpoem}
    
    \stanza{
        In Kellogg’s quest, the question posed, \\
        "The Spirit – how is He composed?" \\
        The issue stirred a great debate, \\
        How does this mystery relate?
    }

    \stanza{
        The question was beyond the seen \\
        To trinity J.H. did lean \\
        The Father wasn’t bound by space? \\
        Without a body or a face?
    }

    \stanza{
        To Ellen, Jesus did inform \\
        Like Him, His Father had a form \\
        “I am His image as express, \\
        Revealing form and righteousness.”
    }

    \stanza{
        In vision was the truth revealed \\
        The inspiration, it was sealed \\
        The Father’s form upon the throne \\
        And Christ with form just like His own.
    }

    \stanza{
        The Spirit’s personality \\
        In actions and in quality \\
        A role distinct, within us dwells. \\
        The mind of Christ the Spirit tells.
    }

    \stanza{
        God’s power and His presence show \\
        Wherever God would have it go \\
        And thus, He’s present everywhere \\
        Invisible, His Spirit there.
    }

    \stanza{
        The Living Temple showed a flaw \\
        A dangerous error Ellen saw \\
        The wayward theories in his mind \\
        Blocked him from truth he could not find.
    }

    \stanza{
        He went off searching on his own \\
        And did not follow what was shown \\
        If he had stayed where God had led, \\
        His teaching would have never spread.
    }
    
\end{titledpoem}


% Dr. Kellogg and the Trinity doctrine

\begin{titledpoem}
    
    \stanza{
        In Kellogg’s quest, the question posed, \\
        "The Spirit – how is He composed?" \\
        The issue stirred a great debate, \\
        How does this mystery relate?
    }

    \stanza{
        The question was beyond the seen \\
        To trinity J.H. did lean \\
        The Father wasn’t bound by space? \\
        Without a body or a face?
    }

    \stanza{
        To Ellen, Jesus did inform \\
        Like Him, His Father had a form \\
        “I am His image as express, \\
        Revealing form and righteousness.”
    }

    \stanza{
        In vision was the truth revealed \\
        The inspiration, it was sealed \\
        The Father’s form upon the throne \\
        And Christ with form just like His own.
    }

    \stanza{
        The Spirit’s personality \\
        In actions and in quality \\
        A role distinct, within us dwells. \\
        The mind of Christ the Spirit tells.
    }

    \stanza{
        God’s power and His presence show \\
        Wherever God would have it go \\
        And thus, He’s present everywhere \\
        Invisible, His Spirit there.
    }

    \stanza{
        The Living Temple showed a flaw \\
        A dangerous error Ellen saw \\
        The wayward theories in his mind \\
        Blocked him from truth he could not find.
    }

    \stanza{
        He went off searching on his own \\
        And did not follow what was shown \\
        If he had stayed where God had led, \\
        His teaching would have never spread.
    }
    
\end{titledpoem}
