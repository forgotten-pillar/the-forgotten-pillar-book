
\qrchapterstar{https://forgottenpillar.com/rsc/en-fp-introduction}{Utangulizi}

\addcontentsline{toc}{chapter}{Utangulizi}

Kitabu hiki kina malengo matatu ya kutimiza. La kwanza ni kufufua nguzo ya zamani ya imani yetu inayoitwa, “\textit{Umbile la Mungu}”. Lengo la pili ni kurejesha imani katika maandishi ya Ellen White, na la tatu ni kuanzisha upya utambulisho wa awali wa Waadventista.

Kabla ya Oktoba 22, 1844, kulikuwa na idadi kubwa ya Waadventista waliokuwa wakimngoja Kristo arudi na mawingu ya mbinguni. Ulikuwa ni vuguvugu la kimataifa la watu waliokuwa wakingoja ujio wake wa pili. Oktoba 22 ilipita bila Kristo kushuka na mawingu ya mbinguni na wengi waliondoka katika harakati, wakaidharau, wakadharau unabii, Biblia, na Mungu. Waaminifu wachache sana, wanyenyekevu, wanaume kwa wanawake walibaki, ambao walikuwa na uhakika kuwa Mungu alikuwa akiongoza harakati hiyo. Walijua kwamba Mungu alikuwa anaangaza nuru ya Kweli na mioyo yao ilikuwa na hamu ya kuipokea. Lakini machoni pa ulimwengu, walionekana kuwa washupavu na waotaji. Hali hii kubwa ya kuvunjika moyo inaweza kulinganishwa na ile ya wanafunzi wa Yesu walipomuona Bwana wao amelazwa kaburini. Walikuwa na uhakika kabisa kwamba Kristo “\textit{alikuwa nabii mwenye nguvu katika matendo na maneno mbele za Mungu na watu wote}”, lakini alipokufa msalabani, walikatishwa tamaa sana, kwa sababu “\textit{waliamini kwamba alikuwa ndiye atakayeikomboa Israeli}.” Hata hivyo katika hali yao ya kukata tamaa, katika hali yao ya kujisikitikia, walikuwa tayari kupokea uwezo wa kuushinda ulimwengu wote kupitia kwa Injili. Walikutana na Kristo na baadaye wakapokea Roho wake. Vile vile ilifanyika kwa Waanzilishi wa Kiadventista. Walikuwa kikundi kidogo cha watu, waliokatishwa tamaa sana; walitafuta Bwana kwa mioyo yao yote na wakampokea katika uweza na Kweli. Kweli za thamani ambazo Mungu alifunua katika wakati huo wa shida ndio msingi wa imani ya Waadventista Wasabato. Kweli hizi zilijaribiwa na nadharia zote za kudanganya, za ulimwengu, za wale waliokidharau kikundi hiki kidogo, lakini kweli hizi kuu zilitawala. Wakati wa mahitaji makubwa, Yesu alitoa ushuhuda Wake kwa kumkuza msichana mdogo, aliyekuwa dhaifu zaidi kati ya wanyonge, ili kuwaidhinisha wote kwa Ukweli wake. Ellen White hakupaswa kuwa chanzo cha ukweli; badala yake, kuwasaidia ndugu ambao walikuwa wakitafuta ukweli katika Biblia. Mungu alimtumia Ellen White kuidhinisha masomo yao na kuwaelekeza kwenye Biblia. Matokeo ya mwisho yalikuwa kuanzishwa kwa msingi wa imani kwa msingi wa Biblia, ambao unasimama imara mpaka mwisho wa dunia.

Je, unaweza kushangaa kujua kwamba msingi wa imani ya Waadventista Wasabato, ambayo iliwekwa mwanzoni mwa kazi yetu, ni kwa kiasi fulani tofauti na ilivyo sasa? Leo, zaidi ya karne moja na nusu baadaye, tunastaajabishwa na masimulizi ya uzoefu wa waanzilishi wetu; lakini tangu wakati huo, Kanisa la Waadventista Wasabato limekuwa chini ya harakati kadhaa mpya. Tangu wakati huo, kanisa limepata mabadiliko mengi, ikijumuisha mabadiliko katika mafundisho yetu. Wengine wanasema kuwa mabadiliko haya ni mazuri na ya maendeleo; wengine wanabisha na kusema kuwa ni haribifu na danganyifu. Kuhamisha uangalizi kwa Uadventista wa awali, unaanzisha pambano kuu katika siku hizi. Sisi binafsi tumekuwa kwenye utata huu kwa zaidi ya miaka 6 sasa na tumeona kuwa utata huu utazidi kuwa mkubwa na wenye nguvu zaidi, mara nyingi matokeo yake yatakuwa ya kutisha. Watu wengi kutoka pande zote mbili za pambano hili wanakataa Roho ya Unabii kwa njia moja au nyingine. Wengine wameondoka kutoka kwa Kanisa la Waadventista Wasabato. Utambulisho wa Waadventista umepotea au umebadilika sana kutoka ule wa awali.

Kwa sasa tunashuhudia mtikisiko wa kanisa la Waadventista Wasabato, tunamuona akitupwa kutoka wimbi moja la mgogoro hadi jingine. Wengi wanapoteza imani yao na utambulisho wao kama Waadventista Wasabato. Lakini tunaamini katika suluhisho ambalo Bwana, kwa rehema zake, ametutolea. Suluhisho linaweza kupatikana katika historia ya harakati ya Waadventista Wasabato.

\egw{\textbf{Katika kukagua historia yetu ya zamani}, baada ya kusafiri kila hatua ya maendeleo yetu hadi tusimamapo leo, naweza kusema, Mungu asifiwe! Ninapoona kile ambacho Bwana amefanya, ninajazwa mshangao, na vilevile uhakika katika Kristo kama kiongozi. \textbf{Hatuna cha kuogopa kwa ajili ya yajayo, \underline{isipokuwa tutakaposahau} njia ambayo Bwana ametuongoza, na \underline{mafundisho Yake} katika historia yetu iliyopita}.}[LS 196.2; 1915][https://egwwritings.org/read?panels=p41.1083]

Hatutaogopa! Hii ni hakikisho kubwa na ahadi—ingawa ina masharti. Ni lazima \textit{kukumbuka} jinsi Bwana ametuongoza, na \textit{mafundisho Yake katika historia yetu iliyopita}. Tunapoangalia kwa yale ambayo Bwana ametufundisha katika historia yetu iliyopita, tunashangaa kuona jinsi mambo yalivyobadilika. Mabadiliko yamechukua miaka kadhaa na migogoro mingi. Ili kutua maamuzi kwa mabadiliko haya katika mafundisho, ikiwa ni mazuri na ya maendeleo au mabaya na yenye uharibifu, tathmini inapaswa kujengwa juu ya uzoefu uliopita, kwa kuwa Bwana aliliongoza kanisa lake waziwazi.

Kwa wakati huu, tunatoa dai la kijasiri—ambalo linafaa kukufanya ushikilie kitabu hiki hadi mwisho wake. Kwa kuhimizwa na mawaidha ya Ellen White kuwa tuangazie historia yetu ya zamani, sisi tumehitimisha kwamba tumesahau nguzo moja muhimu ya imani yetu, ambayo ilikuwa somo kuu la mgogoro wa Kellogg—\emcap{Umbile la Mungu}. Moja ya migogoro mikubwa ambayo Kanisa la SDA liliwahi kuwa nalo wakati wa nabii aliyekuwa hai lilikuwa ni shida ya Kellogg. Ni kutoka kwenye mgogoro huu ambapo migogoro mingine mingi, leo, inapata mizizi yake. Katika mwanga huu, somo la \emcap{Umbile la Mungu} ni muhimu katika wakati wetu huu.

Dada White alimwandikia Kellogg kwamba \emcap{Umbile la Mungu} na \emcap{Umbile la Kristo} ilikuwa nguzo ya imani yetu iliyokuwa daraja sawa na ujumbe wa patakatifu:

\egw{Wale wanaotafuta kuondoa \textbf{alama za zamani} hawashikilii kwa dhati; \textbf{hawaweki \underline{pembeni mwa fikira zao} jinsi walivyopokea na kusikia}. Wale wanaojaribu \textbf{\underline{kuleta} nadharia ambazo zingeondoa \underline{nguzo za imani yetu} kuhusu patakatifu, \underline{au habari ya Umbile la Mungu au la Kristo}, wanafanya kazi kama vipofu}. Wanatafuta kuingiza mashaka na kuwafanya watu wa Mungu watikiswe bila nanga.}[Ms62-1905.14][https://egwwritings.org/read?panels=p14070.10026020]

\emcap{Umbile la Mungu} linapokea usikivu mdogo sana leo kama somo, lakini ni mojawapo ya mambo muhimu katika kushughulikia mafundisho mengine ya Uadventista, kama vile fundisho la Utatu, huduma ya Patakatifu, 1844 na fundisho lingine lolote linalohusu mambo ya Mbinguni.

\emcap{Umbile la Mungu} lilikuwa nguzo ya imani yetu. Leo, inakaribia kusahaulika. Tunapendekeza ufafanuzi wa busara kwa hilo. Ni kutokana na mageuzi ya lugha ya Kiingereza. Nini kinachomaanishwa na maneno, “\textit{Umbile la Mungu}”? Uelewa wa jumla wa neno la Kiingereza ‘\textit{personality}’ limebadilika kwa miaka mingi. Leo, ‘\textit{personality}’ kwa ujumla huonekana kama, “\textit{seti ya tabia, utambuzi, na mifumo ya kihisia}”\footnote{Wikipedia Contributors. “\textit{Personality.}” Wikipedia, Wikimedia Foundation, 19 Apr. 2019, \href{https://en.wikipedia.org/wiki/Personality}{en.wikipedia.org/wiki/Personality}.}, lakini katika karne ya kumi na tisa, na mwanzo wa karne ya ishirini, ilimaanisha “\textit{ubora au hali ya \textbf{kuwa Nafsi}}”\footnote{\href{https://www.merriam-webster.com/dictionary/personality}{Merriam-Webster Dictionary}, - ‘personality’} \footnote{\href{https://babel.hathitrust.org/cgi/pt?id=mdp.39015050663213&view=1up&seq=780}{Hunter Robert, The American encyclopaedic dictionary}, ‘\textit{personality}’ - “\textit{the quality or state of being personal}”; Kamusi iliyotajwa ilikuwa miliki ya Ellen White (angalia \href{https://repo.adventistdigitallibrary.org/PDFs/adl-22/adl-22251050.pdf?_ga=2.116010630.1065317374.1621993520-1506151612.1617862694&fbclid=IwAR3vwmp8jxtnpPEKv0KD9mCv8dJpmRGoyIXW0CkbQAjbU0h6YaBGqhgBzbk}{EGW Private and Office Libraries})}. Tunasoma ufafanuzi huu kama ufafanuzi wa msingi wa neno ‘\textit{personality}’ kutoka kwa Kamusi ya Merriam-Webster\footnote{\href{https://www.merriam-webster.com/dictionary/personality\#word-history}{Merriam-Webster Dictionary} inaonyesha kuwa rekodi ya kwanza ya ufafanuzi “the quality or state of being a person” iliandikwa katika karne ya 15.}. Wakati Dada White na waanzilishi wetu waliandika kuhusu \emcap{Umbile la Mungu}, walirejelea \textit{ubora au hali ya Mungu kuwa Nafsi}. Kwa maneno mengine, walishughulikia swali, “\textit{je Mungu ni Nafsi}”, na, “\textit{ni nini kinachomfanya awe Nafsi}” au “\textit{ubora au hali ya Mungu kuwa Nafsi ni ipi}”? Jaribu kukumbuka mara ya mwisho ulipokuwa na mafunzo ya Biblia juu ya swali, “\textit{je, Mungu ni Nafsi?}” Fikiria jinsi unavyoweza kuthibitisha kwako mwenyewe, kutoka Biblia, kwamba Mungu ni Nafsi. Fikiria juu ya hilo. Ni swali muhimu. Juu ya swali hili unaegemeza mtazamo wako wa Mungu na uhusiano wako kwake. \emcap{Umbile la Mungu} ni msingi wa hali ya kweli ya kiroho; hali ya kweli ya kiroho inatokana na uhusiano wako binafsi na Mungu. Hakuna uhusiano wa kweli wa aina yoyote unaweza kuundwa na yeyote asiye Nafsi. Labda hujawahi kujiuliza swali hili kwa sababu hujawahi kuhisi haja ya kuuliza kama Mungu ni Nafsi, na ni nini (ubora au hali) kinachomfanya kuwa Nafsi. Au, labda ulijiepusha na swali hili kwa sababu ulihisi kuwa ni fumbo ambalo Mungu hakukusudia kulifunua. Labda itakushangaza kujua kuwa Mungu ametoa jibu la uhakika na thabiti katika Neno Lake kwa swali “\textit{ni nini ubora au hali ya Mungu kuwa Nafsi}”. Kilichotushangaza zaidi, ni kwamba Waanzilishi wa Kiadventista, akiwemo Dada White, walikuwa na mwanga wa uhakika kuhusu mada hii, na waliishikilia kama nguzo ya imani yetu, kama sehemu ya msingi wa imani ya Waadventista Wasabato. Wakati \emcap{Umbile la Mungu} linapofahamika vizuri katika mwanga wa historia yetu ya zamani, nukuu za zamani zitang'aa katika mwanga mpya na ushahidi mpya utatokea, ambao utaongeza uelewa wa historia yetu ya zamani na mgogoro wa sasa.

Shida kuu ya mgogoro wa Kellogg ilikuwa juu ya \emcap{Umbile la Mungu}. Ni muhimu kutathmini mgogoro wa Kellogg juu ya \emcap{Umbile la Mungu} kwa kutumia maana iliyokusudiwa wakati huo; yaani, kutumia fasili ya ‘\textit{personality},’ kama ubora au hali ya Mungu kuwa Nafsi. Kuweka ufafanuzi huu akilini, mgogoro wa Kellogg unakuja katika mwanga mpya na ushahidi mpya unaofaa unaletwa kwa ajili yetu leo. Kwa kuzingatia uthibitisho huu, tunaona jinsi Mungu alivyotuongoza zamani; hivyo, hatupaswi kuogopea wakati ujao. Kujua na kuelewa hili, pamoja na umuhimu wake, hutusaidia kutotikiswa na wimbi lolote la udanganyifu katika migogoro ya sasa. Wakati Dada White alikuwa akivuta mawazo ya Kellogg kwa umuhimu wa somo hili, alivuta usikivu wetu pia, kwani ni kila kitu kwetu kama watu.

[Akimwandikia Kellogg] \egw{Kwa hakika huelewi \textbf{Umbile la Mungu}, ambao ni \textbf{\underline{kila kitu} kwetu kama watu}.}[Lt300-1903.7][https://egwwritings.org/read?panels=p14068.7705013]

Masomo haya kuhusu \emcap{Umbile la Mungu} yatazua maswali mengi mapya na magumu. Hatuahidi kuyajibu yote, na labda hutaridhika na majibu yanayotolewa, lakini tunaomba, tunatumaini na kuamini kwamba kitabu hiki kitatimiza malengo matatu yaliyopendekezwa mwanzoni mwa utangulizi huu. Kupitia kuhuisha fundisho la \emcap{Umbile la Mungu}, tunaamini kwamba imani yako katika Roho ya Unabii itaimarishwa, na kwamba utajikita zaidi katika ujumbe wa Waadventista—ambapo tunapata utambulisho wetu kama watu—hivyo kukufanya Mwadventista Msabato mwaminifu zaidi. Muhimu zaidi, tunataka umfahamu Mungu zaidi kama Mungu wako binafsi. Hii hakika itaimarisha na kuongeza uhusiano wako na Yeye.

Tunapata majibu kwa suala la \emcap{Umbile la Mungu} katika kuchunguza mgogoro wa Kellogg, ambapo Dada White alitoa mwanga wa uhakika zaidi juu ya \emcap{Umbile la Mungu} na juu ya msingi wa imani ya Waadventista Wasabato. Ifuatayo ni sura kamili ya kumi kutoka kitabu, \textit{Testimonies for the Church containing Letters to Physicians and Ministers Instruction to Seventh-Day Adventist}. Sura hii, \textit{Msingi wa Imani yetu}, ina ufahamu wa kina katika historia ya mgogoro wa Kellogg. Inatoa muhtasari wa kihistoria wa kweli ambazo Mungu alitoa kama msingi wa imani yetu na katika kweli hizi tunapata utambulisho wetu kama Waadventista—kushika amri za Mungu na kuwa na imani ya Yesu.
