\qrchapterstar{https://forgottenpillar.com/rsc/en-fp-introduction}{Introduction}


\qrchapterstar{https://forgottenpillar.com/rsc/en-fp-introduction}{Utangulizi}


\addcontentsline{toc}{chapter}{Introduction}


\addcontentsline{toc}{chapter}{Utangulizi}


This book has three objectives to fulfill. The first one is to revive the old pillar of our faith called, “\textit{the personality of God}”. The second objective is to re-establish trust in the writings of Ellen White, and the third is to re-establish the original Adventist identity.


Kitabu hiki kina malengo matatu ya kutimiza. La kwanza ni kufufua nguzo ya zamani ya imani yetu inayoitwa, “\textit{Umbile la Mungu}”. Lengo la pili ni kurejesha imani katika maandishi ya Ellen White, na la tatu ni kuanzisha upya utambulisho wa awali wa Waadventista.


Prior to October 22, 1844, there was a great number of Adventists waiting for Christ to return on the clouds of heaven. It was a global movement of people awaiting His second coming. October 22 passed without Christ descending on the clouds and the great majority left the movement, scorning it, scorning the prophecies, the Bible, and God. Very few faithful, humble, men and women remained, who were unquestionably sure that God was leading this movement. They knew that God was shining the light of Truth and their hearts were eager to receive it. But in the eyes of the world, they were just demonstrated fanatics and dreamers. This great disappointment can be compared to the one Jesus’ disciples had after they saw their Lord being laid in the grave. They were unquestionably sure that Christ “\textit{was a prophet mighty in deed and word before God and all the people}”, but as He died on the cross, they were bitterly disappointed, because they “\textit{trusted that it had been He which should have redeemed Israel}.” Yet in their state of despair, in their state of self-disappointment, they were ready to receive the power to conquer the whole world with the Gospel. They met Christ and later received His Spirit. The same happened with the Adventist pioneers. They were a small group of people, bitterly disappointed; they sought the Lord with all their hearts and received Him in power and in Truth. The truths God revealed during this precious time of crisis constitute the foundation of Seventh-day Adventist faith. These truths were tested by all the seductive, deceptive theories of the world, by those scorning this small group, yet these grand truths prevailed. In the time of greatest need, Jesus gave His testimony by raising a little girl, the weakest of the weak, to approve all of His truths. Ellen White was not to be the source of the truths; rather, to support the brethren who were seeking the truth in the Bible. God used Ellen White to approve their studies and to point them to the Bible. The final result was the establishment of the foundation of faith based on the Bible, which standeth sure till the end of the world.


Kabla ya Oktoba 22, 1844, kulikuwa na idadi kubwa ya Waadventista waliokuwa wakimngoja Kristo arudi na mawingu ya mbinguni. Ulikuwa ni vuguvugu la kimataifa la watu waliokuwa wakingoja ujio wake wa pili. Oktoba 22 ilipita bila Kristo kushuka na mawingu ya mbinguni na wengi waliondoka katika harakati, wakaidharau, wakadharau unabii, Biblia, na Mungu. Waaminifu wachache sana, wanyenyekevu, wanaume kwa wanawake walibaki, ambao walikuwa na uhakika kuwa Mungu alikuwa akiongoza harakati hiyo. Walijua kwamba Mungu alikuwa anaangaza nuru ya Kweli na mioyo yao ilikuwa na hamu ya kuipokea. Lakini machoni pa ulimwengu, walionekana kuwa washupavu na waotaji. Hali hii kubwa ya kuvunjika moyo inaweza kulinganishwa na ile ya wanafunzi wa Yesu walipomuona Bwana wao amelazwa kaburini. Walikuwa na uhakika kabisa kwamba Kristo “\textit{alikuwa nabii mwenye nguvu katika matendo na maneno mbele za Mungu na watu wote}”, lakini alipokufa msalabani, walikatishwa tamaa sana, kwa sababu “\textit{waliamini kwamba alikuwa ndiye atakayeikomboa Israeli}.” Hata hivyo katika hali yao ya kukata tamaa, katika hali yao ya kujisikitikia, walikuwa tayari kupokea uwezo wa kuushinda ulimwengu wote kupitia kwa Injili. Walikutana na Kristo na baadaye wakapokea Roho wake. Vile vile ilifanyika kwa Waanzilishi wa Kiadventista. Walikuwa kikundi kidogo cha watu, waliokatishwa tamaa sana; walitafuta Bwana kwa mioyo yao yote na wakampokea katika uweza na Kweli. Kweli za thamani ambazo Mungu alifunua katika wakati huo wa shida ndio msingi wa imani ya Waadventista Wasabato. Kweli hizi zilijaribiwa na nadharia zote za kudanganya, za ulimwengu, za wale waliokidharau kikundi hiki kidogo, lakini kweli hizi kuu zilitawala. Wakati wa mahitaji makubwa, Yesu alitoa ushuhuda Wake kwa kumkuza msichana mdogo, aliyekuwa dhaifu zaidi kati ya wanyonge, ili kuwaidhinisha wote kwa Ukweli wake. Ellen White hakupaswa kuwa chanzo cha ukweli; badala yake, kuwasaidia ndugu ambao walikuwa wakitafuta ukweli katika Biblia. Mungu alimtumia Ellen White kuidhinisha masomo yao na kuwaelekeza kwenye Biblia. Matokeo ya mwisho yalikuwa kuanzishwa kwa msingi wa imani kwa msingi wa Biblia, ambao unasimama imara mpaka mwisho wa dunia.


Would you be surprised to know that the foundation of Seventh-day Adventist faith, which was laid at the beginning of our work, is in a fair degree different from what it is currently? Today, more than a century and a half later, we marvel in amazement over the accounts of the experiences of our pioneers; but since then, the Seventh-day Adventist Church has been subject to several new movements. Since then, the church has experienced many changes, including changes in our doctrine. Some argue that these changes are good and progressive; others argue that they are destructive and deceptive. Moving the spotlight to the original Seventh-day Adventism, it initiates the great controversy in the present days. We have personally been in this controversy for over 6 years now and we have seen that it will only get bigger and stronger, often with results of a sad record. Many people from both sides of this controversy are rejecting the Spirit of Prophecy in one way or another. Some have left the Seventh-day Adventist Church altogether. The Adventist identity is either lost or drastically changed from the initial one.


Je, unaweza kushangaa kujua kwamba msingi wa imani ya Waadventista Wasabato, ambayo iliwekwa mwanzoni mwa kazi yetu, ni kwa kiasi fulani tofauti na ilivyo sasa? Leo, zaidi ya karne moja na nusu baadaye, tunastaajabishwa na masimulizi ya uzoefu wa waanzilishi wetu; lakini tangu wakati huo, Kanisa la Waadventista Wasabato limekuwa chini ya harakati kadhaa mpya. Tangu wakati huo, kanisa limepata mabadiliko mengi, ikijumuisha mabadiliko katika mafundisho yetu. Wengine wanasema kuwa mabadiliko haya ni mazuri na ya maendeleo; wengine wanabisha na kusema kuwa ni haribifu na danganyifu. Kuhamisha uangalizi kwa Uadventista wa awali, unaanzisha pambano kuu katika siku hizi. Sisi binafsi tumekuwa kwenye utata huu kwa zaidi ya miaka 6 sasa na tumeona kuwa utata huu utazidi kuwa mkubwa na wenye nguvu zaidi, mara nyingi matokeo yake yatakuwa ya kutisha. Watu wengi kutoka pande zote mbili za pambano hili wanakataa Roho ya Unabii kwa njia moja au nyingine. Wengine wameondoka kutoka kwa Kanisa la Waadventista Wasabato. Utambulisho wa Waadventista umepotea au umebadilika sana kutoka ule wa awali.


We are currently witnessing the shaking of the Seventh-day Adventist church, seeing her tossed through one wave of crisis after another. Many are losing their faith and their identity as Seventh-day Adventists. But we believe in a solution that the Lord, in His mercy, has already provided. The solution can be found in the history of the Seventh-day Adventist movement.


Kwa sasa tunashuhudia mtikisiko wa kanisa la Waadventista Wasabato, tunamuona akitupwa kutoka wimbi moja la mgogoro hadi jingine. Wengi wanapoteza imani yao na utambulisho wao kama Waadventista Wasabato. Lakini tunaamini katika suluhisho ambalo Bwana, kwa rehema zake, ametutolea. Suluhisho linaloweza kupatikana katika historia ya harakati ya Waadventista Wasabato.


\egw{\textbf{In reviewing our past history}, having traveled over every step of advance to our present standing, I can say, Praise God! As I see what the Lord has wrought, I am filled with astonishment, and with confidence in Christ as leader. \textbf{We have nothing to fear for the future, \underline{except as we shall forget} the way the Lord has led us, and \underline{His teaching} in our past history}.}[LS 196.2; 1915][https://egwwritings.org/read?panels=p41.1083]


\egw{\textbf{Katika kukagua historia yetu ya zamani}, baada ya kusafiri kila hatua ya maendeleo yetu hadi tusimamapo leo, naweza kusema, Mungu asifiwe! Ninapoona kile ambacho Bwana amefanya, ninajazwa mshangao, na vilevile uhakika katika Kristo kama kiongozi. \textbf{Hatuna cha kuogopa kwa ajili ya yajayo, \underline{isipokuwa tutakaposahau} njia ambayo Bwana ametuongoza, na \underline{mafundisho Yake} katika historia yetu iliyopita}.}[LS 196.2; 1915][https://egwwritings.org/read?panels=p41.1083]


We shall not fear! This is a great reassurance and promise—though conditional. We must \textit{remember} how the Lord has led us, and \textit{His teaching in our past history}. When we look at what the Lord has taught us in our past history, we are surprised to see how things have changed. The change has taken several years and many crises. To judge these changes in doctrine, whether good and progressive or bad and destructive, evaluation should be based on past experiences, as the Lord clearly led His church.


Hatutaogopa! Hii ni hakikisho kubwa na ahadi—ingawa ina masharti. Ni lazima \textit{kukumbuka} jinsi Bwana ametuongoza, na \textit{mafundisho Yake katika historia yetu iliyopita}. Tunapoangalia kwa yale ambayo Bwana ametufundisha katika historia yetu iliyopita, tunashangaa kuona jinsi mambo yalivyobadilika. Mabadiliko yamechukua miaka kadhaa na migogoro mingi. Ili kutua maamuzi kwa mabadiliko haya katika mafundisho, ikiwa ni mazuri na ya maendeleo au mabaya na yenye uharibifu, tathmini inapaswa kujengwa juu ya uzoefu uliopita, kwa kuwa Bwana aliliongoza kanisa lake waziwazi.


At this time, we put forth a bold claim—one that is supposed to make you hold this book until the end of its cover. Encouraged by the counsels of Ellen White to review our past history, we have concluded that we have forgotten one crucial pillar of our faith, which was the main subject of Kellogg’s controversy—the \emcap{personality of God}. One of the biggest crises that the SDA Church ever had in the time of the living prophet was the Kellogg crisis. It is out of this crisis that many other crises, today, find their roots. In this light, the subject of the \emcap{personality of God} is pivotal in our present time.


Kwa wakati huu, tunatoa dai la kijasiri—ambalo linafaa kukufanya ushikilie kitabu hiki hadi mwisho wake. Kwa kuhimizwa na mawaidha ya Ellen White kuwa tuangazie historia yetu ya zamani, sisi tumehitimisha kwamba tumesahau nguzo moja muhimu ya imani yetu, ambayo ilikuwa somo kuu la mgogoro wa Kellogg—\emcap{Umbile la Mungu}. Moja ya migogoro mikubwa ambayo Kanisa la SDA liliwahi kuwa nalo wakati wa nabii aliyekuwa hai lilikuwa ni shida ya Kellogg. Ni kutoka kwenye mgogoro huu ambapo migogoro mingine mingi, leo, inapata mizizi yake. Katika mwanga huu, somo la \emcap{Umbile la Mungu} ni muhimu katika wakati wetu huu.


Sister White wrote to Kellogg that the \emcap{personality of God} and the \emcap{personality of Christ} was a pillar of our faith in the same rank as is the sanctuary message:


Dada White alimwandikia Kellogg kwamba \emcap{Umbile la Mungu} na \emcap{Umbile la Kristo} ilikuwa nguzo ya imani yetu iliyokuwa daraja sawa na ujumbe wa patakatifu:


\egw{Those who seek to remove \textbf{the old landmarks} are not holding fast; they \textbf{are \underline{not remembering} how they have received and heard}. Those who try to \textbf{\underline{bring in} theories that would remove \underline{the pillars of our faith} concerning the sanctuary, \underline{or concerning the personality of God or of Christ}, are working as blind men}. They are seeking to bring in uncertainties and to set the people of God adrift, without an anchor.}[Ms62-1905.14][https://egwwritings.org/read?panels=p14070.10026020]


\egw{Wale wanaotafuta kuondoa \textbf{alama za zamani} hawashikilii kwa dhati; \textbf{hawaweki \underline{pembeni mwa fikira zao} jinsi walivyopokea na kusikia}. Wale wanaojaribu \textbf{\underline{kuleta} nadharia ambazo zingeondoa \underline{nguzo za imani yetu} kuhusu patakatifu, \underline{au habari ya Umbile la Mungu au la Kristo}, wanafanya kazi kama vipofu}. Wanatafuta kuingiza mashaka na kuwafanya watu wa Mungu watikiswe bila nanga.}[Ms62-1905.14][https://egwwritings.org/read?panels=p14070.10026020]


The \emcap{personality of God} receives very little attention today as a subject, yet it is one of the crucial elements in dealing with other doctrines pertaining to Adventism, such as the doctrine of Trinity, the Sanctuary service, 1844 and any other doctrine dealing with the Heavenly reality.


\emcap{Umbile la Mungu} linapokea usikivu mdogo sana leo kama somo, lakini ni mojawapo ya mambo muhimu katika kushughulikia mafundisho mengine ya Uadventista, kama vile fundisho la Utatu, huduma ya Patakatifu, 1844 na fundisho lingine lolote linalohusu mambo ya Mbinguni.


The \emcap{personality of God} was a pillar of our faith. Today, it is almost forgotten. We propose a reasonable explanation for that. It is due to the evolution of the English language. What is meant by the term, “\textit{the personality of God}”? The general understanding of the English word ‘\textit{personality}’ has changed over the years. Today, ‘\textit{personality}’ is generally viewed as, “\textit{the characteristic set of behaviors, cognitions, and emotional patterns}”\footnote{Wikipedia Contributors. “\textit{Personality.}” Wikipedia, Wikimedia Foundation, 19 Apr. 2019, \href{https://en.wikipedia.org/wiki/Personality}{en.wikipedia.org/wiki/Personality}.}, but in the nineteenth, and beginning of the twentieth century, it meant “\textit{the quality or state of \textbf{being a person}}”\footnote{\href{https://www.merriam-webster.com/dictionary/personality}{Merriam-Webster Dictionary}, - ‘personality’} \footnote{\href{https://babel.hathitrust.org/cgi/pt?id=mdp.39015050663213&view=1up&seq=780}{Hunter Robert, The American encyclopaedic dictionary}, ‘\textit{personality}’ - “\textit{the quality or state of being personal}”; Mentioned dictionary was in possession of Ellen White (see \href{https://repo.adventistdigitallibrary.org/PDFs/adl-22/adl-22251050.pdf?_ga=2.116010630.1065317374.1621993520-1506151612.1617862694&fbclid=IwAR3vwmp8jxtnpPEKv0KD9mCv8dJpmRGoyIXW0CkbQAjbU0h6YaBGqhgBzbk}{EGW Private and Office Libraries})}. We read this definition as the primary definition of the word ‘\textit{personality}’ from the Merriam-Webster Dictionary\footnote{\href{https://www.merriam-webster.com/dictionary/personality\#word-history}{Merriam-Webster Dictionary} marks that the first record of the definition “the quality or state of being a person” is recorded in the 15th century.}. When Sister White and our pioneers wrote about the \emcap{personality of God}, they referred to \textit{the quality or state of God being a person}. In other words, they dealt with the question, “\textit{is God a person}”, and, “\textit{what is it that makes Him a person}” or “\textit{what is the quality or state of God being a person}”? Try to remember the last time you had a Bible study on the question, “\textit{is God a person?}” Think about how you can prove to yourself, from the Bible, that God is a person. Think about it. It is an important question. Upon this question hangs your view of God and your relationship toward Him. The \emcap{personality of God} is fundamental to true spirituality; true spirituality is based on your personal relationship with God. No real relationship of any kind can be formed with anyone unless he/she is a person. Maybe you have never asked yourself this question because you never felt a need to question if God is a person, and what is it (the quality or state) that makes Him a person. Or, maybe you were refraining from this question because you felt it might be a mystery that God did not intend to reveal. Maybe it will surprise you to know that God has given a definite and affirmative answer in His Word to the question “\textit{what is the quality or state of God being a person}”. What was even more surprising for us, was that the Adventist pioneers, including Sister White, had definite light regarding this topic, and they held it as a pillar of our faith, as part of the foundation of Seventh-day Adventist faith. When the \emcap{personality of God} is rightly understood in light of our historical past, old quotations shine in a new light and new shreds of evidence are presented, which will deepen the understanding of our past history and the present crisis.


\emcap{Umbile la Mungu} lilikuwa nguzo ya imani yetu. Leo, inakaribia kusahaulika. Tunapendekeza ufafanuzi wa busara kwa hilo. Ni kutokana na mageuzi ya lugha ya Kiingereza. Nini kinachomaanishwa na maneno, “\textit{Umbile la Mungu}”? Uelewa wa jumla wa neno la Kiingereza ‘\textit{personality}’ limebadilika kwa miaka mingi. Leo, ‘\textit{personality}’ kwa ujumla huonekana kama, “\textit{seti ya tabia, utambuzi, na mifumo ya kihisia}”\footnote{Wikipedia Contributors. “\textit{Personality.}” Wikipedia, Wikimedia Foundation, 19 Apr. 2019, \href{https://en.wikipedia.org/wiki/Personality}{en.wikipedia.org/wiki/Personality}.}, lakini katika karne ya kumi na tisa, na mwanzo wa karne ya ishirini, ilimaanisha “\textit{ubora au hali ya \textbf{kuwa Nafsi}}”\footnote{\href{https://www.merriam-webster.com/dictionary/personality}{Merriam-Webster Dictionary}, - ‘personality’} \footnote{\href{https://babel.hathitrust.org/cgi/pt?id=mdp.39015050663213&view=1up&seq=780}{Hunter Robert, The American encyclopaedic dictionary}, ‘\textit{personality}’ - “\textit{the quality or state of being personal}”; Kamusi iliyotajwa ilikuwa miliki ya Ellen White (angalia \href{https://repo.adventistdigitallibrary.org/PDFs/adl-22/adl-22251050.pdf?_ga=2.116010630.1065317374.1621993520-1506151612.1617862694&fbclid=IwAR3vwmp8jxtnpPEKv0KD9mCv8dJpmRGoyIXW0CkbQAjbU0h6YaBGqhgBzbk}{EGW Private and Office Libraries})}. Tunasoma ufafanuzi huu kama ufafanuzi wa msingi wa neno ‘\textit{personality}’ kutoka kwa Kamusi ya Merriam-Webster\footnote{\href{https://www.merriam-webster.com/dictionary/personality\#word-history}{Merriam-Webster Dictionary} inaonyesha kuwa rekodi ya kwanza ya ufafanuzi “the quality or state of being a person” iliandikwa katika karne ya 15.}. Wakati Dada White na waanzilishi wetu waliandika kuhusu \emcap{Umbile la Mungu}, walirejelea \textit{ubora au hali ya Mungu kuwa Nafsi}. Kwa maneno mengine, walishughulikia swali, “\textit{je Mungu ni Nafsi}”, na, “\textit{ni nini kinachomfanya awe Nafsi}” au “\textit{ubora au hali ya Mungu kuwa Nafsi ni ipi}”? Jaribu kukumbuka mara ya mwisho ulipokuwa na mafunzo ya Biblia juu ya swali, “\textit{je, Mungu ni Nafsi?}” Fikiria jinsi unavyoweza kuthibitisha kwako mwenyewe, kutoka Biblia, kwamba Mungu ni Nafsi. Fikiria juu ya hilo. Ni swali muhimu. Juu ya swali hili unaegemeza mtazamo wako wa Mungu na uhusiano wako kwake. \emcap{Umbile la Mungu} ni msingi wa hali ya kweli ya kiroho; hali ya kweli ya kiroho inatokana na uhusiano wako binafsi na Mungu. Hakuna uhusiano wa kweli wa aina yoyote unaweza kuundwa na yeyote asiye Nafsi. Labda hujawahi kujiuliza swali hili kwa sababu hujawahi kuhisi haja ya kuuliza kama Mungu ni Nafsi, na ni nini (ubora au hali) kinachomfanya kuwa Nafsi. Au, labda ulijiepusha na swali hili kwa sababu ulihisi kuwa ni fumbo ambalo Mungu hakukusudia kulifunua. Labda itakushangaza kujua kuwa Mungu ametoa jibu la uhakika na thabiti katika Neno Lake kwa swali “\textit{ni nini ubora au hali ya Mungu kuwa Nafsi}”. Kilichotushangaza zaidi, ni kwamba Waanzilishi wa Kiadventista, akiwemo Dada White, walikuwa na mwanga wa uhakika kuhusu mada hii, na waliishikilia kama nguzo ya imani yetu, kama sehemu ya msingi wa imani ya Waadventista Wasabato. Wakati \emcap{Umbile la Mungu} linapofahamika vizuri katika mwanga wa historia yetu ya zamani, nukuu za zamani zitang'aa katika mwanga mpya na ushahidi mpya utatokea, ambao utaongeza uelewa wa historia yetu ya zamani na mgogoro wa sasa.


The root problem of the Kellogg crisis was about the \emcap{personality of God}. It is certainly important to evaluate Kellogg's crisis over the \emcap{personality of God} using the meaning intended at that time; that is, using the definition of ‘\textit{personality},’ as the quality or state of God being a person. With this definition in mind, the Kellogg crisis comes into a new light and new relevant evidence is brought forth for us today. In light of this evidence, we see how God has led us in the past; thus, we should not fear for the future. Knowing and understanding this, as well as its importance, helps us to not be shaken by any wave of deception in present controversies. When Sister White was drawing Kellogg’s attention to the importance of this subject, she was drawing our attention also, as it is everything to us as a people.


Shida kuu ya mgogoro wa Kellogg ilikuwa juu ya \emcap{Umbile la Mungu}. Ni muhimu kutathmini mgogoro wa Kellogg juu ya \emcap{Umbile la Mungu} kwa kutumia maana iliyokusudiwa wakati huo; yaani, kutumia fasili ya ‘\textit{personality},’ kama ubora au hali ya Mungu kuwa Nafsi. Kuweka ufafanuzi huu akilini, mgogoro wa Kellogg unakuja katika mwanga mpya na ushahidi mpya unaofaa unaletwa kwa ajili yetu leo. Kwa kuzingatia uthibitisho huu, tunaona jinsi Mungu alivyotuongoza zamani; hivyo, hatupaswi kuogopea wakati ujao. Kujua na kuelewa hili, pamoja na umuhimu wake, hutusaidia kutotikiswa na wimbi lolote la udanganyifu katika migogoro ya sasa. Wakati Dada White alikuwa akivuta mawazo ya Kellogg kwa umuhimu wa somo hili, alivuta usikivu wetu pia, kwani ni kila kitu kwetu kama watu.


[Writing to Kellogg] \egw{You are not definitely clear on \textbf{the personality of God}, which is \textbf{\underline{everything} to us as a people}.}[Lt300-1903.7][https://egwwritings.org/read?panels=p14068.7705013]


[Akimwandikia Kellogg] \egw{Kwa hakika huelewi \textbf{Umbile la Mungu}, ambao ni \textbf{\underline{kila kitu} kwetu kama watu}.}[Lt300-1903.7][https://egwwritings.org/read?panels=p14068.7705013]


These studies on the \emcap{personality of God} will prompt a lot of new and hard questions. We do not promise to answer all of them, and perhaps you won’t be satisfied with the answers provided, but we pray, hope and believe that this book will fulfill the three objectives proposed in the beginning of this introduction. Through the reviving of the doctrine on the \emcap{personality of God}, we believe that your confidence in the Spirit of Prophecy will strengthen, and that you’ll find yourself rooted deeper in the Adventist message—where we find our identity as people—making you a more faithful Seventh-day Adventist. Most importantly, we want you to become more aware of God as your personal God. This will surely strengthen and deepen your relationship with Him.


Masomo haya kuhusu \emcap{Umbile la Mungu} yatazua maswali mengi mapya na magumu. Hatuahidi kuyajibu yote, na labda hutaridhika na majibu yanayotolewa, lakini tunaomba, tunatumaini na kuamini kwamba kitabu hiki kitatimiza malengo matatu yaliyopendekezwa mwanzoni mwa utangulizi huu. Kupitia kuhuisha fundisho la \emcap{Umbile la Mungu}, tunaamini kwamba imani yako katika Roho ya Unabii itaimarishwa, na kwamba utajikita zaidi katika ujumbe wa Waadventista—ambapo tunapata utambulisho wetu kama watu—hivyo kukufanya Mwadventista Msabato mwaminifu zaidi. Muhimu zaidi, tunataka umfahamu Mungu zaidi kama Mungu wako binafsi. Hii hakika itaimarisha na kuongeza uhusiano wako na Yeye.


We find answers to the issue on the \emcap{personality of God} in examining the Kellogg crisis, where Sister White gave the most definite light on the \emcap{personality of God} and on the foundation of Seventh-day Adventist faith. The following is the complete tenth chapter from the book, \textit{Testimonies for the Church Containing Letters to Physicians and Ministers Instruction to Seventh-Day Adventists}. This chapter, \textit{The Foundation of our Faith}, contains deep insight into the history of Kellogg’s crisis. It gives a historical overview of the truths God gave as the foundation of our faith and in these truths we find our identity as Seventh-day Adventists— keeping the commandments of God and having the faith of Jesus.


Tunapata majibu kwa suala la \emcap{Umbile la Mungu} katika kuchunguza mgogoro wa Kellogg, ambapo Dada White alitoa mwanga wa uhakika zaidi juu ya \emcap{Umbile la Mungu} na juu ya msingi wa imani ya Waadventista Wasabato. Ifuatayo ni sura kamili ya kumi kutoka kitabu, \textit{Testimonies for the Church containing Letters to Physicians and Ministers Instruction to Seventh-Day Adventist}. Sura hii, \textit{Msingi wa Imani yetu}, ina ufahamu wa kina katika historia ya mgogoro wa Kellogg. Inatoa muhtasari wa kihistoria wa kweli ambazo Mungu alitoa kama msingi wa imani yetu na katika kweli hizi tunapata utambulisho wetu kama Waadventista—kushika amri za Mungu na kuwa na imani ya Yesu.
