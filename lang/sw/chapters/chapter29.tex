
\qrchapter{https://forgottenpillar.com/rsc/en-fp-chapter29}{Kukumbuka mafundisho Yake katika historia yetu iliyopita}


Tunaleta safari yetu katika utafiti huu hadi mwisho wake. Tulianza safari hii kwa nguvu kutangaza kile kinachofanya msingi wa imani yetu. Tukafahamiana na historia na uzoefu wa waanzilishi wetu katika kuanzisha Kanisa la Waadventista Wasabato. Tumeona utume na kusudi Mungu alilowapa katika kutangaza ujumbe wa malaika watatu kwa ulimwengu wote. Ujumbe huu umeunganishwa na mafundisho yote mu-himu ya Biblia. Mafundisho haya ndiyo ambayo waanzilishi wetu wali-yaita \emcap{Kanuni za Msingi}, au nguzo za imani yetu. Mafundisho haya yana-wakilisha msingi wa imani yetu.


Mara baada ya kufahamu \emcap{Kanuni za Msingi} ambazo tulikuwa nazo hapo mwanzo, tumetambua tofauti zao kulingana na Imani zetu za Msingi za sasa, hasa kwa swali “\textit{Mungu ni nani}”? Zaidi ya hayo, mafundisho ya sasa ya Mungu hayana ufahamu wa Umbile lake. Kwa maneno mengine, inakosa ufahamu wa \textit{ubora au hali ya Mungu kuwa Nafsi}. Mabadiliko haya katika mafundisho yetu yanakuwa na uzito katika mwanga wa ujumbe wa malaika wa kwanza, ambayo inamhusu Mungu tunayemwabudu. Je, Mungu tunayemwabudu ni Mungu wa Utatu, au Yeye ni Mungu Mmoja, Baba, Mzee wa Siku, Huluki wa kibinafsi na wa kiroho?


Badiliko la ufahamu wetu wa Mungu ni nani kwa Waadventista Wasabato haikuwa ghafla; ilichukua miaka mingi ya mgogoro kufikia msimamo wetu wa sasa. Katika masomo haya, hatukuzama katika historia zaidi ya maisha ya Ellen White. Tuliona mabadiliko yanaanza kuchukua mahali katika wakati wake, wakati Dk. Kellogg aliposisitiza maoni juu ya \emcap{Umbile la Mungu}, ambayo \egwinline{ingepotosha akili za wale ambao hawajaimarishwa kabisa juu ya kanuni za msingi za ukweli wa sasa}[SpTB02 51.3; 1904][https://egwwritings.org/read?panels=p417.262]. Alitia shaka katika funuo wazi za \emcap{Umbile la Mungu} na wa Mwanawe, Yesu Kristo. Maoni yake yalipokelewa na karipio kali kutoka kwa Dada White na maonyo makali kwa kanisa, ili kuepuka njia ya mashaka katika ukweli rahisi unaoonyeshwa katika \emcap{Kanuni za Msingi}—kwamba Mungu mmoja ni wa kibinafsi, wa kiroho, na Kristo ni Mwanawe, \egwinline{aliyezaliwa kwa sura dhahiri ya nafsi ya Baba}[ST May 30, 1895, par. 3; 1895][https://egwwritings.org/read?panels=p820.12891]. Hivyo, Dada White alitetea kwa uthabiti mambo mawili ya kwanza ya \emcap{Kanuni za Msingi}.


Kama vile wakati wa Dk. Kellogg, wakati ndugu wengi walipokuwa wakiacha unyofu na uweupe wa Mafundisho ya Kristo, ndivyo yalivyo leo. Dada White alitabiri kwamba mabadiliko haya juu ya ufahamu wa \emcap{Umbile la Mungu} ungetokea katika kanisa letu, na kuanzishwa upya kwa msingi wetu wa imani ya zamani ungekuwa wa lazima. Je, \emcap{Kanuni za Msingi} zitaanzishwa tena katikati yetu? Matokeo yake yanategemea kabisa kila mtu ambaye ni miongoni mwa washiriki wa Kanisa la Waadventista Wasabato. Leo, siku na zama hii, ndio wakati uanzishwaji upya utafanyika. Maonyo na makemeo yanayotamkwa na kalamu ya Roho wa Unabii haujawahi kuwa muhimu zaidi kuliko leo. Mungu ameweka matokeo ya mwisho katika mikono yako. Maonyo haya yakigusa kiini cha nafsi yako, Mungu anakuita usimame imara kwenye jukwaa la Ukweli wa milele. Anakuita kushikilia kwa uthabiti \emcap{Kanuni za Msingi}, ambazo zinatokana na mamlaka isiyotiliwa shaka.



Ifuatayo, tunawasilisha sehemu ya barua kutoka kwa Dada White kwa Dk. Kellogg, ambamo tunapata onyo zito kwetu leo katika kuweka upya msingi wa imani yetu. Wakati tunafahamiana ukweli juu ya \emcap{Umbile la Mungu} na utata wake juu ya fundisho la Utatu, barua ifuatayo inang'aa katika nuru mpya, yenye jumbe na kanuni ambazo ni muhimu kwetu leo, ili tujue jinsi ya kuishi katika hali yetu ya sasa.


\egw{\textbf{Msingi wa Imani Yetu}}


\egw{Kuhusu kitabu Hekalu Hai, \textbf{nimeelekezwa na mjumbe wa mbinguni} kwamba \textbf{baadhi ya hoja katika kitabu hiki si za kweli, na kwamba hoja hii ingeongoza kupotosha akili za wale ambao hawajaimarishwa kabisa \underline{juu ya kanuni za msingi} za ukweli wa sasa.} \textbf{Inatanguliza yale ambayo si chochote ila ni dhana tu kuhusu \underline{Umbile la Mungu} na \underline{uwepo wake ulipo}}. Hakuna mtu katika dunia hii aliye na haki ya kutafakari juu ya swali hili. ‘Mambo ya siri ni ya Bwana, Mungu wetu, lakini mambo yaliyofunuliwa ni yetu sisi na watoto wetu milele.’}[Lt232-1903.39; 1903][https://egwwritings.org/read?panels=p10197.48]


\egwnogap{\textbf{Nimeidhinishwa na Bwana kusema, Maoni yaliyomo katika Hekalu Hai kuhusu Umbile la Mungu \underline{ni kinyume na kweli ambayo Mungu ametupa}}. \textbf{Ukweli wa wakati huu sasa unapaswa kuletwa mbele ya watu.} Ndugu na dada zetu katika kila kanisa na kila mahali wanapaswa kujilinda kwa uangalifu dhidi ya kuruhusu akili zao kuzama kwa mambo yanayowavuta mbali na mambo ya milele. Adui atatumia baadhi ya kauli zilizotolewa katika Hekalu Hai ili kuwajaribu wengine kama wakati alipowajaribu Adamu na Hawa katika Edeni. \textbf{Nawaonya ndugu zetu wasiingie kwenye mabishano juu ya uwepo na Umbile la Mungu. \underline{Kauli zilizotolewa katika Hekalu Hai kuhusiana na jambo hili si sahihi}.} Maandiko yaliyotumika kuthibitisha fundisho lililowekwa hapo ni Maandiko yaliyotumika vibaya.}[Lt232-1903.40; 1903][https://egwwritings.org/read?panels=p10197.49]


\egwnogap{Nilionywa nisiingie kwenye mabishano kuhusu swali litakalojitokeza katika mambo haya, \textbf{kwa sababu mabishano yanaweza kuwaongoza watu kutumia \underline{hila}, na akili zao zingeongozwa mbali na ukweli wa Neno la Mungu hadi kwenye \underline{makisio na kazi ya kubahatisha}}. Kadiri nadharia potofu zinavyojadiliwa, ndivyo watu watakavyomjua Mungu kwa kiwango kidogo na wa ile kweli inayotakasa nafsi.}[Lt232-1903.41; 1903][https://egwwritings.org/read?panels=p10197.50]


\egwnogap{Sisi ni watu wa Mungu wanaozishika amri. \textbf{Kwa miaka hamsini iliyopita kila awamu ya uzushi umeletwa juu yetu, ili kubomoa \underline{kanuni za msingi za imani yetu}}. Jumbe za kila utaratibu na aina zimehimizwa kwa Waadventista Wasabato kuchukua nafasi ya ukweli ambao \textbf{pointi baada ya pointi} umeshuhudiwa na uweza wa Bwana utendao miujiza. \textbf{Lakini alama za njia ambazo zimetufanya tulivyo zinapaswa kuhifadhiwa, na \underline{zitahifadhiwa}, kama vile Mungu ameonyesha kupitia Neno Lake na ushuhuda wa Roho wake. Kutoka kwa mfumo mkuu wa ukweli kama ulivyokuwa umewasilishwa na wajumbe wa Mungu, \underline{hakuna pini itakayoondolewa}}.}[Lt232-1903.42; 1903][https://egwwritings.org/read?panels=p10197.51]


\egwnogap{\textbf{Nimeitwa na Mungu kusimama katika kutetea ukweli ambao tumepewa kadri tulivyofuata maongozi yake yeye aliye njia, na kweli, na uzima. Wacha kila mwanzilishi katika kazi ashikilie kwa uthabiti ukweli huu. \underline{Sifa za kipekee za imani yetu zinapaswa kushikiliwa sana na mshiko wa imani}.}}[Lt232-1903.43; 1903][https://egwwritings.org/read?panels=p10197.52]


\egwnogap{Hadithi ambazo kwa wakati huu zinatungwa na baadhi ya wahudumu wamishonari wa kitiba hazipaswi kuchukuliwa kama ukweli. \textbf{\underline{Asili yao ya kweli itafichuliwa kwa muda usiokuwa mrefu.}} \textbf{Itaonekana kwamba yalifanywa chini ya uwezo wa hila wa yule mwasi mkuu, ambaye anafanya kazi kama malaika wa nuru, akilitawala akili kwa madanganyo yaliyofichwa hata anatafuta kuwahadaa kama yamkini walio wateule wenyewe}.}[Lt232-1903.44; 1903][https://egwwritings.org/read?panels=p10197.53]


\egwnogap{Ni ushawishi gani isipokuwa ule wa mdanganyifu unaweza kusababisha watu katika \textbf{hatua hii ya historia yetu kufanya kazi kwa njia ya siri, yenye nguvu ya kubomoa \underline{misingi ya imani yetu}—\underline{misingi ambayo iliwekwa mwanzoni mwa kazi yetu} kwa kujifunza kwa maombi Neno na kwa ufunuo}. \textbf{\underline{Juu ya misingi hii tumekuwa tukiijenga katika miaka hamsini iliyopita}}. Je! msingi mpya utajengwa na watu ambao Mungu hakuwajalia uzoefu maalum amewapa watu aliowateua kuusimamisha misingi ya imani yetu? \textbf{Watu wanaojitahidi kujenga msingi huo wa uwongo wanaweza kuseme kwamba wamepata njia mpya, na kwamba wanaweza kuweka msingi imara zaidi kuliko ile iliyowekwa. \underline{Lakini huu ni udanganyifu mkubwa}. \underline{Hakuna mwanadamu mwingine anaweza kujenga msingi mwingine isipokuwa ule uliojengwa}.}}[Lt232-1903.45; 1903][https://egwwritings.org/read?panels=p10197.54]



\egwnogap{Nimeagizwa kuwaambia watu wetu kwamba siku za nyuma wengi wamefanya ujenzi wa imani mpya, uanzishwaji wa kanuni mpya. Lakini jengo lao lilisimama kwa muda gani? Hivi karibuni ukaanguka vipande vipande; \textbf{kwa maana haikujengwa juu ya Mwamba}.}[Lt232-1903.46; 1903][https://egwwritings.org/read?panels=p10197.55]


\egwnogap{Je! Wale wanafunzi wa kwanza hawakukutana na maneno ya wanadamu? Je, hawakupaswa kusikiliza nadharia za uwongo na kisha kusimama imara, baada ya kufanya yote, kusimama, kusema, ‘Hakuna hawezaye kujenga Msingi mwingine isipokuwa ule uliowekwa’? Kundi moja baada ya lingine liliibuka na mafundisho ya uwongo, kwa sababu wanadamu walikuwa na ufahamu mdogo sana wa Mungu.}[Lt232-1903.47; 1903][https://egwwritings.org/read?panels=p10197.56]


\egwnogap{\textbf{Ndugu zangu na dada zangu, jifunzeni sura ya kumi na tatu, kumi na nne, kumi na tano, kumi na sita na kumi na saba ya Yohana. Maneno ya sura hizi yanajieleza yenyewe. ‘Hii ni uzima wa milele,’ Kristo alitangaza, ‘wapate kukujua wewe, Mungu wa pekee wa kweli, na Yesu Kristo, uliyemtuma.’ \underline{Kwa maneno haya ubinafsi wa Mungu na wa Mwana Wake unasemwa waziwazi.} \underline{Ubinafsi wa mmoja hauondoi umuhimu kwa ubinafsi wa mwingine}.}}[Lt232-1903.48; 1903][https://egwwritings.org/read?panels=p10197.57]


\egwnogap{Mungu hatawahi kueleweka na mwanadamu yeyote. Njia zake na kazi zake hazichunguziki. Kuhusiana na mafunuo \textbf{ambayo amejifanyia katika Neno Lake, tunaweza kuzungumza}. \textbf{Lakini linapokuja suala la kuzungumza au kuandika juu ya nafsi na uwepo wa Mungu, tuseme, ‘Wewe ndiwe Mungu, na njia zako hazichunguziki.’}}[Lt232-1903.49; 1903][https://egwwritings.org/read?panels=p10197.58]


\egwnogap{Ni chukizo kuweka katika akili za vijana au wazee \textbf{mbegu za \underline{uvumi} kuhusu somo hili}. Mbegu kama hizo, zikipandwa na kuachwa zikue, zitachipuka na \textbf{kuzaa mavuno ya \underline{hisia za kikafiri}}. Natoa onyo hili kwa wote. \textbf{Hatutaki ujanja kama huo uliotolewa katika Hekalu Hai}. Kuna mambo mazuri katika kitabu. Lakini pia yapo magugu kati ya ngano. Kitabu kina mawazo mengi sahihi, lakini kina pia kauli ambazo zitaleta madhara. Wale wanaokubali makapi kwa ngano watajikuta wenyewe wakipoteza hisia zao za ukuu wa Mungu na kumleta katika hali ya kawaida ya bei nafuu. Hii ni kazi ya mdanganyifu mkuu. \textbf{Ndugu zetu hawatakiwi kuitwa kutoka katika kazi zao ili wasomee swali la Mungu yuko wapi na yeye ni nini. Hatupaswi kuthubutu kushiriki katika haya majadiliano, tusije tukaangamizwa.} Sanduku la Mungu lilipokuwa linachukuliwa kutoka katika nchi ya Wafilisti hadi kwenye kambi ya Israeli, watu wa Bethshemeshi wakaona jambo hilo kwa udadisi. Mungu alichukizwa, na wengi walipigwa na kifo.}[Lt232-1903.50; 1903][https://egwwritings.org/read?panels=p10197.59]


\egwnogap{\textbf{Hebu tuzungumze juu ya Kristo, uwepo wake wa jadi, huduma Yake yenye unyenyekevu, uweza Wake mkuu, Utukufu wake wa kibinafsi unaotarajiwa katika nyua za mbinguni. Mwana wa Mungu anarudisha uzima yule amtakaye. ‘Yote aliyo nayo Baba ni yangu,’ Asema. ‘Mimi na Baba Yangu tu umoja.’ Yeye ana ukuu, ya sasa na inayotarajiwa, ambayo inashangaza dhana ya mwanadamu. Yeye huzunguka wanadamu kwa mkono wake mrefu wa kibinadamu, huku kwa mkono wake wa kimungu akishika kiti cha enzi chake asiye na mwisho}.}[Lt232-1903.51; 1903][https://egwwritings.org/read?panels=p10197.60]


\egwnogap{\textbf{Kuna ujuzi wa Mungu na wa Kristo ambao wote wanaookolewa wanapaswa kuwa nao. ‘\underline{Hii ni uzima wa milele},’ Kristo asema, ‘\underline{wapate kukujua Wewe, Mungu wa pekee wa kweli, na Yesu Kristo uliyemtuma}.’ Naye asema tena, ‘Mtu yeyote akitaka kunifuata, na ajikane mwenyewe, ajitwike msalaba wake, anifuate.’ Kwa wote wanaompokea kama Mkombozi wao, huwapa uwezo wa kufanywa wana wa Mungu. Kila mtu anayemwamini kwa kweli ataongozwa na imani na kuinuliwa kwa mkono wa Uweza.}}[Lt232-1903.52; 1903][https://egwwritings.org/read?panels=p10197.61]


\egwnogap{\textbf{Wale ambao hawapokei kwa imani mpango wa Mungu wa kukomboa wanadamu} wanakataa Roho wa neema, na katika siku kuu ya mwisho hukumu yao itakuwa, ‘Ondokeni Kwangu.’ Wamefanya hivyo kuchukia haki na kuendeleza uovu, na lazima wafukuzwe milele mbali na uwepo wa Mungu, kufukuzwa kutoka kwa furaha hadi kifo—kifo cha milele.}[Lt232-1903.53; 1903][https://egwwritings.org/read?panels=p10197.62]


\egwnogap{Wale ambao katika maisha haya wanampenda Mungu na kuthamini mawazo Yake watatumia uwezo wao na talanta zao kama mawakili waaminifu, wakizitumia vyema, lakini bila kudai malipo yoyote kama haki yao. \textbf{Wanapojikana nafsi zao na kumfuata Yesu, wakiinua msalaba, watapata kwamba msalaba ni mwepesi, na kwamba ni rahani, kama wanavyoubeba, kwamba siku moja watapewa taji ya uzima wa milele}. Je, Utukufu na faida itakuwaje na starehe ya huo uzima wa milele ambao utatolewa kwa wale tu ambao umetayarishwa kwa ajili yao? Furaha kuu ya mshindi itakuwa kwamba yuko katika uwepo wa Kristo. ‘Mahali nilipo, Mtumishi wangu pia atakuwa,’ akasema. Naye akaomba, ‘Baba, nataka hao pia ulionipa wawe pamoja nami mahali nilipo; ili wauone utukufu wangu.’ Kristo anazungumza juu ya utukufu wa uwepo wa Baba yake na nyumba ya Baba yake. \textbf{Utukufu unaopaswa kufunuliwa wote wanaookolewa ni utukufu aliokuwa nao Kristo pamoja na Baba yake kabla ya ulimwengu kuwa—\underline{utukufu usioweza kufikiwa wa mazungumzo yao pamoja}}. \textbf{Malaika hawakuwa kwa mahojiano kati ya Baba na Mwana wakati mpango wa wokovu uliwekwa.} Wanadamu hao wanaotaka kuingilia siri za Aliye Juu, ambaye hukaa milele, waonyesha hawafahamu mambo ya kiroho na ya milele. Wakati huu sauti ya rehema ingali inasikika, ingekuwa Bora zaidi wanyenyekee mavumbini na wamsihi Mungu awafundishe njia zake.}[Lt232-1903.54; 1903][https://egwwritings.org/read?panels=p10197.63]


\egw{Onyo Kwa Wakati}


\egw{Kuna wale ambao wamekuwa wakitafuta kutekeleza mipango yao ya ubinafsi, bila kujali ushawishi ambao jambo hili lingekuwa nalo juu ya tendo na kazi ya Mungu. Ni wakati hao wahisi kazi ya ndani ya neema juu ya mioyo yao, kwamba kazi ya umishonari ya matibabu isiwakilishwe vibaya sana. Wacha wafanyakazi wetu wa kimishenari wa matibabu wasiwe kama ulimwengu kwa mazoea na matendo ambayo walimwengu watawaepuka kwa dharau, wakisema, ‘Mimi ni mzuri kama wao.’ Kuna visa ambapo kazi ya umishonari ya kitiba imekuwa ikifanywa hivi kwamba jina ‘mmishonari wa matibabu’ lingeweza kutupiliwa mbali zaidi; kwa kuwa imesemwa vibaya sana, na Mungu amevunjiwa heshima.}[Lt232-1903.55; 1903][https://egwwritings.org/read?panels=p10197.65]


\egwnogap{\textbf{Tunaishi katikati ya hatari za siku za mwisho. Watu wetu lazima sasa waamkie kazi amabayo iko mbele yao. Tunapaswa kuinua kiwango na kutangaza ujumbe wa mwisho wa onyo kwa ulimwengu unaoangamia. Wale ambao wana ujuzi wa ukweli wa wakati huu sasa wanapaswa kushikilia juu sana ile bendera yenye maandishi, ‘Amri za Mungu na imani ya Yesu.’ }[Ufunuo 14:12.]}[Lt232-1903.56; 1903][https://egwwritings.org/read?panels=p10197.66]


\egwnogap{Nawaomba ndugu zangu wahudumu wajichunguze wenyewe kama wako katika imani au sivyo. \textbf{Ikiwa watakubali maonyesho ya kimizimu yaliyofanywa katika Living Temple, miguu yao hivi karibuni itakuwa inakanyaga katika njia zilizokatazwa. Viwakilishi hivi ni Alfa ya mafundisho ambayo yangetupeleka mbali na kweli kama tulivyoipokea kutoka kwa Neno ya Mungu}. \textbf{\underline{Kukubalika kwa hisia hizi kutasababisha imani dhaifu, yenye kuyumbayumba}}. \textbf{Kama hili ndilo fundisho ambalo linapaswa kutolewa katika kazi ya umishonari ya matibabu, haitakuwa muda mrefu kabla ya kukosa msingi wa kupanda miguu yetu}. \textbf{Nimealikwa kusema kwamba hisia hizi potofu ni hisia za adui mjanja} na hazipaswi kutolewa kwa kijana wetu yeyote ambaye anatafuta kupata elimu ya laini ya umishonari wa matibabu. Kwa ajili ya vijana hawa, nazungumza kwa uamuzi.}[Lt232-1903.57; 1903][https://egwwritings.org/read?panels=p10197.67]


\egwnogap{\textbf{Imani inayoisha ya watu wa Mungu \underline{lazima iwe na ufufuo}}. \textbf{Kuinuliwa kwa akili ya kibinadamu imeanza kazi yake kati yetu na imekwenda mbali sana}. Fikira za kibinadamu imewekwa mahali ambapo ukweli wa kimungu, utakasao unapaswa kuwa. \textbf{Kristo anasubiri kuwasha imani na upendo katika mioyo ya watu wake}. \textbf{Nadharia potofu zisipate mtazamo chanya kutoka kwa watu wanaopaswa kusimama kidete kwenye jukwaa la ukweli wa milele.} \textbf{Mungu anataka tushikilie kwa uthabiti \underline{kanuni za msingi ambazo zimejengwa juu ya mamlaka yasiyo na shaka}}. Anatuita tujifunze maneno na matendo ya Kristo, Mmishonari mkuu zaidi ambaye ulimwengu huu umewahi kumjua.}[Lt232-1903.58; 1903][https://egwwritings.org/read?panels=p10197.68]


\egwnogap{\textbf{Mawazo ya mwalimu wa kweli yanapoachana kwa njia yoyote na udhahiri, ukweli wa injili unaojikana mwenyewe, yuko tayari kupokea hisia za kuwaziwa zinazoitwa ukweli. Zikiwa zimevikwa mavazi ya mwanga, hisia hizi zinawasilishwa kwa wengine, na pia mara nyingi zinapata kibali. Nimeagizwa kuwaambia washiriki wa makanisa yetu, Mjiifadhi mbali na mawazo ya kimizimu}. \textbf{Hatujishughulishi na ngano}. Hasha! ngano hizi ziziwasilishwe kwa watu wetu badala ya ukweli. \textbf{Mungu apishe mbali yeyote miongoni mwetu atakayejenga juu ya mchanga}.}[Lt232-1903.59; 1903][https://egwwritings.org/read?panels=p10197.69]


\egwnogap{\textbf{Bwana ametupa ujumbe ulio wazi na dhahiri wa ukweli kwa wakati huu}. Hebu tutangaze ujumbe huu. \textbf{Hebu tujifunze mafundisho ya Kristo}, na tuwasilishe kile ambacho ametuamuru tuwasilishe. \textbf{Anayezindua kwa hekima yake mwenyewe kuhubiri mambo ya ajabu, ambayo Mungu hajampa, hupata akili tayari kuchachushwa na mawazo mapya ambayo inabidi yeye awasilishe}. \textbf{Shetani hufuata kazi anayofanya, na juhudi za watumishi wa Mungu walio wa kweli inafanywa ngumu zaidi}. \textbf{Kusonga mbele kwa njia yake kumezuiliwa, na Roho yake inahuzunika}.}[Lt232-1903.60; 1903][https://egwwritings.org/read?panels=p10197.70]


Tunaomba kwamba Mungu azungumze kwa uwazi katika moyo wa kila mtu anayesoma maonyo haya, kwamba tujiepushe na kuvuka msingi wa imani yetu. Mungu anatuita tushikilie kwa uthabiti \emcap{Kanuni za Msingi} ambazo zinatokana na mamlaka isiyotiliwa shaka.


% Remembering His teaching in our past history

\begin{titledpoem}
    
    \stanza{
        The pioneers shared a truth profound \\
        Foundations built on solid ground \\
        The pillars set by God’s own hand \\
        The principles on which we stand.
    }

    \stanza{
        The question, “Who is God?” remains \\
        To worship Him, exalt His name \\
        Is He a triune mystery now? \\
        Or Father God, to Whom we bow.
    }

    \stanza{
        The warnings echo through the years \\
        The testimony still appears \\
        Beware to build on sifting sand \\
        For only truth will ever stand.
    }

    \stanza{
        Waymarks which made us what we are \\
        Must be preserved both near and far, \\
        Hold fast the truth with faith’s strong grip \\
        Don’t let our sure foundation slip.
    }
    
\end{titledpoem}
