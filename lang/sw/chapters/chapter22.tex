\qrchapter{https://forgottenpillar.com/rsc/en-fp-chapter22}{The bottom of the issue}


\qrchapter{https://forgottenpillar.com/rsc/en-fp-chapter22}{Kiini cha suala hilo}


Today, when we compare our current Fundamental Beliefs with the previous \emcap{Fundamental Principles} we see the change in the foundation of Seventh-day Adventist faith. This change has occurred in the understanding of God’s person, or the \emcap{personality of God}. Particular to the \emcap{personality of God}, Sister White wrote that the track of truth lies close beside the track of error:


Leo, tunapolinganisha Imani zetu za Msingi za sasa na \emcap{Kanuni za Msingi} iliyotangulia tunaona mabadiliko katika msingi wa imani ya Waadventista Wasabato. Mabadiliko haya yametukia katika ufahamu wa nafsi ya Mungu, au \emcap{Umbile la Mungu}. Hasa kwa \emcap{Umbile la Mungu}, Dada White aliandika kwamba unyayo wa ukweli upo karibu na unyayo wa kosa:


\egw{\textbf{The track of truth lies close beside the track of error}, and both tracks may \textbf{seem }to be one to minds which are not worked by the Holy Spirit, and which, therefore, \textbf{are not quick to discern the difference between truth and error}.}[SpTB02 52.2; 1904][https://egwwritings.org/read?panels=p417.266]


\egw{\textbf{Njia ya ukweli iko karibu na njia ya makosa}, na nyayo zote mbili \textbf{zinaweza }kuonekana kuwa moja kwa akili ambazo hazifanyiwi kazi na Roho Mtakatifu, na ambazo, kwa hiyo, \textbf{si za kupambanua haraka kati ya kweli na uwongo}.}[SpTB02 52.2; 1904][https://egwwritings.org/read?panels=p417.266]


We ask ourselves, how can we draw a clear line between these two tracks? In order to do that we need to get to the bottom of the issue. We need to find a distinguishing principle that separates these two tracks.


Tunajiuliza, tunawezaje kuchora mstari wazi kati ya nyayo hizi mbili? Ili kufanya kwamba tunahitaji kupata undani wa suala hilo. Tunahitaji kupata kanuni ya kutofautisha na kutenganisha nyayo hizi mbili.


By studying our current Trinitarian belief and the works of our pioneers regarding the \emcap{personality of God}, we have found one characterizing principle that distinguishes the truth on the \emcap{personality of God}, as held by our pioneers, from our current Trinitarian belief. Both sides claim the Bible to be their ultimate authority, yet differences can be discerned by the interpretation of the Bible. In the following, we are talking about understanding and interpreting Scripture concerning God’s person. Understanding God’s person can be presented in two distinct, mutually exclusive understandings, which clearly draw a line between the two opposing camps.


Kwa kusoma imani yetu ya sasa ya Utatu na kazi za waanzilishi wetu kuhusu \emcap{Umbile la Mungu}, tumepata kanuni moja ya sifa inayotofautisha ukweli juu ya \emcap{Umbile la Mungu}, kama unavyoshikiliwa na waanzilishi wetu, kutokana na imani yetu ya sasa ya Utatu. Pande zote wanadai Biblia kuwa mamlaka yao kuu, lakini tofauti zinaweza kutambuliwa na tafsiri ya Biblia. Ifuatayo, tunazungumza juu ya uelewa na kutafsiri Maandiko yanayohusu nafsi ya Mungu. Kuelewa nafsi ya Mungu kunaweza kuwa iliyowasilishwa kwa uelewa mbili tofauti, wa kipekee, ambao huchota mstari wazi kati ya kambi mbili zinazopingana.


One, more popular, view is that God presented Himself in a language that is familiar to us in order to explain only the concepts of salvation. So, God presented Himself in words such as ‘\textit{father}’, ‘\textit{son}’, and ‘\textit{spirit}’, to describe the relationships between these concepts. This makes none of these words interpretable by their obvious meaning; rather, they hold symbolic or metaphoric value. The principle behind this reasoning is: \textbf{God adjusted Himself to man}.


Mtazamo mmoja, maarufu zaidi, ni kwamba Mungu alijionyesha katika lugha ambayo tunaizoea ili kueleza dhana tu za wokovu. Kwa hivyo, Mungu alijidhihirisha kwa maneno kama vile ‘\textit{baba}’, ‘\textit{mwana}’, na ‘\textit{roho}’, kuelezea uhusiano kati ya dhana hizi. Hii inafanya maneno haya yasiweze kufasiriwa kwa maana yake dhahiri; badala yake, wanashikilia ishara au thamani ya sitiari. Kanuni iliyo nyuma ya hoja hii ni: \textbf{Mungu alijishusha kwa mwanadamu}.


The other, opposing, view is that \textbf{God adjusted man to Himself}; \textit{He created man in His own image}. Therefore, words like ‘\textit{father}’, ‘\textit{son}’, and ‘\textit{spirit}’, as they address God, imply their obvious meaning. This is the fundamental difference.


Mtazamo mwingine, unaopingana, ni kwamba \textbf{Mungu alimrekebisha mwanadamu kwa nafsi Yake}; \textit{Alimuumba mwanadamu katika nafsi ya picha yake}. Kwa hiyo, maneno kama vile ‘\textit{baba}’, ‘\textit{mwana}’, na ‘\textit{roho}’, wanapozungumza na Mungu, yanamaanisha maana dhahiri. Hii ndiyo tofauti ya kimsingi.


When we come to understand biblical terms like ‘\textit{person}’, ‘\textit{father}’, ‘\textit{son}’ and ‘\textit{spirit}’, we must choose which view we support and apply it accordingly. Either these terms are understood in their obvious meaning, or symbolically or metaphorically. There is no middle ground between these two; we must choose one. The following quotation should settle any dilemma.


Tunapofikia kuelewa maneno ya Biblia kama ‘\textit{nafsi}’, ‘\textit{baba}’, ‘\textit{mwana}’ na ‘\textit{roho}’, ni lazima tuchague ni mtazamo upi tunaounga mkono na tuitumie ipasavyo. Aidha, maneno haya yanaeleweka kidhahiri, au kiishara au kitamathali. Hakuna msingi wa katikati baina ya haya mawili; lazima tuchague moja. Nukuu ifuatayo inapaswa kutatua shida yoyote.


\egw{\textbf{\underline{The language of the Bible should be explained according to its obvious meaning, unless a symbol or figure is employed}}.}[GC 598.3; 1888][https://egwwritings.org/read?panels=p133.2717]


\egw{\textbf{\underline{Lugha ya Biblia inapaswa kufafanuliwa kulingana na maana yake dhahiri, isipokuwa ishara au sura imetumika}}.}[GC 598.3; 1888][https://egwwritings.org/read?panels=p133.2717]


We believe that it is impossible for the Bible to be its own interpreter and not explain its own symbols. If the Bible applies the word ‘father’ to God, but never explains this term, then it should be accepted in its obvious meaning. The same applies to the words ‘son’ and ‘spirit’. Man is created in the image of God. God adjusted man to Himself. The obvious meaning is derived from the experience of man. We understand the obvious meaning of the word ‘father’ through regular, human fathership. But our fathership is the image of our God Who is the Father to His Son. Paul testified:


Tunaamini kwamba haiwezekani kwa Biblia kuwa mfasiri yake yenyewe na isieleze alama yake. Ikiwa Biblia inatumia neno ‘baba’ kwa Mungu, lakini haielezi kamwe neno hili, basi inapaswa kukubaliwa katika maana yake dhahiri. Vivyo hivyo kwa maneno ‘mwana’ na ‘roho’. Mwanadamu ameumbwa kwa mfano wa Mungu. Mungu alimrekebisha mwanadamu kwa nafsi yake. Maana ya wazi ni inayotokana na uzoefu wa mwanadamu. Tunaelewa maana ya wazi ya neno ‘baba’ kupitia ubaba wa kawaida, wa kibinadamu. Lakini ubaba wetu ni mfano wa Mungu wetu Aliyeko Baba kwa Mwanawe. Paulo alishuhudia:


\bible{For this cause I bow my knees unto \textbf{the Father of our Lord Jesus Christ, Of whom the whole \underline{family} in heaven and earth is \underline{named}}}[Ephesians 3:14-15].


\bible{For this cause I bow my knees unto \textbf{the Father of our Lord Jesus Christ, Of whom the whole \underline{family} in heaven and earth is \underline{named}}}[Ephesians 3:14-15].


In Greek, the word ‘\textit{family}’ is the word ‘\textit{patria}’, derived from the word ‘\textit{pater}’, which means ‘\textit{father}’. Some translations even render this verse with \bible{Of whom all \textbf{paternity} in heaven and earth is named} (DRB), which is a more literal translation. The Father of our Lord Jesus Christ is truly the father to His Son, just as truly as we are fathers to our children on Earth. Our paternity on Earth is named according to Paternity in Heaven, where God is the Father of our Lord Jesus Christ. Our earthly paternity is an image of Heavenly Paternity, where God is the Father to His Son. This supports the obvious meaning that Jesus is truly the Son of our God.


Katika Kigiriki, neno ‘\textit{familia}’ ni neno ‘\textit{patria}’, linalotoka-na na neno ‘\textit{pater}’, ambalo maana yake ni ‘\textit{baba}’. Baadhi ya tafsiri hata hutafsiri mstari huu kwa \bible{Of whom all \textbf{paternity} in heaven and earth is named} (DRB), ambayo ni tafsiri halisi zaidi. Baba wa Bwana wetu Yesu Kristo kweli ni baba wa Mwanawe, kama vile sisi tulivyo baba kwa watoto wetu hapa Duniani. Ubaba wetu Duniani unaitwa kulingana na Ubaba wa Mbinguni, ambapo Mungu ni Baba ya Bwana wetu Yesu Kristo. Ubaba wetu wa kidunia ni mfano wa Ubaba wa Mbinguni, ambapo Mungu ni Baba kwa Mwanawe. Hii inaunga mkono maana iliyo wazi kwamba Yesu kweli ni Mwana wa Mungu wetu.


The same underlying principle applies to the understanding behind the word ‘\textit{spirit}’ and the word ‘\textit{being}’. God adjusted man to Himself; He created man in His own image. Man is a being, possessing body and spirit, just like God—and in saying this, we are not saying that man and God possess the same nature. God formed man from the dust of the ground. His physical nature is confined to the elements found on the earth. We do not pry into the nature of God. That will forever remain a mystery to us; it is not revealed unto us. But what is revealed to us is that He has a form, and the form of a man is an image of the form of God. The Bible plainly approves this understanding when describing God sitting upon His throne:


Kanuni hiyo hiyo ya msingi inatumika kwa ufahamu nyuma ya neno ‘\textit{roho}’ na neno ‘\textit{kuwa}’. Mungu alimrekebisha mwanadamu kwa nafsi Yake; Alimuumba mwanadamu kwa mfano wake. Mwanadamu ni kuwa, mwenye mwili na roho, kama Mungu tu--na kwa kusema hivi, hatusemi hivyo mwanadamu na Mungu wana asili moja. Mungu aliumba mtu kwa mavumbi ya ardhi. Yake asili ya kimwili imefungwa kwa vipengele vinavyopatikana duniani. Hatuingii kwenye asili ya Mungu. Hilo litabaki kuwa siri kwetu milele; haikufunuliwa kwetu. Lakini ni nini iliyofunuliwa kwetu ni kwamba ana umbo, na umbo la mwanadamu ni mfano wa umbo la Mungu. Biblia inakubali ufahamu huu kwa uwazi inapomwelezea Mungu aliyeketi kwenye kiti chake cha enzi:


\bible{\textbf{upon the likeness of the throne was the likeness as \underline{the appearance of a man}} above upon it}[Ezekiel 1:26].


\bible{\textbf{juu ya mfano wa kile kiti cha enzi kulikuwa na sura kama \underline{kuonekana kwa mtu}} juu juu yake}[Ezekieli 1:26].


The obvious meaning of the word ‘\textit{spirit}’, applied to the Spirit of God, is derived from the understanding of “\textit{the spirit of man}”. God adjusted man to Himself; He created man in His own image. Just as man possesses a spirit, God possesses a Spirit. The spirit of man has the nature of man, and the spirit of God has the nature of God. With respect to their nature, they are not the same, but respective of their relation to their inner being, they are the same; the Bible puts them on the same level. \bible{\textbf{The \underline{Spirit} itself beareth witness with \underline{our spirit}}, that we are the children of God:}[Romans 8:16]; \bible{For what \textbf{man knoweth the things of a man}, save the \textbf{\underline{spirit of man} which is in him}? \textbf{\underline{even so}} the things of \textbf{God knoweth} no man, \textbf{but \underline{the Spirit of God}}.}[1 Corinthians 2:11].


Maana ya wazi ya neno ‘\textit{roho}’, linalotumika kwa Roho wa Mungu, linatokana na ufahamu wa “\textit{roho ya mwanadamu}”. Mungu alimrekebisha mwanadamu kwa nafsi Yake; Alimuumba mwanadamu katika picha yake mwenyewe. Kama vile mwanadamu alivyo na roho, Mungu ana Roho. Roho ya mwanadamu ina asili ya mwanadamu, na roho ya Mungu ina asili ya Mungu. Kuhusiana na asili yao, si sawa, lakini kwa kuzingatia uhusiano wao na utu wao wa ndani, wao ni sawa; Biblia inawaweka katika kiwango sawa. \bible{\textbf{Roho mwenyewe hushuhudia pamoja na \underline{roho zetu}}, kwamba sisi tu watoto wa Mungu:}[Warumi 8:16]; \bible{Kwa maana ni nani \textbf{mwanadamu ayajuaye mambo ya mwanadamu}, ila \textbf{\underline{roho ya mwanadamu} iliyo ndani yake}? \textbf{\underline{vivyo hivyo}} na mambo ya \textbf{Mungu hakuna ajuaye}, \textbf{ila \underline{Roho wa Mungu}}.}[1 Wakorintho 2:11].


In terms of family relationships and the quality or state of being a person, man and God are alike, because God created man in His own image. God adjusted man unto Himself. But in their nature, God and man are not alike. God is divine and man is earthly.


Kwa upande wa mahusiano ya kifamilia na ubora au hali ya kuwa Nafsi, mwanadamu na Mungu ni sawa, kwa sababu Mungu aliumba mtu kwa mfano wake mwenyewe. Mungu alimfanya mwanadamu awe sawa na nafsi yake. Lakini katika asili yao, Mungu na mwanadamu hawafanani. Mungu ni mungu na mwanadamu ni wa duniani.


The Trinity doctrine adheres to the understanding that God adjusted Himself to man, and that God merely used the terms ‘\textit{father}’, ‘\textit{son}’ and ‘\textit{spirit}’ so that we might understand Him better. This idea underpins and drives the trinitarian paradigm. In what follows, we will not extensively examine our Trinitarian literature, but will support our claim by a few official statements from the Seventh-day Adventist Church.


Fundisho la Utatu linashikilia ufahamu kwamba Mungu alijirekebisha Mwenyewe kwa mwanadamu, na kwamba Mungu alitumia tu maneno ‘\textit{baba}’, ‘\textit{mwana}’ na ‘\textit{roho}’ ili tuweze kumwelewa bora. Wazo hili linasisitiza na kuendesha dhana ya utatu. Katika kile kinachofuata, hatutafanya kuchunguza kwa kina maandiko yetu ya Utatu, lakini itaunga mkono dai letu na kauli chache rasmi kutoka kwa Kanisa la Waadventista Wasabato.


The first statement comes from the Biblical Research Institute, the official institution of the General Conference, which promotes the teachings and doctrines of the Seventh-day Adventist Church. They openly negate the parental relationship between the Father and His Son, in favour of a metaphorical understanding.


Kauli ya kwanza inatoka kwa Taasisi ya Utafiti wa Kibiblia, taasisi rasmi ya Mkutano Mkuu, ambao unakuza mafundisho na doktrini ya Kanisa la Waadventista. Wanakanusha waziwazi uhusiano wa uzazi kati ya Baba na Mwana wake, kwa kupendelea ufahamu wa sitiari.


\others{The father-son image \textbf{cannot be literally applied to the divine Father-Son relationship} within the Godhead. \textbf{The Son is not the natural, literal Son of the Father} ... \textbf{The term ‘Son’ is used metaphorically} when applied to the Godhead.}[Adventist Biblical Research Institute; also published in the official ‘Adventist World’ magazine][https://www.adventistbiblicalresearch.org/materials/a-question-of-sonship/]


\others{Picha ya baba-mwana \textbf{haiwezi kutumika kihalisi kwa uhusiano wa kimungu wa Baba-Mwana} ndani ya Uungu. \textbf{Mwana si Mwana wa asili, halisi wa Baba} ... \textbf{Neno ‘Mwana’ inatumika kwa njia ya sitiari} inapotumika kwa Uungu.}[Adventist Biblical Research Institute; also published in the official ‘Adventist World’ magazine][https://www.adventistbiblicalresearch.org/materials/a-question-of-sonship/]


Regarding the \emcap{personality of God}, in the context of the trinitarian paradigm, the Seventh-day Adventist church issued the following statements in a Sabbath school lesson:


Kuhusu \emcap{Umbile la Mungu}, katika muktadha wa dhana ya utatu, Kanisa la Waadventista Wasabato lilitoa kauli zifuatazo katika somo la shule ya Sabato:


\others{\textbf{The \underline{word persons} used in the title of today's lesson \underline{must be understood in a theological sense}}. \textbf{If we equate human personality with God, we would say that three persons means three individuals. But then we would have three Gods, or tritheism}. \textbf{But \underline{historic Christianity} has given to the word person, when used of God, \underline{a special meaning}}: a personal self-distinction, which gives distinctiveness in the Persons of the Godhead without destroying the concept of oneness. \textbf{This idea is not easy to grasp or to explain! \underline{It is part of the mystery of the Godhead}}.}[“Lesson 3.” Ssnet.org, 2025, \href{http://www.ssnet.org/qrtrly/eng/98d/less03.html}{www.ssnet.org/qrtrly/eng/98d/less03.html}. Accessed 3 Feb. 2025.]


\others{\textbf{Neno \underline{nafsi} lililotumika katika kichwa cha somo la leo \underline{lazima lieleweke katika maana ya kitheolojia}}. \textbf{Ikiwa tunalinganisha ubinafsi wa mwanadamu na Mungu, tungesema nafsi tatu maana yake ni watu watatu. Lakini basi tungekuwa na Miungu watatu, au imani ya watatu}. \textbf{Lakini \underline{Ukristo wa kihistoria} umetoa kwa neno nafsi, linapotumiwa na Mungu, \underline{maana maalum}}: kujipambanua binafsi, ambayo inatoa upambanuzi katika Nafsi za Uungu bila kuharibu dhana ya umoja. \textbf{Wazo hili si rahisi kufahamu au kueleza! \underline{Ni sehemu ya siri ya Uungu}}.}[“Lesson 3.” Ssnet.org, 2025, \href{http://www.ssnet.org/qrtrly/eng/98d/less03.html}{www.ssnet.org/qrtrly/eng/98d/less03.html}. Accessed 3 Feb. 2025.]


\others{These texts and others lead us to believe that \textbf{our wonderful God is \underline{three Persons in one},} a mind-boggling \textbf{mystery }but a truth we accept by faith because Scripture reveals it.}[Ibid.]


\others{Maandiko haya na mengine yanatufanya tuamini kwamba \textbf{Mungu wetu wa ajabu yuko \underline{Nafsi tatu katika moja},} siri ya \textbf{kustaajabisha akili} lakini ukweli tunaukubali kwa imani kwa sababu Maandiko yanaifichua.}[Ibid.]


According to official statements presented in the Sabbath School Lesson, the word \textit{‘persons’},\textit{ }in regard to God, should not be equated with human personality, but should be applied in the theological sense. This is in sharp contrast to the vision Sister White had regarding the \emcap{personality of God}. \egwinline{‘I have often seen the lovely Jesus, that \textbf{He is a person}. I asked Him if \textbf{His Father was a person}, and \textbf{had \underline{a form} like Himself}. Said Jesus, ‘\textbf{I am the express image of My Father’s person!}’ [Hebrews 1:3.]}[Lt253-1903.12; 1903][https://egwwritings.org/read?panels=p9980.18] Her understanding of the quality or state of God being a person is that God is a person in an obvious way—He possesses a form. In the same way she recognized Jesus to be a person, Jesus testified that God is a person, having a form just as He has. Contrary to the obvious and literal view is a spiritual view. She continues to address the error of the spiritual view. \egwinline{\textbf{I have often seen that \underline{the spiritual view} took away all the glory of heaven, and that in many minds the throne of David and the lovely person of Jesus have been burned up in the fire of spiritualism}. I have seen that some who have been deceived and led into this error, will be brought out into the light of truth, \textbf{but it will be almost impossible for them to get entirely rid of the deceptive power of spiritualism. Such should make thorough work in confessing their errors, and leaving them forever}.}[Lt253-1903.13; 1903][https://egwwritings.org/read?panels=p9980.19] According to the Sabbath School Lesson, the obvious understanding of the term \textit{‘person’ }is incorrect because this would \others{\textbf{equate human personality with God}}, meaning that \others{\textbf{three persons means three individuals}}. Opposite to the obvious view is the theological view. For Sister White, the opposite is the spiritual view. This view takes \egwinline{away all the glory of heaven, and that in many minds the throne of David and the lovely person of Jesus have been burned up in the fire of spiritualism}. In the writings of our pioneers, previously examined, we recognize the truthfulness of her claim. The presented theological view of God’s person does away with the truth on the \emcap{personality of God} that Sister White received in a vision. The theological view is explained as one God, Who is a person, yet three persons, made up of three distinct Gods—God the Father, God the Son, and God the Holy Ghost. The Bible never explains God with such a quality or state of being a person. It is simply presumed by trinitarian believers and, because it is never explained, is deemed a mystery of God, but in fact—it is an error.


Kulingana na taarifa rasmi zilizowasilishwa katika Somo la Shule ya Sabato, neno \textit{‘nafsi’},\textit{ }kuhusiana na Mungu, hawapaswi kulinganishwa na ubinafsi wa kibinadamu, bali linapaswa kutumika katika maana ya kitheolojia. Hii ni tofauti kabisa na maono ambayo Dada White alipata kuhusu \emcap{Umbile la Mungu}. \egwinline{‘Mara nyingi nimemwona Yesu mpendwa, kwamba \textbf{Yeye ni Nafsi}. Nilimwuliza kama \textbf{Baba yake alikuwa Nafsi}, na \textbf{alikuwa na \underline{umbo} kama Yeye}. Yesu akasema, ‘\textbf{Mimi ndiye chapa kamili ya Umbile Wake!}’ [Waebrania 1:3.]}[Lt253-1903.12; 1903][https://egwwritings.org/read?panels=p9980.18] Uelewa wake wa ubora au hali ya Mungu kuwa Nafsi ni kwamba Mungu ni Nafsi kwa njia ya wazi—Anamiliki umbo. Kwa njia sawa na yeye alitambua Yesu kuwa Nafsi, Yesu alishuhudia kwamba Mungu ni Nafsi, mwenye umbo kama vile Yeye alivyo. Kinyume na mtazamo wa dhahiri na halisi ni mtazamo wa kimizimu. Anaendelea kushughulikia makosa ya mtazamo wa kimizimu. \egwinline{\textbf{Mara nyingi nimeona kwamba \underline{mtazamo wa kimizimu} uliondoa utukufu wote wa mbinguni, na kwamba katika akili nyingi kiti cha enzi cha Daudi na ule uzuri wa nafsi ya Yesu umeteketezwa kwa moto wa imani ya mizimu}. Nimeona kwamba baadhi ya wale ambao wamedanganywa na kuongozwa katika kosa hili, watatolewa nje ya nuru ya ukweli, \textbf{lakini itakuwa karibu haiwezekani kwao kuiondoa kabisa nguvu ya udanganyifu ya umizimu. Vile wanapaswa kufanya kazi kamili katika kukiri makosa yao, na kuyaacha milele}.}[Lt253-1903.13; 1903][https://egwwritings.org/read?panels=p9980.19] Kulingana na Somo la Shule ya Sabato, ufahamu dhahiri wa neno \textit{‘nafsi’ }si sahihi kwa sababu hii \others{\textbf{itasawazisha ubinafsi wa binadamu na wa Mungu}}, ikimaanisha kwamba \others{\textbf{nafsi tatu maana yake ni watu watatu}}. Kinyume kwa mtazamo ulio wazi ni mtazamo wa kitheolojia. Kwa Dada White, kinyume chake ni mtazamo wa kimizimu. Mtazamo huu unaondoa \egwinline{utukufu wote wa mbinguni, na kwamba katika akili nyingi kiti cha enzi cha Daudi na ule uzuri wa nafsi ya Yesu umeteketezwa kwa moto wa imani ya mizimu}. Katika maandishi ya waanzilishi wetu, yaliyochunguzwa hapo awali, tunatambua ukweli wa madai yake. Mtazamo wa kitheolojia uliowasilishwa wa Nafsi wa Mungu unaondoa ukweli juu ya \emcap{Umbile la Mungu} ambao Dada White alipokea katika ono. Mtazamo wa kitheolojia unafafanuliwa kama Mungu mmoja, ambaye ni Nafsi, lakini nafsi tatu, zilizofanywa na Mungu watatu tofauti—Mungu Baba, Mungu Mwana, na Mungu Roho Mtakatifu. Biblia haisemi kamwe Mungu kwa ubora au hali kama hiyo kuwa Nafsi. Inakisiwa tu na waamini wa utatu na, kwa sababu haijafafanuliwa kamwe, inachukuliwa kuwa fumbo la Mungu, lakini kwa kweli—ni kosa.


When we draw the line between truth and error, we also need to draw the line between the things that are mystery and those that are revealed. Regarding the nature of God, silence is eloquence. Unfortunately, many who are advocating the Trinity doctrine fail to draw this line in the proper place. We protest that the \emcap{personality of God}, that is the quality or state of God being a person, is a mystery. Our pioneers understood it and they clearly explained it from the Bible. If they did not read and accept the Bible in its plain and simple language, they wouldn’t be able to explain the \emcap{personality of God}.


Tunapochora mstari kati ya ukweli na makosa, tunahitaji pia kuchora mstari kati ya mambo ya siri na yaliyofichuliwa. Kuhusu asili ya Mungu, ukimya ni ufasaha. Kwa bahati mbaya, wengi wanaotetea fundisho la Utatu wanashindwa kuteka mstari huu mahali pazuri. Tunapinga kwamba \emcap{Umbile la Mungu}, yaani ubora au hali ya Mungu kuwa Nafsi, ni fumbo. Waanzilishi wetu waliielewa na waliielezea wazi kutoka Bibilia. Ikiwa hawakusoma na kuikubali Biblia katika lugha yake iliyo wazi na rahisi, wao wasingeweza kueleza \emcap{Umbile la Mungu}.


There are brethren who completely agree with the \emcap{personality of God} laid out in the \emcap{Fundamental Principles}. They agree that the terms ‘\textit{father}’, ‘\textit{son}’ and ‘\textit{spirit}’ should be interpreted by their obvious meaning, yet they continue to advocate the Trinity doctrine because they fail to correctly draw the line between what is being revealed by God and what is not. The argument goes something like this: yes, God is a personal, spiritual being; He does have a body of some sort, Christ is His only begotten Son, and the Holy Spirit is Their representative, but that all applies to our physical universe, which is cumbered by space and time; beyond space and time, God is Trinity.


Wapo ndugu ambao wanakubaliana kabisa na \emcap{Umbile la Mungu} uliowekwa katika \emcap{Kanuni za Msingi}. Wanakubali kwamba maneno ‘\textit{baba}’, ‘\textit{mwana}’ na ‘\textit{roho}’ yanapaswa kuwa ikifasiriwa na maana yao iliyo wazi, hata hivyo wanaendelea kutetea fundisho la Utatu kwa sababu wanashindwa kuchora kwa usahihi mstari kati ya kile kinachofunuliwa na Mungu na ni kipi hakijafunuliwa. Hoja inakwenda hivi: ndiyo, Mungu ni Nafsi binafsi, wa kiroho; Yeye ana mwili wa namna fulani, Kristo ni Mwanawe wa pekee, na Roho Mtakatifu ni mwakilishi Wao, lakini hiyo yote inatumika kwa ulimwengu wetu wa kimwili, ambao umezingirwa na nafasi na wakati; zaidi ya nafasi na wakati, Mungu ni Utatu.


Such a view fails to draw the line between what is revealed and what is a mystery. One consequence of such a conception of God is that it casts doubt on the things which are revealed unto us. To recognize that takes honesty because it is very enticing to conceptualize God beyond space and time, but it is, ultimately, unjustifiable because we are finite and bound to space and time. In his book, the Living Temple, Dr. Kellogg conceptualized God beyond \others{the bounds of space and time}. Dr. Kellogg objected to the conception of God depicted by the \emcap{Fundamental Principles}, because God, in His personality, was bound to His body and thus “\textit{circumscribed}” to one locality, say the temple, or the throne in Heaven\footnote{\href{https://archive.org/details/J.H.Kellogg.TheLivingTemple1903/page/n31/mode/2up}{John H. Kellogg, The Living Temple, p. 31}}. This was unprofitable for Dr. Kellogg, and he advocated that God is far beyond our comprehension as are the bounds of space and time.


Mtazamo kama huo unashindwa kuteka mstari kati ya kile kilichofichuliwa na kile ambacho ni fumbo. Baadhi ya matokeo ya dhana kama hiyo ya Mungu ni kwamba inatia shaka juu ya mambo yaliyopo iliyofunuliwa kwetu. Kutambua hilo kunahitaji uaminifu kwa sababu inavutia sana kufikiria Mungu zaidi ya nafasi na wakati, lakini, hatimaye, isiyo na haki kwa sababu sisi ni wenye ukomo na tumefungwa kwa nafasi na wakati. Katika kitabu chake, the Living Temple, Dk. Kellogg alifikiri dhana ya Mungu zaidi ya \others{mipaka ya nafasi na wakati}. Dk. Kellogg alipinga dhana ya Mungu inayoonyeshwa na \emcap{Kanuni za Msingi}, kwa sababu Mungu, katika utu Wake, alikuwa amefungwa kwa mwili Wake na hivyo “kutahiriwa” katika eneo moja, kama vile hekalu, au kiti cha enzi Mbinguni\footnote{\href{https://archive.org/details/J.H.Kellogg.TheLivingTemple1903/page/n31/mode/2up}{John H. Kellogg, The Living Temple, p. 31}}. Hili halikuwa na faida kwa Dk. Kellogg, na alitetea kwamba Mungu yuko mbali zaidi ya ufahamu wetu kama mipaka ya nafasi na wakati.


\others{\textbf{\underline{Discussions respecting the form of God are utterly unprofitable}, and serve only to belittle our conceptions of him who is above all things}, \textbf{and hence not to be compared in form or size or glory or majesty with anything which man has ever seen or which it is within his power to conceive}. In the presence of questions like these, we have only to acknowledge our foolishness and incapacity, and bow our heads with awe and reverence \textbf{in the presence of a Personality, an Intelligent Being} to the existence of which all nature bears definite and positive testimony, \textbf{but which is as far beyond our comprehension \underline{as are the bounds of space and time}}.}[Ibid, p. 33][https://archive.org/details/J.H.Kellogg.TheLivingTemple1903/page/n33/mode/2up]


\others{\textbf{\underline{Majadiliano yanayohusu umbo la Mungu hayana faida kabisa}, na hutumikia tu kudhalilisha dhana zetu za yeye aliye juu ya vitu vyote}, \textbf{na hivyo asilinganishwe kwa umbo au ukubwa au utukufu au ukuu pamoja na kitu chochote ambacho mwanadamu amewahi kukiona au ambacho kiko ndani ya uwezo wake wa kutafakari}. Katika uwepo wa maswali kama haya, lazima tu tukiri upumbavu wetu na kutoweza, na tuinamishe vichwa vyetu kwa kicho na heshima \textbf{katika uwepo wa Nafsi, Huluki wenye utambuzi} ambayo kwayo asili yote ina ushuhuda wa uhakika na chanya, \textbf{lakini ambao ni mbali zaidi ya ufahamu wetu \underline{kama ni mipaka ya nafasi na wakati}}.}[Ibid, p. 33][https://archive.org/details/J.H.Kellogg.TheLivingTemple1903/page/n33/mode/2up]


Dr. Kellogg was reproved for his conceptions of God. His conception of God was God beyond the bounds of space and time. This conception is problematic because it is beyond the bounds of the Scriptures; it is pure conjecture, which casts doubt on the revelation of the Scripture. If the Scriptures testify that God is a definite, tangible being, being present in one place more than another, then any discussions regarding God being beyond space are utterly unprofitable. Such discussions tend to lead toward skepticism on the very conceptions of God that the Scriptures plainly testify of. As we can recall, this was the main problem with Dr. Kellogg, and Sister White gave us many warnings regarding this issue.


Dk. Kellogg alikaripiwa kwa dhana zake za Mungu. Dhana yake juu ya Mungu ilikuwa ni Mungu nje ya mipaka ya nafasi na wakati. Dhana hii ni tatizo kwa sababu ni zaidi ya mipaka ya Maandiko; ni dhana iliyo makisio haswa, inayotia shaka yaliyofunuliwa kwenye Maandiko. Iwapo Maandiko yanashuhudia kwamba Mungu ni kiumbe dhahiri, kinachoshikika, akiwa katika mahali moja zaidi ya nyingine, basi majadiliano yoyote kuhusu Mungu kuwa nje ya nafasi haina faida kabisa. Majadiliano kama haya huwa yanaongoza kuelekea mashaka sana juu ya dhana za Mungu ambazo Maandiko yanashuhudia waziwazi. Kama tunaweza kukumbuka, hii ilikuwa tatizo kuu na Dk. Kellogg, na Dada White alitupa maonyo mengi kuhusu suala hili.


\egw{‘The secret things belong unto the Lord our God: but those things which are revealed belong unto us and to our children forever.’ Deuteronomy 29:29. \textbf{The revelation of Himself that God has given in His word is for our study}. \textbf{This we may seek to understand}. \textbf{\underline{But beyond this we are not to penetrate}}. \textbf{The highest intellect may tax itself until it is wearied out in \underline{conjectures}\footnote{\href{https://www.merriam-webster.com/dictionary/conjectures}{Merriam Webster Dictionary} - ‘\textit{conjecture}’ - “\textit{a: inference formed without proof or sufficient evidence; b: a conclusion deduced by surmise or guesswork}”} \underline{regarding the nature of God}, but the effort will be fruitless}. \textbf{This problem has not been given us to solve. No human mind can comprehend God.} \textbf{None are to indulge in speculation regarding His nature. Here silence is eloquence. The Omniscient One is above discussion}.}[MH 429.3; 1905][https://egwwritings.org/read?panels=p135.2227]


\egw{‘Mambo ya siri ni ya Bwana, Mungu wetu, lakini yaliyofunuliwa ni yetu kwetu sisi na watoto wetu milele.’ Kumbukumbu la Torati 29:29. \textbf{Ufunuo huo wa Mungu Mwenyewe ambao ametoa katika neno lake ni kwa ajili ya kujifunza kwetu}. \textbf{Hii tunaweza kutafuta kuelewa}. \textbf{\underline{Lakini zaidi ya hii hatupaswi kupenya}}. \textbf{Akili ya juu zaidi inaweza kujitoza hadi itakapokuwa imechoka katika \underline{makisio}\footnote{\href{https://www.merriam-webster.com/dictionary/conjectures}{Merriam Webster Dictionary} - ‘\textit{conjecture}’ - “\textit{a: inference formed without proof or sufficient evidence; b: a conclusion deduced by surmise or guesswork}”} \underline{kuhusu asili ya Mungu}, lakini juhudi itakuwa isiyo na matunda}. \textbf{Tatizo hili hatujapewa kulitatua. Hakuna akili ya mwanadamu inayoweza kumfahamu Mungu.} \textbf{Hakuna anayepaswa kujiingiza katika uvumi kuhusu asili yake. Hapa ukimya ni ufasaha. Mwenye kujua yote yuko juu ya majadiliano}.}[MH 429.3; 1905][https://egwwritings.org/read?panels=p135.2227]


\egw{I say, and have ever said, \textbf{that I will not engage in controversy with any one in regard to \underline{the nature} and personality of God}. \textbf{Let those who try to describe God know that on such a subject silence is eloquence}. \textbf{\underline{Let the Scriptures be read in simple faith, and let each one form his conceptions of God from His inspired Word}}.}[Lt214-1903.9; 1903][https://egwwritings.org/read?panels=p10700.15]


\egw{Ninasema, na nimewahi kusema, \textbf{kwamba sitajihusisha na mabishano na mtu yeyote kuhusiana na \underline{asili} na ubinafsi wa Mungu}. \textbf{Wacha wale wanaojaribu kumwelezea Mungu wajue kwamba ukimya wa somo kama hili ni ufasaha}. \textbf{\underline{Maandiko na yasomwe kwa imani yenye utupu, na kila mmoja aumbe mawazo yake juu ya Mungu kutokana na Neno Lake lililovuviliwa}}.}[Lt214-1903.9; 1903][https://egwwritings.org/read?panels=p10700.15]


\egw{No human mind can comprehend God. No man hath seen Him at any time. We are as ignorant of God as little children. But as little children we may love and obey Him. \textbf{Had this been understood, such sentiments as are in this book would never have been expressed}.}[Lt214-1903.10; 1903][https://egwwritings.org/read?panels=p10700.16]


\egw{Hakuna akili ya mwanadamu inayoweza kumwelewa Mungu. Hakuna mtu aliyemwona wakati wowote. Sisi ni kama wasiomjua Mungu kama watoto wadogo. Lakini kama watoto wadogo tunaweza kumpenda na kumtii. \textbf{Kama hili lingekuwa limeeleweka, hisia kama zilizo katika kitabu hiki hazingewasilishwa kamwe}.}[Lt214-1903.10; 1903][https://egwwritings.org/read?panels=p10700.16]


You might wonder why Sister White said that she will not engage in controversy with anyone concerning the nature and \emcap{personality of God}, while she was heavily engaged in the controversy over the \emcap{personality of God}, and wrote many different testimonies regarding it. Discussions regarding the \emcap{personality of God}, to some degree, touch the nature of God; yet, those regarding the nature of God, in connection to the \emcap{personality of God}, Sister White did not engage in. She knew where to draw the line. She pointed out that the Bible should draw this line for us. \egw{\textbf{\underline{Let the Scriptures be read in simple faith, and let each one form his conceptions of God from His inspired Word.}}} The \emcap{Fundamental Principles} obey this rule. Sister White told us that we must not try to explain in regard to the \emcap{personality of God} any further than the Bible has done.


Unaweza kujiuliza kwa nini Dada White alisema kwamba hatajihusisha na mabishano na mtu yeyote kuhusu asili na \emcap{ubinafsi wa Mungu}, alipokuwa akijishughulisha sana na mabishano juu ya \emcap{ubinafsi wa Mungu}, na kuandika shuhuda nyingi tofauti kuhusu hilo. Majadiliano kuhusu \emcap{ubinafsi wa Mungu}, kwa kiasi fulani, yanagusa asili ya Mungu; lakini, zile zinazohusu asili ya Mungu, kuhusiana na \emcap{ubinafsi wa Mungu}, Dada White alifanya hivyo kutojihusisha. Alijua wapi pa kuchora mstari. Alionyesha kwamba Biblia inapaswa kuchora mstari huu kwa ajili yetu. \egw{\textbf{\underline{Maandiko na yasomwe kwa imani yenye unyofu, na kila mmoja aumbe dhana zake za Mungu kutoka katika Neno Lake lililovuviliwa.}}} \emcap{Kanuni za Msingi} zinatii sheria hii. Dada White alituambia kwamba tusijaribu kueleza kuhusiana na \emcap{ubinafsi wa Mungu} zaidi ya Biblia imeeleza.


\egw{Keep your eyes fixed on the Lord Jesus Christ, and by beholding Him you will be changed into His likeness. \textbf{Talk not of these spiritualistic theories. Let them find no place in your mind.} Let our papers be kept free from everything of the kind. Publish the truth; do not publish error. \textbf{Do not try to explain in regard to the personality of God. \underline{You cannot give any further explanation than the Bible has given}}. \textbf{Human theories regarding Him are good for nothing}. Do not soil your minds by studying the misleading theories of the enemy. Labor to draw minds away from everything of this character. It will be better to keep these subjects out of our papers. Let the doctrines of present truth be put into our papers, but give no room to a repeating of erroneous theories.}[Lt179-1904.4; 1904][https://egwwritings.org/read?panels=p7751.11]


\egw{Kaza macho yako kwa Bwana Yesu Kristo, na kwa kumtazama utabadilishwa kwa mfano wake. \textbf{Usizungumze juu ya nadharia hizi za kimizimu. Yaache yasipate nafasi kwa akili yako.} Wacha karatasi zetu zihifadhiwe kutoka kwa kila kitu cha aina hiyo. Chapisha ukweli; usichapishe makosa. \textbf{Usijaribu kueleza kuhusu ubinafsi wa Mungu. \underline{Huwezi kutoa maelezo yoyote zaidi kuliko jinsi Biblia imetoa}}. \textbf{Nadharia za wanadamu kuhusu Yeye sio muhimu}. Usichafue akili zako kwa kusoma nadharia potofu za adui. Kazi ya kuteka akili mbali na kila kitu cha mhusika huyu. Itakuwa bora kuweka hizi mada kutoka kwenye karatasi zetu. Hebu mafundisho ya ukweli wa sasa yawekwe kwenye karatasi zetu, lakini usitoe nafasi ya kurudia nadharia potovu.}[Lt179-1904.4; 1904][https://egwwritings.org/read?panels=p7751.11]


Let the Bible form our conceptions of God. We cannot give any further explanation of the \emcap{personality of God} than the Bible has given. If the Bible speaks of God that, in His person, He is bound to one locality, like His temple, the sanctuary, and His throne, we should accept that regardless of whether it sounds limiting to God. God is limited in space, in His body, but His presence is not limited, for He is everywhere present by His representative, the Holy Spirit.


Hebu Biblia iunde dhana zetu kuhusu Mungu. Hatuwezi kutoa maelezo zaidi ya \emcap{ubinafsi wa Mungu} kuliko jinsi Biblia imewasilisha. Ikiwa Biblia inazungumza juu ya Mungu kwamba, katika nafsi yake, Yeye amefungwa kwa eneo moja, kama hekalu Lake, patakatifu, na kiti chake cha enzi, tunapaswa kukubali kwamba bila kujali kama inaonekana kuwa na mipaka kwa Mungu. Mungu ana mipaka katika nafasi, katika mwili wake, lakini Uwepo wake hauna kikomo, kwa kuwa yuko kila mahali kupitia kwa mwakilishi wake, Roho Mtakatifu.


The revelation of God does express some limitations of His, and some of them are of a salvational matter. For instance, the Bible clearly says that God is omnipotent (Revelation 19:6), He can do all, yet we find that He could save men by no other means than giving His only begotten Son for us. In the garden of Gethsemane, when God handed the cup of His wrath to His Son, Christ prayed for the possibility that this cup could pass from Him, but ultimately for God's will to be done. Here we see all of the available options the Father had in order to save men. It was not possible to save fallen men, other than for God’s Son to die in their stead. Many protest the idea that something was impossible for God. But if it was possible for God to save men, without His Son drinking the cup of His wrath, surely God would have done it. Some protest this idea of God being limited to only one option of saving men, while He might have infinite options—He is omnipotent, after all. With this thinking, God’s salvation of lost men by the sacrifice of His own begotten Son is enshrouded with doubt, and essentially rejected, even scorned, depicting God as a child murderer. But the revelation is clear in the face of these skeptics. It is not God who is heinous for giving His Son for us; it is sin that is heinous. Sin had demanded this infinite sacrifice to be laid, and there was no other way. That was not roleplay\footnote{The Week of Prayer issue by the Adventist Review, October 31, 1996}, but a reality, that caused infinite grief and suffering to our heavenly Father in giving His own begotten\footnote{Read about God’s gift of His \egwinline{own begotten Son} in \href{https://egwwritings.org/?ref=en_Lt13-1894.18&para=5486.24}{{EGW, Lt13-1894.18; 1894}}}, obedient Son to die in our stead.


Ufunuo wa Mungu unaeleza baadhi ya mapungufu yake, na baadhi yake ni jambo la wokovu. Kwa mfano, Biblia inasema waziwazi kwamba Mungu ni muweza wa yote (Ufunuo 19:6), Anaweza yote, lakini tunaona kwamba hangeweza kuwaokoa wanadamu kwa njia nyingine yoyote isipokuwa kutoa Mwana wake wa pekee kwa ajili yetu. Katika bustani ya Gethsemane, wakati Mungu alipotoa kikombe chake cha ghadhabu kwa Mwanawe, Kristo aliomba uwezekano kwamba kikombe hiki kiweze kupita kutoka kwake, lakini hatimaye ili mapenzi ya Mungu yatimizwe. Hapa tunaona chaguzi zote zinazopatikana ambazo Baba alikuwa nazo ili kuokoa wanadamu. Haikuwezekana kuwaokoa wanadamu walioanguka, isipokuwa Mwana wa Mungu afe kwa niaba yao. Wengi hupinga wazo la kwamba kuna jambo lisilowezekana kwa Mungu. Lakini kama ilikuwa inawezekana kwa Mungu kuwaokoa wanadamu, bila Mwanawe kukinywea kikombe cha ghadhabu yake, hakika Mungu angefanya hivyo. Wengine wanapinga wazo hili la Mungu kuwa na kikomo kwa chaguo moja tu la kuokoa wanadamu, ilhali Anaweza kuwa na chaguzi zisizo na kikomo—Yeye ni muweza wa yote, hata hivyo. Kwa mawazo haya, Wokovu wa Mungu wa watu waliopotea kwa dhabihu ya Mwana wake mwenyewe umefunikwa shaka, na kimsingi kukataliwa, hata dharau, inayoonyesha Mungu kama muuaji wa watoto. Lakini ufunuo uko wazi mbele ya wenye shaka hawa. Si Mungu ambaye ni mwovu kwa kutoa Mwana wake kwa ajili yetu; ni dhambi mbaya. Dhambi ilikuwa imedai dhabihu hii isiyo na mwisho kuwekwa, na hapakuwa na njia nyingine. Hilo halikuwa jukumu la kuigiza\footnote{The Week of Prayer issue by the Adventist Review, October 31, 1996}, lakini ukweli, ambao ulisababisha huzuni isiyo na kikomo na kuteseka kwa Baba yetu wa mbinguni kwa kutoa Mwanawe mzaliwa\footnote{Read about God's gift of His \egwinline{own begotten Son} in \href{https://egwwritings.org/?ref=en_Lt13-1894.18&para=5486.24}{{EGW, Lt13-1894.18; 1894}}}, mtiifu kufa kwa ajili yetu.


Let our conceptions of who God is, what God is, and of what character He is, be molded by plain Scripture, and let us not doubt it.


Hebu dhana zetu za Mungu ni nani, Mungu ni nini, na Yeye ni wa tabia gani, zifinyanwe na Maandiko yaliyo wazi, na tusiwe na shaka nayo.


% The bottom of the issue

\begin{titledpoem}
    \stanza{
        Beside the track of truth, error does tread, \\
        A line so fine, where the Holy Spirit lead. \\
        In words familiar, God's personas blend, \\
        Father and son, where meanings extend.
    }

    \stanza{
        Two views diverge on this sacred script, \\
        One symbolic, the other clearly depicted. \\
        As mirrors of man, in His image cast, \\
        God forms our essence, from the first to the last.
    }

    \stanza{
        Father and Son, in literal hues, \\
        Or metaphors for the divine clues? \\
        Truth's narrow path, so closely lain, \\
        Beside the error, we strive to explain.
    }

    \stanza{
        For through the Bible, meanings unfold, \\
        In God's own language, bold and told. \\
        Not just in symbols, but in our frame, \\
        His likeness, His nature, forever the same.
    }

    \stanza{
        As earthly fathers reflect His ways, \\
        So too our spirit His nature portrays. \\
        In discussions of God, where mysteries thrive, \\
        Let scriptures speak, and in faith we dive.
    }

    \stanza{
        Keep to what's revealed, in the Word abide, \\
        Where human theories and errors collide. \\
        God in His fullness, a mystery remains, \\
        Yet in His Word, His truth sustains.
    }
\end{titledpoem}


% The bottom of the issue

\begin{titledpoem}
    \stanza{
        Beside the track of truth, error does tread, \\
        A line so fine, where the Holy Spirit lead. \\
        In words familiar, God's personas blend, \\
        Father and son, where meanings extend.
    }

    \stanza{
        Two views diverge on this sacred script, \\
        One symbolic, the other clearly depicted. \\
        As mirrors of man, in His image cast, \\
        God forms our essence, from the first to the last.
    }

    \stanza{
        Father and Son, in literal hues, \\
        Or metaphors for the divine clues? \\
        Truth's narrow path, so closely lain, \\
        Beside the error, we strive to explain.
    }

    \stanza{
        For through the Bible, meanings unfold, \\
        In God's own language, bold and told. \\
        Not just in symbols, but in our frame, \\
        His likeness, His nature, forever the same.
    }

    \stanza{
        As earthly fathers reflect His ways, \\
        So too our spirit His nature portrays. \\
        In discussions of God, where mysteries thrive, \\
        Let scriptures speak, and in faith we dive.
    }

    \stanza{
        Keep to what's revealed, in the Word abide, \\
        Where human theories and errors collide. \\
        God in His fullness, a mystery remains, \\
        Yet in His Word, His truth sustains.
    }
\end{titledpoem}
