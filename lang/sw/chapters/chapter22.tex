
\qrchapter{https://forgottenpillar.com/rsc/en-fp-chapter22}{Kiini cha suala hilo}


Leo, tunapolinganisha Imani zetu za Msingi za sasa na \emcap{Kanuni za Msingi} iliyotangulia tunaona mabadiliko katika msingi wa imani ya Waadventista Wasabato. Mabadiliko haya yametukia katika ufahamu wa nafsi ya Mungu, au \emcap{Umbile la Mungu}. Hasa kwa \emcap{Umbile la Mungu}, Dada White aliandika kwamba unyayo wa ukweli upo karibu na unyayo wa kosa:


\egw{\textbf{Njia ya ukweli iko karibu na njia ya makosa}, na nyayo zote mbili \textbf{zinaweza }kuonekana kuwa moja kwa akili ambazo hazifanyiwi kazi na Roho Mtakatifu, na ambazo, kwa hiyo, \textbf{si za kupambanua haraka kati ya kweli na uwongo}.}[SpTB02 52.2; 1904][https://egwwritings.org/read?panels=p417.266]


Tunajiuliza, tunawezaje kuchora mstari wazi kati ya nyayo hizi mbili? Ili kufanya hilo tunahitaji kupata undani wa suala hilo. Tunahitaji kupata kanuni ya kutofautisha na kutenganisha nyayo hizi mbili.


Kwa kusoma imani yetu ya sasa ya Utatu na kazi za waanzilishi wetu kuhusu \emcap{Umbile la Mungu}, tumepata kanuni moja ya sifa inayotofautisha ukweli juu ya \emcap{Umbile la Mungu}, kama unavyoshikiliwa na waanzilishi wetu, kutokana na imani yetu ya sasa ya Utatu. Pande zote wanadai Biblia kuwa mamlaka yao kuu, lakini tofauti zinaweza kutambuliwa na tafsiri ya Biblia. Ifuatayo, tunazungumza juu ya uelewa na kutafsiri Maandiko yanayohusu nafsi ya Mungu. Kuelewa nafsi ya Mungu kunaweza kuwa iliyowasilishwa kwa uelewa mbili tofauti, wa kipekee, ambao huchora mstari wazi kati ya kambi mbili zinazopingana.


Mtazamo mmoja, maarufu zaidi, ni kwamba Mungu alijionyesha katika lugha ambayo tunaizoea ili kueleza dhana tu za wokovu. Kwa hivyo, Mungu alijidhihirisha kwa maneno kama vile ‘\textit{baba}’, ‘\textit{mwana}’, na ‘\textit{roho}’, kuelezea uhusiano kati ya dhana hizi. Hii inafanya maneno haya yasiweze kufasiriwa kwa maana yake dhahiri; badala yake, wanashikilia thamani ya ishara au sitiari. Kanuni iliyo nyuma ya hoja hii ni: \textbf{Mungu alijishusha kwa mwanadamu}.


Mtazamo mwingine, unaopingana, ni kwamba \textbf{Mungu alimrekebisha mwanadamu kwa nafsi Yake}; \textit{Alimuumba mwanadamu katika nafsi ya picha yake}. Kwa hiyo, maneno kama vile ‘\textit{baba}’, ‘\textit{mwana}’, na ‘\textit{roho}’, wanapozungumza na Mungu, yanamaanisha maana dhahiri. Hii ndiyo tofauti ya kimsingi.


Tunapofikia kuelewa maneno ya Biblia kama ‘\textit{nafsi}’, ‘\textit{baba}’, ‘\textit{mwana}’ na ‘\textit{roho}’, ni lazima tuchague ni mtazamo upi tunaounga mkono na tuitumie ipasavyo. Aidha, maneno haya yanaeleweka kidhahiri, au kiishara au kitamathali. Hakuna msingi wa katikati baina ya haya mawili; lazima tuchague moja. Nukuu ifuatayo inapaswa kutatua shida yoyote.


\egw{\textbf{\underline{Lugha ya Biblia inapaswa kufafanuliwa kulingana na maana yake dhahiri, isipokuwa ishara au sura imetumika}}.}[GC 598.3; 1888][https://egwwritings.org/read?panels=p133.2717]

Tunaamini kwamba haiwezekani kwa Biblia kuwa mfasiri yake yenyewe na isieleze alama yake. Ikiwa Biblia inatumia neno ‘baba’ kwa Mungu, lakini haielezi kamwe neno hili, basi inapaswa kukubaliwa katika maana yake dhahiri. Vivyo hivyo kwa maneno ‘mwana’ na ‘roho’. Mwanadamu ameumbwa kwa mfano wa Mungu. Mungu alimrekebisha mwanadamu kwa nafsi yake. Maana ya wazi ni inayotokana na uzoefu wa mwanadamu. Tunaelewa maana ya wazi ya neno ‘baba’ kupitia ubaba wa kawaida, wa kibinadamu. Lakini ubaba wetu ni mfano wa Mungu wetu Aliyeko Baba kwa Mwanawe. Paulo alishuhudia:


\bible{For this cause I bow my knees unto \textbf{the Father of our Lord Jesus Christ, Of whom the whole \underline{family} in heaven and earth is \underline{named}}}[Ephesians 3:14-15].


Katika Kigiriki, neno ‘\textit{familia}’ ni neno ‘\textit{patria}’, linalotoka-na na neno ‘\textit{pater}’, ambalo maana yake ni ‘\textit{baba}’. Baadhi ya tafsiri hata hutafsiri mstari huu kwa \bible{Of whom all \textbf{paternity} in heaven and earth is named} (DRB), ambayo ni tafsiri halisi zaidi. Baba wa Bwana wetu Yesu Kristo kweli ni baba wa Mwanawe, kama vile sisi tulivyo baba kwa watoto wetu hapa Duniani. Ubaba wetu Duniani unaitwa kulingana na Ubaba wa Mbinguni, ambapo Mungu ni Baba ya Bwana wetu Yesu Kristo. Ubaba wetu wa kidunia ni mfano wa Ubaba wa Mbinguni, ambapo Mungu ni Baba kwa Mwanawe. Hii inaunga mkono maana iliyo wazi kwamba Yesu kweli ni Mwana wa Mungu wetu.


Kanuni hiyo hiyo ya msingi inatumika kwa ufahamu nyuma ya neno ‘\textit{roho}’ na neno ‘\textit{kuwa}’. Mungu alimrekebisha mwanadamu kwa nafsi Yake; Alimuumba mwanadamu kwa mfano wake. Mwanadamu ni kuwa, mwenye mwili na roho, kama Mungu tu--na kwa kusema hivi, hatusemi hivyo mwanadamu na Mungu wana asili moja. Mungu aliumba mtu kwa mavumbi ya ardhi. Asili ya kimwili imefungwa kwa vipengele vinavyopatikana duniani. Hatuingii kwenye asili ya Mungu. Hilo litabaki kuwa siri kwetu milele; haikufunuliwa kwetu. Lakini ni nini iliyofunuliwa kwetu ni kwamba ana umbo, na umbo la mwanadamu ni mfano wa umbo la Mungu. Biblia inakubali ufahamu huu kwa uwazi inapomwelezea Mungu aliyeketi kwenye kiti chake cha enzi:


\bible{\textbf{juu ya mfano wa kile kiti cha enzi kulikuwa na sura kama \underline{kuonekana kwa mtu}} juu  yake}[Ezekieli 1:26].


Maana ya wazi ya neno ‘\textit{roho}’, linalotumika kwa Roho wa Mungu, linatokana na ufahamu wa “\textit{roho ya mwanadamu}”. Mungu alimrekebisha mwanadamu kwa nafsi Yake; Alimuumba mwanadamu katika picha yake mwenyewe. Kama vile mwanadamu alivyo na roho, Mungu ana Roho. Roho ya mwanadamu ina asili ya mwanadamu, na roho ya Mungu ina asili ya Mungu. Kuhusiana na asili yao, si sawa, lakini kwa kuzingatia uhusiano wao na utu wao wa ndani, wao ni sawa; Biblia inawaweka katika kiwango sawa. \bible{\textbf{Roho mwenyewe hushuhudia pamoja na \underline{roho zetu}}, kwamba sisi tu watoto wa Mungu:}[Warumi 8:16]; \bible{Kwa maana ni nani \textbf{ ayajuaye mambo ya mwanadamu}, ila \textbf{\underline{roho ya mwanadamu} iliyo ndani yake}? \textbf{\underline{vivyo hivyo}} na mambo ya \textbf{Mungu hakuna ajuaye}, \textbf{ila \underline{Roho wa Mungu}}.}[1 Wakorintho 2:11].


Kwa upande wa mahusiano ya kifamilia na ubora au hali ya kuwa Nafsi, mwanadamu na Mungu ni sawa, kwa sababu Mungu aliumba mtu kwa mfano wake mwenyewe. Mungu alimfanya mwanadamu awe sawa na nafsi yake. Lakini katika asili yao, Mungu na mwanadamu hawafanani. Mungu ni mungu na mwanadamu ni wa duniani.


Fundisho la Utatu linashikilia ufahamu kwamba Mungu alijirekebisha Mwenyewe kwa mwanadamu, na kwamba Mungu alitumia tu maneno ‘\textit{baba}’, ‘\textit{mwana}’ na ‘\textit{roho}’ ili tuweze kumwelewa bora. Wazo hili linasisitiza na kuendesha dhana ya utatu. Katika kile kinachofuata, hatutafanya kuchunguza kwa kina maandiko yetu ya Utatu, lakini tutaunga mkono dai letu kwa kauli chache rasmi kutoka kwa Kanisa la Waadventista Wasabato.


Kauli ya kwanza inatoka kwa Taasisi ya Utafiti wa Kibiblia, taasisi rasmi ya Mkutano Mkuu, ambao unakuza mafundisho na doktrini ya Kanisa la Waadventista. Wanakanusha waziwazi uhusiano wa uzazi kati ya Baba na Mwana wake, kwa kupendelea ufahamu wa sitiari.


\others{Picha ya baba-mwana \textbf{haiwezi kutumika kihalisi kwa uhusiano wa kimungu wa Baba-Mwana} ndani ya Uungu. \textbf{Mwana si Mwana wa asili na halisi wa Baba} ... \textbf{Neno ‘Mwana’ inatumika kwa njia ya sitiari} inapotumika kwa Uungu.}[Adventist Biblical Research Institute; also published in the official ‘Adventist World’ magazine][https://www.adventistbiblicalresearch.org/materials/a-question-of-sonship/]


Kuhusu \emcap{Umbile la Mungu}, katika muktadha wa dhana ya utatu, Kanisa la Waadventista Wasabato lilitoa kauli zifuatazo katika somo la shule ya Sabato:


\others{\textbf{Neno \underline{nafsi} lililotumika katika kichwa cha somo la leo \underline{lazima lieleweke katika maana ya kitheolojia}}. \textbf{Ikiwa tunalinganisha ubinafsi wa mwanadamu na Mungu, tungesema nafsi tatu maana yake ni watu watatu. Lakini basi tungekuwa na Miungu watatu, au imani ya watatu}. \textbf{Lakini \underline{Ukristo wa kihistoria} umetoa kwa neno nafsi, linapotumiwa na Mungu, \underline{maana maalum}}: kujipambanua binafsi, ambayo inatoa upambanuzi katika Nafsi za Uungu bila kuharibu dhana ya umoja. \textbf{Wazo hili si rahisi kufahamu au kueleza! \underline{Ni sehemu ya siri ya Uungu}}.}[“Lesson 3.” Ssnet.org, 2025, \href{http://www.ssnet.org/qrtrly/eng/98d/less03.html}{www.ssnet.org/qrtrly/eng/98d/less03.html}. Accessed 3 Feb. 2025.]


\others{Maandiko haya na mengine yanatufanya tuamini kwamba \textbf{Mungu wetu wa ajabu yuko \underline{Nafsi tatu katika moja},} siri ya \textbf{kustaajabisha akili} lakini ukweli tunaukubali kwa imani kwa sababu Maandiko yanaifichua.}[Ibid.]


Kulingana na taarifa rasmi zilizowasilishwa katika Somo la Shule ya Sabato, neno \textit{‘nafsi’},\textit{ }kuhusiana na Mungu, hawapaswi kulinganishwa na ubinafsi wa kibinadamu, bali linapaswa kutumika katika maana ya kitheolojia. Hii ni tofauti kabisa na maono ambayo Dada White alipata kuhusu \emcap{Umbile la Mungu}. \egwinline{‘Mara nyingi nimemwona Yesu mpendwa, kwamba \textbf{Yeye ni Nafsi}. Nilimwuliza kama \textbf{Baba yake alikuwa Nafsi}, na \textbf{alikuwa na \underline{umbo} kama Yeye}. Yesu akasema, ‘\textbf{Mimi ndiye chapa kamili ya Umbile Wake!}’ [Waebrania 1:3.]}[Lt253-1903.12; 1903][https://egwwritings.org/read?panels=p9980.18] Uelewa wake wa ubora au hali ya Mungu kuwa Nafsi ni kwamba Mungu ni Nafsi kwa njia ya wazi—Anamiliki umbo. Kwa njia sawa na yeye alitambua Yesu kuwa Nafsi, Yesu alishuhudia kwamba Mungu ni Nafsi, mwenye umbo kama vile Yeye alivyo. Kinyume na mtazamo wa dhahiri na halisi ni mtazamo wa kimizimu. Anaendelea kushughulikia makosa ya mtazamo wa kimizimu. \egwinline{\textbf{Mara nyingi nimeona kwamba \underline{mtazamo wa kimizimu} uliondoa utukufu wote wa mbinguni, na kwamba katika akili nyingi kiti cha enzi cha Daudi na ule uzuri wa nafsi ya Yesu umeteketezwa kwa moto wa imani ya mizimu}. Nimeona kwamba baadhi ya wale ambao wamedanganywa na kuongozwa katika kosa hili, watatolewa nje ya nuru ya ukweli, \textbf{lakini itakuwa karibu haiwezekani kwao kuiondoa kabisa nguvu ya udanganyifu ya umizimu. Vile wanapaswa kufanya kazi kamili katika kukiri makosa yao, na kuyaacha milele}.}[Lt253-1903.13; 1903][https://egwwritings.org/read?panels=p9980.19] Kulingana na Somo la Shule ya Sabato, ufahamu dhahiri wa neno \textit{‘nafsi’ }si sahihi kwa sababu hii \others{\textbf{itasawazisha ubinafsi wa binadamu na wa Mungu}}, ikimaanisha kwamba \others{\textbf{nafsi tatu maana yake ni watu watatu}}. Kinyume kwa mtazamo ulio wazi ni mtazamo wa kitheolojia. Kwa Dada White, kinyume chake ni mtazamo wa kimizimu. Mtazamo huu unaondoa \egwinline{utukufu wote wa mbinguni, na kwamba katika akili nyingi kiti cha enzi cha Daudi na ule uzuri wa nafsi ya Yesu umeteketezwa kwa moto wa imani ya mizimu}. Katika maandishi ya waanzilishi wetu, yaliyochunguzwa hapo awali, tunatambua ukweli wa madai yake. Mtazamo wa kitheolojia uliowasilishwa wa Nafsi wa Mungu unaondoa ukweli juu ya \emcap{Umbile la Mungu} ambao Dada White alipokea katika ono. Mtazamo wa kitheolojia unafafanuliwa kama Mungu mmoja, ambaye ni Nafsi, lakini nafsi tatu, zilizofanywa na Mungu watatu tofauti—Mungu Baba, Mungu Mwana, na Mungu Roho Mtakatifu. Biblia haisemi kamwe Mungu kwa ubora au hali kama hiyo kuwa Nafsi. Inakisiwa tu na waamini wa utatu na, kwa sababu haijafafanuliwa kamwe, inachukuliwa kuwa fumbo la Mungu, lakini kwa kweli—ni kosa.


Tunapochora mstari kati ya ukweli na makosa, tunahitaji pia kuchora mstari kati ya mambo ya siri na yaliyofichuliwa. Kuhusu asili ya Mungu, ukimya ni ufasaha. Kwa bahati mbaya, wengi wanaotetea fundisho la Utatu wanashindwa kuteka mstari huu mahali pazuri. Tunapinga kwamba \emcap{Umbile la Mungu}, yaani ubora au hali ya Mungu kuwa Nafsi, ni fumbo. Waanzilishi wetu waliielewa na waliielezea wazi kutoka Bibilia. Ikiwa hawakusoma na kuikubali Biblia katika lugha yake iliyo wazi na rahisi, wao wasingeweza kueleza \emcap{Umbile la Mungu}.


Wapo ndugu ambao wanakubaliana kabisa na \emcap{Umbile la Mungu} uliowekwa katika \emcap{Kanuni za Msingi}. Wanakubali kwamba maneno ‘\textit{baba}’, ‘\textit{mwana}’ na ‘\textit{roho}’ yanapaswa kuwa ikifasiriwa na maana yao iliyo wazi, hata hivyo wanaendelea kutetea fundisho la Utatu kwa sababu wanashindwa kuchora kwa usahihi mstari kati ya kile kinachofunuliwa na Mungu na ni kipi hakijafunuliwa. Hoja inakwenda hivi: ndiyo, Mungu ni Nafsi binafsi, wa kiroho; Yeye ana mwili wa namna fulani, Kristo ni Mwanawe wa pekee, na Roho Mtakatifu ni mwakilishi Wao, lakini hiyo yote inatumika kwa ulimwengu wetu wa kimwili, ambao umezingirwa na nafasi na wakati; zaidi ya nafasi na wakati, Mungu ni Utatu.


Mtazamo kama huo unashindwa kuteka mstari kati ya kile kilichofichuliwa na kile ambacho ni fumbo. Baadhi ya matokeo ya dhana kama hiyo ya Mungu ni kwamba inatia shaka juu ya mambo yaliyopo yaliyofunuliwa kwetu. Kutambua hilo kunahitaji uaminifu kwa sababu inavutia sana kufikiria Mungu zaidi ya nafasi na wakati, lakini, hatimaye, isiyo na haki kwa sababu sisi ni wenye ukomo na tumefungwa kwa nafasi na wakati. Katika kitabu chake, the Living Temple, Dk. Kellogg alifikiri dhana ya Mungu zaidi ya \others{mipaka ya nafasi na wakati}. Dk. Kellogg alipinga dhana ya Mungu inayoonyeshwa na \emcap{Kanuni za Msingi}, kwa sababu Mungu, katika utu Wake, alikuwa amefungwa kwa mwili Wake na hivyo “kuzuiliwa” katika eneo moja, kama vile hekalu, au kiti cha enzi Mbinguni\footnote{\href{https://archive.org/details/J.H.Kellogg.TheLivingTemple1903/page/n31/mode/2up}{John H. Kellogg, The Living Temple, p. 31}}. Hili halikuwa na faida kwa Dk. Kellogg, na alitetea kwamba Mungu yuko mbali zaidi ya ufahamu wetu kama mipaka ya nafasi na wakati.


\others{\textbf{\underline{Majadiliano yanayohusu umbo la Mungu hayana faida kabisa}, na hutumikia tu kudhalilisha dhana zetu za yeye aliye juu ya vitu vyote}, \textbf{na hivyo asilinganishwe kwa umbo au ukubwa au utukufu au ukuu pamoja na kitu chochote ambacho mwanadamu amewahi kukiona au ambacho kiko ndani ya uwezo wake wa kutafakari}. Katika uwepo wa maswali kama haya, lazima tu tukiri upumbavu wetu na kutoweza, na tuinamishe vichwa vyetu kwa kicho na heshima \textbf{katika uwepo wa Nafsi, Huluki wenye utambuzi} ambayo kwayo asili yote ina ushuhuda wa uhakika na chanya, \textbf{lakini ambao ni mbali zaidi ya ufahamu wetu \underline{kama ni mipaka ya nafasi na wakati}}.}[Ibid, p. 33][https://archive.org/details/J.H.Kellogg.TheLivingTemple1903/page/n33/mode/2up]


Dk. Kellogg alikaripiwa kwa dhana zake za Mungu. Dhana yake juu ya Mungu ilikuwa ni Mungu nje ya mipaka ya nafasi na wakati. Dhana hii ni tatizo kwa sababu ni zaidi ya mipaka ya Maandiko; ni dhana iliyo makisio haswa, inayotia shaka yaliyofunuliwa kwenye Maandiko. Iwapo Maandiko yanashuhudia kwamba Mungu ni kiumbe dhahiri, kinachoshikika, akiwa katika mahali moja zaidi ya nyingine, basi majadiliano yoyote kuhusu Mungu kuwa nje ya nafasi haina faida kabisa. Majadiliano kama haya huwa yanaongoza kuelekea mashaka sana juu ya dhana za Mungu ambazo Maandiko yanashuhudia waziwazi. Kama tunaweza kukumbuka, hii ilikuwa tatizo kuu na Dk. Kellogg, na Dada White alitupa maonyo mengi kuhusu suala hili.


\egw{‘Mambo ya siri ni ya Bwana, Mungu wetu, lakini yaliyofunuliwa ni yetu kwetu sisi na watoto wetu milele.’ Kumbukumbu la Torati 29:29. \textbf{Ufunuo huo wa Mungu Mwenyewe ambao ametoa katika neno lake ni kwa ajili ya kujifunza kwetu}. \textbf{Hii tunaweza kutafuta kuelewa}. \textbf{\underline{Lakini zaidi ya hii hatupaswi kupenya}}. \textbf{Akili ya juu zaidi inaweza kujitoza hadi itakapokuwa imechoka katika \underline{makisio}\footnote{\href{https://www.merriam-webster.com/dictionary/conjectures}{Merriam Webster Dictionary} - ‘\textit{conjecture}’ - “\textit{a: inference formed without proof or sufficient evidence; b: a conclusion deduced by surmise or guesswork}”} \underline{kuhusu asili ya Mungu}, lakini juhudi itakuwa isiyo na matunda}. \textbf{Tatizo hili hatujapewa kulitatua. Hakuna akili ya mwanadamu inayoweza kumfahamu Mungu.} \textbf{Hakuna anayepaswa kujiingiza katika uvumi kuhusu asili yake. Hapa ukimya ni ufasaha. Mwenye kujua yote yuko juu ya majadiliano}.}[MH 429.3; 1905][https://egwwritings.org/read?panels=p135.2227]


\egw{Ninasema, na nimewahi kusema, \textbf{kwamba sitajihusisha na mabishano na mtu yeyote kuhusiana na \underline{asili} na ubinafsi wa Mungu}. \textbf{Wacha wale wanaojaribu kumwelezea Mungu wajue kwamba ukimya wa somo kama hili ni ufasaha}. \textbf{\underline{Maandiko na yasomwe kwa imani yenye utupu, na kila mmoja aumbe mawazo yake juu ya Mungu kutokana na Neno Lake lililovuviliwa}}.}[Lt214-1903.9; 1903][https://egwwritings.org/read?panels=p10700.15]


\egw{Hakuna akili ya mwanadamu inayoweza kumwelewa Mungu. Hakuna mtu aliyemwona wakati wowote. Sisi ni kama wasiomjua Mungu kama watoto wadogo. Lakini kama watoto wadogo tunaweza kumpenda na kumtii. \textbf{Kama hili lingekuwa limeeleweka, hisia kama zilizo katika kitabu hiki hazingewasilishwa kamwe}.}[Lt214-1903.10; 1903][https://egwwritings.org/read?panels=p10700.16]


Unaweza kujiuliza kwa nini Dada White alisema kwamba hatajihusisha na mabishano na mtu yeyote kuhusu asili na \emcap{ubinafsi wa Mungu}, alipokuwa akijishughulisha sana na mabishano juu ya \emcap{ubinafsi wa Mungu}, na kuandika shuhuda nyingi tofauti kuhusu hilo. Majadiliano kuhusu \emcap{ubinafsi wa Mungu}, kwa kiasi fulani, yanagusa asili ya Mungu; lakini, zile zinazohusu asili ya Mungu, kuhusiana na \emcap{ubinafsi wa Mungu}, Dada White alifanya hivyo kutojihusisha. Alijua wapi pa kuchora mstari. Alionyesha kwamba Biblia inapaswa kuchora mstari huu kwa ajili yetu. \egw{\textbf{\underline{Maandiko na yasomwe kwa imani yenye unyofu, na kila mmoja aumbe dhana zake za Mungu kutoka katika Neno Lake lililovuviliwa.}}} \emcap{Kanuni za Msingi} zinatii sheria hii. Dada White alituambia kwamba tusijaribu kueleza kuhusiana na \emcap{ubinafsi wa Mungu} zaidi ya Biblia imeeleza.


\egw{Kaza macho yako kwa Bwana Yesu Kristo, na kwa kumtazama utabadilishwa kwa mfano wake. \textbf{Usizungumze juu ya nadharia hizi za kimizimu. Yaache yasipate nafasi kwa akili yako.} Wacha karatasi zetu zihifadhiwe kutoka kwa kila kitu cha aina hiyo. Chapisha ukweli; usichapishe makosa. \textbf{Usijaribu kueleza kuhusu ubinafsi wa Mungu. \underline{Huwezi kutoa maelezo yoyote zaidi kuliko jinsi Biblia imetoa}}. \textbf{Nadharia za wanadamu kuhusu Yeye sio muhimu}. Usichafue akili zako kwa kusoma nadharia potofu za adui. Kazi ya kuteka akili mbali na kila kitu cha mhusika huyu. Itakuwa bora kuweka hizi mada kutoka kwenye karatasi zetu. Hebu mafundisho ya ukweli wa sasa yawekwe kwenye karatasi zetu, lakini usitoe nafasi ya kurudia nadharia potovu.}[Lt179-1904.4; 1904][https://egwwritings.org/read?panels=p7751.11]


Hebu Biblia iunde dhana zetu kuhusu Mungu. Hatuwezi kutoa maelezo zaidi ya \emcap{ubinafsi wa Mungu} kuliko jinsi Biblia imewasilisha. Ikiwa Biblia inazungumza juu ya Mungu kwamba, katika nafsi yake, Yeye amefungwa kwa eneo moja, kama hekalu Lake, patakatifu, na kiti chake cha enzi, tunapaswa kukubali kwamba bila kujali kama inaonekana kuwa na mipaka kwa Mungu. Mungu ana mipaka katika nafasi, katika mwili wake, lakini Uwepo wake hauna kikomo, kwa kuwa yuko kila mahali kupitia kwa mwakilishi wake, Roho Mtakatifu.


Ufunuo wa Mungu unaeleza baadhi ya mapungufu yake, na baadhi yake ni jambo la wokovu. Kwa mfano, Biblia inasema waziwazi kwamba Mungu ni muweza wa yote (Ufunuo 19:6), Anaweza yote, lakini tunaona kwamba hangeweza kuwaokoa wanadamu kwa njia nyingine yoyote isipokuwa kutoa Mwana wake wa pekee kwa ajili yetu. Katika bustani ya Gethsemane, wakati Mungu alipotoa kikombe chake cha ghadhabu kwa Mwanawe, Kristo aliomba uwezekano kwamba kikombe hiki kiweze kupita kutoka kwake, lakini hatimaye ili mapenzi ya Mungu yatimizwe. Hapa tunaona chaguzi zote zinazopatikana ambazo Baba alikuwa nazo ili kuokoa wanadamu. Haikuwezekana kuwaokoa wanadamu walioanguka, isipokuwa Mwana wa Mungu afe kwa niaba yao. Wengi hupinga wazo la kwamba kuna jambo lisilowezekana kwa Mungu. Lakini kama ilikuwa inawezekana kwa Mungu kuwaokoa wanadamu, bila Mwanawe kukinywea kikombe cha ghadhabu yake, hakika Mungu angefanya hivyo. Wengine wanapinga wazo hili la Mungu kuwa na kikomo kwa chaguo moja tu la kuokoa wanadamu, ilhali Anaweza kuwa na chaguzi zisizo na kikomo—Yeye ni muweza wa yote, hata hivyo. Kwa mawazo haya, Wokovu wa Mungu wa watu waliopotea kwa dhabihu ya Mwana wake mwenyewe umefunikwa shaka, na kimsingi kukataliwa, hata dharau, inayoonyesha Mungu kama muuaji wa watoto. Lakini ufunuo uko wazi mbele ya wenye shaka hawa. Si Mungu ambaye ni mwovu kwa kutoa Mwana wake kwa ajili yetu; ni dhambi mbaya. Dhambi ilikuwa imedai dhabihu hii isiyo na mwisho kuwekwa, na hapakuwa na njia nyingine. Hilo halikuwa jukumu la kuigiza\footnote{The Week of Prayer issue by the Adventist Review, October 31, 1996}, lakini ukweli, ambao ulisababisha huzuni isiyo na kikomo na kuteseka kwa Baba yetu wa mbinguni kwa kutoa Mwanawe mzaliwa\footnote{Read about God's gift of His \egwinline{own begotten Son} in \href{https://egwwritings.org/?ref=en_Lt13-1894.18&para=5486.24}{{EGW, Lt13-1894.18; 1894}}}, mtiifu kufa kwa ajili yetu.


Hebu dhana zetu za Mungu ni nani, Mungu ni nini, na Yeye ni wa tabia gani, zifinyanwe na Maandiko yaliyo wazi, na tusiwe na shaka nayo.




% The bottom of the issue

\begin{titledpoem}
    \stanza{
        Beside the track of truth, error does tread, \\
        A line so fine, where the Holy Spirit lead. \\
        In words familiar, God's personas blend, \\
        Father and son, where meanings extend.
    }

    \stanza{
        Two views diverge on this sacred script, \\
        One symbolic, the other clearly depicted. \\
        As mirrors of man, in His image cast, \\
        God forms our essence, from the first to the last.
    }

    \stanza{
        Father and Son, in literal hues, \\
        Or metaphors for the divine clues? \\
        Truth's narrow path, so closely lain, \\
        Beside the error, we strive to explain.
    }

    \stanza{
        For through the Bible, meanings unfold, \\
        In God's own language, bold and told. \\
        Not just in symbols, but in our frame, \\
        His likeness, His nature, forever the same.
    }

    \stanza{
        As earthly fathers reflect His ways, \\
        So too our spirit His nature portrays. \\
        In discussions of God, where mysteries thrive, \\
        Let scriptures speak, and in faith we dive.
    }

    \stanza{
        Keep to what's revealed, in the Word abide, \\
        Where human theories and errors collide. \\
        God in His fullness, a mystery remains, \\
        Yet in His Word, His truth sustains.
    }
\end{titledpoem}
