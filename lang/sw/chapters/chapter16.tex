
\qrchapter{https://forgottenpillar.com/rsc/en-fp-chapter16}{Dk. Kellogg na pantheism}

Kutoka katika shajara yake ya kibinafsi, Januari 5, 1902, Dada White aliandika kwamba \egwinline{sayansi ya Kellogg ya Mungu katika asili ni \textbf{kweli}}.

\egw{Nina vitu ambavyo viliwasilishwa kwangu vinavyosumbua akili yangu. Dk. Kellogg anasafiri barabara ambayo alikwisha safiri mara baada ya kuchukua majukumu yake katika Sanitorium. \textbf{Sayansi ya kibinadamu ni uwongo kuhusu Mungu kutokuwa na Umbile}. Najua huu ni uwongo, na bado ikiwa sisi tunaweza kumsaidia daktari kwa njia yoyote lazima tujaribu kufanya hivyo. Nini kinaweza kusemwa? Kuna kuinuliwa amepewa hivi kwamba anakaribia kupinduka juu ya genge hilo. Yeyote kati yetu anaweza kufanya nini? Bwana peke yake anaweza kumwokoa Dk. Kellogg. \textbf{\underline{Sayansi yake ya Mungu katika maumbile ni ya kweli}}, lakini ameweka asili mahali ambapo Mungu anapaswa kuwa. Asili sio Mungu, lakini Mungu aliumba asili. \textbf{\underline{Hii sayansi ya Mungu katika maumbile ni sahihi kwa njia moja}}. \textbf{Mungu huipa asili uhai wake, vipengele vya kuhuisha, uzuri wake}. [Yeye] ndiye mwanzilishi wa uzuri wa asili yote, na huku Yeye akitupa ushahidi huu wa nguvu kuu, \textbf{Yeye ni Mungu binafsi na Kristo ni Mwokozi binafsi}.}[Ms236-1902.1; 1902][https://egwwritings.org/read?panels=p12779.6]

\egwnogap{\textbf{Hatuchukui makosa ya mwanadamu bali Neno la Mungu ambalo mwanadamu aliumbwa baada ya mfano wa Mungu na Kristo}, kwa maana Neno hutangaza ‘Mungu, ambaye alisema zamani na baba zetu katika manabii kwa sehemu nyingi na kwa njia nyingi, mwisho wa siku hizi amesema na sisi katika Mwana, aliyemweka kuwa mrithi wa yote, \textbf{tena kwa yeye aliufanya ulimwengu; Yeye kwa kuwa ni mng'ao wa utukufu wake, na \underline{chapa kamili ya Umbile Wake}}, akivichukua vyote kwa amri ya uweza wake, akiisha kufanya utakaso wa dhambi, \textbf{aliketi mkono wa kuume wa Ukuu huko juu}.’ Waebrania 1:1-3.}[Ms236-1902.4; 1902][https://egwwritings.org/read?panels=p12779.9]

Inafurahisha, Dada White pia alidai kwamba Mungu yuko katika asili, na Anatoa uhai na vitu vya uhai. Kellogg yuko sahihi kwa hoja hii na madai yake hakika yanaungwa mkono na maandishi ya dada White. Kulingana na hoja hii, Kellogg alijitetea, akisema kuwa The Living Temple inapatana na maandishi ya Dada White. Alimwandikia kaka G. I. Butler mahali ambapo Dada White alitetea hisia sawa na yeye.

\others{Dada White amechukua msimamo sawa kwa kuzingatia jambo hili ambalo nimeshikilia pia. Utaipata, katika kazi yake ndogo ya \textbf{Elimu }katika sura za ‘\textbf{Mungu katika Asili}’ na ‘\textbf{Sayansi na Biblia.}’ Utayapata yote kupitia ‘\textbf{Tumaini la vizazi,}’ na ‘\textbf{Mababa na Manabii.}’}[Letter from Dr. Kellogg to Eld. Butler, February 21, 1904]

Acheni tuchunguze “\textit{Mungu katika Asili}”, katika kitabu Elimu, ambapo tunaweza kupata hisia sawa kuhusu Mungu katika Asili ambayo Kellogg alikuza.

\egw{\textbf{Juu ya vitu vyote vilivyoumbwa huonekana mwonekano wa Uungu}. Asili humshuhudia Mungu. Akili nyeti, ilkiletwa pamoja na muujiza na siri ya ulimwengu, haiwezi lakini kutambua \textbf{utendaji kazi wa nguvu isiyo na kikomo}. \textbf{\underline{Dunia haifanyi kwa nishati yake yenyewe} kuzalisha fadhila zake}, na mwaka baada ya mwaka kuendelea na mwendo wake kulizunguka jua. \textbf{Mkono usioonekani huongoza sayari katika mzunguko wao wa mbingu}. \textbf{\underline{Uhai wa ajabu yanaenea popote kwenye asili—uhai yanayotegemeza malimwengu yasiyohesabika katika ukubwa wote}}, \textbf{yanaoishi katika atomi ya wadudu ambayo huelea katika upepo wa kiangazi, ambayo huweka mbawa za ndege kumeza na kuwalisha kunguru wachanga wanaolia, ambayo huleta chipukizi kuchanua na ua kwa matunda}.}[Ed 99.1; 1903][https://egwwritings.org/read?panels=p29.470]

\egwnogap{\textbf{Nguvu ile ile inayoshikilia asili, inafanya kazi pia ndani ya mwanadamu}. \textbf{Sheria kubwa sawa ambayo inaongoza  nyota na atomi hudhibiti maisha ya mwanadamu}. \textbf{Sheria zinazoongoza tendo la moyo, linalodhibiti mtiririko wa mkondo wa maisha kuelekea mwilini, ni sheria za Akili yenye nguvu ambayo ina mamlaka ya nafsi}. \textbf{\underline{Kutoka Kwake uhai wote hutoka}}. Ni katika kupatana Naye pekee ndipo panaweza kupatikana nyanja yake ya kweli ya utendaji. Kwa vitu vyake vyote vya uumbaji hali ni ile ile—\textbf{maisha yanayodumishwa kwa kupokea uzima wa Mungu}, uzima unaotekelezwa kupatana na mapenzi ya Muumba...}[Ed 99.2; 1903][https://egwwritings.org/read?panels=p29.471]

\egw{...Moyo ambao bado haujawa mgumu kutoka na kugusana na uovu ni wepesi \textbf{kutambua \underline{Uwepo} huo unaozunguka vitu vyote vilivyoumbwa}...}[Ed 100.2; 1903][https://egwwritings.org/read?panels=p29.475]

Katika utetezi wake, Kellogg pia alikuwa akirejelea Mababa na Manabii. Hapo tunasoma zifuatazo:

\egw{Wengi hufundisha kwamba asili hii ina nguvu muhimu,—kwamba sifa fulani hugawiwa asili, na kisha inaachwa kutenda kupitia nishati yake ya asili; na kwamba shughuli za asili zinaendeshwa kwa kupatana na sheria zilizowekwa, ambazo Mungu mwenyewe hawezi kuziingilia. \textbf{Hii ni sayansi ya uongo, na haiungwi mkono na neno la Mungu}. Asili ni mtumishi wa Muumba wake. Mungu hazibatilishi sheria zake, au kufanya kazi kinyume nazo; \textbf{lakini yeye ni daima huzitumia kama vyombo vyake. Asili inashuhudia akili, \underline{uwepo}, \underline{na nishati hai}, ambayo inafanya kazi ndanimo na kupitia sheria zake. Kuna katika asili  uendelezaji wa kazi ya \underline{Baba na Mwana}.} Kristo anasema, ‘Baba yangu anafanya kazi hata sasa, nami ninafanya kazi.’ Yohana 5:17.}[PP 114.4; 1980][https://egwwritings.org/read?panels=p84.445]

Manukuu haya yanapatana na manukuu kutoka kwa Hekalu Hai.

\others{Maonyesho ya maisha ni  tofauti jinsi wanyama na mimea ni tofauti, na sehemu za vitu vilivyohuishwa. Kila jani, kila kijani cha nyasi, kila ua, kila ndege, hata kila mdudu, pamoja na kila mnyama au kila mti, unashuhudia utofauti usio na kikomo na raslimali zisizokwisha za \textbf{Uhai ulioenea kote, wa uumbaji wote, wenye kustahimili kila kitu}.}[John H. Kellogg, The Living Temple p. 16][https://archive.org/details/J.H.Kellogg.TheLivingTemple1903/page/n15/]

\others{Akili ni mojawapo ya nguvu za ulimwengu, moja ya maonyesho ya \textbf{\underline{uhai unaoenea kote ambao} uliumba na kuumba, \underline{wenye kuhuisha na kudumisha}}.}[John H. Kellogg, The Living Temple p. 396][https://archive.org/details/J.H.Kellogg.TheLivingTemple1903/page/n425/]

Ikiwa ufahamu wa Kellogg kuhusu Mungu kama chanzo kinachodumisha na kuhuisha asili ni sahihi, basi kosa lake liko wapi? Kwa nini anaitwa mpantheisti? Je, ni haki kumwita mpantheisti? Yeye hakika haifikirii hivyo. Tazama alichoandika kwa Mzee Butler:

\others{\textbf{Ninachukia uholandi} kama wewe. \textbf{Nimejitahidi katika kitabu changu kufundisha tu ukweli kwamba mwanadamu anamtegemea Mungu kwa kila kitu, na kwamba bila nguvu ya uungu itendayo kazi ndani yake Roho wa Mungu akitenda kazi juu ya vipengele vinavyounda mwili wake, angekuwa udongo}.}[Letter from Dr. Kellogg to Eld. Butler, February 21, 1904]

\others{Niko tayari kutupilia mbali mafundisho yote mabaya ambayo wewe na wengine mnanihusisha nayo. Niko tayari kukiri kwamba \textbf{mimi si anayesadiki imani kwa pantheism} wala mmizimu, na kwamba siamini mafundisho yoyote yanayofundishwa na watu hawa au \textbf{kutoka kwa maandishi ya Pantheism au kimizimu}. Sijawahi kusoma kitabu cha pantheism katika maisha yangu. Sijawahi kusoma kitabu kuhusu ‘Fikra Mpya,’ au kitu chochote cha aina hiyo. Mtu yeyote ambaye atasoma kwa umakinifu ‘Hekalu Hai’ kutoka ukurasa wa kwanza moja kwa moja hadi mwisho, na kupea jambo hili usawa na uzingatifu, ataona sana kwa uwazi kabisa kwamba \textbf{sina maelewano yoyote na mambo haya ya pantheism na kimizimu}.}[Ibid.]

Hili ni fumbo gumu sana kusuluhisha, hadi upatane na ukweli juu ya \emcap{Umbile la Mungu}, ambayo tulifanya mwanzoni mwa kitabu hiki. Ndiyo, Mungu hutegemeza uhai katika asili. Katika asili sisi \egwinline{\textbf{tunatambua \underline{Uwepo} unaoenea ndanimo vitu vyote vilivyoumbwa}}[Ed 100.2; 1903][https://egwwritings.org/read?panels=p29.475]. Lakini Mungu \textit{Mwenyewe}—katika Umbile lake—hayuko katika asili, wala asili si Mungu. Mungu ni \textit{huluki binafsi}, na yuko ndani ya hekalu lake takatifu, ameketi juu ya kiti chake cha enzi. Mungu yuko kila mahali kwa \textit{mwakilishi wake}, Roho Mtakatifu.

Wakati Dada White alisema \egwinline{Sayansi ya kibinadamu ni uwongo kuhusu Mungu \textbf{kutokuwa na Umbile},}[Ms236-1902; 1902][https://egwwritings.org/read?panels=p12779.6] alikuwa hasa akirejelea Mungu kuwa na umbo la kimwili la mtu, kama inavyoweza kuonekana katika muktadha wa nukuu hiyo. Lakini wakati Dk. Kellogg alikuwa akishughulikia ‘\textit{Umbile},’ hakuwa akishughulikia umbo au sura ya mtu. Mnamo 1936 katika hotuba yake, alionyesha maoni yale yale aliyokuwa nayo katika Hekalu Hai, lakini kwa uwazi zaidi:

\others{Kwa hiyo unaona haiwezekani kufikiria vitu visivyo na kikomo. Viko nje ya uwezo wetu. Viko \textbf{nje ya ufahamu} na jambo lile lile ni kweli kuhusu \textbf{\underline{Umbile lisilo na kikomo}}. \textbf{Hatuwezi kuunda dhana yoyote ya umbo lake au ukubwa wake au vikwazo vyovyote kwa sababu haina kikomo}. Sasa, labda hiyo ni wazo gumu kwako kuelewa na \textbf{ugumu wa kukubali wazo hili ni ukweli kwamba \underline{hatuna wazo wazi la Umbile}}. \textbf{Tunafikiri Umbile \underline{kama linavyohusiana na umbo}}.}

\others{…\textbf{Ilinipa dhana mpya ya Umbile}. \textbf{\underline{Umbile halimaanishi mtu, mwanamume au mwanamke}}. Halimaanishi aina hiyo ya kitu kabisa. \textbf{Inamaanisha kumiliki uwezo wa kudhamiri na kufanya na kufikiri na kupanga}.}[\href{https://forgotten-pillar.s3.us-east-2.amazonaws.com/Sanitarium+Lecture+1936.pdf}{Dr. Kellogg Sanitarium Lectures, 1936}; For transcript see \href{https://notefp.link/1938-kellogg-lecture}{https://notefp.link/1938-kellogg-lecture}]

Mtazamo kama huo wa Umbile ulipotumiwa kwa Mungu ulimwongoza Dk. Kellogg kwenye pantheism. Fundisho la \emcap{Umbile la Mungu} linashughulika na mtazamo sahihi wa Mungu. Mtazamo wa Dk. Kellogg wa Mungu ulikuwa mtazamo wa utatu.

\others{Yote niliyotaka kueleza katika Hekalu Hai ilikuwa kwamba kazi hii inayoendelea ndani ya mtu hapa \textbf{haiendi yenyewe \underline{kama saa iliyofungwa}; bali ni uweza wa Mungu na \underline{Roho wa Mungu anayeiendesha}}. \textbf{Sasa, nilifikiri nilikuwa nimetoa kabisa upande wa kitheolojia wa maswali ya \underline{utatu na aina hiyo yote ya mambo}}. \textbf{Sikukusudia kuiweka kabisa}, na nilichukua uchungu kusema katika utangulizi kwamba sikufanya hivyo. Sikuwahi kuota \textbf{kitu kama hicho} kwamba swali lolote la kitheolojia \textbf{lingeletwa ndani yake}. Nilitaka tu kuonyesha kwamba \textbf{\underline{moyo haupigi kwa mwendo wake mwenyewe} bali ni \underline{uweza wa Mungu unaoufanya uendelee}}.}[Interview, J. H. Kellogg, G. W. Amadon and A. C. Bourdeau, October 7th 1907 held at Kellogg's residence][https://archive.org/details/KelloggVs.TheBrethrenHisLastInterviewAsAnAdventistoct71907/page/n37]

Moyo haupigi kwa mwendo wake mwenyewe; ni uweza wa Mungu unaoufanya uendelee. Katika hili, Kellogg alikuwa sahihi kabisa.

\egw{\textbf{Mfumo wa kimwili wa mwanadamu uko chini ya usimamizi wa Mungu, lakini \underline{si kama saa ambayo imewekwa kufanya kazi na lazima iende yenyewe}}. \textbf{Moyo unapiga, mshindo unafuata mshindo, pumzi inafuata pumzi, lakini kumbuka kwamba kiumbe kiko chini ya usimamizi wa Mungu}. Ninyi ni shamba la Mungu, ninyi ni jengo la Mungu. \textbf{Katika Mungu tunaishi, tunasogea na kuwa na uhai wetu}. \textbf{Kila mpigo wa moyo, kila pumzi ni uvuvio wa Mungu yule aliyevuvia katika pua za Adamu pumzi ya uhai}, uvuvio wa Mungu aliye hai daima, MKUU AMBAYE YUKO.}[13LtMs, Ms 92, 1898, par. 7][https://egwwritings.org/read?panels=p14063.7342012&index=0]

Sayansi ya Dk. Kellogg ya \egwinline{Mungu katika asili ni kweli.}[Ms236-1902; 1902][https://egwwritings.org/read?panels=p12779.6] Maandiko yanafundisha wazi: \bible{Kama \normaltext{[Mungu]} akiweka moyo wake juu ya mwanadamu, \textbf{kama akijikusanyia mwenyewe \underline{roho yake} na pumzi yake}; \textbf{\underline{Wote wenye mwili wataangamia pamoja}, na mwanadamu atageuka tena kuwa mavumbi}.}[Job 34:14-15] \bible{…hukumu zako ni kina kirefu: \textbf{Ee Bwana, wewe \underline{huhifadhi} mwanadamu na mnyama}… \textbf{Kwa kuwa kwako kuna chemchemi ya uzima}: katika nuru yako tutaona nuru.}[Psalm 36:6b,9]

Ushahidi huu unashuhudia kwamba sayansi ya Dk. Kellogg ya Mungu katika asili ni kweli, lakini matatizo yake yalikuwa maoni ya makosa juu ya Umbile la Mungu, ambayo yalikuwa maoni ya utatu. Hata alipofafanua kwamba \others{Mungu Baba anaketi juu ya kiti chake cha enzi mbinguni ambapo Mwana wa Mungu pia yupo; wakati uhai wa Mungu, au roho au uwepo ni nguvu inayoenea kote ambayo inatekeleza mapenzi ya Mungu katika ulimwengu wote,}[Letter: Dr. Kellogg to W. W. Prescott, October 25, 1903][https://forgotten-pillar.s3.us-east-2.amazonaws.com/1903-10-25-JHKellogg-to-W.W.Prescott.pdf] bado alikuwa na maoni ya makosa juu ya Umbile la Mungu—Mungu katika \others{maana pana} kama \others{Uungu... Mungu Baba, Mungu Mwana, na Mungu Roho Mtakatifu}[Ibid.][https://forgotten-pillar.s3.us-east-2.amazonaws.com/1903-10-25-JHKellogg-to-W.W.Prescott.pdf]. Mtazamo wake wa Utatu \textit{haukuweza} \others{kutatua jambo hilo kwa kuridhisha.}[Letter: A. G. Daniells to W. C. White, October 29, 1903][https://forgotten-pillar.s3.us-east-2.amazonaws.com/Letter-A-G-Daniells-to-W-C-White-October-29-1903.pdf]

Hitimisho linatisha. Ikiwa unaamini kwamba moyo haupigi kwa mwendo wake mwenyewe bali ni uweza wa Mungu unaoufanya uendelee, na unaiunganisha na imani kwamba Mungu Mwenyewe si kiumbe kinachoshikika bali ni roho iliyopo kila mahali, basi machoni mwa Roho ya Unabii, wewe ni mpolandi. Mtazamo wa ubora au hali ya Mungu kuwa Nafsi ndio hufanya tofauti kati ya muumini wa kweli na mpolandi.

% Dr. Kellogg and pantheism

\begin{titledpoem}
    
    \stanza{
        In nature’s vast, a truth untold, \\
        He said God was in every fold. \\
        The trees, the breeze, the soil, the sea, \\
        God’s presence there, for all to see.
    }

    \stanza{
        Yet, in this truth where we concur, \\
        A deeper error did occur. \\
        The Trinity, unsacred bond, \\
        As pantheism and beyond.
    }

    \stanza{
        God’s personality is clear, \\
        Beyond those frontiers, we revere. \\
        For God, who’s more than nature’s face, \\
        Is personal, in sacred space.
    }

    \stanza{
        The doctor’s path did lead astray, \\
        On trinity, we cannot sway. \\
        His view of God, misunderstood, \\
        A misstep from the path of good.
    }

    \stanza{
        In nature, power does reside, \\
        It’s not God’s body that presides. \\
        Beside Him, Christ stands as our guide, \\
        And by His Spirit, life abides.
    }

    \stanza{
        In nature’s charm, God’s hand we see, \\
        Beyond the vastness, He must be. \\
        A precious God, with love so wide, \\
        In whom, in peace, we can confide.  
    }
    
\end{titledpoem}
