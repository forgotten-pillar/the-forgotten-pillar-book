
\qrchapter{https://forgottenpillar.com/rsc/en-fp-chapter28}{Wito wa kinabii wa kufanywa upya kwa nguzo za zamani}

Leo, Mungu yuko katika kazi ya kufanya upya \emcap{Kanuni za Msingi}. Tunayo ahadi kwamba nguzo za zamani za imani yetu zitahifadhiwa kwa sababu Mungu anatuita tufanye upya nguzo hizi. Hebu tuyasikilize mapenzi ya Mungu!

\egw{\textbf{\underline{Watu wetu wanahitaji kuelewa sababu za imani yetu na uzoefu wetu wa zamani}. Inasikitisha jinsi gani kwamba wengi wao wanaonekana kuweka imani isiyo na kikomo kwa watu ambao wanawasilisha nadharia zinazoelekea kung'oa uzoefu wetu wa zamani na kuondoa alama za zamani za kihistoria!} Wale ambao wanaweza kuongozwa kwa urahisi sana na roho ya uongo huonyesha kwamba wameongozwa kumfuata nahodha asiyefaa kwa muda fulani--muda mrefu hivi kwamba hawatambui kwamba wako wakiacha imani, au kwamba hawajengi juu ya msingi wa kweli. Tunahitaji kuwahimiza wote wavae miwani yao ya macho ya kiroho, wapakwe macho yao ili wapate kuona waziwazi na \textbf{kutambua nguzo za kweli za imani}. Ndipo watajua kwamba ‘msingi wa Mungu umesimama imara, wenye muhuri hii, Bwana awajua walio wake’‘ (2 Tim. 2:19). \textbf{\underline{Tunahitajika kufufua ushahidi wa zamani wa imani iliyotolewa mara moja kwa watakatifu}}.}[SW April 5, 1904, Art. B, par. 1; 1904][https://egwwritings.org/read?panels=p489.857]

Tunaposoma, jitihada kubwa zaidi ya Shetani ni kubadili maoni yetu kuhusu \emcap{Umbile la Mungu}. Ili kutuokoa kutokana na udanganyifu wa Shetani, Mungu anataka tufufue ushahidi wa zamani wa imani iliyotolewa kwa waanzilishi wetu. Tunahitaji kuelewa ushahidi wa kibiblia juu ya kwa nini “\textit{Mungu mmoja}” ni Baba, na kwamba Yeye ni Nafsi binafsi, wa kiroho. Kusoma somo hili, tunafanya mazoezi kuhusu historia ya waanzilishi wetu.

\egw{\textbf{Bwana ametangaza kwamba \underline{historia ya zamani itasomwa} tunapoingia kwenye kazi ya kuhitimisha. \underline{Kila ukweli} ambao Ametoa kwa siku hizi za mwisho unapaswa kutangazwa kwa ulimwengu. \underline{Kila nguzo} Aliyoisimamisha \underline{inapaswa kuimarishwa}. Hatuwezi sasa kuondoka kwenye msingi ambao Mungu ameweka... \underline{Kuna haja sasa kurudia uzoefu wa watu} ambao walishiriki katika uanzishwaji wa kazi yetu \underline{hapo mwanzoni}}.}[Ms129-1905.7; 1905][https://egwwritings.org/read?panels=p9797.14]

Tunahitaji kujifunza nguzo zote za imani yetu, mojawapo ni \emcap{Umbile la Mungu}! Mungu anashughulika kuhuisha ukweli wake, pamoja na kanisa lake. Hii haitatokea bila kutetereka kati ya watu wa Mungu. Tuna ushuhuda maalum juu ya kile kitakachosababisha kutetemeka kwa kanisa:

\egw{Niliuliza \textbf{maana ya mtikisiko} niliouona, nikaonyeshwa kuwa \textbf{utasababishwa kwa \underline{ushuhuda wa moja kwa moja} ulioitishwa na shauri la Shahidi wa Kweli kwa Walaodikia}. \textbf{Hii itakuwa na athari yake juu ya moyo wa mpokezi, na itampeleka \underline{kukuza wastani na kumimina ukweli ulionyooka}. Wengine hawatavumilia ushuhuda huu sawa. \underline{Watainuka dhidi yake, na hii itasababisha mtikiso kati ya watu wa Mungu}}.}[T04 34.4; 1857][https://egwwritings.org/read?panels=p12677.185]

Nukuu ifuatayo inatupa maelezo juu ya ujumbe gani utakuwa na ushuhuda wa moja kwa moja.

\egw{\textbf{\underline{Bwana anataka kufanywa upya kwa ushuhuda wa moja kwa moja uliotolewa katika miaka iliyopita}.} \textbf{Anaita kuwa upya maisha ya kiroho. Nguvu za kiroho za watu wake zimekuwa hafifu kwa muda mrefu, lakini kutakuwa na ufufuo kutoka katika kifo dhahiri}.}[8T 297.5; 1904][https://egwwritings.org/read?panels=p112.1796]

\egwnogap{\textbf{\underline{Kwa maombi na maungamo ya dhambi lazima tusafishe njia kuu ya Mfalme}. Tunapofanya hivi, nguvu za Roho zitakuja kwetu. Tunahitaji nishati ya Kipentekoste. \underline{Hii itakuja}, kwa maana Bwana ameahidi kumtuma Roho wake kama uweza ushindao wote}.}[8T 297.6; 1904][https://egwwritings.org/read?panels=p112.1797]

\egwnogap{\textbf{Nyakati za hatari ziko mbele yetu. Kila mtu ambaye ana ujuzi wa ukweli anapaswa kuamka na kujiweka mwenyewe, mwili, nafsi, na roho, \underline{chini ya nidhamu ya Mungu}}. \textbf{Adui yuko kwenye njia yetu. Ni lazima tuwe macho sana, tujilinde dhidi yake}. \textbf{Ni lazima tuvae silaha zote za Mungu}. \textbf{\underline{Ni lazima tufuate maagizo yanayotolewa kupitia roho ya unabii}. \underline{Ni lazima tuupende na kutii ukweli kwa wakati huu}. \underline{Hii itatuokoa kutoka kukubali udanganyifu mkali}. Mungu amesema nasi kupitia neno lake. Amesema kwetu kupitia shuhuda kwa kanisa na kupitia vitabu ambavyo vimesaidia kuweka wazi wajibu wetu wa sasa na nafasi ambayo tunapaswa kuchukua sasa. Maonyo ambayo yametolewa, mstari juu ya mstari, amri juu ya amri, inapaswa kuzingatiwa. Tukipuuza, tutatoa udhuru gani?}}[8T 298.1; 1904][https://egwwritings.org/read?panels=p112.1800]

\egwnogap{“\textbf{Ninawasihi wale wanaofanya kazi kwa ajili ya Mungu wasikubali mambo ya uongo kama ya kweli. Hebu sababu za kibinadamu zisiwekwe mahali ambapo ukweli wa kimungu, utakasao unapaswa kuwa}. \textbf{Kristo anasubiri kuwasha imani na upendo katika mioyo ya watu wake}. \textbf{Nadharia potofu zisipokee uso wa tabasamu kutoka kwa watu ambao wanapaswa kusimama imara kwenye jukwaa ya ukweli wa milele. \underline{Mungu anatuita kushikilia kwa uthabiti kanuni za msingi ambazo zinatoka kwenye mamlaka isiyotiliwa shaka}}”.}[8T 298.2; 1904][https://egwwritings.org/read?panels=p112.1801]

Ushuhuda wa moja kwa moja ambao utasababisha mtikisiko huo ni ushuhuda uliotolewa katika miaka iliyopita. Ushuhuda huu ni ujumbe uliomo katika \emcap{Kanuni za Msingi} zikiambatanishwa na baraza la Ushahidi wa Kweli kwa kanisa la Laodikia.

Matokeo ya mwisho ya kutetereka yatakuwa uamsho wa Mungu wa uzoefu wetu wa kwanza wenye nguvu wa waanzilishi, ambao walikuwa nao baada ya kukatishwa tamaa kuu. Dada White anathibitisha haya mara kadhaa. Mfano mmoja unapatikana katika shajara yake, ya tarehe 27 Novemba 1902.

\egw{\textbf{Nimevutiwa sana na Roho wa Mungu kwamba tunapaswa kupita katika majaribio magumu. Imani ya kila mtu itajaribiwa. \underline{Lazima tujifunze kwa uangalifu alama za zamani}.} \textbf{\underline{Uzoefu huu wa zamani unapaswa kufufuliwa}}. Danieli anapaswa kutokeza waziwazi pamoja na Ufunuo aliopewa Yohana kwenye Kisiwa cha Patmo.}[Ms223-1902.11; 1902][https://egwwritings.org/read?panels=p9124.26]

\egwnogap{\textbf{Katika uzoefu wetu katika siku hizi za mwisho tutakutana na kila jambo ambalo Shetani anaweza kuvumbua ili \underline{kuyabatilisha pointi zilizoimarishwa za imani yetu} ambayo yamekuwa, katika majaliwa ya Mungu, yabarikiwa sana.} \textbf{\underline{Kanuni hizi za msingi} zinapaswa kushikiliwa kwa dhati hadi mwisho. Soma Neno la Mungu.}}[Ms223-1902.13; 1902][https://egwwritings.org/read?panels=p9124.28]

Tena, Mungu anatuita kushikilia kwa dhati \emcap{Kanuni za Msingi} hadi mwisho.

\egw{“Sisi ni watu wa Mungu wanaozishika amri. Kwa miaka hamsini iliyopita kila awamu ya uzushi imeletwa juu yetu, ili kuziba akili zetu kuhusu mafundisho ya neno la Mungu, hasa kuhusu huduma ya Kristo katika patakatifu pa mbinguni, na ujumbe wa mbinguni kwa siku hizi za mwisho, kama ilivyotolewa na malaika wa sura ya kumi na nne Ufunuo. \textbf{Jumbe za kila utaratibu na aina zimehimizwa kwa Waadventista Wasabato, kuchukua nafasi ya ukweli ambao, \underline{pointi baada ya pointi}, umetafutwa kwa kujifunza kwa maombi, na kushuhudiwa kwa uwezo wa kutenda miujiza wa Bwana}. \textbf{Lakini alama za njia ambazo zimetufanya tulivyo, \underline{zinapaswa kuhifadhiwa, na zitahifadhiwa}, kama vile Mungu ameonyesha kupitia neno lake na ushuhuda wa Roho wake. \underline{Yeye anatutaka tushikilie kwa uthabiti}, kwa mshiko wa imani, \underline{kanuni za msingi ambazo zinatokana na mamlaka isiyotiliwa shaka}}.”}[SpTB02 59.1; 1904][https://egwwritings.org/read?panels=p417.299]

Hebu tujifunze kwa makini \emcap{Kanuni za Msingi}.

\egw{“\textbf{Tunaishi katika wakati ambapo kila upepo wa mafundisho unavuma na wale ambao wanadhani wanasimama wanawajibika kuanguka}. Tunaishi katika wakati ambapo Shetani anajitahidi kupanda mbegu za mashaka na ukafiri katika kila akili. \textbf{Tunaishi katika wakati ambapo makosa yanafundishwa kwa hila sana hivi kwamba imani ya wengi inadhoofishwa haraka sana}.”}[Ms143-1907.17; 1907][https://egwwritings.org/read?panels=p7851.23]

\egw{Loo, ni kiasi gani tunapoteza kwa kupuuza pendeleo la kushiriki kwa hiari mkate wa uzima! Je, hatutakataa kwa uthabiti kunaswa na adui wa nafsi zetu? Je, hatutaweka mbali kila kitu ambacho hugeuza akili mbali na kweli ambazo Mungu anatamani tujifunze? \textbf{Hebu tutafute kufahamu vitabu vinavyoeleza kwa uwazi \underline{ukweli wa wakati huu}}. \textbf{\underline{Hebu tujifunze kwa makini kanuni za msingi} za ujumbe ambazo zinatangazwa na watoto wa Mungu ulimwenguni kote}. \textbf{Wacha tuhifadhi taarifa kuhusu maendeleo ya ujumbe huu}. Kazi nzito zaidi sasa inaendelea—kazi ya kuonya ulimwengu usio na toba wa siku ya hukumu na ya kuja kwa Mwokozi wetu hivi karibuni katika mawingu ya mbinguni. Mungu anataka kila mtoto wake apate sehemu ya kutenda kazi hii kubwa. Twendeni kwa msaada wa Bwana, kwa msaada wa Bwana dhidi ya wenye nguvu.}[Ms143-1907.18; 1907][https://egwwritings.org/read?panels=p7851.24]

Katika somo hili tumepata fursa ya kufahamu \egwinline{vitabu vinavyoeleza wazi ukweli wa wakati huu}. Tumeangalia maandishi ya waanzilishi wetu kuhusu \emcap{Umbile la Mungu}. Tumeona \egwinline{ushahidi wa zamani wa imani iliyotolewa mara moja kwa watakatifu}. Katika kuchunguza uthibitisho kuhusu \emcap{Umbile la Mungu}, tumeona ushuhuda wao pia unapinga fundisho la Utatu. Kwa bahati mbaya, tumesahau nguzo ya imani yetu kuhusu \emcap{Umbile la Mungu}, na kwa sababu tumesahau, ni muhimu tukumbuke \egwinline{\textbf{jinsi Bwana ametuongoza, na \underline{mafundisho Yake} katika historia yetu iliyopita.}}[LS 196.2; 1915][https://egwwritings.org/read?panels=p41.1083] Tunapaswa kujifunza kwa makini \emcap{Kanuni za Msingi}. Hili ndilo kusudi la “\textit{The Forgotten Pillar Project}”. Tunakuhimiza kujifunza kwa makini pointi ya kwanza na ya pili za \emcap{Kanuni za Msingi}, zinazohusu \emcap{Umbile la Mungu} na mahali uwepo wake ulipo. Kwa sababu hii, tumefanya utafiti wa kina wa \emcap{Umbile la Mungu} tofauti na uelewaji wa sasa wa fundisho la Utatu. Tunakualika usome na ujifunze “\textit{Rediscovering the Pillar}”\footnote{Unaweza kupata kitabu hiki kwenye tovuti ya The Forgotten Pillar: \href{http://forgottenpillar.com}{forgottenpillar.com}}, ambayo ni mwendelezo wa kitabu hiki.

% The prophetic call for a renewal of the old pillars

\begin{titledpoem}
    
    \stanza{
        God calls us to the faith of old, \\
        The precious pillars are as gold. \\
        Solid foundation we reclaim, \\
        Glory to God, we all exclaim!
    }

    \stanza{
        Straight testimony must be borne, \\
        As in the Advent’s early morn, \\
        Study the waymarks which are true, \\
        Our sacred history we’ll review.
    }

    \stanza{
        For Satan works to change our view \\
        God’s personality, it is true. \\
        The trinity, it is a fraud \\
        God is our Father, one true God.
    }

    \stanza{
        Seek a revival, earnest prayer, \\
        As for the shaking, we prepare. \\
        Look for the old paths, walk therein, \\
        Plead for the Spirit to come in.
    }
    
\end{titledpoem}
