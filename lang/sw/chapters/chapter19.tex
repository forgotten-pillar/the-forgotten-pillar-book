\qrchapter{https://forgottenpillar.com/rsc/en-fp-chapter19}{Ellen White and Matthew 28:19}


\qrchapter{https://forgottenpillar.com/rsc/en-fp-chapter19}{Ellen White na Mathayo 28:19}


Many assert that Ellen White promoted the Trinity doctrine, and that she is the one responsible for accepting it into our ranks. These claims do not consider that she defended the \emcap{personality of God} expressed in the first point of the \emcap{Fundamental Principles}. To support the claims that Ellen White was trinitarian, quotations are presented to her comment on Matthew 28:19:


Wengi hudai kwamba Ellen White aliendeleza fundisho la Utatu, na kwamba yeye ndiye anawajibika kuikubali katika safu zetu. Madai haya hayazingatii kuwa alitetea \emcap{Umbile la Mungu} unaoonyeshwa katika pointi ya kwanza ya \emcap{Kanuni za Msingi}. Ili kuunga mkono madai kwamba Ellen White alikuwa wa utatu, nukuu zinawasilishwa kuhusu maoni yake ya Mathayo 28:19:


\bible{Go ye therefore, and teach all nations, \textbf{baptizing them in the name of \underline{the Father}, and of \underline{the Son}, and of \underline{the Holy Ghost}}.}[Matthew 28:19]


\bible{Basi, enendeni, mkawafanye mataifa yote kuwa wanafunzi, \textbf{mkiwabatiza kwa jina la \underline{Baba}, na la \underline{Mwana}, na la \underline{Roho Mtakatifu}}.}[Mathayo 28:19]


This verse has been most prominent in support of the Trinity doctrine. The Trinity doctrine has propositions about the \emcap{personality of God} of which this text says nothing to support. This verse itself does not teach that the Father, the Son, and the Holy Ghost, comprise \textit{one} God, the God of the Bible. There are other explicit verses in the Bible that exclude such interpretation of the text, i.e. 1 Corinthians 8:4-6; John 17:3; Ephesians 4:4-6; 1 Timothy 2:5.


Mstari huu umekuwa wenye kulazimisha sana kuunga mkono fundisho la Utatu. Fundisho la Utatu ina mapendekezo kuhusu \emcap{Umbile la Mungu} ambayo andiko hili halisemi chochote cha kuunga mkono. Hii mstari wenyewe haufundishi kwamba Baba, Mwana, na Roho Mtakatifu, wanajumuisha Mungu \textit{mmoja}, Mungu wa Biblia. Kuna mistari mingine iliyo wazi katika Biblia ambayo haijumuishi vile tafsiri ya maandishi, yaani 1 Wakorintho 8:4-6; Yohana 17:3; Waefeso 4:4-6; 1 Timotheo 2:5.


Unfortunately, the same unsupported assumptions made about Matthew 28:19 are made about Sister White’s quotations dealing with this verse. For example, Sister White uses terms like \egwinline{three highest powers in heaven}[Lt253a-1903.18; 1903][https://egwwritings.org/read?panels=p10143.25], \egwinline{three great powers of heaven}[8T 254.1; 1904][https://egwwritings.org/read?panels=p112.1450], \egwinline{the three holy dignitaries of heaven}[Ms92-1901.26: 1901][https://egwwritings.org/read?panels=p10732.32] and similar expressions—none of these quotations justify the assumption that these three (the Father, the Son, and the Holy Spirit) make \textit{one} God. On the contrary, as discussed in the previous chapter, keeping William Boardman’s sentiments and \egwinline{the heavenly trio} in context, “\textit{three-in-one}” sentiments \egwinline{should not be trusted}[Ms21-1906.8; 1906][https://egwwritings.org/read?panels=p9754.15].


Kwa bahati mbaya, mawazo yale yale yasiyoungwa mkono yaliyotolewa kuhusu Mathayo 28:19 yanafanywa kuhusu manukuu ya Dada White yanayohusu aya hii. Kwa mfano, Dada White anatumia maneno kama \egwinline{mamlaka tatu kuu mbinguni}[Lt253a-1903.18; 1903][https://egwwritings.org/read?panels=p10143.25], \egwinline{mamlaka kuu tatu za mbinguni}[8T 254.1; 1904][https://egwwritings.org/read?panels=p112.1450], \egwinline{watakatifu watatu wakuu wa mbinguni}[Ms92-1901.26: 1901][https://egwwritings.org/read?panels=p10732.32] na misemo kama hiyo—hakuna mojawapo ya nukuu hizi inayohalalisha dhana kwamba hawa watatu (Baba, Mwana, na Roho Mtakatifu) hufanya Mungu \textit{mmoja}. Kwa kinyume, kama ilivyojadiliwa katika sura iliyotangulia, kuweka hisia za William Boardman na \egwinline{watatu wa mbinguni} katika muktadha, hisia “\textit{tatu-katika-moja}” \egwinline{haifai kuaminiwa}[Ms21-1906.8; 1906][https://egwwritings.org/read?panels=p9754.15].


The heavenly trio (the group of three: the Father, the Son and the Holy Spirit) are also present in other Bible verses, in addition to Matthew 28:19. There are several other instances in the New Testament where the Father, the Son and the Holy Spirit are mentioned, and these verses should be used to interpret the meaning behind the heavenly trio. None of the verses on the heavenly trio prove a three-in-one God; rather, all of them refer to the Father as one God. In the following verses, the heavenly trio is bolded in order to better distinguish the Father, the Son and the Holy Spirit.


Watatu wa mbinguni (kundi la watatu: Baba, Mwana na Roho Mtakatifu) pia wapo katika mistari nyingine ya Biblia, pamoja na Mathayo 28:19. Kuna matukio mengine kadhaa katika Agano Jipya ambapo Baba, Mwana na Roho Mtakatifu wametajwa, na mistari hii inapaswa kutumika kutafsiri maana ya watatu wa mbinguni. Hakuna aya yoyote kwenye utatu wa mbinguni huthibitisha Mungu watatu-katika-mmoja; badala yake, wote wanamtaja Baba kuwa Mungu mmoja. Katika mistari ifuatayo, watatu wa mbinguni wametiwa ujasiri ili kumtofautisha vyema Baba, na Mwana na Roho Mtakatifu.


\bible{There is one body, and \textbf{one Spirit}, even as ye are called in one hope of your calling; \textbf{One Lord}, one faith, one baptism, \textbf{One God and Father} of all, who is above all, and through all, and in you all.}[Ephesians 4:4-6]


\bible{Kuna mwili mmoja, na \textbf{Roho mmoja}, kama vile mlivyoitwa katika tumaini moja la wito wenu; \textbf{Bwana Moja}, imani moja, ubatizo mmoja, \textbf{Mungu mmoja na Baba} wa wote, aliye juu ya yote na katika yote, na ndani yenu nyote.}[Waefeso 4:4-6]


\bible{Now there are diversities of gifts, but the \textbf{same Spirit}. And there are differences of administrations, but the \textbf{same Lord}. And there are diversities of operations, but it is \textbf{the same God} which worketh all in all.}[1 Corinthians 12:4-6]


\bible{Basi pana tofauti za karama, bali \textbf{Roho ni yeye yule}. Na kuna tofauti za huduma, lakini \textbf{Bwana ni yeye yule}. Na kuna anuwai ya shughuli, lakini ni \textbf{Mungu yeye yule} azitendaye kazi zote katika wote.}[1 Wakorintho 12:4-6]


\bible{The grace of \textbf{the Lord Jesus Christ}, and the love of \textbf{God}, and the communion of \textbf{the Holy Ghost}, be with you all. Amen.}[2 Corinthians 13:14]


\bible{Neema ya \textbf{Bwana Yesu Kristo}, na upendo wa \textbf{Mungu}, na ushirika wa \textbf{Roho Mtakatifu}, kuwa nanyi nyote. Amina.}[2 Wakorintho 13:14]


\bible{For through \textbf{him} \normaltext{[Christ]} we both have access by one \textbf{Spirit} unto the \textbf{Father}.}[Ephesians 2:18]


\bible{Kwa maana kwa \textbf{yeye} \normaltext{[Kristo]} sisi sote tumepata njia ya kumkaribia \textbf{Baba} katika \textbf{Roho} mmoja.}[Waefeso 2:18]


\bible{But we are bound to give thanks alway to \textbf{God} for you, brethren beloved of \textbf{the Lord}, because \textbf{God} hath from the beginning chosen you to salvation through sanctification of \textbf{the Spirit} and belief of the truth.}[2 Thessalonians 2:13]


\bible{Lakini imetupasa sisi kumshukuru \textbf{Mungu} siku zote kwa ajili yenu, ndugu mnaopendwa na \textbf{Bwana}; kwa maana \textbf{Mungu} amewateua tangu mwanzo mpate wokovu kwa utakaso wa \textbf{Roho} na imani ya kweli.}[2 Wathesalonike 2:13]


\bible{How much more shall the blood of \textbf{Christ}, who through the eternal \textbf{Spirit} offered himself without spot to \textbf{God}, purge your conscience from dead works to serve \textbf{the living God}?}[Hebrews 9:14]


\bible{Basi si zaidi damu yake \textbf{Kristo}, ambaye kwa \textbf{Roho} wa milele alijitoa nafsi yake bila mawaa kwa \textbf{Mungu}, itaosaje dhamiri zenu kutoka kwa matendo mafu, mpate kumwabudu \textbf{Mungu aliye hai}?}[Waebrania 9:14]


\bible{Elect according to the foreknowledge of \textbf{God the Father}, through sanctification of \textbf{the Spirit}, unto obedience and sprinkling of the blood of \textbf{Jesus Christ}: Grace unto you, and peace, be multiplied.}[1 Peter 1:2]


\bible{Mliochaguliwa kwa kujua tangu zamani kwa \textbf{Mungu Baba}, kwa utakaso wa \textbf{Roho}, kwa utii na kunyunyizwa kwa damu ya \textbf{Yesu Kristo}: Neema na iwe kwenu na amani, na iwe nyingi.}[1 Petro 1:2]


All of the above verses talk about the heavenly trio (the Father, the Son and the Holy Spirit), and all of them consistently testify that the Father is the one referred to as God.
The same reasoning holds ground for Ellen White’s interpretation of Matthew 28:19.


Mistari yote hapo juu inazungumza juu ya watatu wa mbinguni (Baba, Mwana na Roho Mtakatifu), na wote kwa mfululizo hushuhudia kwamba Baba ndiye anayetajwa kuwa Mungu. Hoja hiyo hiyo inashikilia msingi kwa tafsiri ya Ellen White ya Mathayo 28:19.


\egw{Christ gave His followers a positive promise that after His ascension He would send them His Spirit. ‘Go ye therefore,’ He said, ‘and teach all nations, baptizing them in the name of \textbf{the Father (a personal God),} and of \textbf{the Son (a personal Prince and Saviour),} and of \textbf{the Holy Ghost (sent from heaven to represent Christ);} teaching them to observe all things whatsoever I have commanded you, and, lo, I am with you alway, even unto the end of the world.’ Matthew 28:19, 20.}[RH October 26, 1897, par. 9; 1897][https://egwwritings.org/read?panels=p821.16317]


\egw{Kristo aliwapa wafuasi wake ahadi chanya kwamba baada ya kupaa kwake angewatumia Roho Wake. ‘Basi enendeni,’ akasema, ‘mkawafanye mataifa yote kuwa wanafunzi, mkiwabatiza kwa jina la \textbf{Baba (Mungu binafsi),} na wa \textbf{Mwana (Mfalme na Mwokozi binafsi),} na wa \textbf{Roho Mtakatifu (aliyetumwa kutoka mbinguni kumwakilisha Kristo);} kuwafundisha kushika mambo yote niliyowaamuru ninyi, na tazama, mimi nipo pamoja nanyi siku zote, hata ukamilifu wa dahari.’ Mathayo 28:19, 20.}[RH October 26, 1897, par. 9; 1897][https://egwwritings.org/read?panels=p821.16317]


The brackets in this quotation are in the original manuscript written by Ellen White. Here, she gives her own interpretation of Matthew 28:19. The Father is a personal God, the Son is a personal Prince and Saviour, and the Holy Spirit is Christ’s representative. This interpretation is in harmony with the \emcap{personality of God} expressed in the first point of the \emcap{Fundamental Principles}. Matthew 28:19 is a matter of interpretation. The interpretation which makes the Heavenly Trio one God is not inspired. This is not what the text indicates. Rather, let's read Matthew 28:19 within inspired compound: “\textit{Go ye therefore, and teach all nations, baptizing them in the name of a personal God, a personal Prince and Savior, and of the Holy Ghost}.” If one would read the text as such, no one would ever assume that one God is a unity of three persons. Therefore, let's stick to the inspiration, rather than subterfuge\footnote{\href{https://egwwritings.org/?ref=en\_Lt232-1903.41&para=10197.50}{{EGW, Lt232-1903.41; 1903}}}.


Mabano katika nukuu hii yako katika hati asili iliyoandikwa na Ellen White. Hapa, yeye anatoa tafsiri yake mwenyewe ya Mathayo 28:19. Baba ni Mungu binafsi, Mwana ni Mkuu na Mwokozi binafsi, na Roho Mtakatifu ndiye mwakilishi wa Kristo. Tafsiri hii inapatana na \emcap{Umbile la Mungu} unaoonyeshwa katika jambo la kwanza la \emcap{Kanuni za Msingi}. Mathayo 28:19 ni suala la tafsiri. Tafsiri inayofanya Watatu wa Mbinguni kuwa Mungu mmoja haijatokana na uvuvio. Hii sio kile maandiko yanaonyesha. Badala yake, hebu tusome Mathayo 28:19 ndani ya muunganiko wa uvuvio: “\textit{Basi enendeni, mkawafanye mataifa yote kuwa wanafunzi, mkiwabatiza kwa jina la Mungu binafsi, Mfalme na Mwokozi binafsi, na la Roho Mtakatifu}.” Kama mtu angesoma maandiko kama hayo, hakuna ambaye angeweza kudhania kwamba Mungu mmoja ni umoja wa nafsi tatu. Kwa hiyo, hebu tushikilie uvuvio, badala ya hila\footnote{\href{https://egwwritings.org/?ref=en\_Lt232-1903.41&para=10197.50}{{EGW, Lt232-1903.41; 1903}}}.


\egw{Let them be thankful to God for His manifold mercies and be kind to one another. \textbf{They have \underline{one God} and \underline{one Saviour}; and \underline{one Spirit}—\underline{the Spirit of Christ}—is to bring unity into their ranks}.}[9T 189.3; 1909][https://egwwritings.org/read?panels=p115.1057]


\egw{Na wamshukuru Mungu kwa rehema zake nyingi na wawe wema wao kwa wao. \textbf{Wao wanao \underline{Mungu mmoja} na \underline{Mwokozi mmoja}; na \underline{Roho mmoja}—\underline{Roho wa Kristo}—ni kuleta umoja katika safu zao}.}[9T 189.3; 1909][https://egwwritings.org/read?panels=p115.1057]


In light of the presented evidence, we see that simply numbering the Father, the Son and the Holy Spirit, does not prove the \textit{three-in-one} assumption, nor is it in conflict with the \emcap{personality of God} expressed in the \emcap{Fundamental Principles}. There is no denial of three persons of the Godhead, but only a denial of the assumption that these Three Great Worthies make one God.


Kwa kuzingatia uthibitisho uliotolewa, tunaona kwamba kuhesabu tu Baba, Mwana na Roho Mtakatifu, haithibitishi dhana ya \textit{watatu-katika-moja}, wala haipingani na \emcap{Umbile la Mungu} unaoonyeshwa katika \emcap{Kanuni za Msingi}. Hakuna kukataa kwa nafsi tatu za Uungu, lakini kuna kukanushwa kwa dhana kwamba Watatu Wakuu hawa hufanya Mungu mmoja.


Matthew 28:19 is a valuable verse and it opens a new field of study within the Bible and the Spirit of Prophecy. In the context of the Living Temple, and referring to its sentiments, Sister White wrote that this verse should be studied most earnestly because it is not half understood.


Mathayo 28:19 ni mstari muhimu na inafungua uwanja mpya wa kujifunza ndani ya Biblia na Roho ya Unabii. Katika muktadha wa The Living Temple, na kurejelea maoni yake, Dada White aliandika kwamba mstari huu unapaswa kuchunguzwa kwa bidii zaidi kwa sababu sio nusu kueleweka.


\egw{Just before His ascension, Christ gave His disciples a wonderful presentation, \textbf{as recorded in the twenty-eighth chapter of Matthew}. \textbf{This chapter contains instruction} that our ministers, our \textbf{physicians}, our youth, and all our church members need to \textbf{study most \underline{earnestly}}. \textbf{Those who study this instruction as they should will \underline{not dare to advocate theories that have no foundation in the Word of God}}. My brethren and sisters, make the Scriptures, which contain the alpha and omega of knowledge, your study. \textbf{All through the Old Testament and the New, there are things \underline{that are not half understood}}. ‘And Jesus came and spake unto them, saying, All power is given unto Me in heaven and in earth. Go ye therefore, and teach all nations, \textbf{baptizing them in the name of the Father, and of the Son, and of the Holy Ghost}; teaching them to observe all things whatsoever I have commanded you; and, lo, I am with you alway, even unto the end of the world.’ [Verses 18-20.]}[Lt214-1906.10; 1906][https://egwwritings.org/read?panels=p10171.16]


\egw{Kabla tu ya kupaa Kwake, Kristo aliwapa wanafunzi Wake maonyesho ya ajabu, \textbf{kama ilivyorekodiwa katika sura ya ishirini na nane ya Mathayo}. \textbf{Sura hii ina maagizo} ambayo wahudumu wetu, \textbf{waganga} wetu, vijana wetu, na washiriki wetu wote wa kanisa wanahitaji \textbf{kusoma zaidi kwa \underline{bidii}}. \textbf{Wale wanaosoma mafundisho haya inavyopaswa \underline{hawatathubutu kutetea nadharia ambazo hazina msingi katika Neno la Mungu}}. Ndugu na dada zangu, chukua Maandiko, ambayo yana alfa na omega ya maarifa, kuwa somo lako. \textbf{Yote kupitia Agano la Kale na Jipya, kuna mambo ambayo \underline{hayaeleweki hata nusu}}. ‘Na Yesu akaja na kusema nao, akisema, Nimepewa mamlaka yote mbinguni na duniani. Nenda basi ninyi, mkawafundishe mataifa yote, \textbf{mkiwabatiza kwa jina la Baba, na la Mwana, na wa Roho Mtakatifu}; na kuwafundisha kushika mambo yote niliyowaamuru; na tazama, mimi nipo pamoja nanyi siku zote, hata ukamilifu wa dahari.’ [Mistari 18-20.]}[Lt214-1906.10; 1906][https://egwwritings.org/read?panels=p10171.16]


There is a reason why Ellen White pipointed to Matthew 28:19 as a Scripture which is \egwinline{not half understood.} This statement is made in the context of 1906, where many ministers, and physicians were advocating the trinity doctrine. As we have seen, the understanding of God as a trinity, was not something Ellen White supported, and for this reason, herself, she dared not \egwinline{to advocate theories that have no foundation in the Word of God.}


Kuna sababu kwa nini Ellen White alionyesha Mathayo 28:19 kama Andiko ambalo \egwinline{haijafahamika hata nusu.} Taarifa hii ilifanywa katika muktadha wa 1906, ambapo wahudumu wengi na waganga walikuwa wakitetea fundisho la utatu. Kama tulivyoona, uelewa wa Mungu kama utatu, haikuwa jambo ambalo Ellen White aliunga mkono, na kwa sababu hii, yeye mwenyewe, hakuthubutu \egwinline{kutetea nadharia ambazo hazina msingi katika Neno la Mungu.}


\egw{The great Teacher held in His hand \textbf{the entire map of truth. In \underline{simple} language He \underline{made plain} to His disciples} the way to heaven and \textbf{the endless subjects of divine power}. \textbf{The question of \underline{the essence of God} was a subject on which He maintained a wise reserve}, for their entanglements and specifications would bring in science which could not be dwelt upon by unsanctified minds without confusion. \textbf{In regard to God and in regard to His personality, the Lord Jesus said}, ‘Have I been so long time with you, and yet hast thou not known Me, Philip? He that hath seen Me hath seen the Father.’ [John 14:9.] \textbf{Christ was the express image of His Father’s person}.}[19LtMs, Ms 45, 1904, par. 15][https://egwwritings.org/read?panels=p14069.9381023&index=0]


\egw{Mwalimu mkuu alikuwa na \textbf{ramani nzima ya ukweli mkononi mwake. Kwa lugha \underline{rahisi} Aliwafanya \underline{wazi} kwa wanafunzi wake} njia ya kwenda mbinguni na \textbf{masomo yasiyokwisha ya uwezo wa kimungu}. \textbf{Swali la \underline{kiini cha Mungu} lilikuwa somo ambalo alidumisha tahadhari ya hekima}, kwani matatizo yao na maelezo yao yangeleta sayansi ambayo haingeshughulikiwa na akili zisizotakaswa bila kuchanganyikiwa. \textbf{Kuhusu Mungu na kuhusu Umbile lake, Bwana Yesu alisema}, ‘Je, nimekuwa nanyi muda mrefu kiasi hicho, nawe hujanijua, Filipo? Yeye aliyeniona mimi amemwona Baba.’ [Yohana 14:9.] \textbf{Kristo alikuwa chapa kamili ya Umbile la Baba yake}.}[19LtMs, Ms 45, 1904, par. 15][https://egwwritings.org/read?panels=p14069.9381023&index=0]


\egwnogap{The open path, the safe path of walking in the way of His commandments, is a path from which there is no safe departing. \textbf{And when men follow their own human theories dressed up in soft, fascinating representations, they make a snare in which to catch souls}. \textbf{\underline{In the place of devoting your powers to theorizing}}, Christ has given you a work to do. His commission is, Go <throughout the world> and make disciples of all nations, \textbf{baptizing them in the name of the Father, and of the Son, and of the Holy Ghost}. Before the disciples shall compass the threshold, there is to be the imprint of \textbf{the sacred name, baptizing the believers in \underline{the name of the threefold powers} in the heavenly world}. The human mind is impressed in this ceremony, the beginning of the Christian life. It means very much. The work of salvation is not a small matter, but so vast that \textbf{the highest authorities} are taken hold of by the expressed faith of the human agency. \textbf{The Father, the Son, and the Holy Ghost, \underline{the eternal Godhead} is involved in the action required to make assurance to the human agent to unite \underline{all heaven} to contribute to the exercise of human faculties to reach and embrace the fulness of \underline{the threefold powers} to unite in the great work appointed, confederating the heavenly powers with the human, that men may become, through heavenly efficiency, partakers of the divine nature and workers together with Christ}.}[19LtMs, Ms 45, 1904, par. 16][https://egwwritings.org/read?panels=p14069.9381024&index=0]


\egwnogap{Njia iliyo wazi, njia salama ya kutembea katika njia ya amri Zake, ni njia ambayo hakuna kuondoka salama. \textbf{Na wakati watu wanafuata nadharia zao za kibinadamu zilizovishwa katika maonyesho laini, ya kuvutia, wanatengeneza mtego wa kunasa roho}. \textbf{\underline{Badala ya kutumia nguvu zako kwa kutheoretisha}}, Kristo amekupa kazi ya kufanya. Agizo lake ni, Nenda <duniani kote> na kuwafanya mataifa yote kuwa wanafunzi, \textbf{mkiwabatiza kwa jina la Baba, na la Mwana, na la Roho Mtakatifu}. Kabla wanafunzi hawajapita kizingiti, lazima kuwe na alama ya \textbf{jina takatifu, kuwabatiza waaminio kwa \underline{jina la nguvu tatu} katika ulimwengu wa mbinguni}. Akili ya binadamu inapigwa chapa katika sherehe hii, mwanzo wa maisha ya Kikristo. Inamaanisha mengi sana. Kazi ya wokovu si jambo dogo, lakini ni kubwa sana kwamba \textbf{mamlaka za juu zaidi} zinashikiliwa na imani iliyoonyeshwa ya wakala wa kibinadamu. \textbf{Baba, Mwana, na Roho Mtakatifu, \underline{Uungu wa milele} unahusika katika hatua inayohitajika kufanya uhakikisho kwa wakala wa kibinadamu kuunganisha \underline{mbingu yote} kuchangia katika zoezi la uwezo wa kibinadamu kufikia na kukumbatia ukamilifu wa \underline{nguvu tatu} kuunganisha katika kazi kubwa iliyoteuliwa, kuunganisha nguvu za mbinguni na za kibinadamu, ili watu waweze kuwa, kupitia ufanisi wa mbinguni, washiriki wa asili ya kimungu na watendakazi pamoja na Kristo}.}[19LtMs, Ms 45, 1904, par. 16][https://egwwritings.org/read?panels=p14069.9381024&index=0]


This quotation is yet another often misrepresented statement. It has been often used to argue that Ellen White advocated for the Trinity by referencing the Father, the Son and the Holy Spirit by term \egwinline{eternal Godhead.} However, we must peel back the layers of its context. Ellen White was explaining the meaning behind Matthew 28:19. She stated: \egwinline{In the place of devoting your powers to theorizing,} fulfill the commission given by Christ. Theorizing about what? Theorizing about \egwinline{the essence of God.} This is another “smoking gun” for the Trinity doctrine, especially when she referenced the \emcap{personality of God} by stating: \egwinline{\textbf{In regard to God and in regard to His personality}, the Lord Jesus said…[John 14:9.] Christ was the express image of His \textbf{Father’s person}.} John 14:9 does not mean that seeing the Father in Christ implies they are one and the same person, all part of one God. Rather, it affirms that Christ is the express image of the Father’s person. The “God” she referred to was the Father. Indeed, Jesus taught the truth about who and what God is. This is what He \egwinline{made plain} \egwinline{in the simple language.} To claim that by the term \egwinline{eternal Godhead} Ellen White was endorsing the Trinity would contradict the very caution she expressed in the context of this passage.


Nukuu hii ni moja wapo ya taarifa zinazotafsiriwa vibaya mara nyingi. Imekuwa ikitumika mara nyingi kutetea kwamba Ellen White alitetea Utatu kwa kurejelea Baba, Mwana na Roho Mtakatifu kwa neno \egwinline{Uungu wa milele.} Hata hivyo, lazima tuondoe tabaka za muktadha wake. Ellen White alikuwa akielezea maana ya Mathayo 28:19. Alisema: \egwinline{Badala ya kutumia nguvu zako kwa kutheoretisha,} timiza agizo lililotolewa na Kristo. Kutheoretisha kuhusu nini? Kutheoretisha kuhusu \egwinline{kiini cha Mungu.} Hii ni “ushahidi dhahiri” mwingine wa fundisho la Utatu, hasa aliporejea \emcap{Umbile la Mungu} kwa kusema: \egwinline{\textbf{Kuhusu Mungu na kuhusu Umbile lake}, Bwana Yesu alisema...[Yohana 14:9.] Kristo alikuwa chapa kamili ya \textbf{Umbile la Baba yake}.} Yohana 14:9 haimaanishi kwamba kumwona Baba katika Kristo inamaanisha wao ni mtu mmoja na yule yule, wote ni sehemu ya Mungu mmoja. Badala yake, inathibitisha kwamba Kristo ni chapa kamili ya Umbile la Baba. “Mungu” aliyemrejelea alikuwa Baba. Kwa kweli, Yesu alifundisha ukweli kuhusu Mungu ni nani na ni nini. Hiki ndicho \egwinline{alifanya wazi} \egwinline{kwa lugha rahisi.} Kudai kwamba kwa neno \egwinline{Uungu wa milele} Ellen White alikuwa akiidhinisha Utatu itakuwa inapingana na tahadhari aliyoonyesha katika muktadha wa kifungu hiki.


Unfortunately, the desperate desire of Trinitarians to paint Ellen White as a Trinitarian advocate has overshadowed the true, inspired meaning of Matthew 28:19. Her message was: \egwinline{In the place of devoting your powers to theorizing} about \egwinline{the essence of God,} Christ has given us the commission in Matthew 28:19. And she explained the meaning of Matthew 28:19. Her point was: The Father, Son, and Holy Spirit unite all of heaven’s resources with human effort so that, through divine power, people may share in God’s nature and work alongside Christ. That is the meaning of this \egwinline{threefold name.} She continued explaining:


Kwa bahati mbaya, hamu kubwa ya wafuasi wa Utatu kumwonyesha Ellen White kama mtetezi wa Utatu imefunika maana ya kweli, iliyovuviwa ya Mathayo 28:19. Ujumbe wake ulikuwa: \egwinline{Badala ya kutumia nguvu zako kwa kutheoretisha} kuhusu \egwinline{kiini cha Mungu,} Kristo ametupa agizo katika Mathayo 28:19. Na alielezea maana ya Mathayo 28:19. Hoja yake ilikuwa: Baba, Mwana, na Roho Mtakatifu wanaunganisha rasilimali zote za mbinguni na juhudi za kibinadamu ili, kupitia nguvu za kimungu, watu waweze kushiriki katika asili ya Mungu na kufanya kazi pamoja na Kristo. Hiyo ndiyo maana ya \egwinline{jina hili la tatu.} Aliendelea kuelezea:


\egw{\textbf{Man’s capabilities can multiply through the connection of human agencies with divine agencies}. \textbf{United with the heavenly powers}, the human capabilities increase according to that faith that works by love and purifies, sanctifies, and ennobles the whole man. \textbf{\underline{The heavenly powers} have \underline{pledged themselves} to minister to human agents to make the name of God and of Christ and of the Holy Spirit their living efficiency, working and energizing the sanctified man, to make this name above every other name}. \textbf{All the treasures of heaven are under obligation to do for man} infinitely more than human beings can comprehend by multiplying threefold the human with the heavenly agencies.}[19LtMs, Ms 45, 1904, par. 17][https://egwwritings.org/read?panels=p14069.9381026&index=0]


\egw{\textbf{Uwezo wa mwanadamu unaweza kuongezeka kupitia uhusiano wa wakala wa kibinadamu na wakala wa kimungu}. \textbf{Ukiungana na nguvu za mbinguni}, uwezo wa kibinadamu unaongezeka kulingana na imani inayofanya kazi kwa upendo na kusafisha, kutakasa, na kuinua mwanadamu mzima. \textbf{\underline{Nguvu za mbinguni} zimejitolea \underline{ahadi} kuhudumia wakala wa kibinadamu ili kufanya jina la Mungu na la Kristo na la Roho Mtakatifu kuwa ufanisi wao hai, kufanya kazi na kuimarisha mtu aliyetakaswa, ili kufanya jina hili kuwa juu ya kila jina lingine}. \textbf{Hazina zote za mbinguni ziko chini ya wajibu wa kufanya kwa mwanadamu} zaidi sana kuliko vile wanadamu wanaweza kuelewa kwa kuzidisha mara tatu wakala wa kibinadamu na wakala wa mbinguni.}[19LtMs, Ms 45, 1904, par. 17][https://egwwritings.org/read?panels=p14069.9381026&index=0]


\egwnogap{\textbf{\underline{The three great and glorious heavenly characters} are present on the occasion of baptism. All the human capabilities are to be henceforth consecrated powers to do service for God in representing the Father, the Son, and the Holy Ghost upon whom they depend. \underline{All heaven is represented by these three} in covenant relation with the new life}. ‘If ye then be risen with Christ, seek those things that are above, where Christ sitteth at \textbf{the right hand of God}.’ [Colossians 3:1.]}[19LtMs, Ms 45, 1904, par. 18][https://egwwritings.org/read?panels=p14069.9381027&index=0]


\egwnogap{\textbf{\underline{Watakatifu watatu wakuu na watukufu wa mbinguni} wanahudhuria wakati wa ubatizo. Uwezo wote wa kibinadamu utakuwa tangu sasa nguvu zilizowekwa wakfu kufanya huduma kwa Mungu katika kuwakilisha Baba, Mwana, na Roho Mtakatifu ambao wanategemea. \underline{Mbingu yote inawakilishwa na hawa watatu} katika uhusiano wa agano na maisha mapya}. ‘Basi mkiwa mmefufuliwa pamoja na Kristo, yatafuteni yaliyo juu, Kristo alikoketi \textbf{mkono wa kuume wa Mungu}.’ [Wakolosai 3:1.]}[19LtMs, Ms 45, 1904, par. 18][https://egwwritings.org/read?panels=p14069.9381027&index=0]


Many claim that Matthew 28:19 is uninspired because it was inserted by the Catholic Church\footnote{Note, 1 John 5:7 \bible{For there are three that bear record in heaven, the Father, the Word, and the Holy Ghost: and these three are one.} is an interpolation known as “\textit{Johannine Comma}”. Ellen White never used that verse. This was not the case with Matthew 28:19.}. Yet, here we have divine inspiration revealing its true meaning—the significance of baptism in the threefold name as a pledge made by these \egwinline{three great and glorious heavenly characters.} Their pledge is that \egwinline{\textbf{all the treasures of heaven are under obligation to do for man} infinitely more than human beings can comprehend by multiplying threefold the human with the heavenly agencies.}


Wengi hudai kwamba Mathayo 28:19 haijafunuliwa kwa sababu iliingizwa na Kanisa Katoliki\footnote{Kumbuka, 1 Yohana 5:7 \bible{Kwa maana wako watatu wanaoshuhudia mbinguni, Baba, Neno, na Roho Mtakatifu: na hawa watatu ni mmoja.} ni uingizaji unaojulikana kama “\textit{Johannine Comma}”. Ellen White hakutumia mstari huo. Haikuwa hivyo kwa Mathayo 28:19.}. Hata hivyo, hapa tunayo ufunuo wa kimungu ukifunua maana yake ya kweli—umuhimu wa ubatizo katika jina la tatu kama ahadi iliyotolewa na \egwinline{watakatifu watatu wakuu na watukufu wa mbinguni.} Ahadi yao ni kwamba \egwinline{\textbf{hazina zote za mbinguni ziko chini ya wajibu wa kufanya kwa mwanadamu} zaidi sana kuliko vile wanadamu wanaweza kuelewa kwa kuzidisha mara tatu wakala wa kibinadamu na wakala wa mbinguni.}


Ellen White frequently quoted Matthew 28:19, explaining the pledge of the Father, the Son, and the Holy Spirit. This pledge serves as a wonderful encouragement and a promise upheld by Heaven. A detailed study of this pledge is beyond the scope of this book, as it does not directly address the presence and \emcap{personality of God}. However, we encourage you to explore this topic for yourself. When you delve deeper into its meaning, you will come to understand the reality of the ministry of heavenly angels.


Ellen White alinukuu Mathayo 28:19 mara kwa mara, akielezea ahadi ya Baba, Mwana, na Roho Mtakatifu. Ahadi hii ni faraja ya ajabu na ahadi inayoshikiliwa na Mbingu. Uchunguzi wa kina wa ahadi hii uko nje ya upeo wa kitabu hiki, kwani haihusu moja kwa moja uwepo na \emcap{Umbile la Mungu}. Hata hivyo, tunakuhimiza kuchunguza mada hii mwenyewe. Unapozama zaidi katika maana yake, utaelewa ukweli wa huduma ya malaika wa mbinguni.


Sister White stated that \egwinline{all heaven is represented by these three in covenant relation with the new life.} These three are the Father, the Son, and the Holy Spirit. In another instance, she said:


Dada White alisema kwamba \egwinline{mbingu yote inawakilishwa na hawa watatu katika uhusiano wa agano na maisha mapya.} Hawa watatu ni Baba, Mwana, na Roho Mtakatifu. Katika tukio lingine, alisema:


\egw{\textbf{All heaven is interested in your home}. \textbf{God and Christ and \underline{the heavenly angels}} are intensely desirous that you shall so train your children that they will be prepared to enter the family of the redeemed.}[17LtMs, Ms 161, 1902, par. 11][https://egwwritings.org/read?panels=p14067.9877018&index=0]


\egw{\textbf{Mbingu yote inahusika na nyumba yako}. \textbf{Mungu na Kristo na \underline{malaika wa mbinguni}} wana hamu kubwa kwamba utawafundisha watoto wako ili waweze kuwa tayari kuingia katika familia ya waliokombolewa.}[17LtMs, Ms 161, 1902, par. 11][https://egwwritings.org/read?panels=p14067.9877018&index=0]


This is not a contradiction. All of heaven is represented by the Father, the Son, and the Holy Spirit, and in this quote, she specifically mentioned \egwinline{God and Christ and \textbf{the heavenly angels}.} There is a close connection between the workings of the Holy Spirit and the ministry of angels. The Inspiration testifies:


Hii sio ukinzani. Mbingu yote inawakilishwa na Baba, Mwana, na Roho Mtakatifu, na katika nukuu hii, yeye alitaja wazi \egwinline{Mungu na Kristo na \textbf{malaika wa mbinguni}.} Kuna uhusiano wa karibu kati ya kazi za Roho Mtakatifu na huduma ya malaika. Msukumo unashuhudia:


\egw{A measure of \textbf{the Spirit} is given to every man to profit withal. \textbf{Through the ministry of the angels \underline{the Holy Spirit is enabled} to work upon the mind and heart of the human agent}, and draw him to Christ who has paid the ransom money for his soul, that the sinner may be rescued from the slavery of sin and Satan.}[8LtMs, Lt 71, 1893, par. 10][https://egwwritings.org/read?panels=p14058.6086016&index=0]


\egw{Kipimo cha \textbf{Roho} kimepewa kila mtu ili kufaidika. \textbf{Kupitia huduma ya malaika \underline{Roho Mtakatifu anawezeshwa} kufanya kazi katika akili na moyo wa wakala wa kibinadamu}, na kumvuta kwa Kristo ambaye amelipa fidia ya pesa kwa ajili ya nafsi yake, ili mwenye dhambi aweze kuokolewa kutoka utumwa wa dhambi na Shetani.}[8LtMs, Lt 71, 1893, par. 10][https://egwwritings.org/read?panels=p14058.6086016&index=0]


This angelic ministry is one of the elements in the baptismal pledge of Matthew 28:19. When Ellen White said, \egwinline{\textbf{The heavenly powers} have \textbf{pledged themselves} to minister to human agents…,} she was referring to the holy angels. The connection between the Holy Spirit and the holy angels is beyond the scope of this book, but you can explore this topic further in the sequel, \textit{Rediscovering the Pillar}\footnote{Download for free: \href{https://forgottenpillar.com/book/rediscovering-the-pillar}{https://forgottenpillar.com/book/rediscovering-the-pillar}}, in the section on the Holy Spirit\footnote{Also, see the study on the angels \href{https://notefp.link/angels}{https://notefp.link/angels}}.


Huduma hii ya malaika ni moja ya vipengele katika ahadi ya ubatizo ya Mathayo 28:19. Wakati Ellen White alisema, \egwinline{\textbf{Nguvu za mbinguni} zime\textbf{jitolea} kuhudumia wakala wa kibinadamu...}, alikuwa akiwarejea malaika watakatifu. Uhusiano kati ya Roho Mtakatifu na malaika watakatifu uko nje ya upeo wa kitabu hiki, lakini unaweza kuchunguza mada hii zaidi katika toleo linalofuata, \textit{Rediscovering the Pillar}\footnote{Pakua bure: \href{https://forgottenpillar.com/book/rediscovering-the-pillar}{https://forgottenpillar.com/book/rediscovering-the-pillar}}, katika sehemu ya Roho Mtakatifu\footnote{Pia, angalia utafiti juu ya malaika \href{https://notefp.link/angels}{https://notefp.link/angels}}.


% Ellen White and Matthew 28:19

\begin{titledpoem}
    
    \stanza{
        In threefold name we’re baptized true, \\
        Not trinity as some construe. \\
        The Father, Son, and Spirit’s role, \\
        Not one God formed of triple whole.
    }

    \stanza{
        Dear Ellen’s words make clear the case, \\
        This pledge assures us heaven’s grace. \\
        The powers three have pledged their might, \\
        To guide the faithful to the light.
    }

    \stanza{
        Not proof of essence three-in-one, \\
        But heaven’s promise, freely done. \\
        A covenant of help divine, \\
        As new believers cross the line.
    }

    \stanza{
        The Father – God, in person real, \\
        The Son – our Prince, our wounds to heal, \\
        The Spirit – representative, \\
        Through Him Christ’s presence we receive.
    }
    
\end{titledpoem}


% Ellen White and Matthew 28:19

\begin{titledpoem}
    
    \stanza{
        In threefold name we’re baptized true, \\
        Not trinity as some construe. \\
        The Father, Son, and Spirit’s role, \\
        Not one God formed of triple whole.
    }

    \stanza{
        Dear Ellen’s words make clear the case, \\
        This pledge assures us heaven’s grace. \\
        The powers three have pledged their might, \\
        To guide the faithful to the light.
    }

    \stanza{
        Not proof of essence three-in-one, \\
        But heaven’s promise, freely done. \\
        A covenant of help divine, \\
        As new believers cross the line.
    }

    \stanza{
        The Father – God, in person real, \\
        The Son – our Prince, our wounds to heal, \\
        The Spirit – representative, \\
        Through Him Christ’s presence we receive.
    }
    
\end{titledpoem}
