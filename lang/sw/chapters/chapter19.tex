
\qrchapter{https://forgottenpillar.com/rsc/en-fp-chapter19}{Ellen White na Mathayo 28:19}


Wengi hudai kwamba Ellen White aliendeleza fundisho la Utatu, na kwamba yeye ndiye anawajibika kuikubali katika safu zetu. Madai haya hayazingatii kuwa alitetea \emcap{Umbile la Mungu} unaoonyeshwa katika pointi ya kwanza ya \emcap{Kanuni za Msingi}. Ili kuunga mkono madai kwamba Ellen White alikuwa wa utatu, nukuu zinawasilishwa kuhusu maoni yake ya Mathayo 28:19:


\bible{Basi, enendeni, mkawafanye mataifa yote kuwa wanafunzi, \textbf{mkiwabatiza kwa jina la \underline{Baba}, na la \underline{Mwana}, na la \underline{Roho Mtakatifu}}.}[Mathayo 28:19]


Mstari huu umekuwa wenye kulazimisha sana kuunga mkono fundisho la Utatu. Fundisho la Utatu ina mapendekezo kuhusu \emcap{Umbile la Mungu} ambayo andiko hili halisemi chochote cha kuunga mkono. Hii mstari wenyewe haufundishi kwamba Baba, Mwana, na Roho Mtakatifu, wanajumuisha Mungu \textit{mmoja}, Mungu wa Biblia. Kuna mistari mingine iliyo wazi katika Biblia ambayo haijumuishi vile tafsiri ya maandishi, yaani 1 Wakorintho 8:4-6; Yohana 17:3; Waefeso 4:4-6; 1 Timotheo 2:5.


Kwa bahati mbaya, mawazo yale yale yasiyoungwa mkono yaliyotolewa kuhusu Mathayo 28:19 yanafanywa kuhusu manukuu ya Dada White yanayohusu aya hii. Kwa mfano, Dada White anatumia maneno kama \egwinline{mamlaka tatu kuu mbinguni}[Lt253a-1903.18; 1903][https://egwwritings.org/read?panels=p10143.25], \egwinline{mamlaka kuu tatu za mbinguni}[8T 254.1; 1904][https://egwwritings.org/read?panels=p112.1450], \egwinline{watakatifu watatu wakuu wa mbinguni}[Ms92-1901.26: 1901][https://egwwritings.org/read?panels=p10732.32] na misemo kama hiyo—hakuna mojawapo ya nukuu hizi inayohalalisha dhana kwamba hawa watatu (Baba, Mwana, na Roho Mtakatifu) hufanya Mungu \textit{mmoja}. Kwa kinyume, kama ilivyojadiliwa katika sura iliyotangulia, kuweka hisia za William Boardman na \egwinline{watatu wa mbinguni} katika muktadha, hisia “\textit{tatu-katika-moja}” \egwinline{haifai kuaminiwa}[Ms21-1906.8; 1906][https://egwwritings.org/read?panels=p9754.15].


Watatu wa mbinguni (kundi la watatu: Baba, Mwana na Roho Mtakatifu) pia wapo katika mistari nyingine ya Biblia, pamoja na Mathayo 28:19. Kuna matukio mengine kadhaa katika Agano Jipya ambapo Baba, Mwana na Roho Mtakatifu wametajwa, na mistari hii inapaswa kutumika kutafsiri maana ya watatu wa mbinguni. Hakuna aya yoyote kwenye utatu wa mbinguni huthibitisha Mungu watatu-katika-mmoja; badala yake, wote wanamtaja Baba kuwa Mungu mmoja. Katika mistari ifuatayo, watatu wa mbinguni imetiwa mkazo ili kumtofautisha vyema Baba, na Mwana na Roho Mtakatifu.


\bible{Kuna mwili mmoja, na \textbf{Roho mmoja}, kama vile mlivyoitwa katika tumaini moja la wito wenu; \textbf{Bwana Moja}, imani moja, ubatizo mmoja, \textbf{Mungu mmoja na Baba} wa wote, aliye juu ya yote na katika yote, na ndani yenu nyote.}[Waefeso 4:4-6]


\bible{Basi pana tofauti za karama, bali \textbf{Roho ni yeye yule}. Na kuna tofauti za huduma, lakini \textbf{Bwana ni yeye yule}. Na kuna anuwai ya shughuli, lakini ni \textbf{Mungu yeye yule} azitendaye kazi zote katika wote.}[1 Wakorintho 12:4-6]


\bible{Neema ya \textbf{Bwana Yesu Kristo}, na upendo wa \textbf{Mungu}, na ushirika wa \textbf{Roho Mtakatifu}, kuwa nanyi nyote. Amina.}[2 Wakorintho 13:14]


\bible{Kwa maana kwa \textbf{yeye} \normaltext{[Kristo]} sisi sote tumepata njia ya kumkaribia \textbf{Baba} katika \textbf{Roho} mmoja.}[Waefeso 2:18]


\bible{Lakini imetupasa sisi kumshukuru \textbf{Mungu} siku zote kwa ajili yenu, ndugu mnaopendwa na \textbf{Bwana}; kwa maana \textbf{Mungu} amewateua tangu mwanzo mpate wokovu kwa utakaso wa \textbf{Roho} na imani ya kweli.}[2 Wathesalonike 2:13]


\bible{Basi si zaidi damu yake \textbf{Kristo}, ambaye kwa \textbf{Roho} wa milele alijitoa nafsi yake bila mawaa kwa \textbf{Mungu}, itaosaje dhamiri zenu kutoka kwa matendo mafu, mpate kumwabudu \textbf{Mungu aliye hai}?}[Waebrania 9:14]


\bible{Mliochaguliwa kwa kujua tangu zamani kwa \textbf{Mungu Baba}, kwa utakaso wa \textbf{Roho}, kwa utii na kunyunyizwa kwa damu ya \textbf{Yesu Kristo}: Neema na iwe kwenu na amani, na iwe nyingi.}[1 Petro 1:2]


Mistari yote hapo juu inazungumza juu ya watatu wa mbinguni (Baba, Mwana na Roho Mtakatifu), na wote kwa mfululizo hushuhudia kwamba Baba ndiye anayetajwa kuwa Mungu. Hoja hiyo hiyo inashikilia msingi kwa tafsiri ya Ellen White ya Mathayo 28:19.


\egw{Kristo aliwapa wafuasi wake ahadi chanya kwamba baada ya kupaa kwake angewatumia Roho Wake. ‘Basi enendeni,’ akasema, ‘mkawafanye mataifa yote kuwa wanafunzi, mkiwabatiza kwa jina la \textbf{Baba (Mungu binafsi),} na wa \textbf{Mwana (Mfalme na Mwokozi binafsi),} na wa \textbf{Roho Mtakatifu (aliyetumwa kutoka mbinguni kumwakilisha Kristo);} kuwafundisha kushika mambo yote niliyowaamuru ninyi, na tazama, mimi nipo pamoja nanyi siku zote, hata ukamilifu wa dahari.’ Mathayo 28:19, 20.}[RH October 26, 1897, par. 9; 1897][https://egwwritings.org/read?panels=p821.16317]


Mabano katika nukuu hii yako katika hati asili iliyoandikwa na Ellen White. Hapa, yeye anatoa tafsiri yake mwenyewe ya Mathayo 28:19. Baba ni Mungu binafsi, Mwana ni Mkuu na Mwokozi binafsi, na Roho Mtakatifu ndiye mwakilishi wa Kristo. Tafsiri hii inapatana na \emcap{Umbile la Mungu} unaoonyeshwa katika jambo la kwanza la \emcap{Kanuni za Msingi}. Mathayo 28:19 ni suala la tafsiri. Tafsiri inayofanya Watatu wa Mbinguni kuwa Mungu mmoja haijatokana na uvuvio. Hii sio kile maandiko yanaonyesha. Badala yake, hebu tusome Mathayo 28:19 ndani ya muunganiko wa uvuvio: “\textit{Basi enendeni, mkawafanye mataifa yote kuwa wanafunzi, mkiwabatiza kwa jina la Mungu binafsi, Mfalme na Mwokozi binafsi, na la Roho Mtakatifu}.” Kama mtu angesoma maandiko kama hayo, hakuna ambaye angeweza kudhania kwamba Mungu mmoja ni umoja wa nafsi tatu. Kwa hiyo, hebu tushikilie uvuvio, badala ya hila\footnote{\href{https://egwwritings.org/?ref=en\_Lt232-1903.41&para=10197.50}{{EGW, Lt232-1903.41; 1903}}}.


\egw{Na wamshukuru Mungu kwa rehema zake nyingi na wawe wema wao kwa wao. \textbf{Wao wanao \underline{Mungu mmoja} na \underline{Mwokozi mmoja}; na \underline{Roho mmoja}—\underline{Roho wa Kristo}—ni kuleta umoja katika safu zao}.}[9T 189.3; 1909][https://egwwritings.org/read?panels=p115.1057]


Kwa kuzingatia uthibitisho uliotolewa, tunaona kwamba kuhesabu tu Baba, Mwana na Roho Mtakatifu, haithibitishi dhana ya \textit{watatu-katika-moja}, wala haipingani na \emcap{Umbile la Mungu} unaoonyeshwa katika \emcap{Kanuni za Msingi}. Hakuna kukataa kwa nafsi tatu za Uungu, lakini kuna kukanushwa kwa dhana kwamba Watatu Wakuu hawa hufanya Mungu mmoja.


Mathayo 28:19 ni mstari muhimu na inafungua uwanja mpya wa kujifunza ndani ya Biblia na Roho ya Unabii. Katika muktadha wa The Living Temple, na kurejelea maoni yake, Dada White aliandika kwamba mstari huu unapaswa kuchunguzwa kwa bidii zaidi kwa sababu sio nusu umeeleweka.


\egw{Kabla tu ya kupaa Kwake, Kristo aliwapa wanafunzi Wake maonyesho ya ajabu, \textbf{kama ilivyorekodiwa katika sura ya ishirini na nane ya Mathayo}. \textbf{Sura hii ina maagizo} ambayo wahudumu wetu, \textbf{waganga} wetu, vijana wetu, na washiriki wetu wote wa kanisa wanahitaji \textbf{kusoma zaidi kwa \underline{bidii}}. \textbf{Wale wanaosoma mafundisho haya inavyopaswa \underline{hawatathubutu kutetea nadharia ambazo hazina msingi katika Neno la Mungu}}. Ndugu na dada zangu, chukua Maandiko, ambayo yana alfa na omega ya maarifa, kuwa somo lako. \textbf{Yote kupitia Agano la Kale na Jipya, kuna mambo ambayo \underline{hayaeleweki hata nusu}}. ‘Na Yesu akaja na kusema nao, akisema, Nimepewa mamlaka yote mbinguni na duniani. Nenda basi ninyi, mkawafundishe mataifa yote, \textbf{mkiwabatiza kwa jina la Baba, na la Mwana, na wa Roho Mtakatifu}; na kuwafundisha kushika mambo yote niliyowaamuru; na tazama, mimi nipo pamoja nanyi siku zote, hata ukamilifu wa dahari.’ [Mistari 18-20.]}[Lt214-1906.10; 1906][https://egwwritings.org/read?panels=p10171.16]


Kuna sababu kwa nini Ellen White alionyesha Mathayo 28:19 kama Andiko ambalo \egwinline{haijafahamika hata nusu.} Taarifa hii ilifanywa katika muktadha wa 1906, ambapo wahudumu wengi na waganga walikuwa wakitetea fundisho la utatu. Kama tulivyoona, uelewa wa Mungu kama utatu, haikuwa jambo ambalo Ellen White aliunga mkono, na kwa sababu hii, yeye mwenyewe, hakuthubutu \egwinline{kutetea nadharia ambazo hazina msingi katika Neno la Mungu.}


\egw{Mwalimu mkuu alikuwa na \textbf{ramani nzima ya ukweli mkononi mwake. Kwa lugha \underline{rahisi} Aliwafanya \underline{wazi} kwa wanafunzi wake} njia ya kwenda mbinguni na \textbf{masomo yasiyokwisha ya uwezo wa kimungu}. \textbf{Swali la \underline{kiini cha Mungu} lilikuwa somo ambalo alidumisha tahadhari ya hekima}, kwani matatizo yao na maelezo yao yangeleta sayansi ambayo haingeshughulikiwa na akili zisizotakaswa bila kuchanganyikiwa. \textbf{Kuhusu Mungu na kuhusu Umbile lake, Bwana Yesu alisema}, ‘Je, nimekuwa nanyi muda mrefu kiasi hicho, nawe hujanijua, Filipo? Yeye aliyeniona mimi amemwona Baba.’ [Yohana 14:9.] \textbf{Kristo alikuwa chapa kamili ya Umbile la Baba yake}.}[19LtMs, Ms 45, 1904, par. 15][https://egwwritings.org/read?panels=p14069.9381023&index=0]


\egwnogap{Njia iliyo wazi, njia salama ya kutembea katika njia ya amri Zake, ni njia ambayo hakuna kuondoka kuliko salama. \textbf{Na wakati watu wanafuata nadharia zao za kibinadamu zilizovishwa katika maonyesho laini, ya kuvutia, wanatengeneza mtego wa kunasa roho}. \textbf{\underline{Badala ya kutumia nguvu zako kwa kutheoretisha}}, Kristo amekupa kazi ya kufanya. Agizo lake ni, Nenda <duniani kote> na kuwafanya mataifa yote kuwa wanafunzi, \textbf{mkiwabatiza kwa jina la Baba, na la Mwana, na la Roho Mtakatifu}. Kabla wanafunzi hawajapita kizingiti, lazima kuwe na alama ya \textbf{jina takatifu, kuwabatiza waaminio kwa \underline{jina la nguvu tatu} katika ulimwengu wa mbinguni}. Akili ya binadamu inapigwa chapa katika sherehe hii, mwanzo wa maisha ya Kikristo. Inamaanisha mengi sana. Kazi ya wokovu si jambo dogo, lakini ni kubwa sana kwamba \textbf{mamlaka za juu zaidi} zinashikiliwa na imani iliyoonyeshwa ya wakala wa kibinadamu. \textbf{Baba, Mwana, na Roho Mtakatifu, \underline{Uungu wa milele} unahusika katika hatua inayohitajika kufanya uhakikisho kwa wakala wa kibinadamu kuunganisha \underline{mbingu yote} kuchangia katika zoezi la uwezo wa kibinadamu kufikia na kukumbatia ukamilifu wa \underline{nguvu tatu} kuunganisha katika kazi kubwa iliyoteuliwa, kuunganisha nguvu za mbinguni na za kibinadamu, ili watu waweze kuwa, kupitia ufanisi wa mbinguni, washiriki wa asili ya kimungu na watendakazi pamoja na Kristo}.}[19LtMs, Ms 45, 1904, par. 16][https://egwwritings.org/read?panels=p14069.9381024&index=0]


Nukuu hii ni moja wapo ya taarifa zinazotafsiriwa vibaya mara nyingi. Imekuwa ikitumika mara nyingi kutetea kwamba Ellen White alitetea Utatu kwa kurejelea Baba, Mwana na Roho Mtakatifu kwa neno \egwinline{Uungu wa milele.} Hata hivyo, lazima tuondoe tabaka za muktadha wake. Ellen White alikuwa akielezea maana ya Mathayo 28:19. Alisema: \egwinline{Badala ya kutumia nguvu zako kwa kutheoretisha,} timiza agizo lililotolewa na Kristo. Kutheoretisha kuhusu nini? Kutheoretisha kuhusu \egwinline{kiini cha Mungu.} Hii ni “ushahidi dhahiri” mwingine wa fundisho la Utatu, hasa aliporejea \emcap{Umbile la Mungu} kwa kusema: \egwinline{\textbf{Kuhusu Mungu na kuhusu Umbile lake}, Bwana Yesu alisema...[Yohana 14:9.] Kristo alikuwa chapa kamili ya \textbf{Umbile la Baba yake}.} Yohana 14:9 haimaanishi kwamba kumwona Baba katika Kristo inamaanisha wao ni mtu mmoja na yule yule, wote ni sehemu ya Mungu mmoja. Badala yake, inathibitisha kwamba Kristo ni chapa kamili ya Umbile la Baba. “Mungu” aliyemrejelea alikuwa Baba. Kwa kweli, Yesu alifundisha ukweli kuhusu Mungu ni nani na ni nini. Hiki ndicho \egwinline{alifanya wazi} \egwinline{kwa lugha rahisi.} Kudai kwamba kwa neno \egwinline{Uungu wa milele} Ellen White alikuwa akiidhinisha Utatu itakuwa inapingana na tahadhari aliyoonyesha katika muktadha wa kifungu hiki.


Kwa bahati mbaya, hamu kubwa ya wafuasi wa Utatu kumwonyesha Ellen White kama mtetezi wa Utatu imefunika maana ya kweli, iliyovuviwa ya Mathayo 28:19. Ujumbe wake ulikuwa: \egwinline{Badala ya kutumia nguvu zako kwa kutheoretisha} kuhusu \egwinline{kiini cha Mungu,} Kristo ametupa agizo katika Mathayo 28:19. Na alielezea maana ya Mathayo 28:19. Hoja yake ilikuwa: Baba, Mwana, na Roho Mtakatifu wanaunganisha rasilimali zote za mbinguni na juhudi za kibinadamu ili, kupitia nguvu za kimungu, watu waweze kushiriki katika asili ya Mungu na kufanya kazi pamoja na Kristo. Hiyo ndiyo maana ya \egwinline{jina hili la tatu.} Aliendelea kuelezea:


\egw{\textbf{Uwezo wa mwanadamu unaweza kuongezeka kupitia uhusiano wa wakala wa kibinadamu na wakala wa kimungu}. \textbf{Ukiungana na nguvu za mbinguni}, uwezo wa kibinadamu unaongezeka kulingana na imani inayofanya kazi kwa upendo na kusafisha, kutakasa, na kuinua mwanadamu mzima. \textbf{\underline{Nguvu za mbinguni} zimejitolea \underline{ahadi} kuhudumia wakala wa kibinadamu ili kufanya jina la Mungu na la Kristo na la Roho Mtakatifu kuwa ufanisi wao hai, kufanya kazi na kuimarisha mtu aliyetakaswa, ili kufanya jina hili kuwa juu ya kila jina lingine}. \textbf{Hazina zote za mbinguni ziko chini ya wajibu wa kufanya kwa mwanadamu} zaidi sana kuliko vile wanadamu wanaweza kuelewa kwa kuzidisha mara tatu wakala wa kibinadamu na wakala wa mbinguni.}[19LtMs, Ms 45, 1904, par. 17][https://egwwritings.org/read?panels=p14069.9381026&index=0]


\egwnogap{\textbf{\underline{Watakatifu watatu wakuu na watukufu wa mbinguni} wanahudhuria wakati wa ubatizo. Uwezo wote wa kibinadamu utakuwa tangu sasa nguvu zilizowekwa wakfu kufanya huduma kwa Mungu katika kuwakilisha Baba, Mwana, na Roho Mtakatifu ambao wanategemea. \underline{Mbingu yote inawakilishwa na hawa watatu} katika uhusiano wa agano na maisha mapya}. ‘Basi mkiwa mmefufuliwa pamoja na Kristo, yatafuteni yaliyo juu, Kristo alikoketi \textbf{mkono wa kuume wa Mungu}.’ [Wakolosai 3:1.]}[19LtMs, Ms 45, 1904, par. 18][https://egwwritings.org/read?panels=p14069.9381027&index=0]


Wengi hudai kwamba Mathayo 28:19 haijafunuliwa kwa sababu iliingizwa na Kanisa Katoliki\footnote{Kumbuka, 1 Yohana 5:7 \bible{Kwa maana wako watatu wanaoshuhudia mbinguni, Baba, Neno, na Roho Mtakatifu: na hawa watatu ni mmoja.} ni uingizaji unaojulikana kama “\textit{Johannine Comma}”. Ellen White hakutumia mstari huo. Haikuwa hivyo kwa Mathayo 28:19.}. Hata hivyo, hapa tunayo ufunuo wa kimungu ukifunua maana yake ya kweli—umuhimu wa ubatizo katika jina la tatu kama ahadi iliyotolewa na \egwinline{watakatifu watatu wakuu na watukufu wa mbinguni.} Ahadi yao ni kwamba \egwinline{\textbf{hazina zote za mbinguni ziko chini ya wajibu wa kufanya kwa mwanadamu} zaidi sana kuliko vile wanadamu wanaweza kuelewa kwa kuzidisha mara tatu wakala wa kibinadamu na wakala wa mbinguni.}


Ellen White alinukuu Mathayo 28:19 mara kwa mara, akielezea ahadi ya Baba, Mwana, na Roho Mtakatifu. Ahadi hii ni faraja ya ajabu na ahadi inayoshikiliwa na Mbingu. Uchunguzi wa kina wa ahadi hii uko nje ya upeo wa kitabu hiki, kwani haihusu moja kwa moja uwepo na \emcap{Umbile la Mungu}. Hata hivyo, tunakuhimiza kuchunguza mada hii mwenyewe. Unapozama zaidi katika maana yake, utaelewa ukweli wa huduma ya malaika wa mbinguni.


Dada White alisema kwamba \egwinline{mbingu yote inawakilishwa na hawa watatu katika uhusiano wa agano na maisha mapya.} Hawa watatu ni Baba, Mwana, na Roho Mtakatifu. Katika tukio lingine, alisema:


\egw{\textbf{Mbingu yote inahusika na nyumba yako}. \textbf{Mungu na Kristo na \underline{malaika wa mbinguni}} wana hamu kubwa kwamba utawafundisha watoto wako ili waweze kuwa tayari kuingia katika familia ya waliokombolewa.}[17LtMs, Ms 161, 1902, par. 11][https://egwwritings.org/read?panels=p14067.9877018&index=0]


Hii sio ukinzani. Mbingu yote inawakilishwa na Baba, Mwana, na Roho Mtakatifu, na katika nukuu hii, yeye alitaja wazi \egwinline{Mungu na Kristo na \textbf{malaika wa mbinguni}.} Kuna uhusiano wa karibu kati ya kazi za Roho Mtakatifu na huduma ya malaika. uvuvio unashuhudia:


\egw{Kipimo cha \textbf{Roho} kimepewa kila mtu ili kufaidika. \textbf{Kupitia huduma ya malaika \underline{Roho Mtakatifu anawezeshwa} kufanya kazi katika akili na moyo wa wakala wa kibinadamu}, na kumvuta kwa Kristo ambaye amelipa fidia ya pesa kwa ajili ya nafsi yake, ili mwenye dhambi aweze kuokolewa kutoka utumwa wa dhambi na Shetani.}[8LtMs, Lt 71, 1893, par. 10][https://egwwritings.org/read?panels=p14058.6086016&index=0]


Huduma hii ya malaika ni moja ya vipengele katika ahadi ya ubatizo ya Mathayo 28:19. Wakati Ellen White alisema, \egwinline{\textbf{Nguvu za mbinguni} zime\textbf{jitolea} kuhudumia wakala wa kibinadamu...}, alikuwa akiwarejea malaika watakatifu. Uhusiano kati ya Roho Mtakatifu na malaika watakatifu uko nje ya upeo wa kitabu hiki, lakini unaweza kuchunguza mada hii zaidi katika toleo linalofuata, \textit{Rediscovering the Pillar}\footnote{Pakua bure: \href{https://forgottenpillar.com/book/rediscovering-the-pillar}{https://forgottenpillar.com/book/rediscovering-the-pillar}}, katika sehemu ya Roho Mtakatifu\footnote{Pia, angalia utafiti juu ya malaika \href{https://notefp.link/angels}{https://notefp.link/angels}}.




% Ellen White and Matthew 28:19

\begin{titledpoem}
    
    \stanza{
        In threefold name we’re baptized true, \\
        Not trinity as some construe. \\
        The Father, Son, and Spirit’s role, \\
        Not one God formed of triple whole.
    }

    \stanza{
        Dear Ellen’s words make clear the case, \\
        This pledge assures us heaven’s grace. \\
        The powers three have pledged their might, \\
        To guide the faithful to the light.
    }

    \stanza{
        Not proof of essence three-in-one, \\
        But heaven’s promise, freely done. \\
        A covenant of help divine, \\
        As new believers cross the line.
    }

    \stanza{
        The Father – God, in person real, \\
        The Son – our Prince, our wounds to heal, \\
        The Spirit – representative, \\
        Through Him Christ’s presence we receive.
    }
    
\end{titledpoem}
