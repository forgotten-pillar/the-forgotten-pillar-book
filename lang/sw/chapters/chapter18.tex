\qrchapter{https://forgottenpillar.com/rsc/en-fp-chapter18}{Watatu wa Mbinguni}

Mpaka sasa tumeona ushahidi kwamba Ellen White alijua kuhusu maoni ya utatu ya Dk. Kellogg, na tumeona jinsi alivyojibu. Yeye daima alitukuza ukweli juu ya uwepo na \emcap{Umbile la Mungu}, na kuita ili kurudi kwenye msingi wa imani yetu—\emcap{Kanuni za Kimsingi}. Hata hivyo, wakati wasomi wa Waadventista wanapojadili fundisho la Utatu na Ellen White, hawaikaribii kwa njia ile ile kama Ellen White alivyofanya. \emcap{Kanuni za Kimsingi} pamoja na fundisho juu ya \emcap{Umbile la Mungu} zinapuuzwa, na hadithi iliyopotoshwa inawasilishwa kwamba Ellen White alikuwa mwamini wa utatu na alihusika na kanisa kukubali fundisho la Utatu katika safu zetu. Tunataka kupinga hadithi hii iliyopotoshwa kwa kuangalia ushahidi ambao mara nyingi hutumiwa kuunga mkono simulizi hii ya uongo.

Mojawapo ya nukuu maarufu zaidi za kuunga mkono dai kwamba Dada White alihusika kwa kukubali fundisho la Utatu katika safu zetu ni maandishi na maoni yake juu ya Mathayo 28:19\footnote{\bible{Basi, enendeni, mkawafanye mataifa yote kuwa wanafunzi, mkiwabatiza kwa jina la Baba, na Mwana, na Roho Mtakatifu}[Mathayo 28:19]}. Nukuu maarufu zaidi ya kutetea fundisho la Utatu ni nukuu ya “\textit{Watatu wa Mbinguni}”:

\egw{\textbf{Kuna \underline{nafsi tatu hai} wa \underline{watatu wa mbinguni}}; kwa jina la hawa nguvu tatu wakuu—\textbf{Baba, Mwana, na Roho Mtakatifu}—wale wanaompokea Kristo kwa imani iliyo hai wanabatizwa, na nguvu hizi zitashirikiana na raia watiifu wa mbinguni katika juhudi zao za kuishi maisha mapya katika Kristo...}[Ev 615.1; 1946][https://egwwritings.org/read?panels=p30.3407]

Kurudia, nukuu hii mara nyingi inatajwa kusema kwamba Dada White alitetea na kudumisha fundisho la Utatu. Lakini, tukiitazama nukuu hii katika muktadha wake wa kifasihi, tunaona kwamba ndani ya nukuu yenyewe kwa kweli \textit{alikanusha} fundisho hili na kutukuza ukweli juu ya \emcap{Umbile la Mungu}. Kwa wengine hili ni dai la kichekesho, lakini tunakukaribisha ufanye hukumu yako kulingana na data iliyowasilishwa. Hebu tuchunguze muktadha wa nukuu hii.

\egw{Nimeagizwa kusema, \textbf{Maoni} ya wale wanaotafuta mawazo ya sayansi ya hali ya juu \textbf{\underline{hayafai kuaminiwa}}. Uwakilishi kama ufuatao unafanywa: ‘\textbf{Baba ni kama nuru isiyoonekana; Mwana ni kama nuru ilivyomwilishwa; Roho ni kama nuru inavyomwagika nje.}’ ‘\textbf{Baba ni kama umande, mvuke usioonekana; Mwana ni kama umande uliokusanywa kwa umbo la kupendeza; Roho ni kama umande ulioanguka kwenye kiti cha uzima.}’ Uwakilishi mwingine: ‘\textbf{Baba ni kama mvuke usioonekana. Mwana ni kama wingu la risasi. Roho ni mvua iliyonyesha na kufanya kazi katika nguvu za kuburudisha.}’}[Ms21-1906.8; 1906][https://egwwritings.org/read?panels=p9754.15]

Ni maoni gani ambayo hayafai kuaminiwa? Data inaonyesha kwamba maoni hayo ni mawazo ya utatu ya \textit{Mungu mmoja katika nafsi tatu}. Tunajuaje hilo? Tunaona katika muktadha wa kifasihi wa uwakilishi ambao Dada White alinukuu. Kinyume na imani maarufu kwamba alikuwa anarejelea “\textit{utatu wa uongo}” ulioelezwa na Dk. Kellogg,\footnote{Whidden, Woodrow W, et al. \textit{The Trinity : Understanding God's Love, His Plan of Salvation, and Christian Relationships}. Hagerstown, Md, Review And Herald Pub. Association, 2002, p. 216.} kwa kweli alikuwa anarejelea wazo la utatu la \textit{nafsi tatu hai za Mungu mmoja aliye hai}, lililotetewa na William Boardman, katika kitabu chake “Higher Christian Life”, ambacho alinukuu. Muktadha ni muhimu. Muktadha wa nukuu alizozinukuu, unaonyesha kwamba uwakilishi wa Baba, Mwana, na Roho Mtakatifu unatumika kuelezea maoni ya nafsi tatu hai za Mungu mmoja. Hayo ndiyo maoni ambayo tumeagizwa wazi na Mungu, kutoyaamini. Acha data ifasiri yenyewe.

\section*{Maisha ya Juu ya Kikristo, William Boardman}

Ellen White alikuwa na kitabu cha William Boardman “Higher Christian Life.” Kilikuwa kitabu kizuri kuhusu utakaso wa Kikristo, lakini ndani yake kulikuwa na maoni ya utatu, ambayo Dada White aliagizwa na Mungu kuyataja. Hii ni mfano mwingine wa ushahidi ambapo tunaona kwamba Ellen White alikuwa anafahamu msimamo wa utatu, na alikuwa anaukabili moja kwa moja. Hebu tujue maoni ya utatu yaliyotangazwa na William Boardman.

Akizungumzia Mungu wa Utatu, William Boardman anaandika:

\othersQuote{Na kisha, tena, Baba ndiye mwanzilishi na mpangaji wa wokovu kupitia imani katika Mwana wake; na tunapomtumaini Mwana wake tunamheshimu Baba, kwa sababu tunakubali mpango wake wa wokovu kwa ajili yetu, thibitisha hekima yake, na kutenda kulingana na mapenzi yake katika jambo hilo. \textbf{Mtazamo wa mahusiano rasmi na muhimu ya nafsi za Utatu Mtakatifu kwa kila mmoja wao na kwetu, inaweza kutupa nuru ya ziada kwenye njia yetu}. Juu ya somo hili kurupuka itapakana na kufuru. Ni ardhi takatifu. Mwenye kujitosa juu yake anaweza vyema kukanyaga bila viatu, na kichwa kisichofunikwa kimeinamishwa chini.}[William Boardman, The Higher Christian Life, p. 99; 1858][https://archive.org/details/higherchristian02boargoog/page/n106/]

Ndugu Boardman anataka tuangalie \others{mahusiano rasmi na muhimu} ya nafsi tatu za Utatu Mtakatifu. Anadai kwamba \textit{Mungu ni mmoja lakini pia watatu}–\textit{Utatu}–kwa kuwasilisha mahusiano rasmi na muhimu ya nafsi za Utatu Mtakatifu. Taarifa yake ya msingi na muhtasari wa tasnifu yake ni kama ifuatavyo:

\othersQuote{\textbf{Baba ni utimilifu wa Uungu \underline{usioonekana}, hana umbo, ambaye hakuna kiumbe chochote kimeona au kinaweza kuona}. \\
\textbf{Mwana ni utimilifu wa Uungu \underline{unaomwilishwa}, ili viumbe vyake vimwone, na kumjua, na kumtumaini}. \\
\textbf{Roho ndiye utimilifu wa Uungu \underline{katika utendaji kazi wote}, iwe wa uumbaji, majaliwa, ufunuo, au wokovu, ambao kupitia huo Mungu hujidhihirisha mwenyewe kwa na kupitia ulimwengu}.}[William Boardman, The Higher Christian Life, p. 100][https://archive.org/details/higherchristian02boargoog/page/n108/]

Kauli hii ni ya msingi kwa kauli na vielelezo vyake vifuatavyo. Katika aya zifuatazo, William Boardman anatoa nia za kibiblia ili kueleza \others{mahusiano rasmi na muhimu ya Utatu Mtakatifu}—\textit{yaani, Mungu akiwa mmoja, lakini bado watatu}. Anaandika:

\othersQuote{Jina lingine la Yesu litatoa mlinganisho sawa kwa njia ya kushangaza - \textbf{Jua la Haki}. \\
Nuru yote ya jua mbinguni wakati mmoja ilifichwa katika kutoonekana kwa giza kuu; na baada ya hayo, nuru inayowaka sasa katika obi ya mchana ilikuwa, wakati amri ya kwanza ilipotolewa, Nuru iwe! na mwanga ilikuwa, ulionea zaidi ni ukungu wa alfajiri wa kijivu ya asubuhi ya uumbaji kutoka katika giza la usiku wa machafuko, bila umbo, wala mwili, wala katikati, wala mng'ao, au utukufu. Lakini unapotengwa na giza na kuwekwa katikati ya jua, basi katika utukufu wake uling'aa sana hivi kwamba hakuna mtu ila jicho la tai angeweza kustahimili kuiangalia usoni. \\
Lakini tena miale yake ikishuka kwa kasi katika angahewa na mivuke ya dunia, hufurahisha kila kitu ulimwenguni na mwanga ule ule, ukiondoa msimu wa baridi, na baridi, na giza; kuanzia Chipua kwa uzuri wa maua, na Majira ya joto katika mazingira ya asili, na Vuli iliyojaa hazina za dhahabu kwa ghala.
\textbf{Baba ni kama Nuru isiyoonekana}. \\
\textbf{Mwana ni kama Nuru ilivyomwilishwa}. \\
\textbf{Roho ni kama Nuru iangazavyo}.}[William Boardman, The Higher Christian Life, p. 101,102][https://archive.org/details/higherchristian02boargoog/page/n108/]

Kielelezo hiki cha Jua la Haki kinaonyesha kwamba Mungu Baba, ambaye ni \textit{utimilifu wa Uungu usioonekana,} anaweza kufananishwa kiishara na Nuru ambayo \others{ilifichwa katika kutoonekana kwa giza kuu}. Mwana, ambaye ni \textit{utimilifu wa Uungu unaomwilishwa}, ni kama Nuru iliyofumbatwa ndani ya \others{asubuhi ya uumbaji}. Roho Mtakatifu, ambaye ni \textit{utimilifu wa Uungu katika utendaji kazi wote}, ni kama \others{Nuru iliyomwagika}. William Boardman anatupa kielelezo kingine sawa kufafanua \others{mahusiano rasmi ya nafsi za Uungu}:

\othersQuote{Mojawapo ya mifano ya ushawishi uliobarikiwa wa Roho, \textbf{huku ukitoa yale yale mahusiano rasmi ya nafsi za Uungu, kwa kila mmoja wao na sisi}, yanaweza kuyaonyesha bado zaidi,—\textbf{Ule Umande},—\textbf{Ule umande wa Hamoni}—umande kwenye mbuga iliyokatwa. Kabla ya umande kukusanyika kwa matone, huning'inia juu ya mazingira yote katika mvuke usioonekana, ulio kila mahali lakini isiyoonekana. Mara kwa mara usiku unapoingia asubuhi, na joto likizidi kuzama na kugusa umande kwa hatua asiyeonekana inakuwa inayoonekana, ilivyo; na jua linapochomoza ndivyo lilivyo limesimama katika matone ya almasi likitetemeka na kuangaa katika miale michanga ya jua katika uzuri wa lulu juu ya jani na maua, juu ya uso wote wa asili. \\
Lakini sasa tena, upepo unavuma, pumzi ya mbinguni ilipepea kwa upole, ikitetemeka jani na maua, na kwa muda mfupi matone ya lulu hayaonekani tena. Lakini wapi sasa? Imeanguka kwenye mzizi wa mimea na ua ili kutoa maisha mapya, upya, nguvu kwa yote yanayogusa. \\
\textbf{Baba ni kama umande katika mvuke usioonekana}. \\
\textbf{Mwana ni kama umande uliokusanywa kwa umbo la kupendeza}. \\
\textbf{Roho ni kama umande ulioanguka kwenye kiti cha uzima}.}[William Boardman, The Higher Christian Life, p. 102,103][https://archive.org/details/higherchristian02boargoog/page/n110/]

Baba, ambaye ni \textit{utimilifu wa Uungu usioonekana,} anaonyeshwa na \others{umande katika mvuke usioonekana}. Mwana, ambaye ni \textit{utimilifu wa Uungu unaomwilishwa}, anaonyeshwa na \others{umande uliokusanyika kwa umbo la kupendeza}. Roho, ambaye ni \textit{utimilifu wa Uungu katika utendaji kazi wote}, anaonyeshwa na \others{umande ulioanguka kwenye kiti cha uzima}. Kielelezo kinachofuata ambacho ni mfano wa mahusiano rasmi ya nafsi tatu za Mungu mmoja ni kwa mfano mwingine wa Biblia—Mvua.

\othersQuote{\textbf{Bado moja zaidi ya ulinganisho huu wa Biblia} - bila kuwachosha - hautafanyika usiokubalika, au usio na maana, - \textbf{Mvua}. \\
Mvua, kama umande, huelea bila kuonekana, na uwepo wa kila mahali hapo kwanza, juu ya yote, karibu na wote. Bila kuonekana na yeyote. Ijapokuwa inabaki katika kutoonekana kwake, dunia tambara, madongoa yanashikana pamoja ardhi hupasuka, jua humwaga joto lake linalowaka, pepo huinua vumbi ndani tufani zinazozunguka, na mawingu, na njaa imepungua, na mabua ya uchoyo katika nchi; ikifuatiwa na tauni na kifo. Mara kwa mara, mlinzi mwenye shauku anaona mkono mdogo wingu likipanda juu sana juu ya bahari. Inakusanya, inakusanya, inakusanya; huja na kuenea ikija, katika enzi juu ya mbingu zote: - Lakini yote ni kavu na kavu na bado kufa, juu ardhi. \\
Lakini sasa inakuja tone, na tone baada ya tone, haraka, kasi - kuoga, mvua - kufagia na kuipa ardhi hazina zote za mawingu - madongoa yanafunguka, matuta yanalainika, chemchemi; mito, mito, hufurika na kujaa, na nchi yote inafurahishwa tena kwa utele uliorejeshwa. \\
\textbf{Baba ni kama mvuke usioonekana}. \\
\textbf{Mwana ni kama wingu lililoelemewa na mvua inayonyesha}. \\
\textbf{Roho ni Mvua - iliyonyesha na kufanya kazi kwa nguvu ya kuburudisha}.}[William Boardman, The Higher Christian Life, p. 103,104][https://archive.org/details/higherchristian02boargoog/page/n110/]

Hebu tumsikilize William Boardman kwa usawa. Hasemi kwamba Baba ni \others{mvuke usioonekana}; badala yake, anatumia mfano wa mvua na \others{mvuke usioonekana} kuelezea hoja yake kuu kwamba Baba ni utimilifu usioonekana wa Uungu. Ndivyo ilivyo kwa Mwana, ambaye, kama vile mvua inavyodhihirishwa katika mawingu mazito, ni utimilifu wote wa Uungu uliodhihirishwa. Ili kuhakikisha maoni yake hayawakilishwi vibaya, William Boardman alifafanua maoni yake. Haya ndiyo maoni ambayo Ellen White aliagizwa na Mungu kutoyaamini:

\othersQuote{\textbf{Ulinganisho huu wote si kamilifu. Afadhali wao kujificha kuliko kuonyesha \underline{ubinafsi tatu wa Mungu mmoja}, kwa maana wao si watu bali ni vitu, maskini na wa kidunia kwa ubora zaidi, kuwakilisha nafsi hai za Mungu aliye hai. Sana wanaweza kufanya, hata hivyo, kama ili kuonyesha uhusiano rasmi wa kila mmoja kwa wengine na wa kila mmoja na wote kwetu. Na zaidi. Wanaweza pia kuonyesha ukweli kwamba utimilifu wote wa Yeye anayejaza yote katika yote, hukaa ndani ya kila nafsi ya \underline{Mungu wa Utatu}}. \\
\textbf{Baba ni utimilifu wote wa Uungu USIOONEKANA}. \\
\textbf{Mwana ndiye utimilifu wote wa Uungu ULIODHIHIRISHWA}. \\
\textbf{Roho ndiye utimilifu wote wa Uungu AKIDHIHIRISHA}. \\
\textbf{Watu hao si ofisi tu, au njia za ufunuo, bali ni watu walio hai wa Mungu aliye hai}.}[William Boardman, The Higher Christian Life, p. 104,105][https://archive.org/details/higherchristian02boargoog/page/n112/]

Ni muhimu kusisitiza kwamba wakati Boardman anapotumia mifano hii ya Biblia kutoka kwa asili, yeye anazungumza juu ya vielelezo, na sio ukweli. Vielelezo hivi vinaonyesha maoni yake. Katika kukiri kwake mwenyewe, hayo yalikuwa maoni ya \others{nafsi hai za Mungu aliye hai.} Ingawa vielelezo hivi si kamilifu, vinaweza \others{kuonyesha uhusiano rasmi} wa \others{ubinafsi tatu wa Mungu mmoja} na \others{ukweli kwamba utimilifu wote wa Yeye anayejaza yote katika yote, hukaa ndani ya kila nafsi ya Mungu wa Utatu.} Mungu mmoja katika nafsi tatu ni maoni yanayozungumziwa, na maoni hayo ni ya kawaida kwa aina zote na matoleo ya fundisho la utatu—ikiwa ni pamoja na msimamo wetu wa sasa wa utatu katika nukta ya pili ya Mafundisho za Kimsingi.\footnote{\others{Kuna \textbf{Mungu mmoja}: Baba, Mwana, na Roho Mtakatifu, \textbf{umoja wa watatu} wa milele \textbf{Nafsi}…} Nukta ya 2 ya Mafundisho za Kimsingi}

Katika mtazamo huu mfupi wa maoni ya William Boardman, ni wazi kwamba maoni yanayozungumziwa ambayo Ellen White aliagizwa na Mungu kuyataja, yalikuwa maoni ya Mungu wa Utatu, au \textit{nafsi tatu zilizo hai katika Utatu}. Kwa data hiyo akilini, hebu tuchunguze majibu ya Ellen White.

\section*{Ellen White juu ya maoni ya William Boardman}

Kwa nukuu ya Watatu wa Mbinguni, imesemekana kwamba Ellen White alikuwa mwamini wa utatu. Hii inafanywa kwa ujinga au wakati mwingine kwa makusudi kupuuza muktadha wa nukuu hii ya thamani. Wakati wa kusoma majibu ya Ellen White, ambayo anatetea mitazamo yetu ya Mungu, jaribu kutambua ni nani anayemrejelea anaposema juu ya Mungu. Je, Mungu aliyemtetea alikuwa Utatu au Baba? Akirejelea vielelezo vya William Boardman alisema:

\egw{\textbf{Uwakilishi huu wote wa \underline{kimizimu} ni bure bilashi}. Wakilishi hizi si kamilifu, zisizo za kweli. Zinadhoofisha na kupunguza Ukuu ambao hakuna mfano wa dunia unaoweza kulinganishwa. \textbf{Mungu hawezi kulinganishwa na vitu ambavyo mikono yake imetengeneza}. Haya tu ni vitu vya duniani, huteseka chini ya laana ya Mungu kwa sababu ya dhambi za mwanadamu. \textbf{Baba hawezi kuelezewa na vitu vya duniani}. \textbf{Baba ndiye utimilifu wote wa Uungu \underline{kwa jinsi ya kimwili} na \underline{haonekani kwa macho ya kibinadamu}}.}[Ms21-1906.9; 1906][https://egwwritings.org/read?panels=p9754.15]

Kwa kutazama muktadha, ni dhahiri kwamba Dada White anafuata njia ya hoja ya Boardman na kurekebisha makosa. Kwa ulinganisho bora zaidi, hebu tuangalie maandishi yao bega kwa bega:

\begin{table}[H]
\centering
\renewcommand{\arraystretch}{1.5}
\setlength{\tabcolsep}{15pt}
\resizebox{\textwidth}{!}{
\begin{tabular}{|p{0.4\textwidth}|p{0.4\textwidth}|}
\hline
\multicolumn{1}{|c|}{\textbf{William Boardman}} & \multicolumn{1}{c|}{\textbf{Ellen G. White}} \\ \hline
\othersQuote{Ulinganisho huu wote si kamilifu. Afadhali wao kujificha kuliko \textbf{kuonyesha ubinafsi tatu wa \underline{Mungu mmoja}}, kwa maana wao si watu bali ni vitu, maskini na wa kidunia kwa ubora zaidi, kuwakilisha \textbf{nafsi hai za Mungu aliye hai}. \textbf{Sana wanaweza kufanya, hata hivyo, kama ili kuonyesha uhusiano rasmi wa kila mmoja kwa mwingine na wa kila mmoja na wote kwetu. Na zaidi. Wanaweza pia kuonyesha ukweli kwamba utimilifu wote wa Yeye anayejaza yote katika yote, hukaa ndani ya \underline{kila nafsi ya Mungu wa Utatu}}.}[p. 104,105][https://archive.org/details/higherchristian02boargoog/page/n112] & 
\egw{\textbf{Uwakilishi huu wote wa \underline{kimizimu} ni bure bilashi}. Wakilishi hizi si kamilifu, zisizo za kweli. Zinadhoofisha na kupunguza Ukuu ambao hakuna mfano wa dunia unaoweza kulinganishwa. \textbf{Mungu hawezi kulinganishwa na vitu ambavyo mikono yake imetengeneza}. Haya tu ni vitu vya duniani, huteseka chini ya laana ya Mungu kwa sababu ya dhambi za mwanadamu. \textbf{Baba hawezi kuelezewa na vitu vya duniani}.}[Ms21-1906.9; 1906][https://egwwritings.org/read?panels=p9754.15] \\ \hline
\end{tabular}
}
\end{table}

Katika ulinganisho huu, ni wazi Mungu ni nani kwa William Boardman, na Yeye ni nani kwa Dada White. Kwa Boardman, Mungu ni Mungu wa Utatu, ubinafsi tatu wa Mungu mmoja. Kwa Dada White, Mungu ni Baba. Kwa Boardman, uwakilishi huu si kamilifu kwa sababu \others{afadhali wao kujificha kuliko kuonyesha ubinafsi tatu wa Mungu mmoja}, na kwa Dada White uwakilishi huo si kamilifu kwa sababu \egw{Baba hawezi kuelezewa na vitu vya duniani}. Kwa Boardman, Mungu ni \textit{Mungu wa Utatu}; kwa Dada White, Mungu ni \textit{Baba}.

Hoja pekee ya Boardman ambayo Ellen White anathibitisha ni kwamba uwakilishi huu si kamilifu. Hakika, William Boardman hangekubaliana na Ellen White kwamba uwakilishi huu ni wa kimizimu na usio wa kweli. Badala yake, anaamini kwamba vielelezo hivi \others{vinaonyesha ukweli kwamba utimilifu wote wa Yeye anayejaza yote katika yote, hukaa ndani ya kila nafsi ya Mungu wa Utatu}. Kusema kwamba Ellen White alikubaliana na maoni kama hayo ni upotoshaji mkubwa.

Muktadha wa nukuu hii muhimu hutokeza maswali muhimu. Kwa nini nabii wa Mungu anarejelea viwakilishi vinavyoonyesha \others{tri-personality of the one God} kama \egwinline{spiritualistic representations}, ambavyo vinaonyesha hisia kwamba \egwinline{is not to be trusted}? Au kwa nini nabii wa Mungu anarejelea viwakilishi ambavyo \others{represent the living personalities of the living God} kama \egwinline{spiritualistic representations}? Au kwa nini nabii wa Mungu, anaporejelea viwakilishi ambavyo \others{illustrate the truth that all the fullness of Him who filleth all in all, dwells in each person of Triune God}, anavirejelea kama \egwinline{spiritualistic representations}? Viwakilishi hivi vyote vya kimizimu vinaonyesha hisia kwamba \egwinline{is not to be trusted}. Hisia hii dhahiri ni ya utatu.

Dada White anaendelea kufuata hoja za Boardman na kurekebisha makosa.

\begin{table}[H]
\centering
\renewcommand{\arraystretch}{1.5}
\setlength{\tabcolsep}{15pt}
\resizebox{\textwidth}{!}{
\begin{tabular}{|p{0.4\textwidth}|p{0.4\textwidth}|}
\hline
\multicolumn{1}{|c|}{\textbf{William Boardman}} & \multicolumn{1}{|c|}{\textbf{Ellen G. White}} \\ \hline
\othersQuote{Baba ni utimilifu wa Uungu \textbf{bila kuonekana}, \textbf{\underline{bila umbo}}, ambaye \textbf{hakuna kiumbe aliyemwona \underline{wala anaweza kumwona}}.}[p.100][https://archive.org/details/higherchristian02boargoog/page/n108/]

\othersQuote{Baba ni utimilifu wote wa Uungu \textbf{ASIYEONEKANA}.}[p.105][https://archive.org/details/higherchristian02boargoog/page/n112/] & 
\egw{Baba ni utimilifu wote wa Uungu \textbf{\underline{kimwili}}, naye \textbf{haonekani kwa macho ya mwanadamu}.}[Ms21-1906.9; 1906][https://egwwritings.org/read?panels=p9754.15] \\ \hline
\end{tabular} 

\end{table}

Kwa Boardman, Baba hana umbo wala mwili na haonekani na viumbe vyote. Kwa Dada White, Baba ana umbo na mwili na haonekani tu na wanadamu wanaokufa.\footnote{Wakati Dada White anaposema kuhusu wanadamu wanaokufa, anazungumzia wanadamu waliotia doa na dhambi. Baada ya kurejeshwa kwa ubinadamu, wakati wa ufufuo, Kristo atawapa uzima wake usio na kifo kwa watoto wake. Kwa maelezo zaidi soma \href{https://egwwritings.org/?ref=en_RH.July.5.1887.par.5}{EGW, RH July 5, 1887, par. 5; 1887}.}


Nukuu hii ni mojawapo ya manukuu ya moja kwa moja kuhusu \emcap{personality of God}. \egwinline{The Father is all the fullness of the Godhead \textbf{bodily}}[Ms21-1906.9; 1906][https://egwwritings.org/read?panels=p9754.16].


Inaweza kuwa inachanganya kwa mtu kwamba Baba ni utimilifu wote wa Uungu kimwili kwa sababu katika \textit{Wakolosai 2:9}, inapomtaja Yesu, imeandikwa kwamba \bible{in him dwelleth all the fulness of the Godhead bodily.} Maandiko hayajipingani. \textit{Wakolosai 2:9} haifanyi hivyo kumtenga Baba kuwa utimilifu wote wa Uungu kimwili. Maeneo mbalimbali katika Biblia hueleza Baba kuwa na mwili (\textit{umbo: Danieli 7:9,10; Ufunuo 4:2,3; 1 Wafalme 22:19-22; umbo: Yohana 5:37}). Ana sura ya mwanadamu (\textit{Ezekieli 1:26-28}). Ana uso (\textit{Kutoka 33:20; Mathayo 18:10; Ufunuo 22:3, 4}). Hata hivyo, Biblia ni kimya kote juu ya asili ya dutu yake. Biblia inatufundisha kwamba \bible{\textbf{The secret things belong unto the LORD our God}: \textbf{but those things which \underline{are revealed} belong unto us and to our children for ever}, that we may do all the words of this law}[Kumbukumbu la Torati 29:29]. Inafunuliwa kwetu kwamba Baba ana mwili, Yeye ni utimilifu wote wa Uungu kimwili. Pia, inafunuliwa kwamba ndani ya Yesu pia unakaa utimilifu wote wa Uungu kwa jinsi ya kimwili, kwa sababu \bible{it pleased the Father that in him should all fulness dwell}[Wakolosai 1:19]. Hii sio utata kwa vyovyote vile kwa sababu Mwana ni \bible{the \textbf{express image of \underline{His person}}}[Waebrania 1:3].


\begin{table}[H]
\centering
\renewcommand{\arraystretch}{1.5}
\setlength{\tabcolsep}{15pt}
\resizebox{\textwidth}{!}{
\begin{tabular}{|p{0.4\textwidth}|p{0.4\textwidth}|}
\hline
\multicolumn{1}{|c|}{\textbf{William Boardman}} & \multicolumn{1}{c|}{\textbf{Ellen G. White}} \\ \hline
\othersQuote{Mwana ni utimilifu wa Uungu \textbf{aliye katika mwili, ili viumbe vyake wamwone, wamjue, na wamtumaini}.}[p.100][https://archive.org/details/higherchristian02boargoog/page/n108/]

\othersQuote{Mwana ni utimilifu wote wa Uungu \textbf{ALIYEJIDHIHIRISHA}.}[p.105][https://archive.org/details/higherchristian02boargoog/page/n112/] & 
\egw{Mwana ni utimilifu wote wa Uungu \textbf{uliodhihirishwa}. Neno la Mungu linamtangaza kuwa ‘\textbf{mfano halisi wa nafsi yake}’. ‘Kwa maana Mungu aliupenda ulimwengu hata akamtoa \textbf{Mwanawe pekee aliyezaliwa}, ili kila amwaminiye asipotee, bali awe na uzima wa milele’. \textbf{Hapa inaonyeshwa \underline{nafsi ya Baba}}.}[Ms21-1906.10; 1906][https://egwwritings.org/read?panels=p9754.17] \\ \hline
\end{tabular}
}
\end{table}


Dada White alizingatia \emcap{personality of God}, ambao ni ubinafsi wa Baba. Katika Kristo, ambaye \egwinline{begotten in the express image of the Father's person}[ST May 30, 1895, par. 3; 1895][https://egwwritings.org/read?panels=p820.12891], anaonyeshwa ubinafsi wa Baba. Kwa njia sawa na kwamba Yesu ni Nafsi, hivyo ni Baba. Ubora au hali ya Kristo kuwa mtu ni sifa sawa au hali ya Baba kuwa Nafsi. Kama vile Kristo ni kiumbe binafsi, ndivyo Baba alivyo. Kama vile utimilifu wote wa Uungu unakaa kimwili ndani ya Kristo, vivyo hivyo ndani ya Baba, kwa sababu Kristo amezaliwa moja kwa moja kulingana na ubinafsi wa Baba. Ndani yake unaonyeshwa ubinafsi wa Baba. Hitimisho hili rahisi limethibitishwa na Maandiko katika Yohana 3:16 na Waebrania 1:3.


Je, mantiki hiyo hiyo, ya umbile la Baba na Mwana, inatumika kwa Roho Mtakatifu? Akizungumzia Roho Mtakatifu, Dada White anaendelea:


\egw{\textbf{Msaidizi ambaye Kristo} aliahidi kumtuma baada ya kupaa mbinguni, \textbf{ni Roho \underline{ndani} ya utimilifu wote wa Uungu}, akidhihirisha nguvu ya neema ya kimungu kwa wote wanaompokea na kumwamini Kristo kama Mwokozi binafsi.}[Ms21-1906.11; 1906][https://egwwritings.org/read?panels=p9754.18]


Dada White anaonyesha tofauti kati ya Baba na Mwana ambao, kibinafsi, \textbf{ni} utimilifu wote wa Uungu, na Roho aliye \textbf{ndani ya} utimilifu wote wa Uungu. Hii ni tofauti kubwa na mawazo ya William Boardman, ambapo wote watatu ni utimilifu wa Uungu. Dada White hafuati mtindo huu wa utatu. Ufafanuzi ni rahisi katika nuru ya \emcap{umbile la Mungu} na wa Kristo. Roho Mtakatifu ni roho, na roho hukaa \textbf{ndani} ya nyama/mwili. Roho Mtakatifu yuko \textbf{ndani ya} utimilifu wote wa Uungu\footnote{Angalia nukuu kutoka \href{https://egwwritings.org/?ref=en_Ms128-1897.13&para=5426.19}{{EGW, Ms128-1897.13; 1897}}, ambapo Dada White anasema kwamba Baba na Mwana ni Uungu kamili.}.


Hatimaye, nukuu inaendelea hadi sehemu yake maarufu zaidi:



\begin{table}[H]
    \centering
    \renewcommand{\arraystretch}{1.5}
    \setlength{\tabcolsep}{15pt}
    \resizebox{\textwidth}{!}{
    \begin{tabular}{|p{0.4\textwidth}|p{0.4\textwidth}|}
    \hline
    \multicolumn{1}{|c|}{\textbf{William Boardman}} & \multicolumn{1}{c|}{\textbf{Ellen G. White}} \\ \hline
    \othersQuote{\textbf{Baba} ndiye utimilifu wote wa Uungu USIOONEKANE.}

\othersQuote{\textbf{Mwana} ndiye utimilifu wote wa Uungu ULIODHIHIRISHWA.}

\othersQuote{\textbf{Roho} ndiye utimilifu wote wa Uungu AKIDHIHIRISHA.}

\othersQuote{\textbf{Nafsi} hao sio ofisi tu, au njia za ufunuo, \textbf{bali Nafsi walio hai wa Mungu aliye hai}.}[p.105][https://archive.org/details/higherchristian02boargoog/page/n112/] & 
    \egw{Kuna \textbf{nafsi tatu hai wa watatu wa mbinguni}; kwa jina la hizi nguvu tatu kuu—\textbf{Baba, Mwana, na Roho Mtakatifu}—wale wanaompokea Kristo kwa imani iliyo hai wanabatizwa, na nguvu hizi zitashirikiana na raia watiifu wa mbinguni katika juhudi zao za kuishi maisha mapya katika Kristo.}[Ms21-1906.11; 1906][https://egwwritings.org/read?panels=p9754.18] \\ \hline
    \end{tabular}
    }
    \end{table}


Kwa kuzingatia muktadha wa kitabu cha William Boardman, tunaona tofauti kubwa kati ya \others{nafsi tatu hai za \textbf{Mungu mmoja aliye hai}}, ambayo ni hisia ya utatu, na \egwinline{nafsi tatu hai wa \textbf{watatu wa mbinguni}}, ambayo ni kwa mujibu wa ukweli juu ya \emcap{umbile la Mungu}.


Neno ‘\textit{watatu}’ linaonyesha tu kikundi cha watu watatu. \textit{“Watatu wa mbinguni”} wanawakilishwa na Baba, Mwana, na Roho Mtakatifu. Lakini, kinyume na dhana maarufu, hawafanyi Mungu mmoja aliye hai. “Tatu-kwa-moja” na “moja-katika-tatu” ni dhana ambazo huondoa \emcap{umbile la Mungu}. Hii ndiyo sababu Dada White alitaja hisia za utatu kama hisia kwamba \egwinline{hazifai kuaminiwa}[Ms21-1906.8; 1906][https://egwwritings.org/read?panels=p9754.15].


Dada White hakuwahi kufuata mtindo wowote wa utatu—si kwa maneno wala usemi, wala kwa hisia za ndani. Kuna juhudi za utafiti takriban zisizo na jitihada tunazopendekeza uyafanye: katika maandishi ya Ellen White, tafuta maneno ya kawaida ya utatu kama vile “\textit{watatu ni mmoja},” “\textit{mmoja ni watatu},” “\textit{mmoja kati ya watatu},” “\textit{tatu katika moja},” au yoyote kati ya mabadiliko yanayowezekana. Katika kazi yake yenye kuvutia hutapata hata tukio moja la yoyote kati ya haya, achilia mbali neno ‘\textit{trinity}’ likielezea Mungu wetu\footnote{Kuna tukio moja tu, katika maandishi ya Ellen White, ya neno ‘\textit{trinity}’ likirejea \egw{tamaa ya mwili, tamaa ya macho na kiburi cha uzima}[Lt43-1898.25; 1898][https://egwwritings.org/read?panels=p4806.31]}. Yeye kamwe hakutumia vishazi hivi ambavyo ni muhimu kuelezea hisia za utatu. Kwa kuchunguza nukuu ifuatayo, tunaweza kuona kwa nini hakusema kwamba Mungu ni utatu.


\egw{Mada ya \textbf{\underline{dhana} kuhusu \underline{Umbile la Mungu} \underline{hatutahatarisha} kueleza}, \textbf{\underline{isipokuwa katika lugha ya Neno ambayo inawakilisha Umbile lake}}. Hakuna mjadala juu ya swali hili \textbf{isije Mungu akatoa ufunuo usiokosewa wa \underline{kile Yeye ni}} ambao utamzima yule anayethubutu kujaribu katika \textbf{nadharia zake za dhana za kukisiwa}, kama wengine walivyothubutu kufanya katika kufungua sanduku kuona kile kilichomo ndani yake kama nguvu zake na jinsi Mungu alivyodhihirishwa. Watu hao waliuawa kwa udadisi wao wa kisayansi.}[17LtMs, Ms 223, 1902, par. 16][https://egwwritings.org/read?panels=p14067.9124037&index=0]


Je, ulifahamu hilo? Hakuna mjadala juu ya swali la kile Mungu ni, \egwinline{isije Mungu akatoa ufunuo usiokosewa} wa \egwinline{kile Yeye ni}. Kusema “Mungu ni \_\_\_\_\_\_\_“, pengo hilo lazima lijazwe na \egwinline{lugha ya Neno ambayo inawakilisha Umbile lake.} Biblia kwa uwazi inafundisha kwamba Mungu ni kiumbe binafsi, cha kiroho—ukweli uliothibitishwa na Kristo mwenyewe katika mafunuo yake kwa Ellen White. Hii inaendana na lugha ya kibiblia inayoelezea Umbile la Mungu. Hata hivyo, kulingana na taarifa hapo juu, tunaweza kusema “\textit{Mungu ni utatu}?” La! Hiyo si \egwinline{lugha ya Neno ambayo inawakilisha Umbile lake.} Kwa hiyo, katika muktadha uliochunguzwa, tunaweza kuhitimisha kwa usalama kwamba, mtazamo wa Utatu kuhusu Mungu ni sehemu ya \egwinline{nadharia za dhana za kukisia} za \egwinline{kile Yeye ni}.


Hii ikiwa imesemwa, neno \egwinline{Watatu wa Mbinguni} si ufafanuzi wa kile Mungu ni. Mungu wetu ni Baba—si \egwinline{Watatu wa Mbinguni.} Neno Watatu wa Mbinguni halitumiki kama mbadala wa wazo la Utatu la \textit{nafsi tatu hai za Mungu mmoja}. Hii inakuwa wazi, tunapoangalia muktadha. Ellen White aliagizwa kutuonya dhidi ya hisia za Utatu, si kuzitumaini. Hakuwa akiziunga mkono.


Ingawa mifano Ellen White aliyonukuu haikuwa kutoka kwa Dk. Kellogg, inaonekana kwamba wafuasi wa Kellogg, kama si Kellogg mwenyewe, walikuwa wakimtetea kwa hisia za William Boardman. Hatuna data ya moja kwa moja kuthibitisha hili, lakini tunajua kwamba Dk. Kellogg aliibua \others{upande wa kitheolojia wa maswali ya \textbf{utatu na mambo yote ya aina hiyo}.}[Interview, J. H. Kellogg, G. W. Amadon and A. C. Bourdeau, October 7th 1907 held at Kellogg's residence][https://archive.org/details/KelloggVs.TheBrethrenHisLastInterviewAsAnAdventistoct71907/page/n37] Aya tatu za mwisho katika hati ya watatu wa mbinguni \href{https://egwwritings.org/?ref=en_Ms21-1906&para=9754.1}{(Ms21-1906; 1906)} zinafunua uhusiano na Dk. Kellogg, ambao ni “ushahidi dhahiri” wa msimamo wa utatu wa Dk. Kellogg.


\egw{Ninaandika haya kwa sababu wakati wowote maisha yangu yanaweza kumalizika. \textbf{Isipokuwa kuna njia ya kujiondoa kutoka kwa ushawishi ambao Shetani ametayarisha, na \underline{kuhuisha shuhuda ambazo Mungu ametoa}, roho zitaangamia katika udanganyifu wao. Watakubali upotofu baada ya upotofu na hivyo itaendeleza mfarakano utakaokuwepo daima hadi wale ambao wamedanganywa kuchukua \underline{msimamo wao kwenye jukwaa sahihi}}. Elimu hii yote ya juu ambayo inapangwa itazimwa; maana ni uwongo. Kadiri elimu ya wafanyikazi wetu inavyokuwa ya \underline{wazi na nyepesi}, ndivyo inavyopungua uhusiano walio nao na watu ambao Mungu hawaongozi, ndivyo zaidi yatakavyotimizwa. Kazi itafanywa katika usahili wa utauwa wa kweli, na nyakati za kale, za kale zitarudi, chini ya uongozi wa Roho Mtakatifu, maelfu ya watu waliongoka kwa siku moja. Wakati ukweli katika usahili wake unapoishi kila mahali, basi Mungu atafanya kazi kupitia malaika wake kama alivyofanya kazi siku ya Pentekoste, na mioyo itabadilishwa hivyo kwa uamuzi kwamba kutakuwa na udhihirisho wa ushawishi wa ukweli wa kweli, kama ulivyo kuwakilishwa katika kushuka kwa Roho Mtakatifu.}[Ms21-1906.18; 1906][https://egwwritings.org/read?panels=p9754.25]


\egwnogap{Roho Mtakatifu hajawahi na kamwe hataachana na kazi ya umishonari ya matibabu katika siku zijazo kutoka kwa huduma ya injili. Haiwezi kutenganishwa. Imefungwa pamoja na Yesu Kristo, huduma ya Neno na uponyaji wa wagonjwa ni kitu kimoja.}[Ms21-1906.19; 1906][https://egwwritings.org/read?panels=p9754.26]


\egwnogap{Sura ya hamsini na nane ya Isaya ina maagizo ya leo. \textbf{‘Lieni kwa sauti kubwa, msiache, Paza sauti yako kama tarumbeta, uwahubirie watu wangu kosa lao, na nyumba ya Yakobo dhambi yao.’ Mungu hamkubali \underline{Dk. Kellogg kama mtenda kazi Wake}, isipokuwa atakubali sasa kutengana na Shetani}. Kazi hiyo isingezuiwa, kama ilivyokuwa zamani miaka kadhaa, \textbf{ikiwa Dk. Kellogg angekuwa mtu aliyeongoka. ‘Njoo,’ naita, ‘toka na ujitenge naye na washirika wake aliowachacha.’ Sasa ninatoa ujumbe ambao Mungu amenipa, kuwapa wote wanaodai kuamini ukweli, \underline{‘jiondoeni miongoni mwao, na kujitenga},’ sivyo dhambi yao katika kuhalalisha makosa na kutunga udanganyifu utaendelea kuwa uharibifu wa roho}. Hatuwezi kumudu kuwa upande mbaya. Hatuwezi kumudu kuficha ukweli kwa matatizo ya kisayansi. Tunaomba hilo liamuliwe mabadiliko yafanywe na hakuna vikwazo tena vinavyowekwa mbele ya miguu ya watu wa Mungu. Hebu kila nafsi ivae viatu vya injili. \textbf{Hebu kila nafsi iombe na kufanya kazi, wakiweka miguu yao juu ya \underline{msingi ambao Kristo aliweka} katika kutoa maisha yake kwa ajili ya uzima wa dunia}.}[Ms21-1906.20; 1906][https://egwwritings.org/read?panels=p9754.27]


Nukuu ya watatu wa mbinguni ilikuwa sehemu ya utata wa Kellogg. Huu ni ushahidi kwamba utata wa Kellogg ulitia ndani fundisho la Utatu. Tunaambiwa tujitenge \egwinline{na ushawishi wa Shetani} na kuhuisha \egw{ushuhuda ambao Mungu ametupa}, ama sivyo roho zetu zitaangamia katika udanganyifu. Athari hizi na udanganyifu hutoka kwa waamini utatu kama vile \textit{William Boardman} na \textit{Dk. John H. Kellogg}. Anatuelekeza nyuma ili kuweka miguu yetu juu ya msingi huo uliojengwa na Fundi Stadi.\footnote{\href{https://egwwritings.org/?ref=en_SpTB02.54.2&para=417.276}{EGW, SpTB02 54.2; 1904}}


Tunatumaini kwamba muktadha huu unafichua hadithi ya uongo ya kuunga mkono kwa Ellen White kwa fundisho la Utatu, inayoenezwa na wasomi wetu wa Kiadventista. Dk. Kellogg alikuwa katika ukengeufu kwa kuondoka kwenye msingi wa imani yetu, na fundisho la Utatu lilikuwa haki yake. Kwa data kama hiyo akilini, mtu lazima aulize: Ikiwa Utatu ulikuwa kweli, na Ellen White aliuunga mkono, na Utatu huu “wa kweli” ulichanganywa na makosa ya Dk. Kellogg, tungetarajia kwamba angetenganisha Utatu na makosa. Lakini hiki sio alichofanya. Badala yake, alituonyesha kwa uthabiti kurudi kwenye msingi wa imani yetu, ambapo tulikuwa na mafundisho wazi juu ya uwepo na \emcap{Umbile la Mungu}. Lakini kwa suala la Utatu, alibeba kwa uaminifu ujumbe kutoka Mbinguni: “\textit{\textbf{Nimeagizwa kusema}, hisia za wale wanaotafuta \textbf{mawazo ya utatu hazifai kuaminiwa}}.”


% The Heavenly Trio

\begin{titledpoem}
    
    \stanza{
        In heaven’s realm, where truths unfold, \\
        A message clear, so brave and bold. \\
        God spoke through Ellen, clear and bright, \\
        Revealing depths of heavenly light.
    }

    \stanza{
        Misunderstood by some who read, \\
        Her words of God that all must heed. \\
        Not as triune, but trio three \\
        Distinct as persons, heavenly.
    }

    \stanza{
        The Father, not a formless feel, \\
        Invisible to us, yet real. \\
        He is the fullness, all complete, \\
        The Godhead, bodily, concrete.
    }

    \stanza{
        The Son, God’s fullness, manifest \\
        In Him, divinity does rest. \\
        God’s character, seen in His face, \\
        In Christ, we see His Father’s grace.
    }

    \stanza{
        The Spirit, in all fullness dwells, \\
        A mystery nature, Ellen tells. \\
        With forms, the Father and His Son \\
        With Them, in Spirit, we are one.
    }

    \stanza{
        Distinct and clear, Their roles unfold, \\
        The Father, Son, in form behold. \\
        Yet present everywhere we find, \\
        Their Spirit shows Their heart and mind.
    }

    \stanza{
        God’s message true, from up above. \\
        Reveals to us the Father’s love. \\
        To know this truth about our God— \\
        It lights the path that we must trod.
    }

    \stanza{
        Dear Ellen’s words, in context found, \\
        Reveal a truth that’s so profound \\
        Not trinity did she embrace, \\
        But trio persons in their place.
    }

    \stanza{
        The pillar stands, our platform firm, \\
        God’s personality we learn. \\
        The trio that is heavenly, \\
        Exposes falsehood—trinity.
    }
    
\end{titledpoem}
