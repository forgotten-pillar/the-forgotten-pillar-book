\qrchapter{https://forgottenpillar.com/rsc/en-fp-chapter18}{The Heavenly Trio}


\qrchapter{https://forgottenpillar.com/rsc/en-fp-chapter18}{Watatu wa Mbinguni}


So far we have seen the evidence that Ellen White knew about Dr. Kellogg's trinitarian sentiments, and we have seen how she responded to it. She always uplifted the truth on the presence and the \emcap{personality of God}, and called to come back to the foundation of our faith—\emcap{Fundamental Principles}. However, when Adventist scholars discuss the doctrine of the Trinity and Ellen White, they do not approach it in the same manner as Ellen White did. The \emcap{Fundamental Principles} together with the doctrine on the \emcap{personality of God} is downplayed, and the twisted story is presented that Ellen White was trinitarian and responsible for the church's acceptance of the Trinity doctrine into our ranks. We want to challenge this twisted story by looking at the evidence that is often used to support this false narrative.


Mpaka sasa tumeona ushahidi kwamba Ellen White alijua kuhusu maoni ya utatu ya Dk. Kellogg, na tumeona jinsi alivyojibu. Yeye daima alitukuza ukweli juu ya uwepo na \emcap{Umbile la Mungu}, na kuita kurudi kwenye msingi wa imani yetu—\emcap{Kanuni za Kimsingi}. Hata hivyo, wakati wasomi wa Waadventista wanapojadili fundisho la Utatu na Ellen White, hawakaribi kwa njia ile ile kama Ellen White alivyofanya. \emcap{Kanuni za Kimsingi} pamoja na fundisho juu ya \emcap{Umbile la Mungu} zinapuuzwa, na hadithi iliyopotoshwa inawasilishwa kwamba Ellen White alikuwa mwamini wa utatu na alihusika na kanisa kukubali fundisho la Utatu katika safu zetu. Tunataka kupinga hadithi hii iliyopotoshwa kwa kuangalia ushahidi ambao mara nyingi hutumiwa kuunga mkono simulizi hii ya uongo.


One of the most prominent quotations to support the claim that Sister White was responsible for accepting the Trinity doctrine into our ranks is her writings and comments on Matthew 28:19\footnote{\bible{Go ye therefore, and teach all nations, baptizing them in the name of the Father, and of the Son, and of the Holy Ghost}[Matthew 28:19]}. The most prominent quotation to stand out in defense of the Trinity doctrine is “\textit{the Heavenly Trio}” quotation:


Mojawapo ya nukuu maarufu zaidi za kuunga mkono dai kwamba Dada White alihusika kwa kukubali fundisho la Utatu katika safu zetu ni maandishi na maoni yake juu ya Mathayo 28:19\footnote{\bible{Basi, enendeni, mkawafanye mataifa yote kuwa wanafunzi, mkiwabatiza kwa jina la Baba, na Mwana, na Roho Mtakatifu}[Mathayo 28:19]}. Nukuu maarufu zaidi ya kutetea fundisho la Utatu ni nukuu ya “\textit{Watatu wa Mbinguni}”:


\egw{\textbf{There are \underline{three living persons} of the \underline{heavenly trio}}; in the name of these three great powers—\textbf{the Father, the Son, and the Holy Spirit}—those who receive Christ by living faith are baptized, and these powers will co-operate with the obedient subjects of heaven in their efforts to live the new life in Christ...}[Ev 615.1; 1946][https://egwwritings.org/read?panels=p30.3407]


\egw{\textbf{Kuna \underline{nafsi tatu hai} wa \underline{watatu wa mbinguni}}; kwa jina la hawa nguvu tatu wakuu—\textbf{Baba, Mwana, na Roho Mtakatifu}—wale wanaompokea Kristo kwa imani iliyo hai wanabatizwa, na nguvu hizi zitashirikiana na raia watiifu wa mbinguni katika juhudi zao za kuishi maisha mapya katika Kristo...}[Ev 615.1; 1946][https://egwwritings.org/read?panels=p30.3407]


To reiterate, this quotation is often cited to argue that Sister White defended and advocated the Trinity doctrine. But, if we take a look at this quotation in its literary context, we see that within the quotation itself she actually \textit{refuted} this doctrine and exalted the truth on the \emcap{personality of God}. To some this is a ludicrous claim, but we invite you to make your judgment based on presented data. Let us examine the context of this quotation.


Kurudia, nukuu hii mara nyingi inatajwa kusema kwamba Dada White alitetea na kudumisha fundisho la Utatu. Lakini, tukiitazama nukuu hii katika muktadha wake wa kifasihi, tunaona kwamba ndani ya nukuu yenyewe kwa kweli \textit{alikanusha} fundisho hili na kutukuza ukweli juu ya \emcap{Umbile la Mungu}. Kwa wengine hili ni dai la kichekesho, lakini tunakukaribisha ufanye hukumu yako kulingana na data iliyowasilishwa. Hebu tuchunguze muktadha wa nukuu hii.


\egw{I am instructed to say, \textbf{The sentiments} of those who are searching for advanced scientific ideas \textbf{\underline{are not to be trusted}}. Such representations as the following are made: ‘\textbf{The Father is as the light invisible; the Son is as the light embodied; the Spirit as the light shed abroad.}’ ‘\textbf{The Father is like the dew, invisible vapor; the Son is like the dew gathered in beauteous form; the Spirit is like the dew fallen to the seat of life.}’ Another representation: ‘\textbf{The Father is like the invisible vapor. The Son is like the leaden cloud. The Spirit is rain fallen and working in refreshing power.}’}[Ms21-1906.8; 1906][https://egwwritings.org/read?panels=p9754.15]


\egw{Nimeagizwa kusema, \textbf{Maoni} ya wale wanaotafuta mawazo ya sayansi ya hali ya juu \textbf{\underline{hayafai kuaminiwa}}. Uwakilishi kama ufuatao unafanywa: ‘\textbf{Baba ni kama nuru isiyoonekana; Mwana ni kama nuru ilivyo; Roho ni kama nuru inavyomwagika nje.}’ ‘\textbf{Baba ni kama umande, mvuke usioonekana; Mwana ni kama umande uliokusanywa kwa umbo la kupendeza; Roho ni kama umande ulioanguka kwenye kiti cha uzima.}’ Uwakilishi mwingine: ‘\textbf{Baba ni kama mvuke usioonekana. Mwana ni kama wingu la risasi. Roho ni mvua iliyonyesha na kufanya kazi katika nguvu za kuburudisha.}’}[Ms21-1906.8; 1906][https://egwwritings.org/read?panels=p9754.15]


What sentiments are not to be trusted? The data suggest that those sentiments are trinitarian ideas of \textit{one God in three persons}. How do we know that? We see in the literary context of the representations Sister White was quoting. Contrary to the popular belief that she was referencing the “\textit{false}” trinity expressed by Dr. Kellogg,\footnote{Whidden, Woodrow W, et al. \textit{The Trinity : Understanding God’s Love, His Plan of Salvation, and Christian Relationships}. Hagerstown, Md, Review And Herald Pub. Association, 2002, p. 216.} she was actually referencing trinitarian idea of \textit{three living persons of one living God}, advocated by William Boardman, in his book “Higher Christian Life”, which she quoted. The context matters. The context of the quotations she quoted, shows that the representations of the Father, the Son, and the Holy Spirit are serving to illustrate the sentiment of three living persons of one God. That is the sentiment we have been clearly instructed by God, not to trust. Let the data be its own interpreter.


Ni maoni gani ambayo hayafai kuaminiwa? Data inaonyesha kwamba maoni hayo ni mawazo ya utatu ya \textit{Mungu mmoja katika nafsi tatu}. Tunajuaje hilo? Tunaona katika muktadha wa kifasihi wa uwakilishi ambao Dada White alinukuu. Kinyume na imani maarufu kwamba alikuwa anarejelea “\textit{utatu wa uongo}” ulioelezwa na Dk. Kellogg,\footnote{Whidden, Woodrow W, et al. \textit{The Trinity : Understanding God's Love, His Plan of Salvation, and Christian Relationships}. Hagerstown, Md, Review And Herald Pub. Association, 2002, p. 216.} kwa kweli alikuwa anarejelea wazo la utatu la \textit{nafsi tatu hai za Mungu mmoja aliye hai}, lililotetewa na William Boardman, katika kitabu chake “Higher Christian Life”, ambacho alinukuu. Muktadha ni muhimu. Muktadha wa nukuu alizozinukuu, unaonyesha kwamba uwakilishi wa Baba, Mwana, na Roho Mtakatifu unatumika kuelezea maoni ya nafsi tatu hai za Mungu mmoja. Hayo ndiyo maoni ambayo tumeagizwa wazi na Mungu, kutoyaamini. Acha data ifasiri yenyewe.


\section*{The Higher Christian Life, William Boardman}


\section*{Maisha ya Juu ya Kikristo, William Boardman}


Ellen White owned William Boardman's book “Higher Christian Life.” It was a good book about Christian sanctification, but in it there was trinitarian sentiment, which Sister White was particularly instructed by God to call out. This is another instance of evidence where we see that Ellen White was familiar with the trinitarian stance, and she was addressing it directly. Let's get familiar with the trinitarian sentiments promoted by William Boardman.


Ellen White alikuwa na kitabu cha William Boardman “Higher Christian Life.” Kilikuwa kitabu kizuri kuhusu utakaso wa Kikristo, lakini ndani yake kulikuwa na maoni ya utatu, ambayo Dada White aliagizwa na Mungu kuyataja. Hii ni mfano mwingine wa ushahidi ambapo tunaona kwamba Ellen White alikuwa anafahamu msimamo wa utatu, na alikuwa anaukabili moja kwa moja. Hebu tujue maoni ya utatu yaliyotangazwa na William Boardman.


Speaking of Triune God, William Boardman writes:


Akizungumzia Mungu wa Utatu, William Boardman anaandika:


\othersQuote{And then, again, the Father is the author and planner of salvation through faith in his Son; and when we trust in his Son we honor the Father, because we accept of his plan of salvation for us, justify his wisdom, and act in accordance with his will in the matter. \textbf{A glance at the official and essential relations of the persons of the Holy Trinity to each other and to us, may throw additional light upon our pathway}. Upon this subject flippancy would border upon blasphemy. It is holy ground. He who ventures upon it may well tread with unshod foot, and uncovered head bowed low.}[William Boardman, The Higher Christian Life, p. 99; 1858][https://archive.org/details/higherchristian02boargoog/page/n106/]


\othersQuote{Na kisha, tena, Baba ndiye mwanzilishi na mpangaji wa wokovu kupitia imani katika Mwana wake; na tunapomtumaini Mwana wake tunamheshimu Baba, kwa sababu tunakubali mpango wake wa wokovu kwa ajili yetu, thibitisha hekima yake, na kutenda kulingana na mapenzi yake katika jambo hilo. \textbf{Mtazamo wa mahusiano rasmi na muhimu ya nafsi za Utatu Mtakatifu kwa kila mmoja wao na kwetu, inaweza kutupa nuru ya ziada kwenye njia yetu}. Juu ya somo hili kurupuka itapakana na kufuru. Ni ardhi takatifu. Mwenye kujitosa juu yake anaweza vyema kukanyaga bila viatu, na kichwa kisichofunikwa kimeinamishwa chini.}[William Boardman, The Higher Christian Life, p. 99; 1858][https://archive.org/details/higherchristian02boargoog/page/n106/]


Brother Boardman wants us to take \others{a glance at the official and essential relations} of the three persons of the Holy Trinity. He asserts that \textit{God is one but also three}–\textit{Triune}–by presenting official and essential relations of the persons of the Holy Trinity. His fundamental statement and outline for his thesis is as follows:


Ndugu Boardman anataka tuangalie \others{mahusiano rasmi na muhimu} ya nafsi tatu za Utatu Mtakatifu. Anadai kwamba \textit{Mungu ni mmoja lakini pia watatu}–\textit{Utatu}–kwa kuwasilisha mahusiano rasmi na muhimu ya nafsi za Utatu Mtakatifu. Taarifa yake ya msingi na muhtasari wa tasnifu yake ni kama ifuatavyo:


\othersQuote{\textbf{The Father is fullness of the Godhead \underline{invisibly}, without form, whom no creature hath seen or can see}. \\
\textbf{The Son is the fullness of the Godhead \underline{embodied}, that his creatures may see him, and know him, and trust him}. \\
\textbf{The Spirit is the fullness of the Godhead \underline{in all active workings}, whether of creation, providence, revelation, or salvation, by which God manifests himself to and through the universe}.}[William Boardman, The Higher Christian Life, p. 100][https://archive.org/details/higherchristian02boargoog/page/n108/]


\othersQuote{\textbf{Baba ni utimilifu wa Uungu \underline{usioonekana}, hana umbo, ambaye hakuna kiumbe chochote kimeona au kinaweza kuona}. \\
\textbf{Mwana ni utimilifu wa Uungu \underline{unaomwilishwa}, ili viumbe vyake vimwone, na kumjua, na kumtumaini}. \\
\textbf{Roho ndiye utimilifu wa Uungu \underline{katika utendaji kazi wote}, iwe wa uumbaji, majaliwa, ufunuo, au wokovu, ambao kupitia huo Mungu hujidhihirisha mwenyewe kwa na kupitia ulimwengu}.}[William Boardman, The Higher Christian Life, p. 100][https://archive.org/details/higherchristian02boargoog/page/n108/]


This statement is foundational to his following statements and illustrations. In the following paragraphs, William Boardman gives the biblical motives to illustrate \others{the official and essential relations of the Holy Trinity}—\textit{that is, God being one, but yet three}. He writes:


Kauli hii ni ya msingi kwa kauli na vielelezo vyake vifuatavyo. Katika aya zifuatazo, William Boardman anatoa nia za kibiblia ili kueleza \others{mahusiano rasmi na muhimu ya Utatu Mtakatifu}—\textit{yaani, Mungu akiwa mmoja, lakini bado watatu}. Anaandika:


\othersQuote{Another of the names of Jesus will give the same analogies in a light not less striking - \textbf{The Sun of Righteousness}. \\
All the light of the sun in the heavens was once hidden in the invisibility of primal darkness; and after this, the light now blazing in the orb of day was, when first the command when forth, Let light be! and light was, at most only the diffused haze of the gray dawn of the morn of creation out of the darkness of chaotic night, without form, or body, or centre, or radiance, or glory. But when separated from the darkness and centered in the sun, then in its glorious glitter it became so resplendent that none but the eagle eye could bear to look it in the face. \\
But then again its rays falling aslant through earth’s atmosphere and vapors, gladdens all the world with the same light, dispelling the winter, and the cold, and the darkness; starting Spring forth in floral beauty, and Summer in vernal luxuriance, and Autumn laden with golden treasures for the garner.
\textbf{The Father is as the Light invisible}. \\
\textbf{The Son is as the Light embodied}. \\
\textbf{The Spirit is as the Light shed down}.}[William Boardman, The Higher Christian Life, p. 101,102][https://archive.org/details/higherchristian02boargoog/page/n108/]


\othersQuote{Jina lingine la Yesu litatoa mlinganisho sawa kwa njia ya kushangaza - \textbf{Jua la Haki}. \\
Nuru yote ya jua mbinguni wakati mmoja ilifichwa katika kutoonekana kwa giza kuu; na baada ya hayo, nuru inayowaka sasa katika obi ya mchana ilikuwa, wakati amri ya kwanza ilipotolewa, Nuru iwe! na mwanga ilikuwa, ulionea zaidi ni ukungu wa alfajiri wa kijivu ya asubuhi ya uumbaji kutoka katika giza la usiku wa machafuko, bila umbo, wala mwili, wala katikati, wala mng'ao, au utukufu. Lakini unapotengwa na giza na kuwekwa katikati ya jua, basi katika utukufu wake uling'aa sana hivi kwamba hakuna mtu ila jicho la tai angeweza kustahimili kuiangalia usoni. \\
Lakini tena miale yake ikishuka kwa kasi katika angahewa na mivuke ya dunia, hufurahisha kila kitu ulimwenguni na mwanga ule ule, ukiondoa msimu wa baridi, na baridi, na giza; kuanzia Chipua kwa uzuri wa maua, na Majira ya joto katika mazingira ya asili, na Vuli iliyojaa hazina za dhahabu kwa ghala.
\textbf{Baba ni kama Nuru isiyoonekana}. \\
\textbf{Mwana ni kama Nuru ilivyo}. \\
\textbf{Roho ni kama Nuru iangazavyo}.}[William Boardman, The Higher Christian Life, p. 101,102][https://archive.org/details/higherchristian02boargoog/page/n108/]


This illustration of the Sun of Righteousness shows that God the Father, who is \textit{the fullness of the Godhead invisible,} can be symbolically illustrated as a Light that \others{was once hidden in the invisibility of primal darkness}. The Son, who is \textit{the fullness of the Godhead embodied}, is like a Light that is embodied in \others{the morn of creation}. The Holy Spirit, who is \textit{the fullness of the Godhead in all active workings}, is like a \others{Light shed down}. William Boardman gives us another similar illustration to clarify the \others{official relations of the persons of the Godhead}:


Kielelezo hiki cha Jua la Haki kinaonyesha kwamba Mungu Baba, ambaye ni \textit{utimilifu wa Uungu usioonekana,} anaweza kufananishwa kiishara na Nuru ambayo \others{ilifichwa katika kutoonekana kwa giza kuu}. Mwana, ambaye ni \textit{utimilifu wa Uungu unaomwilishwa}, ni kama Nuru iliyofumbatwa ndani ya \others{asubuhi ya uumbaji}. Roho Mtakatifu, ambaye ni \textit{utimilifu wa Uungu katika utendaji kazi wote}, ni kama \others{Nuru iliyomwagika}. William Boardman anatupa kielelezo kingine sawa kufafanua \others{mahusiano rasmi ya nafsi za Uungu}:


\othersQuote{One of the similies for blessed influences of the Spirit, \textbf{while giving the self-same official relations of the persons of the Godhead, to each other and to us}, may illustrate them still further,—\textbf{The Dew},—\textbf{The dew of Hermon} - the dew on the mown meadow. Before the dew gathers at all in drops, it hangs over all the landscape in visible vapor, omnipresent but unseen. By and by as the light wanes into morning, and as the temperature sinks and touches the dew point the invisible becomes the visible, the embodied; and, as the sun rises, it stands in diamond drops trembling and glittering in the sun’s young beams in pearly beauty upon leaf and flower, over all the face of nature. \\
But now again, a breeze springs up, the breath of heaven is wafted gently along, shaking leaf and flower, and in a moment the pearly drops are invisible angina. But where now? Fallen at the root of herb and flower to impart new life, freshness, vigor to all it touches. \\
\textbf{The Father is like the dew in invisible vapor}. \\
\textbf{The Son is like the dew gathered in beauteous form}. \\
\textbf{The Spirit is like the dew fallen to the seat of life}.}[William Boardman, The Higher Christian Life, p. 102,103][https://archive.org/details/higherchristian02boargoog/page/n110/]


\othersQuote{Mojawapo ya mifano ya ushawishi uliobarikiwa wa Roho, \textbf{huku ukitoa yale yale mahusiano rasmi ya nafsi za Uungu, kwa kila mmoja wao na sisi}, yanaweza kuyaonyesha bado zaidi,—\textbf{Ule Umande},—\textbf{Ule umande wa Hamoni}—umande kwenye mbuga iliyokatwa. Kabla ya umande kukusanyika kwa matone, huning'inia juu ya mazingira yote katika mvuke usioonekana, ulio kila mahali lakini isiyoonekana. Mara kwa mara usiku unapoingia asubuhi, na joto likizidi kuzama na kugusa umande hatua asiyeonekana inakuwa inayoonekana, ilivyo; na jua linapochomoza ndivyo lilivyo limesimama katika matone ya almasi likitetemeka na kuangaa katika miale michanga ya jua katika uzuri wa lulu juu ya jani na maua, juu ya uso wote wa asili. \\
Lakini sasa tena, upepo unavuma, pumzi ya mbinguni ilipepea kwa upole, ikitetemeka jani na maua, na kwa muda mfupi matone ya lulu hayaonekani tena. Lakini wapi sasa? Imeanguka kwenye mzizi wa mimea na ua ili kutoa maisha mapya, upya, nguvu kwa yote yanayogusa. \\
\textbf{Baba ni kama umande katika mvuke usioonekana}. \\
\textbf{Mwana ni kama umande uliokusanywa kwa umbo la kupendeza}. \\
\textbf{Roho ni kama umande ulioanguka kwenye kiti cha uzima}.}[William Boardman, The Higher Christian Life, p. 102,103][https://archive.org/details/higherchristian02boargoog/page/n110/]


The Father, who is \textit{the fullness of the Godhead invisible,} is illustrated by the \others{dew in invisible vapor}. The Son, who is \textit{the fullness of the Godhead embodied}, is illustrated by \others{the dew gathered in beauteous form}. The Spirit, who is \textit{the fullness of the Godhead in all active works}, is illustrated by \others{the dew fallen to the seat of life}. The next illustration that exemplifies the official relations of the three personalities of one God is by another Bible likening—the Rain.


Baba, ambaye ni \textit{utimilifu wa Uungu usioonekana,} anaonyeshwa na \others{umande katika mvuke usioonekana}. Mwana, ambaye ni \textit{utimilifu wa Uungu unaomwilishwa}, anaonyeshwa na \others{umande uliokusanyika kwa umbo la kupendeza}. Roho, ambaye ni \textit{utimilifu wa Uungu katika utendaji kazi wote}, anaonyeshwa na \others{umande ulioanguka kwenye kiti cha uzima}. Kielelezo kinachofuata ambacho ni mfano wa mahusiano rasmi ya nafsi tatu za Mungu mmoja ni kwa mfano mwingine wa Biblia—Mvua.


\othersQuote{\textbf{Yet one more of these Bible likenings} – by no means exhausting them – will not be unwelcome, or useless, - \textbf{the Rain}. \\
Rain, like the dew, floats in invisibility, and omnipresence at the first, over all, around all. Seen by none. While it remains in its invisibility, the earth parches, clods cleave together, the ground cracks open, the sun pours down his burning heat, the winds lift up the dust in circling whirls, and rolling clouds, and famine gaunt and greedy stalks through the land, followed by pestilence and death. By and by, the eager watcher sees the little hand-like cloud rising far out over the sea. It gathers, gathers, gathers; comes and spreads as it comes, in majesty over the whole heavens: - But all is parched and dry and dead yet, upon earth. \\
But now comes a drop, and drop after drop, quicker, faster – the shower, the rain – sweeping on, and giving to earth all the treasures of the clouds – clods open, furrows soften, springs, rivulets, rivers, swell and fill, and all the land is gladdened again with restored abundance. \\
\textbf{The Father is like to the invisible vapor}. \\
\textbf{The Son is as the laden cloud and falling rain}. \\
\textbf{The Spirit is the Rain – fallen and working in refreshing power}.}[William Boardman, The Higher Christian Life, p. 103,104][https://archive.org/details/higherchristian02boargoog/page/n110/]


\othersQuote{\textbf{Bado moja zaidi ya ulinganisho huu wa Biblia} - bila kuwachosha - hautafanyika wasiokubalika, au wasio na maana, - \textbf{Mvua}. \\
Mvua, kama umande, huelea bila kuonekana, na uwepo wa kila mahali hapo kwanza, juu ya yote, karibu na wote. Bila kuonekana na yeyote. Ijapokuwa inabaki katika kutoonekana kwake, dunia tambara, madongoa yanashikana pamoja ardhi hupasuka, jua humwaga joto lake linalowaka, pepo huinua vumbi ndani tufani zinazozunguka, na mawingu, na njaa imepungua, na mabua ya uchoyo katika nchi; ikifuatiwa na tauni na kifo. Mara kwa mara, mlinzi mwenye shauku anaona mkono mdogo wingu likipanda juu sana juu ya bahari. Inakusanya, inakusanya, inakusanya; huja na kuenea ikija, katika enzi juu ya mbingu zote: - Lakini yote ni kavu na kavu na bado kufa, juu ardhi. \\
Lakini sasa inakuja tone, na tone baada ya tone, haraka, kasi - kuoga, mvua - kufagia na kuipa ardhi hazina zote za mawingu - madongoa yanafunguka, matuta yanalainika, chemchemi; mito, mito, hufurika na kujaa, na nchi yote inafurahishwa tena kwa utele uliorejeshwa. \\
\textbf{Baba ni kama mvuke usioonekana}. \\
\textbf{Mwana ni kama wingu lililoelemewa na mvua inayonyesha}. \\
\textbf{Roho ni Mvua - iliyonyesha na kufanya kazi kwa nguvu ya kuburudisha}.}[William Boardman, The Higher Christian Life, p. 103,104][https://archive.org/details/higherchristian02boargoog/page/n110/]


Let's give William Boardman a fair hearing. He is not saying that the Father is \others{invisible vapor}; rather, he uses a metaphor of rain and \others{invisible vapor} to illustrate his main point that the Father is the invisible fullness of the Godhead. So it is with the Son, who, just like rain manifested in leaden clouds, is all the fullness of the Godhead manifested. To ensure his sentiments are not potentially misrepresented, William Boardman clarified his sentiment. This was the very sentiment that Ellen White was instructed by God not to trust:


Hebu tumsikilize William Boardman kwa usawa. Hasemi kwamba Baba ni \others{mvuke usioonekana}; badala yake, anatumia mfano wa mvua na \others{mvuke usioonekana} kuelezea hoja yake kuu kwamba Baba ni utimilifu usioonekana wa Uungu. Ndivyo ilivyo kwa Mwana, ambaye, kama vile mvua inavyodhihirishwa katika mawingu mazito, ni utimilifu wote wa Uungu uliodhihirishwa. Ili kuhakikisha maoni yake hayawakilishwi vibaya, William Boardman alifafanua maoni yake. Haya ndiyo maoni ambayo Ellen White aliagizwa na Mungu kutoyaamini:


\othersQuote{\textbf{These likenings are all imperfect. They rather hide than illustrate \underline{the tri-personality of the one God}, for they are not persons but things, poor and earthly at best, to represent the living personalities of the living God. So much they may do, however, as to illustrate the official relations of each to the others and of each and all to us. And more. They may also illustrate the truth that all the fulness of Him who filleth all in all, dwells in each person of \underline{the Triune God}}. \\
\textbf{The Father is all the fulness of the Godhead INVISIBLE}. \\
\textbf{The Son is all the fulness of the Godhead MANIFESTED}. \\
\textbf{The Spirit is all the fulness of the Godhead MAKING MANIFEST}. \\
\textbf{The persons are not mere offices, or modes of revelation, but living persons of the living God}.}[William Boardman, The Higher Christian Life, p. 104,105][https://archive.org/details/higherchristian02boargoog/page/n112/]


\othersQuote{\textbf{Ulinganisho huu wote si kamilifu. Afadhali wao kujificha kuliko kuonyesha \underline{ubinafsi tatu wa Mungu mmoja}, kwa maana wao si watu bali ni vitu, maskini na wa kidunia kwa ubora zaidi, kuwakilisha nafsi hai za Mungu aliye hai. Sana wanaweza kufanya, hata hivyo, kama ili kuonyesha uhusiano rasmi wa kila mmoja kwa wengine na wa kila mmoja na wote kwetu. Na zaidi. Wanaweza pia kuonyesha ukweli kwamba utimilifu wote wa Yeye anayejaza yote katika yote, hukaa ndani ya kila nafsi ya \underline{Mungu wa Utatu}}. \\
\textbf{Baba ni utimilifu wote wa Uungu USIOONEKANA}. \\
\textbf{Mwana ndiye utimilifu wote wa Uungu ULIODHIHIRISHWA}. \\
\textbf{Roho ndiye utimilifu wote wa Uungu AKIDHIHIRISHA}. \\
\textbf{Watu hao si ofisi tu, au njia za ufunuo, bali ni watu walio hai wa Mungu aliye hai}.}[William Boardman, The Higher Christian Life, p. 104,105][https://archive.org/details/higherchristian02boargoog/page/n112/]


It is crucial to emphasize that when Boardman uses these Bible likenings from nature, he speaks of the illustrations, and not reality. These representations are illustrating his sentiments. In his own admission, that was the sentiment of three \others{living personalities of the living God.} Though these illustrations are imperfect, they may \others{illustrate the official relations} of \others{the tri-personality of the one God} and \others{the truth that all the fullness of Him who filleth all in all dwells in each person of the Triune God.} One God in three persons is the sentiment in question, and that sentiment is common to all types and versions of the trinity doctrine—including our current trinitarian stance in the second point of the Fundamental Beliefs.\footnote{\others{There is \textbf{one God}: Father, Son, and Holy Spirit, \textbf{a unity of three} coeternal \textbf{Persons}…} 2nd point of the Fundamental Beliefs}


Ni muhimu kusisitiza kwamba wakati Boardman anapotumia mifano hii ya Biblia kutoka kwa asili, yeye anazungumza juu ya vielelezo, na sio ukweli. Vielelezo hivi vinaonyesha maoni yake. Katika kukiri kwake mwenyewe, hayo yalikuwa maoni ya \others{nafsi hai za Mungu aliye hai.} Ingawa vielelezo hivi si kamilifu, vinaweza \others{kuonyesha uhusiano rasmi} wa \others{ubinafsi tatu wa Mungu mmoja} na \others{ukweli kwamba utimilifu wote wa Yeye anayejaza yote katika yote, hukaa ndani ya kila nafsi ya Mungu wa Utatu.} Mungu mmoja katika nafsi tatu ni maoni yanayozungumziwa, na maoni hayo ni ya kawaida kwa aina zote na matoleo ya fundisho la utatu—ikiwa ni pamoja na msimamo wetu wa sasa wa utatu katika nukta ya pili ya Mafundisho za Kimsingi.\footnote{\others{Kuna \textbf{Mungu mmoja}: Baba, Mwana, na Roho Mtakatifu, \textbf{umoja wa watatu} wa milele \textbf{Nafsi}…} Nukta ya 2 ya Mafundisho za Kimsingi}


In this brief look at William Boardman's sentiments, it is clear that the sentiments in question which Ellen White was instructed by God to call out, were the sentiments of the Triune God, or \textit{three living persons in the Trinity}. With that data in mind, let's examine Ellen White's response.


Katika mtazamo huu mfupi wa maoni ya William Boardman, ni wazi kwamba maoni yanayozungumziwa ambayo Ellen White aliagizwa na Mungu kuyataja, yalikuwa maoni ya Mungu wa Utatu, au \textit{nafsi tatu zilizo hai katika Utatu}. Kwa data hiyo akilini, hebu tuchunguze majibu ya Ellen White.


\section*{Ellen White on William Boardman’s sentiment}


\section*{Ellen White juu ya maoni ya William Boardman}


With the Heavenly Trio quotation, it has been asserted that Ellen White was trinitarian. This is done by ignorantly or sometimes purposely ignoring the context of this valuable quotation. When reading Ellen White’s response, in which she defends our perceptions of God, try to recognize whom she is addressing when she speaks of God. Was the God she defended the Trinity or the Father? Referencing William Boardmans illustrations she said:


Kwa nukuu ya Watatu wa Mbinguni, imesemekana kwamba Ellen White alikuwa mwamini wa utatu. Hii inafanywa kwa ujinga au wakati mwingine kwa makusudi kupuuza muktadha wa nukuu hii ya thamani. Wakati wa kusoma majibu ya Ellen White, ambayo anatetea mitazamo yetu ya Mungu, jaribu kutambua ni nani anayemrejelea anaposema juu ya Mungu. Je, Mungu aliyemtetea alikuwa Utatu au Baba? Akirejelea vielelezo vya William Boardman alisema:


\egw{\textbf{All these \underline{spiritualistic} representations are simply nothingness}. They are imperfect, untrue. They weaken and diminish the Majesty which no earthly likeness can be compared to. \textbf{God cannot be compared with the things His hands have made}. These are mere earthly things, suffering under the curse of God because of the sins of man. \textbf{The Father cannot be described by the things of earth}. \textbf{The Father is all the fulness of the Godhead \underline{bodily} and is \underline{invisible to mortal sight}}.}[Ms21-1906.9; 1906][https://egwwritings.org/read?panels=p9754.15]


\egw{\textbf{Uwakilishi huu wote wa \underline{kimizimu} ni bure bilashi}. Wakilishi hizi si kamilifu, zisizo za kweli. Zinadhoofisha na kupunguza Ukuu ambao hakuna mfano wa dunia unaoweza kulinganishwa. \textbf{Mungu hawezi kulinganishwa na vitu ambavyo mikono yake imetengeneza}. Haya tu ni vitu vya duniani, huteseka chini ya laana ya Mungu kwa sababu ya dhambi za mwanadamu. \textbf{Baba hawezi kuelezewa na vitu vya duniani}. \textbf{Baba ndiye utimilifu wote wa Uungu \underline{kwa jinsi ya kimwili} na \underline{haonekani kwa macho ya kibinadamu}}.}[Ms21-1906.9; 1906][https://egwwritings.org/read?panels=p9754.15]


By observing the context, it is obvious that Sister White follows Boardman’s line of reasoning and corrects the mistakes. For better comparison, let us look at their writings side by side:


Kwa kutazama muktadha, ni dhahiri kwamba Dada White anafuata njia ya hoja ya Boardman na kurekebisha makosa. Kwa ulinganisho bora zaidi, hebu tuangalie maandishi yao bega kwa bega:


\begin{table}[H]
\centering
\renewcommand{\arraystretch}{1.5}
\setlength{\tabcolsep}{15pt}
\resizebox{\textwidth}{!}{
\begin{tabular}{|p{0.4\textwidth}|p{0.4\textwidth}|}
\hline
\multicolumn{1}{|c|}{\textbf{William Boardman}} & \multicolumn{1}{c|}{\textbf{Ellen G. White}} \\ \hline
\othersQuote{These likenings are all imperfect. They rather hide than \textbf{illustrate the tri-personality of the \underline{one God}}, for they are not persons but things, poor and earthly at best, to represent \textbf{the living personalities of the living God}. \textbf{So much they may do, however, as to illustrate the official relations of each to the other and of each and all to us. And more. They may also illustrate the truth that all the fulness of Him who filleth all in all, dwells in \underline{each person of Triune God}}.}[p. 104,105][https://archive.org/details/higherchristian02boargoog/page/n112] & 
\egw{\textbf{All these \underline{spiritualistic} representations are simply nothingness}. They are imperfect, untrue. They weaken and diminish the Majesty which no earthly likeness can be compared to. \textbf{God cannot be compared with the things His hands have made}. These are mere earthly things, suffering under the curse of God because of the sins of man. \textbf{The Father cannot be described by the things of earth}.}[Ms21-1906.9; 1906][https://egwwritings.org/read?panels=p9754.15] \\ \hline
\end{tabular}
}
\end{table}


\begin{table}[H]
\centering
\renewcommand{\arraystretch}{1.5}
\setlength{\tabcolsep}{15pt}
\resizebox{\textwidth}{!}{
\begin{tabular}{|p{0.4\textwidth}|p{0.4\textwidth}|}
\hline
\multicolumn{1}{|c|}{\textbf{William Boardman}} & \multicolumn{1}{c|}{\textbf{Ellen G. White}} \\ \hline
\othersQuote{Ulinganisho huu wote si kamilifu. Afadhali wao kujificha kuliko \textbf{kuonyesha ubinafsi tatu wa \underline{Mungu mmoja}}, kwa maana wao si watu bali ni vitu, maskini na wa kidunia kwa ubora zaidi, kuwakilisha \textbf{nafsi hai za Mungu aliye hai}. \textbf{Sana wanaweza kufanya, hata hivyo, kama ili kuonyesha uhusiano rasmi wa kila mmoja kwa mwingine na wa kila mmoja na wote kwetu. Na zaidi. Wanaweza pia kuonyesha ukweli kwamba utimilifu wote wa Yeye anayejaza yote katika yote, hukaa ndani ya \underline{kila nafsi ya Mungu wa Utatu}}.}[p. 104,105][https://archive.org/details/higherchristian02boargoog/page/n112] & 
\egw{\textbf{Uwakilishi huu wote wa \underline{kimizimu} ni bure bilashi}. Wakilishi hizi si kamilifu, zisizo za kweli. Zinadhoofisha na kupunguza Ukuu ambao hakuna mfano wa dunia unaoweza kulinganishwa. \textbf{Mungu hawezi kulinganishwa na vitu ambavyo mikono yake imetengeneza}. Haya tu ni vitu vya duniani, huteseka chini ya laana ya Mungu kwa sababu ya dhambi za mwanadamu. \textbf{Baba hawezi kuelezewa na vitu vya duniani}.}[Ms21-1906.9; 1906][https://egwwritings.org/read?panels=p9754.15] \\ \hline
\end{tabular}
}
\end{table}


In this comparison, it is clear who God is for William Boardman, and who He is for Sister White. For Boardman, God is the Triune God, a tri-personality of the one God. For Sister White, God is the Father. For Boardman, these representations are imperfect because they \others{rather hide than illustrate the tri-personality of the one God}, and for Sister White these representations are imperfect because \egw{The Father cannot be described by the things of earth}. For Boardman, God is the \textit{Triune God}; for Sister White, God is \textit{the Father}.


Katika ulinganisho huu, ni wazi Mungu ni nani kwa William Boardman, na Yeye ni nani kwa Dada White. Kwa Boardman, Mungu ni Mungu wa Utatu, ubinafsi tatu wa Mungu mmoja. Kwa Dada White, Mungu ni Baba. Kwa Boardman, uwakilishi huu si kamilifu kwa sababu \others{afadhali wao kujificha kuliko kuonyesha ubinafsi tatu wa Mungu mmoja}, na kwa Dada White uwakilishi huo si kamilifu kwa sababu \egw{Baba hawezi kuelezewa na vitu vya duniani}. Kwa Boardman, Mungu ni \textit{Mungu wa Utatu}; kwa Dada White, Mungu ni \textit{Baba}.


Boardman’s only point that Ellen White affirms is that these representations are imperfect. Surely, William Boardman would not agree with Ellen White that these representations are \textit{spiritualistic} and \textit{untrue}. On the contrary, he believes that these illustrations \others{illustrate the truth that all the fulness of Him who filleth all in all, dwells in each person of Triune God}. To say that Ellen White agreed with such sentiment is gross misrepresentation.


Hoja pekee ya Boardman ambayo Ellen White anathibitisha ni kwamba uwakilishi huu si kamilifu. Hakika, William Boardman hangekubaliana na Ellen White kwamba uwakilishi huu ni wa kimizimu na usio wa kweli. Badala yake, anaamini kwamba vielelezo hivi \others{vinaonyesha ukweli kwamba utimilifu wote wa Yeye anayejaza yote katika yote, hukaa ndani ya kila nafsi ya Mungu wa Utatu}. Kusema kwamba Ellen White alikubaliana na maoni kama hayo ni upotoshaji mkubwa.


The context of this important quotation prompts important questions. Why does the prophet of God refer to the representations that illustrate the \others{tri-personality of the one God} as \egwinline{spiritualistic representations}, which illustrate the sentiment that \egwinline{is not to be trusted}? Or why does the prophet of God refer to the representations that \others{represent the living personalities of the living God} as \egwinline{spiritualistic representations}? Or why does the prophet of God, when referring to the representations that \others{illustrate the truth that all the fullness of Him who filleth all in all, dwells in each person of Triune God}, refer to them as \egwinline{spiritualistic representations}? All of these spiritualistic representations illustrate the sentiment that \egwinline{is not to be trusted}. This sentiment is clearly the trinitarian sentiment.


Muktadha wa nukuu hii muhimu hutokeza maswali muhimu. Kwa nini nabii wa Mungu anarejelea viwakilishi vinavyoonyesha \others{tri-personality of the one God} kama \egwinline{spiritualistic representations}, ambavyo vinaonyesha hisia kwamba \egwinline{is not to be trusted}? Au kwa nini nabii wa Mungu anarejelea viwakilishi ambavyo \others{represent the living personalities of the living God} kama \egwinline{spiritualistic representations}? Au kwa nini nabii wa Mungu, anaporejelea viwakilishi ambavyo \others{illustrate the truth that all the fullness of Him who filleth all in all, dwells in each person of Triune God}, anavirejelea kama \egwinline{spiritualistic representations}? Viwakilishi hivi vyote vya kimizimu vinaonyesha hisia kwamba \egwinline{is not to be trusted}. Hisia hii dhahiri ni ya utatu.


Sister White continues to follow Boardman’s line of reasoning and corrects the error.


Dada White anaendelea kufuata hoja za Boardman na kurekebisha makosa.


\begin{table}[H]
\centering
\renewcommand{\arraystretch}{1.5}
\setlength{\tabcolsep}{15pt}
\resizebox{\textwidth}{!}{
\begin{tabular}{|p{0.4\textwidth}|p{0.4\textwidth}|}
\hline
\multicolumn{1}{|c|}{\textbf{William Boardman}} & \multicolumn{1}{c|}{\textbf{Ellen G. White}} \\ \hline
\othersQuote{The Father is fullness of the Godhead \textbf{invisibly}, \textbf{\underline{without form}}, whom \textbf{no creature hath seen \underline{or can see}}.}[p.100][https://archive.org/details/higherchristian02boargoog/page/n108/]

\othersQuote{The Father is all the fullness of the Godhead \textbf{INVISIBLE}.}[p.105][https://archive.org/details/higherchristian02boargoog/page/n112/] & 
\egw{The Father is all the fulness of the Godhead \textbf{\underline{bodily}}, and is \textbf{invisible to mortal sight}.}[Ms21-1906.9; 1906][https://egwwritings.org/read?panels=p9754.15] \\ \hline
\end{tabular}
}
\end{table}


\begin{table}[H]
\centering
\renewcommand{\arraystretch}{1.5}
\setlength{\tabcolsep}{15pt}
\resizebox{\textwidth}{!}{
\begin{tabular}{|p{0.4\textwidth}|p{0.4\textwidth}|}
\hline
\multicolumn{1}{|c|}{\textbf{William Boardman}} & \multicolumn{1}{c|}{\textbf{Ellen G. White}} \\ \hline
\othersQuote{The Father is fullness of the Godhead \textbf{invisibly}, \textbf{\underline{without form}}, whom \textbf{no creature hath seen \underline{or can see}}.}[p.100][https://archive.org/details/higherchristian02boargoog/page/n108/]

\othersQuote{The Father is all the fullness of the Godhead \textbf{INVISIBLE}.}[p.105][https://archive.org/details/higherchristian02boargoog/page/n112/] & 
\egw{The Father is all the fulness of the Godhead \textbf{\underline{bodily}}, and is \textbf{invisible to mortal sight}.}[Ms21-1906.9; 1906][https://egwwritings.org/read?panels=p9754.15] \\ \hline
\end{tabular}
}
\end{table}


For Boardman, the Father does not have a form nor body and is invisible to all creatures. For Sister White, the Father has a form and body and is invisible only to mortal human beings.\footnote{When Sister White talks about mortals, she talks about sin polluted humanity. After the restoration of humanity, at the resurrection, Christ will give His immortal life to His children. For more information read \href{https://egwwritings.org/?ref=en_RH.July.5.1887.par.5}{EGW, RH July 5, 1887, par. 5; 1887}.}


Kwa Boardman, Baba hana umbo wala mwili na haonekani na viumbe vyote. Kwa Dada White, Baba ana umbo na mwili na haonekani tu na wanadamu wanaokufa.\footnote{Wakati Dada White anaposema kuhusu wanadamu wanaokufa, anazungumzia wanadamu waliotia doa na dhambi. Baada ya kurejeshwa kwa ubinadamu, wakati wa ufufuo, Kristo atawapa uzima wake usio na kifo kwa watoto wake. Kwa maelezo zaidi soma \href{https://egwwritings.org/?ref=en_RH.July.5.1887.par.5}{EGW, RH July 5, 1887, par. 5; 1887}.}


This quotation is one of the most direct quotations regarding the \emcap{personality of God}. \egwinline{The Father is all the fullness of the Godhead \textbf{bodily}}[Ms21-1906.9; 1906][https://egwwritings.org/read?panels=p9754.16].


Nukuu hii ni mojawapo ya manukuu ya moja kwa moja kuhusu \emcap{personality of God}. \egwinline{The Father is all the fullness of the Godhead \textbf{bodily}}[Ms21-1906.9; 1906][https://egwwritings.org/read?panels=p9754.16].


It might be confusing to someone that the Father is all the fullness of the Godhead bodily because in \textit{Colossians 2:9}, when referring to Jesus, it is written that \bible{in him dwelleth all the fulness of the Godhead bodily.} Scripture does not contradict itself. \textit{Colossians 2:9} does not exclude the Father to be all the fulness of the Godhead bodily. Various places in the Bible describe the Father having a body (\textit{a form: Daniel 7:9,10; Revelation 4:2,3; 1 Kings 22:19-22; a shape: John 5:37}). He has the appearance of a man (\textit{Ezekiel 1:26-28}). He has a face (\textit{Exodus 33:20; Matthew 18:10; Revelation 22:3, 4}). However, the Bible is completely silent about the nature of its substance. The Bible teaches us that \bible{\textbf{The secret things belong unto the LORD our God}: \textbf{but those things which \underline{are revealed} belong unto us and to our children for ever}, that we may do all the words of this law}[Deuteronomy 29:29]. It is revealed to us that the Father has body, He is all the fulness of the Godhead bodily. Also, it is revealed that in Jesus also dwells all the fulness of the Godhead bodily, because \bible{it pleased the Father that in him should all fulness dwell}[Colossians 1:19]. This is not a contradiction whatsoever because the Son is \bible{the \textbf{express image of \underline{His person}}}[Hebrews 1:3].


Inaweza kuwa inachanganya kwa mtu kwamba Baba ni utimilifu wote wa Uungu kimwili kwa sababu katika \textit{Wakolosai 2:9}, inapomtaja Yesu, imeandikwa kwamba \bible{in him dwelleth all the fulness of the Godhead bodily.} Maandiko hayajipingani. \textit{Wakolosai 2:9} haifanyi hivyo kumtenga Baba kuwa utimilifu wote wa Uungu kimwili. Maeneo mbalimbali katika Biblia hueleza Baba kuwa na mwili (\textit{umbo: Danieli 7:9,10; Ufunuo 4:2,3; 1 Wafalme 22:19-22; umbo: Yohana 5:37}). Ana sura ya mwanadamu (\textit{Ezekieli 1:26-28}). Ana uso (\textit{Kutoka 33:20; Mathayo 18:10; Ufunuo 22:3, 4}). Hata hivyo, Biblia ni kimya kote juu ya asili ya dutu yake. Biblia inatufundisha kwamba \bible{\textbf{The secret things belong unto the LORD our God}: \textbf{but those things which \underline{are revealed} belong unto us and to our children for ever}, that we may do all the words of this law}[Kumbukumbu la Torati 29:29]. Inafunuliwa kwetu kwamba Baba ana mwili, Yeye ni utimilifu wote wa Uungu kimwili. Pia, inafunuliwa kwamba ndani ya Yesu pia unakaa utimilifu wote wa Uungu kwa jinsi ya kimwili, kwa sababu \bible{it pleased the Father that in him should all fulness dwell}[Wakolosai 1:19]. Hii sio utata kwa vyovyote vile kwa sababu Mwana ni \bible{the \textbf{express image of \underline{His person}}}[Waebrania 1:3].


\begin{table}[H]
\centering
\renewcommand{\arraystretch}{1.5}
\setlength{\tabcolsep}{15pt}
\resizebox{\textwidth}{!}{
\begin{tabular}{|p{0.4\textwidth}|p{0.4\textwidth}|}
\hline
\multicolumn{1}{|c|}{\textbf{William Boardman}} & \multicolumn{1}{c|}{\textbf{Ellen G. White}} \\ \hline
\othersQuote{The Son is the fullness of the Godhead \textbf{embodied, that his creatures may see him, and know him, and trust him}.}[p.100][https://archive.org/details/higherchristian02boargoog/page/n108/]

\othersQuote{The Son is all the fulness of the Godhead \textbf{MANIFESTED}.}[p.105][https://archive.org/details/higherchristian02boargoog/page/n112/] & 
\egw{The Son is all the fulness of the Godhead \textbf{manifested}. The Word of God declares Him to be ‘\textbf{the express image of His person}’. ‘God so loved the world that He gave \textbf{His only begotten Son}, that whosoever believeth in Him should not perish, but have everlasting life’. \textbf{Here is shown \underline{the personality of the Father}}.}[Ms21-1906.10; 1906][https://egwwritings.org/read?panels=p9754.17] \\ \hline
\end{tabular}
}
\end{table}


\begin{table}[H]
\centering
\renewcommand{\arraystretch}{1.5}
\setlength{\tabcolsep}{15pt}
\resizebox{\textwidth}{!}{
\begin{tabular}{|p{0.4\textwidth}|p{0.4\textwidth}|}
\hline
\multicolumn{1}{|c|}{\textbf{William Boardman}} & \multicolumn{1}{c|}{\textbf{Ellen G. White}} \\ \hline
\othersQuote{The Son is the fullness of the Godhead \textbf{embodied, that his creatures may see him, and know him, and trust him}.}[p.100][https://archive.org/details/higherchristian02boargoog/page/n108/]

\othersQuote{The Son is all the fulness of the Godhead \textbf{MANIFESTED}.}[p.105][https://archive.org/details/higherchristian02boargoog/page/n112/] & 
\egw{The Son is all the fulness of the Godhead \textbf{manifested}. The Word of God declares Him to be ‘\textbf{the express image of His person}’. ‘God so loved the world that He gave \textbf{His only begotten Son}, that whosoever believeth in Him should not perish, but have everlasting life’. \textbf{Here is shown \underline{the personality of the Father}}.}[Ms21-1906.10; 1906][https://egwwritings.org/read?panels=p9754.17] \\ \hline
\end{tabular}
}
\end{table}


Sister White focused on the \emcap{personality of God}, which is the personality of the Father. In Christ, who is \egwinline{begotten in the express image of the Father’s person}[ST May 30, 1895, par. 3; 1895][https://egwwritings.org/read?panels=p820.12891], is shown the personality of the Father. In the same way that Jesus is a person, so is the Father. The quality or state of Christ being a person is the same quality or state of the Father being a person. As Christ is a personal being, so is the Father. Just as all the fullness of the Godhead bodily dwells in Christ, so it does in the Father, because Christ is begotten in the express image of the Father’s person. In Him is shown the personality of the Father. These simple conclusions have been asserted by Scripture in John 3:16 and Hebrews 1:3.


Dada White alizingatia \emcap{personality of God}, ambao ni ubinafsi wa Baba. Katika Kristo, ambaye \egwinline{begotten in the express image of the Father's person}[ST May 30, 1895, par. 3; 1895][https://egwwritings.org/read?panels=p820.12891], anaonyeshwa ubinafsi wa Baba. Kwa njia sawa na kwamba Yesu ni Nafsi, hivyo ni Baba. Ubora au hali ya Kristo kuwa mtu ni sifa sawa au hali ya Baba kuwa Nafsi. Kama vile Kristo ni kiumbe binafsi, ndivyo Baba alivyo. Kama vile utimilifu wote wa Uungu unakaa kimwili ndani ya Kristo, vivyo hivyo ndani ya Baba, kwa sababu Kristo amezaliwa ndani ya taswira ya ubinafsi wa Baba. Ndani yake unaonyeshwa ubinafsi wa Baba. Hitimisho hili rahisi limethibitishwa na Maandiko katika Yohana 3:16 na Waebrania 1:3.


Does the same reasoning, of the personality of the Father and Son, apply to the Holy Spirit? Speaking of the Holy Spirit, Sister White continues:


Je, mantiki hiyo hiyo, ya umbile la Baba na Mwana, inatumika kwa Roho Mtakatifu? Akizungumzia Roho Mtakatifu, Dada White anaendelea:


\egw{\textbf{The Comforter that Christ} promised to send after He ascended to heaven, \textbf{is the Spirit \underline{in} all the fulness of the Godhead}, making manifest the power of divine grace to all who receive and believe in Christ as a personal Saviour.}[Ms21-1906.11; 1906][https://egwwritings.org/read?panels=p9754.18]


\egw{\textbf{Msaidizi ambaye Kristo} aliahidi kumtuma baada ya kupaa mbinguni, \textbf{ni Roho \underline{ndani} ya utimilifu wote wa Uungu}, akidhihirisha nguvu ya neema ya kimungu kwa wote wanaompokea na kumwamini Kristo kama Mwokozi binafsi.}[Ms21-1906.11; 1906][https://egwwritings.org/read?panels=p9754.18]


Sister White draws a distinction between Father and Son who \textbf{are}, individually, \textbf{all} the fullness of the Godhead, and the Spirit that is \textbf{in all} the fullness of the Godhead. This is a marked contrast to William Boardman’s reasoning, where all three are the fullness of the Godhead. Sister White does not follow this trinitarian fashion. The explanation is simple in light of the \emcap{personality of God} and of Christ. The Holy Spirit is a spirit, and the spirit dwells \textbf{in} the flesh/body. The Holy Spirit is \textbf{in all} the fullness of the Godhead\footnote{Take a look at the quotation from \href{https://egwwritings.org/?ref=en_Ms128-1897.13&para=5426.19}{{EGW, Ms128-1897.13; 1897}}, where Sister White states that the Father and the Son are the absolute Godhead.}.


Dada White anaonyesha tofauti kati ya Baba na Mwana ambao, kibinafsi, \textbf{ni} utimilifu wote wa Uungu, na Roho aliye \textbf{ndani ya} utimilifu wote wa Uungu. Hii ni tofauti kubwa na mawazo ya William Boardman, ambapo wote watatu ni utimilifu wa Uungu. Dada White hafuati mtindo huu wa utatu. Ufafanuzi ni rahisi katika nuru ya \emcap{umbile la Mungu} na wa Kristo. Roho Mtakatifu ni roho, na roho hukaa \textbf{ndani} ya nyama/mwili. Roho Mtakatifu yuko \textbf{ndani ya} utimilifu wote wa Uungu\footnote{Angalia nukuu kutoka \href{https://egwwritings.org/?ref=en_Ms128-1897.13&para=5426.19}{{EGW, Ms128-1897.13; 1897}}, ambapo Dada White anasema kwamba Baba na Mwana ni Uungu kamili.}.


Finally, the quotation continues to its most renowned part:


Hatimaye, nukuu inaendelea hadi sehemu yake maarufu zaidi:


\begin{table}[H]
    \centering
    \renewcommand{\arraystretch}{1.5}
    \setlength{\tabcolsep}{15pt}
    \resizebox{\textwidth}{!}{
    \begin{tabular}{|p{0.4\textwidth}|p{0.4\textwidth}|}
    \hline
    \multicolumn{1}{|c|}{\textbf{William Boardman}} & \multicolumn{1}{c|}{\textbf{Ellen G. White}} \\ \hline
    \othersQuote{\textbf{The Father} is all the fulness of the Godhead INVISIBLE.}

    \othersQuote{\textbf{The Son} is all the fulness of the Godhead MANIFESTED.}

    \othersQuote{\textbf{The Spirit} is all the fulness of the Godhead MAKING MANIFEST.}

    \othersQuote{\textbf{The persons} are not mere offices, or modes of revelation, \textbf{but living persons of the living God}.}[p.105][https://archive.org/details/higherchristian02boargoog/page/n112/] & 
    \egw{There are \textbf{three living persons of the heavenly trio}; in the name of these three great powers—\textbf{the Father, the Son, and the Holy Spirit}—those who receive Christ by living faith are baptized, and these powers will co-operate with the obedient subjects of heaven in their efforts to live the new life in Christ.}[Ms21-1906.11; 1906][https://egwwritings.org/read?panels=p9754.18] \\ \hline
    \end{tabular}
    }
    \end{table}


\begin{table}[H]
    \centering
    \renewcommand{\arraystretch}{1.5}
    \setlength{\tabcolsep}{15pt}
    \resizebox{\textwidth}{!}{
    \begin{tabular}{|p{0.4\textwidth}|p{0.4\textwidth}|}
    \hline
    \multicolumn{1}{|c|}{\textbf{William Boardman}} & \multicolumn{1}{c|}{\textbf{Ellen G. White}} \\ \hline
    \othersQuote{\textbf{Baba} ndiye utimilifu wote wa Uungu USIOONEKANE.}

\othersQuote{\textbf{Mwana} ndiye utimilifu wote wa Uungu ULIODHIHIRISHWA.}

\othersQuote{\textbf{Roho} ndiye utimilifu wote wa Uungu AKIDHIHIRISHA.}

\othersQuote{\textbf{Nafsi} hao sio ofisi tu, au njia za ufunuo, \textbf{bali Nafsi walio hai wa Mungu aliye hai}.}[p.105][https://archive.org/details/higherchristian02boargoog/page/n112/] & 
    \egw{Kuna \textbf{nafsi tatu hai wa watatu wa mbinguni}; kwa jina la hizi nguvu tatu kuu—\textbf{Baba, Mwana, na Roho Mtakatifu}—wale wanaompokea Kristo kwa imani iliyo hai wanabatizwa, na nguvu hizi zitashirikiana na raia watiifu wa mbinguni katika juhudi zao za kuishi maisha mapya katika Kristo.}[Ms21-1906.11; 1906][https://egwwritings.org/read?panels=p9754.18] \\ \hline
    \end{tabular}
    }
    \end{table}


In light of the context of William Boardman’s book, we see a marked contrast between \others{three living persons of \textbf{one living God}}, which is the trinitarian sentiment, and \egwinline{the three living persons of \textbf{the heavenly trio}}, which is in accordance with the truth on the \emcap{personality of God}.


Kwa kuzingatia muktadha wa kitabu cha William Boardman, tunaona tofauti kubwa kati ya \others{nafsi tatu hai za \textbf{Mungu mmoja aliye hai}}, ambayo ni hisia ya utatu, na \egwinline{nafsi tatu hai wa \textbf{watatu wa mbinguni}}, ambayo ni kwa mujibu wa ukweli juu ya \emcap{umbile la Mungu}.


The word ‘\textit{trio}’ simply indicates the group of three. The \textit{“heavenly trio}” is represented by the Father, the Son, and the Holy Spirit. But, contrary to popular assumption, they do not make one living God. Three-in-one and one-in-three are concepts that do away with the \emcap{personality of God}. This is why Sister White referred to trinitarian sentiments as sentiments that \egwinline{are not to be trusted}[Ms21-1906.8; 1906][https://egwwritings.org/read?panels=p9754.15].


Neno ‘\textit{watatu}’ linaonyesha tu kikundi cha watu watatu. \textit{“Watatu wa mbinguni”} wanawakilishwa na Baba, Mwana, na Roho Mtakatifu. Lakini, kinyume na dhana maarufu, hawafanyi Mungu mmoja aliye hai. “Tatu-kwa-moja” na “moja-katika-tatu” ni dhana ambazo huondoa \emcap{umbile la Mungu}. Hii ndiyo sababu Dada White alitaja hisia za utatu kama hisia kwamba \egwinline{hazifai kuaminiwa}[Ms21-1906.8; 1906][https://egwwritings.org/read?panels=p9754.15].


Sister White never followed any trinitarian fashion—neither in words and expressions, nor in sentiments. There is an almost effortless research endeavor we encourage you to take: in the writings of Ellen White, search for standard trinitarian terms like “\textit{three are one},” “\textit{one are three},” “\textit{one in three},” “\textit{three in one},” or any of the permutations possible. In her impressive oeuvre you will not find a single occurrence of any of these, let alone the word ‘\textit{trinity}’ describing our God\footnote{There is but one occurrence, in the writings of Ellen White, of the word ‘\textit{trinity}’ referring to \egw{the lust of the flesh, the lust of the eyes and the pride of life}[Lt43-1898.25; 1898][https://egwwritings.org/read?panels=p4806.31]}. She never used these phrases that are necessary to explain the trinitarian sentiment. Examining the following quote, we can see why she never said that God is trinity.


Dada White hakuwahi kufuata mtindo wowote wa utatu—si kwa maneno wala usemi, wala kwa hisia za ndani. Kuna juhudi za utafiti takriban zisizo na jitihada tunazopendekeza uyafanye: katika maandishi ya Ellen White, tafuta maneno ya kawaida ya utatu kama vile “\textit{watatu ni mmoja},” “\textit{mmoja ni watatu},” “\textit{mmoja kati ya watatu},” “\textit{tatu katika moja},” au yoyote kati ya mabadiliko yanayowezekana. Katika kazi yake yenye kuvutia hutapata hata tukio moja la yoyote kati ya haya, achilia mbali neno ‘\textit{trinity}’ likielezea Mungu wetu\footnote{Kuna tukio moja tu, katika maandishi ya Ellen White, ya neno ‘\textit{trinity}’ likirejea \egw{tamaa ya mwili, tamaa ya macho na kiburi cha uzima}[Lt43-1898.25; 1898][https://egwwritings.org/read?panels=p4806.31]}. Yeye kamwe hakutumia vishazi hivi ambavyo ni muhimu kuelezea hisia za utatu. Kwa kuchunguza nukuu ifuatayo, tunaweza kuona kwa nini hakusema kwamba Mungu ni utatu.


\egw{The subject of \textbf{\underline{speculation} regarding \underline{God’s personality} \underline{we will not venture} to express}, \textbf{\underline{except in the language of the Word which represents His personality}}. There is to be no discussion over this question \textbf{lest God would give unmistakable revelation of \underline{what He is}} that would extinguish the one who dares venture on the holy ground in \textbf{his speculative theories}, as some ventured to do in opening the ark to see what was in it as its power and how God was manifested. The men were slain for their curiosity science.}[17LtMs, Ms 223, 1902, par. 16][https://egwwritings.org/read?panels=p14067.9124037&index=0]


\egw{Mada ya \textbf{\underline{dhana} kuhusu \underline{Umbile la Mungu} \underline{hatutahatarisha} kueleza}, \textbf{\underline{isipokuwa katika lugha ya Neno ambayo inawakilisha Umbile lake}}. Hakuna mjadala juu ya swali hili \textbf{isije Mungu akatoa ufunuo usiokosewa wa \underline{kile Yeye ni}} ambao utamzima yule anayethubutu kujaribu katika \textbf{nadharia zake za dhana}, kama wengine walivyothubutu kufanya katika kufungua sanduku kuona kile kilichomo ndani yake kama nguvu zake na jinsi Mungu alivyodhihirishwa. Watu hao waliuawa kwa udadisi wao wa kisayansi.}[17LtMs, Ms 223, 1902, par. 16][https://egwwritings.org/read?panels=p14067.9124037&index=0]


Did you catch that? There is to be no discussion over the question of what God is, \egwinline{lest God would give unmistakable revelation} of \egwinline{what He is}. To say “God is \_\_\_\_\_\_\_”, the blank must be filled with \egwinline{the language of the Word which represents His personality.} The Bible clearly teaches that God is a personal, spiritual being—a truth confirmed by Christ Himself in His revelations to Ellen White. This fits within the biblical language that describes God’s personality. However, according to above statement, can we say “\textit{God is trinity}?” No! That is not \egwinline{the language of the Word which represents His personality.} Therefore, within explored context, we can safely conclude that, the Trinitarian view of God is part of \egwinline{speculative theories} of \egwinline{what He is}.


Je, ulifahamu hilo? Hakuna mjadala juu ya swali la kile Mungu ni, \egwinline{isije Mungu akatoa ufunuo usiokosewa} wa \egwinline{kile Yeye ni}. Kusema “Mungu ni \_\_\_\_\_\_\_“, pengo hilo lazima lijazwe na \egwinline{lugha ya Neno ambayo inawakilisha Umbile lake.} Biblia kwa uwazi inafundisha kwamba Mungu ni kiumbe binafsi, cha kiroho—ukweli uliothibitishwa na Kristo mwenyewe katika mafunuo yake kwa Ellen White. Hii inaendana na lugha ya kibiblia inayoelezea Umbile la Mungu. Hata hivyo, kulingana na taarifa hapo juu, tunaweza kusema “\textit{Mungu ni utatu}?” La! Hiyo si \egwinline{lugha ya Neno ambayo inawakilisha Umbile lake.} Kwa hiyo, katika muktadha uliochunguzwa, tunaweza kuhitimisha kwa usalama kwamba, mtazamo wa Utatu kuhusu Mungu ni sehemu ya \egwinline{nadharia za dhana} za \egwinline{kile Yeye ni}.


This being said, the phrase \egwinline{Heavenly Trio} is not a definition of what God is. Our God is the Father—not \egwinline{the Heavenly Trio.} The term Heavenly Trio does not serve as a replacement for the Trinitarian idea of \textit{three living persons of one God}. This becomes obvious, when we examine the context. Ellen White was instructed to warn us against Trinitarian sentiments, not to trust them. She was not endorsing them.


Hii ikiwa imesemwa, neno \egwinline{Watatu wa Mbinguni} si ufafanuzi wa kile Mungu ni. Mungu wetu ni Baba—si \egwinline{Watatu wa Mbinguni.} Neno Watatu wa Mbinguni halitumiki kama mbadala wa wazo la Utatu la \textit{nafsi tatu hai za Mungu mmoja}. Hii inakuwa wazi, tunapoangalia muktadha. Ellen White aliagizwa kutuonya dhidi ya hisia za Utatu, si kuzitumaini. Hakuwa akiziunga mkono.


Although the illustrations Ellen White quoted were not from Dr. Kellogg, it seems that Kellogg's proponents, if not Kellogg himself, were defending him with William Boardman's sentiments. We do not have direct data to confirm this, but we do know that Dr. Kellogg raised \others{the theological side of questions of \textbf{the trinity and all that sort of things}.}[Interview, J. H. Kellogg, G. W. Amadon and A. C. Bourdeau, October 7th 1907 held at Kellogg’s residence][https://archive.org/details/KelloggVs.TheBrethrenHisLastInterviewAsAnAdventistoct71907/page/n37] The last three paragraphs in the heavenly trio manuscript \href{https://egwwritings.org/?ref=en_Ms21-1906&para=9754.1}{(Ms21-1906; 1906)} reveal the connection with Dr. Kellogg, which is another “smoking gun” of Dr. Kellogg's trinitarian stance.


Ingawa mifano Ellen White aliyonukuu haikuwa kutoka kwa Dk. Kellogg, inaonekana kwamba wafuasi wa Kellogg, kama si Kellogg mwenyewe, walikuwa wakimtetea kwa hisia za William Boardman. Hatuna data ya moja kwa moja kuthibitisha hili, lakini tunajua kwamba Dk. Kellogg aliibua \others{upande wa kitheolojia wa maswali ya \textbf{utatu na mambo yote ya aina hiyo}.}[Interview, J. H. Kellogg, G. W. Amadon and A. C. Bourdeau, October 7th 1907 held at Kellogg's residence][https://archive.org/details/KelloggVs.TheBrethrenHisLastInterviewAsAnAdventistoct71907/page/n37] Aya tatu za mwisho katika hati ya watatu wa mbinguni \href{https://egwwritings.org/?ref=en_Ms21-1906&para=9754.1}{(Ms21-1906; 1906)} zinafunua uhusiano na Dk. Kellogg, ambao ni “ushahidi dhahiri” wa msimamo wa utatu wa Dk. Kellogg.


\egw{I write this because any moment my life may be ended. \textbf{Unless there is a breaking away from the influence that Satan has prepared, and a \underline{reviving of the testimonies that God has given, souls will perish in their delusion}. They will accept fallacy after fallacy and will thus keep up a disunion that will always exist until those who have been deceived take \underline{their stand on the right platform}}. All this higher education that is being planned will be extinguished; for it is spurious. The more simple the education of our workers, the less connection they have with the men whom God is not leading, the more will be accomplished. \textbf{Work will be done in the \underline{simplicity} of true godliness, and the old, old times will be back when, under the Holy Spirit’s guidance, thousands were converted in a day. When the truth in its simplicity is lived in every place, then God will work through His angels as He worked on the day of Pentecost, and hearts will be changed so decidedly that there will be a manifestation of the influence of genuine truth, as is represented in the descent of the Holy Spirit}.}[Ms21-1906.18; 1906][https://egwwritings.org/read?panels=p9754.25]


\egw{Ninaandika haya kwa sababu wakati wowote maisha yangu yanaweza kumalizika. \textbf{Isipokuwa kuna njia ya kujiondoa kutoka kwa ushawishi ambao Shetani ametayarisha, na \underline{kuhuisha shuhuda ambazo Mungu ametoa}, roho zitaangamia katika udanganyifu wao. Watakubali upotofu baada ya upotofu na hivyo itaendeleza mfarakano utakaokuwepo daima hadi wale ambao wamedanganywa kuchukua \underline{msimamo wao kwenye jukwaa sahihi}}. Elimu hii yote ya juu ambayo inapangwa itazimwa; maana ni uwongo. Kadiri elimu ya wafanyikazi wetu inavyokuwa ya \underline{wazi na nyepesi}, ndivyo inavyopungua uhusiano walio nao na watu ambao Mungu hawaongozi, ndivyo zaidi yatakavyotimizwa. Kazi itafanywa katika usahili wa utauwa wa kweli, na nyakati za kale, za kale zitarudi, chini ya uongozi wa Roho Mtakatifu, maelfu ya watu waliongoka kwa siku moja. Wakati ukweli katika usahili wake unapoishi kila mahali, basi Mungu atafanya kazi kupitia malaika wake kama alivyofanya kazi siku ya Pentekoste, na mioyo itabadilishwa hivyo kwa uamuzi kwamba kutakuwa na udhihirisho wa ushawishi wa ukweli wa kweli, kama ulivyo kuwakilishwa katika kushuka kwa Roho Mtakatifu.}[Ms21-1906.18; 1906][https://egwwritings.org/read?panels=p9754.25]


\egwnogap{The Holy Spirit never has and never will in the future divorce the medical missionary work from the gospel ministry. They cannot be divorced. Bound up with Jesus Christ, the ministry of the Word and the healing of the sick are one.}[Ms21-1906.19; 1906][https://egwwritings.org/read?panels=p9754.26]


\egwnogap{Roho Mtakatifu hajawahi na kamwe hataachana na kazi ya umishonari ya matibabu katika siku zijazo kutoka kwa huduma ya injili. Haiwezi kutenganishwa. Imefungwa pamoja na Yesu Kristo, huduma ya Neno na uponyaji wa wagonjwa ni kitu kimoja.}[Ms21-1906.19; 1906][https://egwwritings.org/read?panels=p9754.26]


\egwnogap{The fifty-eighth chapter of Isaiah contains instruction for today. \textbf{‘Cry aloud, spare not, lift up thy voice like a trumpet, and show My people their transgression, and the house of Jacob their sin.’ God does not accept \underline{Dr. Kellogg as His laborer}, unless he will now break with Satan}. The work would not have been hindered, as it has been for the past several years, \textbf{if Dr. Kellogg were a converted man. ‘Come,’ I call, ‘come ye out and be separate from him and his associates whom he has leavened.’ I am now giving the message God has given me, to give to all who claim to believe the truth, \underline{‘Come ye out from among them, and be separate},’ else their sin in justifying wrongs and framing deceits will continue to be the ruin of souls. We cannot afford to be on the wrong side. We cannot afford to cover the truth with scientific problems. We urge that decided changes be made and no more stumbling blocks be placed before the feet of the people of God}. Let every soul put on the gospel shoes. \textbf{Let every soul pray and work, placing their feet upon \underline{the foundation Christ laid} in giving His life for the life of the world}.}[Ms21-1906.20; 1906][https://egwwritings.org/read?panels=p9754.27]


\egwnogap{Sura ya hamsini na nane ya Isaya ina maagizo ya leo. \textbf{‘Lieni kwa sauti kubwa, msiache, Paza sauti yako kama tarumbeta, uwahubiri watu wangu kosa lao, na nyumba ya Yakobo dhambi yao.’ Mungu hamkubali \underline{Dk. Kellogg kama mtenda kazi Wake}, isipokuwa atakubali sasa kutengana na Shetani}. Kazi hiyo isingezuiwa, kama ilivyokuwa zamani miaka kadhaa, \textbf{ikiwa Dk. Kellogg angekuwa mtu aliyeongoka. ‘Njoo,’ naita, ‘toka na ujitenge naye na washirika wake aliowachacha.’ Sasa ninatoa ujumbe ambao Mungu amenipa, kuwapa wote wanaodai kuamini ukweli, \underline{‘jiondoeni miongoni mwao, na kujitenga},’ sivyo dhambi yao katika kuhalalisha makosa na kutunga udanganyifu utaendelea kuwa uharibifu wa roho}. Hatuwezi kumudu kuwa upande mbaya. Hatuwezi kumudu kuficha ukweli kwa matatizo ya kisayansi. Tunaomba hilo liamuliwe mabadiliko yafanywe na hakuna vikwazo tena vinavyowekwa mbele ya miguu ya watu wa Mungu. Hebu kila nafsi ivae viatu vya injili. \textbf{Hebu kila nafsi iombe na kufanya kazi, wakiweka miguu yao juu ya \underline{msingi ambao Kristo aliweka} katika kutoa maisha yake kwa ajili ya uzima wa dunia}.}[Ms21-1906.20; 1906][https://egwwritings.org/read?panels=p9754.27]


The heavenly trio quotation was part of Kellogg's controversy. This is evidence that Kellogg’s controversy included the Trinity doctrine. We are told to break \egwinline{away from the influence of Satan} and to revive the \egw{testimony that God has given} us, or else our souls will perish in delusions. These influences and delusions come from trinitarians such as \textit{William Boardman} and \textit{Dr. John H. Kellogg}. She is pointing us back to place our feet upon the foundation that was built by the Masterworker.\footnote{\href{https://egwwritings.org/?ref=en_SpTB02.54.2&para=417.276}{EGW, SpTB02 54.2; 1904}}


Nukuu ya watatu wa mbinguni ilikuwa sehemu ya utata wa Kellogg. Huu ni ushahidi kwamba utata wa Kellogg ulitia ndani fundisho la Utatu. Tunaambiwa tujitenge \egwinline{na ushawishi wa Shetani} na kuhuisha \egw{ushuhuda ambao Mungu ametupa}, ama sivyo roho zetu zitaangamia katika udanganyifu. Athari hizi na udanganyifu hutoka kwa waamini utatu kama vile \textit{William Boardman} na \textit{Dk. John H. Kellogg}. Anatuelekeza nyuma ili kuweka miguu yetu juu ya msingi huo uliojengwa na Fundi Stadi.\footnote{\href{https://egwwritings.org/?ref=en_SpTB02.54.2&para=417.276}{EGW, SpTB02 54.2; 1904}}


We hope that this context exposes the false narrative of Ellen White's endorsement of the Trinity doctrine, propagated by our Adventist scholars. Dr. Kellogg was in apostasy for stepping off from the foundation of our faith, and the Trinity doctrine was his justification. With such data in mind, one must ask: If the Trinity was true, and Ellen White endorsed it, and this “true” Trinity was mixed with Dr. Kellogg's error, we should expect her to separate the Trinity from error. But this is not what she did. Instead, she consistently pointed us back to the foundation of our faith, where we had a clear teaching on the presence and the \emcap{personality of God}. But for the case of Trinity, she faithfully bore the message from Heaven: “\textit{\textbf{I am instructed to say}, the sentiments of those who are searching for \textbf{trinitarian ideas are not to be trusted}}.”


Tunatumaini kwamba muktadha huu unafichua hadithi ya uongo ya kuunga mkono kwa Ellen White kwa fundisho la Utatu, inayoenezwa na wasomi wetu wa Kiadventista. Dk. Kellogg alikuwa katika ukengeufu kwa kuondoka kwenye msingi wa imani yetu, na fundisho la Utatu lilikuwa haki yake. Kwa data kama hiyo akilini, mtu lazima aulize: Ikiwa Utatu ulikuwa kweli, na Ellen White aliuunga mkono, na Utatu huu “wa kweli” ulichanganywa na makosa ya Dk. Kellogg, tungetarajia kwamba angetenganisha Utatu na makosa. Lakini hiki sio alichofanya. Badala yake, alituonyesha kwa uthabiti kurudi kwenye msingi wa imani yetu, ambapo tulikuwa na mafundisho wazi juu ya uwepo na \emcap{Umbile la Mungu}. Lakini kwa suala la Utatu, alibeba kwa uaminifu ujumbe kutoka Mbinguni: “\textit{\textbf{Nimeagizwa kusema}, hisia za wale wanaotafuta \textbf{mawazo ya utatu hazifai kuaminiwa}}.”


% The Heavenly Trio

\begin{titledpoem}
    
    \stanza{
        In heaven’s realm, where truths unfold, \\
        A message clear, so brave and bold. \\
        God spoke through Ellen, clear and bright, \\
        Revealing depths of heavenly light.
    }

    \stanza{
        Misunderstood by some who read, \\
        Her words of God that all must heed. \\
        Not as triune, but trio three \\
        Distinct as persons, heavenly.
    }

    \stanza{
        The Father, not a formless feel, \\
        Invisible to us, yet real. \\
        He is the fullness, all complete, \\
        The Godhead, bodily, concrete.
    }

    \stanza{
        The Son, God’s fullness, manifest \\
        In Him, divinity does rest. \\
        God’s character, seen in His face, \\
        In Christ, we see His Father’s grace.
    }

    \stanza{
        The Spirit, in all fullness dwells, \\
        A mystery nature, Ellen tells. \\
        With forms, the Father and His Son \\
        With Them, in Spirit, we are one.
    }

    \stanza{
        Distinct and clear, Their roles unfold, \\
        The Father, Son, in form behold. \\
        Yet present everywhere we find, \\
        Their Spirit shows Their heart and mind.
    }

    \stanza{
        God’s message true, from up above. \\
        Reveals to us the Father’s love. \\
        To know this truth about our God— \\
        It lights the path that we must trod.
    }

    \stanza{
        Dear Ellen’s words, in context found, \\
        Reveal a truth that’s so profound \\
        Not trinity did she embrace, \\
        But trio persons in their place.
    }

    \stanza{
        The pillar stands, our platform firm, \\
        God’s personality we learn. \\
        The trio that is heavenly, \\
        Exposes falsehood—trinity.
    }
    
\end{titledpoem}
