
\qrchapter{https://forgottenpillar.com/rsc/en-fp-chapter25}{Kuweka Kanuni za Msingi zisizo sahihi}

Unaweza kujiuliza: itawezekanaje sisi kama kanisa tumepotoka kwa nuru ambayo Mungu alitupa hapo mwanzo? Jibu la swali hili ni jibu sawa kwa swali kwa nini Wayahudi walipotea kutoka kwa nuru ambayo Mungu ali-wapa kuhusu Mwanawe. Tafadhali, angalia nguvu ya kuendesha kanisa katika nyakati za Mitume na wakati wetu.

\egw{‘Malaika wa Bwana nyakati za usiku akaifungua milango ya gereza, akawatoa nje, akasema, Nendeni mkasimame hekaluni mkaseme na watu maneno yote ya uzima huu.’ [Matendo 5:19, 20.] Sisi tunaona hapa kwamba wanaume wenye mamlaka hawatiiwi siku zote, wengawa wanaweza kukiri kuwa walimu wa mafundisho ya Biblia. \textbf{Kuna wengi leo ambao huhisi hasira na huzuni kwamba sauti yoyote inapaswa kupazwa ikiwasilisha mawazo ambayo ni tofauti na yao kuhusiana na pointi za imani ya kidini}. \textbf{Je, si muda mrefu wametetea mawazo yao kama ukweli?} Kwa hivyo makuhani na marabi walijadiliana katika siku za mitume. Wanamaanisha nini hawa watu wasio na elimu, baadhi yao ni wavuvi tu, ambao wanawasilisha mawazo kinyume na mafundisho ambayo makuhani na watawala wataalamu wanafundisha watu? \textbf{Hawana haki ya kuingilia kati kanuni za msingi za imani yetu}.}[Lt38-1896.23; 1896][https://egwwritings.org/read?panels=p5631.29]

\egwnogap{“\textbf{Lakini tunaona kwamba Mungu wa mbinguni wakati mwingine huwaagiza wanadamu \underline{kufundisha jambo ambalo linachukuliwa kuwa kinyume na mafundisho yaliyowekwa}. Kwa sababu wale ambao walikuwa wawekaji wa awali wa ukweli \underline{wakawa si waaminifu kwa amana yao takatifu}, Bwana aliwachagua wengine ambao wangepokea miale angavu ya Jua la Haki, na wangepokea kutetea kweli ambazo hazikuwa zinalingana na mawazo ya viongozi wa kidini. Na kisha viongozi hawa, katika upofu wa akili zao, wanatoa uwezo kamili wa kile ambacho kinapaswa kuwa hasira ya haki dhidi ya wale ambao wameweka kando hekaya zao za kuthaminiwa. Wanafanya kama watu ambao wamepoteza akili zao. Hawazingatii uwezekano kwamba wao wenyewe hawajaelewa Neno sawasawa. Hawatafungua macho yao ili kubaini ukweli kwamba wameitafsiri vibaya na kuitumia vibaya Maandiko, na wamejenga nadharia za uongo, \underline{na kuziita mafundisho ya msingi ya imani}}.”}[Lt38-1896.24; 1896][https://egwwritings.org/read?panels=p5631.30]

\egwnogap{\textbf{Lakini Roho Mtakatifu mara kwa mara atafunua ukweli kupitia mashirika wake wateule; wala hakuna mtu, hata kuhani au mkuu, aliye na haki ya kusema, Usitangaze maoni yako, kwa sababu siyaamini. Hiyo ya ajabu “mimi” naweza jaribu kuweka chini mafundisho ya Roho Mtakatifu. Watu wanaweza, kwa muda, kujaribu kuivunja na kuiua; lakini hilo halitafanya kosa kuwa ukweli au ukweli kuwa kosa. Uvumbuzi wa akili za watu na maoni ya kubahatisha katika mistari mbalimbali, na wakati Roho Mtakatifu huacha nuru iangaze katika akili za wanadamu, haiheshimu kila nuru ya mwanadamu ya matumizi ya neno. Mungu aliwatia ukakamavu watumishi wake waseme kweli bila kujali yale ambayo watu walikuwa wameyachukulia kuwa ni ukweli}.}[Lt38-1896.25; 1896][https://egwwritings.org/read?panels=p5631.31]

\egwnogap{\textbf{\underline{Hata Waadventista Wasabato wako katika hatari ya kuufumbia macho ukweli jinsi ulivyo katika Yesu}, kwa sababu inapingana na kitu ambacho wamekichukulia kuwa ni kweli, lakini ambayo Roho Mtakatifu anafundisha si kweli. Hebu wote wawe na kiasi sana, na watafute zaidi kwa bidii kujiweka nje ya swali, na kumwinua Yesu.} \textbf{Katika mabishano mengi ya dini, shida ya msingi ni kwamba ubinafsi unajitahidi kupata ukuu}. Kuhusu nini? Kuhusu mambo ambayo si mambo muhimu hata kidogo, na ambayo yanachukuliwa kuwa hivyo kwa sababu tu watu wamewapa umuhimu. Tazama Mathayo 12:31-37; Marko 14:56; Luka 5:21; Mathayo 9:3.}[Lt38-1896.26; 1896][https://egwwritings.org/read?panels=p5631.32]

Hali ya kiburi ya moyo inapinga mapenzi ya Mungu na ndiyo nguvu inayoongoza nyuma ya ukengeufu; moyo mnyenyekevu ni mtiifu kwa mapenzi ya Mungu na ni nguvu inayoongoza matengenezo ya ukweli. Manukuu yafuatayo yanaelezea unabii wa siku zijazo, ambapo dhana za makisio kuhusu Mungu yataletwa na \egwinline{mambo mengi yanayofanana na haya yatatokea siku zijazo}[Ms137-1903.10; 1903][https://egwwritings.org/read?panels=p9939.17]. Mawazo haya ni ya namna sawa na mawazo yaliyomo katika Hekalu Hai. Wataondoa \emcap{Umbile la Mungu}. Ellen White atoa onyo baada ya onyo la kuzingatia \emcap{Kanuni za Msingi}, na kufahamu jinsi viongozi watakavyobomoa Kanuni za Msingi za zamani.

\egw{Kwa kuzingatia Maandiko haya, ni nani atakayethubutu kumfasiri Mungu na kuweka katika akili za wengine maoni kuhusu Yeye zilizomo katika Hekalu Hai? \textbf{Nadharia hizi ni nadharia za mdanganyifu mkuu, na katika maisha ya \underline{wale wanaozipokea zitakuwa sura za huzuni}}. \textbf{Hiki ni mbinu ya Shetani \underline{ili kuyumbisha msingi wa imani yetu}, kutikisika imani yetu katika mwongozo wa Bwana na uzoefu ambao ametupa. \underline{Mambo mengi ya tabia kama hiyo yatatokea katika siku zijazo}}. Ninawasihi mamishonari wetu wa matibabu wasiamini dhana za kubuni za binadamu yeyote ambaye anaendeleza mawazo kwamba \textbf{njia ambayo watu wa Mungu wameongozwa kwa ajili yake miaka hamsini iliyopita ni njia mbaya}. \textbf{\underline{Jihadharini na wale ambao}, bila kuwa na uzoefu wa uamuzi wowote katika uongozi wa Roho wa Bwana, \underline{ungedhani kwamba uongozi huu ni udanganyifu}; kwamba hatuna ukweli}; kwamba sisi si watu wa Bwana, tuliokusanywa naye kutoka nchi na mataifa yote. \textbf{\underline{Jihadharini na wale ambao watabomoa msingi, ambayo tumekuwa tukijenga kwa miaka hamsini iliyopita, ili kuanzisha fundisho jipya}}. \textbf{Ninajua kwamba nadharia hizi mpya zinatoka kwa adui}.}[Ms137-1903.10; 1903][https://egwwritings.org/read?panels=p9939.17]

\egwnogap{\textbf{Acheni wale ambao \underline{wangeleta ndani} mawazo ya kidhahania juu ya Mungu waamke na kuhisi hatari yao. Hili ni somo zito sana haliwezi kuchezewa}.}[Ms137-1903.11; 1903][https://egwwritings.org/read?panels=p9939.18]

\egwnogap{Mzizi wa kuabudu sanamu ni moyo mbaya wa kutoamini katika kujitenga na Mungu aliye hai. Ni kwa sababu watu hawana imani katika uwepo na nguvu za Mungu \textbf{ndiposa wamekuwa wakitegemea hekima zao wenyewe}. Wamekuwa wakibuni na kupanga kujiinua na kupata wokovu katika kazi zao wenyewe. \textbf{\underline{Ushawishi wa udanganyifu kutoka kwa mashirika ya kishetani unakuja ndani}, kwa sababu  viongozi ambao Bwana amewaonya na kuwasihi na kuwashauri wamechagua hekima yao wenyewe badala ya hekima ya Mungu}. Kwa watu kama hao onyo laja, ‘Msiseme tena kwa majivuno kupita kiasi; jeuri isitoke kweye mdomo yenu; kwa kuwa Bwana ni Mungu wa maarifa, na matendo hupimwa na yeye.’}[Ms137-1903.12; 1903][https://egwwritings.org/read?panels=p9939.19]

Tofauti kati ya \emcap{Kanuni za Msingi} za zamani na Mafundisho za Kimsingi mpya iko katika \egwinline{mawazo yetu ya Mungu.} Wazo la Utatu kuhusu Mungu halikuwa sehemu ya msingi wa imani yetu, ambayo Dada White alitetea. Mabadiliko haya yalitokea vipi? Yalifanywa kupitia viongozi ambao walichagua \egwinline{hekima yao wenyewe badala ya hekima ya Mungu.} Tunapaswa \egwinline{Jihadhari na wale ambao wangebomoa msingi, ambao tumekuwa tukijenga kwa miaka hamsini iliyopita, ili kuanzisha fundisho jipya.} Katika uchunguzi huu, tunatambua kwamba wazo hili jipya la Utatu kuhusu Mungu lilikuwa \egwinline{ushawishi wa udanganyifu kutoka kwa mashirika ya kishetani} ambao uliingia katika safu zetu.

% Setting up the wrong Fundamental Principles

\begin{titledpoem}
    
    \stanza{
        In the silence of our own church’s walls, \\
        We’ve strayed from the light that first on us falls. \\
        An ancient query resounds with age: \\
        Why from sacred paths do we disengage?
    }

    \stanza{
        Ellen spoke of priests and rabbis of old, \\
        Men of cloth, with hearts yet so cold. \\
        They spurned the truth when it brightly shone, \\
        Proud hearts rejecting what was divinely shown.
    }

    \stanza{
        Behold the fishermen with unlearned tongues, \\
        Who dared to challenge the learned ones. \\
        "The Holy Spirit guides," they boldly claimed, \\
        While leaders scorned and fiercely blamed.
    }

    \stanza{
        Do not the humble hearts perceive \\
        The quiet whispers they should believe? \\
        While proud hearts preach their own command, \\
        True faith slips like fine sand from hand.
    }

    \stanza{
        Even now, echoes of the past, \\
        Warn us of shadows that leaders cast. \\
        God’s own voice, through years does span, \\
        Yet resisted by the inventions of man.
    }

    \stanza{
        To the foundations we must hold fast, \\
        Not swayed by shadows that leaders cast. \\
        For in the Scripture’s unerring light, \\
        Lies the path that is just and right.
    }

    \stanza{
        Beware of new doctrines, thinly veiled, \\
        On old, firm rocks they have not sailed. \\
        Let not man’s wisdom lead astray, \\
        But God’s own Spirit show the way.
    }

    \stanza{
        In faith, let us each day commence, \\
        With Bible as shield, no human pretense. \\
        For truth in Christ alone is found, \\
        And on this rock, our faith is sound.
    }
    
\end{titledpoem}
