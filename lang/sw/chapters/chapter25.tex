\qrchapter{https://forgottenpillar.com/rsc/en-fp-chapter25}{Setting up the wrong Fundamental Principles}


\qrchapter{https://forgottenpillar.com/rsc/en-fp-chapter25}{Kuweka Kanuni za Msingi zisizo sahihi}


You might ask yourself: how could it be possible that we, as a church, have gone astray from the light God gave us in the beginning? The answer to this question is the same answer to the question why the Jews went astray from the light God gave them concerning His Son. Please, take a look at the driving force behind the church in Apostolic times and our time.


Unaweza kujiuliza: itawezekanaje sisi kama kanisa tumepotoka kwa nuru ambayo Mungu alitupa hapo mwanzo? Jibu la swali hili ni jibu sawa kwa swali kwa nini Wayahudi walipotea kutoka kwa nuru ambayo Mungu ali-wapa kuhusu Mwanawe. Tafadhali, angalia nguvu ya kuendesha kanisa katika nyakati za Mitume na wakati wetu.


\egw{‘The angel of the Lord by night opened the prison doors, and brought them forth, and said, Go, stand and speak in the temple to the people all the words of this life.’ [Acts 5:19, 20.] We see here that the men in authority are not always obeyed, even though they may profess to be teachers of Bible doctrines. \textbf{There are many today who feel indignant and aggrieved that any voice should be raised presenting ideas that differ from their own in regard to points of religious belief}. \textbf{Have they not long advocated their ideas as truth?} So the priests and rabbis reasoned in apostolic days. What mean these men who are unlearned, some of them mere fishermen, who are presenting ideas contrary to the doctrines which the learned priests and rulers are teaching the people? \textbf{They have no right to meddle with the fundamental principles of our faith}.}[Lt38-1896.23; 1896][https://egwwritings.org/read?panels=p5631.29]


\egw{‘Malaika wa Bwana nyakati za usiku akaifungua milango ya gereza, akawatoa nje, akasema, Nendeni mkasimame hekaluni mkaseme na watu maneno yote ya uzima huu.’ [Matendo 5:19, 20.] Sisi tunaona hapa kwamba wanaume wenye mamlaka hawatiiwi siku zote, wengawa wanaweza kukiri kuwa walimu wa mafundisho ya Biblia. \textbf{Kuna wengi leo ambao huhisi hasira na huzuni kwamba sauti yoyote inapaswa kupazwa ikiwasilisha mawazo ambayo ni tofauti na yao kuhusiana na pointi za imani ya kidini}. \textbf{Je, si muda mrefu wametetea mawazo yao kama ukweli?} Kwa hivyo makuhani na marabi walijadiliana katika siku za mitume. Wanamaanisha nini hawa watu wasio na elimu, baadhi yao ni wavuvi tu, ambao wanawasilisha mawazo kinyume na mafundisho ambayo makuhani na watawala wataalamu wanafundisha watu? \textbf{Hawana haki ya kuingilia kati kanuni za msingi za imani yetu}.}[Lt38-1896.23; 1896][https://egwwritings.org/read?panels=p5631.29]


\egwnogap{“\textbf{But we see that the God of heaven sometimes commissions men to \underline{teach that which is regarded as contrary to the established doctrines}. Because those who were once the depositaries of truth \underline{became unfaithful to their sacred trust}, the Lord chose others who would receive the bright beams of the Sun of Righteousness, and would advocate truths that were not in accordance with the ideas of the religious leaders. And then these leaders, in the blindness of their minds, give full sway to what is supposed to be righteous indignation against the ones who have set aside cherished fables. They act like men that have lost their reason. They do not consider the possibility that they themselves have not rightly understood the Word. They will not open their eyes to discern the fact that they have misinterpreted and misapplied the Scriptures, and have built up false theories, \underline{calling them fundamental doctrines of the faith}}.“}[Lt38-1896.24; 1896][https://egwwritings.org/read?panels=p5631.30]


\egwnogap{“\textbf{Lakini tunaona kwamba Mungu wa mbinguni wakati mwingine huwaagiza wanadamu \underline{kufundisha jambo ambalo linachukuliwa kuwa kinyume na mafundisho yaliyowekwa}. Kwa sababu wale ambao walikuwa wawekaji wa awali wa ukweli \underline{wakawa si waaminifu kwa amana yao takatifu}, Bwana aliwachagua wengine ambao wangepokea miale angavu ya Jua la Haki, na wangepokea kutetea kweli ambazo hazikuwa zinalingana na mawazo ya viongozi wa kidini. Na kisha viongozi hawa, katika upofu wa akili zao, wanatoa uwezo kamili wa kile ambacho kinapaswa kuwa hasira ya haki dhidi ya wale ambao wameweka kando hekaya zao za kuthaminiwa. Wanafanya kama watu ambao wamepoteza akili zao. Hawazingatii uwezekano kwamba wao wenyewe hawajaelewa Neno sawasawa. Hawatafungua macho yao ili kubaini ukweli kwamba wameitafsiri vibaya na kuitumia vibaya Maandiko, na wamejenga nadharia za uongo, \underline{na kuziita mafundisho ya msingi ya imani}}.”}[Lt38-1896.24; 1896][https://egwwritings.org/read?panels=p5631.30]


\egwnogap{\textbf{But the Holy Spirit will from time to time reveal the truth through its own chosen agencies; and no man, not even a priest or ruler, has a right to say, You shall not give publicity to your opinions, because I do not believe them. That wonderful ‘I’ may attempt to put down the Holy Spirit’s teaching. Men may, for a time, attempt to smother it and kill it; but that will not make error truth or truth error. The inventive minds of men have advanced speculative opinions in various lines, and when the Holy Spirit lets light shine into human minds, it does not respect every point of man’s application of the word. God impressed his servants to speak the truth irrespective of what men had taken for granted as truth}.}[Lt38-1896.25; 1896][https://egwwritings.org/read?panels=p5631.31]


\egwnogap{\textbf{Lakini Roho Mtakatifu mara kwa mara atafunua ukweli kupitia mashirika wake wateule; wala hakuna mtu, hata kuhani au mkuu, aliye na haki ya kusema, Usitangaze maoni yako, kwa sababu siyaamini. Hiyo ya ajabu “mimi” naweza jaribu kuweka chini mafundisho ya Roho Mtakatifu. Watu wanaweza, kwa muda, kujaribu kuivunja na kuiua; lakini hilo halitafanya kosa kuwa ukweli au ukweli kuwa kosa. Uvumbuzi wa akili za watu na maoni ya kubahatisha katika mistari mbalimbali, na wakati Roho Mtakatifu huacha nuru iangaze katika akili za wanadamu, haiheshimu kila nuru ya mwanadamu ya matumizi ya neno. Mungu aliwatia ukakamavu watumishi wake waseme kweli bila kujali yale ambayo watu walikuwa wameyachukulia kuwa ni ukweli}.}[Lt38-1896.25; 1896][https://egwwritings.org/read?panels=p5631.31]


\egwnogap{\textbf{\underline{Even Seventh-day Adventists are in danger of closing their eyes to truth as it is in Jesus}, because it contradicts something which they have taken for granted as truth, but which the Holy Spirit teaches is not truth. Let all be very modest, and seek most earnestly to put self out of the question, and to exalt Jesus.} \textbf{In most of the religious controversies, the foundation of the trouble is that self is striving for the supremacy}. About what? About matters which are not vital points at all, and which are regarded as such only because men have given importance to them. See Matthew 12:31-37; Mark 14:56; Luke 5:21; Matthew 9:3.}[Lt38-1896.26; 1896][https://egwwritings.org/read?panels=p5631.32]


\egwnogap{\textbf{\underline{Hata Waadventista Wasabato wako katika hatari ya kuufumbia macho ukweli jinsi ulivyo katika Yesu}, kwa sababu inapingana na kitu ambacho wamekichukulia kuwa ni kweli, lakini ambayo Roho Mtakatifu anafundisha si kweli. Hebu wote wawe na kiasi sana, na watafute zaidi kwa bidii kujiweka nje ya swali, na kumwinua Yesu.} \textbf{Katika mabishano mengi ya dini, shida ya msingi ni kwamba ubinafsi unajitahidi kupata ukuu}. Kuhusu nini? Kuhusu mambo ambayo si mambo muhimu hata kidogo, na ambayo yanachukuliwa kuwa hivyo kwa sababu tu watu wamewapa umuhimu. Tazama Mathayo 12:31-37; Marko 14:56; Luka 5:21; Mathayo 9:3.}[Lt38-1896.26; 1896][https://egwwritings.org/read?panels=p5631.32]


The proud state of the heart resists the will of God and is the driving force behind apostasy; the humble heart is obedient to the will of God and is the driving force behind true reformation. The following quotations express future, concrete prophecies where the fanciful ideas of God will be brought in and \egwinline{many things of like character will in the future arise}[Ms137-1903.10; 1903][https://egwwritings.org/read?panels=p9939.17]. These ideas are of like character to the ideas contained in the Living Temple. They will do away with the \emcap{personality of God}. Ellen White gives warning after warning to adhere to the \emcap{Fundamental Principles}, and to be aware of the leaders who will tear down the old foundation.


Hali ya kiburi ya moyo inapinga mapenzi ya Mungu na ndiyo nguvu inayoongoza nyuma ya ukengeufu; moyo mnyenyekevu ni mtiifu kwa mapenzi ya Mungu na ni nguvu inayoongoza matengenezo ya ukweli. Manukuu yafuatayo yanaelezea unabii wa siku zijazo, ambapo dhana za makisio kuhusu Mungu yataletwa na \egwinline{mambo mengi yanayofanana na haya yatatokea siku zijazo}[Ms137-1903.10; 1903][https://egwwritings.org/read?panels=p9939.17]. Mawazo haya ni ya tabia sawa na mawazo yaliyomo katika Hekalu Hai. Wataondoa \emcap{Umbile la Mungu}. Ellen White atoa onyo baada ya onyo la kuzingatia \emcap{Kanuni za Msingi}, na kufahamu jinsi viongozi watakavyobomoa Kanuni za Msingi za zamani.


\egw{In view of these Scriptures, who will dare to interpret God and place in the minds of others the sentiments regarding Him that are contained in Living Temple? \textbf{These theories are the theories of the great deceiver, and in the lives of \underline{those who receive them there will be sad chapters}}. \textbf{This is Satan’s device \underline{to unsettle the foundation of our faith}, to shake our confidence in the Lord’s guidance and in the experience that He has given us. \underline{Many things of like character will in the future arise}}. I entreat our medical missionary workers to be afraid to trust the suppositions and devising of any human being who entertains the thought that \textbf{the path over which the people of God have been led for the last fifty years is a wrong path}. \textbf{\underline{Beware of those who}, not having had any decided experience in the leading of the Lord’s Spirit, \underline{would suppose that this leading is all a fallacy}; that we have not the truth}; that we are not the people of the Lord, gathered by Him from all countries and nations. \textbf{\underline{Beware of those who would tear down the foundation, upon which we have been building for the last fifty years, to establish a new doctrine}}. \textbf{I know that these new theories are from the enemy}.}[Ms137-1903.10; 1903][https://egwwritings.org/read?panels=p9939.17]


\egw{Kwa kuzingatia Maandiko haya, ni nani atakayethubutu kumfasiri Mungu na kuweka katika akili za wengine maoni kuhusu Yeye zilizomo katika Hekalu Hai? \textbf{Nadharia hizi ni nadharia za mdanganyifu mkuu, na katika maisha ya \underline{wale wanaozipokea zitakuwa sura za huzuni}}. \textbf{Hiki ni mbinu ya Shetani \underline{ili kuyumbisha msingi wa imani yetu}, kutikisika imani yetu katika mwongozo wa Bwana na uzoefu ambao ametupa. \underline{Mambo mengi ya tabia kama hiyo yatatokea katika siku zijazo}}. Ninawasihi mamishonari wetu wa matibabu wasiamini dhana za kubuni za binadamu yeyote ambaye anaendeleza mawazo kwamba \textbf{njia ambayo watu wa Mungu wameongozwa kwa ajili yake miaka hamsini iliyopita ni njia mbaya}. \textbf{\underline{Jihadharini na wale ambao}, bila kuwa na uzoefu wa uamuzi wowote katika uongozi wa Roho wa Bwana, \underline{ungedhani kwamba uongozi huu ni udanganyifu}; kwamba hatuna ukweli}; kwamba sisi si watu wa Bwana, tuliokusanywa naye kutoka nchi na mataifa yote. \textbf{\underline{Jihadharini na wale ambao watabomoa msingi, ambayo tumekuwa tukijenga kwa miaka hamsini iliyopita, ili kuanzisha fundisho jipya}}. \textbf{Ninajua kwamba nadharia hizi mpya zinatoka kwa adui}.}[Ms137-1903.10; 1903][https://egwwritings.org/read?panels=p9939.17]


\egwnogap{\textbf{Let those who would \underline{bring in} fanciful ideas of God awake to a sense of their danger. This is too solemn a subject to be trifled with}.}[Ms137-1903.11; 1903][https://egwwritings.org/read?panels=p9939.18]


\egwnogap{\textbf{Acheni wale ambao \underline{wangeleta ndani} mawazo ya kidhahania juu ya Mungu waamke na kuhisi hatari yao. Hili ni somo zito sana haliwezi kuchezewa}.}[Ms137-1903.11; 1903][https://egwwritings.org/read?panels=p9939.18]


\egwnogap{The root of idolatry is an evil heart of unbelief in departing from the living God. It is because men have not faith in the presence and power of God \textbf{that they have been putting their trust in their own wisdom}. They have been devising and planning to exalt themselves and find salvation in their own works. \textbf{\underline{A deceptive influence from satanic agencies is coming in}, because leaders whom the Lord has warned and entreated and counseled are choosing their own wisdom in the place of the wisdom of God}. To such ones the warning comes, ‘Talk no more exceedingly proudly; let not arrogancy come out of your mouth; for the Lord is a God of knowledge, and by Him actions are weighed.’}[Ms137-1903.12; 1903][https://egwwritings.org/read?panels=p9939.19]


\egwnogap{Mzizi wa kuabudu sanamu ni moyo mbaya wa kutoamini katika kujitenga na Mungu aliye hai. Ni kwa sababu watu hawana imani katika uwepo na nguvu za Mungu \textbf{ndiposa wamekuwa wakitegemea hekima zao wenyewe}. Wamekuwa wakibuni na kupanga kujiinua na kupata wokovu katika kazi zao wenyewe. \textbf{\underline{Ushawishi wa udanganyifu kutoka kwa mashirika ya kishetani unakuja ndani}, kwa sababu ni viongozi ambao Bwana amewaonya na kuwasihi na kuwashauri wamechagua hekima yao wenyewe badala ya hekima ya Mungu}. Kwa watu kama hao onyo laja, ‘Msiseme tena kwa majivuno kupita kiasi; jeuri isitoke kweye mdomo yenu; kwa kuwa Bwana ni Mungu wa maarifa, na matendo hupimwa na yeye.’}[Ms137-1903.12; 1903][https://egwwritings.org/read?panels=p9939.19]


The difference between the old \emcap{Fundamental Principles} and the new Fundamental Beliefs is in our \egwinline{ideas of God.} The Trinitarian idea of God was not part of the foundation of our faith, which Sister White defended. How did this change take place? It was done through the leaders who chose \egwinline{their own wisdom in the place of the wisdom of God.} We should \egwinline{Beware of those who would tear down the foundation, upon which we have been building for the last fifty years, to establish a new doctrine.} In this observation, we recognize that this new Trinitarian idea of God was \egwinline{a deceptive influence from satanic agency} that came into our ranks.


Tofauti kati ya \emcap{Kanuni za Msingi} za zamani na Mafundisho za Kimsingi mpya iko katika \egwinline{mawazo yetu ya Mungu.} Wazo la Utatu kuhusu Mungu halikuwa sehemu ya msingi wa imani yetu, ambayo Dada White alitetea. Mabadiliko haya yalitokea vipi? Yalifanywa kupitia viongozi ambao walichagua \egwinline{hekima yao wenyewe badala ya hekima ya Mungu.} Tunapaswa \egwinline{Jihadhari na wale ambao wangebomoa msingi, ambao tumekuwa tukijenga kwa miaka hamsini iliyopita, ili kuanzisha fundisho jipya.} Katika uchunguzi huu, tunatambua kwamba wazo hili jipya la Utatu kuhusu Mungu lilikuwa \egwinline{ushawishi wa udanganyifu kutoka kwa mashirika ya kishetani} ambao uliingia katika safu zetu.


% Setting up the wrong Fundamental Principles

\begin{titledpoem}
    
    \stanza{
        Sadly, within our own church walls \\
        From our own pulpits, error falls \\
        Members want smooth words for their ears \\
        Don’t step on toes, Allay their fears.
    }

    \stanza{
        Pastors and elders do preside \\
        While sins remain, untouched inside \\
        Laodicean comfort zone \\
        But they will reap what they have sown.
    }

    \stanza{
        Ask for the old paths, walk therein \\
        From the old truth, don’t move a pin. \\
        They spurned the truth which brightly shone, \\
        The Spirit’s pow’r, to them unknown.
    }

    \stanza{
        Do not the humble hearts perceive \\
        Whispers of truth they should believe? \\
        Meanwhile the stories ease concern. \\
        What God would tell them, they won’t learn.
    }

    \stanza{
        Beware of error, thinly veiled, \\
        God’s Word is true, but men have failed. \\
        Beware of shadows leaders cast. \\
        To the foundations true, hold fast,
    }

    \stanza{
        Let not man’s wisdom lead astray, \\
        Let God’s own Spirit show the way. \\
        For in the Scripture’s glowing light, \\
        We find the path of safety bright.
    }

    \stanza{
        Let us, in faith, each day commence, \\
        God’s Word our shield, not man’s pretense. \\
        For truth in Christ alone is found, \\
        And on this rock, our faith is sound.
    }
    
\end{titledpoem}


% Setting up the wrong Fundamental Principles

\begin{titledpoem}
    
    \stanza{
        Sadly, within our own church walls \\
        From our own pulpits, error falls \\
        Members want smooth words for their ears \\
        Don’t step on toes, Allay their fears.
    }

    \stanza{
        Pastors and elders do preside \\
        While sins remain, untouched inside \\
        Laodicean comfort zone \\
        But they will reap what they have sown.
    }

    \stanza{
        Ask for the old paths, walk therein \\
        From the old truth, don’t move a pin. \\
        They spurned the truth which brightly shone, \\
        The Spirit’s pow’r, to them unknown.
    }

    \stanza{
        Do not the humble hearts perceive \\
        Whispers of truth they should believe? \\
        Meanwhile the stories ease concern. \\
        What God would tell them, they won’t learn.
    }

    \stanza{
        Beware of error, thinly veiled, \\
        God’s Word is true, but men have failed. \\
        Beware of shadows leaders cast. \\
        To the foundations true, hold fast,
    }

    \stanza{
        Let not man’s wisdom lead astray, \\
        Let God’s own Spirit show the way. \\
        For in the Scripture’s glowing light, \\
        We find the path of safety bright.
    }

    \stanza{
        Let us, in faith, each day commence, \\
        God’s Word our shield, not man’s pretense. \\
        For truth in Christ alone is found, \\
        And on this rock, our faith is sound.
    }
    
\end{titledpoem}
