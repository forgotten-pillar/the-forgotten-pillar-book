\qrchapter{https://forgottenpillar.com/rsc/en-fp-chapter6}{Examining the test}


\qrchapter{https://forgottenpillar.com/rsc/en-fp-chapter6}{Kuchunguza mtihani}


In Sister White's reply, to Dr. Kellogg's belief on the Trinity doctrine and his attempts to \textit{patch up} the Living Temple, we see that she viewed the Trinity doctrine as contradicting the light given her regarding \emcap{the personality of God}. If she had actually embraced the Trinity doctrine, we would expect her to carefully separate it from pantheism and preserve its legitimate aspects. However, this is not what we see in her response. Instead, her response was to contrast the Trinity doctrine with the truth about the \emcap{personality of God}, recalling her past visions which showed that this doctrine would rob God's people of their past experiences. In her reactive recalling of how God established the \emcap{fundamental principles}, she indicated that the Trinity doctrine \textit{tears down the pillars of our faith} and \textit{leads us astray from the foundation principles}. This stark difference can be clearly seen by comparing our current Fundamental Beliefs with the \emcap{Fundamental Principles} held in the past.


Katika jibu la Dada White kwa imani ya Dk. Kellogg kuhusu fundisho la Utatu na juhudi zake za \textit{kuweka viraka} Hekalu Hai, tunaona kwamba aliona fundisho la Utatu kama linapingana na nuru aliyopewa kuhusu \emcap{Umbile la Mungu}. Kama angekuwa amekubali fundisho la Utatu, tungetegemea awe makini kulitenganisha na upantheisti na kuhifadhi vipengele vyake halali. Hata hivyo, hili sio tunachoona katika jibu lake. Badala yake, jibu lake lilikuwa kulinganisha fundisho la Utatu na ukweli kuhusu \emcap{Umbile la Mungu}, akikumbuka maono yake ya zamani ambayo yalionyesha kwamba fundisho hili lingeibia watu wa Mungu uzoefu wao wa zamani. Katika kukumbuka kwake jinsi Mungu alivyoweka \emcap{kanuni za kimsingi}, alionyesha kwamba fundisho la Utatu \textit{linaangusha nguzo za imani yetu} na \textit{hutupotosha kutoka kwa kanuni za msingi}. Tofauti hii kubwa inaweza kuonekana wazi kwa kulinganisha Imani zetu za Msingi za sasa na \emcap{Kanuni za Msingi} zilizoshikiliwa zamani.


Keeping in mind Sister White’s reply to Dr. Kellogg's belief on the Trinity doctrine, let us review the characteristics of the theories she described in the chapter “\textit{The Foundation of our Faith}”. When Sister White is speaking of Kellogg’s theories of God, our question should be, “do her quotations make sense if the Trinity doctrine is applied to their context?” Let’s examine each characteristic.


Tukiweka akilini jibu la Dada White kwa imani ya Dk. Kellogg kuhusu fundisho la Utatu, tuchunguze upya sifa za nadharia alizozieleza katika sura ya “\textit{Msingi wa imani yetu}”. Dada White anapozungumza kuhusu nadharia za Kellogg za Mungu, swali letu linafaa kuwa, “je, manukuu yake yana maana ikiwa fundisho la Utatu linatumiwa kwa muktadha wao?” Hebu tuchunguze kila sifa.


\subsection*{Does the Trinity “rob the people of God of their past experience”?}


\subsection*{Je, Utatu “huwaibia watu wa Mungu mambo yao ya uzoefu wa zamani”?}


\egw{They \normaltext{[the spiritualistic theories]} make of no effect the truth of heavenly origin, and \textbf{rob the people of God of their past experience}, giving them instead a false science.}[SpTB02 54.1; 1904][https://egwwritings.org/read?panels=p417.275]


\egw{\normaltext{[Hizi nadharia za umizimu]} zinafanya kuwa bure ukweli wa asili ya mbinguni, na \textbf{kuwaibia watu wa Mungu uzoefu wao wa zamani}, na kuwapa badala yake sayansi ya uwongo.}[SpTB02 54.1; 1904][https://egwwritings.org/read?panels=p417.275]


\egw{This foundation was built by the Masterworker, and will stand storm and tempest. Will they permit this man \normaltext{[Kellogg]} to present \textbf{doctrines that deny the past experience of the people of God}? The time has come to take decided action.}[SpTB02 54.2; 1904][https://egwwritings.org/read?panels=p417.276]


\egw{Msingi huu ulijengwa na Mfanyakazi Mkuu, na utasimama dhoruba na tufani. Je! Watamruhusu mtu huyu \normaltext{[Kellogg]} kuwasilisha \textbf{mafundisho ambayo yanakinzana na uzoefu wa zamani wa watu wa Mungu}? Wakati umefika wa kuchukua hatua za dhati.}[SpTB02 54.2; 1904][https://egwwritings.org/read?panels=p417.276]


\egw{\textbf{What influence is it that would lead men at this stage of our history to work in an underhanded, powerful way to \underline{tear down the foundation of our faith},—the foundation that was laid \underline{at the beginning of our work} by prayerful study of the word and by revelation? Upon this foundation \underline{we have been building for the past fifty years}}. Do you wonder that when I see the beginning of a work \textbf{that would \underline{remove some of the pillars of our faith},} I have something to say? I must obey the command, ‘Meet it!’}[SpTB02 58.1; 1904][https://egwwritings.org/read?panels=p417.295]


\egw{\textbf{Ni ushawishi gani ambao unaweza kusababisha wanaume katika hatua hii ya historia yetu kufanya kazi katika njia ya chinichini, yenye nguvu ya \underline{kubomoa msingi wa imani yetu},—msingi ambayo iliwekwa \underline{mwanzoni mwa kazi yetu} kwa kujifunza neno kwa maombi na kwa ufunuo? Juu ya msingi huu \underline{tumekuwa tukijenga kwa miaka hamsini iliyopita}}. Unashangaa kwamba ninapoona chemichemi ya kazi ambayo \textbf{ingeondoa \underline{baadhi ya nguzo za imani yetu}}, nina la kusema? Ni lazima nitii amri, ‘Kutana nayo!’}[SpTB02 58.1; 1904][https://egwwritings.org/read?panels=p417.295]


According to Sister White’s testimony, the foundation of our faith was the \emcap{Fundamental Principles}. Currently, these do not represent our beliefs. Most objectionable is the first point, concerning who God is. Instead of the belief that there is one God—the Father, a personal spiritual being, we have a new belief that there is one God—Father, Son, and Holy Spirit, a unity of three Persons. From the light and the experiences of how God established the first point of the \emcap{Fundamental Principles}, does the newly formed doctrine about who God is and what He is, has robbed the people of God of their past experience?


Kulingana na ushuhuda wa Dada White, msingi wa imani yetu ilikuwa wa \emcap{Kanuni za Msingi}. Kwa sasa, hayawakilishi imani zetu. Jambo la kwanza lenye utatanishi zaidi ni, kuhusu Mungu ni nani. Badala ya imani kwamba kuna Mungu mmoja—Baba, huluki binafsi wa kiroho, tunayo imani mpya kwamba kuna Mungu mmoja—Baba, Mwana, na Roho Mtakatifu, umoja wa Nafsi tatu. Kutoka kwa nuru na uzoefu wa jinsi Mungu alivyoweka hoja ya kwanza ya \emcap{Kanuni za Msingi}, je, fundisho jipya kuhusu Mungu ni nani na yeye ni nini, limewaibia watu wa Mungu uzoefu wao wa zamani?


\subsection*{Does the Trinity tear down the pillars of our faith, or lead astray from foundation principles?}


\subsection*{Je, Utatu unabomoa nguzo za imani yetu, au unatupotosha kutoka kwa kanuni za msingi?}


\egw{I have been instructed by the heavenly messenger that some of the reasoning in the book, ‘Living Temple,’ is unsound and that \textbf{this reasoning would lead astray the minds of those who are not thoroughly established on the foundation principles of present truth.}}[SpTB02 51.3; 1904][https://egwwritings.org/read?panels=p417.262]


\egw{Nimefundishwa na mjumbe wa mbinguni kwamba baadhi ya hoja zilizomo ndani ya kitabu, ‘Hekalu Hai,’ halifai na kwamba \textbf{mawazo haya yangepotosha akili za hao ambao hawajathibitishwa kikamili kuhusu kanuni za msingi za ukweli wa sasa.}}[SpTB02 51.3; 1904][https://egwwritings.org/read?panels=p417.262]


\egw{About the time that ‘Living Temple’ was published, there passed before me in the night season, representations indicating that some \textbf{danger was approaching}, and that I must prepare for it by writing out the things God has revealed to me \textbf{regarding the foundation principles of our faith}.}[SpTB02 52.3; 1904][https://egwwritings.org/read?panels=p417.267]


\egw{Takriban wakati ambapo ‘Hekalu Hai’ lilipochapishwa, lilipita mbele yangu msimu wa usiku, uwakilishi unaoonyesha kwamba \textbf{hatari fulani ilikuwa inakaribia}, na kwamba lazima nijitayarishe kwa kuandika mambo ambayo Mungu amenifunulia \textbf{kuhusu kanuni za msingi za imani yetu}.}[SpTB02 52.3; 1904][https://egwwritings.org/read?panels=p417.267]


\egw{\textbf{The enemy of souls has sought to bring in the supposition that a great reformation was to take place among Seventh-day Adventists, and that this reformation would consist in \underline{giving up the doctrines which stand as the pillars of our faith,} and engaging in a process of reorganization}. Were this reformation to take place, what would result? \textbf{The principles of truth} that God in His wisdom has given to the remnant church, \textbf{would be discarded}. Our religion would be changed. \textbf{The fundamental principles} that have sustained the work for the last fifty years \textbf{would be accounted as error}. A new organization would be established. Books of a new order would be written. A system of intellectual philosophy would be introduced.}[SpTB02 54.3; 1904][https://egwwritings.org/read?panels=p417.277]


\egw{\textbf{Adui wa roho ametaka kuleta dhana kwamba matengenezo makubwa yangetukia kati ya Waadventista Wasabato, na kwamba matengenezo haya yangefanyika kwa \underline{kuacha mafundisho ambayo yanasimama kama nguzo za imani yetu,} na kujihusisha katika mchakato wa kujipanga upya}. Je, matengenezo haya yangefanyika, matokeo yangekuwa nini? \textbf{Kanuni za ukweli} ambazo Mungu katika hekima yake amepokeza kanisa la masalio, \textbf{zingetupiliwa mbali}. Dini yetu ingebadilishwa. \textbf{Kanuni za msingi} ambazo zimeendeleza kazi kwa miaka hamsini iliyopita \textbf{ingehesabiwa kama makosa}. Shirika jipya ingeanzishwa. Vitabu vya aina jipya vingeandikwa. Mfumo wa falsafa ya kiakili ingeanzishwa.}[SpTB02 54.3; 1904][https://egwwritings.org/read?panels=p417.277]


Dr. Kellogg’s theories on the \emcap{personality of God}, if accepted, would ignite a reformation within the Seventh-day Adventist Church. Based on intellectual philosophy, they would cause us to renounce some of the doctrines that stand as the pillars of our faith, condemning the \emcap{Fundamental Principles} as error. Could it be that by adhering to the Trinity doctrine we entered into a new organization?


Nadharia za Dk. Kellogg kuhusu \emcap{Umbile la Mungu}, ikiwa zitakubaliwa, zingechochea matengenezo ndani ya Kanisa la Waadventista Wasabato. Kulingana na falsafa ya kiakili, wangeweza kutufanya tukane baadhi ya mafundisho ambayo yanasimama kama nguzo ya imani yetu, na kuhukumu \emcap{Kanuni za Msingi} kama makosa. Je, inaweza kuwa hivi kwamba kwa kuambatana na fundisho la Utatu sisi tumeingia kwenye shirika jipya?


\egw{Shortly before I sent out the testimonies \textbf{regarding the efforts of the enemy to undermine the foundation of our faith through the dissemination of seductive theories}, I had read an incident about a ship in a fog meeting an iceberg…}[SpTB02 55.3; 1904][https://egwwritings.org/read?panels=p417.282]


\egw{Muda mfupi kabla sijatuma shuhuda \textbf{kuhusu juhudi za adui kudhoofisha msingi wa imani yetu kupitia kueneza nadharia potofu}, nilikuwa nimesoma tukio kuhusu meli kwenye ukungu ikikutana na jiwe la barafu…}[SpTB02 55.3; 1904][https://egwwritings.org/read?panels=p417.282]


\egw{Messages of every order and kind have been \textbf{urged upon Seventh-day Adventists, to take the place of the truth which, \underline{point by point}, has been sought out by prayerful study, and testified to by the miracle-working power of the Lord}. \textbf{But the way-marks which have made us what we are, are to be preserved, and they will be preserved}, as God has signified through His word and the testimony of His Spirit. \textbf{He calls upon us to hold firmly}, with the grip of faith, \textbf{to \underline{the fundamental principles} that are based upon \underline{unquestionable authority}}.}[SpTB02 59.1; 1904][https://egwwritings.org/read?panels=p417.299]


\egw{Ujumbe wa kila utaratibu na aina \textbf{umehimizwa kwa Waadventista Wasabato, ili kuchukua nafasi ya ukweli ambao, \underline{pointi kwa pointi}, umetafutwa kwa kujifunza Neno kwa maombi, na kushuhudiwa kwa uwezo wa kutenda miujiza wa Bwana}. \textbf{Lakini alama za njia ambazo zimetufanya tulivyo, zinapaswa kuhifadhiwa, na zitahifadhiwa}, kama Mungu ameonyesha kupitia neno lake na ushuhuda wa Roho wake. \textbf{Anatuita tuzishike kwa uthabiti}, kwa mshiko wa imani, \textbf{\underline{kanuni za msingi} ambazo mamlaka yao ni \underline{msingi usio na shaka}}.}[SpTB02 59.1; 1904][https://egwwritings.org/read?panels=p417.299]


The \emcap{personality of God} was the pillar of our faith\footnote{\href{https://egwwritings.org/?ref=en_Ms62-1905.14}{EGW, Ms62-1905.14; 1905}}. The \emcap{personality of God} was expressed in the first point of the \emcap{Fundamental Principles}. Could it be that by adhering to the Trinity doctrine we have torn down this particular pillar of our faith? Is it possible that by accepting the Trinity doctrine we were led astray from this foundation principle—the \emcap{personality of God}?


\emcap{Umbile la Mungu} ulikuwa nguzo ya imani yetu\footnote{\href{https://egwwritings.org/?ref=en_Ms62-1905.14}{EGW, Ms62-1905.14; 1905}}. \emcap{Umbile la Mungu} ulionyeshwa katika hoja ya kwanza ya \emcap{Kanuni za Msingi}. Je, inaweza kuwa hivyo kwa kushikamana na fundisho la Utatu tumeibomoa nguzo hii hasa ya imani yetu? Je, inawezekana kwa kukubali fundisho la Utatu tulipotoshwa kutoka kwenye kanuni hii ya msingi—\emcap{Umbile la Mungu}?


\subsection*{Does the Trinity do away with the personality of God?}


\subsection*{Je, Utatu unaondoa Umbile la Mungu?}


\egw{\textbf{It \normaltext{[The Living Temple]} introduces that which is naught but \underline{speculation} in regard to \underline{the personality of God} and where His presence is.}}[SpTB02 51.3; 1904][https://egwwritings.org/read?panels=p417.262]


\egw{\textbf{[Hekalu Hai] \normaltext{inatanguliza yale ambayo si kitu ila \underline{uvumi} kuhusiana na \underline{Umbile la Mungu} na mahali uwepo wake ulipo.}}}[SpTB02 51.3; 1904][https://egwwritings.org/read?panels=p417.262]


\egw{\textbf{The spiritualistic theories \underline{regarding the personality of God}, followed to their logical conclusion, sweep away the whole Christian economy.}}[SpTB02 54.1; 1904][https://egwwritings.org/read?panels=p417.275]


\egw{\textbf{Nadharia za kimizimu \underline{kuhusu Umbile la Mungu}, zikifuatwa hadi hitimisho zao kimantiki, hufagia utaratibu wote wa Kikristo.}}[SpTB02 54.1; 1904][https://egwwritings.org/read?panels=p417.275]


\egw{‘Living Temple’ contains the alpha of these theories. I knew that the omega would follow in a little while; and I trembled for our people. I knew that \textbf{I must warn our brethren and sisters not to enter into controversy over \underline{the presence} and \underline{personality of God}. The statements made in ‘Living Temple’ \underline{in regard to this point are incorrect}. The scripture used to substantiate the doctrine there set forth, is scripture misapplied}.}[SpTB02 53.2; 1904][https://egwwritings.org/read?panels=p417.271]


\egw{‘Hekalu Hai’ lina alfa ya nadharia hizi. Nilijua kuwa omega ingefuata kwa muda kidogo; na nikatetemeka kwa ajili ya watu wetu. Nilijua kwamba \textbf{lazima niwaonye ndugu na dada zetu kutoingia katika mabishano kuhusu \underline{uwepo} na \underline{Umbile la Mungu}. Kauli zilizotolewa katika ‘Hekalu Lililo Hai’ \underline{kuhusiana na jambo hili si sahihi}. Maandiko yanayotumiwa kuthibitisha fundisho lililowekwa hapo, yametumiwa vibaya}.}[SpTB02 53.2; 1904][https://egwwritings.org/read?panels=p417.271]


The theories Kellogg presented in the Living Temple are speculative in regard to the \emcap{personality of God} and where His presence is. These theories deal with the question of the quality or state of God being a person\footnote{The Merriam-Webster definition of ‘\textit{personality}’ - “\textit{the quality or state of being a person}”}. God has given us definite light regarding this issue in our \emcap{Fundamental Principles}. Could it be that the Trinity doctrine is casting doubt on this definite light regarding the \emcap{personality of God}?


Nadharia ambazo Kellogg aliwasilisha katika The Living Temple ni za kubahatisha kuhusiana na \emcap{ubinafsi wa Mungu} na mahali uwepo wake ulipo. Nadharia hizi zinahusika na swali la ubora au hali ya Mungu kuwa Nafsi\footnote{Ufafanuzi wa Merriam-Webster wa ‘\textit{personality}’ - “\textit{ubora au hali ya kuwa Nafsi}”}. Mungu ametupa nuru ya uhakika kuhusiana na suala hili katika \emcap{Kanuni zetu za Msingi}. Je, yawezekana kwamba fundisho la Utatu linatia shaka kuhusu nuru hii hususa kuhusu \emcap{ubinafsi wa Mungu}?


\subsection*{Is the Trinity doctrine presented as if Mrs. White supported it?}


\subsection*{Je, fundisho la Utatu linawasilishwa kana kwamba Bibi White aliliunga mkono?}


\egw{In the controversy that arose among our brethren \textbf{regarding the teachings of this book,} those in favor of giving it a wide circulation \textbf{declared: ‘It contains the very sentiments that Sister White has been teaching.’ This assertion struck right to my heart. I felt heart-broken; for I knew that this representation of the matter was not true}.}[SpTB02 53.1; 1904][https://egwwritings.org/read?panels=p417.270]


\egw{Katika mabishano yaliyotokea kati ya ndugu zetu \textbf{kuhusu mafundisho ya kitabu hiki,} wale waliopendelea kuisambaza \textbf{walitangaza hivi: ‘Ina hisia zile zile Dada White amekuwa akifundisha.’ Usemi huo uligusa moyo wangu kabisa. Nilihisi nimevunjika moyo; kwa maana nilijua kwamba uwakilishi huu wa mambo haukuwa wa kweli}.}[SpTB02 53.1; 1904][https://egwwritings.org/read?panels=p417.270]


\egw{\textbf{I am compelled to speak in denial of the claim that the teachings of ‘Living Temple’ can be sustained by statements from my writings}. There may be in this book expressions and sentiments that are in harmony with my writings. And there may be in my writings many statements which, taken from their connection, and interpreted according to the mind of the writer of ‘Living Temple,’ would seem to be in harmony with the teachings of this book. This may give apparent support to the assertion that the sentiments in ‘Living Temple’ are in harmony with my writings. \textbf{But God forbid that this sentiment should prevail}.}[SpTB02 53.3; 1904][https://egwwritings.org/read?panels=p417.272]


\egw{\textbf{Ninalazimika kusema kwa kukanusha madai kwamba mafundisho ya ‘Living Temple’ yanaweza kudumishwa na taarifa kutoka kwa maandishi yangu}. Kunaweza kuwa katika kitabu hiki maneno na hisia zinazopatana na maandishi yangu. Na kunaweza kuwa katika maandishi yangu taarifa nyingi ambazo, zikichukuliwa kutoka kwa uhusiano wao, na kufasiriwa kulingana na akili ya mwandishi wa ‘Living Temple,’ zingeonekana kupatana na mafundisho ya kitabu hiki. Hii inaweza kuunga mkono kwa dhahiri madai kwamba hisia katika ‘Living Temple’ zinapatana na maandishi yangu. \textbf{Lakini Mungu apishe mbali kwamba hisia hii idumu}.}[SpTB02 53.3; 1904][https://egwwritings.org/read?panels=p417.272]


At this point, we have many unanswered questions. But, as we continue to study the first point of the \emcap{Fundamental Principles}, we will find answers to all of these questions. So far, in light of the \emcap{Fundamental Principles}, belief in the Trinity doctrine—as a Seventh-day Adventist—becomes very questionable. In order to defend the Trinity doctrine, the authority of the \emcap{Fundamental Principles} must be compromised. In what follows, we will briefly study their authority, context in Adventist history, and God’s purpose in giving them. We will also look at the true authorship of the \emcap{Fundamental Principles} and their role in present days.


Katika hatua hii, tuna maswali mengi ambayo hayajajibiwa. Lakini, tunapoendelea kujifunza kipengele cha kwanza ya \emcap{Kanuni za Msingi}, tutapata majibu ya maswali haya yote. Hadi sasa, katika mwanga wa \emcap{Kanuni za Msingi}, imani katika fundisho la Utatu—kama Muadventista Msabato—inakuwa ya kutiliwa shaka sana. Ili kutetea fundisho la Utatu, mamlaka ya \emcap{Kanuni za Msingi} lazima zivunjwe. Katika mambo yafuatayo, tutajifunza kwa ufupi mamlaka yao, muktadha yao katika historia ya Waadventista, na kusudi la Mungu katika kuwapa. Tutachunguza vilevile uandishaji wa kweli wa \emcap{Kanuni za Msingi} na wajibu wao katika siku hizi.


% Examining Test

\begin{titledpoem}
    
    \stanza{
        The visions stand against the tide \\
        And all false doctrines are denied \\
        The testimony, clear and bright \\
        Expose the false, and bring forth light.
    }

    \stanza{
        The pillars which were set with care \\
        Now face a challenge, so beware \\
        The platform built by God’s wise plan \\
        Is weakened now by wayward man.
    }

    \stanza{
        God is a person, God’s church knew \\
        But since forgot, by words untrue \\
        Our past experience was robbed \\
        Untempered mortar has been daubed.
    }

    \stanza{
        The waymarks made us what we are, \\
        Should guide us still, our guiding star. \\
        Hold principles with faith’s strong grip, \\
        Lest in the fog we lose our ship.
    }
    
\end{titledpoem}


% Examining Test

\begin{titledpoem}
    
    \stanza{
        The visions stand against the tide \\
        And all false doctrines are denied \\
        The testimony, clear and bright \\
        Expose the false, and bring forth light.
    }

    \stanza{
        The pillars which were set with care \\
        Now face a challenge, so beware \\
        The platform built by God’s wise plan \\
        Is weakened now by wayward man.
    }

    \stanza{
        God is a person, God’s church knew \\
        But since forgot, by words untrue \\
        Our past experience was robbed \\
        Untempered mortar has been daubed.
    }

    \stanza{
        The waymarks made us what we are, \\
        Should guide us still, our guiding star. \\
        Hold principles with faith’s strong grip, \\
        Lest in the fog we lose our ship.
    }
    
\end{titledpoem}
