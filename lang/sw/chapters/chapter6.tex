\qrchapter{https://forgottenpillar.com/rsc/sw-fp-chapter6}{Kuchunguza mtihani}

Katika jibu la Dada White kwa imani ya Dk. Kellogg kuhusu fundisho la Utatu na juhudi zake za \textit{kuweka viraka} Hekalu Hai, tunaona kwamba aliona fundisho la Utatu kama linapingana na nuru aliyopewa kuhusu \emcap{Umbile la Mungu}. Kama angekuwa amekubali fundisho la Utatu, tungetegemea awe makini kulitenganisha na upantheisti na kuhifadhi vipengele vyake halali. Hata hivyo, hili sio tunachoona katika jibu lake. Badala yake, jibu lake lilikuwa kulinganisha fundisho la Utatu na ukweli kuhusu \emcap{Umbile la Mungu}, akikumbuka maono yake ya zamani ambayo yalionyesha kwamba fundisho hili lingeibia watu wa Mungu uzoefu wao wa zamani. Katika kukumbuka kwake jinsi Mungu alivyoweka \emcap{kanuni za kimsingi}, alionyesha kwamba fundisho la Utatu \textit{linaangusha nguzo za imani yetu} na \textit{hutupotosha kutoka kwa kanuni za msingi}. Tofauti hii kubwa inaweza kuonekana wazi kwa kulinganisha Imani zetu za Msingi za sasa na \emcap{Kanuni za Msingi} zilizoshikiliwa zamani.

Tukiweka akilini jibu la Dada White kwa imani ya Dk. Kellogg kuhusu fundisho la Utatu, tuchunguze upya sifa za nadharia alizozieleza katika sura ya “\textit{Msingi wa imani yetu}”. Dada White anapozungumza kuhusu nadharia za Kellogg za Mungu, swali letu linafaa kuwa, “je, manukuu yake yana maana ikiwa fundisho la Utatu linatumiwa kwa muktadha wao?” Hebu tuchunguze kila sifa.

\subsection*{Je, Utatu “huwaibia watu wa Mungu mambo yao ya uzoefu wa zamani”?}

\egw{\normaltext{[Hizi nadharia za umizimu]} zinafanya kuwa bure ukweli wa asili ya mbinguni, na \textbf{kuwaibia watu wa Mungu uzoefu wao wa zamani}, na kuwapa badala yake sayansi ya uwongo.}[SpTB02 54.1; 1904][https://egwwritings.org/read?panels=p417.275]

\egw{Msingi huu ulijengwa na Mfanyakazi Mkuu, na utasimama dhoruba na tufani. Je! Watamruhusu mtu huyu \normaltext{[Kellogg]} kuwasilisha \textbf{mafundisho ambayo yanakinzana na uzoefu wa zamani wa watu wa Mungu}? Wakati umefika wa kuchukua hatua za dhati.}[SpTB02 54.2; 1904][https://egwwritings.org/read?panels=p417.276]

\egw{\textbf{Ni ushawishi gani ambao unaweza kusababisha wanaume katika hatua hii ya historia yetu kufanya kazi katika njia ya chinichini, yenye nguvu ya \underline{kubomoa msingi wa imani yetu},—msingi ambayo iliwekwa \underline{mwanzoni mwa kazi yetu} kwa kujifunza neno kwa maombi na kwa ufunuo? Juu ya msingi huu \underline{tumekuwa tukijenga kwa miaka hamsini iliyopita}}. Unashangaa kwamba ninapoona chemichemi ya kazi ambayo \textbf{ingeondoa \underline{baadhi ya nguzo za imani yetu}}, nina la kusema? Ni lazima nitii amri, ‘Kutana nayo!’}[SpTB02 58.1; 1904][https://egwwritings.org/read?panels=p417.295]

Kulingana na ushuhuda wa Dada White, msingi wa imani yetu ilikuwa wa \emcap{Kanuni za Msingi}. Kwa sasa, hayawakilishi imani zetu. Jambo la kwanza lenye utatanishi zaidi ni, kuhusu Mungu ni nani. Badala ya imani kwamba kuna Mungu mmoja—Baba, huluki binafsi wa kiroho, tunayo imani mpya kwamba kuna Mungu mmoja—Baba, Mwana, na Roho Mtakatifu, umoja wa Nafsi tatu. Kutoka kwa nuru na uzoefu wa jinsi Mungu alivyoweka hoja ya kwanza ya \emcap{Kanuni za Msingi}, je, fundisho jipya kuhusu Mungu ni nani na yeye ni nini, limewaibia watu wa Mungu uzoefu wao wa zamani?

\subsection*{Je, Utatu unabomoa nguzo za imani yetu, au unatupotosha kutoka kwa kanuni za msingi?}

\egw{Nimefundishwa na mjumbe wa mbinguni kwamba baadhi ya hoja zilizomo ndani ya kitabu, ‘Hekalu Hai,’ halifai na kwamba \textbf{mawazo haya yangepotosha akili za hao ambao hawajathibitishwa kikamili kwenye kanuni za msingi za ukweli wa sasa.}}[SpTB02 51.3; 1904][https://egwwritings.org/read?panels=p417.262]

\egw{Takriban wakati ambapo ‘Hekalu Hai’ lilipochapishwa, lilipita mbeleni mwangu msimu wa usiku, uwakilishi unaoonyesha kwamba \textbf{hatari fulani ilikuwa inakaribia}, na kwamba lazima nijitayarishe kwa kuandika mambo ambayo Mungu amenifunulia \textbf{kuhusu kanuni za msingi za imani yetu}.}[SpTB02 52.3; 1904][https://egwwritings.org/read?panels=p417.267]

\egw{\textbf{Adui wa roho ametaka kuleta dhana kwamba matengenezo makubwa yangetukia kati ya Waadventista Wasabato, na kwamba matengenezo haya yangefanyika kwa \underline{kuacha mafundisho ambayo yanasimama kama nguzo za imani yetu,} na kujihusisha katika mchakato wa kujipanga upya}. Je, matengenezo haya yangefanyika, matokeo yangekuwa nini? \textbf{Kanuni za ukweli} ambazo Mungu katika hekima yake amepokeza kanisa la masalio, \textbf{zingetupiliwa mbali}. Dini yetu ingebadilishwa. \textbf{Kanuni za msingi} ambazo zimeendeleza kazi kwa miaka hamsini iliyopita \textbf{ingehesabiwa kama makosa}. Shirika jipya ingeanzishwa. Vitabu vya aina jipya vingeandikwa. Mfumo wa falsafa ya kiakili ingeanzishwa.}[SpTB02 54.3; 1904][https://egwwritings.org/read?panels=p417.277]

Nadharia za Dk. Kellogg kuhusu \emcap{Umbile la Mungu}, ikiwa zitakubaliwa, zingechochea matengenezo ndani ya Kanisa la Waadventista Wasabato. Kulingana na falsafa ya kiakili, wangeweza kutufanya tukane baadhi ya mafundisho ambayo yanasimama kama nguzo ya imani yetu, na kuhukumu \emcap{Kanuni za Msingi} kama makosa. Je, inaweza kuwa hivi kwamba kwa kuambatana na fundisho la Utatu sisi tumeingia kwenye shirika jipya?

\egw{Muda mfupi kabla sijatuma shuhuda \textbf{kuhusu juhudi za adui kudhoofisha msingi wa imani yetu kupitia kueneza nadharia potofu}, nilikuwa nimesoma tukio kuhusu meli kwenye ukungu ikikutana na jiwe la barafu…}[SpTB02 55.3; 1904][https://egwwritings.org/read?panels=p417.282]

\egw{Ujumbe wa kila utaratibu na aina \textbf{umehimizwa kwa Waadventista Wasabato, ili kuchukua nafasi ya ukweli ambao, \underline{pointi kwa pointi}, umetafutwa kwa kujifunza Neno kwa maombi, na kushuhudiwa kwa uwezo wa kutenda miujiza wa Bwana}. \textbf{Lakini alama za njia ambazo zimetufanya tulivyo, zinapaswa kuhifadhiwa, na zitahifadhiwa}, kama Mungu ameonyesha kupitia neno lake na ushuhuda wa Roho wake. \textbf{Anatuita tuzishike kwa uthabiti}, kwa mshiko wa imani, \textbf{\underline{kanuni za msingi} ambazo mamlaka yao ni \underline{msingi usio na shaka}}.}[SpTB02 59.1; 1904][https://egwwritings.org/read?panels=p417.299]

\emcap{Umbile la Mungu} ulikuwa nguzo ya imani yetu\footnote{\href{https://egwwritings.org/?ref=en_Ms62-1905.14}{EGW, Ms62-1905.14; 1905}}. \emcap{Umbile la Mungu} ulionyeshwa katika hoja ya kwanza ya \emcap{Kanuni za Msingi}. Je, inaweza kuwa hivyo kwa kushikamana na fundisho la Utatu tumeibomoa nguzo hii hasa ya imani yetu? Je, inawezekana kwa kukubali fundisho la Utatu tulipotoshwa kutoka kwenye kanuni hii ya msingi—\emcap{Umbile la Mungu}?

\subsection*{Je, Utatu unaondoa Umbile la Mungu?}

\egw{\textbf{[Hekalu Hai] \normaltext{inatanguliza yale ambayo si kitu ila \underline{uvumi} kuhusiana na \underline{Umbile la Mungu} na mahali uwepo wake ulipo.}}}[SpTB02 51.3; 1904][https://egwwritings.org/read?panels=p417.262]

\egw{\textbf{Nadharia za kimizimu \underline{kuhusu Umbile la Mungu}, zikifuatwa hadi hitimisho zao kimantiki, hufagia utaratibu wote wa Kikristo.}}[SpTB02 54.1; 1904][https://egwwritings.org/read?panels=p417.275]

\egw{‘Hekalu Hai’ lina alfa ya nadharia hizi. Nilijua kuwa omega ingefuata kwa muda kidogo; na nikatetemeka kwa ajili ya watu wetu. Nilijua kwamba \textbf{lazima niwaonye ndugu na dada zetu kutoingia katika mabishano kuhusu \underline{uwepo} na \underline{Umbile la Mungu}. Kauli zilizotolewa katika ‘Hekalu Lililo Hai’ \underline{kuhusiana na jambo hili si sahihi}. Maandiko yanayotumiwa kuthibitisha fundisho lililowekwa hapo, yametumiwa vibaya}.}[SpTB02 53.2; 1904][https://egwwritings.org/read?panels=p417.271]

Nadharia ambazo Kellogg aliwasilisha katika The Living Temple ni za kubahatisha kuhusiana na \emcap{ubinafsi wa Mungu} na mahali uwepo wake ulipo. Nadharia hizi zinahusika na swali la ubora au hali ya Mungu kuwa Nafsi\footnote{Ufafanuzi wa Merriam-Webster wa ‘\textit{personality}’ - “\textit{ubora au hali ya kuwa Nafsi}”}. Mungu ametupa nuru ya uhakika kuhusiana na suala hili katika \emcap{Kanuni zetu za Msingi}. Je, yawezekana kwamba fundisho la Utatu linatia shaka kuhusu nuru hii hususa kuhusu \emcap{ubinafsi wa Mungu}?

\subsection*{Je, fundisho la Utatu linawasilishwa kana kwamba Bibi White aliliunga mkono?}

\egw{Katika mabishano yaliyotokea kati ya ndugu zetu \textbf{kuhusu mafundisho ya kitabu hiki,} wale waliopendelea kuisambaza \textbf{walitangaza hivi: ‘Ina hisia zile zile Dada White amekuwa akifundisha.’ Usemi huo uligusa moyo wangu kabisa. Nilihisi nimevunjika moyo; kwa maana nilijua kwamba uwakilishi huu wa mambo haukuwa wa kweli}.}[SpTB02 53.1; 1904][https://egwwritings.org/read?panels=p417.270]

\egw{\textbf{Ninalazimika kusema kwa kukanusha madai kwamba mafundisho ya ‘Living Temple’ yanaweza kudumishwa na taarifa kutoka kwa maandishi yangu}. Kunaweza kuwa katika kitabu hiki maneno na hisia zinazopatana na maandishi yangu. Na kunaweza kuwa katika maandishi yangu taarifa nyingi ambazo, zikichukuliwa kutoka kwa uhusiano wao, na kufasiriwa kulingana na akili ya mwandishi wa ‘Living Temple,’ zingeonekana kupatana na mafundisho ya kitabu hiki. Hii inaweza kuunga mkono kwa dhahiri madai kwamba hisia katika ‘Living Temple’ zinapatana na maandishi yangu. \textbf{Lakini Mungu apishe mbali kwamba hisia hii idumu}.}[SpTB02 53.3; 1904][https://egwwritings.org/read?panels=p417.272]

Katika hatua hii, tuna maswali mengi ambayo hayajajibiwa. Lakini, tunapoendelea kujifunza kipengele cha kwanza ya \emcap{Kanuni za Msingi}, tutapata majibu ya maswali haya yote. Hadi sasa, katika mwanga wa \emcap{Kanuni za Msingi}, imani katika fundisho la Utatu—kama Muadventista Msabato—inakuwa ya kutiliwa shaka sana. Ili kutetea fundisho la Utatu, mamlaka ya \emcap{Kanuni za Msingi} lazima zivunjwe. Katika mambo yafuatayo, tutajifunza kwa ufupi mamlaka yao, muktadha yao katika historia ya Waadventista, na kusudi la Mungu katika kuwapa. Tutachunguza vilevile uandishaji wa kweli wa \emcap{Kanuni za Msingi} na wajibu wao katika siku hizi.

% Examining Test

\begin{titledpoem}
    
    \stanza{
        Sister White's vision stands against the tide, \\
        As Trinity's doctrine she firmly denied. \\
        Her words a warning, clear and bright, \\
        Against teachings that dimmed revealed light.
    }

    \stanza{
        The pillars of faith, established with care, \\
        Now face a challenge, a doctrine to beware. \\
        For what was built through prayer and revelation, \\
        Faces change through doctrinal innovation.
    }

    \stanza{
        The personality of God, once clearly known, \\
        Now wrapped in theories not heaven's own. \\
        Past experiences of God's people at stake, \\
        When foundations new doctrines would break.
    }

    \stanza{
        The waymarks that made us what we are, \\
        Should guide us still, like a guiding star. \\
        Hold firmly to principles with faith's strong grip, \\
        Lest in confusion's fog we lose our ship.
    }
\end{titledpoem}
