
\qrchapter{https://forgottenpillar.com/rsc/en-fp-chapter20}{Dr. Kellogg na maandiko ya Ellen White}

Dk. Kellogg alidai kwamba katika The Living Temple aliwakilisha hisia sawa zilizo-tetewa na Dada White. Vivyo hivyo, leo wengi wanadai kwamba Dada White alikuwa mfuata utatu na anawajibika kwa kukubali kwa kanisa fundisho la Utatu\footnote{William Johnsson, Adventist Review, January 6th, 1994, ‘\textit{Present Truth –Walking in God's Light}’}. Dada White mwenyewe, alitangaza madai hayo kuwa ya uongo.

\egw{\textbf{Adui anatafuta \underline{kuingiza} miongoni mwa watu wa Mungu nadharia za kimizimu, ambazo \underline{zikikubaliwa, zingedhoofisha msingi wa imani} ambayo imetufanya kuwa tulivyo}. Anawaongoza watu kuwasilisha hadithi zillizovikwa na Maandiko. \textbf{Wapo wanaodai kwamba maandishi ya Dada White yanapatana na mafundisho haya}.\textbf{ \underline{Natangaza kuwa hii ni uongo}. Watu wanaweza kutumia Maandiko vibaya; wanatafsiri vibaya maneno yangu; lakini Mungu anaelewa mipango yao}. Ninashukuru jinsi gani kwa hili! Adui anapoingia kama mafuriko, \textbf{Roho wa Bwana atatuinulia kiwango juu yake}.}[Ms137-1903.21; 1903][https://egwwritings.org/read?panels=p9939.30]

Dk. Kellogg alitetea nadharia ambazo zingekubaliwa zingedhoofisha msingi wa imani yetu. Ni muhimu kuelewa kwa usahihi kile kinachofanya msingi wa imani yetu, ambayo Dada White alirejelea. Tumeona kwamba inarejelea \emcap{Kanuni za Msingi}. Kuangalia katika maandishi yake, na maandishi ya waanzilishi wetu, tunaona kwamba fundisho la Utatu linapingana na \emcap{Umbile la Mungu} na ukweli kuhusu uwepo wa Mungu. Leo, na fundisho la Utatu kama sehemu ya imani yetu, tunatambua kwamba tumetoka kwenye \emcap{Kanuni za Msingi} na kuunda msingi mwingine. Dada White hakuhusika na mabadiliko haya. Ni tafsiri isiyo sahihi ya kazi zake. Maandishi yake hayavunji msingi wa imani ambayo imetufanya tulivyo. Kazi yake ya baadaye inapatana kabisa na ukweli uliotolewa hapo mwanzo.

\egw{\textbf{Miaka hamsini iliyopita haijafifisha yodi moja au kanuni ya imani yetu kama tulivyopokea ushahidi mkubwa na wa ajabu ambao ulifanywa kuwa uhakika kwetu mnamo 1844, baada ya kupita kwa wakati.} ... \textbf{\underline{Hakuna neno limebadilishwa au kukataliwa}. Ambacho Roho Mtakatifu alishuhudia kama ukweli baada ya kupita kwa wakati, katika kukatishwa tamaa kwetu kuu, \underline{ni msingi imara wa ukweli}. \underline{Nguzo za ukweli zilifunuliwa}, nasi tukakubali \underline{kanuni za msingi} ambazo zimetufanya tulivyo—Waadventista Wasabato, wazishikao amri za Mungu na kuwa na imani ya Yesu.}}[Lt326-1905.3; 1905][https://egwwritings.org/read?panels=p7678.9]

\section*{Upotoshaji wa msimamo wa kanisa}

Kwa kupotosha maandishi ya Dada White, Dk. Kellogg hakupotosha kazi yake tu, lakini msimamo rasmi wa kanisa unaoonyeshwa katika \emcap{Kanuni za Msingi}. Ellen White alimkemea Kellogg kwa kupotosha msimamo wa kanisa. Tunaposoma karipio hili, tukumbuk msimamo wa sasa wa kanisa juu ya \emcap{Umbile la Mungu} kama inavyolinganishwa na hatua ya kwanza ya \emcap{Kanuni za Msingi}.

\egw{Wewe \textbf{si timamu katika ukweli}. Kauli zako zilizopewa kwa waumini na makafiri \textbf{imetuwasilisha vibaya sisi kama watu ambao hawakuibadili kweli kwa upotofu}. Kauli hizo hufifisha ushawishi \textbf{ambao Mungu angetutaka tuwe nao mbele ya ulimwengu katika kufunua kwa wazi, kwa lugha isiyo na shaka kwamba sisi ni \underline{wa kusimama kidete kwa kanuni za imani yetu} na kwamba sisi tunashikilia mwanzo wa imani yetu imara hadi mwisho}. Sisi ni shirika madhubuti! \textbf{Tunaamini mnamo 1903 kweli zile zile tulizoamini tulipoanzisha Sanitarium na Chuo pale Battle Creek, na \underline{tunajua kwamba hatukuwa na ‘makama’ au ‘mana’ kuhusu hili jambo}}.}[Lt300-1903.4; 1903][https://egwwritings.org/read?panels=p7705.10]

\egwnogap{Wakati umewaambia mambo uliyo nayo na kutoa kauli ulizo nazo hapo awali kwa wasioamini, moyo wangu umekuwa na huzuni kweli. \textbf{Umethibitisha kuwa umeondoka kutoka kwa imani}. Maneno yale yale uliyoyasema mbele ya watu wenye ushawishi wa kidunia, kama karatasi zimeripoti maneno yako, zimewasilishwa kwangu wazi kutoka kwa midomo yako kama wewe wamezungumza nao. Hatuwezi kufanya kazi kukupa ushawishi kama mtu ambaye tunaweza kukwamini kazi takatifu inayohusiana na taasisi zetu, kwani unahitaji kwanza kuokoka na kuongozwa.}[Lt300-1903.5; 1903][https://egwwritings.org/read?panels=p7705.11]

\egwnogap{Wewe si timamu kiimani. Nimeeleza haya katika shajara yangu miezi kadhaa iliyopita. \textbf{Kwa hakika umeweka watu wa Mungu, ambao Bwana amewaongoza hatua kwa hatua katika njia za ukweli na kuwekwa juu ya \underline{msingi imara}, kwa upotovu mbele ya makafiri. Baadhi wameiacha imani na \underline{wataendelea kupotosha kazi ambayo Mungu alinipa}}.}[Lt300-1903.6; 1903][https://egwwritings.org/read?panels=p7705.12]

\egwnogap{\textbf{Swali la patakatifu ni fundisho lililo wazi na dhahiri kama tulivyolishikilia kama watu. \underline{Kwa hakika hauko wazi juu ya ubinafsi wa Mungu, ambao ni kila kitu kwetu kama watu}. \underline{Hakika umemwangamiza Bwana Mungu Mwenyewe}}.}[Lt300-1903.7; 1903][https://egwwritings.org/read?panels=p7705.13]

\egwnogap{Kwa nini uchukue uhuru wa kutoa kauli ulizotoa, kana kwamba ulikuwa na mamlaka ya kusema hivyo, ikiwa ni uwongo? \textbf{Umesababisha ukweli wa imani yetu kukosa maana mbele ya makafiri,} \textbf{na ukweli ambao unapaswa kudumishwa kwa umaarufu na kuinuliwa na watu hawa, umekaana na kupuuza katika kauli zako nyingi. Ulithubutu vipi kufanya hivi?} \textbf{Inatulazimu sasa kuwasilisha msimamo wetu wa kweli ambao unatufanya sisi kuwa Waadventista Wasabato}. Ushawishi wowote ambao Mungu alikupea hapo awali imekuwa katika rehema kwako, kwa kukuangazia nuru.}[Lt300-1903.8; 1903][https://egwwritings.org/read?panels=p7705.14]

\egwnogap{\textbf{Hatuwezi kwa wakati wowote ule kuwa na upotoshaji wowote juu ya haya maneno mazito na masomo muhimu ya ukweli ambayo yamekuwa imani ya watu wetu tangu 1844. Hii ina maana kubwa kwetu.} Bwana angetaka nikuambie kwamba adui, kupitia kwa madanganyo ya ajabu, ameweka kutokuamini akilini mwako, na umedhihirisha hiyo waziwazi. \textbf{\underline{Wote wanaopokea mawasilisho yako wataingia kwenye njia za ajabu ikiwa wataunganishwa nawe}}. \textbf{Unaleta \underline{ndani} moto wa ajabu, wa kawaida}, \textbf{lakini si moto wa kuwasha wa Mungu mwenyewe}; na sasa \textbf{lazima nizungumze kwa uwazi kwa watu wetu kwamba Bwana ametuongoza hatua kwa hatua na alituonyesha mwanga wa wazi juu ya patakatifu pa mbinguni katika patakatifu pa patakatifu sana ambapo \underline{Mungu alijidhihirisha yeye} kwa wateule wake.}}[Lt300-1903.9; 1903][https://egwwritings.org/read?panels=p7705.15]

Dk. Kellogg alipotosha ukweli uliokuwa msingi wa imani yetu; haswa kwa wingi, aliwakilisha vibaya ukweli juu ya \emcap{Umbile la Mungu}, ambao ulikuwa kila kitu kwetu sisi kama watu. Ikiwa mnamo 1903, ililazimu \egwinline{\textbf{kuwasilisha msimamo wetu wa kweli ambao unatufanya sisi kuwa Waadventista Wasabato}}, je, ni muhimu zaidi kwetu leo? Dada White alifanya sehemu yake katika kushikilia msingi wa imani yetu hapo mwanzo, lakini inaonekana ni kama tumeshasahau.

% Dr. Kellogg and Ellen White writings

\begin{titledpoem}
    \stanza{
        In faith's foundation, once so clear, \\
        Ellen White's words, we ought to revere. \\
        Kellogg claimed harmony, yet in deceit, \\
        Misrepresented truths, in his conceit.
    }

    \stanza{
        "The enemy seeks," she solemnly warned, \\
        To twist our beliefs, until they're scorned. \\
        Spiritualistic theories, cleverly dressed, \\
        In Scripture's garb, they falsely impressed.
    }

    \stanza{
        "False!" she declared, against the tide, \\
        Her writings and the Trinity, wrongly tied. \\
        Her defense was strong, her vision, broad, \\
        The Fundamental Principles, by God, lauded.
    }

    \stanza{    
        Not a word changed, nor principle dimmed, \\
        Since 1844, when light on faith brimmed. \\
        The pillars of truth, so firmly believed, \\
        By misinterpretations, were not deceived.
    }

    \stanza{
        Kellogg's stance, a misrepresentation, \\
        Of the faith's core, and its foundation. \\
        The personality of God, so crucial, so dear, \\
        By his theories, was muddled, we all fear.
    }

    \stanza{   
        Yet, Ellen White stood, unyielding, firm, \\
        Against falsehoods, her teachings confirm. \\
        Against the tide of Trinity's doctrine, she stood, \\
        Uplifting the truth of God and His Son, as she should.
    }

    \stanza{    
        Today, as then, let's hold the line, \\
        To the original faith, divine. \\
        Ellen White's legacy, let's rightly claim, \\
        In truth and spirit, always the same.
    }

    \stanza{    
        In battles of faith, let's not forget, \\
        Her defense of principles, firmly set. \\
        Against the currents of change and doubt, \\
        Her writings guide, without and within, throughout.
    }
\end{titledpoem}
