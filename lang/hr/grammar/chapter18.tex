I've reviewed the Croatian text and found a few grammar issues to correct:

# Corrected “sentimentima” to “sentimente” for proper accusative case
[“Upućena sam da kažem, \textbf{sentimentima} onih koji traže napredne znanstvene ideje \textbf{\underline{ne smije se vjerovati}}”]
->
[“Upućena sam da kažem, \textbf{sentimente} onih koji traže napredne znanstvene ideje \textbf{\underline{ne smije se vjerovati}}”]
---------

# Corrected “Kellogga” to “Kelloggu” for proper dative case
[“Bog ne prihvaća \underline{dr. Kellogga kao svog radnika}, osim ako se sada ne prekine sa Sotonom”]
->
[“Bog ne prihvaća \underline{dr. Kelloggu kao svog radnika}, osim ako se sada ne prekine sa Sotonom”]
---------

# Corrected “primjetili” to “primijetili” - proper spelling of the verb
[“Jeste li primjetili? Ne smije biti rasprave o pitanju što je Bog, osim ukoliko \egwinline{Bog ne bi dao nedvojbeno otkrivenje} o \egwinline{tome što On jest}”]
->
[“Jeste li primijetili? Ne smije biti rasprave o pitanju što je Bog, osim ukoliko \egwinline{Bog ne bi dao nedvojbeno otkrivenje} o \egwinline{tome što On jest}”]
---------