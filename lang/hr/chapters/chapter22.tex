\chapter{Suština problema}

Danas, kada uspoređujemo naša trenutna Osnovna Vjerovanja s prethodnim \emcap{Fundamentalnim Principima}, vidimo promjenu u temeljima vjere Adventista Sedmog Dana. Ova promjena se dogodila u razumijevanju Božje osobe, ili \emcap{ličnosti Boga}. Posebno u vezi s \emcap{ličnosti Boga}, sestra White je napisala da staza istine leži blizu staze zablude:

\egw{\textbf{Staza istine leži blizu staze zablude}, i obje staze mogu \textbf{izgledati} jednake u umovima koji nisu vođeni Duhom Svetim, i koji, stoga, \textbf{nisu sposobni} brzo razlučiti razliku između istine i zablude.}[SpTB02 52.2; 1904][https://egwwritings.org/read?panels=p417.266]

Pitamo se, kako možemo povući jasnu liniju između ove dvije staze? Da bismo to učinili, moramo doći do suštine problema. Moramo pronaći razlikovni princip koji razdvaja ove dvije staze.

Proučavajući naše trenutno trinitarijansko vjerovanje i spise naših pionira u vezi s \emcap{ličnosti Boga}, pronašli smo jedno karakteristično načelo koje razlikuje istinu o \emcap{ličnosti Boga}, kako su je držali naši pioniri, od našeg trenutnog trinitarijanskog vjerovanja. Obje strane tvrde da je Biblija njihov krajnji autoritet, no razlike se mogu razaznati prema tumačenju Biblije. U nastavku govorimo o razumijevanju i tumačenju Svetog pisma u vezi s ličnosti Boga. Razumijevanje Božje osobe može se predstaviti u dva različita, međusobno isključiva shvaćanja, koja jasno povlače liniju između dvaju suprotstavljenih tabora.

Jedno, popularnije, gledište je da se Bog predstavio u jeziku koji nam je poznat kako bi objasnio samo koncepte spasenja. Dakle, Bog se predstavio riječima poput ‘\textit{otac}’, ‘\textit{sin}’ i ‘\textit{duh}’, kako bi opisao odnose između tih pojmova. To znači da nijedna od tih riječi nije tumačiva prema njihovom očiglednom značenju; umjesto toga, one imaju simboličku ili metaforičku vrijednost. Načelo iza ovog razmišljanja je: \textbf{Bog je sebe prilagodio čovjeku}.

Drugo, dijametralno suprotno gledište je da je \textbf{Bog čovjeka prilagodio sebi}; \textit{stvorio je čovjeka na svoju sliku}. Stoga, riječi poput ‘\textit{otac}’, ‘\textit{sin}’ i ‘\textit{duh}’, kada se odnose na Boga, podrazumijevaju njihovo očito značenje. To je temeljna razlika.

Kada dođemo do razumijevanja biblijskih pojmova poput ‘\textit{osoba}’, ‘\textit{otac}’, ‘\textit{sin}’ i ‘\textit{duh}’, moramo odabrati koje gledište podržavamo i primijeniti ga u skladu s tim. Ili se ti pojmovi razumiju u njihovom očiglednom značenju, ili simbolički ili metaforički. Nema srednjeg puta između ovih dvaju; moramo odabrati jedno. Sljedeći citat trebao bi razriješiti svaku dilemu.


\egw{\textbf{\underline{Jezik Biblije treba tumačiti u njegovu očitom značenju, osim ukoliko se ne radi o simbolu ili slici}}.}[GC 598.3; 1888][https://egwwritings.org/read?panels=p133.2717]

Vjerujemo da je nemoguće da Biblija bude vlastiti tumač a da ne objašnjava vlastite simbole. Ako Biblija primjenjuje riječ ‘otac’ na Boga, ali nikada ne objašnjava taj pojam, tada bi ga trebalo prihvatiti u njegovom očiglednom značenju. Isto vrijedi i za riječi ‘sin’ i ‘duh’. Čovjek je stvoren na sliku Božju. Bog je prilagodio čovjeka sebi. Očigledno značenje proizlazi iz ljudskog iskustva. Očigledno značenje riječi ‘otac’ razumijemo kroz doslovno, ljudsko očinstvo. Ali naše očinstvo je slika našeg Boga koji je Otac svome Sinu. Pavao je svjedočio:

\bible{Stoga prigibam svoja koljena pred \textbf{Ocem Gospodina našega Isusa Krista, od koga se imenuje svako \underline{očinstvo} na nebesima i na zemlji}}[Efežanima 3:14-15].

U grčkom, riječ ‘\textit{očinstvo}’ je riječ ‘\textit{patria}’, izvedena iz riječi ‘\textit{pater}’, što znači ‘\textit{otac}’. Otac našega Gospodina Isusa Krista je zaista otac svome Sinu, baš kao što smo i mi očevi svojoj djeci na Zemlji. Naše očinstvo na Zemlji imenovano je prema Očinstvu na Nebu, gdje je Bog Otac našega Gospodina Isusa Krista. Naše zemaljsko očinstvo je slika Nebeskog Očinstva, gdje je Bog Otac svome Sinu. Ovo podržava očigledno značenje da je Isus zaista Sin našega Boga.

Isto temeljno načelo primjenjuje se na razumijevanje iza riječi ‘\textit{duh}’ i riječi ‘\textit{biće}’. Bog je prilagodio čovjeka sebi; stvorio je čovjeka na svoju sliku. Čovjek je biće koje posjeduje tijelo i duh, baš kao i Bog — i u tome ne tvrdimo da čovjek i Bog posjeduju istu prirodu. Bog je oblikovao čovjeka od praha zemaljskog. Njegova fizička priroda je ograničena na elemente pronađene na zemlji. Ne zalazimo u prirodu Boga. To će nam zauvijek ostati tajna; nije nam otkriveno. Ali ono što nam je otkriveno je da On ima oblik, i oblik čovjeka je slika oblika Božjeg. Biblija jasno odobrava ovo razumijevanje kada opisuje Boga kako sjedi na svom prijestolju:

\bible{\textbf{na tom obličju prijestolja, gore na njemu, bijaše obličje izgledom nalik \underline{čovjeku}}}[Ezekiel 1:26].

Očigledno značenje riječi ‘\textit{duh}’, primijenjeno na Božjeg Duha, proizlazi iz razumijevanja “duha čovjeka”. Bog je prilagodio čovjeka sebi; stvorio je čovjeka na svoju sliku. Baš kao što čovjek posjeduje duha, Bog posjeduje Duha. Duh čovjeka ima prirodu čovjeka, a Duh Božji ima prirodu Boga. Što se tiče njihove prirode, nisu isti, ali s obzirom na njihov odnos prema unutarnjem biću, oni su isti; Biblija ih stavlja na istu razinu. \bible{\textbf{Sam \underline{Duh} svjedoči s \underline{duhom našim}} da smo djeca Božja:}[Rimljanima 8:16]; \bible{Jer tko od \textbf{ljudi zna što je u čovjeku} osim \textbf{\underline{duha čovječjega} koji je u njemu}? \textbf{\underline{Tako i}} što je u \textbf{Bogu}, nitko ne zna, \textbf{osim \underline{Duha Božjega}}.}[1. Korinćanima 2:11].

Što se tiče obiteljskih odnosa i kvalitete ili stanja koja nekog čine osobom, čovjek i Bog su slični, jer je Bog stvorio čovjeka na svoju sliku. Bog je prilagodio čovjeka sebi. Ali u svojoj prirodi, Bog i čovjek nisu slični. Bog je božanski, a čovjek je zemaljski.

Doktrina o Trojstvu pridržava se razumijevanja da je Bog sebe prilagodio čovjeku i da je Bog samo koristio pojmove ‘\textit{otac}’, ‘\textit{sin}’ i ‘\textit{duh}’ kako bismo ga bolje razumjeli. Ova ideja podupire i pokreće trinitarijanski pogled. U daljnjem tekstu nećemo opsežno razlagati našu trinitarijansku literaturu, već ćemo podržati našu tvrdnju s nekoliko službenih izjava Crkve Adventista Sedmoga Dana.

Prva izjava dolazi od Biblijskog Istraživačkog Instituta, službene ustanove Generalne Konferencije, koja promiče učenja i doktrine Crkve Adventista Sedmoga Dana. Oni otvoreno negiraju roditeljski odnos između Oca i Njegovog Sina, u korist metaforičkog razumijevanja.

\others{Slika otac-sin \textbf{ne može se doslovno primijeniti na nebeski odnos Otac-Sin unutar Božanstva}. \textbf{Sin nije prirodno, doslovno Sin od Oca} ... \textbf{Termin ‘Sin’ se koristi metaforički} kada se primjenjuje na Božanstvo.}[Adventistički Biblijski Istraživački Institut; također objavljeno u službenom časopisu ‘Adventist World’][https://www.adventistbiblicalresearch.org/materials/a-question-of-sonship/]

Što se tiče \emcap{ličnosti Boga} u kontekstu trinitarijanskog pogleda, Crkva Adventista Sedmoga Dana izdala je sljedeće izjave u subotnjoj školskoj pouci:

\others{\textbf{Riječ osobe korištene u naslovu današnje pouke mora se razumijeti u teološkom smislu}. \textbf{Ako izjednačujemo ljudsku ličnost sa Bogom, tada bi rekli da tri osobe znači tri individue. Tada bi onda imali tri Boga, ili triteizam}. \textbf{Ali povijesno Kršćanstvo je dalo riječi osoba, kada se primjenjuje za Boga, \underline{posebno značenje}}: osobno samoodvojivo, koje daje razdvojenost u Osobama Božanstva bez uništavanja koncepta jedinstva. \textbf{Ovu ideju je teško shvatiti ili objasniti! \underline{Ona je dio misterije Božanstva}}.}[“Lesson 3.” Ssnet.org, 2025, \href{http://www.ssnet.org/qrtrly/eng/98d/less03.html}{www.ssnet.org/qrtrly/eng/98d/less03.html}. Accessed 3 Feb. 2025.]

\others{Ovi tekstovi i drugi nas navode da vjerujemo da je \textbf{naš predivan Bog \underline{tri Osobe u jednoj}}, umno nedokućiva \textbf{misterija} ali istina koju prihvaćamo vjerom jer Pismo tako kaže.}[Ibid.]

Prema službenim izjavama predstavljenim u subotnjoj školskoj pouci, riječ \textit{‘osoba’}, u vezi s Bogom, ne treba izjednačavati s ljudskom ličnošću, već je treba primijeniti u teološkom smislu. To je u oštrom kontrastu s viđenjem koje je sestra White imala u vezi s ličnosti Boga. \egwinline{‘Često sam viđala dragog Isusa, da \textbf{je On osoba}. Upitala sam Ga je li \textbf{Njegov Otac osoba}, i \textbf{da li ima \underline{oblik} kao i On}. Isus je odgovorio: ‘\textbf{Ja sam savršena slika osobe mojega Oca!}’ [Hebrejima 1:3.]}[Lt253-1903.12; 1903][https://egwwritings.org/read?panels=p14068.9980018] Njeno razumijevanje kvalitete ili stanja koja Boga čine osobom je da je Bog osoba u očitom smislu—On posjeduje oblik. Na isti način na koji je prepoznala Isusa kao osobu, Isus je svjedočio da je Bog osoba, imajući oblik kao i On. Suprotno očiglednom i doslovnom pogledu je duhovni pogled. Ona nastavlja govoriti o pogrešci duhovnog pogleda. \egwinline{\textbf{Često sam vidjela da je \underline{duhovni pogled} oduzeo Nebu svu slavu i da u mislima mnogih Davidovo prijestolje i ljupka osoba Isusa izgaraju u ognju spiritualizma}. Vidjela sam da će neki koji su bili prevareni i dovedeni u tu zabludu upoznati svjetlo istine, \textbf{ali će biti gotovo nemoguće da se potpuno oslobode zavodničke sile spiritualizma. Takvi trebaju učiniti temeljito djelo u priznavanju svojih zabluda i zauvijek ih napustiti}.}[Lt253-1903.13; 1903][https://egwwritings.org/read?panels=p14068.9980019] Prema subotnjoj školskoj pouci, očito razumijevanje pojma \textit{‘osoba’} je netočno jer bi to \others{\textbf{izjednačilo ljudsku ličnost sa Bogom}}, što znači da \others{\textbf{tri osobe znači tri individue}}. Suprotno očitom pogledu je teološki pogled. Za sestru White, suprotno je duhovni pogled. Ovaj pogled oduzima \egwinline{Nebu svu slavu i da u mislima mnogih Davidovo prijestolje i ljupka osoba Isusa izgaraju u ognju spiritualizma}. U spisima naših pionira, koje smo prethodno ispitali, prepoznajemo istinitost njezine tvrdnje. Predstavljeni teološki pogled na Božju ličnost ukida istinu o ličnosti Boga koju je sestra White primila u viziji. Teološki pogled objašnjava jednog Boga, koji je osoba, ali tri osobe, sastavljene od tri različita Boga—Boga Oca, Boga Sina i Boga Duha Svetog. Biblija nikada ne objašnjava Boga s takvom kvalitetom ili stanjem koja bi ga činila osobom. To se jednostavno pretpostavlja od strane trinitarijanskih vjernika i, budući da ono nikada nije objašnjeno, smatra se Božjom misterijom, dok zapravo—to je zabluda.

Kada povučemo liniju između istine i zablude, također moramo povući liniju između stvari koje su tajna i onih koje su otkrivene. Što se tiče prirode Boga, šutnja je elokventnost. Nažalost, mnogi koji zagovaraju doktrinu o Trojstvu ne uspijevaju povući ovu liniju na pravom mjestu. Protestiramo protiv ideje da je \emcap{ličnost Boga}, odnosno kvaliteta ili stanje koja Boga čini osobom jeste tajna. Naši pioniri su to razumjeli i jasno su to objasnili iz Biblije. Da nisu čitali i prihvatili Bibliju u njenom jasnom i jednostavnom jeziku, ne bi mogli objasniti \emcap{ličnost Boga}.

Postoje braća koja se u potpunosti slažu s ličnosti Boga izloženom u Fundamentalnim Principima. Slažu se da pojmove ‘\textit{otac}’, ‘\textit{sin}’ i ‘\textit{duh}’ treba tumačiti prema njihovom očitom značenju, ali i dalje zagovaraju doktrinu o Trojstvu jer ne uspijevaju pravilno povući liniju između onoga što Bog otkriva i onoga što nije. Argument ide otprilike ovako: da, Bog jeste osobno, duhovno biće; On ima neku vrstu tijela, Krist je Njegov jedinorođeni Sin, a Duh Sveti je Njihov predstavnik, ali sve to vrijedi za naš fizički svemir, koji je ograničen prostorom i vremenom; izvan prostora i vremena, Bog je Trojstvo.

Takvo gledište ne uspijeva povući liniju između onoga što je otkriveno i onoga što je tajna. Jedna posljedica takve koncepcije Boga je da baca sumnju na stvari koje su nam otkrivene. Prepoznati to zahtijeva iskrenost jer je vrlo primamljivo zamišljati Boga izvan prostora, ali to je, u konačnici, neopravdano jer smo mi konačni i vezani prostorom i vremenom. U svojoj knjizi, Živi Hram, dr. Kellogg je zamišljao Boga izvan \others{granice prostora i vremena}. Dr. Kellogg se protivio koncepciji Boga prikazanoj u \emcap{Fundamentalnim Principima}, jer je Bog, u svojoj ličnosti, bio vezan za svoje tijelo i tako ‘\textit{ograničen}’ na jednu lokaciju, recimo hram ili prijestolje na nebu\footnote{\href{https://archive.org/details/J.H.Kellogg.TheLivingTemple1903/page/n31/mode/2up}{John H. Kellogg, The Living Temple, p. 31}}. To je za dr. Kellogga bilo neprofitabilno, pa je zagovarao da je Bog daleko izvan našeg shvaćanja, kao i granica prostora i vremena.

\others{\textbf{\underline{Rasprave o Božjem obliku su krajnje besmislene}, i služe samo da podcjenjuju naše koncepcije o onome tko je iznad svih stvari}, \textbf{i stoga se ne može usporediti u obliku ili veličini ili slavi ili veličanstvu s bilo čim što je čovjek ikada vidio ili koji je unutar svoje moći da zamisli}. U prisutnosti ovakvih pitanja moramo samo priznati našu glupost i nesposobnost, i pokloniti naše glave strahopoštovanjem i poštovanjem \textbf{u prisutnosti Ličnosti, Inteligentnog Bića} za postojanje koje sva priroda nosi određeno i pozitivno svjedočanstvo, \textbf{ali što je daleko izvan našeg shvaćanja \underline{kao što su granice prostora i vremena}}.}[The Living Temple, pg. 33][https://archive.org/details/J.H.Kellogg.TheLivingTemple1903/page/n33/mode/2up]

Dr. Kellogg je bio ukoren zbog svojih koncepcija o Bogu. Njegova koncepcija Boga bila je Bog izvan granica prostora i vremena. Ova koncepcija je problematična jer je izvan granica Svetog pisma; to je čista pretpostavka koja baca sumnju na objavu iz Svetog pisma. Ako Sveto pismo svjedoči da je Bog određen, opipljiv i prisutan na jednom mjestu više nego na drugom, tada su sve rasprave o Bogu izvan prostora potpuno neprofitabilne. Takve rasprave obično vode prema skepticizmu o samim koncepcijama Boga o kojima Sveto pismo jasno svjedoči. Kao što se sjećamo, to je bio glavni problem s dr. Kelloggom, a sestra White nam je dala mnogo upozorenja u vezi s tim pitanjem.

\egw{‘Što je sakriveno, pripada GOSPODU, Bogu našemu, a što je otkriveno, pripada nama i sinovima našim dovijeka.’ Ponovljeni zakon 29:29. \textbf{Objavu samoga sebe koju je Bog dao u svojoj riječi je za naše proučavanje}. \textbf{To možemo nastojati razumjeti}. \textbf{\underline{Ali izvan toga ne smijemo prodirati}}. \textbf{Najviši intelekt može se naprezati dok se ne iscrpi u \underline{pretpostavkama}\footnote{\href{https://www.merriam-webster.com/dictionary/conjectures}{Merriam Webster Dictionary} - \textit{conjectures} - ‘\textit{pretpostavka}’ - “\textit{a: zaključak formiran bez dokaza ili dovoljno dokaza; b: zaključak izveden nagađanjem ili pogađanjem}”} \underline{o prirodi Boga}, ali trud će biti uzaludan}. \textbf{Ovaj problem nam nije dan da ga riješimo. Nijedan ljudski um ne može shvatiti Boga.} \textbf{Nitko ne smije ulaziti u špekulacije o Njegovoj prirodi. Ovdje je šutnja elokventnost. Sveznajući je iznad rasprave}.}[MH 429.3; 1905][https://egwwritings.org/read?panels=p135.2227]

\egw{Kažem, i uvijek sam govorila, \textbf{da se neću upuštati u kontroverzu s ikim u vezi s prirodom i ličnosti Boga}. \textbf{Neka oni koji pokušavaju opisati Boga znaju da je na takvu temu šutnja elokventnost}. \textbf{\underline{Neka se Sveto pismo čita u jednostavnoj vjeri, i neka svatko oblikuje svoje koncepcije o Bogu iz Njegove nadahnute Riječi}}.}[Lt214-1903.9; 1903][https://egwwritings.org/read?panels=p10700.15]

\egw{Nijedan ljudski um ne može shvatiti Boga. Nitko Ga nikada nije vidio. Mi smo u vezi s Bogom neuki kao mala djeca. Ali kao mala djeca možemo Ga voljeti i slušati. \textbf{Da je to bilo shvaćeno, takvi sentimenti kakvi su u ovoj knjizi nikada ne bi bili izrečeni}.}[Lt214-1903.10; 1903][https://egwwritings.org/read?panels=p10700.16]

Možda se pitate zašto je sestra White rekla da se neće upuštati u kontroverzu s ikim u vezi s prirodom i \emcap{ličnosti Boga}, dok je bila snažno angažirana u kontroverzi oko \emcap{ličnosti Boga} i shodno s time napisala mnoga različita svjedočanstva. Rasprave o \emcap{ličnosti Boga}, donekle, dotiču Božju prirodu; ipak, one koje se odnose na prirodu Boga, u vezi s \emcap{ličnosti Boga}, sestra White nije vodila. Znala je gdje povući liniju. Ukazala je da bi Biblija trebala povući tu liniju za nas. \egw{\textbf{\underline{Neka se Sveto pismo čita u jednostavnoj vjeri, i neka svatko oblikuje svoje koncepcije o Bogu iz Njegove nadahnute Riječi.}}} \emcap{Fundamentalni Principi} poštuju ovo pravilo. Sestra White nam je rekla da ne smijemo pokušavati objašnjavati \emcap{ličnost Boga} dalje nego što je to učinila Biblija.

\egw{Drži pogled uperen u Gospodina Isusa Krista, i gledajući Njega biti ćeš promijenjen u Njegovu sliku. \textbf{Nemoj govoriti o tim spiritualističkim teorijama. Neka oni ne nađu mjesta u tvojemu umu.} Neka naši časopisi budu čuvani čistima od svega te vrste. Objavljuj istinu; nemoj objavljivati zabludu. \textbf{Nemoj pokušati objašnjavati vezano uz ličnost Boga. \underline{Ne možeš dati nikakvo dodatno objašnjenje pored onoga kojeg je Biblija dala}}. \textbf{Ljudske teorije vezano za Njega su dobre nizašto}. Nemoj zamrljati svoj um proučavajući varljive teorije neprijatelja. Pokušaj odvuću umove od svega sličnoga. Bolje je držati te teme izvan naših časopisa. Neka doktrine sadašnje istine budu stavljene u naše časopise, ali ne daj prostora ponavljanju pogrešnih teorija.}[Lt179-1904.4; 1904][https://egwwritings.org/read?panels=p7751.11]

Neka Biblija oblikuje naše koncepcije o Bogu. Ne možemo dati nikakvo daljnje objašnjenje o \emcap{ličnosti Boga} nego što je to dala Biblija. Ako Biblija govori o Bogu da je, u svojoj osobi, vezan za jednu lokaciju, poput Njegovog hrama, svetišta i prijestolja, trebali bismo to prihvatiti bez obzira na to zvuči li to ograničavajuće za Boga. Bog je ograničen u prostoru, u svom tijelu, ali Njegova prisutnost nije ograničena, jer je On svugdje prisutan preko svog predstavnika, Duha Svetoga.

Objava Boga doista izražava neka Njegova ograničenja, a neka od njih su od značaja za spasenje. Na primjer, Biblija jasno kaže da je Bog svemoguć (Otkrivenje 19:6), može sve učiniti, a ipak nalazimo da je mogao spasiti ljude jedino tako što je dao svoga jedinorođenog Sina za nas. U Getsemanskom vrtu, kada je Bog predao čašu svog gnjeva svome Sinu, Krist je molio ako je moguće da Ga ova čaša mimoiđe, ali je na kraju molio da bude volja Božja. Ovdje vidimo sve dostupne opcije koje je Otac imao kako bi spasio ljude. Nije bilo moguće spasiti pali ljudski rod, osim da Božji Sin umre umjesto njih. Mnogi protestiraju protiv ideje da nešto nije bilo moguće za Boga. Ali ako je bilo moguće za Boga spasiti ljude bez da Njegov Sin ispije čašu Njegova gnjeva, zasigurno bi to Bog učinio. Neki protestiraju protiv ove ideje da je Bog bio ograničen na samo jednu opciju za spašavanje ljudi, dok bi On mogao imati beskonačne opcije — On je, naposljetku, svemoguć. S ovakvim razmišljanjem, Božje spasenje izgubljenih ljudi žrtvom Njegovog jedinorođenog Sina obavijeno je sumnjom, i u suštini odbačeno, čak i prezreno, prikazujući Boga kao ubojicu djece. Ali objava je jasna i prkosna ovim skepticima. Nije Bog taj koji je okrutan zbog toga što je dao svog Sina za nas; grijeh je taj koji je okrutan. Grijeh je zahtijevao ovu beskonačnu žrtvu, i nije bilo drugog načina. To nije bila igra uloga\footnote{Tjedan molitve, izdanje Adventist Review, 31. listopada 1996.}, već stvarnost koja je uzrokovala beskonačnu tugu i patnju našem nebeskom Ocu dajući svoga vlastitog rođenog\footnote{Čitajte o Božjem daru Njegovog \egwinline{vlastitog jedinorođenog Sina} u \href{https://egwwritings.org/?ref=en_Lt13-1894.18&para=5486.24}{{EGW, Lt13-1894.18; 1894}}}, poslušnog Sina da umre umjesto nas.

Neka naše koncepcije o tome tko je Bog, što je Bog i kakav je Njegov karakter budu oblikovane jasnim Svetim Pismom, i nemojmo sumnjati u Njegovu objavu.

% Suština Problema

\begin{titledpoem}
    \stanza{
        U riječima jasnim Bog se predstavlja, \\
        Ne kroz simbole što um ih sastavlja. \\
        Otac je Otac, Sin je zaista Sin, \\
        To nije metafora, već istine čin.
    }

    \stanza{
        Očigledno značenje riječi treba prihvatiti, \\
        Osim ako simbol Pismo želi otkriti. \\
        Očinstvo zemaljsko slika je nebeskog, \\
        Odraz stvarnosti, ne pojma teološkog.
    }

    \stanza{
        Gdje povući crtu između tajne i znanja? \\
        Gdje prestaje objava, počinju nagađanja. \\
        Šutnja je mudrost gdje Pismo ne zbori, \\
        Ali jasno je ono što nam Bog otvori.
    }

    \stanza{
        Ne traži više od onog što Pismo daje, \\
        U jednostavnosti vjere istina ostaje. \\
        Drži se objave, ne ljudskih teorija, \\
        U tome je mudrost, ne u magli misterija.
    }
\end{titledpoem}