\qrchapterstar{https://forgottenpillar.com/rsc/hr-fp-introduction}{Uvod}

Ova knjiga ima za ispuniti tri cilja. Prvi cilj je oživjeti stari stup vjere “\textit{ličnosti Boga}”. Drugi cilj je obnoviti povjerenje u spise Ellen White, i treći cilj je obnoviti originalni Adventistički identitet.

Prije 22. listopada 1844. godine, postojao je veliki broj Adventista koji su iščekivali Kristov povratak na nebeskim oblacima. Bio je to veliki pokret ljudi koji su čekali Kristov dolazak po drugi put. Kako je 22. listopad prošao bez Kristovog pojavljivanja na oblacima, usljedilo je veliko razočaranje, gdje je velika većina napustila pokret, prezirući ga, prezirući proročanstva, Bibliju, i naposljetku samoga Boga. Ostalo je malen broj poniznih vjernika, koji su nedvojbeno bili sigurni da je Bog vodio ovaj pokret. Znali su da ih je Bog obasjao svjetlom Istine i njihova srca su je željno primila. Ali u očima svijeta oni su bili samo demonstrirani fanatici i sanjari. Ovo veliko razočaranje može se usporediti s onim koje su doživjeli Isusovi učenici nakon što su vidjeli njihovog Gospodina položenog u grob. Bili su nedvojbeno sigurni da je Krist “\textit{bio prorok silan na djelima i riječima pred Bogom i svim narodom}”, ali kada je umro na križu, bili su gorko razočarani, jer su “\textit{se nadali da je on onaj koji ima otkupiti Izraela}.” Ipak u svom stanju očaja, u svom stanju vlastitog razočaranja, bili su spremni primiti silu da osvoje cijeli svijet Evanđeljem. Susreli su Krista i kasnije primili Njegovog Duha. Isto se dogodilo s adventističkim pionirima. Bila je to mala skupina ljudi, gorko razočaranih; tražili su Gospodina svim svojim srcem i primili su Ga u sili i u Istini. Istine koje je Bog otkrio tijekom ovog dragocjenog kriznog vremena čine temelj vjere Adventista Sedmog Dana. Ove istine su bile testirane od strane svih zavodljivih, varljivih teorija u svijetu, od strane onih koji su prezirali ovu malu skupinu, no te velike istine su prevladale. U vrijeme najveće potrebe, Isus je dao svoje svjedočanstvo podižući djevojku, najslabiju od slabih, da potvrdi sve Svoje istine. Ellen White nije bila izvor tih istina, već je bila podrška braći koja su tražila tu istinu u Pismu. Bog je koristio Ellen White kako bi potvrdio njihova proučavanja i uputio ih na Bibliju. Konačni rezultat bio je uspostavljanje temelja vjere baziranog na Bibliji, koji stoji čvrsto do samoga kraja svjetske povijesti.

Da li bi vas iznenadila tvrdnja da se temelji vjere Adventista Sedmog Dana, koji su postavljeni na početku našeg rada, razlikuju u velikoj mjeri od onoga što trenutno vjerujemo? Danas, nakon više od stoljeća i pol, sa divljenjem gledamo iskustva naših pionira; ali od tada je adventistička crkva bila podložna nekoliko novih pokreta. Od tada, Crkva je doživjela mnoge promjene, uključujući promjene u našem nauku. Neki tvrde da su te promjene dobre i progresivne; drugi pak tvrde da su destruktivne i varljive. Stavljanje fokusa na izvorni adventizam danas pokreće veliku kontroverzu. Osobno smo u ovoj kontroverzi već više od 6 godina i vidjeli smo da će ona samo rasti i jačati, često s tužnim posljedicama. Mnogi ljudi s obje strane ove kontroverze na jedan ili drugi način odbacuju Duh proroštva. Neki su potpuno napustili adventističku crkvu. Adventistički identitet je ili izgubljen ili drastično promijenjen od prvobitnog.

Trenutno smo svjedoci rešetanja crkve Adventista Sedmog Dana, koja prolazi kroz jedan val krize za drugim. Mnogi gube svoju vjeru i svoj adventistički identitet. Ali vjerujemo u rješenje koje je Gospodin, u svojoj milosti, već od prije dao. Rješenje se može pronaći u povijesti pokreta Adventista Sedmog Dana.

\egw{\textbf{U razmatranju naše prošlosti}, nakon što smo prošli svaki korak naprijed prema sadašnjem stanju, mogu reći: Hvala Bogu! Dok vidim što je Gospodin učinio, ispunjena sam divljenjem i s povjerenjem u Krista kao vođu. \textbf{Ne moramo se bojati za budućnost, \underline{osim ako zaboravimo} kako nas je Gospodin vodio i \underline{Njegovo učenje} u našoj prošlosti}.}[LS 196.2; 1915][https://egwwritings.org/read?panels=p41.1083]

Ne moramo se bojati! Ovo je veliko obećanje—iako uvjetno. Moramo se \textit{prisjetiti} kako nas je Gospodin vodio i \textit{Njegovog učenja u našoj prošloj povijesti}. Kada pogledamo čemu nas je Gospodin naučio u našoj prošloj povijesti, iznenađeni smo kad vidimo koliko su se stvari promijenile. Promjena je trajala kroz niz godina i mnogo kriza. Za prosuđivanje ovih promjena u doktrini, bilo da su one dobre i progresivne ili loše i destruktivne, prosuđivanje se mora temeljiti na prošlim iskustvima, kako je Gospodin jasno vodio svoju crkvu.

U ovom trenutku iznosimo jednu hrabru tvrdnju—koja bi vas trebala zadržati da pročitate ovu knjigu do kraja. Potaknuti savjetima Ellen White da preispitamo našu povijest, zaključili smo da smo zaboravili jedan ključni stup naše vjere, koji je bio glavni predmet Kelloggove kontroverze—\emcap{ličnost Boga}. Jedna od najvećih kriza koje je adventistička crkva ikada imala u vrijeme živog proroka bila je Kelloggova kriza. Iz te krize mnoge druge krize danas nalaze svoje korijene. U tom svjetlu, tema o \emcap{ličnosti Boga} ključna je u našem današnjem vremenu.

Sestra White je napisala Kelloggu da je \emcap{ličnost Boga} i \emcap{ličnost Krista} bio stup naše vjere jednake važnosti kao i učenje o svetištu:

\egw{Oni koji žele ukloniti \textbf{stare međaše} ne drže se čvrsto; \textbf{\underline{ne pamte} kako su primili i čuli}. Oni koji pokušavaju \textbf{\underline{uvesti} teorije koje bi uklonile \underline{stupove naše vjere} po pitanju svetišta ili \underline{po pitanju ličnosti Boga ili Krista}, rade kao slijepi ljudi}. Trude se unijeti nesigurnosti i uputiti Božji narod na pogrešan put, bez sidra.}[Ms62-1905.14][https://egwwritings.org/read?panels=p14070.10026020]

\emcap{Ličnost Boga} dobiva jako malo na pažnji danas kao tema, iako je ona presudan element u razumijevanju drugih adventističkih doktrina poput doktrine o Trojstvu, doktrine o svetištu, razumijevanje 1844-te i mnogih drugih doktrina koje su vezane za razumijevanje Nebeskih stvarnosti.

Doktrina o \emcap{ličnosti Boga} bila je stup naše vjere. Danas je ona skoro pa zaboravljena. Predložili bismo jedno moguće objašnjenje zašto. To je zbog razvoja Engleskog jezika. Što znači pojam “\textit{ličnost Boga}”? Opće razumijevanje Engleske riječi ‘\textit{personality}’ se izmijenilo tijekom godina. Danas se ‘\textit{personality}’ općenito razumije kao “\textit{karakteristični skup ponašanja, spoznaja i emocionalnih obrazaca}”\footnote{Wikipedia Contributors. “\textit{Personality.}” Wikipedia, Wikimedia Foundation, 19 Travnja. 2019, \href{https://en.wikipedia.org/wiki/Personality}{en.wikipedia.org/wiki/Personality}.}, no u devetnaestom stoljeću i početkom dvadesetog stoljeća značenje je bilo “\textit{kvaliteta ili stanje koja nekog čine osobom}”\footnote{\href{https://www.merriam-webster.com/dictionary/personality}{Merriam-Webster Dictionary}, - ‘personality’} \footnote{\href{https://babel.hathitrust.org/cgi/pt?id=mdp.39015050663213&view=1up&seq=780}{Hunter Robert, The American encyclopaedic dictionary}, ‘\textit{personality}’ - “\textit{the quality or state of being personal}”; Spomenuti rječnik bio je u posjedu Ellen White (vidi \href{https://repo.adventistdigitallibrary.org/PDFs/adl-22/adl-22251050.pdf?_ga=2.116010630.1065317374.1621993520-1506151612.1617862694&fbclid=IwAR3vwmp8jxtnpPEKv0KD9mCv8dJpmRGoyIXW0CkbQAjbU0h6YaBGqhgBzbk}{EGW Private and Office Libraries})}. Navedenu definiciju čitamo kao primarnu definiciju riječi ‘\textit{personality}’ iz uglednog Merriam-Webster rječnika\footnote{\href{https://www.merriam-webster.com/dictionary/personality\#word-history}{Riječnik Merriam-Webster} bilježi da je prvo zabilježeno značenje definicije “\textit{the quality or state of being a person}” zabilježeno u 15. stoljeću.}. Kada su sestra White i naši pioniri govorili i pisali o ličnosti Boga, referirali su se na kvalitetu ili stanje koja Boga čini osobom. Drugim riječima, bavili su se pitanjem “\textit{da li je Bog osoba}” i “\textit{što je to što Ga čini osobom}” ili “\textit{koja je to kvaliteta, ili stanje, koje Boga čini osobom}”? Pokušajte se sjetiti kada ste zadnji put imali Biblijsko proučavanje po pitanju “\textit{da li je Bog osoba}” i zašto? Razmislite, kako si možete potvrditi i dokazati iz Božje svete Riječi da je Bog osoba. Razmislite. To je važno pitanje. Na osnovi tog pitanja počiva naše viđenje Boga i naš odnos s njime. Pitanje ličnosti Boga je temeljno pitanje istinske duhovnosti. Istinska duhovnost je bazirana na osobnom odnosu sa Bogom. Ni jedan odnos, bilo kakve vrste, ne može se formirati sa nikim, ukoliko on ili ona nije osoba. Možda si nikada ni niste postavili to pitanje jer nikada niste osjetili potrebu preispitati stajalište da li je Bog osoba, i što je to (kvaliteta ili stanje) što Boga čini osobom. Ili, možda ste se susprezali od takvih pitanja jer smatrate da je to misterij kojeg Bog nije naumio obznaniti. Možda će vas iznenaditi činjenica da je Bog dao jasan i definiran odgovor u Njegovoj svetoj Riječi na pitanje “\textit{koja je kvaliteta ili stanje koja Boga čini osobom}”. Još više je zapanjujuće da je Bog adventističkim pionirima, uključujući Ellen White, dao jasno svjetlo vezano za ovu temu, te da su tu doktrinu smatrali stupom vjere, kao dijelom temelja vjere Adventista Sedmog Dana. Kada se doktrina o ličnosti Boga ispravno razumije u njezinom povijesnom kontekstu, tada stari poznati citati svijetle u novom svjetlu te razotkrivaju nove pokazatelje i dokaze koji razrješavaju naše trenutne krize i utvrđavaju nas u razumijevanju naše povijesti i našeg naslijeđa.

Suština Kellogove kriza bila je oko razumijevanja \emcap{ličnosti Boga}. Veoma je važno razmotriti Kelloggovu krizu kroz razumijevanje riječi ‘\textit{ličnost}’ (eng. personality) onog vremena, kao kvaliteta ili stanje koja Boga čini osobom. S tom definicijom na umu, Kelloggova kriza dolazi pod jasnije svjetlo, i razotkrivaju nam se novi pokazatelji važnosti Kelloggove kontroverzi, koja se presilkava na našu sadašnjicu. U svjetlu tih pokazatelja, vidimo kako nas je Gospodin vodio u prošlosti prošlosti, te se ne moramo bojati za budućnost. Poznavajući ta iskustva iz prošlosti, pomoći će nam kako ne bismo bili izrešetani od strane raznih zavodljivih ideja današnjice. Kada je sestra White upućivala doktora Kellogga na važnost doktrine o \emcap{ličnosti Boga}, ona je time i nama danas pokazala važnost iste—ona je sve nama kao narodu.

[Pišući dr. Kelloggu] \egw{Nije ti definitivno jasno u vezi \textbf{ličnosti Boga}, \textbf{što je \underline{sve} nama kao narodu}.}[Lt300-1903.7][https://egwwritings.org/read?panels=p14068.7705013]

Ovo proučavanje o \emcap{ličnosti Boga} će potaknuti mnoga nova i teška pitanja. Ne obećavamo odgovore na sva pitanja, i možda neki naši odgovori neće biti zadovoljavajući, ali se nadamo, i molimo se da će ova knjiga ispuniti tri prvobitna cilja zadanih na početku ovog uvoda. Kroz oživljavanje doktrine o ličnosti Boga, vjerujemo će vaše pouzdanje u dar Duha Proroštva ojačati, te će to rezultirati u dubljem ukorjenjavanju u adventističku poruku—u kojoj pronalazimo svoj identitet—čineći vas vjernijim Adventistom Sedmoga Dana. Ono najvažnije, želimo da postanete osvješteniji o Bogu kao vašem osobnom Boga. To će zasigurno ojačati i utemeljiti vaš odnos s Njime.

Odgovore o \emcap{ličnosti Boga} pronalazimo u Kelloggovoj krizi, gdje je sestra White dala najjasnije svjetlo glede \emcap{ličnosti Boga} i o \textit{temelju vjere Adventista Sedmog Dana}. U nastavku čitamo cijelo deseto poglavlje knjige pod nazivom “\textit{Svjedočanstva za Zajednicu Koja Sadrže Pisma Upute Liječnicima i Propovjednicima Adventistima Sedmog Dana}”. Navedeno poglavlje “\textit{Temelji naše vjere}” daje duboki uvid u povijest Kelloggove krize. Ono daje povijesni pregled istina koje je Bog dao kao temelj naše vjere i u tim istinama nalazimo svoj identitet adventista sedmog dana—čuvajući zapovjedi Božje i imajući vjeru Isusovu.
