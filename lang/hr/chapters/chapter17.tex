\chapter{Odgovor na Kellogove trinitarijanske sentimente}

Kada pogledamo Kelloggovu krizu kroz perspektivu \emcap{ličnosti Boga} i \emcap{Fundamentalnih Principa}, citati Sestre White neizbježno sjaje u novom svjetlu. U tom svjetlu vidimo sukob između istine koju smo primili na početku, o \emcap{ličnosti Boga}, i doktrine o Trojstvu. Da bi se izbjegle nesuglasice, u interesu obrane doktrine o Trojstvu, teolozi uvijek pretjerano naglašavaju panteističku stranu problema.

Želimo izazvati ovu tendenciju prekomjernog naglašavanja panteističke strane Kelloggove kontroverze. Sestra White općenito je pristupala istini proaktivno; njezin pristup zabludi bio je uzdizanje istine. Zato je toliko pisala o \emcap{ličnosti Boga}. U većini njezinih citata na temu ličnosti Boga, vidimo kako ona raspršuje zabludu trojstva, a ne panteizma. Čitamo jedan takav primjer gdje ona uspostavlja istinu o \emcap{ličnosti Boga} pozivajući se na sedamnaesto poglavlje Ivana.

\egw{\textbf{Ličnost Oca i Sina, kao i jedinstvo koje postoji između Njih, je prikazano u sedamnaestom poglavlju Ivana, u Kristovoj molitvi za Svoje učenike:}}[MH 421.7; 1905][https://egwwritings.org/read?panels=p135.2173]

Postoji pregršt citata u kojima Sestra White citira Ivan 17 u vezi s Kelloggovom krizom. Oni koji tvrde da je Kellogova kriza bila isključivo vezana za panteizam, trebali bi se pitati kakve veze ima Ivan 17 s problemom Boga u prirodi. I nije samo Ivan 17, već i poglavlja 13-16. U svom pismu Kelloggu, napisala je:

\egw{\textbf{\underline{…proučavajte trinaesto, četrnaesto, petnaesto, šesnaesto i sedamnaesto poglavlje Ivana}. Riječi ovih poglavlja same se objašnjavaju. ‘A ovo je vječni život,’ Krist je izjavio, ‘da poznaju \underline{tebe, jedinoga istinskog Boga}, i onoga koga si poslao — Isusa Krista.’ \underline{U ovim riječima jasno se govori o ličnosti Boga i Njegova Sina.} \underline{Ličnost jednoga ne ukida potrebu za ličnošću drugog}.}}[Lt232-1903.48, 1903][https://egwwritings.org/read?panels=p10197.57]

U gore spomenutim poglavljima Ivana, Ivan nije spominjao ništa što se odnosi na Boga u prirodi. Sadržaj tih poglavlja govori o tome tko je jedini istinski Bog, kako su Otac i Sin jedno, njihov pravi odnos, i kako Isus može biti svugdje prisutan nakon što će se fizički uznijeti Ocu.

\egw{Isus je rekao Židovima: ‘Otac moj sve do sada radi, i ja radim.... Sin sȃm od sebe ne može činiti ništa, osim ono što vidi da čini Otac. Jer što god On čini, to isto čini i Sin. Ta Otac voli Sina i pokazuje mu sve što sȃm čini.’ Ivan 5:17-20.}[8T 268.4, 1904][https://egwwritings.org/read?panels=p112.1557]

\egwnogap{\textbf{I ovdje se ponovo ukazuje na \underline{ličnost Oca i Sina}, i na jedinstvo koje postoji među Njima}.}[8T 269.1; 1904][https://egwwritings.org/read?panels=p112.1560]

\egwnogap{\textbf{To jedinstvo je također izraženo u \underline{sedamnaestom poglavlju Ivana}, u Kristovoj molitvi za učenike:}}[8T 269.2; 1904][https://egwwritings.org/read?panels=p112.1561]

\egwnogap{‘Ne molim pak samo za njih, nego i za one koji me uzvjeruju njihove riječi radi: da svi budu jedno, \textbf{kao Ti, Oče, što si u meni i ja u Tebi; da i oni u nama jedno budu}: da i svijet vjeruje da si me Ti poslao. I \textbf{slavu koju si mi dao} ja dadoh njima; \textbf{da budu jedno kao mi što smo jedno. Ja u njima i Ti u meni; da budu sasvijem u jedno}; da pozna svijet da si me Ti poslao i da si imao ljubav k njima kao i k meni što si ljubav imao.’ Ivan 17:20-23.}[8T 269.3; 1904][https://egwwritings.org/read?panels=p112.1562]

\egwnogap{Čudesne li izjave! \textbf{Jedinstvo koje postoji između Krista i Njegovih učenika \underline{ne uništava ni Njegovu, ni njihovu ličnost}. Oni su jedno u cilju, u umu, u karakteru, ali \underline{ne i u osobi}. Na takav način su jedno i Bog i Krist}.}[8T 269.4; 1904][https://egwwritings.org/read?panels=p112.1563]

\egwnogap{\textbf{Odnos između Oca i Sina, kao i ličnost obojice, jasno su objašnjeni i u sljedećem tekstu:}}[8T 269.5; 1904][https://egwwritings.org/read?panels=p112.1564]

\egwnogap{Ovako veli \textbf{GOSPOD Nad Vojskama} govoreći,} \\
\egw{Evo \textbf{čovjeka} kojemu je ime \textbf{Izdanak};} \\
\egw{jer on će niknuti iz svojega mjesta} \\
\egw{\textbf{I sagraditi svetište GOSPODNJE;...}} \\
\egw{\textbf{I on će pronijeti slavu,}} \\
\egw{\textbf{I sjedit će i vladati na prijestolju svojemu;}} \\
\egw{\textbf{Bit će on i svećenik na prijestolju svojemu;}} \\
\egw{\textbf{I \underline{naum mira bit će između to dvoje}}.’}[8T 269.6; 1904][https://egwwritings.org/read?panels=p112.1565]

Gore spomenuta poglavlja Evanđelja po Ivanu bave se \emcap{ličnošću Boga}, koja je bila izražena u prve dvije točke \emcap{Fundamentalnih Principa}. Protiv koje zablude se borila Sestra White kada je citirala stihove o tome kako je Otac jedini istiniti Bog, i kako Otac i Sin nisu jedno u osobi? Panteizam? Svakako ne; već najvjerojatnije trinitarijanske sentimente, ili vjerovanje u Boga koji je jedan-u-tri, ili troje-u-jednom.

Brat J. N. Loughborough, jedan od prvih braće koji je pisao o \emcap{ličnosti Boga}, napisao je sljedeći komentar na Ivanovo poglavlje 17:

\others{\textbf{\underline{Samo sedamnaesto poglavlje Ivana je dostatno da pobije doktrinu o Trojstvu}}. \textbf{...Pročitajte sedamnaesto poglavlje Ivana, i pogledajte ukoliko li to kompletno uznemirava doktrinu o Trojstvu}.}[John N. Loughborough, The Adventist Review, and Sabbath Herald, November 5, 1861, p. 184.10][https://egwwritings.org/read?panels=p1685.6615]

Proaktivno pisanje Sestre White u prilog istini o \emcap{ličnosti Boga} i Njegove prisutnosti bilo je jednako kao i kod drugih adventističkih pionira. Ako su naši pioniri opovrgavali doktrinu o Trojstvu uzdižući istinu o \emcap{ličnosti Boga} i Božjoj prisutnosti, zašto mislimo da Ellen White nije činila isto, tada kada je teološka strana pitanja Trojstva bila postavljena? Time ne negiramo panteističku stranu Kellogove kontroverze, ali prekomjerno naglašavanje panteizma ne donosi točan opis pravog problema. Ispravno razumijevanje Kellogove kontroverze može se postići samo primarnim fokusiranjem na istinu koju je Sestra White uzdizala, namjesto zanimanjem za zablude, bilo da je riječ o panteizmu ili o trojstvu. Ta istina koju je Sestra White uzdizala bila je istina o \emcap{ličnosti Boga} i gdje je Njegova prisutnost. To je izraženo u prvoj točki \emcap{Fundamentalnih Principa}, što je bio javan i službeni sinopsis vjerovanja Adventista Sedmog Dana u vrijeme Ellen White; istina koju smo, kao crkva, \egwinline{primili i čuli i zagovarali}[Ms124-1905.12; 1905][https://egwwritings.org/read?panels=p9099.18] na početku.

\egw{\textbf{Sve vas preklinjem da budete jasni i čvrsti u vezi određenih istina koje smo čuli, primili i zagovarali. Izjave Božje Riječi su jasne. Posadite svoje noge čvrsto na \underline{platformi vječne istine}. \underline{Odbacite svaku fazu pogreške}, čak i ako je \underline{pokrivena sličnošću istini, koja negira ličnost Boga i Krista}}.}[Ms124-1905.12; 1905][https://egwwritings.org/read?panels=p9099.18]

Upozorenje iz prethodnih citata nije se smanjilo tijekom vremena. Danas je čak i relevantnije. Trebali bismo \egwinline{odbaciti svaku fazu pogreške, čak i ako je pokrivena sličnošću istini, koja negira ličnost Boga i Krista}. U nastavku želimo istaknuti jednu specifičnu fazu pogreške koja je pokrivena sličnošću istini, a koja negira ličnost Boga i Krista—tri žive osobe \textit{jednoga} Boga, nasuprot \egwinline{tri žive osobe nebeskog trija.}[Ms21-1906.11; 1906][https://egwwritings.org/read?panels=p9754.18]
