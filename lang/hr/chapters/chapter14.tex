\chapter{Adventistički pioniri i doktrina o Trojstvu}

Sestra White je napisala da rani adventistički pioniri \egwinline{trebaju iznijeti svoje svjedočanstvo o tome što čini istinu za ovo vrijeme}[Lt329-1905.18; 1905][https://egwwritings.org/?ref=en\_Lt329-1905.18&para=8455.24] jer \egwinline{su naučili izbjeći zablude i opasnosti, i nisu li onda sposobni dati mudre savjete}[7T 287.3; 1902][https://egwwritings.org/?ref=en\_7T.287.3&para=117.1637]? U njihovim spisima vidimo njihove jednoglasne stavove o \emcap{ličnosti Boga}, i da su izbjegli trinitarijansku zabludu. Ima mnogo za pisati o ovoj temi jer su adventistički pioniri ostavili mnogo materijala koji se izravno ili neizravno bavi doktrinom o Trojstvu. Ali pogledat ćemo neka svjedočanstva od Jamesa Whitea i brata Loughborougha jer smo pročitali neke od njihovih članaka o \emcap{ličnosti Boga}. Također, usporedit ćemo njihovo svjedočanstvo s Duhom Proroštva kao što smo to činili do sada.

James White je u Review and Heraldu naveo \others{neke od \textbf{popularnih bajki} toga doba}”, rekavši: “\others{Ovdje bismo mogli spomenuti \textbf{Trojstvo, koje \underline{poništava ličnost Boga i Njegovog Sina Isusa Krista,}} i škropljenje ili polijevanje umjesto da budemo ‘pokopani s Kristom u krštenju,’ ‘zasađeni u sličnosti njegove smrti:’ ali prelazimo od ovih \textbf{bajki} da bismo primijetili jednu koju gotovo svi koji ispovijedaju kršćanstvo, i katolici i protestanti, smatraju svetom. To je promjena Šabata četvrte zapovijedi sa sedmog na prvi dan tjedna.}[James S. White, Review \& Herald, December 11, 1855, p. 85.15][http://documents.adventistarchives.org/Periodicals/RH/RH18551211-V07-11.pdf].

Što James White misli kada kaže da Trojstvo \others{poništava ličnost Boga i Njegovog Sina Isusa Krista}? U Day Staru je napisao:

\others{…određena skupina koja \textbf{poriče jedinog Gospodina Boga i našeg Gospodina Isusa Krista}. Ova skupina ne može biti nitko drugi nego oni koji \textbf{produhovljuju postojanje Oca i Sina}, \textbf{kao \underline{dvije zasebne}, \underline{doslovne}, \underline{opipljive osobe}}, također i doslovni Sveti grad i Davidovo prijestolje… Način na koji su spiritualisti \textbf{uklonili ili zanijekali jedinog Gospodina Boga i našeg Gospodina Isusa Krista je prvo korištenjem \underline{starog nebiblijskog trinitarijanskog vjerovanja}}, odnosno, da je Isus Krist vječni Bog, iako nemaju ni jedan odlomak koji to podupire, dok mi imamo obilje jasnih biblijskih svjedočanstava \textbf{da je On Sin vječnoga Boga}.}[James White, Day Star, 24. siječnja 1846.][https://m.egwwritings.org/en/book/741.25\#27]

Poništavanje ličnosti Boga i Njegovog Sina postiže se njihovim poricanjem kao dvije zasebne, doslovne i opipljive osobe. Nauk o ličnosti Boga uči da Otac ima doslovno, \textit{opipljivo} biće.

U članku Adventist Review and Sabbath Herald od 4. travnja 1854., James White je naveo 10 točaka o \textit{katoličkim razlozima za držanje nedjelje}”, gdje je citirao da je nedjelja “\others{dan koji su apostoli posvetili \textbf{u čast presvetog Trojstva}}[The Advent Review, and Sabbath Herald, vol. 5 April 4, 1854, p. 86][https://egwwritings.org/?ref=en\_ARSH.April.4.1854.p.83.9&para=1643.2867]. Ovdje također vidimo sklad između J. B. Frisbiea i Jamesa Whitea u njihovom stavu da je Subota posvećena biblijskom Bogu izraženom u prvoj točki \emcap{Fundamentalnih Principa}, a nedjelja je posvećena trojedinom bogu. Glavni problem s doktrinom o Trojstvu je da “\others{poništava ličnost Boga i Njegovog Sina Isusa Krista}”. U Life Incidents napisao je više o tome zašto je to tako.

\others{\textbf{Isus se molio da njegovi učenici budu jedno kao što je on bio \underline{jedno sa svojim Ocem}}. \textbf{Ova molitva nije razmatrala jednog učenika s dvanaest glava, već dvanaest učenika, učinjenih jednima u cilju i naporu u djelu njihovog Učitelja}. \textbf{\underline{Ni Otac ni Sin nisu dijelovi ‘tro-jednog Boga’.}}\footnote{Isti citat se nalazi u knjizi Jamesa Whitea “\textit{Zakon i Evanđelje}” s jednom razlikom. On navodi, “\textit{Ni Otac ni Sin nisu dijelovi \underline{jednog bića}}”; u “\textit{Life Incidents}”, napisao je “dijelovi ‘\underline{tro-jednog Boga}’“. Vidi \href{https://egwwritings.org/?ref=en_LAGO.1.2&para=1492.10}{James S. White, The Law and the Gospel p. 1.2}.} \textbf{\underline{Oni su dva različita bića}}, \textbf{ipak jedno u naumu i ostvarenju otkupljenja}. Otkupljeni, od prvog koji sudjeluje u velikom otkupljenju, do posljednjeg, svi pripisuju čast, slavu i hvalu svog spasenja \textbf{i Bogu i Janjetu}.}[James S. White, Life Incidents, p.343.2][https://egwwritings.org/?ref=en\_LIFIN.343.2&para=1462.1743]

Sestra White je napisala slično o Kristovoj molitvi:

\egw{Teret te molitve bio je da Njegovi učenici budu \textbf{jedno kao što je On bio jedno s Ocem}; jedinstvo tako blisko da, \textbf{iako \underline{dva različita bića}}, postojalo je \textbf{savršeno jedinstvo duha, svrhe i djelovanja}. Um Oca bio je um Sina.}[Lt1-1882.1; 1882][https://egwwritings.org/?ref=en\_Lt1-1882.1&para=4120.5]

\egw{\textbf{Jedinstvo koje postoji između Krista i Njegovih učenika \underline{ne uništava ličnost nijednog}}. Oni su jedno u svrsi, u umu, u karakteru, \textbf{ali \underline{ne u osobi}}. \textbf{Tako su i Bog i Krist jedno}.}[MH, 421 422; 1905][https://egwwritings.org/?ref=en\_MH.422.1&para=135.2177]

Otac i Sin ne čine jednu osobu niti biće. Otac i Sin su jedno, baš kao što su Krist i Njegovi učenici jedno—jedno u duhu, svrsi, umu i karakteru.

Mnogi adventistički trinitarijanski teolozi optužuju Jamesa Whitea i druge rane pionire za arijanizam ili semi-arijanizam, tvrdeći da su učinili Krista inferiornim Ocu. To nije istina. Pročitajmo svjedočanstvo Jamesa Whitea o ovom pitanju.

\others{Pavao potvrđuje za \textbf{Sina Božjega da je bio u obličju Božjem}, i da je \textbf{\underline{bio jednak s Bogom}}. ‘\textbf{Koji budući u obličju Božjem nije smatrao otimanjem biti \underline{jednak s Bogom}}.’ Fil. 2:6. Razlog zašto nije otimanje da Sin \textbf{bude jednak s Ocem je činjenica da jest jednak}. Ako Sin nije jednak s Ocem, onda je otimanje da se izjednačava s Ocem.}

\othersnogap{\textbf{\underline{Neobjašnjivo trojstvo koje čini božanstvo tri u jednom i jedno u tri, dovoljno je loše}}; \textbf{ali onaj ultra unitarijanizam koji čini Krista inferiornim Ocu je još gori}. Je li Bog rekao inferiornijem, Načinimo čovjeka na svoju sliku?’}[James S. White, The Advent Review and Sabbath Herald, November 29, 1877, p. 171][https://documents.adventistarchives.org/Periodicals/RH/RH18771129-V50-22.pdf]

Problem adventističkih trinitarijanskih skolara leži u tome što oni sami ne mogu potpuno objasniti Kristovo božanstvo osim kroz trinitarijansku paradigmu. Adventistički pioniri su vjerovali u Kristovo potpuno božanstvo ali su odbacili Trojstvo jer ono uništava \emcap{ličnost Boga}. \others{Neobjašnjivo trojstvo koje čini božanstvo tri u jednom i jedno u tri, \textbf{dovoljno je loše}}. Ispod je još jedna izjava Jamesa Whitea gdje je usporedio vjerovanje Adventista sedmog dana s vjerovanjem Baptista sedmog dana. Adventisti sedmog dana nisu vjerovali u Trojstvo za razliku od Baptista sedmog dana. James White je spomenuo da, što se tiče Kristovog božanstva, Adventisti sedmog dana drže tako blisko s Baptistima sedmog dana da ne predviđaju nikakav problem po pitanju Kristovog potpunog božanstva.

\others{\textbf{Glavna razlika između dva tijela je pitanje besmrtnosti}. \textbf{Adventisti sedmog dana drže \underline{božanstvo Krista tako blisko s trinitarijancima}, da ne predviđamo nikakav problem ovdje}. I kako naša braća Baptisti sedmog dana bolje razumiju praktičnu primjenu teme Darova Duha našem narodu i našem radu, oni pokazuju manju zabrinutost za nas po tom pitanju.}[James S. White, The Advent Review and Sabbath Herald, October 12, 1876, p. 116][https://documents.adventistarchives.org/Periodicals/RH/RH18761012-V48-15.pdf]

Ovi pokazatelji bi trebali postaviti pitanja svakom adventističkom trinitarijanskom skolaru. Kako je moguće da su adventistički pioniri u potpuno Kristovo božanstvo, baš kao i trinitarijanci, a ipak su odbacili doktrinu o Trojstvu? Na koji način je Krist bio u potpunosti Božanske prirode, ako nije bio dio amalgamiranog tri-u-jednom Boga? Odgovor je jednostavan i potpuno biblijski. Krist je u potpuno Božanske prirode, baš kao i Njegov Otac, jer je bio rođen na savršenu sliku Očeve osobe; tako je naslijedio potpunu božansku prirodu od svog Oca.

\egw{Učinjena je potpuna žrtva; jer ‘Bog je tako ljubio svijet da je dao svog jedinorođenog Sina,’—\textbf{ne sina po stvaranju}, kao što su anđeli, niti sina po posvojenju, kao što je oprošteni grešnik, već \textbf{Sina \underline{rođenog} na savršenu sliku Očeve osobe}, i u svoj svjetlosti njegovog veličanstva i slave, \textbf{jednakog s Bogom} u autoritetu, dostojanstvu, i \textbf{božanskom savršenstvu}. \textbf{U njemu je prebivala sva punina božanstva tjelesno}.}[ST May 30, 1895, par. 3; 1895][https://egwwritings.org/?ref=en\_ST.May.30.1895.par.3&para=820.12891]

Kristovo potpuno božanstvo nije zasnovano na amalgamiranoj \emcap{ličnosti Boga}, već na Njegovom Sinovstvu s Ocem. Biblija nikada ne naziva Krista terminom “\textit{jedan Bog}”—samo se Otac naziva terminom “\textit{jedan Bog}”\footnote{Ivan 17:3; 1. Korinćanima 8:6; 1. Timoteju 2:5; Efežanima 4:6} \footnote{Za detaljno proučavanje o potpunom Kristovom Božanstvu, vidi nastavak ove knjige—\textit{Rediscovering the Pillar}”}. Isus, Sin Božji, je u potpunosti Božanstvo ali se ne naziva \others{\textbf{jednim Bogom}, \textbf{osobnim, duhovnim bićem}} u prvoj točki \emcap{Fundamentalnih Principa}.

\egw{Gospodin Isus Krist, jedinorođeni Sin Očev, \textbf{uistinu je Bog u beskonačnosti, \underline{ali ne u ličnosti}}.}

Brat J. N. Loughborough bio je zamoljen da odgovori na pitanje, \others{Koji je ozbiljan prigovor na nauk o Trojstvu?}[Pitanje je postavio brat W. W. Giles i poslao ga je Jamesu S. Whiteu, koji je proslijedio pitanje bratu Johnu N. Loughboroughu.]. Dok čitamo njegov odgovor, pokušajmo razumjeti neke od razloga zašto se rani pioniri nisu držali ovog nauka.

\others{Postoji mnogo prigovora koje bismo mogli iznijeti, ali zbog našeg ograničenog prostora svest ćemo ih na sljedeća tri: \textbf{1. Protivi se zdravom razumu. 2. Protivi se Pismu. 3. Njegovo porijeklo je pogansko i bajkovito.}}

\othersnogap{O ovim stajalištima ćemo ukratko govoriti po redu. I 1. \textbf{Nije u skladu sa zdravim razumom govoriti o tome da su tri jedno, a jedno tri}. \textbf{Ili kako neki to izražavaju, nazivajući Boga ‘trojedinim Bogom’ ili ‘tri-jedan-Bog.’} \textbf{Ako su Otac, Sin i Sveti Duh svaki Bog, to bi bila tri Boga; jer tri puta jedan nije jedan, nego tri}. \textbf{\underline{Postoji smisao u kojem su oni jedno, ali ne jedna osoba, kako tvrde trinitarijanci}}}.

\othersnogap{2. \textbf{To je protivno Pismu}. \textbf{Gotovo svaki dio Novog zavjeta koji govori o Ocu i Sinu, predstavlja ih kao dvije različite osobe}. \textbf{\underline{Sedamnaesto poglavlje Ivana je samo po sebi dovoljno da pobije doktrinu o Trojstvu}}. \textbf{Više od četrdeset puta u tom jednom poglavlju Krist govori o svom Ocu kao osobi različitoj od sebe}. Njegov Otac je bio na nebu, a on na zemlji. Otac ga je poslao. Dao mu je one koji su vjerovali. On je tada trebao ići k Ocu. \textbf{I u ovom svjedočanstvu nam pokazuje u čemu se sastoji jedinstvo Oca i Sina}. \textbf{\underline{To je isto kao jedinstvo članova Kristove crkve}}. ‘\textbf{Da \underline{oni} svi budu jedno; \underline{kao} što si ti, Oče, u meni, i ja u tebi, \underline{da i oni} budu jedno u nama}; da svijet vjeruje da si me ti poslao. I \textbf{slavu koju si mi dao ja sam dao njima}; da \textbf{budu jedno}, \textbf{kao što smo mi jedno.}’ \textbf{Jednog srca i jednog uma}. \textbf{Jedne svrhe} u svem planu osmišljenom za čovjekovo spasenje. \textbf{\underline{Pročitajte sedamnaesto poglavlje Ivana i vidite ne pobija li ono potpuno doktrinu o Trojstvu}}.}

\othersnogap{\textbf{Da bismo vjerovali tu doktrinu, kad čitamo Pismo moramo vjerovati da je Bog poslao samog sebe na svijet, umro da pomiri svijet sa samim sobom, uskrsnuo samog sebe od mrtvih, uzašao k samom sebi na nebo, moli pred samim sobom na nebu da pomiri svijet sa samim sobom, i jedini je posrednik između čovjeka i samog sebe}. Neće biti dovoljno zamijeniti ljudsku prirodu Krista (prema trinitarijancima) kao Posrednika; jer Clarke kaže: ‘Ljudska krv ne može više umiriti Boga nego svinjska krv.’ Komentar na 2. Samuelovu 21:10. \textbf{Moramo također vjerovati da se u vrtu Bog molio samom sebi, ako je moguće, da ga mimoiđe čaša od samog sebe, i tisuću drugih \underline{takvih apsurdnosti}}.}

\others{\textbf{Pažljivo pročitajte sljedeće tekstove, uspoređujući ih s idejom da je Krist Svemogući, Sveprisutni, Vrhovni i jedini samopostojeći Bog: Ivan 14:28; 17:3; 3:16; 5:19, 26; 11:15; 20:19; 8:50; 6:38; Marko 13:32; Luka 6:12; 22:69; 24:29; Matej 3:17; 27:46; Galaćanima 3:20; 1 Ivanova 2:1; Otkrivenje 5:7; Djela 17:31. Također pogledajte Matej 11:25, 27; Luka 1:32; 22:42; Ivan 3:35, 36; 5:19, 21, 22, 23, 25, 26; 6:40; 8:35, 36; 14:13; 1 Korinćanima 15:28, itd}.}

\othersnogap{\textbf{Riječ Trojstvo se nigdje ne pojavljuje u Pismu}. \textbf{Glavni tekst za koji se pretpostavlja da ga poučava je 1 Ivanova 1:7\footnote{J. N. Loughborough je napravio pogrešku u originalnom dokumentu, htio je ukazati na 1 Ivanovu 5:7}, koji je interpolacija}. Clarke kaže, ‘\textbf{Od sto trinaest rukopisa, tekst nedostaje u sto dvanaest. Ne pojavljuje se ni u jednom rukopisu prije desetog stoljeća. A prvo mjesto gdje se tekst pojavljuje na grčkom je u grčkom prijevodu akata Lateranskog koncila, održanog 1215. godine}’. - Komentar na Ivana 1, i primjedbe na kraju poglavlja.}

\othersnogap{3. \textbf{Njegovo porijeklo je pogansko i basnoslovan}. Umjesto da nas upućuju na Pismo za dokaz trojstva, upućuju nas na trozubac Perzijanaca, s tvrdnjom da su time htjeli poučavati ideju trojstva, i ako su oni imali doktrinu o trojstvu, morali su je primiti predajom od Božjeg naroda. \textbf{Ali sve je to pretpostavljeno, jer je sigurno da židovska crkva nije držala takvu doktrinu. Gospodin Summerbell kaže: ‘Moj prijatelj koji je bio prisutan u jednoj njujorškoj sinagogi, pitao je rabina za objašnjenje \underline{riječi ‘elohim’}. Jedan trinitarski svećenik koji je stajao pored, odgovorio je: ‘Pa, to se \underline{odnosi na tri osobe u Trojstvu},’ kada je jedan Židov istupio naprijed i rekao da ne smije više spominjati tu riječ, ili će ga morati prisiliti da napusti kuću; \underline{jer nije bilo dopušteno spominjati ime bilo kojeg stranog boga u sinagogi}.’}\footnote{Rasprava između Summerbella i Flooda o Trojstvu, str. 38.} Milman kaže da je ideja o Trozupcu bajkovita. (Hist. Christianity, str.34.)}

\others{\textbf{Ova doktrina o trojstvu je uvedena u crkvu otprilike u isto vrijeme kad i štovanje ikona, i držanje dana sunca, i nije ništa drugo nego preoblikovana perzijska doktrina}. \textbf{Trebalo je oko tristo godina od njenog uvođenja da se doktrina dovede do onoga što je danas. Započela je oko 325. godine, a nije bila dovršena do 681.} Vidi Milman's Gibbon's Rome, vol. iv, str.422. Usvojena je u Španjolskoj 589., u Engleskoj 596., u Africi 534. - Gib. vol. iv, str.114,345; Milner, vol. i, str.519.}[John N. Loughborough, The Adventist Review, and Sabbath Herald, November 5, 1861, p. 184][https://egwwritings.org/?ref=en\_ARSH.November.5.1861.p.184.1&para=1685.6615]

Brat Loughborough bio je sin metodističkog propovjednika i odgojen je s vjerovanjem u doktrinu o Trojstvu. Osim razloga koje je spomenuo, on ne pristaje uz ovu doktrinu jer nije u skladu s istinom o \emcap{ličnosti Boga}. Sedamnaesto poglavlje Ivana je u skladu s istinom o \emcap{ličnosti Boga} koju su učili i prakticirali Adventisti sedmog dana; doktrina o Trojstvu to nije.

J. N. Andrews je rekao: \others{\textbf{Doktrina o Trojstvu koja je uspostavljena u crkvi na koncilu u Niceji, 325. godine}. \textbf{Ova doktrina \underline{uništava ličnost Boga i njegovog Sina Isusa Krista našeg Gospoda}}...}[John. N. Andrews, The Advent Review and Sabbath Herald, March 6, 1855, p. 185][http://documents.adventistarchives.org/Periodicals/RH/RH18550306-V06-24.pdf]

U kontekstu trinitarijanskog razumijevanja \emcap{ličnosti Boga}, sigurno je reći da je \emcap{ličnost Boga}, ili kvaliteta ili stanje koja Boga čine osobom, u bilo kojem razumijevanju doktrine o Trojstvu misterija. Problem je u tome što ne postoji jasan pogled o tome tko je taj \textit{jedan Bog} koji je osoba? Temeljna tvrdnja je da je Bog Jedan ali Tri, ili Jedan u Tri; da, Bog je osoba, i On je jedan, ali istovremeno je tri osobe. Ovaj pogled nikada ne može imati jasnu percepciju \emcap{ličnosti Boga}. Također, on će negirati najjasnije svjedočanstvo Pisma da je jedan Bog Otac, i da je Krist uistinu Njegov jedinorođeni Sin. Većina trinitarijanske braće bi se složila da je Krist stvarno i određeno biće, ali ako bi trinitarijanac prihvatio Oca kao stvarno i određeno Biće, morao bi također prihvatiti Svetog Duha kao stvarno i određeno biće, time negirajući Svetog Duha kao \textit{duha}, sredstvo kojim su Otac i Sin sveprisutni. Obrnuto, ako bi trinitarijanac prihvatio Svetog Duha kao doslovnog duha, koji nema tijela ni oblika, tada bi negirao Oca kao stvarno, određeno biće. U razgovoru o kvaliteti ili stanju koja Boga čine osobom, nikada nema jasnog pogleda na stvar kod zagovornika doktrine o Trojstvu; to je subterfuge. \textit{‘Subterfuges’} je riječ koju je Sestra White koristila da opiše obmanu pomoću lukavstva ili strategije kako bi se prikrila, pobjegla ili izbjegla\footnote{\href{https://www.merriam-webster.com/dictionary/subterfuges}{The Merriam-Webster, ‘subterfuges’} - “\textit{obmana pomoću lukavstva ili strategije kako bi se prikrilo, pobjeglo ili izbjeglo}”} istina; drugim riječima, nešto što ne možeš uhvatiti ni za glavu ni za rep. Ovo je glavni razlog zašto se Sestra White nije upuštala u raspravu o Trojstvu koja će se pojaviti u Crkvi Adventista sedmog dana.

\egw{Upozorena sam da ne ulazim u kontroverzu \textbf{u vezi s pitanjem} koje će \textbf{\underline{se pojaviti}} oko \textbf{ovih stvari, jer bi kontroverza \underline{mogla navesti ljude da pribjegnu subterfugama, i njihovi umovi bi bili odvedeni od istine Božje Riječi prema pretpostavkama i nagađanjima}}. \textbf{Što se više raspravlja o maštovitim teorijama, \underline{to će ljudi manje znati o Bogu i istini koja posvećuje dušu}}.}[Lt232-1903.41; 1903][https://egwwritings.org/?ref=en\_Lt232-1903.41&para=10197.50]

Kada čitamo djela pionira Adventista sedmog dana o \emcap{ličnosti Boga}, vidimo da nisu upali u zamku Trojstva. Njihovi netrinitarijanski pogledi na Boga nisu bili zbog neznanja, već znanja istine o \emcap{ličnosti Boga}. Bili su oštrog i plemenitog intelekta, razumijevajući tanku liniju između istine i zablude. Njihovo razumijevanje \emcap{ličnosti Boga} je uravnoteženo i čvrsto, snažno poduprto jednostavnim i jasnim “\textit{tako govori Gospod}”.

Mnogi adventisti danas prihvaćaju doktrinu o Trojstvu jer je Ellen White navodno prihvatila i promovirala. To je daleko od istine i takav zaključak se temelji na nedostatku poznavanja Duha Proroštva. Ako je itko bio upoznat s vjerovanjima sestre White, to je bio njen suprug James White. Evo što on kaže o spisima svoje supruge:

\others{\textbf{Pozivamo sve da usporede svjedočanstva Svetog Duha kroz gđu W. s Božjom riječju}. \textbf{I u tome vas ne pozivamo da ih uspoređujete \underline{sa svojim vjerovanjem}}. To je sasvim druga stvar. \textbf{\underline{Trinitarijanac ih može usporediti sa svojim vjerovanjem i, zato što se ne slažu s njim, osuditi ih}}. Onaj koji svetkuje nedjelju, ili čovjek koji drži vječne muke važnom istinom, i propovjednik koji škropi dojenčad, mogu svaki osuditi svjedočanstva gđe W. jer se ne slažu s njihovim posebnim pogledima. I stotinu drugih, od kojih svaki ima različite poglede, mogu doći do istog zaključka. \textbf{Ali njihova izvornost nikada se ne može ispitati na taj način}.}[James S. White, The Advent Review, and Herald of the Sabbath, June 13, 1871][https://documents.adventistarchives.org/Periodicals/RH/RH18710613-V37-26.pdf]

James White je bio najbliži suradnik Ellen White, osoba koja je bila jedno s njom u Božjem podizanju Crkve Adventista Sedmoga Dana. Imamo jasno i izravno svjedočanstvo od njega da spisi Ellen White nisu trinitarijanski. Danas trinitarijanski teolozi stavljaju lažnu priču da je Ellen White rasla u svom razumijevanju doktrine o Trojstvu i na kraju je prihvatila i propovijedala. Ali vidimo da Ellen White nije promijenila svoje stajalište o \emcap{ličnosti Boga} niti se držala doktrine o Trojstvu. Bila je nedvosmislena u svojoj tvrdnji da to nikada nije učinila. Kada je došla Kelloggova kriza oko \emcap{ličnosti Boga}, ostala je čvrsta u svom stavu, baš kao i svi rani adventistički pioniri—a njeno postupanje s dr. Kelloggom to dokazuje. Istina je, doktrinu o Trojstvu \textit{ne može biti prihvaćena od onih koji su odani vjeri i principima koji su izdržali svu opoziciju sotonskih utjecaja}.\footnote{\egw{Teorije krpanja ne mogu biti prihvaćene od onih koji su lojalni vjeri i principima koji su odoljeli svim protivljenjima sotonskih utjecaja}[Lt253-1903.28; 1903][https://egwwritings.org/?ref=en\_Lt253-1903.28]} Današnja službena priča da je Ellen White poučavala Trojstvo odjekuje tvrdnjom dr. Kellogga da je Živi Hram učio isto što i Ellen White. \egwinline{\textbf{Ali ne dao Bog da ovaj sentiment prevlada}.}[SpTB02 53.3; 1904][https://egwwritings.org/?ref=en\_SpTB02.53.3]

