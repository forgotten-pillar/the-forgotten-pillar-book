\chapter{Dr. Kellogg i panteizam}

Iz svog osobnog dnevnika, 5. siječnja 1902. godine, sestra White napisala je da Kellogova \egwinline{znanost o Bogu u prirodi je \textbf{istinita}}.

\egw{Predstavljaju mi se stvari koje me zabrinjavaju. Dr. Kellogg hoda istim putem kao što je to učinio ubrzo nakon što je preuzeo svoje odgovornosti u Sanatoriju. \textbf{Ljudska znanost je laž u pogledu toga da Bog nema ličnost}. Znam da je to neistina, i ipak, ako na bilo koji način možemo pomoći doktoru, moramo pokušati to učiniti. Što se može reći? Takvo uzdizanje mu je dato da je na rubu da prekorači liticu. Što itko od nas može učiniti? Sam Gospodin može spasiti dr. Kellogga. \textbf{\underline{Njegova znanost o Bogu u prirodi je istinita}}, ali je postavio prirodu tamo gdje bi Bog trebao biti. Priroda nije Bog, ali Bog je stvorio prirodu. \textbf{\underline{Ova znanost o Bogu u prirodi je točna u jednom smislu}}. \textbf{Bog daje prirodi njezin život, njezina životna svojstva, njezinu ljepotu}. [On] je autor sve ljepote prirode, i dok nam daje ovaj dokaz moćne sile, \textbf{On je osobni Bog i Krist je osobni Spasitelj}.}[Ms236-1902.1; 1902][https://egwwritings.org/read?panels=p12779.6]

\egwnogap{\textbf{Mi ne prihvaćamo zablude ljudi već Riječ Božju da je čovjek stvoren prema slici Boga i Krista}, jer Riječ objavljuje ‘Bog, koji je u više navrata i na više načina nekoć govorio očima po prorocima, u ove posljednje dane progovorio nam je po svome Sinu, kojega je postavio baštinikom svega, \textbf{po kome je i načinio svjetove; koji je sjaj slave njegove i \underline{savršena slika Njegove osobe}}, te sve nosi riječju svoje snage; koji je, sam ostvarivši očišćenje naših grijeha, \textbf{sjeo zdesna Veličanstvu u visinama}.’ Hebrejima 1:1-3.}[Ms236-1902.4; 1902][https://egwwritings.org/read?panels=p12779.9]

Zanimljivo, sestra White također je tvrdila da je Bog u prirodi i da On daje život i životna svojstva. Kellogg je u pravu po ovom pitanju i njegova tvrdnja je definitivno podržana njezinim spisima. Na temelju ove točke, Kellogg se branio govoreći da je Živi Hram u skladu sa spisima sestre White. Pisao je bratu G. I. Butleru točno gdje je sestra White zagovarala jednake sentimente kao i on.

\others{Sestra White je jasno zauzela isti stav u odnosu na ovo pitanje koji sam i ja zauzeo. Naći ćeš to u njenoj maloj knjizi \textbf{Odgoj} u poglavljima ‘\textbf{Bog u prirodi}’ i ‘\textbf{Znanost i Biblija}.’ Naći ćeš to kroz cijelu knjigu ‘\textbf{Čežnja Vjekova}’ i ‘\textbf{Patrijarsi i Proroci}.’}[Letter from Dr. Kellogg to Eld. Butler, February 21, 1904]

Pogledajmo “\textit{Bog u prirodi}” u knjizi Odgoj, gdje možemo pronaći isti sentiment o Bogu u prirodi koji je promicao Kellogg.

\egw{\textbf{Na svim stvorenim stvarima vidi se otisak Božanstva}. Priroda svjedoči o Bogu. Osjetljiv um, doveden u dodir s čudom i misterijem svemira, ne može a da ne prepozna \textbf{djelovanje beskonačne moći}. \textbf{\underline{Ne vlastitom inherentnom energijom} zemlja proizvodi svoje blagodati}, i iz godine u godinu nastavlja svoje kretanje oko sunca. \textbf{Nevidljiva ruka vodi planete u njihovom kretanju po nebesima}. \textbf{\underline{Misteriozan život prožima svu prirodu—život koji održava nebrojene svjetove kroz beskonačno prostranstvo}}, \textbf{koji živi u insektu atoma koji lebdi u ljetnom povjetarcu, koji daje krila letu lastavice i hrani mlade gavrane koje viču, koji dovodi pup do cvata i cvijet do ploda}.}[Ed 99.1; 1903][https://egwwritings.org/read?panels=p29.470]

\egwnogap{\textbf{Ista \underline{sila} koja podržava prirodu, djeluje i u čovjeku}. \textbf{Isti veličanstveni zakoni koji jednako vode zvijezdu i atom, kontroliraju i ljudski život}. \textbf{Zakoni koji upravljaju radom srca, regulirajući tok životne struje tijelu, su zakoni moćne Inteligencije koja ima nadležnost nad dušom}. \textbf{\underline{Od Njega sav život dolazi}}. Samo u skladu s Njim može se pronaći njegova prava sfera djelovanja. Za sve objekte Njegovog stvaranja uvjet je isti—\textbf{život održavan primanjem života od Boga}, život proveden u skladu s voljom Stvoritelja...}[Ed 99.2; 1903][https://egwwritings.org/read?panels=p29.471]

\egw{…Srce koje još nije otvrdnulo dodirima sa zlom brzo \textbf{prepoznaje \underline{Prisutnost} koja prožima sve stvorene stvari}…}[Ed 100.2; 1903][https://egwwritings.org/read?panels=p29.475]

U svojoj obrani, Kellogg se također pozivao na knjigu “Patrijarsi i Proroci”. Tamo čitamo sljedeće:

\egw{Mnogi uče da materija posjeduje vitalnu snagu,—da se određena svojstva prenose na materiju, i onda se ona ostavlja da djeluje kroz svoju inherentnu energiju; i da se operacije prirode provode u skladu s fiksnim zakonima, s kojima se sam Bog ne može miješati. \textbf{To je lažna znanost, i nije podržana Božjom riječju}. Priroda je sluga svog Stvoritelja. Bog ne poništava svoje zakone, niti djeluje protivno njima; \textbf{ali ih neprestano koristi kao svoje instrumente. Priroda svjedoči o inteligenciji, \underline{prisutnosti}, \underline{aktivnoj energiji}, koja djeluje u njoj i kroz njezine zakone. U prirodi je kontinuirani rad \underline{Oca i Sina}.} Krist kaže, ‘Moj Otac do sada radi, i ja radim.’ Ivan 5:17.}[PP 114.4; 1980][https://egwwritings.org/read?panels=p84.445]

Ovi citati su u skladu s citatima iz knjige ”Živi Hram”.

\others{Manifestacije života su raznolike kao i različite pojedinačne životinje i biljke, te dijelovi animiranih stvari. Svaki list, svaka vlat trave, svaki cvijet, svaka ptica, čak i svaki insekt, kao i svaka zvijer ili svako drvo, svjedoče o beskrajnoj raznolikosti i neiscrpnim resursima \textbf{jednog sveprožimajućeg, svekreirajućeg, sveodržavajućeg Života}.}[John H. Kellogg, The Living Temple p. 16][https://archive.org/details/J.H.Kellogg.TheLivingTemple1903/page/n15/]

\others{Inteligencija je jedna od sila svemira, jedna od manifestacija \textbf{\underline{sveprožimajućeg života koji} je stvorio i stvara, \underline{animira i održava}}.}[John H. Kellogg, The Living Temple p. 396][https://archive.org/details/J.H.Kellogg.TheLivingTemple1903/page/n425/]

Ako je Kellogovo razumijevanje Boga kao izvora koji održava i animira prirodu ispravno, gdje je onda njegova greška? Zašto ga se naziva panteistom? Je li pošteno nazvati ga panteistom? On definitivno ne misli tako. Pogledajte što je napisao starješini Butleru:

\others{\textbf{Ja se gnušam na panteizma} koliko i ti. \textbf{Nastojao sam u svojoj knjizi jednostavno poučavati činjenicu da čovjek u svemu ovisi o Bogu i da bi bez božanske sile koja djeluje u njemu, bez Božjeg Duha koji djeluje na elemente koji sačinjavaju njegovo tijelo, on bio prah}.}[Letter from Dr. Kellogg to Eld. Butler, February 21, 1904]

\others{Spreman sam se odreći svih groznih doktrina koje mi vi i drugi pripisujete. Spreman sam priznati da \textbf{nisam panteist} niti spiritualist i da ne vjerujem nijednoj od doktrina koje podučavaju ovi ljudi ili \textbf{panteistički ili spiritualistički spisi}. Nikad u životu nisam pročitao panteističku knjigu. Nikada nisam pročitao knjigu ‘Nova Misao’ ili bilo što slično. Svatko tko će pažljivo pročitati ’Živi Hram’ od prve stranice sve do posljednje, i dati će stvar pošteno i dosljedno razmatrati, trebao bi vrlo jasno vidjeti da \textbf{se ja uopće ne slažem s ovim panteističkim i spiritualističkim teorijama}.}[Ibid.]

Ovo je vrlo teška zagonetka za riješiti, osim ako ne uključite istinu o \emcap{ličnosti Boga}, što smo učinili na početku ove knjige. Da, Bog održava život u prirodi. U prirodi mi \egwinline{\textbf{prepoznajemo \underline{Prisutnost} koja prožima sve stvorene stvari}}[Ed 100.2; 1903][https://egwwritings.org/read?panels=p29.475]. No sam Bog—u svojoj ličnosti—nije u prirodi, niti je priroda Bog. Bog je \textit{osobno biće}, i On je u svom svetom hramu, sjedi na Svom prijestolju. Bog je svugdje prisutan preko svog \textit{predstavnika}, Svetog Duha.

Kada je sestra White rekla \egwinline{Ljudska znanost je laž u pogledu toga da Bog \textbf{nema ličnost},}[Ms236-1902; 1902][https://egwwritings.org/read?panels=p12779.6] posebno se referirala na to da Bog ima fizički oblik osobe, kao što se može vidjeti u kontekstu tog citata. No kada je Dr. Kellogg govorio o ‘\textit{ličnosti},’ nije govorio o obliku ili formi osobe. Godine 1936. u svom predavanju, izrazio je iste sentimente koje je zastupao u Živom Hramu, samo direktnije:

\others{Dakle vidite da je nemoguće pojmiti beskonačne stvari. One su izvan nas. One su \textbf{izvan razumijevanja} i ista stvar vrijedi za \textbf{\underline{beskonačnu ličnost}}. \textbf{Ne možemo stvoriti nikakvu predodžbu o njenom obliku ili veličini ili bilo kakvim ograničenjima bilo koje vrste jer je beskonačna}. Sada, možda vam je to teška ideja za prihvatiti i \textbf{poteškoća prihvaćanja ove ideje je činjenica da \underline{nemamo jasnu ideju o ličnosti}}. \textbf{Mi mislimo o ličnosti \underline{kao povezanoj s oblikom}}.}

\others{...\textbf{To mi je dalo novu koncepciju ličnosti}. \textbf{\underline{Ličnost ne označava osobu, muškarca ili ženu}}. Uopće ne znači takvu vrstu stvari. \textbf{To znači posjedovanje moći da se hoće i čini i misli i planira}.}[\href{https://forgotten-pillar.s3.us-east-2.amazonaws.com/Sanitarium+Lecture+1936.pdf}{Dr. Kellogg Sanitarium Lectures, 1936}; Za transkript vidi \href{https://notefp.link/1938-kellogg-lecture}{https://notefp.link/1938-kellogg-lecture}]

Takav pogled na ličnost primijenjen na Boga odveo je dr. Kellogga u panteizam. Doktrina o \emcap{ličnosti Boga} bavi se ispravnom percepcijom Boga. Kellogova percepcija Boga bila je trinitarijanska percepcija.

\others{Sve što sam želio objasniti u Živom Hramu bilo je da ovaj rad koji se odvija u čovjeku \textbf{se ne odvija sam po sebi \underline{kao navijeni sat}; već je to Božja moć i \underline{Božji Duh koji to provodi}}. \textbf{Mislio sam da sam potpuno izostavio teološku stranu pitanja o \underline{trojstvu i sve te stvari}}. \textbf{Nisam to uopće želio staviti unutra}, i potrudio sam se napisati u uvodu da nisam. Nikada nisam ni sanjao \textbf{o takvoj stvari} kao što je bilo kakvo teološko pitanje koje bi bilo \textbf{uneseno u to}. Samo sam želio pokazati da \textbf{\underline{srce ne kuca svojim vlastitim pokretom} već da je \underline{Božja sila ta koja ga održava}}.}[Intervju, J. H. Kellogg, G. W. Amadon i A. C. Bourdeau, 7. listopada 1907. održan u Kelloggovoj rezidenciji][https://archive.org/details/KelloggVs.TheBrethrenHisLastInterviewAsAnAdventistoct71907/page/n37]

Da, srce ne kuca vlastitim pokretom; to je Božja sila koja ga održava. U tome je Kellogg bio apsolutno u pravu.

\egw{\textbf{Fizički organizam čovjeka je pod Božjim nadzorom, ali \underline{nije poput sata koji je postavljen u rad i mora ići sam od sebe}}. \textbf{Srce kuca, puls slijedi puls, dah slijedi dah, ali imajte na umu da je biće pod Božjim nadzorom}. Vi ste Božja svojina, vi ste Božja građevina. \textbf{U Bogu živimo, mičemo se i jesmo}. \textbf{Svaki otkucaj srca, svaki dah je nadahnuće onog Boga koji je udahnuo u Adamove nosnice dah života}, nadahnuće uvijek prisutnog Boga, velikog JA JESAM.}[13LtMs, Ms 92, 1898, par. 7][https://egwwritings.org/read?panels=p14063.7342012&index=0]

Dr. Kellogova \egwinline{znanost o Bogu u prirodi je istinita.}[Ms236-1902; 1902][https://egwwritings.org/read?panels=p12779.6] Sveto pismo to jasno uči: \bible{Kad bi na to srce svoje upravio, \textbf{te \underline{svoj duh} k sebi povukao}; \textbf{sva tijela zajedno \underline{bi izginula} i čovjek bi se vratio u prah}.}[Job 34:14-15] \bible{…sudovi tvojih dubina su velika: \textbf{Ljude i životinje \underline{ti}, GOSPODE, \underline{održavaš}}… \textbf{Jer u tebi je izvor života}, u tvojemu svjetlu svjetlo ćemo vidjeti.}[Psalam 36:6b,9, GBV]

Ovi pokazatelji svjedoče da je Kellogova znanost o Bogu u prirodi istinita, ali njegovi problemi bili su pogrešni pogledi na ličnost Boga, koji su bili trinitarijanski pogledi. Čak i kada je pojasnio da \others{Bog Otac sjedi na svom prijestolju na nebu gdje je i Bog Sin; dok je Božji život, ili duh ili prisutnost sveprožimajuća sila koja izvršava Božju volju u cijelom svemiru,}[Pismo: Dr. Kellogg W. W. Prescottu, 25. listopada 1903.][https://forgotten-pillar.s3.us-east-2.amazonaws.com/1903-10-25-JHKellogg-to-W.W.Prescott.pdf] još uvijek je imao pogrešne poglede na ličnost Boga—Boga u \others{sveobuhvatnom smislu} kao \others{Božanstvo… Bog Otac, Bog Sin i Bog Sveti Duh}[Ibid.][https://forgotten-pillar.s3.us-east-2.amazonaws.com/1903-10-25-JHKellogg-to-W.W.Prescott.pdf]. Njegov trinitarijanski pogled \textit{nije mogao} \others{zadovoljavajuće razjasniti.}[Pismo: A. G. Daniells W. C. Whiteu, 29. listopada 1903.][https://forgotten-pillar.s3.us-east-2.amazonaws.com/Letter-A-G-Daniells-to-W-C-White-October-29-1903.pdf]

Zaključak je zastrašujući. Ako vjerujete da srce ne kuca vlastitim pokretom već da ga Božja snaga održava, i to kombinirate s vjerom da sam Bog nije opipljivo biće već duh svugdje prisutan, onda u očima Duha Proroštva vi ste panteist. Percepcija kvalitete ili stanja koja Boga čine osobom čini razliku između pravog vjernika i panteista.
