\chapter{Autoritet Fundamentalnih Principa} \label{chap:authority}

U 10. poglavlju Posebnih Svjedočanstava, čitamo kako je Bog uspostavio temelj naše vjere. Sestra White koristila je nekoliko različitih izraza za temelj naše vjere. Koristila je izraze kao što su: “\textit{platforma vječne istine}”, “\textit{stupovi naše vjere}”, “\textit{principi istine}”, “\textit{točke principa}”, “\textit{međaši}”, i “\textit{temeljni principi}—svi se oni odnose na \emcap{Fundamentalne Principe}. Na kraju poglavlja, potvrdila je Božju volju da \egwinline{On nas poziva da se držimo čvrstim zahvatom vjere, \textbf{fundamentalnih principa} koji su \textbf{zasnovani na neupitnom \underline{autoritetu}}.}

Autoritet na kojem su \emcap{fundamentalni principi} zasnovani je neupitan. Oni su rezultat dubokog, ozbiljnog proučavanja u vrijeme velikog razočaranja, kada je \egwinline{\textbf{\underline{točka po točka} bila tražena molitvenim proučavanjem, i potvrđena \underline{čudesnom silom Gospodnjom}}}\footnote{Ibid.}. \egwinline{\textbf{Tako su \underline{vodeće točke naše vjere} kako ih držimo danas čvrsto utvrđene}. \textbf{\underline{Točka za točkom} je jasno definirana, i sva braća su došla u harmoniju}.}[Lt253-1903.4; 1903][https://egwwritings.org/?ref=en\_Lt253-1903.4]

Oni su rezultat ozbiljnog proučavanja Biblije naših pionira, nakon isteka vremena 1844. Kako je pokret adventista sedmog dana napredovao, pojavila se potreba za osnivanjem organizacije, što je realizirano 1863. Godine 1872., Crkva Adventista Sedmoga Dana izdala je dokument pod nazivom “\textit{Deklaracija Fundamentalnih Principa koje Uče i Prakticiraju Adventisti Sedmog Dana}”. Ovo je bio prvi pisani dokument koji objavljuje \emcap{fundamentalne principe} kao javne izjave adventističke vjere. Ovaj dokument je bio javni sažetak adventističke vjere i objavio je \others{ono što je, i što je bilo, s velikom složnošću, vjerovano među} adventističkim narodom. Napisan je \others{da odgovori na upite} o tome što Adventisti vjeruju, \others{da ispravi lažne izjave koje kruže} i da \others{ukloni pogrešne dojmove}[FP1872 3.1; 1872][https://egwwritings.org/?ref=en\_FP1872.3.1&para=928.8].

Danas se još uvijek raspravlja o tome tko je autor sinopsisa jer je izvorno, 1872. godine, izdan anonimno. Godine 1874. James White ga je objavio u Signs of the Times\footnote{\href{https://adventistdigitallibrary.org/adl-364148/signs-times-june-4-1874}{Signs of the Times, 4. lipnja 1874}} i Uriah Smith u Review and Heraldu\footnote{\href{http://documents.adventistarchives.org/Periodicals/RH/RH18741124-V44-22.pdf}{The Advent Review and Herald of the Sabbath, 24. studenog 1874}}—obojica su se potpisali svojim potpisima. Godine 1889. Uriah Smith je revidirao deklaraciju dodavši tri točke; objavljena je u Adventističkom godišnjaku s njegovim potpisom. Uriah Smith je umro 1903. i neka naredna uzastopna tiskanja \emcap{Fundamentalnih Principa} tiskana su pod njegovim imenom. Tiskani su u godišnjacima—svake godine od 1905. do 1914\footnote{Za detaljniju kronologiju tiskanja Fundamentalnih Princnipa vidi \hyperref[appendix:timeline]{Dodatak: Fundamentalni Principi - Vremenska Linija}}. Sestra White umrla je 1915. i sljedećih 17 godina \emcap{fundamentalni principi} nisu tiskani. Njihovo sljedeće pojavljivanje bilo je u Godišnjaku 1931. godine kada su doživjeli značajne izmjene.

Godine 1971. LeRoy Froom je napisao o izdanju 1872.: \others{Iako se pojavljuje anonimno, zapravo ju je napisao Smith}[LeRoy Froom, Movement of Destiny, str. 160]. Nažalost, nije dao nikakve podatke koji bi potkrijepili svoju tvrdnju. Žalosno je vidjeti kako pro-trinitarijanski skolari pridaju \emcap{Fundamentalnim Principima} jako malo važnosti. Njihova je prava vrijednost uvelike smanjena pripisivanjem ovih uvjerenja nekoj maloj skupini ljudi, uglavnom osobnim uvjerenjima Jamesa Whitea ili Uriaha Smitha, umjesto vjerovanja koje je bilo \others{s velikom složnošću, vjerovano među}[Predgovor Fundamentalnih Principa 1872] adventistima. Godine 1958. časopis Ministry opisao je \emcap{Fundamentalne Principe} na sljedeći način:

\others{Istina je da je 1872. tiskana ‘Deklaracija Fundamentalnih Principa koje Uče i Prakticiraju Adventisti Sedmog Dana’, \textbf{ali ju denominacija nikada nije prihvatila i stoga se ona ne može smatrati službenom}. Očito je mala skupina, \textbf{možda čak jedna ili dvije osobe, nastojala izraziti riječima ono što su mislili da su stajališta cijele crkve…}}[Ministry Magazine “\textit{Our Declaration of Fundamental Beliefs}”, January 1958, Roy Anderson, J. Arthur Buckwalter, Louise Kleuser, Earl Cleveland and Walter Schubert]

Problematično je to što nema dokaza koji bi poduprli tvrdnju da \emcap{Fundamentalni Principi} nisu predstavljali vjeru cijelog tijela. Svakako znamo da ih je sestra White podržavala i, samo iz njezina utjecaja, znamo da su ta vjerovanja doista prihvaćena od strane denominacije—to je uz činjenicu da su višestruko tiskani tijekom 42 godine, tijekom života Ellen White.

Ali ne bi trebalo biti kontroverze oko autorstva \emcap{Fundamentalnih Principa}. Imamo citat sestre White o tome tko ih je napisao. Govoreći o Uriah Smithu, sestra White je napisala:

\egw{\textbf{Brat Smith je bio s nama u usponu ovog djela. On razumije kako smo \underline{mi—moj suprug i ja}—nosili i napredovali korak po korak i podnosili teškoće, siromaštvo i nedostatak sredstava. S nama su bili ti rani radnici. Posebno je starješina Smith bio jedno s mojim mužem u njegovoj ranoj dobi}. …}[Ms54-1890.6; 1890][https://egwwritings.org/?ref=en\_Ms54-1890.6&para=7213.15]

\egwnogap{\textbf{\underline{Stajali smo rame uz rame sa Starješinom Smithom u ovom djelu dok je Gospod postavljao temeljne principe}}. \textbf{Morali smo stalno raditi protiv ideja pojedinaca}, koji su mislili da ispravni poslovni odnosi u pogledu rada koji se morao obaviti bio rezultat njihovog svetovnog razmišljanja, i onih mrzovoljnih koji bi se predstavljali sposobnima za odgovornosti, a kojima se nije mogao povjeriti rad jer bi ga skrenuli na pogrešnu stazu. \textbf{Korak po korak se morao činiti, \underline{ne prema ljudskoj mudrosti} već prema mudrosti i uputama Onoga koji je premudar da bi pogriješio i predobar da bi nam učinio zlo}. \textbf{Bilo je toliko mnogo elemenata koje je trebalo prokušati i testirati. Zahvaljujem Gospodinu što su starješine Smith, Amadon i Batchellor još uvijek živi. Oni su činili članove naše obitelji u najtežim dijelovima naše povijesti}.}[Ms54-1890.7; 1890][https://egwwritings.org/?ref=en\_Ms54-1890.7&para=7213.16]

Prema ovom citatu, tko je postavio temeljne principe?

\egwinline{\textbf{\underline{Mi smo stajali rame uz rame sa starješinom Smithom u ovom radu dok je Gospodin postavljao temeljne principe}}.} \textbf{Gospodin}! Ali tko ih je sastavio u deklaraciju? To je bio starješina Smith sa Jamesom Whiteom i sestrom White; to vidimo kada sestra White kaže \egwinline{\textbf{mi} smo stajali rame uz rame sa starješinom Smithom}. Ovaj ‘\textit{mi}’ objašnjen je u prethodnom paragrafu: \egwinline{On \normaltext{[starješina Smith]} razumije kako \textbf{smo mi—moj suprug i ja}—nosili i napredovali}. Ovim citatom, sestra White je jasno bila uključena kada je Gospodin postavljao \emcap{Fundamentalne Principe}.

Istina je da je Deklaracija \emcap{Fundamentalnih Principa} napisana od strane male grupe ljudi, poimence starješina Smith, James White i Ellen White, ali oni su nastojali riječima izraziti ono što je bilo pravo gledište cjelokupnog crkvenog tijela. Oni su ispravno predstavljali \emcap{fundamentalne principe}—istine primljene na početku našeg rada. Ako to nije slučaj, onda je ova deklaracija u potpunoj suprotnosti onome što sâma tvrdi. Ona je napisana \others{da odgovori na upite} o tome u što Adventisti Sedmog Dana vjeruju, \others{da ispravi lažne izjave koje kruže} i da \others{ukloni pogrešne dojmove}[FP1872 3.1; 1872][https://egwwritings.org/?ref=en\_FP1872.3.1&para=928.8]. Ako je ovaj dokument lažno predstavio adventističko stajalište, zašto je bilo dopušteno njegovo neprekidno periodično tiskanje tijekom 42 godine? Tiskana je sve do smrti Ellen White. Da je ovaj dokument lažno predstavio stav crkve, ne bi li Ellen White digla glas protiv njega? Uvijek je dizala svoj glas protiv pogrešnog predstavljanja stajališta Adventista Sedmog Dana, kao što je učinila s D. M. Canrightom i dr. Kelloggom. Ako su \emcap{Fundamentalni Principi} lažno predstavljali adventistički stav, onda bi se sva naknadna pretiskavanja trebala pripisati teoriji zavjere. To bi bila najveća teorija zavjere unutar Crkve Adventista Sedmog Dana. Ikad. Sklad između spisa Ellen White, adventističkih pionira i tvrdnji iznesenih u Deklaraciji \emcap{Fundamentalnih Principa} svjedoči o činjenici da je ova deklaracija točan \others{sažetak glavnih obilježja} vjere Adventista Sedmog Dana, \others{nad kojima, koliko znamo, stoji jednoglasnost cijeloga tijela}[Predgovor Fundamentalnih Principa 1889].

Smrću sestre White 1915. prestalo je tiskanje \emcap{Fundamentalnih Principa}. Od 1915. nadalje Godišnjak nije tiskao nijednu izjavu vjere sve do 1931. U to su vrijeme \emcap{Fundamentalni Principi} doživjeli značajne promjene. Po prvi put, Trojstvo je uvedeno u \emcap{fundamentalne principe}. U točkama 2. i 3. stoji:

\others{2. \textbf{Božanstvo, ili Trojstvo, sastoji se od Vječnog Oca, \underline{osobnog, duhovnog Bića}}, svemoćnog, \underline{sveprisutnog}, sveznajućeg, beskonačnog u mudrosti i ljubavi; \textbf{Gospodina Isusa Krista, Sina Vječnog Oca, po kojem je sve stvoreno} i po kome će se postići spasenje otkupljenih domaćina; \textbf{Duh Sveti, treća osoba Božanstva}, velika moć obnavljanja u djelu otkupljenja. Mat. 28:19.}

\others{3. \textbf{Isus Krist je sâm Bog, koji je iste prirode i esencije kao i Vječni Otac}…}[Yearbook of the Seventh-day Adventist Denomination, 1931, page. 377][https://static1.squarespace.com/static/554c4998e4b04e89ea0c4073/t/59d17eec12abd9c6194cd26d/1506901758727/SDA-YB1931-22+\%28P.+377-380\%29.pdf]

Ova promjena, u korist Trojstva, pojavila se šesnaest godina nakon smrti sestre White. Usporedba ove izjave s originalnim \emcap{Fundamentalnim Principima} pokazuje nekoliko upečatljivih razlika. Otac je još uvijek osobno, duhovno Biće, stvoritelj svih stvari, ali se više ne naziva “\textit{jednim Bogom}”. Isus Krist je još uvijek Sin Vječnog Oca, preko kojega je Otac sve stvorio; te je također, iste naravi i bîti kao i Otac. Iako su to bili pojmovi korišteni za objašnjenje \emcap{ličnosti Boga} u prijašnjoj izjavi, postavlja se pitanje značenja pojma “\textit{osobnog duhovnog bića}” koje je sâm po sebi svugdje prisutno? Duh Sveti više nije predstavnik Oca, niti sredstvo Očeve sveprisutnosti. Ova izjava, iako ima jako sličnu retoriku originalnih \emcap{Fundamentalnih Principa}, odstupa od našeg originalnog učenja o prisutnosti i \emcap{ličnosti Boga}.

Prema LeRoyu Froomu, ovu izjavu je u potpunosti napisao Francis Wilcox, uz odobrenje trojice drugih braće (C.H. Watson, M.E. Kern i E.R. Palmer).\footnote{LeRoy Froom, Movement of Destiny, p. 411, 413, 414} U neobjavljenom radu \textit{The Seventh-day Adventist Church in Mission: 1919-1979}, čitamo kako je starješina Wilcox dao ovu izjavu suprotno vjerovanju crkvenog tijela i objavio je bez njihovog odobrenja.

\others{\textbf{Shvativši da Odbor Generalne konferencije ili bilo koje drugo crkveno tijelo nikada ne bi prihvatilo dokument u obliku u kojem je pisano}, starješina Wilcox, u potpunom saznanju grupe \normaltext{[C.H. Watson, M.E. Kern i E.R. Palmer]}, predao je Izjavu izravno Edsonu Rogersu, statističaru Generalne konferencije, koji je objavio u izdanju Godišnjaka iz 1931. godine, gdje se od tada pojavljuje. Stoga je bez službene suglasnosti Odbora Generalne konferencije i bez ikakvog formalnog denominacijskog usvajanja, izjava starješine Wilcoxa postala prihvaćena izjava naše vjere.}[Dwyer, Bonnie. “A New Statement of Fundamental Beliefs (1980) - Spectrum Magazine.” \textit{Spectrum Magazine}, 7. lipnja 2009, \href{https://spectrummagazine.org/news/new-statement-fundamental-beliefs-1980/}{spectrummagazine.org/news/new-statement-fundamental-beliefs-1980/}. Accessed 30 Jan. 2025.]

Godine 1980., napravljena je konačna promjena javne deklaracije vjere Crkve Adventista Sedmoga Dana. Generalna Konferencija je izglasala usvajanje današnje službene izjave:

\others{\textbf{Postoji jedan Bog: Otac, Sin i Sveti Duh, jedinstvo tri vječne Osobe}. Bog je besmrtan, svemoguć, sveznajuć, iznad svega i \textbf{uvijek nazočan}. On je neograničen i izvan moći ljudske spoznaje; pa ipak poznat po onome kako se sam objavio. On je vječno dostojan da Ga se obožava, poštuje i da Mu služi sve što je stvoreno.}[Seventh-day Adventists Believe: A Biblical Exposition of 27 Fundamental Doctrines, p. 16]

U ovom kratkom povijesnom pregledu vidimo da je izjava iz 1931. godine “među korak” između izvornog adventističkog vjerovanja prema potpunoj trinitarijanskom vjerovanju.

Promjena u našim vjerovanjima dogodila se tijekom velikog vremenskog razdoblja kroz mnoge rasprave. Naša adventistička povijest ostavila je trag tih promjena. Ako smo iskreni tragači za istinom, valja nam detaljno proučiti ovo pitanje. Možemo li vidjeti, u našoj adventističkoj povijesti, zašto smo napustili prvu točku \emcap{Fundamentalnih Principa} u korist doktrine o Trojstvu? Svakako! U sljedećim poglavljima pogledat ćemo neke povijesne dokumente koji pokazuju zašto smo se pomaknuli od prve točke \emcap{Fundamentalnih Principa}, kojih smo se držali u ranim godinama, u prilog prihvaćanja doktrine o Trojstvu. Tijekom proučavanja ovih dokumenata, pozivamo vas da s molitvom usporedite te promjene sa svojim vlastitim vjerovanjima.
