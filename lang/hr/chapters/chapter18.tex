\qrchapter{https://forgottenpillar.com/rsc/hr-fp-chapter18}{Nebeski Trio}

Do sada smo vidjeli dokaze da je Ellen White znala o trinitarijanskim sentimentima dr. Kellogga, i vidjeli smo kako je na to odgovorila. Ona je uvijek uzdizala istinu o prisutnosti i \emcap{ličnosti Boga}, i pozivala na povratak temeljima naše vjere—\emcap{Fundamentalnim Principima}. Međutim, kada adventistički teolozi raspravljaju o doktrini o Trojstvu i Ellen White, oni ne pristupaju tome na isti način kao što je to činila Ellen White. \emcap{Fundamentalni Principi} zajedno s doktrinom o \emcap{ličnosti Boga} su zapostavljeni, i predstavljen je iskrivljen narativ da je Ellen White bila trinitarijanka i odgovorna za prihvaćanje doktrine o Trojstvu u naše redove. Želimo osporiti ovu iskrivljenu priču gledajući dokaze koji se često koriste za podržavanje ovog lažnog narativa.

Jedan od najistaknutijih citata koji podržava tvrdnju da je sestra White bila odgovorna za prihvaćanje doktrine o Trojstvu u naše redove su njezini spisi i komentari o Mateju 28:19\footnote{\bible{Idite dakle i učinite sve narode učenicima, krsteći ih u ime Oca i Sina i Duha Svetoga}[Matej 28:19]}. Najistaknutiji citat koji se ističe u obrani doktrine o Trojstvu je citat “\textit{Nebeski Trio}”:

\egw{\textbf{Postoje \underline{tri žive osobe} \underline{nebeskog tria}}; u ime ove tri velike sile—\textbf{Oca, Sina, i Svetoga Duha}—oni koji primaju Krista živom vjerom su kršteni, i ove sile će surađivati sa poslušnim podanicima neba u njihovim naporima da žive novi život u Kristu...}[Ev 615.1; 1946][https://egwwritings.org/read?panels=p30.3407]

Da ponovimo, ovaj se citat često navodi kao argument da je sestra White branila i zagovarala doktrinu o Trojstvu. Ali, ako pogledamo ovaj citat u njegovom književnom kontekstu, vidimo da je unutar samog citata ona zapravo \textit{opovrgavala} ovu doktrinu i uzvisivala istinu o \emcap{ličnosti Boga}. Nekima je ovo apsurdna tvrdnja, ali pozivamo vas da donesete svoj sud na temelju predstavljenih podataka. Ispitajmo kontekst ovog citata.

\egw{Upućena sam da kažem, \textbf{sentimentima} onih koji traže napredne znanstvene ideje \textbf{\underline{ne smije se vjerovati}}. Učinjeni su slijedeći prikazi: ‘\textbf{Otac je kao Svjetlo nevidljivo; Sin je kao Svjetlo otjelovljeno; Duh je kao Svjetlo rasvjetljeno.}’ ‘\textbf{Otac je poput rose u nevidljivoj pari; Sin je poput rose skupljene u prekrasnom obliku; Duh je poput rose pale na sjedište života.}’ Drugi prikaz: ‘\textbf{Otac je poput nevidljive pare. Sin je kao natovaren oblak. Duh je Kiša – pala i djelujuća u osvježavajućoj snazi.}’}[Ms21-1906.8; 1906][https://egwwritings.org/read?panels=p9754.15]

Kojim sentimentima ne treba vjerovati? Podaci sugeriraju da su to trinitarijanski sentimenti \textit{jednog Boga u tri osobe}. Kako to znamo? Vidimo u književnom kontekstu ilustracija koje je sestra White citirala. Suprotno popularnom vjerovanju da se referirala na “\textit{lažno}” trojstvo koje je izrazio dr. Kellogg,\footnote{Whidden, Woodrow W. et al. The Trinity: Understanding God’s Love, His Plan of Salvation, and Christian Relationships. Hagerstown, MD, Review and Herald Pub. Association, 2002, str. 216.} ona je zapravo referirala na trinitarijansku ideju o \textit{tri žive osobe jednog živog Boga}, koju je zagovarao William Boardman u svojoj knjizi “Uzvišeniji Kršćanski Život”\footnote{U originalu "\textit{Higher Christian Life}"}, koju je citirala. Kontekst je važan. Kontekst citata koje je citirala pokazuje da prikazi Oca, Sina i Svetoga Duha služe za ilustraciju sentimenta o tri žive osobe \textit{jednog} Boga. To je sentiment za koji nam je Bog jasno dao upute da mu ne vjerujemo. Neka podaci budu svoj vlastiti tumač.

\section*{Uzvišeniji Kršćanski Život, William Boardman}

Ellen White je posjedovala knjigu Williama Boardmana “Uzvišeniji Kršćanski Život”. Bila je to dobra knjiga o kršćanskom posvećenju, ali je posjedovala trinitarijanski sentiment, za koji je Bog preko sestre White specifično dao upute da upozori Njegov narod. Ovo je još jedan primjer pokazatelja gdje vidimo da je Ellen White bila upoznata s trinitarijanskim stavom i da ga je izravno adresirala. Upoznajmo se s trinitarijanskim sentimentima koje je promovirao William Boardman.

Govoreći o Trojedinom Bogu, William Boardman piše:

\othersQuote{I ponovno, Otac je autor i projektant spasenja kroz vjeru u njegovog Sina; i kada vjerujemo u Sina mi častimo Oca, jer prihvaćamo njegov plan spasenja za nas, opravdavajući njegovu mudrost, i djelujemo u skladu s njegovom voljom. \textbf{Pogled na službene i esencijalne odnose osoba Svetog Trojstva jednih prema drugima i prema nama, može baciti dodatno svjetlo na našu stazu}. O ovoj temi lakomislenost bi graničila s bogohuljenjem. To je sveto tlo. Onaj tko se usuđuje stupiti na to neka zagazi bosonoga stopala, i otkrivene pognute glave.}[William Boardman, The Higher Christian Life, str. 99; 1858][https://archive.org/details/higherchristian02boargoog/page/n106/]

Brat Boardman želi da bacimo \others{pogled na službene i esencijalne odnose} triju osoba Svetog Trojstva. On tvrdi da je \textit{Bog jedan ali također i tri}–\textit{Trojedini}–predstavljajući službene i esencijalne odnose osoba Svetog Trojstva. Njegova temeljna izjava i okvir za njegovu tezu je sljedeći:

\othersQuote{\textbf{Otac je punina Božanstva \underline{nevidljivo}, bez forme, kojeg nijedno stvorenje nije vidjelo niti može vidjeti}. \\
\textbf{Sin je punina Božanstva \underline{utjelovljeno}, kako bi ga njegova stvorenja mogla vidjeti, i poznavati ga, i vjerovati mu}. \\
\textbf{Duh je punina Božanstva \underline{u svim aktivnim djelovanjima}, bilo stvaranja, providnosti, otkrivenja ili spasenja, kroz koje Bog očituje sebe svemiru i kroz svemir}.}[William Boardman, The Higher Christian Life, str. 100][https://archive.org/details/higherchristian02boargoog/page/n108/]

Ova izjava je temeljna za njegove sljedeće izjave i ilustracije. U sljedećim odlomcima, William Boardman daje biblijske motive kako bi ilustrirao \others{službene i esencijalne odnose Svetog Trojstva}—\textit{to jest, Bog je jedan, ali ipak tri}. On piše:

\othersQuote{Jedno od Isusovih imena dat će iste analogije u ne manje upečatljivom svjetlu - \textbf{Sunce Pravde}. \\
Sva svjetlost sunca na nebu jednom je bila skrivena u nevidljivosti prvobitne tame; i nakon toga, svjetlost koja sada blista u dnevnoj zvijezdi bila je, kada je prva zapovijed izašla, Neka bude svjetlo! i svjetlo bi, u najboljem slučaju samo difuzna izmaglica sivog svitanja jutra stvaranja iz tame kaotične noći, bez forme, ili tijela, ili središta, ili sjaja, ili slave. Ali kada je odvojena od tame i smještena u sunce, tada je u svom veličanstvenom sjaju postala tako blistava da je nitko osim orlovskog oka nije mogao gledati u lice. \\
Ali tada opet, njegove zrake padajući ukoso kroz Zemljinu atmosferu i pare, razveseljuju cijeli svijet istom svjetlošću, rastjerujući zimu, i hladnoću, i tamu; pokrećući Proljeće u cvjetnoj ljepoti, i Ljeto u proljetnom obilju, i Jesen natovarenu zlatnim blagom za žitnicu.
\textbf{Otac je kao Svjetlo nevidljivo}. \\
\textbf{Sin je kao Svjetlo otjelovljeno}. \\
\textbf{Duh je kao Svjetlo rasvjetljeno}.}[William Boardman, The Higher Christian Life, str. 101,102][https://archive.org/details/higherchristian02boargoog/page/n108/]

Ova ilustracija Sunca Pravde pokazuje da je Bog Otac, koji je \textit{punina Božanstva nevidljiva}, simbolički prikazan kao Svjetlo koje \others{je jednom bilo skriveno u nevidljivosti prvobitne tame}. Sin, koji je \textit{punina Božanstva otjelovljeno}, je poput Svjetla koje je utjelovljeno u \others{jutru stvaranja}. Duh Sveti, koji je \textit{punina Božanstva u svim aktivnim djelovanjima}, je poput \others{Svjetla rasvjetljenog}. William Boardman nam daje još jednu sličnu ilustraciju kako bi pojasnio \others{službene odnose osoba Božanstva}:

\othersQuote{Jedna od usporedbi za blagoslovljene utjecaje Duha, \textbf{dok daje iste službene odnose osoba Božanstva, jednih prema drugima i prema nama}, može ih ilustrirati još dalje,—\textbf{Rosa},—\textbf{Rosa Hermona} - rosa na pokošenoj livadi. Prije nego što se rosa uopće skupi u kapljice, ona visi nad cijelim krajolikom u nevidljivoj pari, sveprisutna ali nevidljiva. Malo-pomalo kako svjetlost blijedi u jutro, i kako temperatura pada i doseže točku rose, nevidljivo postaje vidljivo, utjelovljeno; i, kako sunce izlazi, stoji u dijamantnim kapljicama koje drhte i blistaju na mladim zrakama sunca u bisernoj ljepoti na listu i cvijetu, preko cijelog lica prirode. \\
Ali sada opet, povjetarac se diže, dah neba nježno se nosi duž, tresući list i cvijet, i u trenutku biserne kapljice su ponovno nevidljive. Ali gdje sada? Pale su na korijen bilja i cvijeća da daju novi život, svježinu, snagu svemu što dodirnu. \\
\textbf{Otac je poput rose u nevidljivoj pari}. \\
\textbf{Sin je poput rose skupljene u prekrasnom obliku}. \\
\textbf{Duh je poput rose pale na sjedište života}.}[William Boardman, The Higher Christian Life, str. 102,103][https://archive.org/details/higherchristian02boargoog/page/n110/]

Otac, koji je \textit{punina Božanstva nevidljiva}, ilustriran je \others{rosom u nevidljivoj pari}. Sin, koji je \textit{punina Božanstva utjelovljeno}, ilustriran je \others{rosom skupljenom u prekrasnom obliku}. Duh, koji je \textit{punina Božanstva u svim aktivnim djelovanj}, ilustriran je \others{rosom palom na sjedište života}. Sljedeća ilustracija koja prikazuje službene odnose triju ličnosti jednog Boga je kroz još jednu biblijsku usporedbu—Kišu.

\othersQuote{\textbf{Još jedna od ovih biblijskih usporedbi} – nipošto ih ne iscrpljujući – neće biti nedobrodošla, niti beskorisna, - \textbf{Kiša}. \\
Kiša, poput rose, lebdi u nevidljivosti i sveprisutnosti u početku, nad svime, oko svega. Nitko je ne vidi. Dok ostaje u svojoj nevidljivosti, zemlja se suši, grude se sljepljuju, tlo puca, sunce izlijeva svoju žarku toplinu, vjetrovi podižu prašinu u kružnim vrtlozima i kotrljajućim oblacima, a glad mršava i pohlepna korača zemljom, praćena kugom i smrću. Malo-pomalo, željni promatrač vidi mali oblak poput ruke kako se diže daleko nad morem. On se skuplja, skuplja, skuplja; dolazi i širi se dok dolazi, veličanstveno preko cijelog neba: - Ali sve je još uvijek isušeno i suho i mrtvo na zemlji. \\
Ali sada dolazi kap, i kap za kapi, brže, brže – pljusak, kiša – jureći naprijed i dajući zemlji sva blaga oblaka – grude se otvaraju, brazde omekšavaju, izvori, potoci, rijeke, bujaju i pune se, i cijela zemlja je ponovno razveseljena obnovljenim obiljem. \\
\textbf{Otac je poput nevidljive pare}. \\
\textbf{Sin je kao natovaren oblak i padajuća kiša}. \\
\textbf{Duh je Kiša – pala i djelujuća u osvježavajućoj snazi}.}[William Boardman, The Higher Christian Life, str. 103,104][https://archive.org/details/higherchristian02boargoog/page/n110/]

Dajmo Williamu Boardmanu pošteno saslušanje. On ne govori da je Otac \others{nevidljiva para}; radije, koristi metaforu kiše i \others{nevidljive pare} kako bi ilustrirao svoju glavnu poantu da je Otac nevidljiva punina Božanstva. Tako je i sa Sinom, koji je, baš kao kiša manifestirana u natovarenim oblacima, sva punina Božanstva očitovana. Kako bi osigurao da njegovi sentimenti ne budu potencijalno pogrešno predstavljeni, William Boardman je pojasnio svoj sentiment. To je bio upravo onaj sentiment za koji je Ellen White bila upućena od Boga da mu \egwinline{ne smije se vjerovati}:

\othersQuote{\textbf{Ove usporedbe su sve nesavršene. One više skrivaju nego što ilustriraju \underline{tri-ličnost jednoga Boga}, jer one nisu osobe nego stvari, siromašne i zemaljske u najboljem slučaju, kako bi predstavile žive ličnosti živoga Boga. One mogu, međutim, ilustrirati službene odnose svakoga prema drugima i svakoga i svih prema nama. I više. One također mogu ilustrirati istinu da sva punina Onoga koji ispunjava sve u svemu, prebiva u svakoj osobi \underline{Trojedinog Boga}}. \\
\textbf{Otac je sva punina Božanstva NEVIDLJIVO}. \\
\textbf{Sin je sva punina Božanstva OČITOVANO}. \\
\textbf{Duh je sva punina Božanstva KOJE ČINI OČITOVANIM}. \\
\textbf{Osobe nisu puke službe, ili načini objave, već žive osobe živoga Boga}.}[William Boardman, The Higher Christian Life, str. 104,105][https://archive.org/details/higherchristian02boargoog/page/n112/]

Ključno je naglasiti da kada Boardman koristi ove biblijske usporedbe iz prirode, govori o ilustracijama, a ne o stvarnosti. Ove reprezentacije ilustriraju njegove sentimente. Prema njegovom vlastitom priznanju, to je bio sentiment o tri \others{žive ličnosti živoga Boga.} Iako su ove ilustracije nesavršene, one mogu \others{ilustrirati službene odnose} \others{tri-ličnosti jednoga Boga} i \others{istinu da sva punina Onoga koji ispunjava sve u svemu, prebiva u svakoj osobi Trojedinog Boga.} Jedan Bog u tri osobe je sentiment o kojem je riječ, i taj sentiment je zajednički svim vrstama i verzijama doktrine o Trojstvu—uključujući i naš trenutni trinitarijanski stav u drugoj točki Temeljnih Vjerovanja.\footnote{\others{Postoji \textbf{jedan Bog}: Otac, Sin i Sveti Duh, \textbf{jedinstvo tri} vječne \textbf{Osobe}…} 2. točka Temeljnih vjerovanja}

U ovom kratkom pregledu sentimenta Williama Boardmana, jasno je da su sentimenti o kojima je Ellen White bila upućena od Boga da ih osudi, bili sentimenti o Trojedinom Bogu, ili \textit{tri žive osobe Trojstva}. S tim činjenicama na umu, pogledajmo odgovor Ellen White.

\section*{Ellen White o sentimentu Williama Boardmana}

Koristeći citat o Nebeskom Triu, tvrdi se da je Ellen White bila trinitarijanka. To se tvrdi u neznanju ili ponekad namjerno ignorirajući kontekst ovog vrijednog citata. Kada čitate odgovor Ellen White, u kojem brani naše percepcije o Bogu, pokušajte prepoznati kome se obraća kada govori o Bogu. Je li Bog kojeg je branila Trojstvo ili Otac? Referirajući se na Boardmanove ilustracije, rekla je:

\egw{\textbf{Svi ti \underline{spiritualistički} prikazi su obično ništavilo}. Oni su nesavršeni, neistiniti. Oni oslabljuju i umanjuju Veličanstvo s kojim se ni jedna zemaljska usporedba ne može usporediti. \textbf{Bog se ne može usporediti sa stvarima koje je Njegova ruka načinila}. One su samo zemaljske stvari, koje pate pod Božjim prokletstvom radi čovjekova grijeha. \textbf{Otac se ne može opisati stvarima sa zemlje}. \textbf{Otac je sva punina Božanstva \underline{tjelesno} i \underline{nevidljiv je smrtnom pogledu}}.}[Ms21-1906.9; 1906][https://egwwritings.org/read?panels=p9754.15]

Promatrajući kontekst, očito je da sestra White slijedi Boardmanovu liniju razmišljanja i ispravlja pogreške. Za bolju usporedbu, pogledajmo njihove spise jedan pored drugoga:

\begin{table}[H]
\centering
\renewcommand{\arraystretch}{1.5}
\setlength{\tabcolsep}{15pt}
\begin{tabular}{|p{0.4\textwidth}|p{0.4\textwidth}|}
\hline
\multicolumn{1}{|c|}{\textbf{William Boardman}} & \multicolumn{1}{c|}{\textbf{Ellen G. White}} \\ \hline
\othersQuote{Ove usporedbe su sve nesavršene. One više skrivaju nego što \textbf{ilustriraju tri-ličnost \underline{jednoga Boga}}, jer one nisu osobe nego stvari, siromašne i zemaljske u najboljem slučaju, kako bi predstavile \textbf{žive ličnosti živoga Boga}. \textbf{One mogu, međutim, ilustrirati službene odnose svakoga prema drugima i svakoga i svih prema nama. I više. One također mogu ilustrirati istinu da sva punina Onoga koji ispunjava sve u svemu, prebiva u \underline{svakoj osobi Trojedinog Boga}}.}[str. 104,105][https://archive.org/details/higherchristian02boargoog/page/n112] & 
\egw{\textbf{Svi ti \underline{spiritualistički} prikazi su obično ništavilo}. Oni su nesavršeni, neistiniti. Oni oslabljuju i umanjuju Veličanstvo s kojim se ni jedna zemaljska usporedba ne može usporediti. \textbf{Bog se ne može usporediti sa stvarima koje je Njegova ruka načinila}. One su samo zemaljske stvari, koje pate pod Božjim prokletstvom radi čovjekova grijeha. \textbf{Otac se ne može opisati stvarima sa zemlje}.}[Ms21-1906.9; 1906][https://egwwritings.org/read?panels=p9754.15] \\ \hline
\end{tabular}
\end{table}

U ovoj usporedbi, jasno je tko je Bog za Williama Boardmana, a tko za sestru White. Za Boardmana, Bog je Trojedini Bog, tri-ličnost jednoga Boga. Za sestru White, Bog je Otac. Za Boardmana, te su reprezentacije nesavršene jer \others{više skrivaju nego što ilustriraju tri-ličnost jednoga Boga}, a za sestru White te reprezentacije su nesavršene jer \egw{Otac se ne može opisati stvarima sa zemlje}. Za Boardmana, Bog je \textit{Trojedini Bog}; za sestru White, Bog je \textit{Otac}.

Jedina točka Boardmana koju Ellen White potvrđuje jest da su te reprezentacije nesavršene. Sigurno je da William Boardman ne bi se složio s Ellen White da su te reprezentacije \textit{spiritualističke} i \textit{neistinite}. Naprotiv, on vjeruje da te ilustracije \others{ilustriraju istinu da sva punina Onoga koji ispunjava sve u svemu, prebiva u svakoj osobi Trojedinog Boga}. Reći da se Ellen White složila s takvim sentimentom je teška pogrešna interpretacija.

Kontekst ovog važnog citata postavlja važna pitanja. Zašto Božji prorok naziva prikaze koji ilustriraju \others{tri-ličnosti jednoga Boga} \egwinline{spiritualističkim prikazima}, koji ilustriraju sentiment koji \egwinline{ne smije se vjerovati}? Ili zašto Božji prorok naziva prikaze koji \others{predstavljaju živuće ličnosti živoga Boga} \egwinline{spiritualističkim prikazima}? Ili zašto Božji prorok, kada se referira na prikaze koji \others{ilustriraju istinu da sva punina Onoga koji ispunjava sve u svemu, prebiva u svakoj osobi Trojedinog Boga}, naziva ih \egwinline{spiritualističkim prikazima}? Svi ti spiritualistički prikazi ilustriraju sentiment koji \egwinline{ne smije se vjerovati}. Taj sentiment je jasno trinitarijanski sentiment.

Sestra White nastavlja slijediti Boardmanovu liniju razmišljanja i ispravlja pogreške.

\begin{table}[H]
\centering
\renewcommand{\arraystretch}{1.5}
\setlength{\tabcolsep}{15pt}
\begin{tabular}{|p{0.4\textwidth}|p{0.4\textwidth}|}
\hline
\multicolumn{1}{|c|}{\textbf{William Boardman}} & \multicolumn{1}{c|}{\textbf{Ellen G. White}} \\ \hline
\othersQuote{Otac jeste punina Božanstva \textbf{nevidljivo}, \textbf{\underline{bez forme}}, koje \textbf{ni jedno stvorenje nije vidjelo \underline{niti može}}.}[str.100][https://archive.org/details/higherchristian02boargoog/page/n108/]

\othersQuote{Otac jeste sva punina Božanstva \textbf{NEVIDLJIVO}.}[str.105][https://archive.org/details/higherchristian02boargoog/page/n112/] & 
\egw{Otac jeste sva punina Božanstva \textbf{\underline{tjelesno}}, i \textbf{nevidljiv je smrtnom pogledu}.}[Ms21-1906.9; 1906][https://egwwritings.org/read?panels=p9754.15] \\ \hline
\end{tabular}
\end{table}

Za Boardmana, Otac nema formu niti tijelo i nevidljiv je svim stvorenjima. Za sestru White, Otac ima formu i tijelo i nevidljiv je samo smrtnim ljudima.\footnote{Kada sestra White govori o smrtnicima, govori o grijehom zagađenom čovječanstvu. Nakon obnove čovječanstva, pri uskrsnuću, Krist će dati svoj besmrtni život svojoj djeci. Za više informacija pročitajte \href{https://egwwritings.org/?ref=en_RH.July.5.1887.par.5}{EGW, RH July 5, 1887, par. 5; 1887}.}

Ovaj citat je jedan od najizravnijih citata koji se tiču \emcap{ličnosti Boga}. \egwinline{Otac je sva punina Božanstva \textbf{tjelesno}}[Ms21-1906.9; 1906][https://egwwritings.org/read?panels=p9754.16].

Možda će nekome biti zbunjujuće da je Otac sva punina Božanstva tjelesno jer u \textit{Kološanima 2:9}, kada se odnosi na Isusa, piše da \bible{u njemu tjelesno prebiva sva punina božanstva.} Sveto pismo ne proturječi sâmo sebi. \textit{Kološanima 2:9} ne isključuje mogućnost da Otac bude sva punina Božanstva tjelesno. Na raznim mjestima u Bibliji opisuje se da Otac ima tijelo (\textit{formu: Daniel 7:9,10; Otkrivenje 4:2,3; 1. Kraljevima 22:19-22; oblik: Ivan 5:37}). On ima izgled čovjeka (\textit{Ezekiel 1:26-28}). On ima lice (\textit{Izlazak 33:20; Matej 18:10; Otkrivenje 22:3, 4}). Međutim, Biblija potpuno šuti vezano za prirodu njegove supstance. Biblija nas uči da \bible{\textbf{Što je sakriveno, pripada GOSPODU, Bogu našemu}: \textbf{a što je \underline{otkriveno}, pripada nama i sinovima našim dovijeka}, da bismo vršili sve riječi ovoga zakona}[Ponovljeni zakon 29:29]. Otkriveno nam je da Otac ima tijelo, On je sva punina Božanstva tjelesno. Također je otkriveno da u Isusu također prebiva sva punina Božanstva tjelesno, jer \bible{svidjelo se Ocu u njemu nastaniti svu puninu}[Kološanima 1:19]. Ovo nije nikakva kontradikcija jer je Sin \bible{\textbf{otisak \underline{Njegove osobe}}}[Hebrejima 1:3].

\begin{table}[H]
\centering
\renewcommand{\arraystretch}{1.5}
\setlength{\tabcolsep}{15pt}
\begin{tabular}{|p{0.4\textwidth}|p{0.4\textwidth}|}
\hline
\multicolumn{1}{|c|}{\textbf{William Boardman}} & \multicolumn{1}{c|}{\textbf{Ellen G. White}} \\ \hline
\othersQuote{Sin jeste punina Božanstva \textbf{otjelovljeno, kako bi ga njegova stvorenja mogli vidjeti i poznati ga, i vjerovati mu}.}[str. 100][https://archive.org/details/higherchristian02boargoog/page/n108/]

\othersQuote{Sin jeste sva punina Božanstva \textbf{OČITOVANO}.}[str. 105][https://archive.org/details/higherchristian02boargoog/page/n112/] & 
\egw{Sin jeste sva punina Božanstva \textbf{očitovana}. Riječ Božja proglašava da je On ‘\textbf{savršena slika Njegove osobe}’. ‘Bog tako uzljubi svijet te dade \textbf{Sina svojega jedinorođenoga}, da svaki koji vjeruje u njega ne propadne, nego ima život vječni’. \textbf{Ovdje je prikazana \underline{ličnost Oca}}.}[Ms21-1906.10; 1906][https://egwwritings.org/read?panels=p9754.17] \\ \hline
\end{tabular}
\end{table}

Sestra White je usredotočena na \emcap{ličnost Boga}, što je zapravo ličnost Oca. U Kristu, koji je \egwinline{rođen na savršenu sliku Očeve osobe}[ST May 30, 1895, par. 3; 1895][https://egwwritings.org/read?panels=p820.12891], prikazana je Očeva ličnost. Na isti način na koji je Isus osoba, tako je i Otac. Kvaliteta ili stanje koja Krista čini osobom je ista kvaliteta ili stanje koja Oca čini osobom. Kao što je Krist osobno biće, tako je i Otac. Kao što sva punina Božanstva tjelesno prebiva u Kristu, tako i u Ocu, jer je Krist rođen na savršenu sliku Očeve osobe. U Njemu je prikazana Očeva ličnost. Ovi jednostavni zaključci potvrđeni su u Svetom Pismu u Ivan 3:16 i Hebrejima 1:3.

Da li se isti način razmišljanja o ličnosti Oca i Sina primjenjuje i na Duha Svetoga? Govoreći o Duhu Svetom, sestra White nastavlja:

\egw{\textbf{Utješitelj kojega je Krist} obećao poslati nakon što uzađe na nebo, \textbf{je Duh \underline{u} svoj punini Božanstva}, otkrivajući silu božanske milosti svima koji primaju i vjeruju u Krista kao osobnog Spasitelja.}[Ms21-1906.11; 1906][https://egwwritings.org/read?panels=p9754.18]

Sestra White pravi razliku između Oca i Sina koji \textbf{jesu}, pojedinačno, \textbf{sva} punina Božanstva, i Duha koji je \textbf{u svoj} punini Božanstva. Ovo je značajan kontrast u odnosu na razmišljanje Williama Boardmana, gdje sva trojica jesu punina Božanstva. Sestra White ne slijedi ovaj trinitarijanski način razmišljanja. Objašnjenje je jednostavno u svjetlu \emcap{ličnosti Boga} i Krista. Duh Sveti je duh, a duh prebiva \textbf{u} tijelu. Duh Sveti je \textbf{u svoj} punini Božanstva\footnote{Pogledajte citat iz \href{https://egwwritings.org/?ref=en_Ms128-1897.13&para=5426.19}{{EGW, Ms128-1897.13; 1897}}, gdje sestra White navodi da su Otac i Sin apsolutno Božanstvo.}.

Konačno, citat se nastavlja svojim najpoznatijim dijelom:

\begin{table}[H]
\centering
\renewcommand{\arraystretch}{1.5}
\setlength{\tabcolsep}{15pt}
\begin{tabular}{|p{0.4\textwidth}|p{0.4\textwidth}|}
\hline
\multicolumn{1}{|c|}{\textbf{William Boardman}} & \multicolumn{1}{c|}{\textbf{Ellen G. White}} \\ \hline
\othersQuote{\textbf{Otac} jeste sva punina Božanstva NEVIDLJIVO.}

\othersQuote{\textbf{Sin} jeste sva punina Božanstva OČITOVANO.}

\othersQuote{\textbf{Duh} jeste sva punina Božanstva što ČINI OČITOVANIM.}

\othersQuote{\textbf{Osobe} nisu puke dužnosti, ili način objave, \textbf{već žive osobe živog Boga}.}[str. 105][https://archive.org/details/higherchristian02boargoog/page/n112/] & 
\egw{Postoje \textbf{tri žive osobe nebeskog tria}; u ime ove tri velike sile—\textbf{Oca, Sina, i Svetoga Duha}—oni koji primaju Krista živom vjerom su kršteni, i ove sile će surađivati sa poslušnim podanicima neba u njihovim naporima da žive novi život u Kristu.}[Ms21-1906.11; 1906][https://egwwritings.org/read?panels=p9754.18] \\ \hline
\end{tabular}
\end{table}

U svjetlu konteksta knjige Williama Boardmana, vidimo značajan kontrast između \others{tri žive osobe \textbf{jednog živog Boga}}, što je trinitarijanski sentiment, i \egwinline{tri žive osobe \textbf{nebeskog tria}}, što je u skladu s istinom o \emcap{ličnosti Boga}.

Riječ ‘\textit{trio}’ jednostavno označava grupu od tri. \textit{“Nebeski trio”} predstavlja Otac, Sin i Duh Sveti. Ali, suprotno popularnom mišljenju, oni ne čine jednog živog Boga. Koncepti tri-u-jednom i jedan-u-tri uništavaju \emcap{ličnost Boga}. Zato je sestra White trinitarijanske sentimente nazvala sentimentima kojima \egwinline{ne smije se vjerovati}[Ms21-1906.8; 1906][https://egwwritings.org/read?panels=p9754.15].

Sestra White nikada nije slijedila nikakvu trinitarijansku teološku strukturu—niti u riječima i izrazima, niti u sentimentima. Postoji gotovo jednostavno istraživanje koje vam preporučujemo da poduzmete: u spisima Ellen White potražite standardne trinitarijanske termine poput “\textit{troje su jedno},” “\textit{jedno su troje},” “\textit{jedan u tri},” “\textit{tri u jednom},” ili bilo koju od mogućih permutacija. U njezinom impresivnom opusu nećete pronaći nijednu pojavu bilo kojeg od ovih izraza, a kamoli riječ ‘\textit{trojstvo}’ koja opisuje našeg Boga\footnote{Postoji samo jedan slučaj, u spisima Ellen White, riječi ‘\textit{trojstvo}’ koja se odnosi na \egw{požudu tijela, požudu očiju i oholost života}[Lt43-1898.25; 1898][https://egwwritings.org/read?panels=p4806.31]}. Nikada nije koristila te fraze koje su nužne za objašnjenje trinitarijanskog sentimenta. Promatrajući sljedeći citat, možemo vidjeti zašto nikada nije rekla da je Bog trojstvo.

\egw{O predmetu \textbf{\underline{spekulacije} o \underline{Božjoj ličnosti} \underline{nećemo se usuditi} izraziti}, \textbf{\underline{osim jezikom Riječi koja predstavlja Njegovu ličnost}}. O ovom pitanju ne smije biti rasprave \textbf{da Bog ne bi dao nedvojbeno otkrivenje o \underline{tome što On jest}} koje bi uništilo onoga tko se usuđuje stupiti na sveto tlo sa \textbf{svojim spekulativnim teorijama}, kao što su se neki usudili učiniti otvarajući kovčeg da vide što je u njemu kao njegova moć i kako se Bog očitovao. Ljudi su bili ubijeni zbog svoje znatiželje znanosti.}[17LtMs, Ms 223, 1902, par. 16][https://egwwritings.org/read?panels=p14067.9124037&index=0]

Jeste li primijetili? Ne smije biti rasprave o pitanju što je Bog, osim ukoliko \egwinline{Bog ne bi dao nedvojbeno otkrivenje} o \egwinline{tome što On jest}. Da bismo rekli “Bog je \_\_\_\_\_\_\_“, praznina mora biti ispunjena \egwinline{jezikom Riječi koja predstavlja Njegovu ličnost.} Biblija jasno uči da je Bog osobno, duhovno biće—istina koju je Krist sam potvrdio u svojim otkrivenjem Ellen White. To se uklapa u biblijski jezik koji opisuje Božju ličnost. Međutim, prema gore navedenom iskazu, možemo li reći “\textit{Bog je trojstvo}?” Ne! To nije izraženo \egwinline{jezikom Riječi koja predstavlja Njegovu ličnost.} Stoga, u istraženom kontekstu, možemo sa sigurnošću zaključiti da trinitarijanski pogled na Boga je dio \egwinline{spekulativnih teorija} o \egwinline{tome što On jest}.

To rečeno, izraz \egwinline{Nebeski Trio} nije definicija onoga što Bog jest. Naš Bog je Otac—ne \egwinline{Nebeski Trio.} Pojam Nebeski Trio ne služi kao zamjena za trinitarijansku ideju o \textit{tri žive osobe jednog Boga}. To postaje očito kada ispitamo kontekst. Ellen White je bila upućena da nas upozori protiv trinitarijanskih sentimenta, a ne da im vjerujemo. Ona ih nije podržavala.

Iako ilustracije koje je Ellen White citirala nisu bile od dr. Kellogga, čini se da su Kelloggovi zagovornici, ako ne i sam Kellogg, branili ga Boardmanovim sentimentima. Nemamo izravne podatke koji bi to potvrdili, ali znamo da je dr. Kellogg pokrenuo \others{teološku stranu pitanja o \textbf{trojstvu i svim takvim stvarima}.}[Intervju, J. H. Kellogg, G. W. Amadon i A. C. Bourdeau, 7. listopada 1907. održan u Kelloggovoj rezidenciji][https://archive.org/details/KelloggVs.TheBrethrenHisLastInterviewAsAnAdventistoct71907/page/n37] Posljednja tri odlomka u rukopisu o nebeskom triu \href{https://egwwritings.org/?ref=en_Ms21-1906&para=9754.1}{(Ms21-1906; 1906)} otkrivaju povezanost s dr. Kelloggom. Ovdje je još jednom Dr. Kellog "uhvaćen u djelu" u svojim trinitarijanskim stavima.

\egw{Ovo pišem jer svaki trenutak moj život može završiti. \textbf{Ako se ne prekine utjecaj koji je Sotona pripremio i ne \underline{ožive svjedočanstva koja je Bog dao, duše će propasti u svojoj zabludi}. Prihvaćat će zabludu za zabludom i tako održavati nesklad koji će uvijek postojati dok oni koji su bili zavedeni ne \underline{stanu na ispravnu platformu}}. Sva ta viša obrazovanja koja se planira bit će ugašena; jer je lažna. Što je jednostavnije obrazovanje naših radnika, što manje imaju veze s ljudima koje Bog ne vodi, to će se više postići. \textbf{Rad će se obavljati u \underline{jednostavnosti} prave pobožnosti, i stara, stara vremena će se vratiti kada su, pod vodstvom Duha Svetoga, tisuće bile obraćene u jednom danu. Kada se istina u svojoj jednostavnosti živi na svakom mjestu, tada će Bog djelovati kroz svoje anđele kao što je djelovao na dan Pedesetnice, i srca će se tako odlučno mijenjati da će biti očit utjecaj istinske istine, kako je prikazano u silasku Duha Svetoga}.}[Ms21-1906.18; 1906][https://egwwritings.org/read?panels=p9754.25]

\egwnogap{Duh Sveti nikada nije i nikada neće u budućnosti odvojiti zdravstveno misijski rad od službe evanđelja. Oni se ne mogu odvojiti. Povezani s Isusom Kristom, služba Riječi i iscjeljivanje bolesnih su jedno.}[Ms21-1906.19; 1906][https://egwwritings.org/read?panels=p9754.26]

\egwnogap{Pedeset i osmo poglavlje Izaije sadrži upute za danas. \textbf{‘Viči iz sveg grla, ne štedi, podigni svoj glas kao trubu, i pokaži mom narodu njihove prijestupe, i domu Jakovovu njihove grijehe.’ Bog ne prihvaća \underline{dr. Kellogga kao svog radnika}, osim ako sada ne prekine sa Sotonom}. Djelo ne bi bilo ometano, kao što je bilo proteklih nekoliko godina, \textbf{da je dr. Kellogg bio obraćen čovjek. ‘Dođite,’ pozivam, ‘dođite i odvojite se od njega i njegovih suradnika nad kojima ima utjecaj.’ Sada prenosim poruku koju mi je Bog dao, da dam svima koji tvrde da vjeruju u istinu, \underline{‘Izađite iz njih i budite odvojeni},’ inače će njihov grijeh u opravdavanju pogrešaka i smišljanju prijevara nastaviti biti propast duša. Ne možemo si priuštiti biti na pogrešnoj strani. Ne možemo si priuštiti pokrivati istinu znanstvenim problemima. Potičemo da se naprave odlučne promjene i da se više ne postavljaju spoticanja pred noge Božjeg naroda}. Neka svaka duša obuje cipele evanđelja. \textbf{Neka svaka duša moli i radi, stavljajući svoje noge \underline{na temelj koji je Krist postavio} dajući svoj život za život svijeta}.}[Ms21-1906.20; 1906][https://egwwritings.org/read?panels=p9754.27]

Citat o nebeskom triu bio je dio Kelloggove kontroverze. To je dokaz da je Kelloggova kontroverza uključivala doktrinu o Trojstvu. Rečeno nam je da se odvojimo \egwinline{od utjecaja Sotone} i oživimo \egw{svjedočanstvo koje nam je Bog dao}, inače će naše duše propasti u zabludama. Ti utjecaji i zablude dolaze od trinitarijanaca poput \textit{Williama Boardmana} i \textit{dr. Johna H. Kellogga}. Ona nas upućuje da ponovno postavimo svoje noge na temelj koji je izgradio Najveći Radnik.\footnote{\href{https://egwwritings.org/?ref=en_SpTB02.54.2&para=417.276}{EGW, SpTB02 54.2; 1904}}

Nadamo se da ovaj kontekst razotkriva lažni narativ koji šire naši adventistički teolozi da je Ellen White podržavala doktrinu o Trojstvu. Dr. Kellogg je bio u otpadu jer je odstupio s temelja naše vjere, a doktrina o Trojstvu bila je njegovo opravdanje. S takvim podacima na umu, mora se postaviti pitanje: Ako je Trojstvo bilo istinito, i Ellen White ga je podržavala, i to “istinito” Trojstvo bilo je pomiješano s Kelloggovom zabludom, zar ne bismo očekivali da ona odvoji Trojstvo od zablude. Ali to nije ono što je učinila. Umjesto toga, uporno nas je upućivala natrag na temelj naše vjere, gdje smo imali jasno učenje o prisutnosti i \emcap{ličnosti Boga}. Ali u slučaju Trojstva, vjerno je prenjela uputu s Neba: “\textit{\textbf{Upućena sam da kažem}, sentimentima onih koji traže \textbf{trinitarijanske ideje ne smije se vjerovati}}.”

% The Heavenly Trio

\begin{titledpoem}

    \stanza{
        Ličnost Boga, istina je sveta, \\
        Ellen je branila kroz mnoga ljeta. \\
        Protiv trojstva je stala, \\
        Istinu o Bogu jasno je znala.
    }

    \stanza{
        Mnogi su krivo shvatili njezin glas, \\
        Ali istina stoji, čeka na nas. \\
        Ne trojedini Bog, već nebeski trio, \\
        Tri žive osobe, svaka svoj dio.
    }

    \stanza{
        Otac nije bezobličan, već pun i sjajan, \\
        Smrtniku nevidljiv, ali moćan i trajan. \\
        Sin je punina Božanstva očitovana, \\
        U Njemu je ličnost Oca prikazana.
    }

    \stanza{
        Duh je u punini Božanstva darovan, \\
        Kao utješitelj, od Boga poslan. \\
        Ne jedan u tri, niti tri u jednom, \\
        Već tri osobe u krištenju vrijednom.
    }

    \stanza{
        Trinitarijanske ideje, spiritualističke i prazne, \\
        Ellen ih odbacuje kao neistinite i lažne. \\
        Sentimentima trinitarijanskim ne vjeruj, \\
        Već za istinu o ličnosti Boga revnuj.
    }
    
\end{titledpoem}
