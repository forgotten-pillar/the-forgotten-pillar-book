\chapter{Dr. Kellogg and Ellen White writings}


\chapter{Dr. Kellogg i spisi Ellen White}


Dr. Kellogg asserted that in the Living Temple he represented the same sentiments advocated by Sister White. Likewise, today many claim that Sister White was trinitarian and is responsible for the church's acceptance of the Trinity doctrine\footnote{William Johnsson, Adventist Review, January 6th, 1994, ‘\textit{Present Truth –Walking in God’s Light}’}. Sister White, herself, declared such claims to be false.


Dr. Kellogg je tvrdio da je u Živom Hramu zastupao ista stajališta koja je i sestra White zagovarala. Također, danas mnogi tvrde da je sestra White bila trinitarijanac i da je ona ima zasluge za prihvaćanje doktrine o Trojstvu u našoj crkvi\footnote{William Johnsson, Adventist Review, 6. siječnja 1994., ‘\textit{Present Truth –Walking in God's Light}’}. Sestra White je sama izjavila da su takve tvrdnje lažne.


\egw{\textbf{The enemy is seeking to \underline{bring in} among the people of God spiritualistic theories, which \underline{if accepted, would undermine the foundation of the faith} that has made us what we are}. He leads men to present fables clothed with Scripture. \textbf{There are those who assert that Sister White’s writings are in harmony with these teachings}.\textbf{ \underline{I declare this to be false}. Men may misapply Scripture; they may misinterpret my words; but God understands their devising}. How thankful I am for this! When the enemy comes in like a flood, \textbf{the Spirit of the Lord will lift up a standard for us against him}.}[Ms137-1903.21; 1903][https://egwwritings.org/?ref=en\_Ms137-1903.21&para=9939.30]


\egw{\textbf{Neprijatelj nastoji \underline{unijeti} među Božji narod spiritualističke teorije koje, \underline{ako budu prihvaćene, bi potkopale temelje vjere} koja nas je učinila onim što jesmo}. On vodi ljude da predstavljaju bajke obučene u Pismo. \textbf{Postoje oni koji tvrde da su spisi sestre White u skladu s ovim učenjima}. \textbf{\underline{Ovo izjavljujem da je to lažno}. Ljudi mogu pogrešno primijeniti Pismo; mogu pogrešno tumačiti moje riječi; ali Bog razumije njihovo smišljanje}. Kako sam zahvalna za ovo! Kada neprijatelj navali kao poplava, \textbf{Duh Gospodnji će za nas podići standard protiv njega}.}[Ms137-1903.21; 1903][https://egwwritings.org/?ref=en\_Ms137-1903.21&para=9939.30]


Dr. Kellogg advocated the theories that, if accepted, would undermine the foundation of our faith. It is crucial to correctly understand what constitutes the foundation of our faith, which Sister White referred to. We have seen that it refers to the \emcap{Fundamental Principles}. Looking at her writings, and the writings of our pioneers, we see that the Trinity doctrine contradicts the \emcap{personality of God} and the truth about God’s presence. Today, with the Trinity doctrine as part of our belief, we recognize that we have moved away from the \emcap{Fundamental Principles} and formed another foundation. Sister White was not responsible for this transition. It is purely a misinterpretation of her works. Her writings do not undermine the foundation of the faith that has made us what we are. Her later work is completely in harmony with the truth given in the beginning.


Dr. Kellogg je zagovarao teorije koje, ako bi se prihvatile, bi potkopale temelj naše vjere. Ključno je pravilno razumjeti što čini temelj naše vjere, na što se sestra White referirala. Vidjeli smo da se to odnosi na \emcap{Fundamentalne Principe}. Gledajući njezine spise i spise naših pionira, vidimo da doktrina o Trojstvu proturječi \emcap{ličnosti Boga} i istini o Božjoj prisutnosti. Danas, s doktrinom o Trojstvu kao dijelom našeg vjerovanja, prepoznajemo da smo se udaljili od \emcap{Fundamentalnih Principa} i formirali drugi temelj. Sestra White nije bila odgovorna za ovu tranziciju. To je čisto pogrešno tumačenje njezinih djela. Njezini spisi ne potkopavaju temelj vjere koji nas je učinio onim što jesmo. Njezin kasniji rad je potpuno u skladu s istinom danoj na početku.


\egw{\textbf{The past fifty years have not dimmed one jot or principle of our faith as we received the great and wonderful evidences that were made certain to us in 1844, after the passing of the time.} ... \textbf{\underline{Not a word is changed or denied}. That which the Holy Spirit testified to as truth after the passing of the time, in our great disappointment, \underline{is the solid foundation of truth}. \underline{Pillars of truth were revealed}, and we accepted \underline{the foundation principles} that have made us what we are—Seventh-day Adventists, keeping the commandments of God and having the faith of Jesus.}}[Lt326-1905.3; 1905][https://egwwritings.org/?ref=en\_Lt326-1905.3&para=7678.9]


\egw{\textbf{Posljednjih pedeset godina nije izblijedilo ni jednu jotu ili princip naše vjere kao što smo primili velike i čudesne dokaze koji su nam učinjeni sigurnim 1844, nakon proteka vremena}... \textbf{\underline{Ni jedna riječ nije promijenjena ili poreknuta}. Ono za što je Duh Sveti svjedočio da je istina nakon proteka vremena, u našem velikom razočarenju, \underline{je čvrsti temelj istine}. \underline{Stupovi istine su bili otkriveni}, i mi smo prihvatili \underline{temeljne principe} koji su nas učinili onim što jesmo—Adventisti Sedmog dana, držeći zapovijedi Božje i imajući vjeru Isusovu.}}[Lt326-1905.3; 1905][https://egwwritings.org/?ref=en\_Lt326-1905.3&para=7678.9]


\section*{Misrepresentation of the church standpoint}


\section*{Pogrešno predstavljanje crkvenog stajališta}


By misrepresenting Sister White’s writings, Dr. Kellogg did not only misrepresent her work, but the church’s official standpoint expressed in the \emcap{Fundamental Principles}. Ellen White rebuked Kellogg for misrepresenting the church’s standpoint. As we read this rebuke, let us keep in mind the church’s current standpoint on the \emcap{personality of God} as it compares to the first point of the \emcap{Fundamental Principles}.


Dr. Kellogg, krivo predstavljajući spise Sestre White, nije samo krivo predstavio njezin rad, već i službeno stajalište crkve izraženo u \emcap{Fundamentalnim Principima}. Ellen White je ukorila Kellogga zbog krivog predstavljanja stajališta crkve. Dok čitamo ovaj ukor, imajmo na umu trenutačno stajalište crkve o \emcap{ličnosti Boga} u usporedbi s prvom točkom \emcap{Fundamentalnih Principa}.


\egw{You \textbf{are not sound in the truth}. Your statements made to believers and unbelievers \textbf{misrepresent us as a people who have not changed the truth for error}. They detract from the influence \textbf{God would have us possess before the world in revealing in plain, unmistakable language that we are \underline{true to the principles of our faith} and that we hold the beginning of our confidence firm unto the end}. We are strictly denominational. \textbf{We believe in 1903 the same truths we did believe when we established the Sanitarium and the College in Battle Creek, and \underline{we know that we had no ifs or ands about this matter}}.}[Lt300-1903.4; 1903][https://egwwritings.org/?ref=en\_Lt300-1903.4&para=7705.10]


\egw{Ti \textbf{nisi čvrst u istini}. Tvoje izjave upućene vjernicima i nevjernicima \textbf{krivo nas predstavljaju kao narod koji nije zamijenio istinu za zabludom}. One umanjuju utjecaj \textbf{koji bi Bog želio da imamo pred svijetom, otkrivajući jasnim, nedvosmislenim jezikom da smo \underline{vjerni principima naše vjere} i da držimo početak našeg vjere čvrstim do kraja}. Mi smo strogo denominacijski. \textbf{Vjerujemo u 1903. iste istine koje smo vjerovali kada smo osnovali Sanatorij i školu u Battle Creeku, i \underline{znamo da u ovoj stvari nismo imali nikakvih ‘ako’ ili ‘ali’}}.}[Lt300-1903.4; 1903][https://egwwritings.org/?ref=en\_Lt300-1903.4&para=7705.10]


\egwnogap{While you have told the things that you have and made the statements you have before unbelievers, my heart has been sad indeed. \textbf{You have evidenced that you have departed from the faith}. The very statements you have made before worldly men of influence, as the papers have reported your words, have been presented to me distinctly from your lips as you have spoken them. We cannot labor to give you influence as one whom we can trust with the sacred work connected with our institutions, for you need first to be converted and led.}[Lt300-1903.5; 1903][https://egwwritings.org/?ref=en\_Lt300-1903.5&para=7705.11]


\egwnogap{Dok si govorio stvari koje si govorio i davao svoje izjave koje pred nevjernicima, moje je srce doista bilo tužno. \textbf{Dokazao si da si odstupio od vjere}. Same izjave koje si dao pred svjetovnim utjecajnim ljudima, kako su novine prenijele tvoje riječi, prezentirane su mi jasno s tvojih usana dok si ih govorio. Ne možemo se truditi da ti damo utjecaj kao nekome kome možemo povjeriti sveti posao povezan s našim institucijama, jer se najprije trebaš obratiti i biti vođen.}[Lt300-1903.5; 1903][https://egwwritings.org/?ref=en\_Lt300-1903.5&para=7705.11]


\egwnogap{You are not sound in the faith. I have stated this in my diary months ago. \textbf{You have certainly placed the people of God, whom the Lord has led step by step in the ways of truth and placed upon \underline{a solid foundation}, in a false showing before unbelievers. Some have departed from the faith and \underline{will continue to misrepresent the work God has given me}}.}[Lt300-1903.6; 1903][https://egwwritings.org/?ref=en\_Lt300-1903.6&para=7705.12]


\egwnogap{Nisi čvrst u vjeri. To sam zapisala u svom dnevniku prije nekoliko mjeseci. \textbf{Svakako si postavio narod Božji, kojeg je Gospodin korak po korak vodio putevima istine i postavio na \underline{čvrst temelj}, u lažno svjetlo pred nevjernicima. Neki su se odmaknuli od vjere i \underline{nastavit će krivo predstavljati rad koji mi je Bog dao}}.}[Lt300-1903.6; 1903][https://egwwritings.org/?ref=en\_Lt300-1903.6&para=7705.12]


\egwnogap{\textbf{The sanctuary question is a clear and definite doctrine as we have held it as a people. \underline{You are not definitely clear on the personality of God, which is everything to us as a people}. \underline{You have virtually destroyed the Lord God Himself}}.}[Lt300-1903.7; 1903][https://egwwritings.org/?ref=en\_Lt300-1903.7&para=7705.13]


\egwnogap{Pitanje svetišta je jasna i određena doktrina kakvu smo držali kao narod. \underline{Nisi definitivno jasan po pitanju ličnosti Boga, što je za nas kao narod sve}. \underline{U suštini si uništio samog Gospodina Boga}.}[Lt300-1903.7; 1903][https://egwwritings.org/?ref=en\_Lt300-1903.7&para=7705.13]


\egwnogap{Why should you take the liberty to make the statements which you have made, as though you had authority for thus stating, when they are falsehoods? \textbf{You have made the facts of our faith of none effect before unbelievers,} \textbf{and the truth which should ever be kept prominent and exalted with this people you have virtually denied and ignored in your many statements. How dared you to do this?} \textbf{It necessitates us now to present our true position which constitutes us Seventh-day Adventists}. Whatever influence God has given you in the past has been in mercy to you, letting the light shine upon you.}[Lt300-1903.8; 1903][https://egwwritings.org/?ref=en\_Lt300-1903.8&para=7705.14]


\egwnogap{Zašto si uzeo slobodu iznositi izjave koje si iznio, kao da imaš ovlasti tako govoriti, kada su to neistine? \textbf{Učinio si činjenice naše vjere nevažnima pred nevjernicima}, \textbf{i istinu koja bi uvijek trebala biti istaknuta i uzdignuta kod ovog naroda, ti si u suštini zanijekao i ignorirao u svojim mnogim izjavama. Kako si se usudio to učiniti?} \textbf{Sada nas to tjera da predstavimo naš pravi položaj koji nas čini Adventistima Sedmog Dana}. Kakav god utjecaj ti je Bog dao u prošlosti, bio je to čin milosti prema tebi, dopuštajući da svjetlost sja na tebe.}[Lt300-1903.8; 1903][https://egwwritings.org/?ref=en\_Lt300-1903.8&para=7705.14]


\egwnogap{\textbf{We cannot for a moment have any misrepresentation upon these solemn and important subjects of truth which have been the faith of our people since 1844. This means much to us.} The Lord would have me say to you that the enemy has, through his specious deceptions, placed his unbelief in your mind, and you have been working it out. \textbf{\underline{All who receive your presentations will enter upon strange paths if they connect with you}}. \textbf{You are \underline{bringing in} strange, common fire}, \textbf{but not the fire of God’s own kindling}; and now \textbf{I must speak plainly to our people that the Lord has led us step by step and shown us clear light upon the heavenly sanctuary in the most holy of holies where \underline{God revealed Himself} to His appointed ones.}}[Lt300-1903.9; 1903][https://egwwritings.org/?ref=en\_Lt300-1903.9&para=7705.15]


\egwnogap{\textbf{Ne možemo ni trenutka dopustiti bilo kakvo krivo predstavljanje ovih svečanih i važnih predmeta istine koji su bili vjera našeg naroda od 1844. To nam puno znači.} Gospodin bi htio da ti kažem da je neprijatelj, kroz svoje lukave obmane, postavio svoje nevjerovanje u tvoj um, i ti si to provodio. \textbf{\underline{Svi koji prihvate tvoje prezentacije krenut će na čudne staze ako se povežu s tobom}}. \textbf{\underline{Unosiš} čudnu, običnu vatru}, \textbf{ali ne vatru koju je Bog sam zapalio}; i sada \textbf{moram jasno govoriti našem narodu da nas je Gospodin korak po korak vodio i pokazao nam jasno svjetlo o nebeskom svetištu u Svetinji nad Svetinjama gdje se \underline{Bog otkrio} svojim odabranima}.}[Lt300-1903.9; 1903][https://egwwritings.org/?ref=en\_Lt300-1903.9&para=7705.15]


Dr. Kellogg misrepresented the truth that constituted the foundation of our faith; most specifically, he misrepresented the truth on the \emcap{personality of God}, which was everything to us as people. If in 1903, it necessitated \egwinline{\textbf{to present our true position which constitutes us Seventh-day Adventists}}, how much more is it important for us today? Sister White did her part in upholding the foundation of our faith in the beginning, but it seems like we have forgotten.


Dr. Kellogg je krivo predstavio istinu koja je činila temelj naše vjere; posebno je krivo predstavio istinu o ličnosti Boga, što je za nas kao narod bilo sve. Ako je 1903. bilo potrebno \egwinline{\textbf{predstaviti “naš pravi položaj koji nas čini Adventistima Sedmog Dana”}}, koliko je to važnije za nas danas? Sestra White je učinila svoj dio u očuvanju temelja naše vjere na početku, ali čini se da smo zaboravili.
