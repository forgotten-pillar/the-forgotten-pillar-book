\chapter{Heaven's reality}


\chapter{Nebeska stvarnost}


The \emcap{personality of God} deals with the quality or state of God being a person. Whenever we look at the pioneer's work on the \emcap{personality of God}, we see that they were all in harmony with the view that God is a tangible \textit{being}, possessing both body and parts. We always see the same underlying reasoning, which differentiates the term ‘\textit{spirit}’ and the term ‘\textit{being}’. By differentiating these terms, they explain the quality or state of God being a person\footnote{\href{https://www.merriam-webster.com/dictionary/personality}{Merriam-Webster Dictionary} defines the word ‘\textit{personality}’ as “\textit{quality or state of being a person}”.}—a \emcap{personality of God}. All their conclusions are summed up in the first point of the \emcap{Fundamental Principles}. \others{There is \textbf{one God}, a \textbf{personal}, \textbf{spiritual being}, the Creator of all things, omnipotent, omniscient, … and \textbf{every-where present by his representative, the Holy Spirit}. Psalm 139:7.}[FPSDA 1.2][https://egwwritings.org/?ref=en\_FPSDA.1.2&para=1299.6]


\emcap{Ličnost Boga} se bavi kvalitetom ili stanjem koja Boga čine osobom. Kad god pogledamo rad pionira o \emcap{ličnosti Boga}, vidimo da su svi bili u skladu s pogledom da je Bog opipljivo \textit{biće}, koje posjeduje i tijelo i dijelove. Uvijek vidimo isto temeljno rasuđivanje, koje razlikuje pojam '\textit{duh}' i pojam '\textit{biće}'. Razlikujući ove pojmove, oni objašnjavaju kvalitetu ili stanje koja Boga čine osobom\footnote{\href{https://www.merriam-webster.com/dictionary/personality}{Merriam-Webster Dictionary} definira riječ '\textit{personality}' kao "\textit{kvaliteta ili stanje koja nekog čine osobom}".}—\emcap{ličnost Boga}. Svi njihovi zaključci su sažeti u prvoj točki \emcap{Fundamentalnih Principa}. \others{Postoji \textbf{jedan Bog}, \textbf{osobno}, \textbf{duhovno biće}, Stvoritelj svega, svemoguć, sveznajući, … i \textbf{svugdje prisutan preko svog predstavnika, Svetog Duha}. Psalam 139:7.}[FPSDA 1.2][https://egwwritings.org/?ref=en\_FPSDA.1.2&para=1299.6]


So far, in the pioneers’ work, we have seen that the \emcap{personality of God} is tightly connected to the reality of God’s presence. God is a personal spiritual being, having a body and shape; as such, His presence is cumbered to one locality—as the Bible says, in His temple, at His throne where He is surrounded with unapproachable glory. But He is everywhere present by His representative, the Holy Spirit. Obviously, the Holy Spirit is a spirit, and not a being, \bible{for a spirit hath not flesh and bones as ye see me have}, said Jesus (Luke 24:39). Christ is also a Being, like His Father. He is an express image of the Father’s person; therefore, He bears the same personality, or quality or state of being a person, as His Father does.


Do sada smo u radu pionira vidjeli da je \emcap{ličnost Boga} usko povezana sa stvarnošću Božje prisutnosti. Bog je osobno duhovno biće, koje ima tijelo i oblik; kao takav, Njegova prisutnost je ograničena na jedno mjesto—kao što Biblija kaže, u Njegovom hramu, na Njegovom prijestolju gdje je okružen nepristupačnom slavom. Ali On je svugdje prisutan preko svog predstavnika, Svetog Duha. Očito, Sveti Duh je duh, a ne biće, \bible{jer duh nema mesa i kostiju kao što vidite da ja imam}, rekao je Isus (Luka 24:39). Krist je također Biće, poput svog Oca. On je savršena slika Očeve osobe; stoga, On nosi istu ličnost, ili kvalitetu ili stanje koja nekog čine osobom, kao i Njegov Otac.


In our experience, when we present the original Seventh-day Adventist beliefs on the \emcap{personality of God} to our trinitarian brothers, as expressed in the first two points of the \emcap{Fundamental Principles}, they often claim that the statements in the \emcap{Fundamental Principles} are correct in some way, but the understanding attributed to the terms “\textit{personal spiritual being}” are false. They usually attempt to harmonize the \emcap{Fundamental Principles} with the Trinity doctrine by twisting the words “\textit{spiritual being}”, as if the word ‘\textit{spiritual}’ means something mysterious, suitable to equalize the \emcap{personality of God} and of Christ with the personality of the Holy Ghost. The underlying problem comes down to the understanding of the heavenly realities. The Bible is not silent about heaven and its realities, and our pioneers understood it well. Below we read about the explanation of the terms “\textit{spiritual being}” from James White and Uriah Smith in their book, “\textit{The Biblical Institute}”. The Bible explains these terms using the example of angels, which are “\textit{spiritual beings}”.


U našem iskustvu, kada predstavljamo izvorna adventistička vjerovanja o \emcap{ličnosti Boga} našoj trinitarijanskoj braći, kako su izražena u prve dvije točke \emcap{Fundamentalnih Principa}, oni često tvrde da su izjave u \emcap{Fundamentalnim Principima} na neki način ispravne, ali da je razumijevanje pripisano pojmovima "\textit{osobno duhovno biće}" pogrešno. Obično pokušavaju uskladiti \emcap{Fundamentalne Principe} s doktrinom o Trojstvu iskrivljavanjem riječi "\textit{duhovno biće}", kao da riječ '\textit{duhovno}' znači nešto tajanstveno, prikladno za izjednačavanje \emcap{ličnosti Boga} i Krista s ličnošću Svetog Duha. Temeljni problem svodi se na razumijevanje nebeskih stvarnosti. Biblija nije šutljiva o nebu i njegovim stvarnostima, a naši pioniri su to dobro razumjeli. U nastavku čitamo objašnjenje pojmova "\textit{duhovno biće}" od Jamesa Whitea i Uriaha Smitha u njihovoj knjizi "\textit{The Biblical Institute}". Biblija objašnjava ove pojmove koristeći primjer anđela, koji su "\textit{duhovna bića}".


\others{\textbf{Angels are real beings}. They are described in the Bible as \textbf{possessing face, feet, wings} \&x. Ezekiel says of the cherubim, \textbf{‘Their whole \underline{body} and their backs and their hands and their wings},’ \&c. Eze. 10:12. Angels \textbf{appeared }unto Abraham. Gen. 18:1-8. They talked and ate with him. They went on to Sodom and communed with Lot, who, entering into his house baked unleavened bread for them and they did eat. \textbf{These person were called angels}. David speaks of the manna as the corn of Heaven and angel’s food. Ps. 78:23-25.}


\others{\textbf{Anđeli su stvarna bića}. Oni su opisani u Bibliji kao \textbf{bića koja posjeduju lice, noge, krila} \&x. Ezekiel kaže o kerubinima, \textbf{'Cijelo njihovo \underline{tijelo} i njihova leđa i njihove ruke i njihova krila},' \&c. Ez. 10:12. Anđeli su se \textbf{pojavili} Abrahamu. Post. 18:1-8. Razgovarali su i jeli s njim. Otišli su u Sodomu i družili se s Lotom, koji je, ušavši u svoju kuću, ispekao beskvasni kruh za njih i oni su jeli. \textbf{Te osobe su nazvane anđelima}. David govori o mani kao o žitu nebeskom i anđeoskoj hrani. Ps. 78:23-25.}


\othersnogap{The case of Balaam, Num. 22:22-31, is an interesting incident.The angel \textbf{appeared }to Balaam with a sword \textbf{drawn in his hand}. The question is sometimes asked \textbf{how angels can be \underline{material beings since we cannot see them}. This case illustrates it}. The record says the \textbf{Lord opened the eyes of Balaam and he saw the angel}. \textbf{The angel did not create a body for that occasion}.\textbf{ He was just the same as he was before Balaam saw him; \underline{but the change took place in Balaam}. His eyes were opened, then he beheld the angel}. It was the same with the servant of Elisha when he and his master were brought into a straight place, surrounded by the army of the king of Syria. 2 Kings 6:17. Elisha prayed that \textbf{the eyes of his servant might be opened}; and he immediately saw the whole mountain full of horses and chariots round about Elisha.}


\othersnogap{Slučaj Bileama, Br. 22:22-31, je zanimljiv događaj. Anđeo se \textbf{pojavio} Bileamu s \textbf{isukanim mačem u ruci}. Ponekad se postavlja pitanje \textbf{kako anđeli mogu biti \underline{materijalna bića kad ih ne možemo vidjeti}. Ovaj slučaj to ilustrira}. Zapis kaže da je \textbf{Gospodin otvorio Bileamove oči i on je vidio anđela}. \textbf{Anđeo nije stvorio tijelo za tu priliku}. \textbf{Bio je isti kao i prije nego što ga je Bileam vidio; \underline{ali promjena se dogodila u Bileamu}}. Njegove oči su bile otvorene, tada je ugledao anđela. Isto je bilo sa slugom Elišinim kada su on i njegov gospodar bili dovedeni u težak položaj, okruženi vojskom sirijskog kralja. 2 Kr 6:17. Eliša se molio da \textbf{se otvore oči njegovom sluzi}; i on je odmah vidio cijelu goru punu konja i kola oko Eliše.}


\othersnogap{\textbf{This may be further illustrated referring to things which we know are material and yet which we cannot see}. Air is material, light is material, even thought itself is only the result of material organizations — matter acting upon matter — and yet we can see none of these things. \textbf{Just so with the angels}.}


\othersnogap{\textbf{Ovo se može dodatno ilustrirati pozivanjem na stvari za koje znamo da su materijalne, a ipak ih ne možemo vidjeti}. Zrak je materijalan, svjetlost je materijalna, čak je i sama misao samo rezultat materijalnih organizacija — materija koja djeluje na materiju — a ipak ne možemo vidjeti ništa od toga. \textbf{Isto je tako s anđelima}.}


\othersnogap{\textbf{It is further objected to the materiality of the angels that they are called spirits. }Heb. 1:13, 14.\textbf{\underline{But this is no objection to their being literal beings}}. \textbf{They are simply spiritual beings organized differently from these earthly bodies which we possess}. Paul says, 1 Cor. 15:44, ‘\textbf{There is a natural body and there is \underline{a spiritual body}}.’ \textbf{The natural body we now have; the spiritual body we shall have in the resurrection}. ‘\textbf{It is raised a spiritual body}.’ Verse 44. \textbf{But then we are equal unto the angels}, Luke 20:36; \textbf{then we have bodies like unto Christ’s most glorious body}. Phil. 3:4 \textbf{and Christ is no less a spirit than the angels}. \textbf{We read that God is a spirit, that is, simply \underline{a spiritual being}}.}[James White and Uriah Smith, The Biblical Institute (Kindle Locations 2537-2553). Kindle Edition.]


\othersnogap{\textbf{Nadalje se prigovara materijalnosti anđela da su nazvani duhovima}. Heb. 1:13, 14. \textbf{\underline{Ali to nije prigovor njihovom postojanju kao doslovnih bića}}. \textbf{Oni su jednostavno duhovna bića organizirana drugačije od ovih zemaljskih tijela koja mi posjedujemo}. Pavao kaže, 1 Kor. 15:44, '\textbf{Postoji prirodno tijelo i postoji \underline{duhovno tijelo}}.' \textbf{Prirodno tijelo sada imamo; duhovno tijelo ćemo imati u uskrsnuću}. '\textbf{Uskrsnut će kao duhovno tijelo}.' Stih 44. \textbf{Tada ćemo biti jednaki anđelima}, Luka 20:36; \textbf{tada ćemo imati tijela poput Kristovog najslavnijeg tijela}. Fil. 3:4 \textbf{i Krist nije ništa manje duh nego anđeli}. \textbf{Čitamo da je Bog duh, to jest, jednostavno \underline{duhovno biće}}.}[James White and Uriah Smith, The Biblical Institute (Kindle Locations 2537-2553). Kindle Edition.]


The Bible gives us the insight that angels are spiritual beings that possess material bodies, but are still unseen to us, unless the Lord opens our eyes to see them. When the righteous will rise up in their new glorified bodies, they will rise in a spiritual body, an incorruptible one. This body will be tangible and material just as the new Earth will be tangible and material. And with our spiritual bodies we will possess the renewed Earth, we will replenish it \bible{and subdue it: and have dominion over the fish of the sea, and over the fowl of the air, and over every living thing that moveth upon the earth}[Genesis 1:28].


Biblija nam daje uvid da su anđeli duhovna bića koja posjeduju materijalna tijela, ali su nam ipak nevidljivi, osim ako nam Gospodin ne otvori oči da ih vidimo. Kada pravednici ustanu u svojim novim proslavljenim tijelima, ustat će u duhovnom tijelu, neraspadljivom. To tijelo će biti opipljivo i materijalno baš kao što će i nova Zemlja biti opipljiva i materijalna. I s našim duhovnim tijelima posjedovat ćemo obnovljenu Zemlju, napučit ćemo je \bible{i podložiti je sebi: i vladati nad ribama morskim, i nad pticama nebeskim, i nad svim živim stvorenjima što se miču po zemlji}[Postanak 1:28].
