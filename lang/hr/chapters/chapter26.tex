\chapter{Koraci do Omege}

U našem dosadašnjem istraživanju vidjeli smo pokazatelje da je Kelloggova kontroverza bila povezana s doktrinom o Trojstvu i \emcap{ličnošću Boga} izraženom u prvoj točki \emcap{Fundamentalnih Principa}. Nažalost, danas ne stojimo na tom temelju u vezi s \emcap{ličnošću Boga}; izgradili smo drugi temelj koji je promijenio istinu o \emcap{ličnosti Boga} u misterioznog Trojedinog Boga. Sestra White je jasno bila protiv ove reorganizacije i prorekla je da će Bog u završetku svoga djela ponovno oživjeti povijest adventnog pokreta i ponovno uspostaviti svaki stup naše vjere koji je bio prihvaćen na početku.

\egw{\textbf{\underline{Gospod je izjavio kako će se prošlost morati ponovno obnoviti kako ulazimo u završno djelo}. \underline{Svaka istina} koju je On dao za ove posljednje dane ima biti objavljena svijetu. \underline{Svaki stup} koji je On utvrdio \underline{ima biti učvršćen}. Ne možemo sada sići sa temelja kojega je Bog učvrstio. Ne možemo sada ući u nikakvu novu organizaciju; jer to bi značilo otpadanje od istine}.}[Ms129-1905.6; 1905][https://egwwritings.org/read?panels=p9797.13]

Uspoređujući \emcap{Fundamentalne Principe} s trenutnim Temeljnim Vjerovanjima Adventista Sedmog Dana, vidimo da smo ušli u novu organizaciju. Božje upozorenje, dano kroz sestru White, da ponovno uspostavimo sve stupove naše vjere u ovim posljednjim danima, postaje imperativ. Dok smo pratili doktrinu o Trojstvu preko Kelloggove kontroverze, naišli smo na upozorenja Ellen White protiv alfe i omega otpada, koje će ući u našu crkvu.

\egw{\textbf{‘Živi Hram’ sadrži alfu tih teorija. Znala sam da \underline{će omega ubrzo uslijediti}; i drhtala sam za naš narod}. Znala sam da \textbf{moram upozoriti našu braću i sestre da ne ulaze u kontroverzu \underline{oko prisutnosti i ličnosti Boga}. Sentimenti u knjizi ‘Živi Hram’ \underline{vezani za tu točku su netočni}. }Sveto Pismo koje se koristilo za potkrijepljenje te doktrine, pogrešno je primijenjeno.}[SpTB02 53.2; 1904][https://egwwritings.org/read?panels=p417.271]

U kontekstu reorganizacije Adventista sedmog dana, identificiramo nekoliko koraka koji su bili nužni za ostvarenje ove reorganizacije i koji su nužni za njeno održavanje.

\subsection*{Korak 1: Negirati da su Fundamentalni Principi temelj naše vjere te službeni i točni prikaz vjerovanja adventista sedmog dana}

Prvi nužan korak k optpadništvu jeste sakriti izvorni temelj naše vjere time što ćemo ga odvojiti od \emcap{Fundamentalnih Principa}.

\egw{\textbf{Kao narod moramo \underline{čvrsto stajati na platformi vječne istine} koja je izdržala test i probu. Moramo se \underline{držati sigurnih stupova naše vjere}. \underline{Principi istine} koje nam je Bog objavio \underline{je naš jedini istiniti temelj}. Oni su nas učinili onim što jesmo. Protek vremena nije umanjilo njihovu vrijednost. \underline{Stalni je napor neprijatelja da ukloni ove istine sa njihovog mjesta}, i umjesto njih postavi \underline{lažne teorije}. On će \underline{unijeti} sve što je u mogućnosti kako bi ostvario svoje varljive dizajne.}}[SpTB02 51.2; 1904][https://egwwritings.org/read?panels=p417.261]

\egw{\textbf{Poruke svakog reda i vrste nametane su Adventistima sedmog dana, da zauzmu mjesto istine koja je, \underline{točku po točku}, bila istražena kroz molitveno proučavanje i posvjedočena čudotvornom silom Gospodnjom}. \textbf{Ali \underline{međaši} \underline{koji su nas učinili onim što jesmo}, \underline{moraju biti sačuvani}, i \underline{bit će sačuvani}, kako je Bog pokazao kroz svoju riječ i svjedočanstvo svoga Duha}. \textbf{On nas poziva da \underline{čvrsto držimo}, sa stiskom vjere, \underline{fundamentalne principe} koji su \underline{utemeljeni na neupitnom autoritetu}}.}[SpTB02 59.1; 1904][https://egwwritings.org/read?panels=p417.299]

\emcap{Fundamentalni Principi} bili su istine koje je Bog otkrio pionirima nakon proteka vremena 1844. godine. Vidjeli smo svjedočanstva naših pionira, uključujući i Ellen White, u vezi s prvom točkom \emcap{Fundamentalnih Principa}. Svi su bili u skladu s tim posebnim točkama naše vjere. Godine 1863. adventisti sedmog dana organizirali su se u crkvu, kao organizirano tijelo. Od tada, mnogi su pogrešno predstavljali stavove Crkve Adventista Sedmoga Dana i pioniri su smatrali potrebnim odgovarati na upite, \others{i ponekad ispraviti pogrešne izjave koje su kružile protiv} vjerovanja i praksi crkve. Kao rezultat toga, 1872. godine, pioniri su izdali dokument nazvan “\textit{Deklaracija Fundamentalnih Principa koje Uče i Prakticiraju Adventisti Sedmog Dana}”\footnote{“A Declaration of the Fundamental Principles, Taught and Practiced by the Seventh-Day Adventists (1872) : MVT : Free Download, Borrow, and Streaming : Internet Archive.” Internet Archive, 2025, \href{https://archive.org/details/ADeclarationOfTheFundamentalPrinciplesTaughtAndPracticedByThe}{archive.org/details/ADeclarationOfTheFundamentalPrinciplesTaughtAndPracticedByThe}. Accessed 3 Feb. 2025.}. Ova izjava je javnosti predstavila \others{kratak sažetak onoga što jest i što je bilo, s velikom jednoglasnošću, držano od}[Predgovor Fundamentalnih Principa iz 1872.] Adventista Sedmog Dana.

U poglavlju “\hyperref[chap:authority]{Autoritet Fundamentalnih Principa}” raspravljali smo o tome kako su pro-trinitarijanski teolozi ugrozili autoritet \emcap{Fundamentalnih Principa}, poričući njihovu pravu vrijednost u našoj adventističkoj povijesti.

Pro-trinitarijanski teolozi tvrde da ova izjava nije ono što tvrdi da jest—deklaracija \emcap{fundamentalnih principa}, učeno i prakticirano od strane Adventista Sedmoga Dana. Ova izjava bila je sažetak glavnih obilježja adventističke vjere, a nijedna točka nije toliko problematična ili sporna kao prva točka, koja se bavi s \emcap{ličnošću Boga} i time gdje je Njegova prisutnost. Ali dokazi u korist \emcap{Fundamentalnih Principa}, posebno prve točke, su nepobitni.

Sve te tvrdnje lako su opovrgnute činjenicom da su se \emcap{Fundamentalni Principi} redovito izdavali i ponovno tiskali tijekom cijelog života sestre White, sve do 1914. godine. Da su to bila samo privatna mišljenja nekolicine pojedinaca, kako tvrde teolozi\footnote{Ministry Magazine “Our Declaration of Fundamental Beliefs”: Sječanj 1958, Roy Anderson, J. Arthur Buckwalter, Louise Kleuser, Earl Cleveland i Walter Schubert}, bi li se dosljedno ponovno tiskali tijekom 42 godine\footnote{Za detaljan popis publikacija tijekom ovih godina, pogledajte Dodatak.}, javno označavajući sažetak vjere adventista sedmog dana? Da su izdani samo jednom, mogli bismo smatrati da je to zavjera nekih pojedinaca kako bi namjerno pogrešno predstavili vjeru adventista sedmog dana. Naprotiv, \emcap{Fundamentalni Principi} su se redovito iznova tiskali što dokazuje da su predstavljali službenu vjeru i praksu adventista sedmog dana.

Drugi važan argument je taj da je sestra White odobrila \emcap{Fundamentalne Principe} u svojim spisima, izričito se pozivajući na njih, ali i poučavala te iste istine koje su sadržane u \emcap{Fundamentalnim Principima}. Spisi naših pionira također su u skladu s izjavama u ovoj Deklaraciji \emcap{Fundamentalnih Principa}. Uzimajući u obzir sve te činjenice, neizbježno je da je ova izjava bila istinita u svojim tvrdnjama. Ovaj dokument doista je deklaracija \emcap{fundamentalnih principa}, učena i prakticirana od adventista sedmog dana, predstavljajući javni \others{sažetak naše vjere}, \others{kratak sažetak onoga što jest i što je bilo, s velikom jednoglasnošću, držano od} Adventista Sedmog Dana.\footnote{Predgovor Fundamentalnih Principa iz 1872.} Kao takva, ona točno predstavlja vjerovanje i praksu adventista sedmog dana te predstavlja temelj vjere adventista sedmog dana u vrijeme Ellen White.

Danas, u obrani doktrine o Trojstvu, adventistički povjesničari hrabro tvrde da naši pioniri, kada su proučavali adventističke istine kao što su svetište, istražni sud, subota i druge doktrine, \others{nisu proučavali temu doktrine o Bogu}. Ovi adventistički povjesničari lažno tvrde da doktrina o Bogu \others{nije bila pitanje kojim su se bavili u to vrijeme}[Denis Kaiser. “From Antitrinitarianism to Trinitarianism: The Adventist story” and Panelist. The God We Worship: A Godhead Symposium. Central California Conference, Dinuba, CA. March 23-24, 2018.]. Slijedeći ovu lažnu tvrdnju, oni predstavljaju povijesne podatke o tome kako se adventistička doktrina postupno razvijala prema trinitarijanskom razumijevanju. Istina je da postoji nekoliko slučajeva iz ranog perioda\footnote{Najranije spominjanje doktrine Trojstva u pozitivnom smislu bilo je kada je M.C. Wilcox ponovno objavio neadventistički članak Samuela Speara u časopisu Signs of the Times 7. prosinca 1891. i 14. prosinca 1891.} kada se doktrina Trojstva spominje u pozitivnom svjetlu u našoj literaturi. Ali kada uzmete u obzir činjenicu da je adventistička crkva imala svoj pozitivan stav o doktrini o Bogu, kako je izraženo u \emcap{Fundamentalnim Principima}, ti se slučajevi ne mogu tumačiti kao napredak u razumijevanju, već kao prodor doktrine Trojstva u Crkvu Adventista Sedmoga Dana.

Lako je opovrgnuti tvrdnju da pioniri adventista nisu razumjeli doktrinu o Bogu. Da je nisu razumjeli, ne bi uspjeli proglasiti prvu anđeosku vijest. Ovu smo točku detaljno raspravili u poglavlju “\hyperref[chap:remembering-the-beginning]{Prisjećanje početaka}”. Pokret Adventista Sedmoga Dana nije bio neuspjeh, već Božji vođen, proročki pokret.

\subsection*{Korak 2: Zanemariti upozorenja o izgradnji novog temelja}

Kada se \emcap{Fundamentalni Principi} uklone iz jednadžbe, mnoga upozorenja Ellen White ne sjaje u svom pravom svjetlu i njihovo pravo značenje ne odjekuje kod čitatelja.

Citirali smo mnoge izjave gdje je sestra White upozoravala crkvu da ne odstupa od \emcap{Fundamentalnih Principa}. Bavili smo se njima u poglavlju “\hyperref[chap:apostasy]{Veliko otpadništvo uskoro će se ostvariti}”, ali ćemo ponovno spomenuti jednu od najistaknutijih izjava.

\egw{\textbf{Neprijatelj duša tražio je da uvede pretpostavku da će nastati velika reformacija među Adventistima Sedmog Dana, i da će se ta reforma sastojati od \underline{odstupanja od doktrina koje stoje kao stupovi naše vjere} i da će se pokrenuti proces reorganizacije}. Kada bi takva reformacija nastupila, kakav bi bio rezultat? \textbf{Principi istine koje je Bog u svojoj mudrosti dao crkvi ostatka bili bi odbačeni. Naša religija bi se promijenila. \underline{Fundamentalni principi koji su održavali posao posljednjih pedeset godina bili bi proglašeni greškom}}. \textbf{Nova organizacija bi se osnovala. Knjige novog reda bi se napisale. Uveo bi se sistem intelektualne filozofije}...}[Lt242-1903.13; 1903][https://egwwritings.org/read?panels=p7767.20]

\egwnogap{Tko ima ovlasti započeti takav pokret? \textbf{Imamo naše Biblije. Imamo svoje iskustvo koje je potvrđeno čudesnim radom Duha Svetoga}. \textbf{Imamo istinu koja ne prihvaća kompromise.} \textbf{\underline{Ne bismo li trebali odbaciti sve što nije u skladu s tom istinom}?}}[Lt242-1903.14; 1903][https://egwwritings.org/read?panels=p7767.21]

\subsection*{Korak 3: Negirati da je ličnost Boga bio stup naše vjere i dio temelja naše vjere}

Postoji jedna izjava Ellen White koja naizgled podržava tvrdnju da \emcap{ličnost Boga} nije bila stup naše vjere. Drugi izraz za “\textit{stupove naše vjere}” su “\textit{međaši}”. U sljedećim citatima, sestra White nabraja nekoliko međaša: čišćenje svetišta, poruke triju anđela, hram Božji, subotu i smrtnost duše.

\egw{Protek vremena 1844. bio je razdoblje velikih događaja, otvarajući našim zadivljenim očima \textbf{čišćenje svetišta koje se događa na nebu}, i koje ima odlučujući odnos prema Božjem narodu na zemlji, [također] \textbf{poruke prvog i drugog anđela i trećeg}, razvijajući zastavu na kojoj je bilo ispisano: ‘Zapovijedi Božje i vjera Isusova.’ [Otkrivenje 14:12.] Jedan od međaša pod ovom porukom bio je \textbf{hram Božji}, viđen od Njegovih ljudi koji vole istinu na nebu, i kovčeg koji sadrži Božji zakon. Svjetlo \textbf{subote} iz četvrte zapovijedi bacalo je svoje jake zrake na put prijestupnika Božjeg zakona. \textbf{Nebesmrtnost zlih} je stari međaš. \textbf{Ne mogu se sjetiti ničega više što bi moglo spadati pod naziv stare međaše}. Sav ovaj vapaj o promjeni starih međaša je sama izmišljotina.}[Ms13-1889.9; 1889][https://egwwritings.org/read?panels=p4179.14]

Na kraju ovog popisa međaša, ili stupova naše vjere, ona kaže da se ne može sjetiti ničega više što bi moglo spadati pod naziv starih međaša. Za mnoge, ovaj citat je dokaz da \emcap{ličnost Boga} nije bila stari međaš niti stup naše vjere. Istina je da u ovom citatu sestra White nije eksplicitno spomenula \emcap{ličnost Boga}, ali ona je implicitno uključena pod porukom prvog anđela, kao i u temeljnu doktrinu poruke Svetišta. Nadalje, postoje i drugi citati sestre White koji eksplicitno uključuju \emcap{ličnost Boga} kao stari međaš, ili stup naše vjere.

\egw{Oni koji žele ukloniti \textbf{stare međaše} ne drže se čvrsto; \textbf{ne pamte onako kako su primili i čuli}. Oni koji pokušavaju \textbf{\underline{uvesti} teorije koje bi uklonile \underline{stupove naše vjere}} \textbf{po pitanju svetišta}, \textbf{\underline{ili o ličnosti Boga ili Krista}, rade kao slijepi ljudi}. Trude se unijeti nesigurnosti i uputiti Božji narod \textbf{na pogrešan put}, bez sidra.}[Ms62-1905.14; 1905][https://egwwritings.org/read?panels=p14070.10026020]

Sestra White nas također uči da stupovi naše vjere čine temelj naše vjere.

\egw{\textbf{Koji bi to utjecaj mogao, u ovom dijelu povijesti, na nepošten, snažan način \underline{srušiti temelje naše vjere},—temelj koji je bio postavljen na početku našeg rada sa molitvom i proučavanjem Riječi i Otkrivenja? Na \underline{tom temelju} smo gradili \underline{posljednjih pedeset godina}. Smatraš li da nemam što reći kada vidim početak rada koji bi \underline{uklonio neke od stupova naše vjere}? Moram poslušati naredbu, ‘Suoči se!’}}[SpTB02 58.1; 1904][https://egwwritings.org/read?panels=p417.295]

Uklanjanje nekih od stupova naše vjere znači rušenje temelja naše vjere. Na drugom mjestu, sestra White je rekla da se rušenje ili podrivanje temelja naše vjere vrši indoktrinacijom sentimenata u vezi s \emcap{ličnošću Boga}.

\egw{Studij je premješten iz Battle Creeka; ipak, studenti se i dalje pozivaju da dođu tamo, i tamo bivaju \textbf{indoktrinirani samim sentimentima u vezi s ličnošću Boga i Krista koji bi podrivali temelj naše vjere}.}[Lt72-1906.5; 1906][https://egwwritings.org/read?panels=p10013.11]

U svjetlu ovih citata vidimo pozitivno svjedočanstvo da je doktrina o \emcap{ličnosti Boga} bila dio temelja naše vjere. Nadalje, u poglavlju 10 Specijalnih Svjedočanstava, pod naslovom “\textit{Temelj naše vjere}”, sestra White je spomenula “\textit{Fundamentalne Principe}” koristeći sinonime “\textit{stupovi naše vjere}”, “\textit{međaši}” i “\textit{orijentiri}”, kada je govorila o temelju naše vjere.

\subsection*{Korak 4: Promijeniti značenje izraza “ličnost Boga”}

Riječ ‘\textit{ličnost}’ ima dvije različite primjene i najčešća definicija u svakodnevnoj uporabi je iz područja psihologije. ‘\textit{Ličnost}’ se definira kao “\textit{karakteristični skupovi ponašanja, kognicija i emocionalnih obrazaca koji se razvijaju iz bioloških i okolišnih čimbenika}”\footnote{Wikipedia Contributors. “Personality.” Wikipedia, Wikimedia Foundation, 19 Apr. 2019, \href{https://en.wikipedia.org/wiki/Personality}{en.wikipedia.org/wiki/Personality}.}. Od najveće je važnosti prepoznati da kada govorimo o stupu naše vjere—“\textit{ličnosti Boga}”—ne govorimo o pojmu iz područja psihologije. Točna primjena riječi ‘\textit{ličnost}’, vezana za doktrinu o \emcap{ličnosti Boga}, nalazi se u Merriam-Webster rječniku, kao: “\textit{kvaliteta ili stanje koja nekoga čini osobom}”\footnote{\href{https://www.merriam-webster.com/dictionary/personality}{Merriam-Webster Dictionary} - ‘\textit{ličnost}’}. Prema Merriam-Webster rječniku, ova se definicija koristi od 15. stoljeća\footnote{Vidi “\href{https://www.merriam-webster.com/dictionary/personality\#word-history}{Prva poznata uporaba}” riječi ‘ličnost’ u Merriam Webster rječniku}. U izdanju Merriam-Webster rječnika iz 1828. godine čitamo definiciju riječi ‘\textit{ličnost}’ kao: “\textit{ono što nekoga čini zasebnom osobom}”\footnote{\href{https://archive.org/details/americandictiona02websrich/page/272/mode/2up}{Merriam-Webster Dictionary, izdanje iz 1828.} - ‘\textit{ličnost}’} \footnote{\href{https://archive.org/details/websterscomplete00webs/page/974/mode/2up}{Izdanje Merriam-Webster rječnika iz 1886.} definira riječ ‘\textit{ličnost}’ kao: “\textit{ono što čini ili se odnosi na osobu}”}. Obje definicije nalaze se u Enciklopedijskom rječniku, autora Hunter Roberta\footnote{\href{https://babel.hathitrust.org/cgi/pt?id=mdp.39015050663213&view=1up&seq=780}{Hunter Robert, The Encyclopaedic Dictionary} - ‘\textit{ličnost}’}—rječnik u vlasništvu Ellen White. Primjena ovih definicija može se vidjeti iz članaka napisanih o \emcap{ličnosti Boga}.

Godine 1903., kada je sestra White pisala dr. Kelloggu, \egwinline{ja sam \textbf{uvijek} imala nositi isto svjedočanstvo koje i sada nosim \textbf{po pitanju ličnosti Boga}}[Lt253-1903.9; 1903][https://egwwritings.org/read?panels=p14068.9980015], prisjetila se svoje vizije kada je vidjela Oca i Sina.

\egw{‘Često sam viđala dragog Isusa, da je \textbf{On osoba}. \textbf{Upitala sam Ga je li Njegov Otac osoba}, i \textbf{ima li \underline{oblik} kao i On}. Isus je odgovorio: ‘\textbf{Ja sam savršena slika osobe Moga Oca!}’ [Hebrejima 1:3.]}[Lt253-1903.12; 1903][https://egwwritings.org/read?panels=p14068.9980018]

Kvaliteta ili stanje po kojom sestra White definira Boga kao osobom jeste Njegov \textit{oblik}—\textit{fizički izgled}. Dr. Kellogg slijedi istu primjenu riječi \textit{‘ličnost’}, iako kroz spekulacije.

\others{Činjenica da je Bog tako velik da ne možemo stvoriti jasnu mentalnu sliku o \textbf{njegovom fizičkom izgledu} ne treba umanjivati u našem umu stvarnost \textbf{Njegove ličnosti}...}[John H. Kellogg, The Living Temple, str. 31][https://archive.org/details/J.H.Kellogg.TheLivingTemple1903/page/n31/mode/2up]

Kao što smo ranije vidjeli, naši adventistički pioniri također su istaknuli fizički izgled kao kvalitetu koja Boga čini osobom. James White je napisao, \others{Oni koji poriču \textbf{ličnost Boga}, kažu da ‘slika’ ovdje ne znači \textbf{fizički oblik}, već moralnu sliku...}[James S. White, PERGO 1.1; 1861][https://egwwritings.org/read?panels=p1471.3]. J. B. Frisbie je napisao, \others{Nekima se čini se ovo protivi \textbf{Božjoj ličnosti}, jer je on Duh, i onda kažu da je On bez \textbf{tijela ili dijelova}...}[\href{https://documents.adventistarchives.org/Periodicals/RH/RH18540307-V05-07.pdf}{Adventist Review and Sabbath Herald, 7. ožujka 1854.}, J. B. Frisbie, “The Seventh-Day Sabbath Not Abolished”, str. 50]

U svjetlu činjenica, prepoznajemo primjenu riječi ‘\textit{ličnost}’. Kada se tema o \emcap{ličnosti Boga} predstavlja u vezi s doktrinom Trojstva, često postoji tendencija da se promijeni značenje riječi ‘\textit{ličnost}’. Također je važno napomenuti da tema o \emcap{ličnosti Boga} govori o ličnosti Oca. To je jasno vidljivo iz prikazanih podataka.

\subsection*{Korak 5: U ispitivanju Kelloggove krize, premještanje glavnog fokusa s ličnosti Boga na panteizam}

Podaci o Kelloggovoj krizi, u vezi s doktrinom o Trojstvu, su uvjerljivi ako se \emcap{ličnost Boga} uzme u obzir u jednadžbi. Jedini način da se ne povežu točke jest ignorirati \emcap{ličnost Boga} i prebaciti fokus isključivo na panteizam. Ne poričemo panteističku prirodu Kelloggove kontroverze. Vjerujemo da se panteistička priroda Kelloggove kontroverze ne može ispravno razumjeti ako se ne ispituje u pravom svjetlu \emcap{ličnosti Boga}. Ali, nažalost, u ispitivanju Kelloggove krize, pažnja koju panteizam dobiva nadmašuje ispitivanje istine o \emcap{ličnosti Boga}.

Možete pretražiti kompilacije Ellen White kako biste vidjeli koliko je više pažnje panteizam dobio nego \emcap{ličnost Boga}. Ako biste pretražili njezine spise za ‘panteizam’ ili ‘panteistički’, isključujući kompilacije nakon njezine smrti, pronašli biste 36 pojavljivanja. Među njima je nekoliko istih citata koje je sestra White kopirala iz jednog pisma u drugo, ili u posebna svjedočanstva za crkvu. Ako biste brojali različita pojavljivanja, pronašli biste samo 12 različitih citata koji sadrže riječi poput ‘\textit{panteizam}’ ili ‘\textit{panteistički}’\footnote{Na traci za pretraživanje \href{https://egwwritings.org/}{https://egwwritings.org/}, unesite riječ “\textit{pantheis*} “; to će uključiti sve riječi koje počinju s ‘\textit{pantheis...}’, (uključujući ‘\textit{panteizam}’ i ‘\textit{panteistički}’). Rezultati se mogu usporediti u podskupovima korpusa spisa Ellen White uključivanjem ili isključivanjem kompilacija nakon njezine smrti. Ova opcija je dostupna u padajućem izborniku ispod trake za pretraživanje.}. Ako biste proveli istu pretragu, ali samo u kompilacijama izdanim nakon njezine smrti, pronašli biste 140 pojavljivanja! Svi oni spadaju u jedan od dvanaest različitih slučajeva o kojima je sestra White pisala na temu panteizma.

U pretrazi spisa Ellen White o frazi “\textit{ličnost Boga}”, isključujući kompilacije nakon njezine smrti, pronašli biste 58 pojavljivanja. Među njima su također nekoliko istih citata koje je sestra White kopirala u nekoliko različitih pisama i u svjedočanstva za crkvu. Ipak, ako biste pretražili ovu frazu unutar kompilacija koje su izdane nakon njezine smrti, pronašli biste samo 52 pojavljivanja.

Ove jednostavne statistike pokazuju fokus kompilatora nakon smrti sestre White. Takav naglasak na panteizam promijenio je naše javno mnjenje o Kelloggovoj krizi. Četrdeset tri, od pedeset osam, citata o frazi “\textit{ličnost Boga}” nalaze se u pismima i rukopisima, dostupnim javnosti od 2015. nadalje. To znači da tri četvrtine (\textit{74 posto}) citata o \emcap{ličnosti Boga}, prije 2015., nije bilo dostupno javnosti. Prije 2015. nismo imali mnogo dostupnih podataka za proučavanje Kelloggove krize u svjetlu \emcap{ličnosti Boga} i u njezinom kontekstu.

% Koraci do Omege

\begin{titledpoem}
    \stanza{
        Pet koraka do Omege, put je mračan, dug, \\
        Gdje istina se gubi, a raste novi krug. \\
        Adventni temelj nekad čvrst i jasan, \\
        Danas zaboravljen, ali još uvijek spasan.
    }

    \stanza{
        Prvo se niječe da Principi su temelj vjere, \\
        Da su službeni i točni, puni prave mjere. \\
        Zatim se zanemaruju proročka upozorenja, \\
        O novom temelju, o putu bez spasenja.
    }

    \stanza{
        Treće, poriče se da ličnost Boga stup je vjere, \\
        Da je dio temelja, srž adventističke ere. \\
        Četvrto, riječi se mijenjaju, značenja se kriju, \\
        Dok teolozi nove teorije šiju.
    }

    \stanza{
        Peto, u Kelloggovoj krizi, fokus se premješta, \\
        S ličnosti Boga na panteizam, istina se smješta. \\
        Kelloggova kriza, samo je početak bila, \\
        Omega dolazi, Ellen je upozorila. \\
    }

    \stanza{
        O, Crkvo Božja, vrati se na stare staze, \\
        Gdje Otac i Sin u jasnoj slici dolaze. \\
        Na Fundamentalne Principe se vrati, \\
        Gdje istina o Bogu svjetlost prati.
    }
\end{titledpoem}