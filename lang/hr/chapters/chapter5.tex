\qrchapter{https://forgottenpillar.com/rsc/hr-fp-chapter5}{Teorije Krpanja - Lt253-1903}

\egw{Dragi Brate,—}

\egwnogap{\textbf{Moram ti reći da su tvoje ideje glede nekih stvari \underline{odlučno pogrešne}.} Voljela bih da vidiš svoje pogreške. \textbf{Knjiga Živi Hram \underline{ne treba biti zakrpana}, sa nekoliko učinjenih promjena u njoj, i tada je oglašavati i veličati kao vrijednu nakladu}. Bolje bi bilo predstaviti fiziološke dijelove u drugoj knjizi pod drugim naslovom. \textbf{Kada si pisao tu knjigu}, \textbf{nisi bio pod Božjim nadahnućem}. Pored tebe bio je onaj koji je nadahnuo Adama da gleda na Boga u pogrešnom svjetlu. Tvoje cjelokupno srce mora biti promijenjeno, temeljito i u potpunosti očišćeno.}[Lt253-1903.1; 1903][https://egwwritings.org/read?panels=p9980.7]

\egwnogap{\textbf{Moj brate, nemoj si dopustiti da se otuđiš od svoje braće propovjednika koji te upućuju na tvoje opasnosti. Oni koji ti vjerno i iskreno govore o tvojim pogreškama tvoji su najbolji prijatelji.} Žao mi je, jako žao, tvojih zdravstvenih suradnika. Oni su nevjerni Bogu i neistiniti prema tebi propuštajući ti ljubazno, ali odlučno reći gdje nisi radio pravedno.}[Lt253-1903.2; 1903][https://egwwritings.org/read?panels=p9980.8]

\egwnogap{Postoje mnoge stvari koje moraš nadvladati prije nego što možeš biti spašen. U srcu koje nije vođeno od Boga postoji nešto što ga navodi na želju da se održi na pogrešnom putu. Ljudi koji ti vjerno govore istinu, ukazujući na tvoje pogreške, smatrao si svojim neprijateljima. Ali često su upravo oni tvoji najbolji prijatelji i, kada su te upozorili na stramputicu, obavili su vrlo neugodnu dužnost. Gospodnje sluge ne smiju laskati tvojem ponosu; ne smiju šutjeti, bojeći se reći: ‘Zašto to činiš?’ Oni te vjerno upozoravaju na opasnost.}[Lt253-1903.3; 1903][https://egwwritings.org/read?panels=p9980.9]

\egwnogap{\textbf{Moj suprug, starješina Joseph Bates, otac Pierce, starješina Edson, i mnogi drugi koji su bili oštroumni, plemeniti, i odani bili su među onima koji su nakon isteka vremena 1844. tragali za istinom}. \textbf{Na našim važnim sastancima, ovi ljudi bi se sastajali međusobno i tragali za istinom kao za skrivenim blagom}. Sastajala bih se s njima, i mi bismo proučavali i molili se usrdno; jer smo osjećali da moramo naučiti Božju istinu. Često bismo ostajali zajedno do kasno u noć, a nekada kroz cijelu noć, moleći za svjetlost i proučavajući Riječ. Dok smo postili i molili, velika sila je došla na nas. Ali nisam mogla razumjeti rasuđivanje braće. Moj um je bio zaključan, i nisam mogla razumjeti što smo proučavali. Tada bi Duh Božji došao na mene, bila bih odnesena u viziji, i jasno objašnjenje odlomaka koje smo proučavali bi mi bilo dano s uputama vezano za poziciju koju moramo zauzeti glede istine i dužnosti. Iznova i iznova bi se to događalo. \textbf{Linija istine koja se proteže od tog vremena do vremena kada ćemo ući u Božji grad mi je jednostavno prikazana}, i dala sam svojim braćama i sestrama upute koje je Gospod meni dao. Oni su znali da kada nisam u viziji, da ne mogu razumjeti ove stvari, i oni su prihvatili otkrivenja koja su mi dana kao izravnu svjetlost s neba. \textbf{Tako su vodeće točke naše vjere, kako ih držimo danas, čvrsto utvrđene}. \textbf{\underline{Točka za točkom} bila je jasno definirana, i sva braća su došla u harmoniju}.}[Lt253-1903.4; 1903][https://egwwritings.org/read?panels=p14068.9980010]

\egwnogap{\textbf{Cijela zajednica vjernika je bila ujedinjena u istini}. \textbf{Bilo je onih koji su došli unutra s čudnim doktrinama, ali nismo se nikada bojali suočiti se s njima. Naše iskustvo je bilo čudesno ustanovljeno otkrivenjima Svetoga Duha}.}[Lt253-1903.5; 1903][https://egwwritings.org/read?panels=p9980.11]

\egwnogap{Tijekom dvije ili tri godine moj um je nastavljao biti zaključan za Pismo. 1846. sam se udala za starješinu Jamesa Whitea. Bilo je to neko vrijeme nakon poroda mojega drugog sina da smo bili u teškim nedoumicama glede nekih točaka doktrine. Molila sam se Gospodu da mi otključa um, da mogu razumjeti Njegovu Riječ. Odjednom se činilo kao da sam obavijena u jasnoj, prelijepoj svjetlosti, i od tada pa nadalje, \textbf{Pismo je bilo otvorena knjiga za mene}.}[Lt253-1903.6; 1903][https://egwwritings.org/read?panels=p14068.9980012]

\egwnogap{U to vrijeme bila sam u Parizu, Maineu. Stari Otac Andrews je bio jako bolestan. Za neko vrijeme bio je u velikim bolovima radi upalnog reumatizma. On se nije mogao pomaknuti bez snažnih bolava. Molili smo se za njega. Položila sam svoju ruku na njegovu glavu, i rekla: ‘Oče Andrews, Gospodin Isus te iscjeljuje.’ Trenutno je bio iscijeljen. On je ustao i hodao po sobi, slaveći Boga, govoreći: ‘Nikada nisam to prije tako vidio. Anđeli Božji su u ovoj sobi.’ Božja slava se očitovala. \textbf{Činilo se da svjetlost sija kroz čitavu kuću, i anđeoska ruka je bila položena na moju glavu. Od tog vremena do sada bila sam sposobna razumjeti Riječ Božju}.}[Lt253-1903.7; 1903][https://egwwritings.org/read?panels=p9980.13]

\egwnogap{\textbf{Nakon proteka vremena, bili smo suprotstavljeni i okrutno izvrtani. Pogrešne teorije su nam bile nametane od ljudi i žena koji su otišli u fanatizam}. Bila sam usmjerena da idem na mjesta gdje su ovi ljudi zagovarali ove pogrešne teorije, i dok sam išla, sila Duha se predivno pokazala u ukoravanju pogrešaka koje su se uvlačile unutra. \textbf{\underline{Sam Sotona, u osobi čovjeka}, se trudio učiniti neučinkovitim moja svjedočanstva glede pozicije za koje mi sada znamo da su potkrijepljene Pismom}.}[Lt253-1903.8; 1903][https://egwwritings.org/read?panels=p9980.14]

\egwnogap{\textbf{Upravo takve teorije kakve si ti predstavio u Živom Hramu su tada bile predstavljene}. \textbf{Te suptilne, varljive mudrolije su iznova i iznova tražile mjesto među nama. \underline{Ali ja sam uvijek imala nositi isto svjedočanstvo koje i sada nosim po pitanju ličnosti Boga}}.}[Lt253-1903.9; 1903][https://egwwritings.org/read?panels=p14068.9980015]

\egwnogap{U (Ranim Spisima, 60, 66, 67)\footnote{Čini se da su stranice netočne. Spomenuti odlomci mogu se naći u Ranim Spisima na stranicama \href{https://egwwritings.org/read?panels=p28.462&index=0}{70.2}, \href{https://egwwritings.org/read?panels=p28.490&index=0}{77}, i \href{https://egwwritings.org/read?panels=p28.390&index=0}{54.2}.}, nalaze se sljedeće izjave:}[Lt253-1903.10; 1903][https://egwwritings.org/read?panels=p9980.16]

\egwnogap{‘14. svibnja 1851. vidjela sam Isusovu ljepotu i ljupkost. Dok sam promatrala Njegovu slavu, nisam ni pomislila da bih se ikad mogla odvojiti od Njegove prisutnosti. \textbf{Vidjela sam svjetlo koje je dolazilo od slave koja je okruživala Oca}, i kad mi se približilo, tijelo mi je zadrhtalo i ja sam ustreptala kao list. Pomislila sam da ću, ako mi se približi, nestati, ali svjetlo je prošlo pokraj mene. \textbf{Tada sam mogla donekle shvatiti koliko je velik i strašan \underline{Bog} s kojim imamo posla}.’}[Lt253-1903.11; 1903][https://egwwritings.org/read?panels=p9980.17]

\egwnogap{‘Često sam viđala \textbf{dragog Isusa, da je On osoba}. \textbf{Upitala sam Ga je li Njegov Otac osoba i ims li \underline{oblik} kao i On}. Isus je odgovorio: ‘\textbf{Ja sam savršena slika osobe Moga Oca}!’ [Hebrejima 1:3.]}[Lt253-1903.12; 1903][https://egwwritings.org/read?panels=p14068.9980018]

\egwnogap{‘\textbf{Često sam vidjela da je duhovni pogled oduzeo Nebu svu slavu i da u mislima mnogih Davidovo prijestolje i ljupka osoba Isusa izgaraju u ognju spiritualizma}. Vidjela sam da će neki koji su bili prevareni i dovedeni u tu zabludu upoznati svjetlo istine, \textbf{ali će biti gotovo nemoguće da se potpuno oslobode zavodničke sile spiritualizma. Takvi trebaju učiniti temeljito djelo u priznavanju svojih zabluda i zauvijek ih napustiti}.’}[Lt253-1903.13; 1903][https://egwwritings.org/read?panels=p14068.9980019]

\egwnogap{\textbf{Postoji trag spiritualizma koji \underline{dolazi} među naš narod, i \underline{potkopat će vjeru} onih koji mu daju prostora, vodeći ih da se priklanjaju zavodničkim duhovima i đavolskim doktrinama}. Greške će se prezentirati na ugodan i laskav način. Neprijatelj čezne odvratiti umove naše braće i sestara od djela pripremanja naroda da stoje u ovim posljednjim danima.}[Lt253-1903.14; 1903][https://egwwritings.org/read?panels=p9980.21]

\egwnogap{Upućena sam da upozorim našu braću i sestre \textbf{da ne raspravljaju o prirodi našeg Boga}. Mnogi od znatiželjnika koji su pokušali otvoriti Kovčeg zavjeta, da vide što je unutra, bili su kažnjeni zbog svoje drskosti. \textbf{Ne smijemo reći da je Gospodin Bog neba u lišću ili u drvetu; jer On nije tamo. \underline{On sjedi na svom prijestolju u nebesima}}.}[Lt253-1903.15; 1903][https://egwwritings.org/read?panels=p14068.9980022]

\egwnogap{Djelo Stvoritelja, vidljivo u prirodi, otkriva Njegovu silu. Ali priroda nije iznad Boga, niti je Bog u prirodi onako kako Ga neki predstavljaju. Bog je stvorio svijet, ali svijet nije Bog; on je samo djelo Njegovih ruku. \textbf{Priroda otkriva djelo stvarnog, \underline{osobnog Boga}, pokazujući da Bog postoji i da nagrađuje one koji Ga revno traže}.}[Lt253-1903.16, 1903][https://egwwritings.org/read?panels=p9980.23]

\egwnogap{Mogla bih puno reći o svetištu; o kovčegu koji sadrži Božji zakon; o poklopcu Kovčega, koji je prijestolje milosti; o anđelima na oba kraja kovčega; i druge stvari povezane s nebeskim svetištem i s velikim danom pomirenja. Mogla bih mnogo reći o tajnama neba; ali usne su mi zatvorene. Nemam sklonost pokušati ih opisati.}[Lt253-1903.17; 1903][https://egwwritings.org/read?panels=p9980.25]

\egwnogap{\textbf{Ne bih se usudila govoriti o Bogu kao što si ti govorio o Njemu}. On je visok i uzvišen, i njegova slava ispunjava nebesa. ‘Glas Gospodnji lomi cedre; da, Gospod lomi cedre libanonske. \textbf{Gospod je u svom svetom hramu}; neka sva zemlja zašuti pred njim.’ [Vidi Psalam 29:5; Habakuk 2:20.]}[Lt253-1903.18; 1903][https://egwwritings.org/read?panels=p14068.9980026]

\egwnogap{\textbf{Moj brate, kad si u iskušenju da govoriš o Bogu, \underline{gdje je On, ili što je On}, sjeti se da je po tom pitanju šutnja rječitost}. Izuj obuću s nogu; jer tlo na koje stavljaš svoje nemarne, neposvećene noge je sveto tlo.}[Lt253-1903.19; 1903][https://egwwritings.org/read?panels=p14068.9980027]

\egwnogap{\textbf{Upućena sam da kažem da ne postoji ništa u Riječi Božjoj što bi podržavalo tvoje spiritualističke teorije. Nećeš li odmah odbaciti te teorije? Tvoj um je već dugo vremena počivao na njima, ali one nisu imale nikakav posvećujući, pročišćujući, oplemenjujući utjecaj na tvoj život. Gospod nema nikakve koristi od tih teorija, i On ne bi želio da ih Njegov narod opravdava ili širi.}}[Lt253-1903.20; 1903][https://egwwritings.org/read?panels=p9980.28]


\egwnogap{\textbf{Otac, Sveznajući, je stvorio svijet \underline{kroz} Krista Isusa}. Krist je svjetlo svijeta, put u vječni život. On, Pomazanik, je Onaj kojega je Bog dao za okajavanje grijeha svijetu. Moraš razumjeti da ukoliko ne vjeruješ \textbf{u to pomirenje}, i znaš da si kupljen po cijeni krvi \textbf{jedino rođenoga Sina Božjega}, sigurno ćeš biti vezan za zloga. \textbf{Ukoliko nastaviš njegovati teorije koje si njegovao, biti ćeš ostavljen kao igračka Sotoninih iskušenja}. On igra igru života za tvoju dušu. Ostaneš li još malo duže povezan s njime, budi siguran da ćeš izgubiti svoju dušu.}[Lt253-1903.21; 1903][https://egwwritings.org/read?panels=p14068.9980029]

\egwnogap{Izjavljujući da su naše institucije nedenominacijske, postavio si naše ljude i naš rad u pogrešan položaj. Hodao si strašnim putem čije opasnosti nisi znao, ali ćeš ih možda jednom vidjeti. Još nije prekasno da se greške isprave. Ima nade za tebe. \textbf{Pratio si neprijatelja korak po korak, nastojeći zaviriti u tajne previsoke i svete za tvoje razumijevanje}. \textbf{Onda je u tvojem učenju Svetac sveden na ljudske \underline{znanstvene, spiritualističke ideje}}. Hodao si krivim stazama. Izgubio si moralnu sliku Boga. Ali ima nade za tebe. Još uvijek možeš upraviti svoja stopala na pravi put. Nećeš li sada ispraviti svoje staze, kako hromi ne bi skrenuli s puta? Hoćeš li sada odbiti posijati još jedno sjeme skepticizma i sofizma u umove drugih? Hoćeš li sada doći Kristu i biti izliječen?}[Lt253-1903.22; 1903][https://egwwritings.org/read?panels=p14068.9980030]

\egwnogap{\textbf{Oklijevala sam i odgađala u slanju onoga što mi je Duh Gospodnji naložio napisati}. Nisam željela biti prisiljena predstaviti sotonske utjecaje ovih mudrolija. Ali ukoliko ne dođe do odlučne promjene u tebi samome i tvojim suradnicima, morati ću to učiniti, da spasim druge od hodanja putem kojim si ti hodao. Morati ću poslušati zapovijed koja mi je data od Boga, ‘\textbf{Suoči se s time}.’ To je jedino što mogu učiniti.}[Lt253-1903.23; 1903][https://egwwritings.org/read?panels=p9980.31]

\egwnogap{Predstavljam ti stvari koje mi je Gospodin predstavio. Veliki posao valja se odraditi. Moramo se uhvatiti posla sa razumijevanjem, moleći, vjerujući i primajući Svetoga Duha. Jedino tako možemo odraditi nama podaren posao. \textbf{Bog zahtjeva od mene da nosim svjedočanstvo protiv Živog Hrama}. Što god tvoji suradnici imaju za reći glede te knjige, \textbf{ja sada i za uvijek uzimam poziciju da je to zamka}. \textbf{Nikakvo jedinstvo neće biti oblikovano od našeg naroda kao cijeline na \underline{teorijama} koje si ti počeo predstavljati u toj knjizi}. \textbf{Možeš to smatrati zauvijek određenim}. \textbf{Kao narod stajati ćemo čvrsto na \underline{platformi koja je izdržala ispit i probu}. Držati ćemo se \underline{sigurnih stupova naše vjere}. \underline{Principi istine} koje nam je Bog otkrio su naš jedini temelj. Oni su nas učinili onim što jesmo. Te nove, nestvarne teorije su očaravajuće i varljive. Oni ugrožavaju vječne interese duše. Pismo ih ne podupire}. Odjeven u kršćanski oklop, obuveni u spremnost za evanđelje mira, čvrsto ćemo stajati \textbf{protiv ovih varljivih teorija}. Možeš odbiti i nastaviti iskrivljavati Božju Riječ na vlastitu propast, ali preklinjem te da to ne činiš.}[Lt253-1903.24; 1903][https://egwwritings.org/read?panels=p9980.32]

\egwnogap{\textbf{Nebo nije para. Ono je mjesto}. \textbf{Krist je otišao pripremiti stanove za one koji ga ljube}, one koji, u poslušnosti Njegovim zapovijedima, izađu iz svijeta i odvojeni su. Principi neba moraju se unijeti u naše iskustvo, kako bismo se mogli razlikovati od svijeta. \textbf{Između nas i svijeta mora postojati jasan kontrast; jer mi smo Božji denominirani narod}.}[Lt253-1903.25; 1903][https://egwwritings.org/read?panels=p14068.9980033]

\egwnogap{Gospodin ti je podario priliku ispraviti stvari. \textbf{Radujem se što si napravio početak. Nemoj misliti da mi nemamo pravo pokušati ispraviti tvoje pogreške i rezultate ovih pogrešaka. Dokle god mi Bog daje daha, i nalaže mi da koristim pero i glas u odbijanju ove zle stvari koja je ušla među nas, odraditi ću svoj dio u borbi. Još od kada sam imala sedamnaest godina, morala sam voditi ovu borbu protiv lažnih teorija, u obrani istine}. \textbf{Povijest našeg prošlog iskustva je neizbrisivo upečaćena u mojem umu, i ja sam odlučna da \underline{nijedna teorija takvog reda koje ti prihvaćaš} ne uđu u naše redove}. Ukoliko se budeš odbio promijeniti i nastaviš voditi svoje suradnike za sobom, i oni se izlože opasnosti tvojega vodstva, odgovornost leži na tebi i na njima, ne na mojoj duši.}[Lt253-1903.26, 1903][https://egwwritings.org/read?panels=p9980.34]

\egwnogap{\textbf{Govorim odlučno, kako bi mogao znati, da ukoliko ne postoji odlučna promjena u tebi, ne može postojati nada za jedinstvo između tebe i onih koji drže početak svojeg pouzdanja čvrsto do kraja}. Ti si napravio podjelu. \textbf{\underline{Moramo čvrsto stajati za istine koje nam je Bog dao kao stupove naše vjere}}.}[Lt253-1903.27; 1903][https://egwwritings.org/read?panels=p9980.35]

\egwnogap{Molim te da se okreneš Gospodu punog srca, prije nego što je zauvijek prekasno. Odvoji se od utjecaja koji su te odvojili od tvoje braće koja su uključena u službu evanđelja i od naroda kojega Bog vodi. \textbf{\underline{Teorije krpanja} ne mogu biti prihvaćene od onih koji su lojalni vjeri i \underline{principima} koji su odoljeli svim protivljenjima sotonskih utjecaja}.}[Lt253-1903.28; 1903][https://egwwritings.org/read?panels=p14068.9980036]

\egwnogap{Ako se isprazniš od svega što te je odvojilo od Krista i primiš Spasitelja u svoje srce, tvoj karakter će se izmijeniti. Odloži svoje odgovornosti na neko vrijeme i otiđi nekamo s nekolicinom svoje braće i s njima istraži Sveto pismo. Ponizi svoje srce pred Gospodinom i temeljito radi na pokajanju. \textbf{Kristova religija je duhovni kvasac koji valja unijeti u srce. To mijenja život i karakter}. Ova religija je nebeski princip, koji se vidi u životu i razgovoru kršćanina. Otkriva se u kršćanskoj čistoći. Kristova se ljubav vidi u nježnosti i milosti posvećenog čovječanstva. Spašeni smo po Riječi koja je tijelom postala. Naše otkupljenje je izvršeno \textbf{ne time što je Sin Božji ostao na nebu, već time što se Sin Božji utjelovio—uzevši čovjekovu prirodu na sebe i došao na ovaj svijet}. Tako nam je donesen vječni život. Ono što autoritet, zapovijedi i obećanja nisu mogli učiniti, Bog je učinio došavši na ovaj svijet u obličju grešnog tijela.}[Lt253-1903.29; 1903][https://egwwritings.org/read?panels=p9980.37]

\egwnogap{Krist je došao na Zemlju živjeti kao čovjek među ljudima, ne da bi ga pokvarila ljudska slabost, nego da u ljudske umove postavi principe istine koja se nikada ne mogu izbrisati, jer su vječno istiniti. Došao je donijeti novi život palim ljudskim bićima—izvrsnost koja se ne može okaljati ili pokvariti grijehom.}[Lt253-1903.30; 1903][https://egwwritings.org/read?panels=p9980.38]

\egwnogap{\textbf{Moj brate, moram ti reći da imaš malo svjesti o tome kuda su tvoje noge nagnale}. Vezao si se za one koje pripadaju vojsci velikog opadnika. \textbf{Tvoj um je bio mračan kao što je Egipat}. \textbf{Ukoliko padneš na Stijenu i razbiješ se}, Krist će te prihvatiti. Ali ti si stajao na neprijateljevom tlu, radeći njegov posao. \textbf{Religijski svijet brzo hita istom cestom koju si ti slijedio. Ukoliko nastaviš slijediti tu cestu, imati ćeš obilje društva. Ali koji će svršetak tome biti?}}[Lt253-1903.31; 1903][https://egwwritings.org/read?panels=p14068.9980039]

\egwnogap{Toliko si dugo hodao u tami, tako dugo si slijedio svoj vlastiti put, da bi mogao biti u velikom iskušenju oduprijeti se ovom pozivu kojeg ti upućujem. Da nije riječ o tvojim vječnim interesima, ne bih s tobom razgovarala o ovoj temi. Činilo bi se da sam dovoljno napisala, da nema potrebe više nametati ovu temu. \textbf{Ali kažem ti u istini da mi je sasvim jasno što činim}. Dano ti je dovoljno svjetla. Ali već nekoliko godina nisi obraćao pažnju na ovo svjetlo. Da si želio znati što je Gospodin rekao, mogao si znati; \textbf{jer imaš knjige koje su napisane pod vodstvom Njegovog Duha}. Imao si sva uputstva koja su se mogla potražiti da pokažu pravi put. Poslano ti je izravno svjetlo. Ali to si smatrao manje važnim od svojih vlastitih planova i mišljenja. Da si poslušao svjedočanstva koja su ti poslana, Živi Hram nikada ne bi bio napisan.}[Lt253-1903.32; 1903][https://egwwritings.org/read?panels=p14068.9980040]

\egwnogap{Nećeš li učiniti temeljit, odlučan, Kristov napor da razbiješ urok koju je Sotona bacio na tebe? On je imao veliku moć nad tvojim umom i pokolebao te je u krivim smjerovima. On misli da te sada može držati. Nećeš li ga poraziti i razočarati?}[Lt253-1903.33; 1903][https://egwwritings.org/read?panels=p9980.41]

\egwnogap{Pišem ti kao što bih pisala svojemu sinu. Otrgni se od neprijatelja — tužitelja braće. Reci mu: ‘Odlazi od mene Sotono. Počinio sam težak grijeh poslušavši tvoje prijedloge. Neću ih više slušati.’ Molim te, za svoju dušu, odoli kušaču, da pobjegne od tebe. Približi se Bogu, i On će se približiti tebi. \textbf{Izgubit ćeš nebo ako ne padneš na Stijenu i ne slomiš se}.}[Lt253-1903.34; 1903][https://egwwritings.org/read?panels=p9980.42]

Mnoge stvari u ovom pismu dr. Kelloggu ostaju nerečene, ali su objašnjene kada se shvati kontekst. Ellen White je pročitala pismo brata Daniellsa koji je ispričao kako Kellogg želi revidirati knjigu Živi Hram jer je\others{razmišljao o tome ponovo i počeo uviđati da je učinio malu grešku u \textbf{izražavanju} svojih gledišta}, i\others{za kratko vrijeme došao je do toga da je povjerovao u Trojstvo i sada može vrlo jasno vidjeti gdje su bile sve teškoće i vjeruje da može zadovoljavajuće razjasniti to pitanje}. Kellogg je priznao\others{da on sada vjeruje u Boga Oca, Boga Sina i Boga Svetog Duha}. Kao odgovor na to, Sestra White mu je osobno napisala:\egwinline{Knjiga Živi Hram \textbf{ne treba biti zakrpana}, sa nekoliko učinjenih promjena u njoj, i tada je oglašavati i veličati kao vrijednu nakladu}. Kako je Kellogg želio pokrpati svoju knjigu? Prema svjedočanstvu A. G. Daniellsa, mislio je promijeniti nekoliko izraza eksplicitno izražavajući svoj trinitarijanski sentiment. Ali izražavanje sâmih stavova nije bio pravi problem, već sâmi stavovi. Sestra White ga nije štedjela prekoriti radi njegovih pogleda na Boga, koji su bili trinitarijanski pogledi. Rekla mu je da je\egwinline{\textbf{odlučna da \underline{nijedna teorija takvog reda koje ti prihvaćaš} ne uđu u naše redove}}. To je jako snažna izjava. Je li moguće da, budući da je Kellogg priznao da prihvaća doktrinu Trojstva, sestra White, također i tu doktrinu podrazumijevala u svojoj izjavi? Čini se nezamislivim, jer danas, ta doktrina jeste u našim redovima danas. Ali kada je rekla:\egwinline{\textbf{Teorije krpanja} ne mogu biti prihvaćene od onih koji su lojalni \textbf{vjeri i principima} koji su odoljeli svim protivljenjima sotonskih utjecaja,} ona je zapravo istaknula doktrinu o Trojstvu. Kellogg je želio pokrpati “Živi Hram” eksplicitno spominjući doktrinu o Trojstvu. Zašto je sestra White bila odlučna zadržati ovu doktrinu izvan naših redova, a ipak je ona u našim redovima danas? Valja je istaknuti da doktrina o Trojstvu nije bila dio vjere Adventista Sedmog Dana u njeno vrijeme i da je u naše redove došlo tek kasnije. Danas mnogi tvrde da je to zbog njezinih djela što je Trojstvo dio naših vjerovanja, ali reakcija Ellen White i njezin odgovor na Kelloggovo vjerovanje u to pokazuju kako se ona nosila s takvom doktrinom. Što možemo naučiti iz toga?

Uzeto u svom kontekstu, ovo pismo baca novo svjetlo na Kelloggovu kontroverzu i pokazuje kako bismo se trebali nositi s doktrinom o Trojstvu. Prvo što se pitamo je zašto sestra White nikada nije koristila riječ “Trojstvo” u svojim spisima, čak i kada se direktno bavila tom doktrinom? Drugdje, ona odgovara:

\egw{Upozorena sam da ne ulazim u kontroverzu \textbf{glede pitanja} koje će \textbf{\underline{se pojaviti}} oko \textbf{ovih stvari, jer bi rasprava mogla \underline{navesti ljude da pribjegnu izvrtanju, i njihovi umovi bili bi odvedeni daleko od istine Božje Riječi prema pretpostavkama i nagađanjima}}. \textbf{Što se više raspravlja o tim izmišljenim teorijama, to će ljudi \underline{manje znati o Bogu i istini koja posvećuje dušu}}.}[Lt232-1903.41; 1903][https://egwwritings.org/read?panels=p14068.10197050]

Ovo je vrlo važna lekcija i princip koje nas ovdje podučava sestra White. Kad je došlo do kontroverze oko Kelloggovih teorija, ona se nije upuštala u te same teorije, jer bi to odvelo umove ljudi od istine Riječi Božje prema pretpostavkama i nagađanjima. Umjesto toga, ona je vodila umove ljudi u istinu, koja posvećuje dušu. Vodila je svojim primjerom, što je vidljivo ovdje u njezinom pismu dr. Kelloggu. Ova istina do koje je vodila umove ljudi bila je istina o \emcap{ličnosti Boga}. Ona jeste ukorila Kellogga zbog njegovih teorija ali, što je vrlo važno, mi ispravno identificiramo te teorije prema njihovom kontekstu i njezinim implicitnom referiranjem.

Vidimo kako je postavila doktrinu o Trojstvu u kontrast naspram istine o \emcap{ličnosti Boga}. Napravila je razliku između starih principa naše vjere i ovih novih teorija. Najprije, prisjetila nas je na početak naše duhovne baštine, \egwinline{nakon proteka vremena u 1844}, kada su njen suprug James White, Joseph Bates, otac Pierce, starješina Edson, i mnogi drugi koji su bili oštroiumni, plemeniti, i odani tragali za istinom. Ponovno je ukazala na divna i moćna iskustva kako su vodeće točke naše vjere, koje su se tada držale, bile ustanovljene. \egwinline{\textbf{Tako su \underline{vodeće točke naše vjere}} kako ih držimo danas, čvrsto utvrđene.} \egwinline{\textbf{\underline{Točka za točkom} se jasno izdefinirala, i sva braća su došli u harmoniju}.} \egwinline{\textbf{Cijela zajednica vjernika je bila ujedinjena u istini}}. Očito, iz konteksta 10. poglavlja Posebnih Svjedočanstava, znamo da ta iskustva objašnjavaju \egwinline{\textbf{koliko čvrsto su temelji naše vjere postavljeni}}[SpTB02 56.4; 1904][https://egwwritings.org/read?panels=p417.288]. Ti temelji su izraženi u \emcap{Fundamentalnim Principima}\footnote{\href{https://static1.squarespace.com/static/554c4998e4b04e89ea0c4073/t/59d17e24c027d84167e17617/1506901547915/SDA-YB1905+\%28P.+188-192\%29.pdf}{Yearbook Of Seventh-day Adventist denomination 1905, p. 188-192}}. Taj temelj su istine koje su, \egwinline{\textbf{\underline{točku po točku}}, \textbf{bile tražene molitvenim proučavanjem, i potvrđene čudesnom silom Gospodnjom}}. Bog \egwinline{\textbf{nas poziva da \underline{držimo čvrsto}, zahvatom vjere, \underline{fundamentalne principe} koji su \underline{zasnovani na neupitnom autoritetu}}.}[SpTB02 59.1; 1904][https://egwwritings.org/read?panels=p417.299] U svjetlu tih iskustava i istine izražene u \emcap{fundamentalnim principima}, \egwinline{\textbf{\underline{Teorije krpanja} ne mogu biti prihvaćene od onih koji su lojalni \underline{vjeri} i \underline{principima} koji su odoljeli svim protivljenjima sotonskih utjecaja}}[Lt253-1903.28; 1903][https://egwwritings.org/read?panels=p14068.9980036]. Iz povijesnih zapisa o ovoj braći koja su bila oštroumna, plemenita i odana, imamo pokazatelje da su se i oni suprotstavili doktrinu o Trojstvu s istinom o \emcap{ličnosti Boga}. James White u članku iz Review and Herald časopisa, navodi \others{neke od popularnih zabluda tog doba}, govoreći: \others{Ovdje bismo mogli spomenuti \textbf{Trojstvo, koje \underline{ukida ličnost Boga, i njegova Sina Isusa Krista}}}[James White, Review \& Herald, 11. prosinca 1855, p. 85.15][http://documents.adventistarchives.org/Periodicals/RH/RH18551211-V07-11.pdf]. J. N. Andrews je rekao,\others{\textbf{Doktrina o Trojstvu koju je u crkvi uspostavio koncil u Niceji, 325. godine po Kr}. \textbf{Ova doktrina \underline{uništava ličnost Boga, i njegova Sina Isusa Krista našega Gospodina}}...}[J. N. Andrews, Review \& Herald, 6. ožujka 1855, p. 185][http://documents.adventistarchives.org/Periodicals/RH/RH18550306-V06-24.pdf] J. B. Frisbie, u svojem članku “\textit{Sedmi dan Subota nije ukinuta}”, uspoređuje Boga Subote sa bogom Nedjelje; opisuje Boga Subote u svjetlu \emcap{ličnosti Boga} izražene u prvoj točki \emcap{Fundamentalnih Principa}. Bog nedjelje je opisan \others{jedinstvom ovog Božanstva, postoje tri osobe jedne substance, moći i vječnosti; Otac, Sin i Duh Sveti}[J. B. Frisbie, Review \& Herald 7 ožujka 1854. p. 50][http://documents.adventistarchives.org/Periodicals/RH/RH18540307-V05-07.pdf]. Objasnio je kako doktrina o \emcap{ličnosti Boga} stoji u konfliktu prema doktrini o Trojstvu, na istoj razini kao što sveti Subotnji dan stoji u konfliktu naspram poganskog bogoštovlja nedjeljom. Također, brat J. N. Loughborough napisao je prigovore doktrini Trojstva u “Adventist Review and Sabbath Herald” časopisu\footnote{\href{https://adventistdigitallibrary.org/adl-349160/advent-review-and-sabbath-herald-november-5-1861}{J. N. Loughborough, 5. studenog 1861, Review \& Herald, vol. 18, p. 184, par. 1-11}}. U drugim izdanjima Review and Herald časopisa, objavio je članak pod naslovom \textit{“Da li je Bog osoba?}”, objašnjavajući poziciju vjerovanja Adventista Sedmoga Dana po pitanju \emcap{ličnosti Boga}, opisanu u prvoj točki \emcap{Fundamentalnih Principa}\footnote{\href{http://documents.adventistarchives.org/Periodicals/RH/RH18550918-V07-06.pdf}{J. N. Loughborough, 18. rujna. 1855, Review \& Herald, vol. 7, p. 6.}}. James White je također objašnjavao istu poziciju u svom višestruko tiskanom pamfletu, “\textit{Ličnost Boga}”\footnote{\href{https://egwwritings.org/?ref=en_PERGO.1.1&para=1471.3}{J. White, The Personality of God, 18. lipnja 1861.}}. Ovo je samo nekoliko primjera gdje su adventistički pioniri objasnili stav o \emcap{ličnosti Boga} izražen prvom točkom \emcap{Fundamentalnih Principa}.

Sestra White se prisjetila iskustava kako su vodeće točke naše vjere, koje su se držale u prijašnja vremena, bile čvrsto utvrđene.\egwinline{\textbf{\underline{Točka za točkom} se jasno izdefinirala, i sva braća su došli u harmoniju}}[Lt253-1903.4; 1903][https://egwwritings.org/read?panels=p14068.9980010]. Te vodeće točke bili su \emcap{Fundamentalni Principi}, od kojih je \emcap{ličnost Boga} bila jedna od njih. Ta točka \emcap{Fundamentalnih Principa} i svjedočanstvo sestre White o njoj ostali su nepromijenjeni tijekom cijelog njezinog života. Rekla je \egwinline{\textbf{\underline{Ja sam uvijek imala nositi isto svjedočanstvo koje i sada nosim po pitanju ličnosti Boga}}}[Lt253-1903.9; 1903][https://egwwritings.org/read?panels=p14068.9980015]. Iz Ranih Spisa, tada je citirala svoje vizije o nebeskoj stvarnosti. Prisjetila se kako je imala privilegiju biti u Božjoj prisutnosti, kako je Bog, obavijen svjetlom svoje slave, prošao pored nje. Nije vidjela Boga od svjetla kojim je bio okružen; bojala ga se, misleći da će, ako joj se približi \egwinline{nestati}. Tada je vidjela\egwinline{\textbf{dragog Isusa, da je On osoba}. \textbf{Upitala sam Ga je li Njegov Otac osoba i dali ima \underline{oblik kao} On}. Isus je odgovorio: ‘\textbf{Ja sam savršena slika osobe Moga Oca!}’}[Lt253-1903.12; 1903][https://egwwritings.org/read?panels=p14068.9980018]. Pitanje koje je postavila bilo je: \textit{je li Bog osoba, ima li oblik poput Isusa}? Odgovor je bio potvrdan—s jakim biblijskim temeljem. Njezine vizije nisu bile izvor istine o \emcap{ličnosti Boga}; nego su potvrdili istinu koju su pioniri otkrili kroz marljivo proučavanje Božje riječi.

Stoga je njihov konačni zaključak o \emcap{ličnosti Boga} bio,\others{Da postoji \textbf{jedan Bog}, \textbf{osobno, duhovno \underline{biće}}, \textbf{stvoritelj svih stvari}, svemoguć, sveznajuć i vječan; beskonačan u mudrosti, svetosti, pravednosti, dobroti, istini i milosti; nepromjenjiv i \textbf{svugdje prisutan po svojem predstavniku, Svetome Duhu}. Ps. 139:7; Da postoji jedan Gospodin Isus Krist, \textbf{Sin Vječnog Oca, Onaj po kojemu je On stvorio sve stvari, i po kojemu one postoje} …i kao završni dio svoga djela kao svećenika, prije nego što preuzme svoje prijestolje kao kralj, učiniti će \textbf{veliko pomirenje} za grijehe svih takvih, i njihovi grijesi će tada biti izbrisani (Djela 3:19) i uklonjeni iz svetišta, kao što je pokazano u službi Levitskog svećenstva, koja je nagovještavala i oslikavala službu našega Gospodina na Nebu. Vidi Lev. 16; Heb. 8:4,5; 9:6,7; itd.}[Prva, i dio druge točke Fundamentalnih Principa, 1905.]

Ellen White podsjetila je dr. Kellogga na ovu točku \emcap{fundamentalnih principa} rekavši:\egwinline{\textbf{Otac, Sveznajući, je stvorio svijet \underline{kroz} Krista Isusa}. Krist je svjetlo svijeta, put u vječni život. \textbf{On, Pomazanik, je Onaj kojega je Bog dao za okajavanje grijeha svijetu}...}[Lt253-1903.21; 1903][https://egwwritings.org/read?panels=p14068.9980029]

Pitanje o \emcap{ličnosti Boga} odnosi se na kvalitetu ili stanje koja Boga čine osobom. Adventistički pioniri dali su odgovor na to pitanje i Bog ga je odobrio kroz spise Ellen White: Bog je \textit{osobno duhovno Biće} i on je naš nebeski Otac. Gdje je njegova prisutnost?\egwinline{\textbf{Ne smijemo reći da je Gospodin Bog neba u lišću ili u drvetu; jer On nije tamo. \underline{On sjedi na svom prijestolju u nebesima}}.}[Lt253-1903.15; 1903][https://egwwritings.org/read?panels=p14068.9980022] \\
Njegova prisutnost je na prijestolju na nebu. \\
\egwinline{\textbf{Nebo nije para. Ono je mjesto}. \textbf{Krist je otišao pripremiti stanove za one koji ga ljube}, one koji, u poslušnosti Njegovim zapovijedima, izađu iz svijeta i odvojeni su...}[EGW, Lt253-1903.25; 1903][https://egwwritings.org/read?panels=p14068.9980033]. \\
“...\egwinline{‘Glas GOSPODNJI lomi cedre; da, GOSPOD lomi cedre libanonske. \textbf{GOSPOD je u svom svetom Hramu}; nek sva zemlja zašuti pred njim.’ [Vidi Psalam 29:5; Habakuk 2:20.]}[Lt253-1903.18; 1903][https://egwwritings.org/read?panels=p14068.9980026]

Prema adventističkim pionirima i sestri White, naš nebeski Otac je jedan Bog. On je osobno duhovno Biće, prisutno na nebu, na Svome prijestolju. Božji tron je realan, fizički tron, na kojem sjedi realna Osoba (Biće, koje posjeduje oblik, baš kao i Isus)—naš Nebeski Otac. To mjesto je realno mjesto; ono nije para, ili neki drugi duhovni pogled.

\egwinline{\textbf{Često sam vidjela da je duhovni pogled oduzeo Nebu svu slavu i da u mislima mnogih Davidovo prijestolje i ljupka osoba Isusa izgaraju u ognju spiritualizma}. Vidjela sam da će neki koji su bili prevareni i dovedeni u tu zabludu upoznati svjetlo istine, ali će biti gotovo nemoguće da se potpuno oslobode zavodničke sile spiritualizma. \textbf{Takvi trebaju učiniti temeljito djelo u priznavanju svojih zabluda i zauvijek ih napustiti}.}[Lt253-1903.13; 1903][https://egwwritings.org/read?panels=p14068.9980019]

Duhovni pogled na Božju osobu je pogrešan pogled. U Bibliji imamo svjedočanstva o nebu, nebeskom prijestolju i Bogu koji sjedi na njemu. Ako prihvatimo ova svjedočanstva u njihovom očitom značenju, tada se doktrina Trojstva ne može održati. Biblija i Duh proroštva predstavljaju jednog Boga na nebu, kao osobno biće, koje ima tijelo i oblik baš kao što ga ima Isus. Ovo gledište nije u skladu s doktrinom Trojedinog Boga, budući da zahtijeva da Sveti Duh bude Biće, ima tijelo i oblik—ova ideja bi kompromitirala Duha Svetoga kao sredstvo Oca i Sina kojim su svugdje prisutni. Kako bi se održala doktrina o Trojstvu, svjedočanstva o Božjem prijestolju i Božjoj osobi moraju biti shvaćena nekim duhovnim gledištem. Ovdje smo vidjeli da je sestra White postavila učenje o \emcap{ličnosti Boga} nasuprot doktrini o Trojstvu. U kontekstu, napravila je kontrast između doktrine o Trojstvu i prve dvije točke Fundamentalnih Principa, koji su bili rezultat biblijskog proučavanja naših pionira. Referirajući se na pionire i \emcap{Fundamentalne Principe}, rekla je: \egwinline{\textbf{\underline{Teorije krpanja} ne mogu biti prihvaćene od onih koji su \underline{lojalni vjeri i principima} koji su odoljeli svim protivljenjima sotonskih utjecaja}}[Lt253-1903.28; 1903][https://egwwritings.org/read?panels=p14068.9980036]

Zaključak je jasan i jednostavan. Oni koji su lojalni našim principima vjere primljenim na početku rada, ne mogu prihvatiti teorije krpanja. U kontekstu, teorija krpanja, koja je doktrina o Trojstvu, ne može biti prihvaćena od strane onih koji se drže\egwinline{\textbf{čvrstim zahvatom vjere, \underline{fundamentalnih principa} koji su \underline{zasnovani na neupitnom autoritetu}}}[SpTB02 59.1; 1904][https://egwwritings.org/read?panels=p417.299]. Ovaj zaključak nas vraća na naš prvi predloženi test temelja naše vjere.

% The Patchwork Theories

\begin{titledpoem}
    \stanza{
        Istina čvrsta, kroz molitvu tražena, \\
        Točka po točka, pažljivo osnažena. \\
        Principi vremenom i kušnjom provjereni, \\
        Protiv Sotonskih napada čvrsto branjeni.
    }

    \stanza{
        Teorije krpanja ne smiju se prihvatiti, \\
        Niti se istina smije zakrpama zamijeniti. \\
        Bog nije u lišću, niti u drvetu skrit, \\
        Na nebeskom prijestolju On je osobno bit.
    }

    \stanza{
        Ličnost Božja nije za raspravu dana, \\
        Niti za ljudska nagađanja stvorena. \\
        Vjeran narod čvrsto na istini stoji, \\
        Dok spiritualizam duše razdvoji.
    }

    \stanza{
        Otac kroz Krista svijet je stvorio, \\
        Sin je za grijehe pomirenje učinio. \\
        Nebo je mjesto, a ne para ili san, \\
        Tamo Krist priprema dom za vječni dan.
    }
\end{titledpoem}