\chapter{Koraci prema otpadu}

U sljedećem citatu, brat J. N. Loughborough, koji je bio jedan od pionira Crkve Adventista Sedmoga Dana, upozorio nas je na pet koraka prema otpadu.

\others{\textbf{Prvi korak} prema otpadu je \textbf{sastaviti kredo}, govoreći nam što trebamo vjerovati. \textbf{Drugi} je \textbf{učiniti to kredo testom zajedništva}. \textbf{Treći} je \textbf{ispitivati članove prema tom kredu}. \textbf{Četvrti} je \textbf{proglasiti hereticima one koji ne vjeruju u taj kredo}. A \textbf{peti}, \textbf{započeti progon protiv istih}. Molim da ne oponašamo crkve u bilo kojem neopravdanom smislu u predloženom koraku.}[John N. Loughborough, Review and Herald, 8. listopada 1861.][https://egwwritings.org/?ref=en\_ARSH.October.8.1861.p.149.7&para=1685.5326]


Ovi principi su važni za imati na umu, i trebali bismo se zapitati oponašamo li mi, danas, crkve u bilo kojem neopravdanom smislu u predloženom koraku. Što bi se dogodilo adventistu sedmoga dana koji bi odbacio doktrinu o Trojstvu u korist \emcap{Fundamentalnih Principa}? Imamo li mi uspostavljen kredo u našoj crkvi? Testiramo li naše članstvo prema njemu?

\emcap{Fundamentalni Principi} su imali drugačiju prirodu i ulogu u Crkvi Adventista Sedmoga Dana, za razliku od obrasca kojeg su slijedile druge crkve. \emcap{Fundamentalni Principi} nisu bili zamišljeni kao kredo. U predgovoru izjave iz 1872. čitamo o njihovoj prirodi:

\others{Predstavljajući \textbf{javnosti} ovaj \textbf{sažetak naše vjere}, želimo da bude jasno shvaćeno da \textbf{\underline{nemamo članke vjere, vjerovanje}, ili disciplinu, \underline{osim Biblije}}. Mi \textbf{ne} iznosimo ovo \textbf{\underline{kao da ima bilo kakav autoritet nad našim narodom}}, \textbf{niti je osmišljeno da osigura jednoobraznost među njima}, \textbf{kao sustav vjere}, \textbf{već je kratak iskaz onoga što jest, i što je, s velikom jednodušnošću, držano od njih}.}[Deklaracija Fundamentalnih Principa koje Uče i Prakticiraju Adventisti Sedmog Dana, 1872]

U predgovoru izjave iz 1889. čitamo slične sentimente:

\others{Kao što je drugdje već bilo navedeno, Adventisti Sedmog Dana \textbf{nemaju kredo osim Biblije}; ali se drže \textbf{određenih dobro definiranih točaka vjere}, za koje se \textbf{osjećaju spremnima da daju razlog ‘svakomu koji ih išće obrazloženje’}. Sljedeće teze se mogu uzeti kao sažetak \textbf{glavnih obilježja njihove vjere}, nad kojima, koliko znamo, \textbf{stoji jednoglasnost cijeloga tijela}.}[Godišnjak statistike Adventista sedmog dana za 1889., str. 147, Fundamentalni Principi Adventista sedmog dana]

\emcap{Fundamentalni Principi} nisu bili zamišljeni da diktiraju nečiju vjeru. Vjernici, vođeni Svetim Duhom, slobodno su podlagali svoju savjest Božjoj Riječi; pod utjecajem Svetog Duha, došli su do istih zaključaka. Postojalo je potpuno jedinstvo u cijelom tijelu. Svi vjernici su se osjećali “\textit{spremni na odgovor svakomu koji ih išće obrazloženje}” u vezi njihove vjere.

Danas vidimo značajnu razliku u načelima i praksi adventističkih vjerovanja u usporedbi s našim pionirima. Održavamo duh jedinstva disciplinirajući naše članove zbog njihovog poricanja Temeljnih Vjerovanja. U našem crkvenom priručniku, pod odjeljkom “\textit{Razlozi za disciplinske mjere}”, čitamo prvu točku koja navodi disciplinsku mjeru za poricanje vjere u Temeljna Vjerovanja adventista sedmog dana.

\others{Razlozi za disciplinske mjere}

\others{1. \textbf{Poricanje vjere} u temeljna načela evanđelja i \textbf{u temeljna vjerovanja Crkve} ili \textbf{poučavanje doktrina koje su suprotne istima}.}[SDA Church Manual, 20th edition, Revised 2022, p. 67][https://www.adventist.org/wp-content/uploads/2023/07/2022-Seventh-day-Adventist-Church-Manual.pdf]

Disciplinirati nekoga zbog njegove vjere nije ništa drugo nego prinuda savjesti. Trebamo podložiti svoju savjest samo Bibliji—ne bilo kojem čovjeku, vijećima ili crkvenim vjerovanjima. Discipliniranje članova radi njihovog poricanja Temeljnih Vjerovanja jasan je dokaz da, zapravo, imamo kredo osim Biblije. Ne možemo prakticirati slobodu svoje savjesti u podložnosti Božjoj Riječi dok smo ograničeni skupom vjerovanja koja, ako se preispituju autoritetom Biblije, podliježu disciplinskim mjerama. U svojoj praksi zaboravili smo temelje protestantizma i reformacije. Svi reformatori imali su svoju savjest ucjenjenu do kraja svojih života. Martin Luther je slavno proveo ovo načelo u djelo u svojoj obrani pred Saborom u Wormsu.

\others{Osim ako nisam \textbf{uvjeren Svetim pismom} i jasnim razumom—ne prihvaćam autoritet papa i koncila, jer su si proturječili—\textbf{\underline{moja je savjest zarobljena Božjom Riječju}}. Ne mogu i neću opozvati ništa, jer \textbf{ići protiv savjesti nije ni ispravno ni sigurno}. Ovdje stojim, ne mogu drugačije. Bog mi pomogao. Amen.}[Bainton, 182]

Ako jedan član Crkve Adventista Sedmog Dana ima svoju savjest zarobljenu Božjom Riječju i nije u skladu s Temeljnim Vjerovanjima adventista sedmog dana, njegova savjest ne bi smjela biti prinuđena crkvenom disciplinom. Znamo da će na kraju vremena cijela Crkva Adventista Sedmog Dana biti izložena ucjenama radi pitanja Subote. Borimo se za vjersku slobodu, a ipak si dopuštamo prinuđivati savjest onih koji nisu u skladu s Temeljnim Vjerovanjima. Ako danas discipliniramo naše članove zato što svoju savjest ne podlažu ljudima, vijećima i vjerovanjima, kako ćemo djelovati sutra kada će vlada disciplinirati svoje građane zbog toga što svoju savjest ne podlažu njezinoj moći, kada će prisiljavati na poslušnost zakonodavstvu suprotnom Svetom Pismu?

Pioniri adventizma bili su vrlo svjesni opasnosti prinuđivanja savjesti crkvenih članova. Izražavanje njihovih vjerovanja nije bilo zamišljeno da stvori jedinstvo. Bili su spremni opravdati svoju vjeru, iz Biblije, kada bi ih se pitalo. Biblija je bila njihov jedini kredo i članak vjere.

Godine 1883. postojala je sugestija za uvođenjem crkvenog priručnika u Crkvu Adventista Sedmog Dana. Ovaj prijedlog je odbijen nakon pomnog istraživanja odbora kojeg je imenovala Generalna Konferencija. U članku “\textit{Bez crkvenog priručnika}” čitamo njihove razloge za neprihvaćanje predloženog crkvenog priručnika.

\others{\textbf{Dok su se braća koja su zagovarala priručnik uvijek tvrdila da takvo djelo ne bi trebalo biti ništa nalik kredo ili disciplini, niti imati ovlasti za rješavanje spornih točaka}, već bi se trebalo smatrati samo knjigom koja sadrži savjete za pomoć onima s malo iskustva, \textbf{ipak mora biti očito da bi takvo djelo, izdano pod okriljem Generalne Konferencije, odmah nosilo sa sobom veliku težinu autoriteta i bilo bi konzultirano od strane većine naših mlađih propovjednika}. \textbf{\underline{Ono bi postupno oblikovalo i formiralo cijelo tijelo}}; \textbf{i oni koji ga ne bi slijedili smatrali bi se izvan sklada s uspostavljenim načelima crkvenog reda}. \textbf{A, zapravo, nije li to cilj priručnika?} I koja bi bila svrha jednog priručnika ako ne postići takav rezultat? Ali, bi li taj rezultat, u cjelini, bio koristan? Bi li naši propovjednici bili širi, originalniji, samopouzdaniji ljudi? Bi li se na njih moglo bolje osloniti u velikim hitnim slučajevima? Bi li njihova duhovna iskustva vjerojatno bila dublja i njihova prosudba pouzdanija? \textbf{Mislimo da bi tendencija bila sasvim suprotna}.}[No Church Manual, The Review and Herald, November 27, 1883, pg. 745][https://documents.adventistarchives.org/Periodicals/RH/RH18831127-V60-47.pdf]

\others{\textbf{Biblija sadrži naš kredo i disciplinu. Ona \underline{temeljito} opskrbljuje čovjeka Božjeg za svako dobro djelo}. Ono što nije otkriveno u odnosu na organizaciju i upravljanje crkve, dužnosti zaposlenika i propovjednika, i drugih subjekata, ne bi se trebalo strogo definirati i izvući u minutne specifikacije radi ujednačenosti, \textbf{nego se prepustiti individualnoj prosudbi pod vodstvom Duha Svetoga}. \textbf{Da je bilo važno imati takvu knjigu sa uputstvima takve vrste, Duh bi bez sumnje to dao i ostavio trag potpisan nadahnućem}.}[Ibid.][https://documents.adventistarchives.org/Periodicals/RH/RH18831127-V60-47.pdf]

Od 1883. godine Crkva adventista sedmog dana značajno je narasla; stoga je, radi praktičnosti, 1931. godine Odbor Generalne Konferencije odlučio objaviti crkveni priručnik.\footnote{Maratas, Prince. “Church Manual.” General Conference of Seventh-Day Adventists, 20 Aug. 2023, \href{https://gc.adventist.org/church-manual/}{gc.adventist.org/church-manual/}. Accessed 3 Feb. 2025.} Crkva, kao organizirano tijelo, trebala bi provoditi red i disciplinu u pitanjima organizacije i planova za prosperitet misije Crkve. No, nijedan odbor ne bi trebao imati ovlasti nad nečijom savješću i vjerovanjem. Samo Bog ima pravo na tu vlast. Zato je Biblija naš jedini kredo. Podlažemo svoju savjest Božjoj Riječi, ne čovjeku, niti skupini ljudi ili odboru. Suprotno tome, mnogi vjeruju da je Bog dao ovu vlast globalnom zasjedanju Generalne Konferencije. Ali takva je ideja temeljena na pogrešnom tumačenju jednog određenog citata Ellen White. Pročitajmo taj citat pažljivo.

\egw{Ponekad, kada mala grupa ljudi povjerena \textbf{općom upravom djela}, u ime Generalne Konferencije, nastoji provesti nerazumne planove i ograničiti Božje djelo, rekla sam da više ne mogu smatrati glas Generalne Konferencije, koju zastupaju ovo nekoliko ljudi, kao glas Božji. \textbf{Ali to ne govori da se odluke Generalne Konferencije sastavljene od skupštine uredno imenovanih, reprezentativnih ljudi iz svih dijelova svijeta ne bi trebali poštivati}. \textbf{Bog je odredio da predstavnici Njegove crkve iz svih dijelova zemlje, kada se okupe na Generalnoj Konferenciji, \underline{imaju autoritet}}. Pogreška koju su neki u opasnosti napraviti, jeste podrediti se umu i rasuđivanju jednog čovjeka ili male skupine ljudi, \textbf{punu mjeru autoriteta i utjecaja kojeg je Bog dao svojoj Crkvi u rasuđivanju i glasu okupljene Generalne Konferencije \underline{kako bi planirala prosperitet i unapređenje Njegova djela}}.}[9T 260.2; 1909][https://egwwritings.org/?ref=en\_9T.260.2&para=115.1474]

Sestra White je istaknula da svjetsko zasjedanje Generalne Konferencije doista ima autoritet kao Božji glas, no ona je vrlo precizna u vezi s kojim stvarima ta skupština ima taj autoritet. Autoritet koji je Bog dao skupštini Generalne konferencije je \egwinline{kako bi planirala prosperitet i unapređenje Njegova djela}. Radi se o planiranju misije, a ne o upravljanju vjerovanjima ili savješću. Božja crkva doista ima Njegov glas u vezi vjerovanja; Božji glas u vezi vjere je Biblija. Biblija je za nas potpuno dovoljna i slobodni smo podložiti joj svoju savjest. Nijedan sažetak bilo koje denominacijske vjere nema autoritet diktirati nečiju vjeru; niti \emcap{Fundamentalni Principi}, niti trenutna Temeljna Vjerovanja.\footnote{Iako Fundamentalni Principi nisu bili zamišljeni da imaju autoritet nad ljudima, niti su bili zamišljeni da osiguraju jednolikost među njima, kao sistem vjerovanja, postoje neki dokazi koji govore suprotno. U svom članku “\textit{Seventh-day Adventists and the Doctrine of the Trinity}”, iz “\textit{Christian Workers Magazine}”, 1915., D.M. Caright je dao dokaz da je predsjednik Konferencije koristio \emcap{Fundamentalne Principe} kao test zajedništva 1911. godine. Takva praksa nije konstruktivna za Istinu, niti je korisna za vjernike.} Sestra White je bila vrlo jasna o tome da je Biblija jedino pravilo vjere, i svaka doktrina treba biti preispitana pomoću Svetog pisma. U knjizi “Velika Borba” čitamo sljedeće:

\egw{Ali Bog će imati narod na zemlji \textbf{koji će održavati Bibliju, i \underline{samo Bibliju}}, \textbf{kao standard svih doktrina i temelj svih reformi}. \textbf{Mišljenja učenih ljudi, zaključci znanosti, \underline{kreda ili odluke crkvenih vijeća}, koliko god brojne i neskladne bile kao i crkve koje predstavljaju, glas većine - nijedno od ovih, niti sva zajedno, ne bi trebali biti smatrani dokazom za ili protiv bilo koje točke vjerske vjere.} \textbf{Prije prihvaćanja bilo koje doktrine ili učenja, trebali bismo zahtijevati jasno ‘Ovako govori Gospod’ kao potporu.}}[GC 595.1; 1888][https://egwwritings.org/?ref=en\_GC.595.1&para=132.2689]

Sloboda savjesti temelj je protestantizma i reformacije. Nadamo se i vjerujemo da svaki adventist sedmog dana može slobodno podložiti svoju savjest Bibliji bez prisile disciplinskih mjera ili bilo kojim drugih sredstva. Pitanje crkvenog kreda i discipline postaje sve relevantnije danas, kada imamo obećanje da će Bog ponovno uspostaviti izvorni temelj naše vjere. Nadamo se i molimo da dokazi izneseni ovdje donesu svjetlo crkvenom vodstvu i potaknu ih da iskorijene lažne prakse u našoj sredini. Kao što su vjerskim vođama u Kristovo vrijeme bile povjerene dužnosti očuvanja Istine i prepoznavanja vremena Božjeg pohođenja, tako je danas s vođama Crkve Adventista Sedmog Dana. U nastavku ćemo predstaviti proročanstva koja je Bog posebno dao Crkvi Adventista Sedmog Dana. U naše vrijeme, vrijeme kraja, svi stupovi naše vjere koji su se držali na početku bit će ponovno uspostavljeni. Neka svaki član Crkve Adventista Sedmog Dana prepozna važnost preporoda koji Bog uskoro uspostavlja.
