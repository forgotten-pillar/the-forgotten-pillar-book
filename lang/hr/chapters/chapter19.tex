\qrchapter{https://forgottenpillar.com/rsc/hr-fp-chapter19}{Ellen White i Matej 28:19}

Mnogi tvrde da je Ellen White promovirala doktrinu o Trojstvu i da je ona odgovorna za prihvaćanje te doktrine u našim redovima. Ove tvrdnje ne uzimaju u obzir da je u svim njenim direktnim osvrtajima na tu doktrinu, branila  doktrinu o \emcap{ličnosti Boga} izraženu u prvoj točki \emcap{Fundamentalnih Principa}. Kako bi podržali tvrdnje da je Ellen White bila trinitarijanka, navode se njeni komentari na Mateja 28:19:

\bible{Idite, dakle, i učinite učenicima sve narode, \textbf{krsteći ih u ime \underline{Oca}, i \underline{Sina}, i \underline{Duha Svetoga}}.}[Matej 28:19]

Ovaj stih se najviše koristi u potpori doktrine o Trojstvu. Doktrina o Trojstvu sadrži premise o \emcap{ličnosti Boga} koje ovaj tekst ne podržava. Sam stih ne uči da Otac, Sin i Duh Sveti čine \textit{jednog} Boga, Boga Biblije. Postoje drugi jasni stihovi u Bibliji koji isključuju takvo tumačenje teksta, npr. 1. Korinćanima 8:4-6; Ivan 17:3; Efežanima 4:4-6; 1. Timoteju 2:5.

Nažalost, iste neosnovane pretpostavke napravljene o Mateju 28:19 koriste se i za citate sestre White koji se bave ovim stihom. Na primjer, sestra White koristi izraze poput \egwinline{tri najviše sile na nebu}[Lt253a-1903.18; 1903][https://egwwritings.org/read?panels=p10143.25], \egwinline{tri velike sile neba}[8T 254.1; 1904][https://egwwritings.org/read?panels=p112.1450], \egwinline{tri sveta dostojanstvenika neba}[Ms92-1901.26: 1901][https://egwwritings.org/read?panels=p10732.32] i slične izraze—nijedan od ovih citata ne opravdava pretpostavku da ta tri (Otac, Sin i Duh Sveti) čine \textit{jednog} Boga. Naprotiv, kao što je raspravljeno u prethodnom poglavlju, uzimajući u obzir sentimente Williama Boardmana i \egwinline{nebeski trio} u kontekstu, sentimenti “\textit{tri u jednom}” \egwinline{ne smije se vjerovati}[Ms21-1906.8; 1906][https://egwwritings.org/read?panels=p9754.15].

Nebeski trio (grupa od tri: Otac, Sin i Duh Sveti) također je prisutan u drugim biblijskim stihovima, osim u Mateju 28:19. Postoji nekoliko drugih stihova u Novom zavjetu gdje se spominju Otac, Sin i Duh Sveti, i ovi stihovi trebali bi se koristiti za tumačenje značenja nebeskog trija. Nijedan od stihova o nebeskom triju ne dokazuje Boga tri-u-jedan; umjesto toga, svi oni referiraju na Oca kao jednog Boga. U sljedećim stihovima, nebeski trio je podebljan kako bi se bolje prepoznali u tekstu.

\bible{Jedno tijelo i \textbf{jedan Duh}, kao što ste i pozvani na jednu nadu svojega poziva; \textbf{jedan Gospodin}, jedna vjera, jedno krštenje, \textbf{jedan Bog i Otac} sviju, koji je nad svima i po svima i u svima vama.}[Efežanima 4:4-6]

\bible{A ima razlika u darovima, ali je \textbf{isti Duh}; i ima razlika u službama, a \textbf{isti je Gospodin}; i ima razlika u djelovanjima, ali je \textbf{isti Bog} koji čini sve u svima.}[1. Korinćanima 12:4-6]

\bible{Milost \textbf{Gospodina Isusa Krista} i ljubav \textbf{Boga} i zajedništvo \textbf{Duha Svetoga} sa svima vama! Amen.}[2. Korinćanima 13:14]

\bible{Jer po \textbf{njemu} \normaltext{[Kristu]} jedni i drugi imamo pristup k \textbf{Ocu} u jednome \textbf{Duhu}.}[Efežanima 2:18]

\bible{A mi moramo svagda zahvaljivati \textbf{Bogu} za vas, braćo od \textbf{Gospodina} ljubljena, što vas je od početka \textbf{Bog} izabrao za spasenje posvećenjem \textbf{Duha} i vjerom u istinu.}[2 Solunjanima 2:13]

\bible{Koliko će više krv \textbf{Krista} — koji po vječnom \textbf{Duhu} samoga sebe bez mane prinese \textbf{Bogu} — očistiti savjest vašu od mrtvih djela za službu \textbf{Bogu živomu}?}[Hebrejima 9:14]

\bible{Po predznanju \textbf{Boga Oca}, posvećenjem \textbf{Duha}, za poslušnost i poškropljenje krvlju \textbf{Isusa Krista}. Milost vam se i mir umnožili!}[1 Petrova 1:2]

Svi gore navedeni stihovi govore o nebeskom triju (Otac, Sin i Duh Sveti) i svi konzistentno svjedoče da je Otac tâj koji je referenciran kao Bog. Isti princip i Ellen White primjenjuje na tumačenje Mateja 28:19.

\egw{Krist je svojim sljedbenicima dao pozitivno obećanje da će im nakon svog uzašašća poslati svog Duha. ‘Idite dakle,’ rekao je, ‘i učite sve narode, krsteći ih u ime \textbf{Oca (osobnog Boga),} i \textbf{Sina (osobnog Princa i Spasitelja),} i \textbf{Duha Svetoga (poslanog s neba da predstavlja Krista);} učeći ih da drže sve što sam vam zapovjedio, i evo, ja sam s vama uvijek, sve do svršetka svijeta.’ Matej 28:19, 20.}[RH October 26, 1897, par. 9; 1897][https://egwwritings.org/read?panels=p821.16317]

Zagrade u ovom citatu nalaze se u izvornom rukopisu Ellen White. Ovdje ona daje svoje vlastito tumačenje Mateja 28:19. Otac je osobni Bog, Sin je osobni Princ i Spasitelj, a Duh Sveti je Kristov predstavnik. Ovo tumačenje je u skladu s \emcap{ličnošću Boga} izraženom u prvoj točki \emcap{Fundamentalnih Principa}. Matej 28:19 je stvar tumačenja. Tumačenje koje čini Nebeski Trio jednim Bogom nije nadahnuto. To nije ono što tekst ukazuje. Umjesto toga, pročitajmo Matej 28:19 unutar nadahnutog sklopa: “\textit{Idite dakle, i učite sve narode, krsteći ih u ime osobnog Boga, osobnog Princa i Spasitelja, i Duha Svetoga}.” Ako bi netko čitao tekst na takav način, nitko nikada ne bi pretpostavio da je jedan Bog jedinstvo tri osobe. Stoga se držimo nadahnuća, a ne izvrtanja\footnote{Izraz Ellen White—‘subterfuge’—vidi \href{https://egwwritings.org/?ref=en\_Lt232-1903.41&para=10197.50}{\{EGW, Lt232-1903.41; 1903\}}}.

\egw{Neka budu zahvalni Bogu za njegovu mnogostruku milost i neka budu ljubazni jedan prema drugom. \textbf{Oni imaju \underline{jednoga Boga} i \underline{jednoga Spasitelja}; i \underline{jednoga Duha}—\underline{Duha Kristovog}—koji će unijeti jedinstvo unutar njihovih redova}.}[9T 189.3; 1909][https://egwwritings.org/read?panels=p115.1057]

U svjetlu predstavljenih činjenica, vidimo da jednostavno enumeriranje Oca, Sina i Duha Svetoga ne dokazuje \textit{tri-u-jedan} pretpostavku, niti je ono u sukobu s \emcap{ličnošću Boga} izraženom u \emcap{Fundamentalnim Principima}. Ne poričemo postojanje tri osobe u Božanstvu, već samo pretpostavku da ova Tri Velika Dostojanstvenika čine jednog Boga.

Matej 28:19 je vrijedan stih i otvara novo područje proučavanja Biblije i Duha Proroštva. U kontekstu Živoga Hrama, i pozivajući se na njegove sentimente, sestra White je napisala da ovaj stih treba proučavati s najvećom revnošću jer nije niti upola shvaćen.

\egw{Neposredno prije svog uzašašća, Krist je svojim učenicima dao predivnu prezentaciju, \textbf{kako je zabilježeno u dvadeset osmom poglavlju Mateja}. \textbf{Ovo poglavlje sadrži upute} koje naši propovjednici, naši \textbf{liječnici}, naša mladež i svi naši članovi crkve trebaju \textbf{proučavati s najvećom \underline{revnošću}}. \textbf{Oni koji proučavaju ovu uputu kako bi trebali, \underline{neće se usuditi zagovarati teorije koje nemaju temelja u Riječi Božjoj}}. Braćo i sestre, proučavajte Pisma koja sadrže alfu i omegu znanja. \textbf{Kroz cijeli Stari i Novi Zavjet postoje stvari \underline{koje nisu niti upola shvaćene}}. ‘Isus im pristupi i prozbori govoreći: »Dana mi je sva vlast na nebu i na zemlji. Idite, dakle, i učinite učenicima sve narode, \textbf{krsteći ih u ime Oca i Sina i Duha Svetoga}; učeći ih držati sve što sam vama zapovjedio. I evo, ja sam s vama u sve dane do svršetka svijeta.«‘ [Stihovi 18-20.]}[Lt214-1906.10; 1906][https://egwwritings.org/read?panels=p10171.16]

Postoji razlog zašto je Ellen White ukazala na Matej 28:19 kao na Pismo koje \egwinline{nije niti upola shvaćeno.} Ova izjava je dana u kontekstu 1906. godine, kada su mnogi propovjednici i liječnici zagovarali doktrinu o Trojstvu. Kao što smo vidjeli, razumijevanje Boga kao trojedinog nije nešto što je Ellen White podržavala, i iz tog razloga, ona se sama nije usudila \egwinline{zagovarati teorije koje nemaju temelja u Riječi Božjoj.}

\egw{Veliki Učitelj držao je u svojoj ruci \textbf{cijelu kartu istine. Jednostavnim jezikom On je \underline{jasno objasnio} svojim učenicima} put prema nebu i \textbf{beskrajne teme božanske moći}. \textbf{Pitanje \underline{Božje esencije} bila je tema o kojoj je zadržavao mudru rezerviranost}, jer bi njihova zaplitanja i specifikacije donijeli znanost o kojoj neposvećeni umovi ne bi mogli razmišljati bez zbunjenosti. \textbf{U pogledu Boga i u pogledu Njegove ličnosti, Gospodin Isus je rekao}, ‘Toliko sam vremena s vama i još me nisi upoznao, Filipe? Tko je vidio mene, vidio je i Oca.’ [Ivan 14:9.] \textbf{Krist je bio savršena slika Očeve osobe}.}[19LtMs, Ms 45, 1904, par. 15][https://egwwritings.org/read?panels=p14069.9381023&index=0]

\egwnogap{Otvoreni put, sigurni put hodanja putem Njegovih zapovijedi, put je s kojeg nema sigurnog odstupanja. \textbf{A kada ljudi slijede svoje vlastite ljudske teorije odjevene u meke, fascinantne reprezentacije, oni stvaraju zamku u koju hvataju duše}. \textbf{\underline{Umjesto da posvećujete svoje snage teoretiziranju}}, Krist vam je dao posao za obaviti. Njegova zapovijed je, Idite <po cijelom svijetu> i učinite učenicima sve narode, \textbf{krsteći ih u ime Oca, i Sina, i Duha Svetoga}. Prije nego što učenici prijeđu prag, mora biti otisak \textbf{svetog imena, krsteći vjernike u \underline{ime trostruke sile} u nebeskom svijetu}. Ljudski um je impresioniran u ovoj ceremoniji, početku kršćanskog života. To znači vrlo mnogo. Djelo spasenja nije mala stvar, već toliko ogromna da su \textbf{najviši autoriteti} uključeni u izraženu vjeru ljudskog djelovanja. \textbf{Otac, Sin i Duh Sveti, \underline{vječno Božanstvo} uključeno je u djelovanje potrebno da osigura ljudskom djelatniku ujedinjenje \underline{cijelog neba} da doprinese vježbanju ljudskih sposobnosti da dosegnu i prihvate puninu \underline{trostruke sile} da se ujedini u velikom djelu koje je određeno, udružujući nebeske sile s ljudskim, da ljudi mogu postati, kroz nebesku učinkovitost, sudionici božanske prirode i suradnici s Kristom}.}[19LtMs, Ms 45, 1904, par. 16][https://egwwritings.org/read?panels=p14069.9381024&index=0]

Ovaj citat je još predmet čestog izvrtanja. Često se koristi kako bi se tvrdilo da je Ellen White zagovarala Trojstvo referirajući se na Oca, Sina i Duha Svetoga izrazom \egwinline{vječno Božanstvo.} Međutim, moramo razotkriti slojeve njegovog konteksta. Ellen White je objašnjavala značenje iza Mateja 28:19. Izjavila je: \egwinline{Umjesto da posvećujete svoje snage teoretiziranju,} ispunite nalog koji je dao Krist. Teoretiziranje o čemu? Teoretiziranje o \egwinline{Božjoj esenciji.} U svjetlu dosadašnjih izjava sestre White u pogledu na trojstvo, ovaj citat odgovara njenom standardom odgovoru prema doktrini o Trojstvu. To vidimo po tome da se ograđivala rasprava o Božjoj esenciji, i uzdizala je dokrtinu o ličnosti Boga. U ovom primjeru je rekla: \egwinline{\textbf{U pogledu Boga i u pogledu Njegove ličnosti}, Gospodin Isus je rekao…[Ivan 14:9.] Krist je bio savršena slika \textbf{Očeve osobe}.} Ivan 14:9 ne znači da viđenje Oca u Kristu implicira da su oni jedna te ista osoba, sve dio jednog Boga. Umjesto toga, potvrđuje da je Krist savršena slika Očeve osobe. “Bog” na kojeg se referirala bio je Otac. Zaista, Isus je učio istinu o tome tko i što je Bog. To je ono što je On \egwinline{jasno objasnio} \egwinline{jednostavnim jezikom.} Tvrditi da je izrazom \egwinline{vječno Božanstvo} Ellen White podržavala Trojstvo bilo bi u suprotnosti s upozorenjem koji je izrazila u kontekstu ovog odlomka.

Nažalost, očajnička želja trinitarijanaca da prikažu Ellen White kao zagovornicu Trojstva zasjenila je pravo, nadahnuto značenje Mateja 28:19. Njezina poruka bila je: \egwinline{Umjesto da posvećujete svoje snage teoretiziranju} o \egwinline{Božjoj esenciji,} Krist nam je dao nalog u Mateju 28:19. I objasnila je značenje Mateja 28:19. Njezina poanta bila je: Otac, Sin i Duh Sveti ujedinjuju sve nebeske resurse s ljudskim naporom kako bi, kroz božansku moć, ljudi mogli sudjelovati u Božjoj prirodi i raditi zajedno s Kristom. To je značenje ovog \egwinline{trostrukog imena.} U kontekstu, nastavila je objašnjavati:

\egw{\textbf{Ljudske sposobnosti mogu se umnožiti kroz povezivanje ljudskih djelatnika s božanskim djelatnicima}. \textbf{Ujedinjene s nebeskim silama}, ljudske sposobnosti se povećavaju prema onoj vjeri koja djeluje kroz ljubav i pročišćava, posvećuje i oplemenjuje cijelog čovjeka. \textbf{\underline{Nebeske sile} su se \underline{obvezale} služiti ljudskim djelatnicima kako bi ime Boga i Krista i Duha Svetoga učinile svojom živom učinkovitošću, radeći i osnažujući posvećenog čovjeka, da učine to ime iznad svakog drugog imena}. \textbf{Sva blaga neba su pod obvezom da učine za čovjeka} beskonačno više nego što ljudska bića mogu shvatiti umnožavajući trostruko ljudske s nebeskim djelatnicima.}[19LtMs, Ms 45, 1904, par. 17][https://egwwritings.org/read?panels=p14069.9381026&index=0]

\egwnogap{\textbf{\underline{Tri velika i slavna nebeska karaktera} prisutna su prilikom krštenja. Sve ljudske sposobnosti trebaju od sada biti posvećene sile za službu Bogu u predstavljanju Oca, Sina i Duha Svetoga o kojima ovise. \underline{Cijelo nebo je predstavljeno s ovo troje} u zavjetnom odnosu s novim životom}. ‘Ako ste, dakle, uskrsnuli s Kristom, tražite ono što je gore, gdje Krist sjedi \textbf{zdesna Bogu}.’ [Kološanima 3:1.]}[19LtMs, Ms 45, 1904, par. 18][https://egwwritings.org/read?panels=p14069.9381027&index=0]

Mnogi tvrde da Matej 28:19 nije nadahnut jer ga je umetnula Katolička crkva\footnote{Napomena, 1. Ivanova 5:7 \bible{Jer troje je što svjedoči na nebu: Otac, Riječ i Duh Sveti; i to je troje jedno.} je interpolacija poznata kao “\textit{Johannine Comma}”. Ellen White nikada nije koristila taj stih. To nije bio slučaj s Matejem 28:19.}. Ipak, ovdje imamo božansko nadahnuće koje otkriva njegovo pravo značenje—značaj krštenja u trostruko ime kao zalog koji su dala ova \egwinline{tri velika i slavna nebeska karaktera.} Njihov zalog je da su \egwinline{\textbf{sva blaga neba pod obvezom da učine za čovjeka} beskonačno više nego što ljudska bića mogu shvatiti umnožavajući trostruko ljudske s nebeskim djelatnicima.}

Ellen White često je citirala Matej 28:19, objašnjavajući zalog Oca, Sina i Duha Svetoga. Ovaj zalog služi kao divno ohrabrenje i obećanje koje podržava Nebo. Detaljno proučavanje ovog zaloga izvan je opsega ove knjige, jer se ne bavi izravno prisutnošću i \emcap{ličnošću Boga}. Međutim, potičemo vas da sami istražite ovu temu. Kada dublje uronite u njezino značenje, razumjet ćete stvarnost službe nebeskih anđela.

Sestra White je izjavila da \egwinline{cijelo nebo je predstavljeno s ovo troje u zavjetnom odnosu s novim životom.} Ovo troje su Otac, Sin i Duh Sveti. Na drugom mjestu, rekla je:

\egw{\textbf{Cijelo nebo je zainteresirano za vaš dom}. \textbf{Bog i Krist i \underline{nebeski anđeli}} intenzivno žele da tako odgajate svoju djecu da budu pripremljeni za ulazak u obitelj otkupljenih.}[17LtMs, Ms 161, 1902, par. 11][https://egwwritings.org/read?panels=p14067.9877018&index=0]

Ovo nije proturječje. Cijelo nebo je predstavljeno Ocem, Sinom i Duhom Svetim, a u ovom citatu, pod pojmom “cijelo nebo” ona obuhvaća \textit{Boga i Krista i \textbf{nebeske anđele}.} Postoji bliska veza između djelovanja Duha Svetoga i službe nebeskih anđela. Nadahnuće svjedoči:

\egw{Mjera \textbf{Duha} dana je svakom čovjeku na korist. \textbf{Kroz službu anđela \underline{Duh Sveti je osposobljen} djelovati na um i srce ljudskog djelatnika}, i privući ga Kristu koji je platio otkupninu za njegovu dušu, da bi grešnik mogao biti spašen iz ropstva grijeha i Sotone.}[8LtMs, Lt 71, 1893, par. 10][https://egwwritings.org/read?panels=p14058.6086016&index=0]

Ova anđeoska služba je jedan od elemenata u zavjetu krštenja iz Mateja 28:19. Kada je Ellen White rekla, \egwinline{\textbf{Nebeske sile} su se \textbf{zavjetovale} da će služiti ljudskim djelatnicima...}, ona je govorila o svetim anđelima. Veza između Duha Svetoga i svetih anđela je izvan opsega ove knjige, ali možete istražiti ovu temu detaljnije u nastavku ove knjige, ‘\textit{Rediscovering the Pillar}’\footnote{Preuzmite besplatno: \href{https://forgottenpillar.com/book/rediscovering-the-pillar}{https://forgottenpillar.com/book/rediscovering-the-pillar}}, u odjeljku o Duhu Svetom\footnote{Također, pogledajte proučavanje o anđelima \href{https://notefp.link/angels}{https://notefp.link/angels}}.

% Ellen White and Matthew 28:19

\begin{titledpoem}

    \stanza{
        Tri nebeske sile u krštenju stoje, \\
        Da pomognu onima koji se Bogu mole. \\
        Otac, Sin i Duh Sveti u savezu stoje, \\
        Ne jedan Bog od tri osobe koje broje.
    }

    \stanza{
        Ellen White jasno tumači nam stih, \\
        Nebeski zalog daje snagu svih. \\
        Tri sile zavjet svoj nam daju, \\
        Da vjernicima pomoć pružaju.
    }

    \stanza{
        Otac je Bog osobni i stvaran, \\
        Sin je Knez spasenja, nama darovan. \\
        Duh Sveti Krista svjedoči jasno, \\
        Vodi nas putem spasonosnim krasno.
    }
    
\end{titledpoem}