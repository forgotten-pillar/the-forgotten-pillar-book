\chapter{Uspostava pogrešnih Fundamentalnih Principa}

Možda se pitate: kako je moguće da smo mi, kao crkva, skrenuli s puta svjetla koje nam je Bog dao na početku? Odgovor na ovo pitanje isti je kao i odgovor na pitanje zašto su Židovi skrenuli s puta svjetla koje im je Bog dao u vezi s Njegovim Sinom. Molim vas, promotrite pokretačko načelo iza crkve u apostolskim vremenima i našeg vremena.

\egw{‘No anđeo Gospodnji noću otvori vrata tamnice, izvede ih i reče: »Idite, stanite i govorite u Hramu puku sve riječi života ovoga.«‘ [Djela 5:19, 20.] Ovdje vidimo da ljudi na vlasti ne treba uvijek slušati, bez obzira što se predstavljaju kao učitelji biblijskih doktrina. \textbf{Mnogi danas osjećaju ogorčenost i nezadovoljstvo što se bilo koji glas podiže i predstavlja ideje koje se razlikuju od njihovih u vezi s nekim točkama vjerovanja}. \textbf{Zar oni nisu dugo zagovarali svoje ideje kao istinu?} Tako su svećenici i rabini razmišljali u apostolskim danima. Što si misle ovi ljudi koji su neuki, neki od njih obični ribari, koji predstavljaju ideje suprotne doktrinama koje učeni svećenici i vladari podučavaju narod? \textbf{Nemaju pravo miješati se u temeljne principe naše vjere}.}[Lt38-1896.23; 1896][https://egwwritings.org/read?panels=p5631.29]

\egwnogap{“\textbf{Ali vidimo da Bog neba ponekad zadužuje ljude da \underline{poučavaju ono što se smatra suprotnim ustaljenim doktrinama}. Zato što su oni koji su nekada bili čuvari istine \underline{postali nevjerni svom svetom povjerenju}, Gospodin je izabrao druge koji bi primili sjajne zrake Sunca Pravde i zagovarali istine koje nisu u skladu s idejama vjerskih vođa. Tada ti vođe, u sljepoći svojih umova, daju pun zamah onome što se smatra pravednim gnjevom protiv onih koji su odbacili drage izmišljotine. Ponašaju se kao ljudi koji su izgubili razum. Ne razmatraju mogućnost da oni sami nisu ispravno razumjeli Riječ. Neće otvoriti oči da razaznaju činjenicu da su pogrešno tumačili i primjenjivali Sveto pismo, te su izgradili lažne teorije, \underline{nazivajući ih temeljnim doktrinama vjere}}.”}[Lt38-1896.24; 1896][https://egwwritings.org/read?panels=p5631.30]

\egwnogap{\textbf{Ali Duh Sveti će s vremena na vrijeme otkrivati istinu putem svojih izabranih sredstava; i nitko, čak ni svećenik ili vladar, nema pravo reći: Ne smiješ javno iznositi svoje mišljenje, jer ja ne vjerujem u njih. To čudesno ‘ja’ može pokušati ugušiti učenje Duha Svetoga. Ljudi mogu, na neko vrijeme, pokušati prigušiti i ubiti to učenje; ali to neće učiniti pogrešku istinom ili istinu pogreškom. Maštoviti umovi ljudi napredovali su sa spekulativnim mišljenjima u raznim smjerovima, i kada Duh Sveti pusti svjetlo da zasvijetli u ljudske umove, ono ne poštuje svaku točku čovjekove primjene riječi. Bog je nadahnuo svoje sluge da govore istinu, bez obzira na ono što su ljudi uzeli zdravo za gotovo kao istinu}.}[Lt38-1896.25; 1896][https://egwwritings.org/read?panels=p5631.31]

\egwnogap{\textbf{\underline{Čak su i Adventisti Sedmog Dana u opasnosti da zatvore oči pred istinom kakva je u Isusu}, jer ona proturječi nečemu što su uzeli zdravo za gotovo kao istinu, ali što Duh Sveti uči da nije istina. Neka svi budu vrlo skromni i neka najgorljivije nastoje izbaciti sebe iz pitanja i uzvisiti Isusa.} \textbf{U većini vjerskih kontroverzi, temelj problema je da se ego bori za prevlast}. Oko čega? Oko stvari koje uopće nisu vitalne točke, a koje se smatraju takvima samo zato što su im ljudi pridali važnost. Pogledajte Matej 12:31-37; Marko 14:56; Luka 5:21; Matej 9:3.}[Lt38-1896.26; 1896][https://egwwritings.org/read?panels=p5631.32]

Ponosno stanje srca opire se Božjoj volji i pokretačka je snaga otpadništva; ponizno srce je poslušno Božjoj volji i pokretačka je snaga prave reformacije. Sljedeći citati izražavaju buduća, konkretna proročanstva gdje će maštovite ideje o Bogu biti uvedene i \egwinline{mnoge stvari sličnog karaktera pojavit će se u budućnosti}[Ms137-1903.10; 1903][https://egwwritings.org/read?panels=p9939.17]. Ove ideje su sličnog karaktera idejama sadržanim u Živom Hramu. One će ukloniti \emcap{ličnost Boga}. Ellen White daje upozorenje za upozorenjem da se pridržavamo \emcap{Fundamentalnih Principa} i da budemo svjesni vođa koji će rušiti stari temelj.

\egw{U svjetlu ovih tekstova, tko će se usuditi tumačiti Boga i usaditi sentimente u umove drugih u vezi s Njim koji su sadržani u Živom Hramu? \textbf{Ove teorije su teorije velikog prevaranta, i u životima \underline{onih koji ih prihvate bit će tužnih poglavlja}}. \textbf{Ovo je Sotonino sredstvo \underline{da uzdrma temelj naše vjere}, da poljulja naše povjerenje u Gospodnje vodstvo i u iskustvo koje nam je dao. \underline{Mnoge stvari sličnog karaktera pojavit će se u budućnosti}}. Molim naše zdravstvene misionare da se boje vjerovati pretpostavkama i smišljanjima bilo kojeg čovjeka koji misli da \textbf{je put kojim je Božji narod vođen posljednjih pedeset godina pogrešan put}. \textbf{\underline{Čuvajte se onih koji}, nemajući nikakvo odlučno iskustvo u vođenju Gospodnjeg Duha, \underline{pretpostavljaju da je to vođenje sve laž}; da nemamo istinu}; da nismo narod Gospodnji, okupljen od Njega iz svih zemalja i naroda. \textbf{\underline{Čuvajte se onih koji bi srušili temelj na kojem smo gradili posljednjih pedeset godina kako bi uspostavili novu doktrinu}}. \textbf{Znam da su ove nove teorije od neprijatelja}.}[Ms137-1903.10; 1903][https://egwwritings.org/read?panels=p9939.17]

\egwnogap{\textbf{Neka oni koji bi \underline{unosili} maštovite ideje o Bogu postanu svjesni svoje opasnosti. Ovo je previše ozbiljna tema da bi se s njom šalilo}.}[Ms137-1903.11; 1903][https://egwwritings.org/read?panels=p9939.18]

\egwnogap{Korijen idolopoklonstva je zlo srce nevjerovanja koje se udaljava od živoga Boga. Zbog toga što ljudi nemaju vjeru u prisutnost i moć Božju \textbf{stavljaju svoje povjerenje u vlastitu mudrost}. Oni su smišljali i planirali kako bi uzvisili sebe i pronašli spasenje u svojim vlastitim djelima. \textbf{\underline{Obmanjujući utjecaj sotonske sile dolazi}, jer vođe koje je Gospodin upozorio, molio i savjetovao biraju svoju vlastitu mudrost umjesto Božje mudrosti}. Takvima dolazi upozorenje: ‘Ne govorite više tako ponosito, neka drskost iz vaših ne izlazi usta; jer GOSPOD je Bog znanja i on odmjerava djela.’}[Ms137-1903.12; 1903][https://egwwritings.org/read?panels=p9939.19]

Uspoređujući stare \emcap{Fundamentalne Principe} s novim trinitarijanskim Temeljnim Vjerovanjima, vidimo razliku. Vidimo da smo odstupili od temelja naše vjere i izgradili novi. To je bio proces. Danas, kao i u danima sestre White, sotonske službe su donijele obmanjujući utjecaj u Crkvu Adventista Sedmoga Dana, \egwinline{jer vođe koje je Gospodin upozorio, molio i savjetovao biraju svoju vlastitu mudrost umjesto Božje mudrosti}. Trebali bismo se \egwinline{\textbf{čuvati onih koji bi srušili temelj na kojem smo gradili posljednjih pedeset godina kako bi uspostavili novu doktrinu}}.

Razlika između starih \emcap{Fundamentalnih Principa} i novih Temeljnih Vjerovanja je u našim \egwinline{idejama o Bogu.} Trinitarijanska ideja o Bogu nije bila dio temelja naše vjere, kojeg je branila sestra White. Kako je došlo do te promjene? Dogodilo se to kroz vođe koji su izabrali \egwinline{vlastitu mudrost umjesto Božje mudrosti.} Trebamo se čuvati \egwinline{onih koji bi srušili temelj na kojem smo gradili posljednjih pedeset godina kako bi uspostavili novu doktrinu.} U ovom opažanju prepoznajemo da je ova nova trinitarijanska ideja o Bogu bila \egwinline{obmanjujući utjecaj sotonske sile} koji je ušao u naše redove.

% Uspostava pogrešnih Fundamentalnih Principa

\begin{titledpoem}
    \stanza{
        U tišini crkve naše, \\
        Stare istine se gase. \\
        Trinitarijansko učenje novo, \\
        Nije bilo naše slovo.
    }

    \stanza{
        Ellen White nas upozorava, \\
        Da se temelj vjere potkopava. \\
        Sotonska sila obmanjuje, \\
        Dok se nova doktrina usađuje.
    }

    \stanza{
        Vođe biraju mudrost svoju, \\
        Umjesto Božju, u tom boju. \\
        Ponizno srce istinu traži, \\
        Dok oholo srce samo sebe snaži.
    }

    \stanza{
        Čuvajte se onih koji ruše staro, \\
        Što nam je pedeset godina sjalo. \\
        Maštovite ideje o Bogu unose, \\
        Dok vlastitu mudrost ponosno pronose.
    }
\end{titledpoem}