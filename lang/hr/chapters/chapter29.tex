\chapter{Remembering His teaching in our past history}


\chapter{Sjećanje na Njegovo učenje u našoj prošlosti}


We are bringing our journey in this study to its end. We started this journey with a strong proclamation of what constitutes the foundation of our faith. We became acquainted with the history and experience of our pioneers in establishing the Seventh-day Adventist Church. We have seen the mission and purpose God bestowed upon them in proclaiming the three angels’ messages to the whole world. These messages are interwoven with all important doctrines of the Bible. These doctrines are what our pioneers called the \emcap{fundamental principles}, or the pillars of our faith. These doctrines represent the foundation of our faith.


Privodimo naše putovanje u ovom proučavanju svojemu kraju. Započeli smo ovo putovanje snažnom objavom onoga što čini temelj naše vjere. Upoznali smo se s poviješću i iskustvom naših pionira u osnivanju Crkve Adventista Sedmoga Dana. Vidjeli smo misiju i svrhu koju im je Bog dao u objavi trostruke anđeoske vijesti cijelom svijetu. Te vijesti su isprepletene sa svim važnim doktrinama Biblije. Te doktrine su ono što su naši pioniri nazivali \emcap{fundamentalnim principima}, ili stupovima naše vjere. Te doktrine predstavljaju temelj naše vjere.


Once acquainted with the \emcap{Fundamental Principles} that we had in the beginning, we recognize their difference to our current Fundamental Beliefs, particularly to the question “\textit{who is God}”? Additionally, the present doctrine of God lacks understanding of His personality. In other words, it lacks the understanding of \textit{the quality or state} of God \textit{being a person}. This change in our doctrine becomes important in light of the first angel’s message, which pertains to the God we worship. Is the God we worship a triune God, or is He one God, the Father, the Ancient of Days, a personal and spiritual Being?


Jednom kada se upoznamo s \emcap{Fundamentalnim Principima} koja smo imali na početku, prepoznajemo njihovu razliku u odnosu na naša trenutna Temeljna Vjerovanja, osobito u pitanju “\textit{tko je Bog}”? Osim toga, današnja doktrina o Bogu izostavlja razumijevanje Njegove ličnosti. Drugim riječima, izostavlja razumijevanje \textit{kvalitete ili stanja} koja Boga \textit{čine osobom}. Ova promjena u našoj doktrini postaje važna u svjetlu prve anđeoske vijesti, koja se odnosi na Boga kojeg štujemo. Je li Bog kojeg štujemo trojedini Bog, ili je On jedan Bog, Otac, Pradavni, osobno i duhovno Biće?


The change in our understanding of who God is for Seventh-day Adventists was not instantaneous; it took years of controversy to arrive at our present standing. In these studies, we did not delve into history beyond the life of Ellen White. We saw change begin to take place in her time, when Dr. Kellogg insinuated the sentiment over the \emcap{personality of God}, which would \egwinline{lead astray the minds of those who are not thoroughly established on the foundation principles of present truth}[SpTB02 51.3; 1904][https://egwwritings.org/?ref=en\_SpTB02.51.3&para=417.262]. He insinuated doubt in the clear revelations of the \emcap{personality of God} and of His Son, Jesus Christ. His sentiments were met by fierce rebukes from Sister White and strong warnings for the church, to avoid the path of doubt in the simple truth expressed in the \emcap{Fundamental Principles}—that one God is a personal, spiritual being, and Christ is His Son, \egwinline{begotten in the express image of the Father’s person}[ST May 30, 1895, par. 3; 1895][https://egwwritings.org/?ref=en\_ST.May.30.1895.par.3&para=820.12891]. Thusly, Sister White firmly defended the first two points of the \emcap{Fundamental Principles}.


Promjena u našem razumijevanju tko je Bog za Adventiste sedmog dana nije bila trenutačna; trebalo je mnogo godina kontroverzi da bismo došli do našeg sadašnjeg stajališta. U ovim proučavanjima nismo zalazili u povijest izvan života Ellen White. Vidjeli smo da su se promjene počele događati u njezino vrijeme, kada je dr. Kellogg insinuirao sentiment o \emcap{ličnosti Boga}, koji bi \egwinline{odveo u zabludu umove onih koji nisu temeljito utemeljeni na temeljnim principima sadašnje istine}[SpTB02 51.3; 1904][https://egwwritings.org/?ref=en\_SpTB02.51.3&para=417.262]. Insinuirao je sumnju u jasna otkrivenja o \emcap{ličnosti Boga} i Njegova Sina, Isusa Krista. Njegovi sentimenti su bili oštro ukoreni od sestre White koja je dala snažna upozorenja za crkvu, kako bi izbjegli put sumnje u jednostavnu istinu izraženu u \emcap{Fundamentalnim Principima}—da je jedan Bog osobno, duhovno biće, a Krist je Njegov Sin, \egwinline{rođen na savršenu sliku Očeve osobe}[ST May 30, 1895, par. 3; 1895][https://egwwritings.org/?ref=en\_ST.May.30.1895.par.3&para=820.12891]. Tako je sestra White čvrsto branila prve dvije točke \emcap{Fundamentalnih Principa}.


Just as in the time of Dr. Kellogg, when many brethren were departing from the simplicity of Christ’s teaching, so it is today. Sister White prophesied that this change over the understanding of the \emcap{personality of God} would occur in our church, and re-establishment of our old foundation of faith would be necessary. Will the \emcap{Fundamental Principles} be re-established in our midst? The outcome of it lies completely on every individual that makes up the body of the Seventh-day Adventist Church. Today, this day and age, is when this re-establishment will take place. The warnings and rebukes uttered by the pen of the Spirit of Prophecy have never been more relevant than today. God has put the final outcome into your hands. If these warnings touch the core of your soul, God is calling you to stand firm on the platform of eternal Truth. He calls you to hold firmly to the \emcap{Fundamental Principles}, which are based on unquestionable authority.


Kao i u vrijeme dr. Kellogga, kada su mnogi braća odstupali od jednostavnosti Kristovog učenja, tako je i danas. Sestra White je prorekla da će doći do promjene u razumijevanju \emcap{ličnosti Boga} u našoj crkvi i da će biti potrebno ponovno uspostavljanje našeg starog temelja vjere. Hoće li se \emcap{Fundamentalni Principi} ponovno uspostaviti među nama? Ishod toga potpuno ovisi o svakom pojedincu koji čini tijelo Crkve Adventista Sedmoga Dana. Danas, u ovo doba, je vrijeme kada će se ta obnova dogoditi. Upozorenja i ukori izrečeni perom Duha Proroštva nikada nisu bila relevantnija nego danas. Bog je konačni ishod stavio u tvoje ruke. Ako su te ta upozorenja dotakla u srce tvoje duše, Bog te poziva da čvrsto stojiš na platformi vječne Istine. On te poziva da se čvrsto držiš \emcap{Fundamentalnih Principa}, koji su zasnovani na neupitnom autoritetu.


Below, we present a portion of the letter from Sister White to Dr. Kellogg, in which we find solemn warning for us today in re-establishing the foundation of our faith. When we are acquainted with the truth on the \emcap{personality of God} and its controversy over the Trinity doctrine, the following letter shines in a new light, with messages and the principles that are valuable for us today, that we may know how to behave in our current state of affairs.


U nastavku predstavljamo dio pisma sestre White upućenog dr. Kelloggu, u kojem nalazimo ozbiljno upozorenje za nas danas u ponovnom uspostavljanju temelja naše vjere. Kada se upoznamo s istinom o \emcap{ličnosti Boga} i njezinom kontroverzom u odnosu na doktrinu o Trojstvu, sljedeće pismo sjaji u novom svjetlu, s porukama i načelima koji su vrijedni za nas danas, kako bismo znali kako se ponašati u našoj trenutnoj situaciji.


\egw{\textbf{The Foundation of Our Faith}}


\egw{\textbf{Temelj Naše Vjere}}


\egw{In regard to the book Living Temple, \textbf{I have been instructed by the heavenly messenger} that \textbf{some of the reasoning in this book is untrue, and that this reasoning would lead astray the minds of those who are not thoroughly established \underline{on the foundation principles} of present truth.} \textbf{It introduces that which is naught but speculation in regard to \underline{the personality of God} and \underline{where His presence is}}. No one on this earth has a right to speculate on this question. ‘The secret things belong unto the Lord our God, but those things which are revealed belong unto us, and to our children forever.’}[Lt232-1903.39; 1903][https://egwwritings.org/?ref=en\_Lt232-1903.39&para=10197.48]


\egw{U vezi s knjigom Živi Hram, \textbf{bila sam upućena od strane Nebeskog glasnika} da \textbf{su neka rasuđivanja u ovoj knjizi netočna i da će ta razmišljanja odvesti u zabludu umove onih koji nisu temeljito utemeljeni \underline{na temeljnim principima} sadašnje istine.} \textbf{Ona uvode ono što je ništa više nego obična nagađanja vezano za \underline{ličnost Boga} i \underline{gdje je Njegova prisutnost}}. Nitko na ovoj zemlji nema pravo nagađati o tom pitanju. ‘Što je sakriveno, pripada GOSPODU, Bogu našemu, a što je otkriveno, pripada nama i sinovima našim dovijeka.’}[Lt232-1903.39; 1903][https://egwwritings.org/?ref=en\_Lt232-1903.39&para=10197.48]


\egwnogap{\textbf{I am authorized by the Lord to say, The sentiments contained in Living Temple in regard to the personality of God \underline{are opposed to the truth that God has given us}}. \textbf{The truth for this time is now to be brought before the people.} Our brethren and sisters in every church and in every place are to guard carefully against allowing their minds to be engrossed with matters that draw them away from eternal things. The enemy will use some of the statements made in Living Temple to tempt some as he tempted Adam and Eve in Eden. \textbf{I warn our brethren not to enter into controversy over the presence and personality of God. \underline{The statements made in Living Temple in regard to this point are incorrect}.} The Scripture used to substantiate the doctrine there set forth is Scripture misapplied.}[Lt232-1903.40; 1903][https://egwwritings.org/?ref=en\_Lt232-1903.40&para=10197.49]


\egwnogap{\textbf{Gospodin me je ovlastio da kažem, Sentimenti sadržani u Živom hramu u vezi ličnosti Boga \underline{su suprotni istini koju nam je Bog dao}}. \textbf{Istina za ovo vrijeme sada se treba iznijeti pred ljude.} Naša braća i sestre u svakoj crkvi i na svakom mjestu trebaju pažljivo čuvati svoje umove da ne budu zaokupljeni stvarima koje ih odvlače od vječnih stvari. Neprijatelj će koristiti neke izjave iz Živog Hrama da iskuša neke kao što je iskušao Adama i Evu u Edenu. \textbf{Upozoravam našu braću da ne ulaze u rasprave o prisutnosti i ličnosti Boga. \underline{Izjave dane u Živom Hramu u vezi ove točke su netočne}.} Sveto Pismo korišteno za potkrepljivanje tamo iznesene doktrine je pogrešno primijenjeno.}[Lt232-1903.40; 1903][https://egwwritings.org/?ref=en\_Lt232-1903.40&para=10197.49]


\egwnogap{I was cautioned not to enter into controversy regarding the question that will come up over these things, \textbf{because controversy might lead men to resort to \underline{subterfuges}, and their minds would be led away from the truth of the Word of God \underline{to assumption and guesswork}. The more that fanciful theories are discussed, the less men will know of God and of the truth that sanctifies the soul}.}[Lt232-1903.41; 1903][https://egwwritings.org/?ref=en\_Lt232-1903.41&para=10197.50]


\egwnogap{Upozoren sam da ne ulazim u kontroverzu u vezi s pitanjem koje će se pojaviti oko ovih stvari, \textbf{jer bi rasprava mogla navesti ljude da pribjegnu \underline{izvrtanju}, i njihovi umovi bili bi odvedeni daleko od istine Božje Riječi \underline{prema pretpostavkama i nagađanjima}. Što se više raspravlja o tim izmišljenim teorijama, to će ljudi manje znati o Bogu i istini koja posvećuje dušu}.}[Lt232-1903.41; 1903][https://egwwritings.org/?ref=en\_Lt232-1903.41&para=10197.50]


\egwnogap{We are God’s commandment-keeping people. \textbf{For the last fifty years every phase of heresy has been brought to bear upon us, to tear down \underline{the foundation principles of our faith}}. Messages of every order and kind have been urged upon Seventh-day Adventists to take the place of the truth which \textbf{point by point} has been testified to by the miracle-working power of the Lord. \textbf{But the waymarks which have made us what we are are to be preserved, and they \underline{will be preserved}, as God has signified through His Word and the testimony of His Spirit. From the great system of truth as it has been presented by God’s messengers, \underline{not a pin is to be removed}}.}[Lt232-1903.42; 1903][https://egwwritings.org/?ref=en\_Lt232-1903.42&para=10197.51]


\egwnogap{Mi smo Božji narod koji drži zapovijedi. \textbf{Posljednjih pedeset godina svaka faza hereze je bila usmjerena na nas, kako bi srušila \underline{temeljne principe naše vjere}}. Poruke svakog reda i vrste su nametane adventistima sedmog dana da zauzmu mjesto istine koja je \textbf{točku po točku} posvjedočena čudotvornom silom Gospodnjom. \textbf{Ali međaši koji su nas učinili onim što jesmo trebaju biti sačuvani, i \underline{bit će sačuvani}, kako je Bog naznačio kroz Svoju Riječ i svjedočanstvo Svog Duha. Iz velikog sustava istine kako su ga predstavili Božji glasnici, \underline{ne smije se ukloniti nijedan klin}}.}[Lt232-1903.42; 1903][https://egwwritings.org/?ref=en\_Lt232-1903.42&para=10197.51]


\egwnogap{\textbf{I am called upon by God to stand in defense of the truth that has been given us as we have followed the leading of Him who is the way, the truth, and the life. Let every pioneer in the work adhere firmly to this truth. \underline{The peculiarities of our faith are to be held fast with the grip of faith}.}}[Lt232-1903.43; 1903][https://egwwritings.org/?ref=en\_Lt232-1903.43&para=10197.52]


\egwnogap{\textbf{Bog me poziva da stanem u obranu istine koja nam je dana dok smo slijedili vodstvo Onoga koji je put, istina i život. Neka se svaki pionir u djelu čvrsto drži ove istine. \underline{Posebnosti naše vjere treba držati čvrstim zahvatom vjere}.}}[Lt232-1903.43; 1903][https://egwwritings.org/?ref=en\_Lt232-1903.43&para=10197.52]


\egwnogap{The fables that at the present time are being framed by some medical missionary workers are not to be regarded as truth. \textbf{\underline{Their true origin will ere long be revealed.}} \textbf{It will be seen that they were formed under the subtle power of the great apostate, who works as an angel of light, controlling minds by deceptions so concealed that he seeks by them to deceive if possible the very elect}.}[Lt232-1903.44; 1903][https://egwwritings.org/?ref=en\_Lt232-1903.44&para=10197.53]


\egwnogap{Bajke koje trenutno smišljaju neki zdravstveno-misionarski radnici ne treba smatrati istinom. \textbf{\underline{Njihovo pravo porijeklo uskoro će biti otkriveno.}} \textbf{Vidjet će se da su one formirane pod suptilnom moći velikog otpadnika, koji djeluje kao anđeo svjetlosti, kontrolirajući umove obmanama tako prikrivenim da njima nastoji prevariti, ako je moguće, i same izabrane}.}[Lt232-1903.44; 1903][https://egwwritings.org/?ref=en\_Lt232-1903.44&para=10197.53]


\egwnogap{What influence but that of the deceiver could lead men at \textbf{this stage of our history to work in an underhand, powerful way to tear down \underline{the foundations of our faith}—\underline{the foundations which were laid at the beginning of our work} by prayerful study of the Word and by revelation}. \textbf{\underline{Upon these foundations we have been building for the last fifty years}}. Shall a new foundation be built up by men to whom God has not granted the special experience He has granted to the men whom He ordained to establish the foundations of our faith? \textbf{The men who are striving to build up this false foundation may suppose that they have found a new way, and that they can lay a stronger foundation than that which has been laid. \underline{But this is a great deception}. \underline{Other foundation can no man lay than that which has been laid}.}}[Lt232-1903.45; 1903][https://egwwritings.org/?ref=en\_Lt232-1903.45&para=10197.54]


\egwnogap{Kakav utjecaj osim onoga od varalice bi mogao navesti ljude u \textbf{ovom dijelu povijesti, na nepošten, snažan način u srušiti \underline{temelje naše vjere}—\underline{temelj koji je bio postavljen na početku našeg rada} sa molitvom i proučavanjem Riječi i Otkrivenja}. \textbf{\underline{Na tom temelju smo gradili posljednjih pedeset godina}}. Hoće li novi temelj graditi ljudi kojima Bog nije dao posebno iskustvo koje je dao ljudima koje je odredio da uspostave temelje naše vjere? \textbf{Ljudi koji se trude izgraditi ovaj lažni temelj možda pretpostavljaju da su pronašli novi put i da mogu položiti jači temelj od onoga koji je već položen. \underline{Ali ovo je velika obmana}. \underline{Nitko ne može položiti drugi temelj osim onoga koji je već položen}.}}[Lt232-1903.45; 1903][https://egwwritings.org/?ref=en\_Lt232-1903.45&para=10197.54]


\egwnogap{I am instructed to say to our people that in the past many have undertaken the building of a new faith, the establishment of new principles. But how long did their building stand? It soon fell to pieces; \textbf{for it was not founded upon the Rock}.}[Lt232-1903.46; 1903][https://egwwritings.org/?ref=en\_Lt232-1903.46&para=10197.55]


\egwnogap{Upućena sam da kažem našem narodu da su u prošlosti mnogi poduzeli gradnju nove vjere, uspostavu novih principa. Ali koliko dugo je njihova građevina stajala? Ubrzo se raspala; \textbf{jer nije bila utemeljena na Stijeni}.}[Lt232-1903.46; 1903][https://egwwritings.org/?ref=en\_Lt232-1903.46&para=10197.55]


\egwnogap{Did not the first disciples have to meet the sayings of men? Did they not have to listen to false theories and then stand firm, having done all, to stand, saying, ‘Other foundation can no man lay than that which is laid’? One class after another arose with false doctrines, because men were so little acquainted with God.}[Lt232-1903.47; 1903][https://egwwritings.org/?ref=en\_Lt232-1903.47&para=10197.56]


\egwnogap{Zar se prvi učenici nisu morali suočiti s ljudskim predajama? Zar nisu morali slušati lažne teorije i zatim čvrsto stajati, učinivši sve, da opstanu, govoreći: ‘Nitko ne može položiti drugi temelj osim onoga koji je položen’? Jedna klasa za drugom se pojavljivala s lažnim doktrinama, jer su ljudi tako malo poznavali Boga.}[Lt232-1903.47; 1903][https://egwwritings.org/?ref=en\_Lt232-1903.47&para=10197.56]


\egwnogap{\textbf{My brethren and sisters, study the thirteenth, fourteenth, fifteenth, sixteenth, and seventeenth chapters of John. The words of these chapters explain themselves. ‘This is life eternal,’ Christ declared, ‘that they might know Thee the only true God, and Jesus Christ, whom Thou hast sent.’ \underline{In these words the personality of God and of His Son is clearly spoken of.} \underline{The personality of the one does not do away with the necessity for the personality of the other}.}}[Lt232-1903.48; 1903][https://egwwritings.org/?ref=en\_Lt232-1903.48&para=10197.57]


\egwnogap{\textbf{Moja braćo i sestre, proučavajte trinaesto, četrnaesto, petnaesto, šesnaesto i sedamnaesto poglavlje Ivana. Riječi ovih poglavlja same sebe objašnjavaju. ‘Ovo je život vječni,’ izjavio je Krist, ‘da upoznaju tebe, jedinoga istinskog Boga, i onoga koga si poslao - Isusa Krista.’ \underline{U ovim riječima se jasno govori o ličnosti Boga i Njegovog Sina.} \underline{Ličnost jednoga ne uklanja potrebu za ličnošću drugoga}.}}[Lt232-1903.48; 1903][https://egwwritings.org/?ref=en\_Lt232-1903.48&para=10197.57]


\egwnogap{God is never to be understood by any human being. His ways and His works are past finding out. In regard to the revelations \textbf{that He has made of Himself in His Word, we may talk}. \textbf{But when it comes to talking or writing of God’s person and presence, let us say, ‘Thou art God, and Thy ways are past finding out.’}}[Lt232-1903.49; 1903][https://egwwritings.org/?ref=en\_Lt232-1903.49&para=10197.58]


\egwnogap{Boga nikada ne može u potpunosti shvatiti nijedno ljudsko biće. Njegovi putovi i Njegova djela su nedokučivi. O objavama \textbf{koje je dao o Sebi u Svojoj Riječi, možemo govoriti}. \textbf{Ali kada dođe do razgovora ili pisanja o Božjoj osobi i prisutnosti, recimo: ‘Ti si Bog, i Tvoji su putovi nedokučivi.’}}[Lt232-1903.49; 1903][https://egwwritings.org/?ref=en\_Lt232-1903.49&para=10197.58]


\egwnogap{It is sacrilegious to put into the minds of young or old the \textbf{seeds of \underline{speculation} regarding this subject}. Such seeds, planted and left to grow, will spring up and \textbf{bring forth a harvest of \underline{infidel sentiments}}. I give this warning to all. \textbf{We want no such sophistry as that presented in Living Temple}. There are excellent things in the book. But there are also tares among the wheat. The book contains many correct ideas, but it contains also statements that will do harm. Those who accept the chaff for the wheat will find themselves losing their sense of God’s greatness and bringing Him into cheap commonness. This is the work of the great deceiver. \textbf{Our brethren are not to be called from their work to study the question of where God is and what He is. We are not to dare to engage in this discussion, lest we be destroyed.} When the ark of God was being taken from the land of the Philistines to the camp of Israel, curiosity led the men of Bethshemesh to look into it. God was displeased, and many were smitten with death.}[Lt232-1903.50; 1903][https://egwwritings.org/?ref=en\_Lt232-1903.50&para=10197.59]


\egwnogap{Svetogrđe je usađivati u umove mladih ili starih \textbf{sjeme \underline{spekulacija} o ovoj temi}. Takvo sjeme, posijano i ostavljeno da raste, izniknuti će i \textbf{donijeti žetvu \underline{nevjerničkih sentimenata}}. Dajem ovo upozorenje svima. \textbf{Ne želimo takve sofizme kakvi su predstavljeni u Živom Hramu}. U knjizi ima izvrsnih stvari. Ali ima i kukolja među pšenicom. Knjiga sadrži mnoge ispravne ideje, ali sadrži i izjave koje će nanijeti štetu. Oni koji prihvate pljevu umjesto pšenice, shvatit će da gube osjećaj Božje veličine i svode Ga na jeftinu običnost. To je djelo velikog varalice. \textbf{Našu braću ne treba pozivati s njihovog posla da proučavaju pitanje gdje je Bog i što je On. Ne smijemo se usuditi upustiti u ovu raspravu, da ne bismo bili uništeni.} Kada se Božji kovčeg prenosio iz zemlje Filistejaca u izraelski tabor, znatiželja je navela ljude iz Bet Šemeša da pogledaju u njega. Bog je bio nezadovoljan i mnogi su bili pogođeni smrću.}[Lt232-1903.50; 1903][https://egwwritings.org/?ref=en\_Lt232-1903.50&para=10197.59]


\egwnogap{\textbf{Let us talk of Christ, His preexistence, His humble ministry, His mighty power, His prospective personal glory in the heavenly courts. The Son of God restores to life whom He will. ‘All that the Father hath is Mine,’ He says. ‘I and My Father are one.’ He has greatness, present and prospective, that baffles human conception. He encircles the race with His long human arm, while with His divine arm He grasps the throne of the Infinite}.}[Lt232-1903.51; 1903][https://egwwritings.org/?ref=en\_Lt232-1903.51&para=10197.60]


\egwnogap{\textbf{Govorimo o Kristu, Njegovom preegzistencija, Njegovoj poniznoj službi, Njegovoj moćnoj sili, Njegovoj budućoj osobnoj slavi na nebeskim dvorovima. Sin Božji vraća u život koga On hoće. ‘Sve što ima Otac, moje je,’ kaže On. ‘Ja i Otac jedno smo.’ On ima veličinu, sadašnju i buduću, koja zbunjuje ljudsko poimanje. On obuhvaća ljudski rod svojom dugom ljudskom rukom, dok svojom božanskom rukom drži prijestolje Beskonačnoga}.}[Lt232-1903.51; 1903][https://egwwritings.org/?ref=en\_Lt232-1903.51&para=10197.60]


\egwnogap{\textbf{There is a knowledge of God and of Christ which all who are saved must have. ‘\underline{This is life eternal},’ Christ says, ‘\underline{that they might know Thee, the only true God, and Jesus Christ whom Thou hast sent}.’ And He says again, ‘If any man will come after Me, let him deny himself, and take up his cross, and follow Me.’ To all who receive Him as their Redeemer, He gives power to become the sons of God. Every one who truly believes in Him will be inspired by faith and raised by the arm of Omnipotence.}}[Lt232-1903.52; 1903][https://egwwritings.org/?ref=en\_Lt232-1903.52&para=10197.61]


\egwnogap{\textbf{Postoji spoznaja Boga i Krista koju svi koji će biti spašeni moraju imati. ‘\underline{Ovo je život vječni},’ kaže Krist, ‘\underline{da upoznaju tebe, jedinoga istinskog Boga, i onoga koga si poslao - Isusa Krista}.’ I On ponovno kaže: ‘Ako tko hoće ići za mnom, neka se odrekne samoga sebe, neka uzme svoj križ i neka me slijedi.’ Svima koji Ga prime kao svog Otkupitelja, On daje moć da postanu sinovi Božji. Svatko tko istinski vjeruje u Njega bit će nadahnut vjerom i uzdignut rukom Svemogućega.}}[Lt232-1903.52; 1903][https://egwwritings.org/?ref=en\_Lt232-1903.52&para=10197.61]


\egwnogap{\textbf{Those who do not receive in faith God’s plan for redeeming the race} do despite to the Spirit of grace, and at the last great day their sentence will be, ‘Depart from Me.’ They have hated righteousness and fostered iniquity, and they must be banished forever from the presence of God, exiled from happiness to death—eternal death.}[Lt232-1903.53; 1903][https://egwwritings.org/?ref=en\_Lt232-1903.53&para=10197.62]


\egwnogap{\textbf{Oni koji s vjerom ne prihvate Božji plan za otkupljenje ljudskog roda} vrijeđaju Duha milosti, i na posljednji veliki dan njihova presuda će biti: ‘Odlazite od mene.’ Oni su mrzili pravednost i gajili bezakonje, i moraju biti zauvijek prognani iz Božje prisutnosti, protjerani iz sreće u smrt - vječnu smrt.}[Lt232-1903.53; 1903][https://egwwritings.org/?ref=en\_Lt232-1903.53&para=10197.62]


\egwnogap{Those who in this life love God and cherish the thought of Him will employ their faculties and their talents as faithful stewards, making the very best use of them, but not claiming any reward as their due. \textbf{As they deny self and follow Jesus, lifting the cross, they will find that the cross is light, and that it is a pledge, as they bear it, that they will one day be given a crown of everlasting life}. What will be the glory and the gain and the enjoyment of that eternal life that is to be given to those only for whom it has been prepared? The great joy of the overcomer will be that he is in the presence of Christ. ‘Where I am, there shall also My servant be,’ He declared. And He prayed, ‘Father, I will that they also whom Thou hast given Me be with Me where I am; that they may behold My glory.’ Christ is speaking of the glory of His Father’s presence and His Father’s house. \textbf{The glory that is to be revealed to all who are saved is the glory which Christ had with His Father before the world was—\underline{the unapproachable splendor of their converse together}}. \textbf{The angels were not admitted to the interviews between the Father and the Son when the plan of salvation was laid.} Those human beings who seek to intrude into the secrets of the Most High, who inhabiteth eternity, show their ignorance of spiritual and eternal things. Far better might they, while mercy’s voice is still heard, humble themselves in the dust and plead with God to teach them His ways.}[Lt232-1903.54; 1903][https://egwwritings.org/?ref=en\_Lt232-1903.54&para=10197.63]


\egwnogap{Oni koji u ovom životu vole Boga i cijene misao o Njemu, koristit će svoje sposobnosti i talente kao vjerni upravitelji, koristeći ih na najbolji mogući način, ali ne tražeći nikakvu nagradu kao nešto što im pripada. \textbf{Dok se odriču sebe i slijede Isusa, noseći križ, otkrit će da je križ lagan i da je on, dok ga nose, zalog da će im jednoga dana biti dana kruna vječnoga života}. Kakva će biti slava, dobitak i užitak tog vječnog života koji će biti dan samo onima za koje je pripremljen? Velika radost pobjednika bit će to što je u Kristovoj prisutnosti. ‘Gdje sam ja, ondje će biti i moj sluga,’ izjavio je On. I molio je: ‘Oče, hoću da i oni koje si mi dao budu gdje sam ja, da gledaju moju slavu.’ Krist govori o slavi prisutnosti svog Oca i Očeve kuće. \textbf{Slava koja će biti otkrivena svima koji su spašeni je slava koju je Krist imao sa svojim Ocem prije nego što je svijet postojao—\underline{nepristupačni sjaj njihovog zajedništva}}. \textbf{Anđelima nije bilo dopušteno pristupiti razgovorima između Oca i Sina kada je plan spasenja bio položen.} Ona ljudska bića koja nastoje proniknuti u tajne Svevišnjega, koji nastanjuje vječnost, pokazuju svoje neznanje o duhovnim i vječnim stvarima. Daleko bi bolje bilo da se, dok se još čuje glas milosti, ponize u prahu i mole Boga da ih pouči svojim putovima.}[Lt232-1903.54; 1903][https://egwwritings.org/?ref=en\_Lt232-1903.54&para=10197.63]


\egw{A Timely Warning}


\egw{Pravovremeno upozorenje}


\egw{There are those who have been seeking to carry out their own selfish designs, without regard to the influence that this would have upon the cause and work of God. It is time that such ones felt the inward work of grace upon their hearts, that the medical missionary work may not be grossly misrepresented. Let not our medical missionary workers become so like the world in habit and practice that worldlings will turn away from them with scorn, saying, ‘I am just as good as they are.’ There are instances where the medical missionary work has been so conducted that the name ‘medical missionary’ might better be dropped; for it has been badly misrepresented, and God has been dishonored.}[Lt232-1903.55; 1903][https://egwwritings.org/?ref=en\_Lt232-1903.55&para=10197.65]


\egw{Postoje oni koji su nastojali provoditi svoje sebične planove, ne obazirući se na utjecaj koji bi to imalo na Božje djelo. Vrijeme je da takvi osjete unutarnje djelovanje milosti na svojim srcima, kako zdravstveno-misionarski rad ne bi bio pogrešno predstavljen. Neka naši zdravstveno-misionarski radnici ne postanu toliko slični svijetu u navikama i praksi da će se svjetovnjaci od njih odvratiti s prezirom, govoreći: ‘Ja sam jednako dobar kao i oni.’ Postoje slučajevi gdje je zdravstveno-misionarski rad bio tako vođen da bi bilo bolje ispustiti naziv ‘zdravstveno-misionarski’; jer je bio loše predstavljen, a Bog je bio obeščašćen.}[Lt232-1903.55; 1903][https://egwwritings.org/?ref=en\_Lt232-1903.55&para=10197.65]


\egwnogap{\textbf{We are living amidst the perils of the last days. Our people must now arouse to the work before them. We are to lift up the standard and proclaim the last message of warning to a perishing world. Those who have a knowledge of the truth for this time are now to hold firmly aloft the banner bearing the inscription, ‘The commandments of God and the faith of Jesus.’ }[Revelation 14:12.]}[Lt232-1903.56; 1903][https://egwwritings.org/?ref=en\_Lt232-1903.56&para=10197.66]


\egwnogap{\textbf{Živimo usred opasnosti posljednjih dana. Naš narod se sada mora probuditi za posao koji je pred njima. Moramo podići zastavu i objaviti posljednju poruku upozorenja svijetu koji propada. Oni koji imaju spoznaju istine za sadašnje vrijeme sada trebaju čvrsto držati visoko zastavu s natpisom, ‘Zapovijedi Božje i vjera Isusova.’ }[Otkrivenje 14:12.]}[Lt232-1903.56; 1903][https://egwwritings.org/?ref=en\_Lt232-1903.56&para=10197.66]


\egwnogap{I ask my ministering brethren to examine themselves, to see whether they are in the faith or not. \textbf{If they accept the spiritualistic representations made in Living Temple, their feet will soon be treading in forbidden paths. These representations are the Alpha of doctrines that would lead far away from the truth as we have received it from the Word of God}. \textbf{\underline{The acceptation of these sentiments will result in a weak, wavering faith}}. \textbf{If this is the teaching that is to be given in the medical missionary work, it will not be long before we have no foundation upon which to plant our feet}. \textbf{I am bidden to say that these erroneous sentiments are the sentiments of the wily foe} and should not be presented to any of our youth who are seeking to gain an education in medical missionary lines. For the sake of these youth, I speak decidedly.}[Lt232-1903.57; 1903][https://egwwritings.org/?ref=en\_Lt232-1903.57&para=10197.67]


\egwnogap{Molim moju braću u službi da ispitaju sebe, da vide jesu li u vjeri ili nisu. \textbf{Ako prihvate spiritističke prikaze iznesene u Živom Hramu, njihove će noge uskoro kročiti zabranjenim stazama. Ovi prikazi su Alfa doktrina koje bi odvele daleko od istine kakvu smo primili iz Božje Riječi}. \textbf{\underline{Prihvaćanje ovih sentimenata rezultirat će slabom, kolebljivom vjerom}}. \textbf{Ako je to učenje koje se treba dati u zdravstveno-misionarskom radu, neće dugo potrajati prije nego što ne budemo imali temelja na koji bismo stali}. \textbf{Naloženo mi je da kažem da ovi pogrešni sentimenti su sentimenti lukavog neprijatelja} i ne bi se trebali predstavljati nijednom od naših mladih koji traže obrazovanje u zdravstveno-misionarskim pravcima. Radi ovih mladih, govorim odlučno.}[Lt232-1903.57; 1903][https://egwwritings.org/?ref=en\_Lt232-1903.57&para=10197.67]


\egwnogap{\textbf{The expiring faith of the people of God \underline{must have a resurrection}}. \textbf{The exaltation of human reason has begun its work among us and has gone altogether too far}. Human reason is placed where divine, sanctifying truth should be. \textbf{Christ is waiting to kindle faith and love in the hearts of His people}. \textbf{Let not erroneous theories receive countenance from the people who ought to be standing firm on the platform of eternal truth.} \textbf{God calls upon us to hold firmly to \underline{the fundamental principles that are based upon unquestionable authority}}. He calls upon us to study the words and works of Christ, the greatest missionary that this world has ever known.}[Lt232-1903.58; 1903][https://egwwritings.org/?ref=en\_Lt232-1903.58&para=10197.68]


\egwnogap{\textbf{Umirujuća vjera Božjeg naroda \underline{mora doživjeti uskrsnuće}}. \textbf{Uzdizanje ljudskog razuma započelo je svoje djelo među nama i otišlo je predaleko}. Ljudski razum je postavljen tamo gdje bi trebala biti božanska, posvećujuća istina. \textbf{Krist čeka da zapali vjeru i ljubav u srcima svog naroda}. \textbf{Neka pogrešne teorije ne prime podršku od ljudi koji bi trebali čvrsto stajati na platformi vječne istine.} \textbf{Bog nas poziva da se čvrsto držimo \underline{fundamentalnih principa koji se zasnovani na neupitnom autoritetu}}. On nas poziva da proučavamo riječi i djela Krista, najvećeg misionara kojeg je ovaj svijet ikada upoznao.}[Lt232-1903.58; 1903][https://egwwritings.org/?ref=en\_Lt232-1903.58&para=10197.68]


\egwnogap{\textbf{When the mind of a teacher of truth becomes in any way divorced from plain, self-denying gospel truth, he is prepared to receive fanciful sentiments called truth. Arrayed in the garments of light, these sentiments are presented to others, and too often they find favor. I am instructed to say to the members of our churches, Keep away from spiritualistic ideas}. \textbf{We are not dealing in fables}. God forbid that fables in the disguise of truth shall be presented to our people. \textbf{God forbid that any among us shall build upon the sand}.}[Lt232-1903.59; 1903][https://egwwritings.org/?ref=en\_Lt232-1903.59&para=10197.69]


\egwnogap{\textbf{Kada se um učitelja istine na bilo koji način odvoji od jasne, samoodricajuće evanđeoske istine, on je spreman primiti maštovite sentimente koji se nazivaju istinom. Odjeveni u odjeću svjetlosti, ti se sentimenti predstavljaju drugima, i prečesto nailaze na odobravanje. Upućena sam da kažem članovima naših crkava, Držite se podalje od spiritističkih ideja}. \textbf{Mi se ne bavimo bajkama}. Ne dao Bog da se bajke prerušene u istinu predstavljaju našem narodu. \textbf{Ne dao Bog da itko među nama gradi na pijesku}.}[Lt232-1903.59; 1903][https://egwwritings.org/?ref=en\_Lt232-1903.59&para=10197.69]


\egwnogap{\textbf{The Lord has given us a clear, distinct message of truth for this time}. Let us proclaim this message. \textbf{Let us study the teaching of Christ}, and present what He has commanded us to present. \textbf{He who launches out in his own wisdom to preach strange things, which God has not given him, finds minds ready to be leavened with the new ideas that he has to present}. \textbf{Satan follows up the work that he does, and the efforts of the true servants of God are made much harder}. \textbf{The advancement of His cause is hindered, and His Spirit is grieved}.}[Lt232-1903.60; 1903][https://egwwritings.org/?ref=en\_Lt232-1903.60&para=10197.70]


\egwnogap{\textbf{Gospodin nam je dao jasnu, određenu poruku istine za ovo vrijeme}. Objavljujmo tu poruku. \textbf{Proučavajmo Kristovo učenje}, i predstavljajmo ono što nam je On zapovjedio da predstavimo. \textbf{Onaj tko se u svojoj vlastitoj mudrosti upušta u propovijedanje čudnih stvari, koje mu Bog nije dao, nalazi umove spremne da budu ukvasani novim idejama koje on ima za predstaviti}. \textbf{Sotona slijedi rad koji on čini, i napori istinskih Božjih slugu postaju mnogo teži}. \textbf{Napredak Njegove stvari je ometan, a Njegov Duh je ožalošćen}.}[Lt232-1903.60; 1903][https://egwwritings.org/?ref=en\_Lt232-1903.60&para=10197.70]


We pray that God will speak decisively clear into everyone's heart reading these warnings, to keep from stepping off the foundation of our faith. God calls upon us to hold firmly to the \emcap{Fundamental Principles} that are based upon unquestionable authority.


Molimo se da Bog govori odlučno jasno u srce svakoga tko čita ova upozorenja, te da ne odstupa od temelja naše vjere. Bog nas poziva da se čvrsto držimo \emcap{Fundamentalnih Principa} koji su zasnovani na neupitnom autoritetu.
