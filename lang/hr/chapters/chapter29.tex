\chapter{Sjećanje na Njegovo učenje u našoj prošlosti}

Privodimo naše putovanje u ovom proučavanju svojemu kraju. Započeli smo ovo putovanje snažnom objavom onoga što čini temelj naše vjere. Upoznali smo se s poviješću i iskustvom naših pionira u osnivanju Crkve Adventista Sedmoga Dana. Vidjeli smo misiju i svrhu koju im je Bog dao u objavi trostruke anđeoske vijesti cijelom svijetu. Te vijesti su isprepletene sa svim važnim doktrinama Biblije. Te doktrine su ono što su naši pioniri nazivali \emcap{fundamentalnim principima}, ili stupovima naše vjere. Te doktrine predstavljaju temelj naše vjere.

Jednom kada se upoznamo s \emcap{Fundamentalnim Principima} koja smo imali na početku, prepoznajemo njihovu razliku u odnosu na naša trenutna Temeljna Vjerovanja, osobito u pitanju “\textit{tko je Bog}”? Osim toga, današnja doktrina o Bogu izostavlja razumijevanje Njegove ličnosti. Drugim riječima, izostavlja razumijevanje \textit{kvalitete ili stanja} koja Boga \textit{čine osobom}. Ova promjena u našoj doktrini postaje važna u svjetlu prve anđeoske vijesti, koja se odnosi na Boga kojeg štujemo. Je li Bog kojeg štujemo trojedini Bog, ili je On jedan Bog, Otac, Pradavni, osobno i duhovno Biće?

Promjena u našem razumijevanju tko je Bog za Adventiste sedmog dana nije bila trenutačna; trebalo je mnogo godina kontroverzi da bismo došli do našeg sadašnjeg stajališta. U ovim proučavanjima nismo zalazili u povijest izvan života Ellen White. Vidjeli smo da su se promjene počele događati u njezino vrijeme, kada je dr. Kellogg insinuirao sentiment o \emcap{ličnosti Boga}, koji bi \egwinline{odveo u zabludu umove onih koji nisu temeljito utemeljeni na temeljnim principima sadašnje istine}[SpTB02 51.3; 1904][https://egwwritings.org/read?panels=p417.262]. Insinuirao je sumnju u jasna otkrivenja o \emcap{ličnosti Boga} i Njegova Sina, Isusa Krista. Njegovi sentimenti su bili oštro ukoreni od sestre White koja je dala snažna upozorenja za crkvu, kako bi izbjegli put sumnje u jednostavnu istinu izraženu u \emcap{Fundamentalnim Principima}—da je jedan Bog osobno, duhovno biće, a Krist je Njegov Sin, \egwinline{rođen na savršenu sliku Očeve osobe}[ST May 30, 1895, par. 3; 1895][https://egwwritings.org/read?panels=p820.12891]. Tako je sestra White čvrsto branila prve dvije točke \emcap{Fundamentalnih Principa}.

Kao i u vrijeme dr. Kellogga, kada su mnoga braća odstupali od jednostavnosti Kristovog učenja, tako je i danas. Sestra White je prorekla da će doći do promjene u razumijevanju \emcap{ličnosti Boga} u našoj crkvi i da će biti potrebno ponovno uspostavljanje našeg starog temelja vjere. Hoće li se \emcap{Fundamentalni Principi} ponovno uspostaviti među nama? Ishod toga potpuno ovisi o svakom pojedincu koji čini tijelo Crkve Adventista Sedmoga Dana. Danas, u ovo doba, je vrijeme kada će se ta obnova dogoditi. Upozorenja i ukori izrečeni perom Duha Proroštva nikada nisu bila relevantnija nego danas. Bog je konačni ishod stavio u tvoje ruke. Ako su te ta upozorenja dotakla u srce tvoje duše, Bog te poziva da čvrsto stojiš na platformi vječne Istine. On te poziva da se čvrsto držiš \emcap{Fundamentalnih Principa}, koji su zasnovani na neupitnom autoritetu.

U nastavku predstavljamo dio pisma sestre White upućenog dr. Kelloggu, u kojem nalazimo ozbiljno upozorenje za nas danas u ponovnom uspostavljanju temelja naše vjere. Kada se upoznamo s istinom o \emcap{ličnosti Boga} i njezinom kontroverzom u odnosu na doktrinu o Trojstvu, sljedeće pismo sjaji u novom svjetlu, s porukama i načelima koji su vrijedni za nas danas, kako bismo znali kako se ponašati u našoj trenutnoj situaciji.

\egw{\textbf{Temelj Naše Vjere}}

\egw{U vezi s knjigom Živi Hram, \textbf{bila sam upućena od strane Nebeskog glasnika} da \textbf{su neka rasuđivanja u ovoj knjizi netočna i da će ta razmišljanja odvesti u zabludu umove onih koji nisu temeljito utemeljeni \underline{na temeljnim principima} sadašnje istine.} \textbf{Ona uvode ono što je ništa više nego obična nagađanja vezano za \underline{ličnost Boga} i \underline{gdje je Njegova prisutnost}}. Nitko na ovoj zemlji nema pravo nagađati o tom pitanju. ‘Što je sakriveno, pripada GOSPODU, Bogu našemu, a što je otkriveno, pripada nama i sinovima našim dovijeka.’}[Lt232-1903.39; 1903][https://egwwritings.org/read?panels=p10197.48]

\egwnogap{\textbf{Gospodin me je ovlastio da kažem, Sentimenti sadržani u Živom hramu u vezi ličnosti Boga \underline{su suprotni istini koju nam je Bog dao}}. \textbf{Istina za ovo vrijeme sada se treba iznijeti pred ljude.} Naša braća i sestre u svakoj crkvi i na svakom mjestu trebaju pažljivo čuvati svoje umove da ne budu zaokupljeni stvarima koje ih odvlače od vječnih stvari. Neprijatelj će koristiti neke izjave iz Živog Hrama da iskuša neke kao što je iskušao Adama i Evu u Edenu. \textbf{Upozoravam našu braću da ne ulaze u rasprave o prisutnosti i ličnosti Boga. \underline{Izjave dane u Živom Hramu u vezi ove točke su netočne}.} Sveto Pismo korišteno za potkrepljivanje tamo iznesene doktrine je pogrešno primijenjeno.}[Lt232-1903.40; 1903][https://egwwritings.org/read?panels=p10197.49]

\egwnogap{Upozoren sam da ne ulazim u kontroverzu u vezi s pitanjem koje će se pojaviti oko ovih stvari, \textbf{jer bi rasprava mogla navesti ljude da pribjegnu \underline{izvrtanju}, i njihovi umovi bili bi odvedeni daleko od istine Božje Riječi \underline{prema pretpostavkama i nagađanjima}. Što se više raspravlja o tim izmišljenim teorijama, to će ljudi manje znati o Bogu i istini koja posvećuje dušu}.}[Lt232-1903.41; 1903][https://egwwritings.org/read?panels=p10197.50]

\egwnogap{Mi smo Božji narod koji drži zapovijedi. \textbf{Posljednjih pedeset godina svaka faza hereze je bila usmjerena na nas, kako bi srušila \underline{temeljne principe naše vjere}}. Poruke svakog reda i vrste su nametane adventistima sedmog dana da zauzmu mjesto istine koja je \textbf{točku po točku} posvjedočena čudotvornom silom Gospodnjom. \textbf{Ali međaši koji su nas učinili onim što jesmo trebaju biti sačuvani, i \underline{bit će sačuvani}, kako je Bog naznačio kroz Svoju Riječ i svjedočanstvo Svog Duha. Iz velikog sustava istine kako su ga predstavili Božji glasnici, \underline{ne smije se ukloniti nijedan klin}}.}[Lt232-1903.42; 1903][https://egwwritings.org/read?panels=p10197.51]

\egwnogap{\textbf{Bog me poziva da stanem u obranu istine koja nam je dana dok smo slijedili vodstvo Onoga koji je put, istina i život. Neka se svaki pionir u djelu čvrsto drži ove istine. \underline{Posebnosti naše vjere treba držati čvrstim zahvatom vjere}.}}[Lt232-1903.43; 1903][https://egwwritings.org/read?panels=p10197.52]

\egwnogap{Bajke koje trenutno smišljaju neki zdravstveno-misionarski radnici ne treba smatrati istinom. \textbf{\underline{Njihovo pravo porijeklo uskoro će biti otkriveno.}} \textbf{Vidjet će se da su one formirane pod suptilnom moći velikog otpadnika, koji djeluje kao anđeo svjetlosti, kontrolirajući umove obmanama tako prikrivenim da njima nastoji prevariti, ako je moguće, i same izabrane}.}[Lt232-1903.44; 1903][https://egwwritings.org/read?panels=p10197.53]

\egwnogap{Kakav utjecaj osim onoga od varalice bi mogao navesti ljude u \textbf{ovom dijelu povijesti, na nepošten, snažan način srušiti \underline{temelje naše vjere}—\underline{temelj koji je bio postavljen na početku našeg rada} sa molitvom i proučavanjem Riječi i Otkrivenja}. \textbf{\underline{Na tom temelju smo gradili posljednjih pedeset godina}}. Hoće li novi temelj graditi ljudi kojima Bog nije dao posebno iskustvo koje je dao ljudima koje je odredio da uspostave temelje naše vjere? \textbf{Ljudi koji se trude izgraditi ovaj lažni temelj možda pretpostavljaju da su pronašli novi put i da mogu položiti jači temelj od onoga koji je već položen. \underline{Ali ovo je velika obmana}. \underline{Nitko ne može položiti drugi temelj osim onoga koji je već položen}.}}[Lt232-1903.45; 1903][https://egwwritings.org/read?panels=p10197.54]

\egwnogap{I am instructed to say to our people that in the past many have undertaken the building of a new faith, the establishment of new principles. But how long did their building stand? It soon fell to pieces; \textbf{for it was not founded upon the Rock}.}[Lt232-1903.46; 1903][https://egwwritings.org/read?panels=p10197.55]


\egwnogap{Upućena sam da kažem našem narodu da su u prošlosti mnogi pokušali izgraditi novu vjeru, uspostaviti nove principe. Ali koliko dugo je njihova građevina stajala? Ubrzo se raspala; \textbf{jer nije bila utemeljena na Stijeni}.}[Lt232-1903.46; 1903][https://egwwritings.org/read?panels=p10197.55]

\egwnogap{Zar se prvi učenici nisu morali suočiti s ljudskim predajama? Zar nisu morali slušati lažne teorije i zatim čvrsto stajati, učinivši sve, da opstanu, govoreći: ‘Nitko ne može položiti drugi temelj osim onoga koji je položen’? Jedna klasa za drugom se pojavljivala s lažnim doktrinama, jer su ljudi tako malo poznavali Boga.}[Lt232-1903.47; 1903][https://egwwritings.org/read?panels=p10197.56]

\egwnogap{\textbf{Moja braćo i sestre, proučavajte trinaesto, četrnaesto, petnaesto, šesnaesto i sedamnaesto poglavlje Ivana. Riječi ovih poglavlja same sebe objašnjavaju. ‘Ovo je život vječni,’ izjavio je Krist, ‘da upoznaju tebe, jedinoga istinskog Boga, i onoga koga si poslao - Isusa Krista.’ \underline{U ovim riječima se jasno govori o ličnosti Boga i Njegovog Sina.} \underline{Ličnost jednoga ne uklanja potrebu za ličnošću drugoga}.}}[Lt232-1903.48; 1903][https://egwwritings.org/read?panels=p10197.57]

\egwnogap{Boga nikada ne može u potpunosti shvatiti nijedno ljudsko biće. Njegovi putovi i Njegova djela su nedokučivi. O objavama \textbf{koje je dao o Sebi u Svojoj Riječi, možemo govoriti}. \textbf{Ali kada dođe do razgovora ili pisanja o Božjoj osobi i prisutnosti, recimo: ‘Ti si Bog, i Tvoji su putovi nedokučivi.’}}[Lt232-1903.49; 1903][https://egwwritings.org/read?panels=p10197.58]

\egwnogap{Svetogrđe je usađivati u umove mladih ili starih \textbf{sjeme \underline{spekulacija} o ovoj temi}. Takvo sjeme, posijano i ostavljeno da raste, izniknuti će i \textbf{donijeti žetvu \underline{nevjerničkih sentimenata}}. Dajem ovo upozorenje svima. \textbf{Ne želimo takve sofizme kakvi su predstavljeni u Živom Hramu}. U knjizi ima izvrsnih stvari. Ali ima i kukolja među pšenicom. Knjiga sadrži mnoge ispravne ideje, ali sadrži i izjave koje će nanijeti štetu. Oni koji prihvate pljevu umjesto pšenice, shvatit će da gube osjećaj Božje veličine i svode Ga na jeftinu običnost. To je djelo velikog varalice. \textbf{Našu braću ne treba pozivati s misijskog djela da proučavaju pitanje gdje je Bog i što je On. Ne smijemo se usuditi upustiti u ovu raspravu, da ne bismo bili uništeni.} Kada se Božji kovčeg prenosio iz zemlje Filistejaca u izraelski tabor, znatiželja je navela ljude iz Bet Šemeša da pogledaju u njega. Bog je bio nezadovoljan i mnogi su bili pogođeni smrću.}[Lt232-1903.50; 1903][https://egwwritings.org/read?panels=p10197.59]

\egwnogap{\textbf{Govorimo o Kristu, Njegovom preegzistencija, Njegovoj poniznoj službi, Njegovoj moćnoj sili, Njegovoj budućoj osobnoj slavi na nebeskim dvorovima. Sin Božji vraća u život koga On hoće. ‘Sve što ima Otac, moje je,’ kaže On. ‘Ja i Otac jedno smo.’ On ima veličinu, sadašnju i buduću, koja zbunjuje ljudsko poimanje. On obuhvaća ljudski rod svojom dugom ljudskom rukom, dok svojom božanskom rukom drži prijestolje Beskonačnoga}.}[Lt232-1903.51; 1903][https://egwwritings.org/read?panels=p10197.60]

\egwnogap{\textbf{Postoji spoznaja Boga i Krista koju svi koji će biti spašeni moraju imati. ‘\underline{Ovo je život vječni},’ kaže Krist, ‘\underline{da upoznaju tebe, jedinoga istinskog Boga, i onoga koga si poslao - Isusa Krista}.’ I On ponovno kaže: ‘Ako tko hoće ići za mnom, neka se odrekne samoga sebe, neka uzme svoj križ i neka me slijedi.’ Svima koji Ga prime kao svog Otkupitelja, On daje moć da postanu sinovi Božji. Svatko tko istinski vjeruje u Njega bit će nadahnut vjerom i uzdignut rukom Svemogućega.}}[Lt232-1903.52; 1903][https://egwwritings.org/read?panels=p10197.61]

\egwnogap{\textbf{Oni koji s vjerom ne prihvate Božji plan za otkupljenje ljudskog roda} vrijeđaju Duha milosti, i na posljednji veliki dan njihova presuda će biti: ‘Odlazite od mene.’ Oni su mrzili pravednost i gajili bezakonje, i moraju biti zauvijek prognani iz Božje prisutnosti, protjerani iz sreće u smrt - vječnu smrt.}[Lt232-1903.53; 1903][https://egwwritings.org/read?panels=p10197.62]

\egwnogap{Oni koji u ovom životu vole Boga i njeguju misao o Njemu koristit će svoje sposobnosti i talente kao vjerni upravitelji, koristeći ih na najbolji mogući način, ali ne tražeći nagradu kao nešto što im pripada. \textbf{Dok se odriču sebe i slijede Isusa, noseći križ, otkrit će da je križ lagan i da je, dok ga nose, obećanje da će jednoga dana dobiti krunu vječnoga života}. Kakva će biti slava, dobitak i užitak tog vječnog života koji će biti dan samo onima za koje je pripremljen? Velika radost pobjednika bit će da je u Kristovoj prisutnosti. ‘Gdje sam ja, ondje će biti i moj sluga,’ izjavio je. I molio je: ‘Oče, hoću da i oni koje si mi dao budu sa mnom gdje sam ja; da gledaju moju slavu.’ Krist govori o slavi prisutnosti svoga Oca i Očeva doma. \textbf{Slava koja će biti otkrivena svima koji su spašeni jest slava koju je Krist imao sa svojim Ocem prije nego što je svijeta bilo—\underline{nepristupačni sjaj njihovog zajedništva}}. \textbf{Anđelima nije bilo dopušteno pristupiti razgovorima između Oca i Sina kada je bio postavljen plan spasenja.} Ona ljudska bića koja nastoje proniknuti u tajne Svevišnjega, koji prebiva u vječnosti, pokazuju svoje neznanje o duhovnim i vječnim stvarima. Daleko bi bolje bilo da se, dok se još čuje glas milosti, ponize u prahu i mole Boga da ih pouči svojim putevima.}[Lt232-1903.54; 1903][https://egwwritings.org/read?panels=p10197.63]

\egw{Pravovremeno upozorenje}

\egw{Postoje oni koji su nastojali provoditi svoje sebične planove, ne obazirući se na utjecaj koji bi to imalo na Božje djelo. Vrijeme je da takvi osjete unutarnje djelovanje milosti na svojim srcima, kako zdravstveno-misionarski rad ne bi bio pogrešno predstavljen. Neka naši zdravstveno-misionarski radnici ne postanu toliko slični svijetu u navikama i praksi da će se svjetovnjaci od njih odvratiti s prezirom, govoreći: ‘Ja sam jednako dobar kao i oni.’ Postoje slučajevi gdje je zdravstveno-misionarski rad bio tako vođen da bi bilo bolje ispustiti naziv ‘zdravstveno-misionarski’; jer je bio loše predstavljen, a Bog je bio obeščašćen.}[Lt232-1903.55; 1903][https://egwwritings.org/read?panels=p10197.65]

\egwnogap{\textbf{Živimo usred opasnosti posljednjih dana. Naš narod se sada mora probuditi za posao koji je pred njima. Moramo podići zastavu i objaviti posljednju poruku upozorenja svijetu koji propada. Oni koji imaju spoznaju istine za sadašnje vrijeme sada trebaju čvrsto držati visoko zastavu s natpisom, ‘Zapovijedi Božje i vjera Isusova.’ }[Otkrivenje 14:12.]}[Lt232-1903.56; 1903][https://egwwritings.org/read?panels=p10197.66]


\egwnogap{I ask my ministering brethren to examine themselves, to see whether they are in the faith or not. \textbf{If they accept the spiritualistic representations made in Living Temple, their feet will soon be treading in forbidden paths. These representations are the Alpha of doctrines that would lead far away from the truth as we have received it from the Word of God}. \textbf{\underline{The acceptation of these sentiments will result in a weak, wavering faith}}. \textbf{If this is the teaching that is to be given in the medical missionary work, it will not be long before we have no foundation upon which to plant our feet}. \textbf{I am bidden to say that these erroneous sentiments are the sentiments of the wily foe} and should not be presented to any of our youth who are seeking to gain an education in medical missionary lines. For the sake of these youth, I speak decidedly.}[Lt232-1903.57; 1903][https://egwwritings.org/read?panels=p10197.67]


\egwnogap{Molim moju braću u službi da ispitaju sebe, da vide jesu li u vjeri ili nisu. \textbf{Ako prihvate spiritualističke prikaze iznesene u Živom Hramu, njihove će noge uskoro kročiti zabranjenim stazama. Ovi prikazi su Alfa doktrina koje bi odvele daleko od istine kakvu smo primili iz Božje Riječi}. \textbf{\underline{Prihvaćanje ovih sentimenata rezultirat će slabom, kolebljivom vjerom}}. \textbf{Ako je to učenje koje se treba dati u zdravstveno-misionarskom radu, neće dugo potrajati prije nego što ne budemo imali temelja na koji bismo stali}. \textbf{Naloženo mi je da kažem da ovi pogrešni sentimenti su sentimenti lukavog neprijatelja} i ne bi se trebali predstavljati nijednom od naših mladih koji traže obrazovanje u zdravstveno-misionarskim pravcima. Radi ovih mladih, govorim odlučno.}[Lt232-1903.57; 1903][https://egwwritings.org/read?panels=p10197.67]


\egwnogap{\textbf{The expiring faith of the people of God \underline{must have a resurrection}}. \textbf{The exaltation of human reason has begun its work among us and has gone altogether too far}. Human reason is placed where divine, sanctifying truth should be. \textbf{Christ is waiting to kindle faith and love in the hearts of His people}. \textbf{Let not erroneous theories receive countenance from the people who ought to be standing firm on the platform of eternal truth.} \textbf{God calls upon us to hold firmly to \underline{the fundamental principles that are based upon unquestionable authority}}. He calls upon us to study the words and works of Christ, the greatest missionary that this world has ever known.}[Lt232-1903.58; 1903][https://egwwritings.org/read?panels=p10197.68]


\egwnogap{\textbf{Umirujuća vjera Božjeg naroda \underline{mora doživjeti uskrsnuće}}. \textbf{Uzdizanje ljudskog razuma započelo je svoje djelo među nama i otišlo je predaleko}. Ljudski razum je postavljen tamo gdje bi trebala biti božanska, posvećujuća istina. \textbf{Krist čeka da zapali vjeru i ljubav u srcima svog naroda}. \textbf{Neka pogrešne teorije ne prime podršku od ljudi koji bi trebali čvrsto stajati na platformi vječne istine.} \textbf{Bog nas poziva da se čvrsto držimo \underline{fundamentalnih principa koji se zasnovani na neupitnom autoritetu}}. On nas poziva da proučavamo riječi i djela Krista, najvećeg misionara kojeg je ovaj svijet ikada upoznao.}[Lt232-1903.58; 1903][https://egwwritings.org/read?panels=p10197.68]

\egwnogap{\textbf{Kada se um učitelja istine na bilo koji način odvoji od jasne, samoodricajuće evanđeoske istine, on je spreman primiti maštovite sentimente koji se nazivaju istinom. Odjeveni u odjeću svjetlosti, ti se sentimenti predstavljaju drugima, i prečesto nailaze na odobravanje. Upućena sam da kažem članovima naših crkava, Držite se podalje od spiritualističkih ideja}. \textbf{Mi se ne bavimo bajkama}. Ne dao Bog da se bajke prerušene u istinu predstavljaju našem narodu. \textbf{Ne dao Bog da itko među nama gradi na pijesku}.}[Lt232-1903.59; 1903][https://egwwritings.org/read?panels=p10197.69]

\egwnogap{\textbf{Gospodin nam je dao jasnu, određenu poruku istine za ovo vrijeme}. Objavljujmo tu poruku. \textbf{Proučavajmo Kristovo učenje}, i predstavljajmo ono što nam je On zapovjedio da predstavimo. \textbf{Onaj tko se u svojoj vlastitoj mudrosti upušta u propovijedanje čudnih stvari, koje mu Bog nije dao, nalazi umove spremne da budu ukvasani novim idejama koje on ima za predstaviti}. \textbf{Sotona slijedi rad koji on čini, i napori istinskih Božjih slugu postaju mnogo teži}. \textbf{Napredak Njegove stvari je ometan, a Njegov Duh je ožalošćen}.}[Lt232-1903.60; 1903][https://egwwritings.org/read?panels=p10197.70]

Molimo se da Bog govori odlučno jasno u srce svakoga tko čita ova upozorenja, te da ne odstupa od temelja naše vjere. Bog nas poziva da se čvrsto držimo \emcap{Fundamentalnih Principa} koji su zasnovani na neupitnom autoritetu.
