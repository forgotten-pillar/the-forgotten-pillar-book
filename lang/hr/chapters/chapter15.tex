\chapter{Dr. Kellogg and the Trinity doctrine}


\chapter{Dr. Kellogg i doktrina o Trojstvu}


The key problem with the Kellogg controversy was the sentiments over the \emcap{personality of God}, which were departing from the foundation of our faith, that God established at the beginning of our work. We have been told that \egwinline{Many things of like character will in the future arise}[Ms137-1903.10; 1903][https://egwwritings.org/?ref=en\_Ms137-1903.10&para=9939.17]. In the book, the Living Temple, we see the sentiments regarding the \emcap{personality of God} and where His presence is, which were stepping off of the \emcap{Fundamental Principles}. This step was never supposed to be made! But we raise the question, where was this step heading? We will see the evidence that this step was heading toward the Trinity doctrine. Sister White prophesied that Kellogg’s step would lead toward the Omega heresy. Can we see the connection between Kellogg’s controversy and the Trinity doctrine?


Ključni problem Kellogg-ove kontroverze bio su sentimenti o \emcap{ličnosti Boga}, koji su odstupali od temelja naše vjere, koje je Bog uspostavio na početku našeg rada. Rečeno nam je da \egwinline{Mnoge stvari sličnog karaktera pojaviti će se u budućnosti}[Ms137-1903.10; 1903][https://egwwritings.org/?ref=en\_Ms137-1903.10&para=9939.17]. U knjizi, Živi Hram, vidimo sentimente o \emcap{ličnosti Boga} i Njegovoj prisutnosti, koji su odstupali od \emcap{Fundamentalnih Principa}. Ovaj korak nikada nije trebao biti napravljen! Ali postavljamo pitanje, kamo je ovaj korak vodio? Vidjet ćemo pokazatelje da je ovaj korak vodio prema doktrini o Trojstvu. Sestra White je prorekla da će Kellogg-ov korak voditi prema Omega odpadu. Možemo li vidjeti vezu između Kellogg-ove kontroverze i doktrine o Trojstvu?


In the following section, we want to present you with the connection between Kellogg’s controversy and the doctrine of Trinity. It is important to emphasize that the Living Temple does not contain this doctrine as it is believed today. The main problem with Kellogg’s teaching was the \textit{stepping off} of the \emcap{Fundamental Principles}, which were the foundation of our faith. The information we will present to you reveals that Dr. Kellogg justified his actions in stepping off of the foundation through his belief in the doctrine of Trinity. Therefore, our main focus should not be in recognizing the Trinity doctrine in Kellogg's arguments, but rather in understanding the differences between Kellogg’s teachings and the teachings of the \emcap{Fundamental Principles} regarding \egwinline{the personality of God and where His presence is}[SpTB02 51.3; 1903][https://egwwritings.org/?ref=en\_SpTB02.51.3&para=417.262]. In other words, what were the steps Kellogg made in stepping off of the foundation of our faith? This approach is advocated by the Spirit of Prophecy and it will help us to avoid speculations regarding Kellogg’s motives—it will help us to focus upon the truth. Ellen White tells us that there are many good things written in the Living Temple, but they are mingled with specious, deceptive theories regarding the \emcap{personality of God} and \emcap{of Christ}.


U sljedećem dijelu želimo vam predstaviti vezu između Kelloggove kontroverze i doktrine o Trojstvu. Važno je naglasiti da Živi Hram ne sadrži ovu doktrinu kako se ona danas vjeruje. Glavni problem s Kelloggovim učenjem bio je \textit{odstupanje} od \emcap{Fundamentalnih Principa}, koji su bili temelj naše vjere. Informacije koje ćemo vam predstaviti otkrivaju da je dr. Kellogg opravdao svoje postupke odstupanjem od temelja kroz svoje vjerovanje u doktrinu o Trojstvu. Stoga, naš glavni fokus ne bi trebao biti prepoznavanje doktrine Trojstva u Kelloggovim argumentima, već u razumijevanju razlika između Kelloggovih učenja i učenja \emcap{Fundamentalnih Principa} u vezi s \egwinline{ličnosti Boga i gdje je Njegova prisutnost}[SpTB02 51.3; 1903][https://egwwritings.org/?ref=en\_SpTB02.51.3&para=417.262]. Drugim riječima, koji su to koraci koje je Kellogg napravio odstupajući od temelja naše vjere? Ovaj pristup zagovara Duh Proroštva i pomoći će nam da izbjegnemo spekulacije u vezi s Kelloggovim motivima - pomoći će nam da se usredotočimo na istinu. Ellen White nam govori da u Živom Hramu postoje mnoge dobre stvari, ali su pomiješane sa lažnim, obmanjujućim teorijama o ličnosti Boga i Krista.


\egw{\textbf{The book Living Temple contains specious, \underline{deceptive sentiments regarding the personality of God and of Christ}}. The Lord opened before me the true meaning of these sentiments, showing me that unless they were steadfastly repudiated, they would deceive the very elect. \textbf{Precious truth and beautiful sentiments were woven in with false, misleading theories. Thus truth was used to substantiate the \underline{most dangerous errors}. The precious representations of God are so misconstrued as to appear to uphold falsehoods \underline{originated by the great apostate}. Sentiments that belong to the revealings of God are mingled with specious, deceptive theories of satanic agencies}.}[Lt146-1905.2; 1905][https://egwwritings.org/?ref=en\_Lt146-1905.2&para=9430.8]


\egw{\textbf{Knjiga Živi hram sadrži lažne, \underline{obmanjujuće sentimente o ličnosti Boga i Krista}}. Gospodin mi je otvorio pravo značenje tih sentimenta, pokazujući mi da, ako se čvrsto ne odbace, mogu prevariti čak i odabrane. \textbf{Dragocjena istina i lijepe misli isprepletene su s lažnim, obmanjujućim teorijama}. Tako je istina korištena za potkrepljivanje \underline{najopasnijih zabluda}. Dragocjene predstave o Bogu su tako pogrešno prikazane da izgleda kao da podržavaju \underline{zablude koje je izvorio veliki otpadnik}. Sentimenti koji pripadaju Božjim otkrivenjima pomiješani su s zabludama, obmanjujućim teorijama sotonskih agencija.}[Lt146-1905.2; 1905][https://egwwritings.org/?ref=en\_Lt146-1905.2&para=9430.8]


\egw{In the controversy over these theories \textbf{it has been asserted that I believed and taught the same things} that I have been instructed to condemn in the book Living Temple. \textbf{This I deny}. In the name of Jesus Christ of Nazareth, \textbf{I say that this is not so}.}[Lt146-1905.3; 1905][https://egwwritings.org/?ref=en\_Lt146-1905.3&para=9430.9]


\egw{U kontroverzi oko ovih teorija \textbf{tvrdilo se da sam vjerovala i učila iste stvari} koje sam bila navedena osuditi u knjizi Živi hram. \textbf{To poričem}. U ime Isusa Krista iz Nazareta, \textbf{kažem da to nije tako}.}[Lt146-1905.3; 1905][https://egwwritings.org/?ref=en\_Lt146-1905.3&para=9430.9]


This mixture of truth and error makes the matter difficult. In the eyes of pro-trinitarian scholars, the problem is solely attributed to pantheism, and the evidence of Kellogg’s belief in the Trinity doctrine is interpreted as a belief in a false Trinity\footnote{Woodrow Wilson Whidden, Jerry Moon, John W. Reeve, \textit{The Trinity: Understanding God's Love, His Plan of Salvation, and Christian relationships}, page. 217}; Sister White’s rebuke is attributed to the defense of the “correct” Trinity, which she supposedly believed. Unfortunately, such interpretation does not attribute Sister White’s defense of the \emcap{Fundamental Principles} regarding the \emcap{personality of God} and of Christ, thus it is a misinterpretation of her work. In the following studies, we will look at historical data in the perspective of the Adventist truth on the \emcap{personality of God}, which constituted the foundation of our faith. With this perspective, we believe that the historical data will shine in a new light, and spark honest and constructive dialogue in our church.


Ova mješavina istine i zablude čini stvar teškom. U očima pristaša doktrine o Trojstvu, problem je isključivo pripisan panteizmu, a dokazi Kelloggovog vjerovanja u doktrinu Trojstva tumače se kao vjerovanje u lažno Trojstvo\footnote{Woodrow Wilson Whidden, Jerry Moon, John W. Reeve, \textit{The Trinity: Understanding God's Love, His Plan of Salvation, and Christian relationships}, page. 217}; ukor sestre White pripisuje se obrani "ispravnog" Trojstva, za koje se pretpostavlja da je vjerovala. Nažalost, takva interpretacija ne pripisuje obranu sestre White Fundamentalnih Principa u vezi ličnosti Boga i Krista, stoga je to pogrešno tumačenje njezinog rada. U nastavku pogledat ćemo povijesne podatke iz perspektive adventističke istine o ličnosti Boga, koja je činila temelj naše vjere. S ovom perspektivom, vjerujemo da će povijesni podaci zasjati u novom svjetlu i potaknuti iskren i konstruktivan dijalog u našoj crkvi.


\section*{Correspondence of Dr. Kellogg and Brother Butler}


\section*{Prepiska Dr. Kellogga i brata Butlera}


In the following section we briefly present you with the well-known correspondence between Dr. Kellogg and G. I. Butler over the book, the Living Temple. Here, we see Dr. Kellogg’s objections regarding the controversy. He wrote to Brother Butler:


U nastavku slijedi poznatu prepisku između dr. Kellogga i G. I. Butlera o knjizi, Živi Hram. Ovdje vidimo dr. Kelloggove prigovore vezane za kontroverzu. Pisao je bratu Butleru:


\others{As far as I can fathom, the \textbf{difficulty }which is found \textbf{in ‘The Living Temple’,} \textbf{the whole thing may be simmered down to the question}: \textbf{\underline{Is the Holy Ghost a person}?} You say no. I had supposed the Bible said this for the reason that the personal pronoun ‘he’ is used in speaking of the Holy Ghost. \textbf{Sister White uses the pronoun ‘he’ and has said in so many words that the Holy Ghost is \underline{the third person of the Godhead}}. \textbf{How the Holy Ghost can be the third person and not be a person at all is difficult for me to see}.}[Letter: J. H. Kellogg to G I Butler. Oct 28. 1903][https://static1.squarespace.com/static/554c4998e4b04e89ea0c4073/t/5db9fbc96defed1e45b497a4/1572469707862/1903-10-28-Kellog-to-Butler.pdf]


\others{Koliko ja mogu razumjeti, \textbf{teškoća} koja se pronalazi \textbf{u 'Živom Hramu',} \textbf{može se sumirati u sljedećem pitanju}: \textbf{\underline{Je li Sveti Duh osoba}?} Ti kažeš ne. Ja sam pretpostavljao da to proizlazi iz Svetog Pisma iz razloga što se osobna zamjenica 'on' koristi onda kada se govori o Svetom Duhu. \textbf{Sestra White koristi zamjenicu 'on' i rekla je više puta da je Sveti Duh \underline{treća osoba Božanstva}}. \textbf{Kako Sveti Duh može biti treća osoba, a da uopće ne bude osoba, teško mi je uvidjeti}.}[Letter: J. H. Kellogg to G I Butler. Oct 28. 1903][https://static1.squarespace.com/static/554c4998e4b04e89ea0c4073/t/5db9fbc96defed1e45b497a4/1572469707862/1903-10-28-Kellog-to-Butler.pdf]


According to Dr. Kellogg’s perspective, the whole problem with the book ‘The Living Temple’ comes down to the question “\textit{Is the Holy Ghost a person?}”. Obviously, he does not advocate an impersonal God, as he is often accused of. Moreover, he even believes that the Holy Ghost is a \textit{third person of the Godhead}. Also, he claims that Brother Butler does not believe that the Holy Ghost is a person. The problem obviously lies in the definition of the word \textit{‘person’}. On this point, Kellogg continues:


Prema perspektivi dr. Kellogga, cijeli problem s knjigom 'Živi hram' svodi se na pitanje "\textit{Je li Sveti Duh osoba?}". Očito je da on ne zagovara neosobnog Boga, kako ga se često optužuje. Štoviše, čak vjeruje da je Sveti Duh \textit{treća osoba Božanstva}. Također tvrdi da brat Butler ne vjeruje da je Sveti Duh osoba. Problem očito leži u definiciji riječi \textit{'osoba'}. Na tom pitanju, Kellogg nastavlja:


\others{I believe this Spirit of God to be a personality you don’t. But this is purely a question of definition. \textbf{I believe the Spirit of God is a personality}; you say, No, it is not a personality. Now the only reason why we differ is because we \textbf{differ in our ideas as to \underline{what a personality is}}. \textbf{Your idea of personality is perhaps that of \underline{semblance to a person} or a human being}.}[Ibid.][https://static1.squarespace.com/static/554c4998e4b04e89ea0c4073/t/5db9fbc96defed1e45b497a4/1572469707862/1903-10-28-Kellog-to-Butler.pdf]


\others{Ja vjerujem da je ovaj Duh Božji ličnost, u što ti ne vjeruješ. Ali ovo je čisto pitanje definicije. \textbf{Ja vjerujem da je Duh Božji ličnost}; ti kažeš: Ne, nije ličnost. Sada je jedini razlog zbog kojeg se mimoilazimo u našim pojmovima o tome što je to ličnost. \textbf{Tvoja ideja ličnosti je možda \underline{izvanjska sličnost s osobom} ili ljudskim bićem}.}[Ibid.][https://static1.squarespace.com/static/554c4998e4b04e89ea0c4073/t/5db9fbc96defed1e45b497a4/1572469707862/1903-10-28-Kellog-to-Butler.pdf]


Brother Butler replied:


Brat Butler odgovara:


\others{\textbf{So far as Sister White and you being in perfect agreement, I shall have to leave that entirely between you and Sister White. \underline{Sister White says there is not perfect agreement; you claim there is}. \underline{I know some of her remarks seem to give you strong ground for claiming that she does}. I am candid enough to say that, but I must give her the credit until she disowns it of saying there is a difference too, and I do not believe you can fully tell just what she means. \underline{God dwells in us by His Holy Spirit}, as a Comforter, as a Reprover, especially the former. When we come to Him we partake of Him in that sense, because the Spirit comes forth from Him; \underline{it comes forth from the Father and the Son}. It is not a person walking around on foot, or flying \underline{as a literal being}, \underline{in any such sense as Christ and the Father are} – at least, if it is, it is utterly beyond my comprehension of the meaning of language or words}.}[Letter: G. I. Butler to J. H. Kellogg. April 5. 1904]


\others{\textbf{S obzirom na to da ste do sada ti i sestra White bili u potpunom skladu, morat ću u potpunosti napustiti cijelu stvar i ostaviti je tebi i sestri White. \underline{Sestra White kaže da ne postoji potpuni sklad, a ti kažeš da postoji}. \underline{Znam da neke od njenih napomena izgledaju kao da ti daju snažno uporište da tvrdiš da te ona podržava}. Dovoljno sam otvoren da to kažem, ali moram joj dati naklonost sve dok se ona ne odrekne stava da postoji različitost, i ne vjerujem da si u stanju kazati što ona točno misli. \underline{Bog prebiva u nama Svojim Svetim Duhom}, kao Utješitelj, kao Ukoravatelj, ali posebice ovo drugo. Kada dođemo Njemu, mi imamo udjela u Njemu u tom smislu jer Duh proizlazi od Njega; \underline{on proizlazi od Oca i Sina}. To nije neka osoba koja se šeta naokolo ili leti \underline{kao neko doslovno biće}, \underline{u bilo kojem takvom smislu kao što su to Krist i Otac} - ukoliko je to tako onda je to apsolutno iznad mog razumijevanja značenja jezika ili riječi}.}[Letter: G. I. Butler to J. H. Kellogg. April 5. 1904]


The given correspondence is crucial for understanding the Kellogg controversy. Kellogg himself stated, \others{the whole thing may be simmered down to the question: \textbf{Is the Holy Ghost a person?}} Similarly Dr. Kellogg wrote to William White: \others{I have been studying very carefully to see what is \textbf{the real root of the difficulty with the Living Temple}, and as far as I can see \textbf{\underline{the whole question} resolves itself into this: \underline{Is the Holy Ghost, a person}?}}[Letter J. H. Kellogg to William White, October 28, 1903][https://drive.google.com/file/d/1\_S4S-Hc0K7Ka8gda9oRhPuAb9XzBTwmb/view] How does Kellogg's conclusion compare to the review and instruction of heavenly origin, which clearly told us that the reasoning in the Living Temple is \egwinline{naught but speculation in regard to \textbf{the personality of God and where His presence is}}[SpTB02 51.3; 1904][https://egwwritings.org/?ref=en\_SpTB02.51.3&para=417.262]? In the writings of Ellen White and the pioneers, the term '\textit{personality of God}' refers specifically to the personality of the Father. So, why does Kellogg claim that the real issue is the personality of the Holy Spirit, when God indicated that the issue concerns the personality of the Father?


Dana korespondencija je ključna za razumijevanje Kellogove kontroverze. Kellogg je sâm izjavio, \others{sve teškoće mogu se sumirati u sljedećem pitanju: \textbf{Je li Sveti Duh osoba?}} Slično tome, dr. Kellogg je pisao Williamu Whiteu: \others{Vrlo pažljivo sam proučavao da vidim što je \textbf{pravi korijen problema s 'Živim hramom'}, i koliko mogu vidjeti \textbf{\underline{cijelo pitanje} se svodi na ovo: \underline{Je li Sveti Duh, osoba}?}}[Letter J. H. Kellogg to William White, October 28, 1903][https://drive.google.com/file/d/1\_S4S-Hc0K7Ka8gda9oRhPuAb9XzBTwmb/view] Kako se Kelloggov zaključak može usporediti s uputom nebeskog podrijetla, koje nam jasno kaže da su rasuđivanja u Živom Hramu \egwinline{obična nagađanja vezano za \textbf{ličnost Boga i gdje je Njegova prisutnost}}[SpTB02 51.3; 1904][https://egwwritings.org/?ref=en\_SpTB02.51.3&para=417.262]? U spisima Ellen White i pionira, izraz '\textit{ličnost Boga}' odnosi se specifično na ličnost Oca. Dakle, zašto Kellogg tvrdi da je pravi korijen problema ličnost Duha Svetoga, kada je Bog naznačio da je korijen problema ličnost Oca?


Many assume that Dr. Kellogg is being manipulative, evading the core issue. However, under a particular premise, his arguments concerning the personality of the Holy Spirit logically support his controversial views on the \emcap{personality of God}. This premise becomes evident within the data itself when we closely follow his reasoning.


Mnogi pretpostavljaju da je dr. Kellogg manipulativan, te izbjegava glavno pitanje. Međutim, pod određenom pretpostavkom, njegovi argumenti o ličnosti Duha Svetoga logički podržavaju njegove kontroverzne poglede na \emcap{ličnost Boga}. Ta pretpostavka postaje očita unutar samih podataka kada pažljivo pratimo njegovo rasuđivanje.


As we have seen earlier, the doctrine on the \emcap{personality of God} teaches that God, the Father, possesses a form—a tangible, material body. Dr. Kellogg concurred that this assertion holds true within the bounds of our finite conception of God\footnote{\href{https://archive.org/details/J.H.Kellogg.TheLivingTemple1903/page/n33/}{Dr. John H. Kellogg, The Living Temple, p.31.}}. However, he argued that, in reality, God transcends our conceptions regarding His form, as He is beyond the constraints of space\footnote{\href{https://archive.org/details/J.H.Kellogg.TheLivingTemple1903/page/n33/}{Dr. John H. Kellogg, The Living Temple, p.33.}}. In this sense, Kellogg effectively does away with the reality of God’s physical, material body. The premise that would validate Dr. Kellogg’s viewpoint is the \textit{exclusive equivalence} in understanding the \emcap{personality of God} and that of the Holy Spirit. Is the Holy Spirit constrained by space? No, He is not. Does the Holy Spirit have a physical body? No! According to Jesus, \bible{for a spirit hath not flesh and bones}[Luke 24:39]. Is the Holy Ghost a person? The answer hinges on our interpretation of what it means to be a person. What is that quality or state of the Holy Spirit being a person?\footnote{Direct application of the definition on the word ‘\textit{personality}’ from the \href{https://www.merriam-webster.com/dictionary/personality}{Merriam Webster Dictionary}} When comparing Dr. Kellogg's belief in the personality of the Holy Spirit with Brother Butler's views, it becomes evident that the quality of the Holy Spirit being a person does not align with \others{that of \textbf{semblance to a person} or a human being}. Butler explicitly stated his criteria for this determination\footnote{In his letter to Dr. Kellogg, Brother Butler further asserted that there is no distinction between the person and the bodily presence. See \href{https://c7da.us/egwdl/Butler\%20to\%20Kellogg\%20Aug121904.pdf}{Letter from Butler to Kellogg, August 12, 1904, p.6}}: \others{\textbf{It is not a person walking around on foot, or flying \underline{as a literal being}, \underline{in any such sense as Christ and the Father are} – at least, if it is, it is utterly beyond my comprehension of the meaning of language or words}}.


Kao što smo ranije vidjeli, doktrina o \emcap{ličnosti Boga} uči da Bog, Otac, posjeduje oblik—opipljivo, materijalno tijelo. Dr. Kellogg se složio da je ta tvrdnja istinita unutar granica našeg konačnog shvaćanja Boga\footnote{\href{https://archive.org/details/J.H.Kellogg.TheLivingTemple1903/page/n33/}{Dr. John H. Kellogg, The Living Temple, p.31.}}. Međutim, on je tvrdio da, u stvarnosti, Bog nadilazi naše pojmove o Njegovom obliku, budući da je izvan ograničenja prostora\footnote{\href{https://archive.org/details/J.H.Kellogg.TheLivingTemple1903/page/n33/}{Dr. John H. Kellogg, The Living Temple, p.33.}}. U tom smislu, Kellogg učinkovito odbacuje stvarnost Božjeg fizičkog, materijalnog tijela. Pretpostavka koja bi validirala dr. Kelloggov pogled je \textit{isključivo izjednačavanje} razumijevanja \emcap{ličnosti Boga} i ličnosti Duha Svetoga. Da li je Duh Sveti ograničen prostorom? Ne, nije. Ima li Duh Sveti fizičko tijelo? Ne! Prema Isusu, \bible{duh nema tijela i kostiju}[Luke 24:39]. Je li Sveti Duh osoba? Odgovor ovisi o našem tumačenju što znači biti osoba. Koji je to kvalitet ili stanje koje Svetog Duha čini osobom?\footnote{Direktna primjena definicije riječi '\textit{ličnost}' iz \href{https://www.merriam-webster.com/dictionary/personality}{Merriam Webster rječnika}} Kada usporedimo dr. Kelloggovu razumijevanje ličnosti Duha Svetoga s pogledima brata Butlera, postaje očito da kvalitet Svetog Duha kao osobe ne odgovara \others{\textbf{izvanjskoj sličnosti s osobom} ili ljudskim bićem}. Butler je potvrdio da je upravo to njegovo razumijevanje\footnote{U svom pismu dr. Kelloggu, brat Butler je dalje tvrdio da nema razlike između osobe i tjelesne prisutnosti. Vidi \href{https://c7da.us/egwdl/Butler\%20to\%20Kellogg\%20Aug121904.pdf}{Pismo od Butlera Kelloggu, 12. kolovoza 1904., str.6}}: \others{\textbf{To nije neka osoba koja se šeta naokolo ili leti \underline{kao neko doslovno biće}, \underline{u bilo kojem takvom smislu kao što su to Krist i Otac} - ukoliko je to tako onda je to apsolutno iznad mog razumijevanja značenja jezika ili riječi}}.


Have you noticed that Brother Butler addressed Kellogg’s unspoken premise? Butler drew a distinction between the Father and Christ, in relation to the Holy Spirit. Brother Butler is correct. There exists a contrast between the personality of the Holy Spirit and that of God and Christ. Christ and the Father possess a physical form of a person, whereas the Holy Spirit does not. To do away with the physical form of a person of the Father is to \textit{exclusively equate} the understanding of the personality of the Father with that of the Holy Spirit. Kellogg’s approach is compelling, because it was backed by valid arguments regarding the personality of the Holy Spirit.


Jeste li primijetili da je brat Butler adresirao neizrečenu pretpostavku Dr. Kellogga? Butler je napravio razliku između Oca i Krista, u odnosu na Svetog Duha. Brat Butler je u pravu. Postoji kontrast između ličnosti Svetog Duha i ličnosti Boga i Krista. Krist i Otac posjeduju fizički oblik osobe, dok Sveti Duh ne. Odbaciti fizički oblik osobe Oca usljedilo bi \textit{isključivim izjednačavanjem} razumijevanja ličnosti Oca s ličnošću Duha Svetoga. Kelloggov pristup je uvjerljiv, jer je podržan valjanim argumentima o ličnosti Svetog Duha.


Let us briefly examine the personality of the Holy Spirit. What is the quality or state of the Holy Spirit being a person?


Preispitajmo nakratko ličnost Duha Svetoga. Koji kvalitet ili stanje čini Duha Svetoga osobom?


\egw{\textbf{The Holy Spirit has a personality}, \textbf{\underline{else} }He could not \textbf{bear witness} to our spirits and with our spirits that we are the children of God. \textbf{He must also be a \underline{divine person}}, \textbf{\underline{else}} He could not \textbf{search out the secrets} which lie hidden \textbf{in the mind of God}.}[21LtMs, Ms 20, 1906, par. 32; 1906][https://egwwritings.org/read?panels=p14071.10296041&index=0]


\egw{\textbf{Duh Sveti ima ličnost}, \textbf{\underline{inače}} ne bi mogao \textbf{svjedočiti} našim duhovima i s našim duhovima da smo djeca Božja. \textbf{On također mora biti \underline{božanska osoba}}, \textbf{\underline{inače}} ne bi mogao \textbf{istraživati tajne} koje leže skrivene \textbf{u Božjem umu}.}[21LtMs, Ms 20, 1906, par. 32; 1906][https://egwwritings.org/read?panels=p14071.10296041&index=0]


\egw{\textbf{The Holy Spirit is a person}; \textbf{\underline{for}} He \textbf{beareth witness} with our spirits that we are the children of God.}[21LtMs, Ms 20, 1906, par. 31; 1906][https://egwwritings.org/read?panels=p14071.10296040&index=0]


\egw{\textbf{Duh Sveti je osoba}; \textbf{\underline{jer}} On \textbf{svjedoči} duhu našem da smo djeca Božja.}[21LtMs, Ms 20, 1906, par. 31; 1906][https://egwwritings.org/read?panels=p14071.10296040&index=0]


The qualities or states that define the Holy Spirit as a person are explicitly mentioned in the provided quotations. These include the ability to bear witness and search out the mind. Further support can be found in Scripture, which attributes actions to the Holy Spirit such as speaking (\textit{Acts 13:2}), teaching (\textit{John 14:26; 1 Corinthians 2:13}), making decisions (\textit{Acts 15:28}), and experiencing emotions (\textit{Ephesians 4:30}), among others. These \textit{qualities }collectively affirm the personality of the Holy Spirit. Can these same qualities be also applied to the Father and the Son? Most certainly. However, unlike the Father and the Son, the Holy Spirit is distinguished by the absence of a material, tangible form. When Ellen White questioned Christ about the \emcap{personality of God}, her inquiry specifically targeted the personal form as the defining quality of the Father's personality.


Kvalitete ili stanja koja definiraju Svetog Duha kao osobu izričito su prikazani u navedenim citatima. Navedene kvalitete uključuju sposobnost svjedočenja i istraživanja uma. Dodatna podrška može se pronaći u Svetom Pismu, koje pripisuje Svetom Duhu sposobnosti poput govorenja (\textit{Djela 13:2}), učenja (\textit{Ivan 14:26; 1 Korinćanima 2:13}), donošenja odluka (\textit{Djela 15:28}) i iskustvo emocija (\textit{Efežanima 4:30}), među ostalim. Te \textit{kvalitete} zajedno potvrđuju ličnost Duha Svetoga. Mogu li se te iste kvalitete primijeniti i na Oca i Sina? Svakako. Međutim, za razliku od Oca i Sina, Sveti Duh se razlikuje po odsutnosti materijalnog, opipljivog oblika. Kada je Ellen White pitala Krista o \emcap{ličnosti Boga}, njezino pitanje je specifično bilo usmjereno na osobni oblik kao definirajuću kvalitetu ličnosti Oca.


\egw{I have often \textbf{seen }the lovely Jesus, that \textbf{He is a person}. \textbf{I asked Him if His Father \underline{was a person} and \underline{had a form} like Himself}. Said Jesus, ‘\textbf{I am in the express image of My Father's person}.’}[EW 77.1; 1882][https://egwwritings.org/read?panels=p28.490&index=0]


\egw{Često sam \textbf{viđala} dragog Isusa, da \textbf{je On osoba}. \textbf{Upitala sam Ga je li Njegov Otac \underline{osoba} i \underline{ima li oblik} kao i On}. Isus je odgovorio: '\textbf{Ja sam savršena slika osobe mojega Oca}.'}[EW 77.1; 1882][https://egwwritings.org/read?panels=p28.490&index=0]


This brings us to a profound distinction in how the personality of the Holy Spirit is understood, as opposed to that of the Father and the Son. Ellen White describes the Holy Spirit as a spiritual manifestation of Christ, drawing a clear line between the outward, visible manifestation of Christ and His spiritual manifestation. This contrast underscores the unique nature of the Holy Spirit's presence and action in the world, distinct from the physical presence of Christ and the Father. Pay attention to the contrast between the outward, visible manifestation of Christ, and His spiritual manifestation:


To nas dovodi do duboke razlike u razumijevanju ličnosti Duha Svetoga, u usporedbi s ličnošću Oca i Sina. Ellen White opisuje Svetog Duha kao duhovnu manifestaciju Krista, povlačeći jasnu liniju između vanjske, vidljive manifestacije Krista i Njegove duhovne manifestacije. Ovaj kontrast naglašava jedinstvenu prirodu prisutnosti i djelovanja Svetog Duha u svijetu, različitu od fizičke prisutnosti Krista i Oca. Obratite pažnju na kontrast između vanjske, vidljive manifestacije Krista i Njegove duhovne manifestacije:


\egw{That \textbf{Christ }should \textbf{manifest Himself} to them, and yet \textbf{be invisible to the world}, was a mystery to the disciples. They could not understand \textbf{the words of Christ in their \underline{spiritual sense}}. \textbf{They were thinking of \underline{the outward, visible manifestation}}. They could not take in the fact that they could have \textbf{the presence of Christ with them}, and \textbf{yet He be unseen by the world}. \textbf{They did not understand the meaning of \underline{a spiritual manifestation}}.}[ST November 18, 1897, par. 6; 1897][https://egwwritings.org/read?panels=p820.14727&index=0]


\egw{Da će se \textbf{Krist} \textbf{manifestirati} njima, a \textbf{ostati nevidljiv svijetu}, učenicima je bila tajna. Nisu mogli razumjeti \textbf{Kristove riječi u njihovom \underline{duhovnom značenju}}. \textbf{Oni su razmišljali o \underline{vanjskom, vidljivoj manifestaciji}}. Nisu mogli uzeti u obzir činjenicu da oni mogu imati \textbf{Kristovu prisutnost sa njima}, a \textbf{da On ne bude viđen od svijeta}. \textbf{Nisu razumjeli značenje \underline{duhovne manifestacije}}.}[ST November 18, 1897, par. 6; 1897][https://egwwritings.org/read?panels=p820.14727&index=0]


The Holy Spirit is not a person in the physical sense but is manifested in spiritual sense. If the exclusive understanding of the personality of the Holy Spirit is applied to the Father, then consequently His physical form of a person is done away. His personality is spiritualized. This is why Ellen White critically labeled Kellogg's perspective as spiritualism. Do you know which doctrine, in particular, has a core tenet, that the Father and the Holy Spirit are equal in their personalities? It is \textit{the doctrine of the trinity}. Could it be possible that Dr. Kellogg was actually raising the theological side of questions of the trinity?


Sveti Duh nije osoba u fizičkom smislu, već se manifestira u duhovnom smislu. Ako se ekskluzivno razumijevanje ličnosti Duha Svetoga primijeni na Oca, tada se posljedično njegova fizička forma osobe odbacuje. Njegova ličnost je spiritualizirana. To je razlog zašto je Ellen White kritički označila Kelloggovu perspektivu kao spiritualizam. Znate li koja doktrina, posebno, ima temeljno načelo da su Otac i Sveti Duh jednaki u svojim ličnostima? To je \textit{doktrina o trojstvu}. Bi li bilo moguće da je dr. Kellogg zapravo postavio teološku stranu pitanja doktrine o trojstvu?


\section*{Kellogg’s confession about the Living Temple}


\section*{Kelloggova ispovijest o Živom hramu}


In his interview with G. W. Amadon and A. C. Bourdeau, one month after being disfellowshipped, he confessed that he unintentionally brought the theological side of the question of the Trinity into his book.


U svom intervjuu s G. W. Amadonom i A. C. Bourdeauom, mjesec dana nakon što je isključen iz zajednice, priznao je da je nenamjerno uveo teološku stranu pitanja o trojstvu u svoju knjigu.


\others{\textbf{Now, I thought I had cut out entirely the theological side of questions of \underline{the trinity and all that sort of things}}. \textbf{I didn't mean to \underline{put it in} at all}, and I took pains to state in the preface that I did not. I never dreamed of such a thing as \textbf{any theological question being} \textbf{\underline{brought into it}}. I only wanted to show that \textbf{the heart does not beat of its own motion but that it is the power of God that keeps it going}.}[Kellogg vs. The Brethren: His Last Interview as an Adventist, p. 58.][https://archive.org/details/KelloggVs.TheBrethrenHisLastInterviewAsAnAdventistoct71907/page/n37/mode/2up]


\others{\textbf{Mislio sam da sam potpuno izostavio teološku stranu pitanja o \underline{trojstvu i sve te stvari}}. \textbf{Nisam to uopće želio \underline{staviti unutra}}, i potrudio sam se napisati u uvodu da nisam. Nikada nisam ni sanjao da će \textbf{takva teološka pitanja biti} \textbf{\underline{unesena u nju}}. \textbf{Samo sam želio pokazati da srce ne kuca svojim vlastitim pokretom nego da je sila Božja ta koja ga održava}.}[Kellogg vs. The Brethren: His Last Interview as an Adventist, p. 58.][https://archive.org/details/KelloggVs.TheBrethrenHisLastInterviewAsAnAdventistoct71907/page/n37/mode/2up]


If we were to look in his book for trinitarian expressions, we would not find any. Would that be a proof that Kellogg is disingenuous in his confession? The only thing we find is the teaching that is stepping off of the foundation of our faith—the \emcap{fundamental principles}—regarding the \emcap{personality of God} and where His presence is. The trinitarian expressions are not there but his sentiments regarding the \emcap{personality of God} are in line with the trinitarian sentiments on God’s person. These sentiments are deceptive and Kellogg was rebuked for them. When he wanted to explicitly state the belief in the Trinity doctrine, in hopes of fixing the book, he was again rebuked by the words, \egwinline{\textbf{Patchwork theories} cannot be accepted by those who are loyal to the faith} and to the \emcap{Fundamental Principles}\footnote{\href{https://egwwritings.org/?ref=en_Lt253-1903.28&para=9980.36}{EGW, Lt253-1903.28; 1903}}. The crucial problem of the Trinity doctrine, in regard to the \emcap{personality of God}, is the underlying assumption that all Three, the Father, the Son, and the Holy Spirit, possess the same type of personality in such a way that They make one monotheistic God. In this light, we may understand Kellogg's assertions over the personality of the Holy Spirit, that the Holy Spirit is the third person of the Godhead. Dr. Kellogg quoted Ellen White when asserting his claims; although he used the same words, he had a wrong sentiment. In light of Dr. Kellogg’s confession, for including “\others{\textbf{the theological side of questions of \underline{the trinity}}}”, and His assertion that “\others{\textbf{the whole thing may be simmered down to the question}: \textbf{\underline{Is the Holy Ghost a person}}?}”, we may see the unspoken premise that the Father and the Son are in the same way persons as is the Holy Spirit. This is why Brother Butler wrote to him regarding the personality of the Holy Spirit: \others{\textbf{It is not a person walking around on foot, or flying \underline{as a literal being}, \underline{in any such sense as Christ and the Father are} – at least, if it is, it is utterly beyond my comprehension of the meaning of language or words.}}[Letter from G. I. Butler to J. H. Kellogg, April 5 1904.]


Ako bismo tražili izraze vezane za Trojstvo u njegovoj knjizi, ne bismo ih pronašli. Da li bi to bio dokaz da Kellogg nije iskren u svojoj ispovijesti? Jedino što nalazimo je učenje koje odstupa od temelja naše vjere—\emcap{fundamentalnih principa}—u vezi s \emcap{ličnosti Boga} i gdje je Njegova prisutnost. Izrazi vezani za Trojstvo nisu prisutni, ali njegovi sentimenti o \emcap{ličnosti Boga} su u skladu s trinitarijanskim sentimentima o Božjoj ličnosti. Ti su sentimenti obmanjujući i Kellogg je bio ukoren za njih. Kada je želio izričito izraziti vjerovanje u doktrinu o Trojstvu, u nadi da će pokrpati knjigu, ponovno je bio ukoren riječima, \egwinline{\textbf{Teorije krpanja} ne mogu biti prihvaćene od onih koji su lojalni vjeri} i \emcap{Fundamentalnim Principima}\footnote{\href{https://egwwritings.org/?ref=en_Lt253-1903.28&para=9980.36}{EGW, Lt253-1903.28; 1903}}. Ključni problem doktrine o Trojstvu, u pogledu \emcap{ličnosti Boga}, je temeljna pretpostavka da sva Trojica, Otac, Sin i Sveti Duh, posjeduju istu vrstu ličnosti na takav način da čine jednog monoteističkog Boga. U ovom svjetlu, možemo razumjeti Kellogove tvrdnje o ličnosti Svetog Duha, da je Sveti Duh treća osoba Božanstva. Dr. Kellogg je citirao Ellen White prilikom iznošenja svojih tvrdnji; iako je koristio iste riječi, imao je pogrešan sentiment. U svjetlu dr. Kellogove ispovijesti, za uključivanje "\others{\textbf{teološke strane pitanja o \underline{trojstvu}}}", i njegove tvrdnje da "\others{\textbf{sve se može sumirati u sljedećem pitanju}: \textbf{\underline{Je li Sveti Duh osoba}}?}", možemo vidjeti neizgovorenu pretpostavku da su Otac i Sin na isti način osobe kao što je to Sveti Duh. Zato je Brat Butler pisao njemu u vezi s osobnošću Svetog Duha: \others{\textbf{To nije neka osoba koja se šeta naokolo ili leti \underline{kao neko doslovno biće}, \underline{u bilo kojem takvom smislu kao što su to Krist i Otac} – ukoliko je to tako onda je to apsolutno iznad mog razumijevanja značenja jezika ili riječi.}}[Letter from G. I. Butler to J. H. Kellogg, April 5 1904.]


\section*{The presence of God manifested in nature}


\section*{Božja prisutnost očitovana u prirodi}


From the works of our pioneers we have seen that the personality of the Holy Ghost is most clearly expressed in terms of God’s presence. Sister White told us that the Living Temple \egwinline{introduces that which is naught but speculation in \textbf{regard to the personality of God and where His presence is}.}[SpTB02 51.3; 1904][https://egwwritings.org/?ref=en\_SpTB02.51.3&para=417.262] The \emcap{personality of God} and where His presence is are two mutually inclusive doctrines; one affirms the other. Deny one, you deny the other. This notion is clearly seen in the book, the Living Temple. In the previous sections, we read Kellogg’s arguments for the \emcap{personality of God} taken from his book. He argued that it is unprofitable to talk about God’s shape or any tangible form. He denied the reality of God as a definite, material and tangible Being. If God is spirit, possessing no form nor body, then He is not restricted in His presence to one locality; this was the sentiment Kellogg advocated in the Living Temple.


Iz spisa naših pionira vidjeli smo da se ličnost Svetog Duha najjasnije izražava u smislu Božje prisutnosti. Sestra White nam je rekla da Živi Hram \egwinline{uvodi ono što je ništa više nego obična nagađanja vezano za \textbf{ličnost Boga i gdje je Njegova prisutnost}.}[SpTB02 51.3; 1904][https://egwwritings.org/?ref=en\_SpTB02.51.3&para=417.262] \emcap{Ličnost Boga} i gdje je Njegova prisutnost su dva međusobno uključiva učenja; jedno potvrđuje drugo. Poricati jedno, znači poricati i drugo. Ova se ideja jasno vidi u knjizi Živi hram. U prethodnim odjeljcima, čitali smo Kellogove argumente za \emcap{ličnost Boga} uzete iz njegove knjige. Tvrdio je da je krajnje besmisleno govoriti o Božjem obliku ili bilo kojem opipljivom obliku. Negirao je stvarnost Boga kao određenog, materijalnog i opipljivog Bića. Ako je Bog duh, koji ne posjeduje oblik ni tijelo, onda Njegova prisutnost nije ograničena na jednu lokaciju; to je bio sentiment koji je Kellogg zagovarao u Živom Hramu.


\others{Says one, ‘\textbf{God may be \underline{present by his Spirit}, or by his power, but \underline{certainly God himself} cannot be present everywhere at once}.’ We answer: How can power be separated from the source of power? \textbf{Where God’s Spirit is at work}, where God’s power is manifested, \textbf{God \underline{himself} is actually and truly present…}}[John H. Kellogg, The Living Temple, p.28.][https://archive.org/details/J.H.Kellogg.TheLivingTemple1903/page/n29/]


\others{Kaže jedan: '\textbf{Bog može biti \underline{prisutan putem svojeg Duha}, ili svoje moći, ali \underline{sigurno sâm Bog} ne može biti prisutan svugdje odjednom}.' Mi odgovaramo: Kako moć može biti odvojena od izvora moći? \textbf{Gdje je Duh Božji na djelu}, gdje se manifestira Božja moć, \textbf{Bog \underline{sâm} je stvarno i istinski prisutan...}}[John H. Kellogg, The Living Temple, p.28.][https://archive.org/details/J.H.Kellogg.TheLivingTemple1903/page/n29/]


When Dr. Kellogg wrote \others{Says one, ‘God may be present by His Spirit…’}, he referred to the sentiments of our pioneers who were loyal to the \emcap{Fundamental Principles}. This is the most obvious point where Dr. Kellogg stepped off from the \emcap{Fundamental Principles}.


Kada je dr. Kellogg napisao \others{Kaže jedan, 'Bog može biti prisutan putem svojeg Duha...'}, referirao se na sentimente naših pionira koji su bili vjerni \emcap{Fundamentalnim Principima}. Ovo je najuočljivija točka na kojoj je dr. Kellogg odstupio od \emcap{Fundamentalnih Principa}.


Further, Kellogg proves his point by his famous illustration of a shoemaker and a living shoe, which to many is proof of pantheism. Kellogg’s main argument in the Living Temple is that God is the One that sustains all life. This is true. But where is the real problem? Where is the false teaching? Is it true that every life is sustained by God? In the following chapters, we will present data that testify to the truthfulness of Kellogg’s claim that God indeed sustains all life in nature. At first, this evidence may make Kellogg seem less pantheistic; but, hopefully, you will be able to recognize the defining characteristic that justifies Ellen White’s claim of his pantheistic views.


Dalje, Kellogg dokazuje svoju točku svojom poznatom ilustracijom o obućaru i živućoj cipeli, što mnogima služi kao dokaz panteizma. Glavni argument Kellogga u Živom hramu je da je Bog Onaj koji održava sav život. To je istina. Ali gdje je pravi problem? Gdje je lažno učenje? Je li istina da svaki život održava Bog? U sljedećim poglavljima predstavit ćemo podatke koji svjedoče o istinitosti Kelloggovog tvrdnje da Bog doista održava sav život u prirodi. Na prvi pogled, ovi dokazi mogu učiniti Kellogga manje panteističkim; ali, nadamo se, moći ćete prepoznati definirajuću karakteristiku koja opravdava tvrdnju Ellen White o njegovim panteističkim pogledima.
