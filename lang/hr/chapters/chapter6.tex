\chapter{Ispitivanje Testa}

U odgovoru Sestre White na Dr. Kellogovo vjerovanje u doktrinu o Trojstvu i njegove pokušaje da \textit{zakrpa} Živi Hram, vidimo da je ona smatrala doktrinu o Trojstvu kontradiktornom svjetlu koje joj je dano u vezi \emcap{ličnosti Boga}. Da je bila pobornik doktrine o Trojstvu, očekivali bismo da ju je pažljivo odvojila od panteizma i sačuvala njene legitimne aspekte. Međutim, to nije ono što vidimo u njenoj reakciji na Trojstvo. Umjesto toga, u svom odgovoru je suprotstavila doktrinu o Trojstvu istini o \emcap{ličnosti Boga}, prisjećajući se svojih prošlih vizija koje su pokazale da bi ova doktrina lišila Božji narod njihovih prošlih iskustava. U njenom reaktivnom prisjećanju kako je Bog ustanovio \emcap{fundamentalne principe}, ukazala je da doktrina o Trojstvu \textit{ruši stupove naše vjere} i \textit{odvodi nas od temeljnih principa}. Ova izrazita razlika može se jasno vidjeti uspoređujući naša trenutna temeljna vjerovanja sa \emcap{fundamentalnim principima} koji su se držali u prošlosti.

Imajući na umu odgovor sestre White na Kelloggovu vjeru o doktrini o Trojstvu, osvrnimo se na karakteristike teorija koje je opisala u poglavlju “\textit{Temelji naše vjere}”. Kada sestra White govori o Kelloggovim teorijama o Bogu, postavimo si pitanje: “imaju li njezini citati smisla ako se doktrina Trojstva primijeni u njihov kontekst?“ Ispitajmo svaku karakteristiku.

\subsection*{Dali doktrina o Trojstvu “liši Božji narod njihovog prošlog iskustva”?}

\egw{One \normaltext{[spiritualističke teorije]} uništavaju učinkovitost istine nebeskog porijekla i \textbf{liše Božji narod njihovog prošlog iskustva}, zauzvrat dajući im lažne znanosti.}[SpTB02 54.1; 1904][https://egwwritings.org/?ref=en\_SpTB02.54.1]

\egw{Ovaj temelj je izgrađen od strane Najvećeg Radnika, i održati će se na oluji i buri. Hoće li mu dopustiti da ovaj čovjek \normaltext{[Kellogg]} predstavi \textbf{doktrine koje opovrgavaju prošlo iskustvo Božjeg naroda}? Došlo je vrijeme za odlučnu akciju.}[SpTB02 54.2; 1904][https://egwwritings.org/?ref=en\_SpTB02.54.2]

\egw{\textbf{Koji bi to utjecaj mogao, u ovom dijelu povijesti, na nepošten, snažan način \underline{srušiti temelje naše vjere},—temelj koji je bio postavljen \underline{na početku našeg rada} sa molitvom i proučavanjem Riječi i Otkrivenja? Na tom temelju \underline{smo gradili posljednjih pedeset godina}}. Smatraš li da nemam što reći kada vidim početak rada koji bi \textbf{\underline{uklonio neke od stupova naše vjere}}? Moram poslušati naredbu, ‘Suoči se!’}[SpTB02 58.1; 1904][https://egwwritings.org/?ref=en\_SpTB02.58.1]

Prema svjedočenju sestre White, temelj naše vjere bili su \emcap{Fundamentalni Principi}. Oni trenutno više ne predstavljaju našu vjeru. Najveća razlika je prva točka, oko razumijevanja tko je Bog. Umjesto vjerovanja da je jedan Bog—Otac, osobno duhovno biće, imamo novo vjerovanje da je jedan Bog—Otac, Sin i Duh Sveti, zajednica triju osoba. Iz perspektive prve točke \emcap{Fundamentalnih Principa}, da li je ova nova doktrina o tome tko je Bog i što je On, lišila Božji narod njegovog prošlog iskustva?

\subsection*{Dali doktrina o Trojstvu ruši stupove naše vjere, ili odvodi od temeljnih principa?}

\egw{Bila sam upućena od strane Nebeskog glasnika da su neka rasuđivanja u knjizi ‘Živi Hram’ netočna i da \textbf{će ta razmišljanja odvesti u zabludu umove onih koji nisu temeljito utemeljeni na temeljnim principima sadašnje istine.}}[SpTB02 51.3; 1904][https://egwwritings.org/?ref=en\_SpTB02.51.3]

\egw{U vremenu kada se izdala knjiga ‘Živi Hram’, u noćnoj viziji, predstavljeni su mi prikazi koji su ukazivali na \textbf{približavanje neke opasnosti}, te da se za nju moram pripremiti pisanjem stvari koje mi je Bog objavio \textbf{vezano za temeljne principe naše vjere}.}[SpTB02 52.3; 1904][https://egwwritings.org/?ref=en\_SpTB02.52.3]

\egw{Neprijatelj duša tražio je da uvede pretpostavku da će nastati velika reformacija među Adventistima Sedmog Dana, i da će se ta reforma sačinjavati od \underline{odstupanja od doktrina koje stoje kao stupovi naše vjere,} i da će se pokrenuti proces reorganizacije. Kada bi takva reformacija nastupila, kakav bi bio rezultat? \textbf{Principi istine} koje Bog u svojoj mudrosti dao crkvi ostatka, \textbf{bili bi odbačeni}. Naša religija bi se promijenila. \textbf{Fundamentalni principi} koji su održavali posao posljednjih pedeset godina \textbf{proglasili bi se greškom}. Nova organizacija bi se osnovala. Knjige novog reda bi se napisale. Uveo bi se sistem intelektualne filozofije.}

Teorije dr. Kellogga o \emcap{ličnosti Boga}, ako budu prihvaćene, pokrenule bi reformaciju unutar Crkve Adventista Sedmoga Dana. Na temelju intelektualne filozofije, oni bi nas naveli da se odreknemo nekih doktrina koje stoje kao stupovi naše vjere, osuđujući Fundamentalne Principe kao zabludu. Da li je moguće da smo prihvaćanjem doktrine o Trojstvu ušli u novu organizaciju?

\egw{Prije nego što sam poslala svjedočanstva \textbf{o naporima neprijatelja da potkopa temelje naše vjere kroz širenje zavodljivih teorija}, pročitala sam o incidentu o brodu koji se suočio sa ledenjakom u magli...}[SpTB02 55.3; 1904][https://egwwritings.org/?ref=en\_SpTB02.55.3]

\egw{Poruke svakog reda i vrste su bile \textbf{nametnute Adventistima Sedmoga Dana, da zauzmu mjesto istine, koja \underline{točku po točku}, je bila tražena sa molitvom i proučavanjem, i potvrđena čudesnom silom Gospodnjom}. \textbf{Ali međaši koji su nas učinili onime što jesmo, biti će očuvani, i biti će očuvani}, kao što je Bog naznačio kroz svoju riječ i svjedočanstvo Njegovog Duha. \textbf{On nas poziva da se držimo čvrstim} zahvatom vjere, \textbf{\underline{fundamentalnih principa} koji su zasnovani na \underline{neupitnom autoritetu}}.}[SpTB02 59.1; 1904][https://egwwritings.org/?ref=en\_SpTB02.59.1]

\emcap{Ličnost Boga} bila je stup naše vjere\footnote{\href{https://egwwritings.org/?ref=en_Ms62-1905.14}{EGW, Ms62-1905.14; 1905}}. \emcap{Ličnost Boga} izražena je u prvoj točki \emcap{Fundamentalnih Principa}. Da li je moguće da smo prihvaćanjem doktrine o Trojstvu srušili upravo taj stup naše vjere? Je li moguće da smo prihvaćanjem doktrine o Trojstvu skrenuli s ovog temeljnog principa—\emcap{ličnosti Boga}?

\subsection*{Ukida li Trojstvo ličnost Boga?}

\egw{\textbf{Ona \normaltext{[Živi Hram]} uvodi ono što je ništa više nego \underline{nagađanja} vezano za \underline{ličnost Boga} i gdje je Njegova prisutnost.}}[SpTB02 51.3; 1904][https://egwwritings.org/?ref=en\_SpTB02.51.3]

\egw{\textbf{Spiritualističke teorije o \underline{ličnosti Boga}, praćene njihovim logičnim zaključkom, oduzimaju cjelokupno kršćansko razumijevanje.}}[SpTB02 54.1; 1904][https://egwwritings.org/?ref=en\_SpTB02.54.1]

\egw{‘Živi Hram’ sadrži alfu tih teorija. Znala sam da će omega ubrzo uslijediti; i drhtala sam za naš narod. Znala sam da \textbf{moram upozoriti našu braću i sestre da ne ulaze u kontroverzu oko \underline{prisutnosti} i \underline{ličnosti Boga}. Sentimenti u knjizi ‘Živi Hram’ \underline{vezani za tu točku su netočni}. Sveto Pismo koje se koristilo za potkrijepljenje te doktrine, je pogrešno primijenjeno}.}[SpTB02 53.2; 1904][https://egwwritings.org/?ref=en\_SpTB02.53.2]

Teorije koje je Kellogg iznio u Živom Hramu su spekulativne u pogledu ličnosti Boga i Njegove prisutnosti. Te se teorije bave pitanjem kvalitete ili stanja koja Bogom čine osobom\footnote{Definicija riječi ‘\textit{personality}’ prema Merriam-Webster rječniku - “\textit{kvaliteta ili stanje koja nekog čine osobom}”, orig.: “\textit{the quality or state of being a person}”}. Bog nam je dao jasno svjetlo u vezi s ovim pitanjem u našim \emcap{Fundamentalnim Principima}. Da li je moguće da doktrina o Trojstvu baca sumnju na ovo jasno svjetlo u pogledu \emcap{ličnosti Boga}?

\subsection*{Da li je doktrina o Trojstvu predstavljena kao da ju Sestra White podržavala?}

\egw{U kontroverzi koja je nastala među našom braćom \textbf{u vezi s učenjima ove knjige,} oni koji podupiru njezinu široku cirkulaciju \textbf{izjavili su: ‘Ona sadrži upravo one sentimente koje Sestra White podučava’. Ova tvrdnja pogodila je moje srce. Osjećala sam se slomljenom; jer sam znala da takva prikazivanja stvari nisu istinita}.}[SpTB02 53.1; 1904][https://egwwritings.org/?ref=en\_SpTB02.53.1]

\egw{\textbf{Primorana sam govoriti i odbiti tvrdnje da se učenja ‘Živog Hrama’ mogu poduprijeti tvrdnjama iz mojih spisa}. U toj knjizi mogu postojati izrazi i sentimenti koji su u skladu s mojim spisima. U mojim spisima mogu biti mnoge tvrdnje koje, uzete iz njihovog konteksta i tumačene prema umu pisca ‘Živog Hrama’, čini se da su u skladu s učenjima ove knjige. To može dati prividnu potporu tvrdnji da su sentimenti u ‘Živom Hramu’ u skladu s mojim spisima. \textbf{Ali, Bože sačuvaj da taj sentiment prevlada}.}[SpTB02 53.3; 1904][https://egwwritings.org/?ref=en\_SpTB02.53.3]


At this point, we have many unanswered questions. But, as we continue to study the first point of the \emcap{Fundamental Principles}, we will find answers to all of these questions. So far, in light of the \emcap{Fundamental Principles}, belief in the Trinity doctrine—as a Seventh-day Adventist—becomes very questionable. In order to defend the Trinity doctrine, the authority of the \emcap{Fundamental Principles} must be compromised. In what follows, we will briefly study their authority, context in Adventist history, and God’s purpose in giving them. We will also look at the true authorship of the \emcap{Fundamental Principles} and their role in present days.


U ovom trenutku imamo mnogo neodgovorenih pitanja. No, dok nastavljamo proučavati prvu točku \emcap{Fundamentalnih Principa}, pronaći ćemo odgovore na sva ova pitanja. Do sada, u svjetlu \emcap{Fundamentalnih Principa}, doktrina o Trojstvu postaje vrlo upitna. Kako bi se obranila doktrina o Trojstvu, autoritet \emcap{Fundamentalnih Principa} mora biti kompromitiran. U nastavku ćemo ukratko proučiti njihov autoritet, kontekst u adventističkoj povijesti i Božju svrhu kojom ih je dao. Također ćemo razmotriti tko je bio pravi autor \emcap{Fundamentalnih Principa} i njihovu ulogu u sadašnjem vremenu.
