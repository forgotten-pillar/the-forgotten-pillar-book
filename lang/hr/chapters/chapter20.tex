\chapter{Dr. Kellogg i spisi Ellen White}

Dr. Kellogg je tvrdio da je u Živom Hramu zastupao ista stajališta koja je i sestra White zagovarala. Također, danas mnogi tvrde da je sestra White bila trinitarijanac i da je ona ima zasluge za prihvaćanje doktrine o Trojstvu u našoj crkvi\footnote{William Johnsson, Adventist Review, 6. siječnja 1994., ‘\textit{Present Truth –Walking in God's Light}’}. Sestra White je sama izjavila da su takve tvrdnje lažne.

\egw{\textbf{Neprijatelj nastoji \underline{unijeti} među Božji narod spiritualističke teorije koje, \underline{ako budu prihvaćene, potkopale bi temelje vjere} koja nas je učinila onim što jesmo}. On vodi ljude da predstavljaju bajke obučene u Pismo. \textbf{Postoje oni koji tvrde da su spisi sestre White u skladu s ovim učenjima}. \textbf{\underline{Izjavljujem da je ovo laž}. Ljudi mogu pogrešno primijeniti Pismo; mogu pogrešno tumačiti moje riječi; ali Bog razumije njihove namjere}. Kako sam zahvalna za ovo! Kada neprijatelj navali kao poplava, \textbf{Duh Gospodnji će za nas podići standard protiv njega}.}[Ms137-1903.21; 1903][https://egwwritings.org/read?panels=p9939.30]

Dr. Kellogg je zagovarao teorije koje, ako bi se prihvatile, potkopale bi temelj naše vjere. Ključno je pravilno razumjeti što čini temelj naše vjere, na što se sestra White referirala. Vidjeli smo da se to odnosi na \emcap{Fundamentalne Principe}. Gledajući njezine spise i spise naših pionira, vidimo da doktrina o Trojstvu proturječi \emcap{ličnosti Boga} i istini o Božjoj prisutnosti. Danas, s doktrinom o Trojstvu kao dijelom našeg vjerovanja, prepoznajemo da smo se udaljili od \emcap{Fundamentalnih Principa} i formirali drugi temelj. Sestra White nije bila odgovorna za ovu tranziciju. To je čisto pogrešno tumačenje njezinih djela. Njezini spisi ne potkopavaju temelj vjere koji nas je učinio onim što jesmo. Njezin kasniji rad je potpuno u skladu s istinom danoj na početku.

\egw{\textbf{Posljednjih pedeset godina nije izblijedilo ni jednu jotu ili princip naše vjere kao što smo primili velike i čudesne dokaze koji su nam učinjeni sigurnim 1844, nakon proteka vremena}... \textbf{\underline{Nijedna riječ nije promijenjena ili poreknuta}. Ono za što je Duh Sveti svjedočio da je istina nakon proteka vremena, u našem velikom razočarenju, \underline{je čvrsti temelj istine}. \underline{Stupovi istine su bili otkriveni}, i mi smo prihvatili \underline{temeljne principe} koji su nas učinili onim što jesmo—Adventisti Sedmog dana, držeći zapovijedi Božje i imajući vjeru Isusovu.}}[Lt326-1905.3; 1905][https://egwwritings.org/read?panels=p7678.9]

\section*{Pogrešno predstavljanje crkvenog stajališta}

Dr. Kellogg, krivo predstavljajući spise Sestre White, nije samo krivo predstavio njezin rad, već i službeno stajalište crkve izraženo u \emcap{Fundamentalnim Principima}. Ellen White je ukorila Kellogga zbog krivog predstavljanja stajališta crkve. Dok čitamo ovaj ukor, imajmo na umu trenutačno stajalište crkve o \emcap{ličnosti Boga} u usporedbi s prvom točkom \emcap{Fundamentalnih Principa}.

\egw{Ti \textbf{nisi čvrst u istini}. Tvoje izjave upućene vjernicima i nevjernicima \textbf{krivo nas predstavljaju kao narod koji nije zamijenio istinu za zabludom}. One umanjuju utjecaj \textbf{koji bi Bog želio da imamo pred svijetom, otkrivajući jasnim, nedvosmislenim jezikom da smo \underline{vjerni principima naše vjere} i da držimo početak naše vjere čvrstim do kraja}. Mi smo strogo denominacijski. \textbf{Vjerujemo u 1903. godini iste istine koje smo vjerovali kada smo osnovali Sanatorij i školu u Battle Creeku, i \underline{znamo da u ovoj stvari nismo imali nikakvih ‘ako’ ili ‘ali’}}.}[Lt300-1903.4; 1903][https://egwwritings.org/read?panels=p7705.10]

\egwnogap{Dok si govorio stvari koje si govorio i davao svoje izjave pred nevjernicima, moje je srce doista bilo tužno. \textbf{Dokazao si da si odstupio od vjere}. Same izjave koje si dao pred svjetovnim utjecajnim ljudima, kako su novine prenijele tvoje riječi, prezentirane su mi jasno s tvojih usana dok si ih govorio. Ne možemo se truditi da ti damo utjecaj kao nekome kome možemo povjeriti sveti posao povezan s našim institucijama, jer se najprije trebaš obratiti i biti vođen.}[Lt300-1903.5; 1903][https://egwwritings.org/read?panels=p7705.11]

\egwnogap{Nisi čvrst u vjeri. To sam zapisala u svom dnevniku prije nekoliko mjeseci. \textbf{Svakako si postavio narod Božji, kojeg je Gospodin korak po korak vodio putevima istine i postavio na \underline{čvrst temelj}, u lažno svjetlo pred nevjernicima. Neki su se odmaknuli od vjere i \underline{nastavit će krivo predstavljati rad koji mi je Bog dao}}.}[Lt300-1903.6; 1903][https://egwwritings.org/read?panels=p7705.12]

\egwnogap{\textbf{Pitanje svetišta je jasna i određena doktrina kakvu smo držali kao narod. \underline{Nisi definitivno jasan po pitanju ličnosti Boga, što je za nas kao narod sve}. \underline{U suštini si uništio samog Gospodina Boga}}.}[Lt300-1903.7; 1903][https://egwwritings.org/read?panels=p14068.7705013]

\egwnogap{Zašto si uzeo slobodu iznositi izjave koje si iznio, kao da imaš ovlasti tako govoriti, kada su to neistine? \textbf{Učinio si činjenice naše vjere nevažnima pred nevjernicima}, \textbf{i istinu koja bi uvijek trebala biti istaknuta i uzdignuta kod ovog naroda, ti si u suštini zanijekao i ignorirao u svojim mnogim izjavama. Kako si se usudio to učiniti?} \textbf{Sada nas to tjera da predstavimo naš pravi položaj koji nas čini Adventistima Sedmog Dana}. Kakav god utjecaj ti je Bog dao u prošlosti, bio je to čin milosti prema tebi, dopuštajući da svjetlost sja na tebe.}[Lt300-1903.8; 1903][https://egwwritings.org/read?panels=p7705.14]

\egwnogap{\textbf{Ne možemo ni trenutka dopustiti bilo kakvo krivo predstavljanje ovih svečanih i važnih predmeta istine koji su bili vjera našeg naroda od 1844. To nam puno znači.} Gospodin bi htio da ti kažem da je neprijatelj, kroz svoje lukave obmane, postavio svoje nevjerovanje u tvoj um, i ti si to provodio. \textbf{\underline{Svi koji prihvate tvoje prezentacije krenut će na čudne staze ako se povežu s tobom}}. \textbf{\underline{Unosiš} čudnu, običnu vatru}, \textbf{ali ne vatru koju je Bog sam zapalio}; i sada \textbf{moram jasno govoriti našem narodu da nas je Gospodin korak po korak vodio i pokazao nam jasno svjetlo o nebeskom svetištu u Svetinji nad Svetinjama gdje se \underline{Bog otkrio} svojim odabranima}.}[Lt300-1903.9; 1903][https://egwwritings.org/read?panels=p7705.15]

Dr. Kellogg je krivo predstavio istinu koja je činila temelj naše vjere; posebno je krivo predstavio istinu o ličnosti Boga, što je za nas kao narod bilo sve. Ako je 1903. godine bilo potrebno predstaviti \egwinline{\textbf{naš pravi položaj koji nas čini Adventistima Sedmog Dana}}, koliko je to važnije za nas danas? Sestra White je učinila svoj dio u očuvanju temelja naše vjere na početku, ali čini se da smo to zaboravili.

% Dr. Kellogg i spisi Ellen White

\begin{titledpoem}
    \stanza{
        Kellogg je tvrdio da slijedi njene riječi, \\
        No Ellen White je rekla da istinu kriječi. \\
        "Izjavljujem da je ovo laž" jasno je rekla, \\
        Dok je crkva na stranputicu polako tekla.
    }

    \stanza{
        Ličnost Boga za nas je sve, ona piše, \\
        Dok Kellogg o tome krivo govori i diše. \\
        Fundamentalni Principi temelj su vjere, \\
        A ne Trojstvo koje danas mnogi žele.
    }

    \stanza{
        Danas kao i onda, moramo stati, \\
        I pravi položaj vjere predstavljati. \\
        Ellen White nije Trojstvo podržala, \\
        Već istinu o Bogu i Sinu sačuvala.
    }
\end{titledpoem}