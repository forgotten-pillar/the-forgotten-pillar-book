\qrchapter{https://forgottenpillar.com/rsc/en-fp-chapter17}{Reply to Kellogg’s trinitarian sentiments}


\qrchapter{https://forgottenpillar.com/rsc/es-fp-chapter17}{Respuesta a los sentimientos trinitarios de Kellogg}


If we look at the Kellogg crisis through the perspective of the \emcap{personality of God} and the \emcap{Fundamental Principles}, Sister White’s quotations inevitably shine in a new light. In this light we see the conflict between the truth we have received in the beginning, on the \emcap{personality of God}, and the Trinity doctrine. In order to avoid discrepancy, in the interest of defending the Trinity doctrine, scholars always overemphasize the pantheistic side of the problem.


Si miramos la crisis de Kellogg a través de la perspectiva de la \emcap{personalidad de Dios} y los \emcap{Principios Fundamentales}, las citas de la hermana White inevitablemente brillan bajo una nueva luz. A esta luz vemos el conflicto entre la verdad que hemos recibido al principio, sobre la \emcap{personalidad de Dios}, y la doctrina trinitaria. Para evitar la discrepancia, en aras de defender la doctrina trinitaria, los académicos siempre hacen demasiado hincapié en el lado panteísta del problema.


We would like to challenge this tendency to overemphasize the pantheistic side of Kellogg’s controversy. Sister White generally wrote proactive truth; she approached the error by uplifting the truth. This is why she wrote so much about the \emcap{personality of God}. In most of her quotations on this subject, we see her dispelling the Trinitarian error, rather than pantheistic error. We read one such example where she establishes the truth on the \emcap{personality of God} referencing the seventeenth chapter of John.


Nos gustaría desafiar esta tendencia a sobreenfatizar el lado panteísta de la controversia de Kellogg. La hermana White generalmente escribió la verdad proactiva; ella abordó el error elevando la verdad. Por eso escribió tanto sobre la \emcap{personalidad de Dios}. En la mayoría de sus citas sobre este tema, la vemos disipando el error trinitario, en lugar del error panteísta. Leemos un ejemplo en el que establece la verdad sobre la \emcap{personalidad de Dios} refiriéndose al capítulo diecisiete de Juan.


\egw{\textbf{The personality of the Father and the Son, also the unity that exists between Them, are presented in the seventeenth chapter of John}, in the prayer of Christ for His disciples:}[MH 421.7; 1905][https://egwwritings.org/read?panels=p135.2173]


\egw{\textbf{La personalidad del Padre y del Hijo, así como la unidad que existe entre Ellos, se presentan en el capítulo diecisiete de Juan}, en la oración de Cristo por sus discípulos:}[MH 421.7; 1905][https://egwwritings.org/read?panels=p135.2173]


There are many cases where Sister White quotes John 17 in regard to Kellogg’s crisis. Those who assert that Kellogg’s crisis was solely about pantheism should inquire how John 17 addresses God in nature. And it is not only John 17, but also chapters 13-16. In her letter to Kellogg, she wrote:


Hay muchos casos en los que la hermana White cita Juan 17 en relación con la crisis de Kellogg. Aquellos que afirman que la crisis de Kellogg era únicamente sobre el panteísmo, deberían preguntar cómo Juan 17 se refiere a Dios en la naturaleza. Y no es sólo Juan 17, sino también los capítulos 13-16. En su carta a Kellogg, escribió:


\egw{\textbf{\underline{…study the thirteenth, fourteenth, fifteenth, sixteenth, and seventeenth chapters of John}. The words of these chapters explain themselves. ‘This is life eternal,’ Christ declared, ‘that they might know \underline{Thee the only true God}, and Jesus Christ, whom Thou hast sent.’ \underline{In these words the personality of God and of His Son is clearly spoken of.} \underline{The personality of the one does not do away with the necessity for the personality of the other}.}}[Lt232-1903.48, 1903][https://egwwritings.org/read?panels=p10197.57]


\egw{\textbf{\underline{...estudia los capítulos trece, catorce, quince, dieciséis y diecisiete de Juan}. Las palabras de estos capítulos se explican por sí mismas. ‘Esta es la vida eterna’, declaró Cristo, ‘para que te conozcan a \underline{Ti, el único Dios verdadero}, y a Jesucristo, a quien has enviado’. \underline{En estas palabras se habla claramente de la personalidad de Dios y de su Hijo.} \underline{La personalidad de uno no elimina la necesidad de la personalidad del otro}.}}[Lt232-1903.48, 1903][https://egwwritings.org/read?panels=p10197.57]


In the aforementioned chapters of John, John did not reference anything pertaining to God in nature. The content of those chapters covers who is the only true God, how the Father and the Son are one, their true relation, and how Jesus can be everywhere present yet will ascend to the Father.


En los mencionados capítulos de Juan, éste no hace referencia a nada que tenga que ver con Dios en la naturaleza. El contenido de esos capítulos abarca quién es el único Dios verdadero, cómo el Padre y el Hijo son uno, su verdadera relación, y cómo Jesús puede estar presente en todas partes y sin embargo ascenderá al Padre.


\egw{Jesus said to the Jews: ‘My Father worketh hitherto, and I work.... The Son can do nothing of Himself, but what He seeth the Father do: for what things soever He doeth, these also doeth the Son likewise. For the Father loveth the Son, and showeth Him all things that Himself doeth.’ John 5:17-20.}[8T 268.4, 1904][https://egwwritings.org/read?panels=p112.1557]


\egw{Jesús dijo a los judíos: ‘Mi Padre trabaja hasta ahora, y yo trabajo.... El Hijo no puede hacer nada por sí mismo, sino lo que ve hacer al Padre; porque todo lo que Él hace, lo hace también el Hijo. Porque el Padre ama al Hijo, y le muestra todas las cosas que hace’. Juan 5:17-20.}[8T 268.4, 1904][https://egwwritings.org/read?panels=p112.1557]


\egwnogap{\textbf{Here again is brought to view the \underline{personality of the Father and the Son}, showing the unity that exists between them}.}[8T 269.1; 1904][https://egwwritings.org/read?panels=p112.1560]


\egwnogap{\textbf{Aquí nuevamente se pone a la vista la \underline{personalidad del Padre y del Hijo}, mostrando la unidad que existe entre ellos}.}[8T 269.1; 1904][https://egwwritings.org/read?panels=p112.1560]


\egwnogap{\textbf{This unity is expressed also in \underline{the seventeenth chapter of John}}, in the prayer of Christ for His disciples:}[8T 269.2; 1904][https://egwwritings.org/read?panels=p112.1561]


\egwnogap{\textbf{Esta unidad se expresa también en \underline{el capítulo diecisiete de Juan}}, en la oración de Cristo por sus discípulos:}[8T 269.2; 1904][https://egwwritings.org/read?panels=p112.1561]


\egwnogap{‘Neither pray I for these alone, but for them also which shall believe on Me through their word; that they all may be one; \textbf{as Thou, Father, art in Me, and I in Thee, that they also may be one in Us}: that the world may believe that Thou hast sent Me. And \textbf{the glory which Thou gavest Me} I have given them; \textbf{that they may be one, even as We are one: I in them, and Thou in Me, that they may be made perfect in one}; and that the world may know that Thou hast sent Me, and hast loved them, as Thou hast loved Me.’ John 17:20-23.}[8T 269.3; 1904][https://egwwritings.org/read?panels=p112.1562]


\egwnogap{‘Ni ruego solamente por éstos, sino también por los que han de creer en Mí por la palabra de ellos; para que todos sean uno; \textbf{como Tú, oh Padre, en Mí, y Yo en Ti, que también ellos sean uno en Nosotros}: para que el mundo crea que Tú me enviaste. Y \textbf{la gloria que me diste}, Yo les he dado; \textbf{para que sean uno, así como Nosotros somos uno: Yo en ellos, y Tú en Mí, para que sean perfectos en unidad}; y para que el mundo conozca que Tú me enviaste, y que los has amado como también a Mí me has amado.’ Juan 17:20-23.}[8T 269.3; 1904][https://egwwritings.org/read?panels=p112.1562]


\egwnogap{Wonderful statement! \textbf{The unity that exists between Christ and His disciples \underline{does not destroy the personality of either}. They are one in purpose, in mind, in character, but \underline{not in person}. It is thus that God and Christ are one}.}[8T 269.4; 1904][https://egwwritings.org/read?panels=p112.1563]


\egwnogap{¡Maravillosa declaración! \textbf{La unidad que existe entre Cristo y sus discípulos \underline{no destruye la personalidad de ninguno de ellos}. Son uno en propósito, en mente, en carácter, pero \underline{no en persona}. Es así como Dios y Cristo son uno}.}[8T 269.4; 1904][https://egwwritings.org/read?panels=p112.1563]


\egwnogap{\textbf{The relation between the Father and the Son, and the personality of both, are made plain in this scripture also}:}[8T 269.5; 1904][https://egwwritings.org/read?panels=p112.1564]


\egwnogap{\textbf{La relación entre el Padre y el Hijo, y la personalidad de ambos, quedan claras también en esta escritura}:}[8T 269.5; 1904][https://egwwritings.org/read?panels=p112.1564]


\egwnogap{Thus speaketh \textbf{Jehovah of hosts}, saying,} \\
\egw{Behold, \textbf{the man} whose name is\textbf{ the Branch}:} \\
\egw{And He shall grow up out of His place;} \\
\egw{\textbf{And He shall build the temple of Jehovah;... }} \\
\egw{\textbf{And He shall bear the glory,}} \\
\egw{\textbf{And shall sit and rule upon His throne;}} \\
\egw{\textbf{And He shall be a priest upon His throne;}} \\
\egw{\textbf{And \underline{the counsel of peace shall be between Them both}}.’}[8T 269.6; 1904][https://egwwritings.org/read?panels=p112.1565]


\egwnogap{Así habla \textbf{Jehová de los ejércitos}, diciendo,} \\
\egw{He aquí \textbf{el varón} cuyo nombre es \textbf{el Renuevo}:} \\
\egw{Y él brotará de sus raíces;} \\
\egw{\textbf{Y edificará el templo de Jehová;... }} \\
\egw{\textbf{Y él llevará gloria,}} \\
\egw{\textbf{Y se sentará y dominará en su trono;}} \\
\egw{\textbf{Y habrá sacerdote a su lado;}} \\
\egw{\textbf{Y \underline{consejo de paz habrá entre ambos}}.’}[8T 269.6; 1904][https://egwwritings.org/read?panels=p112.1565]


The aforementioned chapters of the Gospel of John deal with the \emcap{personality of God}, which had been expressed in the first two points of the \emcap{Fundamental Principles}. What error did Sister White combat when she referenced verses on how the Father was the only true God, and how the Father and the Son are not one in person? Pantheism? Certainly not; but most probably the trinitarian sentiments, or belief in a one-in-three, or three-in-one God.


Los mencionados capítulos del Evangelio de Juan tratan de la \emcap{personalidad de Dios}, que había sido expresada en los dos primeros puntos de los \emcap{Principios Fundamentales}. ¿Qué error combatió la hermana White cuando hizo referencia a los versículos sobre cómo el Padre era el único Dios verdadero, y cómo el Padre y el Hijo no son uno en persona? ¿Panteísmo? Ciertamente no; pero muy probablemente los sentimientos trinitarios, o la creencia en un Dios uno-en-tres, o tres-en-uno.


Brother J. N. Loughborough, one of the first brethren who wrote on the \emcap{personality of God}, wrote the following comment on John chapter 17:


El hermano J. N. Loughborough, uno de los primeros hermanos que escribió sobre la \emcap{personalidad de Dios}, escribió el siguiente comentario sobre el capítulo 17 de Juan:


\others{\textbf{\underline{The seventeenth chapter of John is alone sufficient to refute the doctrine of the Trinity}}. \textbf{...\underline{Read the seventeenth chapter of John, and see if it does not completely upset the doctrine of the Trinity}}.}[John N. Loughborough, The Adventist Review, and Sabbath Herald, November 5, 1861, p. 184.10][https://egwwritings.org/read?panels=p1685.6615]


\others{\textbf{\underline{El decimoséptimo capítulo de Juan es suficiente para refutar la doctrina de la Trinidad}}. \textbf{...\underline{Lean el capítulo diecisiete de Juan, y vean si no trastorna completamente la doctrina de la Trinidad}}.}[John N. Loughborough, The Adventist Review, and Sabbath Herald, November 5, 1861, p. 184.10][https://egwwritings.org/read?panels=p1685.6615]


Sister White’s proactive writing in support of the truth on the \emcap{personality of God} and His presence is the same as other Adventist pioneers. If Adventist pioneers were debunking the Trinity doctrine by exalting the truth on the \emcap{personality of God} and God’s presence, what makes us think Ellen White was not doing the same, when the theological side of the question of the Trinity was raised? By stating this, we do not deny the pantheistic side of Kellogg’s controversy, but by overemphasizing it, it falls short of accurately describing its real issue. The correct understanding of the Kellogg controversy can only be accomplished by focusing primarily on the truth Sister White uplifted, rather than focusing on error, whether pantheism or Trinity. This truth that Sister White uplifted was the truth on the \emcap{personality of God} and where His presence is. This is expressed in the first point of the \emcap{Fundamental Principles}, which were the official synopsis and representation of Seventh-day Adventist beliefs in the time of Ellen White; the truth which we, as a church, \egwinline{have received and heard and advocated}[Ms124-1905.12; 1905][https://egwwritings.org/read?panels=p9099.18] in the beginning.


La escritura proactiva de la hermana White en apoyo de la verdad sobre la \emcap{personalidad de Dios} y su presencia es la misma que la de otros pioneros adventistas. Si los pioneros adventistas desacreditaban la doctrina trinitaria exaltando la verdad sobre la \emcap{personalidad de Dios} y la presencia de Dios, ¿qué nos hace pensar que Ellen White no hacía lo mismo, cuando se planteaba el lado teológico de la cuestión de la Trinidad? Al afirmar esto, no negamos el lado panteísta de la controversia de Kellogg, pero al enfatizarlo en exceso, no alcanza a describir con exactitud su verdadera cuestión. La comprensión correcta de la controversia de Kellogg sólo puede lograrse centrándose principalmente en la verdad que la hermana White levantó, en lugar de centrarse en el error, ya sea el panteísmo o la Trinidad. Esta verdad que la hermana White levantó fue la verdad sobre la \emcap{personalidad de Dios} y dónde está su presencia. Esto se expresa en el primer punto de los \emcap{Principios Fundamentales}, que eran la sinopsis oficial y la representación de las creencias adventistas del séptimo día en la época de Ellen White; la verdad que nosotros, como iglesia, \egwinline{hemos recibido y oído y defendido}[Ms124-1905.12; 1905][https://egwwritings.org/read?panels=p9099.18] en el principio.


\egw{\textbf{I entreat every one to be clear and firm regarding the certain truths that we have received and heard and advocated. The statements of God’s Word are plain. Plant your feet firmly on \underline{the platform of eternal truth}. \underline{Reject every phase of error}, even \underline{though it be covered with a semblance of reality, which denies the personality of God or of Christ}}.}[Ms124-1905.12; 1905][https://egwwritings.org/read?panels=p9099.18]


\egw{\textbf{Ruego a todos que sean claros y firmes en cuanto a las verdades ciertas que hemos recibido y oído y defendido. Las declaraciones de la Palabra de Dios son claras. Planten sus pies firmemente en \underline{la plataforma de la verdad eterna}. \underline{Rechazad toda fase de error}, aunque \underline{esté cubierta de una apariencia de realidad, que niegue la personalidad de Dios o de Cristo}}.}[Ms124-1905.12; 1905][https://egwwritings.org/read?panels=p9099.18]


The warning from the previous quotations did not lessen in the course of time. Today it is even more relevant. We should \egwinline{reject every phase of error, even though it be covered with a semblance of reality, which denies the personality of God or of Christ}. In the following chapter we want to point out to the specific phase of error that is covered with a semblance of reality, which denies the personality of God and of Christ—three living persons of \textit{one} God, as opposed to \egwinline{three living persons of the heavenly trio.}[Ms21-1906.11; 1906][https://egwwritings.org/read?panels=p9754.18]


La advertencia de las citas anteriores no disminuyó con el paso del tiempo. Hoy es aún más pertinente. Debemos \egwinline{rechazar toda fase de error, aunque esté cubierta con una apariencia de realidad, que niegue la personalidad de Dios o de Cristo}. En el siguiente capítulo queremos señalar la fase específica de error que está cubierta con una apariencia de realidad, que niega la personalidad de Dios y de Cristo—tres personas vivientes de \textit{un} Dios, en oposición a \egwinline{tres personas vivientes del trío celestial.}[Ms21-1906.11; 1906][https://egwwritings.org/read?panels=p9754.18]


% Reply to Kellogg’s trinitarian sentiments

\begin{titledpoem}
    
    \stanza{
        The light of truth, so clear and bold, \\
        A crisis came, a story told. \\
        Not pantheism, dim and wide, \\
        But God’s persona, we confide.
    }

    \stanza{
        But God, through Ellen, did uphold \\
        God’s personality was told. \\
        Against the Trinity, she leaned, \\
        A unity, by John unseen.
    }

    \stanza{
        "The Father and the Son," she wrote, \\
        Are one in purpose was her quote. \\
        John seventeen, her chosen guide, \\
        Where God’s true nature cannot hide.
    }

    \stanza{
        The pioneers, with her agreed, \\
        Of God’s true person, they did plead. \\
        Loughborough echoed, his words clear, \\
        The Trinity dismissed, no fear.
    }

    \stanza{
        The Fundamental Points, so dear, \\
        They make it plain, we must revere. \\
        Not in the trinity’s wrong creed, \\
        But in His presence, faith is freed.
    }

    \stanza{
        So let us stand on truth so bright, \\
        Rejecting wrong, with all our might. \\
        God’s person, where we find our plea, \\
        Truth’s platform for eternity.
    }
    
\end{titledpoem}

% \qrchapter{https://forgottenpillar.com/rsc/en-fp-chapter17}{Reply to Kellogg’s trinitarian sentiments}


\qrchapter{https://forgottenpillar.com/rsc/en-fp-chapter17}{Jibu kwa hisia za utatu za Kellogg}


If we look at the Kellogg crisis through the perspective of the \emcap{personality of God} and the \emcap{Fundamental Principles}, Sister White’s quotations inevitably shine in a new light. In this light we see the conflict between the truth we have received in the beginning, on the \emcap{personality of God}, and the Trinity doctrine. In order to avoid discrepancy, in the interest of defending the Trinity doctrine, scholars always overemphasize the pantheistic side of the problem.


Ikiwa tunatazama shida ya Kellogg kupitia mtazamo wa \emcap{ubinafsi wa Mungu} na \emcap{Kanuni za Msingi}, manukuu ya Dada White bila shaka yanaangaza kwa Mwanga mpya. Katika mwanga huu tunaona mgongano kati ya ukweli tuliopokea hapo mwanzo, juu ya \emcap{ubinafsi wa Mungu}, na fundisho la Utatu. Ili kuepusha hitilafu, kwa maslahi ya kutetea Fundisho la Utatu, wasomi daima husisitiza zaidi upande wa pantheism wa tatizo.


We would like to challenge this tendency to overemphasize the pantheistic side of Kellogg’s controversy. Sister White generally wrote proactive truth; she approached the error by uplifting the truth. This is why she wrote so much about the \emcap{personality of God}. In most of her quotations on this subject, we see her dispelling the Trinitarian error, rather than pantheistic error. We read one such example where she establishes the truth on the \emcap{personality of God} referencing the seventeenth chapter of John.


Tungependa kutoa changamoto kwa tabia hii ya kusisitiza zaidi upande wa pantheism wa mgogoro wa Kellogg. Dada White kwa ujumla aliandika ukweli; alikaripia kosa kwa kuinua ukweli. Hii ndiyo sababu aliandika sana kuhusu \emcap{ubinafsi wa Mungu}. Katika zaidi ya nukuu zake juu ya somo hili, tunamwona akiondoa kosa la Utatu, badala ya kosa la pantheism. Tunasoma mfano mmoja kama huo ambapo anathibitisha ukweli juu ya \emcap{ubinafsi wa Mungu} akiirejelea sura ya kumi na saba ya Yohana.


\egw{\textbf{The personality of the Father and the Son, also the unity that exists between Them, are presented in the seventeenth chapter of John}, in the prayer of Christ for His disciples:}[MH 421.7; 1905][https://egwwritings.org/read?panels=p135.2173]


\egw{\textbf{Ubinafsi wa Baba na Mwana, pia umoja uliopo kati Yao, yametolewa katika sura ya kumi na saba ya Yohana, katika maombi ya Kristo kwa ajili Ya wanafunzi wake:}}[MH 421.7; 1905][https://egwwritings.org/read?panels=p135.2173]


There are many cases where Sister White quotes John 17 in regard to Kellogg’s crisis. Those who assert that Kellogg’s crisis was solely about pantheism should inquire how John 17 addresses God in nature. And it is not only John 17, but also chapters 13-16. In her letter to Kellogg, she wrote:


Kuna matukio mengi ambapo Dada White ananukuu Yohana 17 kuhusiana na mgogoro wa Kellogg. Wale wanaodai kwamba mzozo wa Kellogg ulihusu pantheism tu wanapaswa kuuliza jinsi John 17 inazungumzia Mungu katika asili. Na si Yohana 17 tu, bali pia sura za 13-16. Katika barua yake kwa Kellogg, aliandika:


\egw{\textbf{\underline{…study the thirteenth, fourteenth, fifteenth, sixteenth, and seventeenth chapters of John}. The words of these chapters explain themselves. ‘This is life eternal,’ Christ declared, ‘that they might know \underline{Thee the only true God}, and Jesus Christ, whom Thou hast sent.’ \underline{In these words the personality of God and of His Son is clearly spoken of.} \underline{The personality of the one does not do away with the necessity for the personality of the other}.}}[Lt232-1903.48, 1903][https://egwwritings.org/read?panels=p10197.57]


\egw{\textbf{\underline{...soma sura ya kumi na tatu, ya kumi na nne, ya kumi na tano, ya kumi na sita, na ya kumi na saba ya Yohana}. Maneno ya sura hizi yanajieleza yenyewe. ‘Huu ndio uzima wa milele,’ Kristo alitangaza, ‘wapate kukujua \underline{wewe, Mungu wa pekee wa kweli}, na Yesu Kristo ambaye Wewe amemtuma.’ \underline{Katika maneno haya ubinafsi wa Mungu na wa Mwanawe unasemwa waziwazi.} \underline{Ubinafsi wa Mmoja hauondoi umuhimu wa ubinafsi wa mwingine}.}}[Lt232-1903.48, 1903][https://egwwritings.org/read?panels=p10197.57]


In the aforementioned chapters of John, John did not reference anything pertaining to God in nature. The content of those chapters covers who is the only true God, how the Father and the Son are one, their true relation, and how Jesus can be everywhere present yet will ascend to the Father.


Katika sura zilizotajwa hapo juu za Yohana, Yohana hakurejelea chochote kinachomhusu Mungu katika Asili. Maudhui ya sura hizo yanahusu ni nani aliye Mungu wa pekee wa kweli, jinsi Baba na Mwana ni mmoja, uhusiano wao wa kweli, na jinsi Yesu anavyoweza kuwepo kila mahali na bado kupaa kwa Baba.


\egw{Jesus said to the Jews: ‘My Father worketh hitherto, and I work.... The Son can do nothing of Himself, but what He seeth the Father do: for what things soever He doeth, these also doeth the Son likewise. For the Father loveth the Son, and showeth Him all things that Himself doeth.’ John 5:17-20.}[8T 268.4, 1904][https://egwwritings.org/read?panels=p112.1557]


\egw{Yesu aliwaambia Wayahudi: ‘Baba yangu anafanya kazi hata sasa, nami ninafanya kazi.... Mwana hawezi kufanya lolote bali lile analomuona Baba analifanya; kwa kuwa yote ayatendayo, hayo pia naye Mwana vivyo hivyo. Kwa maana Baba anampenda Mwana, na humwonyesha mambo yote ambayo Mwenyewe anafanya.’ Yohana 5:17-20.}[8T 268.4, 1904][https://egwwritings.org/read?panels=p112.1557]


\egwnogap{\textbf{Here again is brought to view the \underline{personality of the Father and the Son}, showing the unity that exists between them}.}[8T 269.1; 1904][https://egwwritings.org/read?panels=p112.1560]


\egwnogap{\textbf{Hapa tena inaletwa machoni petu, \underline{ubinafsi wa Baba na Mwana}, ikionyesha umoja uliopo kati yao}.}[8T 269.1; 1904][https://egwwritings.org/read?panels=p112.1560]


\egwnogap{\textbf{This unity is expressed also in \underline{the seventeenth chapter of John}}, in the prayer of Christ for His disciples:}[8T 269.2; 1904][https://egwwritings.org/read?panels=p112.1561]


\egwnogap{\textbf{Umoja huu unaonyeshwa pia katika \underline{sura ya kumi na saba ya Yohana}, katika maombi ya Kristo kwa wanafunzi Wake:}}[8T 269.2; 1904][https://egwwritings.org/read?panels=p112.1561]


\egwnogap{‘Neither pray I for these alone, but for them also which shall believe on Me through their word; that they all may be one; \textbf{as Thou, Father, art in Me, and I in Thee, that they also may be one in Us}: that the world may believe that Thou hast sent Me. And \textbf{the glory which Thou gavest Me} I have given them; \textbf{that they may be one, even as We are one: I in them, and Thou in Me, that they may be made perfect in one}; and that the world may know that Thou hast sent Me, and hast loved them, as Thou hast loved Me.’ John 17:20-23.}[8T 269.3; 1904][https://egwwritings.org/read?panels=p112.1562]


\egwnogap{‘Wala siwaombei hawa peke yao, bali na wale watakaoniamini kwa njia ya neno lao; ili wote wawe kitu kimoja; \textbf{kama wewe, Baba, ulivyo ndani yangu, nami ndani yako, hao nao pia wawe wamoja ndani Yetu}: ili ulimwengu upate kusadiki ya kwamba ndiwe uliyenituma. Na \textbf{utukufu ambao Ulinipa Mimi} nimewapa wao; \textbf{ili wawe na umoja kama sisi tulivyo na umoja: mimi ndani yao, nawe ndani yangu, ili wawe wamekamilika katika umoja}; na ili ulimwengu upate kujua hayo Umenituma mimi, nawe umewapenda wao kama vile ulivyonipenda mimi.’ Yohana 17:20-23.}[8T 269.3; 1904][https://egwwritings.org/read?panels=p112.1562]


\egwnogap{Wonderful statement! \textbf{The unity that exists between Christ and His disciples \underline{does not destroy the personality of either}. They are one in purpose, in mind, in character, but \underline{not in person}. It is thus that God and Christ are one}.}[8T 269.4; 1904][https://egwwritings.org/read?panels=p112.1563]


\egwnogap{Kauli ya ajabu! \textbf{Umoja uliopo kati ya Kristo na wanafunzi wake \underline{hauharibu ubinafsi wa mwingine}. Wao ni wamoja katika kusudi, akilini, katika tabia, lakini \underline{si kwa nafsi}. Hivyo ndivyo Mungu na Kristo Wana umoja}.}[8T 269.4; 1904][https://egwwritings.org/read?panels=p112.1563]


\egwnogap{\textbf{The relation between the Father and the Son, and the personality of both, are made plain in this scripture also}:}[8T 269.5; 1904][https://egwwritings.org/read?panels=p112.1564]


\egwnogap{\textbf{Uhusiano kati ya Baba na Mwana, na ubinafsi wa wote wawili, unafanywa wazi katika andiko pia}:}[8T 269.5; 1904][https://egwwritings.org/read?panels=p112.1564]


\egwnogap{Thus speaketh \textbf{Jehovah of hosts}, saying,} \\
\egw{Behold, \textbf{the man} whose name is\textbf{ the Branch}:} \\
\egw{And He shall grow up out of His place;} \\
\egw{\textbf{And He shall build the temple of Jehovah;... }} \\
\egw{\textbf{And He shall bear the glory,}} \\
\egw{\textbf{And shall sit and rule upon His throne;}} \\
\egw{\textbf{And He shall be a priest upon His throne;}} \\
\egw{\textbf{And \underline{the counsel of peace shall be between Them both}}.’}[8T 269.6; 1904][https://egwwritings.org/read?panels=p112.1565]


\egwnogap{Asema hivi \textbf{BWANA wa majeshi},} \\
\egw{Tazama, \textbf{mtu} ambaye jina lake ni \textbf{Tawi}:} \\
\egw{Naye atakua kutoka mahali pake;} \\
\egw{\textbf{Naye atalijenga hekalu la BWANA;... }} \\
\egw{\textbf{Naye atabeba utukufu,}} \\
\egw{\textbf{Naye ataketi na kutawala juu ya kiti chake cha enzi;}} \\
\egw{\textbf{Naye atakuwa kuhani katika kiti chake cha enzi;}} \\
\egw{\textbf{Na \underline{shauri la amani litakuwa kati ya hao wawili}}.’}[8T 269.6; 1904][https://egwwritings.org/read?panels=p112.1565]


The aforementioned chapters of the Gospel of John deal with the \emcap{personality of God}, which had been expressed in the first two points of the \emcap{Fundamental Principles}. What error did Sister White combat when she referenced verses on how the Father was the only true God, and how the Father and the Son are not one in person? Pantheism? Certainly not; but most probably the trinitarian sentiments, or belief in a one-in-three, or three-in-one God.


Sura zilizotajwa hapo juu za Injili ya Yohana zinahusu \emcap{ubinafsi wa Mungu}, ambao umeelezwa katika hoja mbili za kwanza za \emcap{Kanuni za Msingi}. Ni kosa lipi Dada White alipigana aliporejelea mistari kuhusu jinsi Baba ni Mungu wa pekee wa kweli, na jinsi Baba na Mwana si wamoja katika nafsi? Pantheism? Hakika sivyo; lakini wengi pengine hisia za utatu, au imani katika Mungu mmoja-katika-tatu, au watatu-katika-mmoja.


Brother J. N. Loughborough, one of the first brethren who wrote on the \emcap{personality of God}, wrote the following comment on John chapter 17:


Ndugu J. N. Loughborough, mmoja wa ndugu wa kwanza walioandika juu ya \emcap{ubinafsi wa Mungu}, aliandika maelezo yafuatayo juu ya Yohana sura ya 17:


\others{\textbf{\underline{The seventeenth chapter of John is alone sufficient to refute the doctrine of the Trinity}}. \textbf{...\underline{Read the seventeenth chapter of John, and see if it does not completely upset the doctrine of the Trinity}}.}[John N. Loughborough, The Adventist Review, and Sabbath Herald, November 5, 1861, p. 184.10][https://egwwritings.org/read?panels=p1685.6615]


\others{\textbf{\underline{Sura ya kumi na saba ya Yohana pekee inatosha kukanusha fundisho la Utatu}}. \textbf{...Soma sura ya kumi na saba ya Yohana, na uone kama haifanyi hivyo kabisa kuvuruga fundisho la Utatu}.}[John N. Loughborough, The Adventist Review, and Sabbath Herald, November 5, 1861, p. 184.10][https://egwwritings.org/read?panels=p1685.6615]


Sister White’s proactive writing in support of the truth on the \emcap{personality of God} and His presence is the same as other Adventist pioneers. If Adventist pioneers were debunking the Trinity doctrine by exalting the truth on the \emcap{personality of God} and God’s presence, what makes us think Ellen White was not doing the same, when the theological side of the question of the Trinity was raised? By stating this, we do not deny the pantheistic side of Kellogg’s controversy, but by overemphasizing it, it falls short of accurately describing its real issue. The correct understanding of the Kellogg controversy can only be accomplished by focusing primarily on the truth Sister White uplifted, rather than focusing on error, whether pantheism or Trinity. This truth that Sister White uplifted was the truth on the \emcap{personality of God} and where His presence is. This is expressed in the first point of the \emcap{Fundamental Principles}, which were the official synopsis and representation of Seventh-day Adventist beliefs in the time of Ellen White; the truth which we, as a church, \egwinline{have received and heard and advocated}[Ms124-1905.12; 1905][https://egwwritings.org/read?panels=p9099.18] in the beginning.


Uandishi wa umakini wa Dada White katika kuunga mkono ukweli juu ya \emcap{ubinafsi wa Mungu} na uwepo wake ni sawa na waanzilishi wengine wa Kiadventista. Ikiwa waanzilishi wa Kiadventista walikuwa wanakanusha Fundisho la Utatu kwa kuinua ukweli juu ya \emcap{Ubinafsi wa Mungu} na uwepo wa Mungu, nini inatufanya tufikiri Ellen White hakuwa anafanya hivyo, wakati upande wa kitheolojia wa swali la Utatu lilizushwa? Kwa kusema hili, hatukatai upande wa pantheism wa Mzozo wa Kellogg, lakini kwa kuusisitiza kupita kiasi, unashindwa kuelezea kwa usahihi ukweli wa suala Hilo. Uelewa sahihi wa utata wa Kellogg unaweza tu kukamilika kwa kulenga hasa ukweli ulioinuliwa Dada White, badala ya kuzingatia makosa, kama pantheism au Utatu. Ukweli huu ambao Dada White aliinua ulikuwa ukweli juu ya \emcap{ubinafsi wa Mungu} na ulipo uwepo wake. Hii inaonyeshwa katika hoja ya kwanza ya \emcap{Kanuni za Msingi}, ambazo zilikuwa muhtasari rasmi na uwakilishi wa Imani ya Waadventista Wasabato katika wakati wa Ellen White; ukweli ambao sisi kama kanisa \egwinline{tumeupokea na kuusikia na kutetea}[Ms124-1905.12; 1905][https://egwwritings.org/read?panels=p9099.18] hapo mwanzo.


\egw{\textbf{I entreat every one to be clear and firm regarding the certain truths that we have received and heard and advocated. The statements of God’s Word are plain. Plant your feet firmly on \underline{the platform of eternal truth}. \underline{Reject every phase of error}, even \underline{though it be covered with a semblance of reality, which denies the personality of God or of Christ}}.}[Ms124-1905.12; 1905][https://egwwritings.org/read?panels=p9099.18]


\egw{\textbf{Ninasihi kila mmoja awe wazi na thabiti kuhusu kweli fulani tulizopokea na kusikia na zilizotetewa. Kauli za Neno la Mungu ziko wazi. Panda miguu yako imara kwenye \underline{jukwaa la ukweli wa milele}. \underline{Kataa kila awamu ya makosa}, hata \underline{ingawa imefunikwa na mwonekano wa ukweli, ambao unakana ubinafsi wa Mungu au ya Kristo}}.}[Ms124-1905.12; 1905][https://egwwritings.org/read?panels=p9099.18]


The warning from the previous quotations did not lessen in the course of time. Today it is even more relevant. We should \egwinline{reject every phase of error, even though it be covered with a semblance of reality, which denies the personality of God or of Christ}. In the following chapter we want to point out to the specific phase of error that is covered with a semblance of reality, which denies the personality of God and of Christ—three living persons of \textit{one} God, as opposed to \egwinline{three living persons of the heavenly trio.}[Ms21-1906.11; 1906][https://egwwritings.org/read?panels=p9754.18]


Onyo kutoka kwa nukuu zilizopita hazikupungua baada ya muda. Leo ni muhimu zaidi. Tunapaswa \egwinline{kukataa kila awamu ya kosa, ingawa imefunikwa kwa mfano wa ukweli, unaokana ubinafsi wa Mungu au wa Kristo}. Katika sura inayofuata tunataka kuonyesha awamu maalum ya kosa ambalo limefunikwa na mwonekano wa ukweli, ambalo linakana ubinafsi wa Mungu na wa Kristo—nafsi tatu hai za \textit{mmoja} Mungu, kinyume na \egwinline{nafsi tatu hai za utatu wa mbinguni.}[Ms21-1906.11; 1906][https://egwwritings.org/read?panels=p9754.18]


% Reply to Kellogg’s trinitarian sentiments

\begin{titledpoem}
    
    \stanza{
        The light of truth, so clear and bold, \\
        A crisis came, a story told. \\
        Not pantheism, dim and wide, \\
        But God’s persona, we confide.
    }

    \stanza{
        But God, through Ellen, did uphold \\
        God’s personality was told. \\
        Against the Trinity, she leaned, \\
        A unity, by John unseen.
    }

    \stanza{
        "The Father and the Son," she wrote, \\
        Are one in purpose was her quote. \\
        John seventeen, her chosen guide, \\
        Where God’s true nature cannot hide.
    }

    \stanza{
        The pioneers, with her agreed, \\
        Of God’s true person, they did plead. \\
        Loughborough echoed, his words clear, \\
        The Trinity dismissed, no fear.
    }

    \stanza{
        The Fundamental Points, so dear, \\
        They make it plain, we must revere. \\
        Not in the trinity’s wrong creed, \\
        But in His presence, faith is freed.
    }

    \stanza{
        So let us stand on truth so bright, \\
        Rejecting wrong, with all our might. \\
        God’s person, where we find our plea, \\
        Truth’s platform for eternity.
    }
    
\end{titledpoem}


% Reply to Kellogg’s trinitarian sentiments

\begin{titledpoem}
    
    \stanza{
        The light of truth, so clear and bold, \\
        A crisis came, a story told. \\
        Not pantheism, dim and wide, \\
        But God’s persona, we confide.
    }

    \stanza{
        But God, through Ellen, did uphold \\
        God’s personality was told. \\
        Against the Trinity, she leaned, \\
        A unity, by John unseen.
    }

    \stanza{
        "The Father and the Son," she wrote, \\
        Are one in purpose was her quote. \\
        John seventeen, her chosen guide, \\
        Where God’s true nature cannot hide.
    }

    \stanza{
        The pioneers, with her agreed, \\
        Of God’s true person, they did plead. \\
        Loughborough echoed, his words clear, \\
        The Trinity dismissed, no fear.
    }

    \stanza{
        The Fundamental Points, so dear, \\
        They make it plain, we must revere. \\
        Not in the trinity’s wrong creed, \\
        But in His presence, faith is freed.
    }

    \stanza{
        So let us stand on truth so bright, \\
        Rejecting wrong, with all our might. \\
        God’s person, where we find our plea, \\
        Truth’s platform for eternity.
    }
    
\end{titledpoem}

Jibu kwa hisia za utatu za Kellogg

Ikiwa tunatazama shida ya Kellogg kupitia mtazamo wa ubinafsi wa Mungu na Kanuni za Msingi, manukuu ya Dada White bila shaka yanaangaza kwa Mwanga mpya. Katika mwanga huu tunaona mgongano kati ya ukweli tuliopokea hapo mwanzo, juu ya ubinafsi wa Mungu, na fundisho la Utatu. Ili kuepusha hitilafu, kwa maslahi ya kutetea Fundisho la Utatu, wasomi daima husisitiza zaidi upande wa pantheism wa tatizo.

