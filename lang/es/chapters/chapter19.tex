\qrchapter{https://forgottenpillar.com/rsc/en-fp-chapter19}{Ellen White and Matthew 28:19}


\qrchapter{https://forgottenpillar.com/rsc/en-fp-chapter19}{Ellen White and Matthew 28:19}


Many assert that Ellen White promoted the Trinity doctrine, and that she is the one responsible for accepting it into our ranks. These claims do not consider that she defended the \emcap{personality of God} expressed in the first point of the \emcap{Fundamental Principles}. To support the claims that Ellen White was trinitarian, quotations are presented to her comment on Matthew 28:19:


Muchos afirman que Elena G. de White promovió la doctrina trinitaria, y que ella es la responsable de aceptarla en nuestras filas. Estas afirmaciones no consideran que ella defendió la \emcap{personalidad de Dios} expresada en el primer punto de los \emcap{Principios Fundamentales}. Para apoyar las afirmaciones de que Elena G. de White era trinitaria, se presentan citas a su comentario sobre Mateo 28:19:


\bible{Go ye therefore, and teach all nations, \textbf{baptizing them in the name of \underline{the Father}, and of \underline{the Son}, and of \underline{the Holy Ghost}}.}[Matthew 28:19]


\bible{Por tanto, id, y haced discípulos a todas las naciones, \textbf{bautizándolos en el nombre de \underline{el Padre}, y de \underline{el Hijo}, y de \underline{el Espíritu Santo}}.}[Mateo 28:19]


This verse has been most prominent in support of the Trinity doctrine. The Trinity doctrine has propositions about the \emcap{personality of God} of which this text says nothing to support. This verse itself does not teach that the Father, the Son, and the Holy Ghost, comprise \textit{one} God, the God of the Bible. There are other explicit verses in the Bible that exclude such interpretation of the text, i.e. 1 Corinthians 8:4-6; John 17:3; Ephesians 4:4-6; 1 Timothy 2:5.


Este versículo ha sido muy prominente en apoyo de la doctrina trinitaria. La doctrina trinitaria tiene proposiciones sobre la \emcap{personalidad de Dios} de las cuales este texto no dice nada para apoyar. Este versículo en sí mismo no enseña que el Padre, el Hijo y el Espíritu Santo, constituyen \textit{un} solo Dios, el Dios de la Biblia. Hay otros versículos explícitos en la Biblia que excluyen tal interpretación del texto, es decir, 1 Corintios 8:4-6; Juan 17:3; Efesios 4:4-6; 1 Timoteo 2:5.


Unfortunately, the same unsupported assumptions made about Matthew 28:19 are made about Sister White’s quotations dealing with this verse. For example, Sister White uses terms like \egwinline{three highest powers in heaven}[Lt253a-1903.18; 1903][https://egwwritings.org/read?panels=p10143.25], \egwinline{three great powers of heaven}[8T 254.1; 1904][https://egwwritings.org/read?panels=p112.1450], \egwinline{the three holy dignitaries of heaven}[Ms92-1901.26: 1901][https://egwwritings.org/read?panels=p10732.32] and similar expressions—none of these quotations justify the assumption that these three (the Father, the Son, and the Holy Spirit) make \textit{one} God. On the contrary, as discussed in the previous chapter, keeping William Boardman’s sentiments and \egwinline{the heavenly trio} in context, “\textit{three-in-one}” sentiments \egwinline{should not be trusted}[Ms21-1906.8; 1906][https://egwwritings.org/read?panels=p9754.15].


Lamentablemente, las mismas suposiciones sin fundamento que se hacen sobre Mateo 28:19 se hacen sobre las citas de la hermana White que tratan de este versículo. Por ejemplo, la hermana White utiliza términos como \egwinline{los tres poderes más altos del cielo}[Lt253a-1903.18; 1903][https://egwwritings.org/read?panels=p10143.25], \egwinline{los tres grandes poderes del cielo}[8T 254.1; 1904][https://egwwritings.org/read?panels=p112.1450], \egwinline{los tres santos dignatarios del cielo}[Ms92-1901.26: 1901][https://egwwritings.org/read?panels=p10732.32] y expresiones similares—ninguna de estas citas justifica la suposición de que estos tres (el Padre, el Hijo y el Espíritu Santo) forman \textit{un} solo Dios. Por el contrario, como se discutió en el capítulo anterior, manteniendo los sentimientos de William Boardman y \egwinline{el trío celestial} en contexto, los sentimientos “\textit{tres-en-uno}” \egwinline{no deben ser confiados}[Ms21-1906.8; 1906][https://egwwritings.org/read?panels=p9754.15].


The heavenly trio (the group of three: the Father, the Son and the Holy Spirit) are also present in other Bible verses, in addition to Matthew 28:19. There are several other instances in the New Testament where the Father, the Son and the Holy Spirit are mentioned, and these verses should be used to interpret the meaning behind the heavenly trio. None of the verses on the heavenly trio prove a three-in-one God; rather, all of them refer to the Father as one God. In the following verses, the heavenly trio is bolded in order to better distinguish the Father, the Son and the Holy Spirit.


El trío celestial (el grupo de tres: el Padre, el Hijo y el Espíritu Santo) también está presente en otros versículos bíblicos, además de Mateo 28:19. Hay varios otros casos en el Nuevo Testamento en los que se menciona al Padre, al Hijo y al Espíritu Santo, y estos versículos deben utilizarse para interpretar el significado del trío celestial. Ninguno de los versículos sobre el trío celestial demuestra un Dios de tres-en-uno; más bien, todos ellos se refieren al Padre como un solo Dios. En los siguientes versículos, el trío celestial está en negrita para distinguir mejor al Padre, al Hijo y al Espíritu Santo.


\bible{There is one body, and \textbf{one Spirit}, even as ye are called in one hope of your calling; \textbf{One Lord}, one faith, one baptism, \textbf{One God and Father} of all, who is above all, and through all, and in you all.}[Ephesians 4:4-6]


\bible{Un cuerpo, y \textbf{un Espíritu}, como fuisteis también llamados en una misma esperanza de vuestra vocación; \textbf{un Señor}, una fe, un bautismo, \textbf{un Dios y Padre} de todos, el cual es sobre todos, y por todos, y en todos.}[Efesios 4:4-6]


\bible{Now there are diversities of gifts, but the \textbf{same Spirit}. And there are differences of administrations, but the \textbf{same Lord}. And there are diversities of operations, but it is \textbf{the same God} which worketh all in all.}[1 Corinthians 12:4-6]


\bible{Ahora bien, hay diversidad de dones, pero el \textbf{mismo Espíritu}. Y hay diversidad de ministerios, pero el \textbf{mismo Señor}. Y hay diversidad de operaciones, pero \textbf{Dios, que hace todas las cosas en todos, es el mismo}.}[1 Corintios 12:4-6]


\bible{The grace of \textbf{the Lord Jesus Christ}, and the love of \textbf{God}, and the communion of \textbf{the Holy Ghost}, be with you all. Amen.}[2 Corinthians 13:14]


\bible{La gracia de \textbf{el Señor Jesucristo}, el amor de \textbf{Dios} y la comunión de \textbf{el Espíritu Santo} sean con todos vosotros. Amén.}[2 Corintios 13:14]


\bible{For through \textbf{him} \normaltext{[Christ]} we both have access by one \textbf{Spirit} unto the \textbf{Father}.}[Ephesians 2:18]


\bible{Porque por medio de \textbf{él} \normaltext{[Cristo]} los unos y los otros tenemos entrada por un mismo \textbf{Espíritu} al \textbf{Padre}.}[Efesios 2:18]


\bible{But we are bound to give thanks alway to \textbf{God} for you, brethren beloved of \textbf{the Lord}, because \textbf{God} hath from the beginning chosen you to salvation through sanctification of \textbf{the Spirit} and belief of the truth.}[2 Thessalonians 2:13]


\bible{Pero nosotros debemos dar siempre gracias a \textbf{Dios} respecto a vosotros, hermanos amados por \textbf{el Señor}, de que \textbf{Dios} os haya escogido desde el principio para salvación, mediante la santificación por \textbf{el Espíritu} y la fe en la verdad.}[2 Tesalonicenses 2:13]


\bible{How much more shall the blood of \textbf{Christ}, who through the eternal \textbf{Spirit} offered himself without spot to \textbf{God}, purge your conscience from dead works to serve \textbf{the living God}?}[Hebrews 9:14]


\bible{¿Cuánto más la sangre de \textbf{Cristo}, el cual mediante el \textbf{Espíritu} eterno se ofreció a sí mismo sin mancha a \textbf{Dios}, limpiará vuestras conciencias de obras muertas para que sirváis al \textbf{Dios vivo}?}[Hebreos 9:14]


\bible{Elect according to the foreknowledge of \textbf{God the Father}, through sanctification of \textbf{the Spirit}, unto obedience and sprinkling of the blood of \textbf{Jesus Christ}: Grace unto you, and peace, be multiplied.}[1 Peter 1:2]


\bible{Elegidos según la presciencia de \textbf{Dios Padre} en santificación del \textbf{Espíritu}, para obedecer y ser rociados con la sangre de \textbf{Jesucristo}: Gracia y paz os sean multiplicadas.}[1 Pedro 1:2]


All of the above verses talk about the heavenly trio (the Father, the Son and the Holy Spirit), and all of them consistently testify that the Father is the one referred to as God.
The same reasoning holds ground for Ellen White’s interpretation of Matthew 28:19.


Todos los versículos anteriores hablan del trío celestial (el Padre, el Hijo y el Espíritu Santo), y todos ellos testifican consistentemente que el Padre es el que se refiere como Dios.
El mismo razonamiento es válido para la interpretación de Elena G. de White de Mateo 28:19.


\egw{Christ gave His followers a positive promise that after His ascension He would send them His Spirit. ‘Go ye therefore,’ He said, ‘and teach all nations, baptizing them in the name of \textbf{the Father (a personal God),} and of \textbf{the Son (a personal Prince and Saviour),} and of \textbf{the Holy Ghost (sent from heaven to represent Christ);} teaching them to observe all things whatsoever I have commanded you, and, lo, I am with you alway, even unto the end of the world.’ Matthew 28:19, 20.}[RH October 26, 1897, par. 9; 1897][https://egwwritings.org/read?panels=p821.16317]


\egw{Cristo dio a sus seguidores una promesa positiva de que después de su ascensión les enviaría su Espíritu. ‘Id, pues,’ dijo, ‘y enseñad a todas las naciones, bautizándolos en el nombre del \textbf{Padre (un Dios personal),} y del \textbf{Hijo (un Príncipe y Salvador personal),} y del \textbf{Espíritu Santo (enviado desde el cielo para representar a Cristo);} enseñándoles a guardar todo lo que os he mandado, y he aquí que yo estoy con vosotros todos los días, hasta el fin del mundo’. Mateo 28:19, 20.}[RH October 26, 1897, par. 9; 1897][https://egwwritings.org/read?panels=p821.16317]


The brackets in this quotation are in the original manuscript written by Ellen White. Here, she gives her own interpretation of Matthew 28:19. The Father is a personal God, the Son is a personal Prince and Saviour, and the Holy Spirit is Christ’s representative. This interpretation is in harmony with the \emcap{personality of God} expressed in the first point of the \emcap{Fundamental Principles}. Matthew 28:19 is a matter of interpretation. The interpretation which makes the Heavenly Trio one God is not inspired. This is not what the text indicates. Rather, let's read Matthew 28:19 within inspired compound: “\textit{Go ye therefore, and teach all nations, baptizing them in the name of a personal God, a personal Prince and Savior, and of the Holy Ghost}.” If one would read the text as such, no one would ever assume that one God is a unity of three persons. Therefore, let's stick to the inspiration, rather than subterfuge\footnote{\href{https://egwwritings.org/?ref=en\_Lt232-1903.41&para=10197.50}{{EGW, Lt232-1903.41; 1903}}}.


Los paréntesis de esta cita están en el manuscrito original escrito por Elena G. de White. Aquí, ella da su propia interpretación de Mateo 28:19. El Padre es un Dios personal, el Hijo es un Príncipe y Salvador personal, y el Espíritu Santo es el representante de Cristo. Esta interpretación está en armonía con la \emcap{personalidad de Dios} expresada en el primer punto de los \emcap{Principios Fundamentales}. Mateo 28:19 es un asunto de interpretación. La interpretación que hace del Trío Celestial un solo Dios no está inspirada. Esto no es lo que indica el texto. Más bien, leamos Mateo 28:19 dentro del compuesto inspirado: “\textit{Id, pues, y enseñad a todas las naciones, bautizándolos en el nombre de un Dios personal, un Príncipe y Salvador personal, y del Espíritu Santo}.” Si uno leyera el texto como tal, nadie asumiría jamás que un Dios es una unidad de tres personas. Por lo tanto, atengámonos a la inspiración, en lugar del subterfugio\footnote{\href{https://egwwritings.org/?ref=en\_Lt232-1903.41&para=10197.50}{{EGW, Lt232-1903.41; 1903}}}.


\egw{Let them be thankful to God for His manifold mercies and be kind to one another. \textbf{They have \underline{one God} and \underline{one Saviour}; and \underline{one Spirit}—\underline{the Spirit of Christ}—is to bring unity into their ranks}.}[9T 189.3; 1909][https://egwwritings.org/read?panels=p115.1057]


\egw{Sean agradecidos a Dios por sus múltiples misericordias y sean bondadosos unos con otros. \textbf{Tienen \underline{un solo Dios} y \underline{un solo Salvador}; y \underline{un solo Espíritu}—\underline{el Espíritu de Cristo}—ha de traer la unidad a sus filas}.}[9T 189.3; 1909][https://egwwritings.org/read?panels=p115.1057]


In light of the presented evidence, we see that simply numbering the Father, the Son and the Holy Spirit, does not prove the \textit{three-in-one} assumption, nor is it in conflict with the \emcap{personality of God} expressed in the \emcap{Fundamental Principles}. There is no denial of three persons of the Godhead, but only a denial of the assumption that these Three Great Worthies make one God.


A la luz de las pruebas presentadas, vemos que el simple hecho de numerar al Padre, al Hijo y al Espíritu Santo no demuestra la suposición de \textit{tres-en-uno}, ni está en conflicto con la \emcap{personalidad de Dios} expresada en los \emcap{Principios Fundamentales}. No se niega la existencia de tres personas de la Divinidad, pero sí se niega la afirmación de que estos Tres Grandes Dignatarios forman un solo Dios.


Matthew 28:19 is a valuable verse and it opens a new field of study within the Bible and the Spirit of Prophecy. In the context of the Living Temple, and referring to its sentiments, Sister White wrote that this verse should be studied most earnestly because it is not half understood.


Mateo 28:19 es un verso valioso y abre un nuevo campo de estudio dentro de la Biblia y el Espíritu de Profecía. En el contexto del Living Temple, y refiriéndose a sus sentimientos, la hermana White escribió que este versículo debe ser estudiado con la mayor seriedad porque no se entiende a medias.


\egw{Just before His ascension, Christ gave His disciples a wonderful presentation, \textbf{as recorded in the twenty-eighth chapter of Matthew}. \textbf{This chapter contains instruction} that our ministers, our \textbf{physicians}, our youth, and all our church members need to \textbf{study most \underline{earnestly}}. \textbf{Those who study this instruction as they should will \underline{not dare to advocate theories that have no foundation in the Word of God}}. My brethren and sisters, make the Scriptures, which contain the alpha and omega of knowledge, your study. \textbf{All through the Old Testament and the New, there are things \underline{that are not half understood}}. ‘And Jesus came and spake unto them, saying, All power is given unto Me in heaven and in earth. Go ye therefore, and teach all nations, \textbf{baptizing them in the name of the Father, and of the Son, and of the Holy Ghost}; teaching them to observe all things whatsoever I have commanded you; and, lo, I am with you alway, even unto the end of the world.’ [Verses 18-20.]}[Lt214-1906.10; 1906][https://egwwritings.org/read?panels=p10171.16]


\egw{Justo antes de su ascensión, Cristo dio a sus discípulos una maravillosa presentación, \textbf{como se registra en el capítulo veintiocho de Mateo}. \textbf{Este capítulo contiene instrucciones} que nuestros ministros, nuestros \textbf{médicos}, nuestros jóvenes y todos los miembros de nuestra iglesia necesitan \textbf{estudiar con la mayor \underline{seriedad}}. \textbf{Aquellos que estudian esta instrucción como deberían \underline{no se atreverán a defender teorías que no tienen fundamento en la Palabra de Dios}}. Mis hermanos y hermanas, hagan de las Escrituras, que contienen el alfa y omega del conocimiento, su estudio. \textbf{En todo el Antiguo Testamento y en el Nuevo, hay cosas \underline{que no se entienden a medias}}. ‘Y Jesús se acercó y les habló, diciendo: Todo poder me es dado en el cielo y en la tierra. Id, pues, y enseñad a todas las naciones, \textbf{bautizándolas en el nombre del Padre, y del Hijo, y del Espíritu Santo}; enseñándoles a guardar todo lo que os he mandado; y he aquí que yo estoy con vosotros todos los días, hasta el fin del mundo’. [Versículos 18-20.]}[Lt214-1906.10; 1906][https://egwwritings.org/read?panels=p10171.16]


There is a reason why Ellen White pipointed to Matthew 28:19 as a Scripture which is \egwinline{not half understood.} This statement is made in the context of 1906, where many ministers, and physicians were advocating the trinity doctrine. As we have seen, the understanding of God as a trinity, was not something Ellen White supported, and for this reason, herself, she dared not \egwinline{to advocate theories that have no foundation in the Word of God.}


Hay una razón por la que Elena G. de White señaló Mateo 28:19 como una Escritura que \egwinline{no se entiende a medias.} Esta declaración se hace en el contexto de 1906, donde muchos ministros y médicos estaban defendiendo la doctrina trinitaria. Como hemos visto, la comprensión de Dios como una trinidad no era algo que Elena G. de White apoyaba, y por esta razón, ella misma no se atrevía \egwinline{a defender teorías que no tienen fundamento en la Palabra de Dios.}


\egw{The great Teacher held in His hand \textbf{the entire map of truth. In \underline{simple} language He \underline{made plain} to His disciples} the way to heaven and \textbf{the endless subjects of divine power}. \textbf{The question of \underline{the essence of God} was a subject on which He maintained a wise reserve}, for their entanglements and specifications would bring in science which could not be dwelt upon by unsanctified minds without confusion. \textbf{In regard to God and in regard to His personality, the Lord Jesus said}, ‘Have I been so long time with you, and yet hast thou not known Me, Philip? He that hath seen Me hath seen the Father.’ [John 14:9.] \textbf{Christ was the express image of His Father’s person}.}[19LtMs, Ms 45, 1904, par. 15][https://egwwritings.org/read?panels=p14069.9381023&index=0]


\egw{El gran Maestro tenía en Su mano \textbf{el mapa completo de la verdad. En lenguaje \underline{sencillo} Él \underline{explicó claramente} a Sus discípulos} el camino al cielo y \textbf{los infinitos temas del poder divino}. \textbf{La cuestión de \underline{la esencia de Dios} fue un tema sobre el cual Él mantuvo una sabia reserva}, porque sus enredos y especificaciones traerían una ciencia que no podría ser tratada por mentes no santificadas sin confusión. \textbf{Con respecto a Dios y con respecto a Su personalidad, el Señor Jesús dijo}, ‘Tanto tiempo hace que estoy con vosotros, ¿y no me has conocido, Felipe? El que me ha visto a mí, ha visto al Padre.’ [Juan 14:9.] \textbf{Cristo era la imagen misma de la sustancia de Su Padre}.}[19LtMs, Ms 45, 1904, par. 15][https://egwwritings.org/read?panels=p14069.9381023&index=0]


\egwnogap{The open path, the safe path of walking in the way of His commandments, is a path from which there is no safe departing. \textbf{And when men follow their own human theories dressed up in soft, fascinating representations, they make a snare in which to catch souls}. \textbf{\underline{In the place of devoting your powers to theorizing}}, Christ has given you a work to do. His commission is, Go <throughout the world> and make disciples of all nations, \textbf{baptizing them in the name of the Father, and of the Son, and of the Holy Ghost}. Before the disciples shall compass the threshold, there is to be the imprint of \textbf{the sacred name, baptizing the believers in \underline{the name of the threefold powers} in the heavenly world}. The human mind is impressed in this ceremony, the beginning of the Christian life. It means very much. The work of salvation is not a small matter, but so vast that \textbf{the highest authorities} are taken hold of by the expressed faith of the human agency. \textbf{The Father, the Son, and the Holy Ghost, \underline{the eternal Godhead} is involved in the action required to make assurance to the human agent to unite \underline{all heaven} to contribute to the exercise of human faculties to reach and embrace the fulness of \underline{the threefold powers} to unite in the great work appointed, confederating the heavenly powers with the human, that men may become, through heavenly efficiency, partakers of the divine nature and workers together with Christ}.}[19LtMs, Ms 45, 1904, par. 16][https://egwwritings.org/read?panels=p14069.9381024&index=0]


\egwnogap{El camino abierto, el camino seguro de andar en Sus mandamientos, es un camino del cual no hay desviación segura. \textbf{Y cuando los hombres siguen sus propias teorías humanas vestidas con representaciones suaves y fascinantes, hacen un lazo en el cual atrapar almas}. \textbf{\underline{En lugar de dedicar vuestros poderes a teorizar}}, Cristo os ha dado una obra para hacer. Su comisión es: Id <por todo el mundo> y haced discípulos a todas las naciones, \textbf{bautizándolos en el nombre del Padre, y del Hijo, y del Espíritu Santo}. Antes de que los discípulos crucen el umbral, debe estar la impresión del \textbf{nombre sagrado, bautizando a los creyentes en \underline{el nombre de los tres poderes} en el mundo celestial}. La mente humana queda impresionada en esta ceremonia, el comienzo de la vida cristiana. Significa mucho. La obra de salvación no es un asunto pequeño, sino tan vasto que \textbf{las más altas autoridades} son tomadas por la fe expresada del agente humano. \textbf{El Padre, el Hijo y el Espíritu Santo, \underline{la eterna Deidad} está involucrada en la acción requerida para dar seguridad al agente humano de unir \underline{todo el cielo} para contribuir al ejercicio de las facultades humanas para alcanzar y abrazar la plenitud de \underline{los tres poderes} para unirse en la gran obra designada, confederando los poderes celestiales con los humanos, para que los hombres puedan llegar a ser, mediante la eficiencia celestial, participantes de la naturaleza divina y colaboradores con Cristo}.}[19LtMs, Ms 45, 1904, par. 16][https://egwwritings.org/read?panels=p14069.9381024&index=0]


This quotation is yet another often misrepresented statement. It has been often used to argue that Ellen White advocated for the Trinity by referencing the Father, the Son and the Holy Spirit by term \egwinline{eternal Godhead.} However, we must peel back the layers of its context. Ellen White was explaining the meaning behind Matthew 28:19. She stated: \egwinline{In the place of devoting your powers to theorizing,} fulfill the commission given by Christ. Theorizing about what? Theorizing about \egwinline{the essence of God.} This is another “smoking gun” for the Trinity doctrine, especially when she referenced the \emcap{personality of God} by stating: \egwinline{\textbf{In regard to God and in regard to His personality}, the Lord Jesus said…[John 14:9.] Christ was the express image of His \textbf{Father’s person}.} John 14:9 does not mean that seeing the Father in Christ implies they are one and the same person, all part of one God. Rather, it affirms that Christ is the express image of the Father’s person. The “God” she referred to was the Father. Indeed, Jesus taught the truth about who and what God is. This is what He \egwinline{made plain} \egwinline{in the simple language.} To claim that by the term \egwinline{eternal Godhead} Ellen White was endorsing the Trinity would contradict the very caution she expressed in the context of this passage.


Esta cita es otra declaración frecuentemente mal representada. A menudo se ha utilizado para argumentar que Elena G. de White defendía la Trinidad al referirse al Padre, al Hijo y al Espíritu Santo con el término \egwinline{eterna Deidad.} Sin embargo, debemos desentrañar las capas de su contexto. Elena G. de White estaba explicando el significado detrás de Mateo 28:19. Ella declaró: \egwinline{En lugar de dedicar vuestros poderes a teorizar,} cumplid la comisión dada por Cristo. ¿Teorizar sobre qué? Teorizar sobre \egwinline{la esencia de Dios.} Esta es otra “prueba contundente” para la doctrina de la Trinidad, especialmente cuando ella hizo referencia a la \emcap{personalidad de Dios} al afirmar: \egwinline{\textbf{Con respecto a Dios y con respecto a Su personalidad}, el Señor Jesús dijo...[Juan 14:9.] Cristo era la imagen misma de la \textbf{sustancia de Su Padre}.} Juan 14:9 no significa que ver al Padre en Cristo implica que son una y la misma persona, todos parte de un solo Dios. Más bien, afirma que Cristo es la imagen misma de la sustancia del Padre. El “Dios” al que ella se refería era el Padre. De hecho, Jesús enseñó la verdad sobre quién y qué es Dios. Esto es lo que Él \egwinline{explicó claramente} \egwinline{en lenguaje sencillo.} Afirmar que con el término \egwinline{eterna Deidad} Elena G. de White estaba respaldando la Trinidad contradiría la misma precaución que expresó en el contexto de este pasaje.


Unfortunately, the desperate desire of Trinitarians to paint Ellen White as a Trinitarian advocate has overshadowed the true, inspired meaning of Matthew 28:19. Her message was: \egwinline{In the place of devoting your powers to theorizing} about \egwinline{the essence of God,} Christ has given us the commission in Matthew 28:19. And she explained the meaning of Matthew 28:19. Her point was: The Father, Son, and Holy Spirit unite all of heaven’s resources with human effort so that, through divine power, people may share in God’s nature and work alongside Christ. That is the meaning of this \egwinline{threefold name.} She continued explaining:


Desafortunadamente, el deseo desesperado de los trinitarios de presentar a Elena G. de White como defensora trinitaria ha eclipsado el verdadero significado inspirado de Mateo 28:19. Su mensaje era: \egwinline{En lugar de dedicar vuestros poderes a teorizar} sobre \egwinline{la esencia de Dios,} Cristo nos ha dado la comisión en Mateo 28:19. Y ella explicó el significado de Mateo 28:19. Su punto era: El Padre, el Hijo y el Espíritu Santo unen todos los recursos del cielo con el esfuerzo humano para que, a través del poder divino, las personas puedan participar de la naturaleza de Dios y trabajar junto a Cristo. Ese es el significado de este \egwinline{nombre triple.} Ella continuó explicando:


\egw{\textbf{Man’s capabilities can multiply through the connection of human agencies with divine agencies}. \textbf{United with the heavenly powers}, the human capabilities increase according to that faith that works by love and purifies, sanctifies, and ennobles the whole man. \textbf{\underline{The heavenly powers} have \underline{pledged themselves} to minister to human agents to make the name of God and of Christ and of the Holy Spirit their living efficiency, working and energizing the sanctified man, to make this name above every other name}. \textbf{All the treasures of heaven are under obligation to do for man} infinitely more than human beings can comprehend by multiplying threefold the human with the heavenly agencies.}[19LtMs, Ms 45, 1904, par. 17][https://egwwritings.org/read?panels=p14069.9381026&index=0]


\egw{\textbf{Las capacidades del hombre pueden multiplicarse mediante la conexión de agencias humanas con agencias divinas}. \textbf{Unidas con los poderes celestiales}, las capacidades humanas aumentan de acuerdo con esa fe que obra por el amor y purifica, santifica y ennoblece a todo el hombre. \textbf{\underline{Los poderes celestiales} se han \underline{comprometido} a ministrar a los agentes humanos para hacer del nombre de Dios y de Cristo y del Espíritu Santo su eficiencia viviente, trabajando y energizando al hombre santificado, para hacer este nombre por encima de cualquier otro nombre}. \textbf{Todos los tesoros del cielo están bajo la obligación de hacer por el hombre} infinitamente más de lo que los seres humanos pueden comprender al multiplicar por tres las agencias humanas con las celestiales.}[19LtMs, Ms 45, 1904, par. 17][https://egwwritings.org/read?panels=p14069.9381026&index=0]


\egwnogap{\textbf{\underline{The three great and glorious heavenly characters} are present on the occasion of baptism. All the human capabilities are to be henceforth consecrated powers to do service for God in representing the Father, the Son, and the Holy Ghost upon whom they depend. \underline{All heaven is represented by these three} in covenant relation with the new life}. ‘If ye then be risen with Christ, seek those things that are above, where Christ sitteth at \textbf{the right hand of God}.’ [Colossians 3:1.]}[19LtMs, Ms 45, 1904, par. 18][https://egwwritings.org/read?panels=p14069.9381027&index=0]


\egwnogap{\textbf{\underline{Los tres grandes y gloriosos personajes celestiales} están presentes en la ocasión del bautismo. Todas las capacidades humanas han de ser de ahora en adelante poderes consagrados para hacer servicio para Dios representando al Padre, al Hijo y al Espíritu Santo de quienes dependen. \underline{Todo el cielo está representado por estos tres} en relación de pacto con la nueva vida}. ‘Si, pues, habéis resucitado con Cristo, buscad las cosas de arriba, donde está Cristo sentado a \textbf{la diestra de Dios}.’ [Colosenses 3:1.]}[19LtMs, Ms 45, 1904, par. 18][https://egwwritings.org/read?panels=p14069.9381027&index=0]


Many claim that Matthew 28:19 is uninspired because it was inserted by the Catholic Church\footnote{Note, 1 John 5:7 \bible{For there are three that bear record in heaven, the Father, the Word, and the Holy Ghost: and these three are one.} is an interpolation known as “\textit{Johannine Comma}”. Ellen White never used that verse. This was not the case with Matthew 28:19.}. Yet, here we have divine inspiration revealing its true meaning—the significance of baptism in the threefold name as a pledge made by these \egwinline{three great and glorious heavenly characters.} Their pledge is that \egwinline{\textbf{all the treasures of heaven are under obligation to do for man} infinitely more than human beings can comprehend by multiplying threefold the human with the heavenly agencies.}


Muchos afirman que Mateo 28:19 no es inspirado porque fue insertado por la Iglesia Católica\footnote{Nota, 1 Juan 5:7 \bible{Porque tres son los que dan testimonio en el cielo: el Padre, el Verbo y el Espíritu Santo; y estos tres son uno.} es una interpolación conocida como “\textit{Comma Joánica}”. Elena G. de White nunca usó ese versículo. Este no fue el caso con Mateo 28:19.}. Sin embargo, aquí tenemos la inspiración divina revelando su verdadero significado—la importancia del bautismo en el nombre triple como una promesa hecha por estos \egwinline{tres grandes y gloriosos personajes celestiales.} Su promesa es que \egwinline{\textbf{todos los tesoros del cielo están bajo la obligación de hacer por el hombre} infinitamente más de lo que los seres humanos pueden comprender al multiplicar por tres las agencias humanas con las celestiales.}


Ellen White frequently quoted Matthew 28:19, explaining the pledge of the Father, the Son, and the Holy Spirit. This pledge serves as a wonderful encouragement and a promise upheld by Heaven. A detailed study of this pledge is beyond the scope of this book, as it does not directly address the presence and \emcap{personality of God}. However, we encourage you to explore this topic for yourself. When you delve deeper into its meaning, you will come to understand the reality of the ministry of heavenly angels.


Elena G. de White citó frecuentemente Mateo 28:19, explicando la promesa del Padre, del Hijo y del Espíritu Santo. Esta promesa sirve como un maravilloso estímulo y una promesa sostenida por el Cielo. Un estudio detallado de esta promesa está más allá del alcance de este libro, ya que no aborda directamente la presencia y \emcap{personalidad de Dios}. Sin embargo, te animamos a explorar este tema por ti mismo. Cuando profundices en su significado, llegarás a comprender la realidad del ministerio de los ángeles celestiales.


Sister White stated that \egwinline{all heaven is represented by these three in covenant relation with the new life.} These three are the Father, the Son, and the Holy Spirit. In another instance, she said:


La hermana White declaró que \egwinline{todo el cielo está representado por estos tres en relación de pacto con la nueva vida.} Estos tres son el Padre, el Hijo y el Espíritu Santo. En otra ocasión, ella dijo:


\egw{\textbf{All heaven is interested in your home}. \textbf{God and Christ and \underline{the heavenly angels}} are intensely desirous that you shall so train your children that they will be prepared to enter the family of the redeemed.}[17LtMs, Ms 161, 1902, par. 11][https://egwwritings.org/read?panels=p14067.9877018&index=0]


\egw{\textbf{Todo el cielo está interesado en tu hogar}. \textbf{Dios y Cristo y \underline{los ángeles celestiales}} están intensamente deseosos de que entrenes a tus hijos de tal manera que estén preparados para entrar en la familia de los redimidos.}[17LtMs, Ms 161, 1902, par. 11][https://egwwritings.org/read?panels=p14067.9877018&index=0]


This is not a contradiction. All of heaven is represented by the Father, the Son, and the Holy Spirit, and in this quote, she specifically mentioned \egwinline{God and Christ and \textbf{the heavenly angels}.} There is a close connection between the workings of the Holy Spirit and the ministry of angels. The Inspiration testifies:


Esto no es una contradicción. Todo el cielo está representado por el Padre, el Hijo y el Espíritu Santo, y en esta cita, ella específicamente mencionó \egwinline{Dios y Cristo y \textbf{los ángeles celestiales}.} Existe una estrecha conexión entre la obra del Espíritu Santo y el ministerio de los ángeles. La Inspiración testifica:


\egw{A measure of \textbf{the Spirit} is given to every man to profit withal. \textbf{Through the ministry of the angels \underline{the Holy Spirit is enabled} to work upon the mind and heart of the human agent}, and draw him to Christ who has paid the ransom money for his soul, that the sinner may be rescued from the slavery of sin and Satan.}[8LtMs, Lt 71, 1893, par. 10][https://egwwritings.org/read?panels=p14058.6086016&index=0]


\egw{Una medida de \textbf{el Espíritu} es dada a cada hombre para provecho. \textbf{A través del ministerio de los ángeles \underline{el Espíritu Santo puede} obrar sobre la mente y el corazón del agente humano}, y atraerlo a Cristo quien ha pagado el precio del rescate por su alma, para que el pecador pueda ser rescatado de la esclavitud del pecado y Satanás.}[8LtMs, Lt 71, 1893, par. 10][https://egwwritings.org/read?panels=p14058.6086016&index=0]


This angelic ministry is one of the elements in the baptismal pledge of Matthew 28:19. When Ellen White said, \egwinline{\textbf{The heavenly powers} have \textbf{pledged themselves} to minister to human agents…,} she was referring to the holy angels. The connection between the Holy Spirit and the holy angels is beyond the scope of this book, but you can explore this topic further in the sequel, \textit{Rediscovering the Pillar}\footnote{Download for free: \href{https://forgottenpillar.com/book/rediscovering-the-pillar}{https://forgottenpillar.com/book/rediscovering-the-pillar}}, in the section on the Holy Spirit\footnote{Also, see the study on the angels \href{https://notefp.link/angels}{https://notefp.link/angels}}.


Este ministerio angélico es uno de los elementos en el compromiso bautismal de Mateo 28:19. Cuando Elena de White dijo, \egwinline{\textbf{Los poderes celestiales} se han \textbf{comprometido} a ministrar a los agentes humanos...}, se estaba refiriendo a los santos ángeles. La conexión entre el Espíritu Santo y los santos ángeles está más allá del alcance de este libro, pero puedes explorar este tema más a fondo en la secuela, \textit{Redescubriendo el Pilar}\footnote{Descarga gratuita: \href{https://forgottenpillar.com/book/rediscovering-the-pillar}{https://forgottenpillar.com/book/rediscovering-the-pillar}}, en la sección sobre el Espíritu Santo\footnote{También, ver el estudio sobre los ángeles \href{https://notefp.link/angels}{https://notefp.link/angels}}.


% Ellen White and Matthew 28:19

\begin{titledpoem}
    
    \stanza{
        In threefold name we’re baptized true, \\
        Not trinity as some construe. \\
        The Father, Son, and Spirit’s role, \\
        Not one God formed of triple whole.
    }

    \stanza{
        Dear Ellen’s words make clear the case, \\
        This pledge assures us heaven’s grace. \\
        The powers three have pledged their might, \\
        To guide the faithful to the light.
    }

    \stanza{
        Not proof of essence three-in-one, \\
        But heaven’s promise, freely done. \\
        A covenant of help divine, \\
        As new believers cross the line.
    }

    \stanza{
        The Father – God, in person real, \\
        The Son – our Prince, our wounds to heal, \\
        The Spirit – representative, \\
        Through Him Christ does in us now live.
    }
    
\end{titledpoem}


% Ellen White and Matthew 28:19

\begin{titledpoem}
    
    \stanza{
        In threefold name we’re baptized true, \\
        Not trinity as some construe. \\
        The Father, Son, and Spirit’s role, \\
        Not one God formed of triple whole.
    }

    \stanza{
        Dear Ellen’s words make clear the case, \\
        This pledge assures us heaven’s grace. \\
        The powers three have pledged their might, \\
        To guide the faithful to the light.
    }

    \stanza{
        Not proof of essence three-in-one, \\
        But heaven’s promise, freely done. \\
        A covenant of help divine, \\
        As new believers cross the line.
    }

    \stanza{
        The Father – God, in person real, \\
        The Son – our Prince, our wounds to heal, \\
        The Spirit – representative, \\
        Through Him Christ does in us now live.
    }
    
\end{titledpoem}
