\qrchapterstar{https://forgottenpillar.com/rsc/en-fp-introduction}{Introduction}


\qrchapterstar{https://forgottenpillar.com/rsc/en-fp-introduction}{Introducción}


\addcontentsline{toc}{chapter}{Introduction}


\addcontentsline{toc}{chapter}{Introducción}


This book has three objectives to fulfill. The first one is to revive the old pillar of our faith called, “\textit{the personality of God}”. The second objective is to re-establish trust in the writings of Ellen White, and the third is to re-establish the original Adventist identity.


Este libro tiene tres objetivos que cumplir. El primero es revivir el antiguo pilar de nuestra fe llamado, “\textit{la personalidad de Dios}”. El segundo objetivo es restablecer la confianza en los escritos de Elena G. de White, y el tercero es restablecer la identidad adventista original.


Prior to October 22, 1844, there was a great number of Adventists waiting for Christ to return on the clouds of heaven. It was a global movement of people awaiting His second coming. October 22 passed without Christ descending on the clouds and the great majority left the movement, scorning it, scorning the prophecies, the Bible, and God. Very few faithful, humble, men and women remained, who were unquestionably sure that God was leading this movement. They knew that God was shining the light of Truth and their hearts were eager to receive it. But in the eyes of the world, they were just demonstrated fanatics and dreamers. This great disappointment can be compared to the one Jesus’ disciples had after they saw their Lord being laid in the grave. They were unquestionably sure that Christ “\textit{was a prophet mighty in deed and word before God and all the people}”, but as He died on the cross, they were bitterly disappointed, because they “\textit{trusted that it had been He which should have redeemed Israel}.” Yet in their state of despair, in their state of self-disappointment, they were ready to receive the power to conquer the whole world with the Gospel. They met Christ and later received His Spirit. The same happened with the Adventist pioneers. They were a small group of people, bitterly disappointed; they sought the Lord with all their hearts and received Him in power and in Truth. The truths God revealed during this precious time of crisis constitute the foundation of Seventh-day Adventist faith. These truths were tested by all the seductive, deceptive theories of the world, by those scorning this small group, yet these grand truths prevailed. In the time of greatest need, Jesus gave His testimony by raising a little girl, the weakest of the weak, to approve all of His truths. Ellen White was not to be the source of the truths; rather, to support the brethren who were seeking the truth in the Bible. God used Ellen White to approve their studies and to point them to the Bible. The final result was the establishment of the foundation of faith based on the Bible, which standeth sure till the end of the world.


Antes del 22 de octubre de 1844, había un gran número de adventistas que esperaban el regreso de Cristo en las nubes del cielo. Era un movimiento global de personas que esperaban Su segunda venida. El 22 de octubre pasó sin que Cristo descendiera en las nubes y la gran mayoría abandonó el movimiento, despreciándolo, despreciando las profecías, la Biblia y a Dios. Quedaron muy pocos hombres y mujeres fieles y humildes, que estaban indudablemente seguros de que Dios estaba dirigiendo este movimiento. Sabían que Dios estaba haciendo brillar la luz de la Verdad y sus corazones estaban ansiosos por recibirla. Pero a los ojos del mundo, eran solo fanáticos y soñadores demostrados. Esta gran decepción puede ser comparada con la que tuvieron los discípulos de Jesús después de ver a su Señor ser puesto en la tumba. Ellos estaban indudablemente seguros de que Cristo “\textit{era un profeta poderoso en obra y palabra delante de Dios y de todo el pueblo}”, pero cuando Él murió en la cruz, quedaron amargamente decepcionados, porque “\textit{esperaban que él era el que había de redimir a Israel}.” Sin embargo, en su estado de desesperación, en su estado de autodecepción, estaban listos para recibir el poder para conquistar el mundo entero con el Evangelio. Se encontraron con Cristo y más tarde recibieron Su Espíritu. Lo mismo sucedió con los pioneros adventistas. Eran un pequeño grupo de personas, amargamente decepcionadas; buscaron al Señor con todo su corazón y lo recibieron en poder y en Verdad. Las verdades que Dios reveló durante este precioso tiempo de crisis constituyen el fundamento de la fe Adventista del Séptimo Día. Estas verdades fueron puestas a prueba por todas las teorías seductoras y engañosas del mundo, por aquellos que despreciaban a este pequeño grupo, y sin embargo estas grandes verdades prevalecieron. En el momento de mayor necesidad, Jesús dio su testimonio levantando a una niña, la más débil de las débiles, para aprobar todas sus verdades. Ellen White no debía ser la fuente de las verdades; sino apoyar a los hermanos que buscaban la verdad en la Biblia. Dios utilizó a Ellen White para aprobar sus estudios y dirigirlos a la Biblia. El resultado final fue el establecimiento del fundamento de la fe basado en la Biblia, que permanece firme hasta el fin del mundo.


Would you be surprised to know that the foundation of Seventh-day Adventist faith, which was laid at the beginning of our work, is in a fair degree different from what it is currently? Today, more than a century and a half later, we marvel in amazement over the accounts of the experiences of our pioneers; but since then, the Seventh-day Adventist Church has been subject to several new movements. Since then, the church has experienced many changes, including changes in our doctrine. Some argue that these changes are good and progressive; others argue that they are destructive and deceptive. Moving the spotlight to the original Seventh-day Adventism, it initiates the great controversy in the present days. We have personally been in this controversy for over 6 years now and we have seen that it will only get bigger and stronger, often with results of a sad record. Many people from both sides of this controversy are rejecting the Spirit of Prophecy in one way or another. Some have left the Seventh-day Adventist Church altogether. The Adventist identity is either lost or drastically changed from the initial one.


¿Le sorprendería saber que el fundamento de la fe Adventista del Séptimo Día, que se estableció al principio de nuestra obra, es en gran medida diferente de lo que es actualmente? Hoy, más de un siglo y medio después, nos maravillamos con los relatos de las experiencias de nuestros pioneros; pero desde entonces, la Iglesia Adventista del Séptimo Día ha sido objeto de varios movimientos nuevos. Desde entonces, la iglesia ha experimentado muchos cambios, incluyendo cambios en nuestra doctrina. Algunos sostienen que estos cambios son buenos y progresistas; otros sostienen que son destructivos y engañosos. Al trasladar el foco de atención al Adventismo del Séptimo Día original, se inicia la gran controversia en los días actuales. Nosotros hemos estado en esta controversia personalmente por más de 6 años y hemos visto que sólo se hará más grande y más fuerte, a menudo con resultados de un registro triste. Muchas personas de ambos lados de esta controversia están rechazando el Espíritu de Profecía de una manera u otra. Algunos han abandonado la Iglesia Adventista del Séptimo Día por completo. La identidad adventista se pierde o es cambiada drásticamente con respecto a la inicial.


We are currently witnessing the shaking of the Seventh-day Adventist church, seeing her tossed through one wave of crisis after another. Many are losing their faith and their identity as Seventh-day Adventists. But we believe in a solution that the Lord, in His mercy, has already provided. The solution can be found in the history of the Seventh-day Adventist movement.


Actualmente estamos siendo testigos de la sacudida de la Iglesia Adventista del Séptimo Día, viéndola ser estremecida por una ola de crisis tras otra. Muchos están perdiendo su fe y su identidad como adventistas del séptimo día. Pero creemos en una solución que el Señor, en su misericordia, ya ha proporcionado. La solución puede encontrarse en la historia del movimiento adventista del séptimo día.


\egw{\textbf{In reviewing our past history}, having traveled over every step of advance to our present standing, I can say, Praise God! As I see what the Lord has wrought, I am filled with astonishment, and with confidence in Christ as leader. \textbf{We have nothing to fear for the future, \underline{except as we shall forget} the way the Lord has led us, and \underline{His teaching} in our past history}.}[LS 196.2; 1915][https://egwwritings.org/read?panels=p41.1083]


\egw{\textbf{Al repasar nuestra historia pasada}, habiendo recorrido cada paso de avance hasta nuestra condición presente, puedo decir: ¡Alabado sea Dios! Al ver lo que el Señor ha hecho, me lleno de admiración y de confianza en Cristo como líder. \textbf{No tenemos nada que temer del futuro, \underline{excepto que olvidemos} la manera en que el Señor nos ha conducido, y \underline{Su enseñanza} en nuestra historia pasada}.}[LS 196.2; 1915][https://egwwritings.org/read?panels=p41.1083]


We shall not fear! This is a great reassurance and promise—though conditional. We must \textit{remember} how the Lord has led us, and \textit{His teaching in our past history}. When we look at what the Lord has taught us in our past history, we are surprised to see how things have changed. The change has taken several years and many crises. To judge these changes in doctrine, whether good and progressive or bad and destructive, evaluation should be based on past experiences, as the Lord clearly led His church.


¡No temeremos! Esta es una gran aseguranza y promesa—aunque condicional. Debemos \textit{recordar} cómo nos ha guiado el Señor, y \textit{Su enseñanza en nuestra historia pasada}. Cuando miramos lo que el Señor nos ha enseñado en nuestra historia pasada, nos sorprende ver cómo han cambiado las cosas. El cambio ha tomado varios años y muchas crisis. Para juzgar estos cambios en la doctrina, ya sean buenos y progresivos o malos y destructivos, la evaluación debe basarse en las experiencias pasadas, tal como el Señor dirigió claramente a su iglesia.


At this time, we put forth a bold claim—one that is supposed to make you hold this book until the end of its cover. Encouraged by the counsels of Ellen White to review our past history, we have concluded that we have forgotten one crucial pillar of our faith, which was the main subject of Kellogg’s controversy—the \emcap{personality of God}. One of the biggest crises that the SDA Church ever had in the time of the living prophet was the Kellogg crisis. It is out of this crisis that many other crises, today, find their roots. In this light, the subject of the \emcap{personality of God} is pivotal in our present time.


En este momento, planteamos una afirmación audaz—una que debe hacer que usted sostenga este libro hasta el final de su cubierta. Alentados por los consejos de Ellen White para revisar nuestra historia pasada, hemos llegado a la conclusión de que hemos olvidado un pilar crucial de nuestra fe, que fue el tema principal de la controversia de Kellogg—la \emcap{personalidad de Dios}. Una de las mayores crisis que tuvo la Iglesia ASD en la época del profeta viviente fue la crisis de Kellogg. Es a partir de esta crisis que muchas otras crisis, hoy en día, encuentran sus raíces. A la luz de esto, el tema de la \emcap{personalidad de Dios} es de importancia central en nuestro tiempo actual.


Sister White wrote to Kellogg that the \emcap{personality of God} and the \emcap{personality of Christ} was a pillar of our faith in the same rank as is the sanctuary message:


La hermana White escribió a Kellogg que la \emcap{personalidad de Dios} y la \emcap{personalidad de Cristo} era un pilar de nuestra fe del mismo rango que el mensaje del santuario:


\egw{Those who seek to remove \textbf{the old landmarks} are not holding fast; they \textbf{are \underline{not remembering} how they have received and heard}. Those who try to \textbf{\underline{bring in} theories that would remove \underline{the pillars of our faith} concerning the sanctuary, \underline{or concerning the personality of God or of Christ}, are working as blind men}. They are seeking to bring in uncertainties and to set the people of God adrift, without an anchor.}[Ms62-1905.14][https://egwwritings.org/read?panels=p14070.10026020]


\egw{Los que procuran mover \textbf{los antiguos pilares} no están afirmando las cosas; \textbf{\underline{no recuerdan} lo que han recibido y oído}. Los que tratan de \textbf{\underline{introducir} teorías que mueven \underline{los pilares de nuestra fe} con respecto al santuario, \underline{la personalidad de Dios o de Cristo}, están trabajando como ciegos}. Procuran introducir incertidumbre y dejar al pueblo de Dios sin ancla, a la deriva.}[Ms62-1905.14][https://egwwritings.org/read?panels=p14070.10026020]


The \emcap{personality of God} receives very little attention today as a subject, yet it is one of the crucial elements in dealing with other doctrines pertaining to Adventism, such as the doctrine of Trinity, the Sanctuary service, 1844 and any other doctrine dealing with the Heavenly reality.


La \emcap{personalidad de Dios} recibe muy poca atención hoy en día como tema, sin embargo es uno de los elementos cruciales al tratar otras doctrinas pertenecientes al adventismo, como la doctrina trinitaria, el servicio del santuario, 1844 y cualquier otra doctrina que trate de la realidad celestial.


The \emcap{personality of God} was a pillar of our faith. Today, it is almost forgotten. We propose a reasonable explanation for that. It is due to the evolution of the English language. What is meant by the term, “\textit{the personality of God}”? The general understanding of the English word ‘\textit{personality}’ has changed over the years. Today, ‘\textit{personality}’ is generally viewed as, “\textit{the characteristic set of behaviors, cognitions, and emotional patterns}”\footnote{Wikipedia Contributors. “\textit{Personality.}” Wikipedia, Wikimedia Foundation, 19 Apr. 2019, \href{https://en.wikipedia.org/wiki/Personality}{en.wikipedia.org/wiki/Personality}.}, but in the nineteenth, and beginning of the twentieth century, it meant “\textit{the quality or state of \textbf{being a person}}”\footnote{\href{https://www.merriam-webster.com/dictionary/personality}{Merriam-Webster Dictionary}, - ‘personality’} \footnote{\href{https://babel.hathitrust.org/cgi/pt?id=mdp.39015050663213&view=1up&seq=780}{Hunter Robert, The American encyclopaedic dictionary}, ‘\textit{personality}’ - “\textit{the quality or state of being personal}”; Mentioned dictionary was in possession of Ellen White (see \href{https://repo.adventistdigitallibrary.org/PDFs/adl-22/adl-22251050.pdf?_ga=2.116010630.1065317374.1621993520-1506151612.1617862694&fbclid=IwAR3vwmp8jxtnpPEKv0KD9mCv8dJpmRGoyIXW0CkbQAjbU0h6YaBGqhgBzbk}{EGW Private and Office Libraries})}. We read this definition as the primary definition of the word ‘\textit{personality}’ from the Merriam-Webster Dictionary\footnote{\href{https://www.merriam-webster.com/dictionary/personality\#word-history}{Merriam-Webster Dictionary} marks that the first record of the definition “the quality or state of being a person” is recorded in the 15th century.}. When Sister White and our pioneers wrote about the \emcap{personality of God}, they referred to \textit{the quality or state of God being a person}. In other words, they dealt with the question, “\textit{is God a person}”, and, “\textit{what is it that makes Him a person}” or “\textit{what is the quality or state of God being a person}”? Try to remember the last time you had a Bible study on the question, “\textit{is God a person?}” Think about how you can prove to yourself, from the Bible, that God is a person. Think about it. It is an important question. Upon this question hangs your view of God and your relationship toward Him. The \emcap{personality of God} is fundamental to true spirituality; true spirituality is based on your personal relationship with God. No real relationship of any kind can be formed with anyone unless he/she is a person. Maybe you have never asked yourself this question because you never felt a need to question if God is a person, and what is it (the quality or state) that makes Him a person. Or, maybe you were refraining from this question because you felt it might be a mystery that God did not intend to reveal. Maybe it will surprise you to know that God has given a definite and affirmative answer in His Word to the question “\textit{what is the quality or state of God being a person}”. What was even more surprising for us, was that the Adventist pioneers, including Sister White, had definite light regarding this topic, and they held it as a pillar of our faith, as part of the foundation of Seventh-day Adventist faith. When the \emcap{personality of God} is rightly understood in light of our historical past, old quotations shine in a new light and new shreds of evidence are presented, which will deepen the understanding of our past history and the present crisis.


La \emcap{personalidad de Dios} era un pilar de nuestra fe. Hoy en día, está casi olvidada. Proponemos una explicación razonable para ello. Se debe a la evolución de la lengua inglesa. ¿Qué significa el término “la personalidad de Dios”? La comprensión general de la palabra inglesa ‘\textit{personalidad}’ ha cambiado a lo largo de los años. Hoy en día, ‘\textit{personalidad}’ se considera generalmente como “\textit{el conjunto característico de comportamientos, cogniciones y pautas emocionales}”\footnote{Wikipedia Contributors. “\textit{Personality.}” Wikipedia, Wikimedia Foundation, 19 Apr. 2019, \href{https://en.wikipedia.org/wiki/Personality}{en.wikipedia.org/wiki/Personality}.}, pero en el siglo XIX y principios del XX, significaba “\textit{la cualidad o estado de \textbf{ser una persona}}”\footnote{\href{https://www.merriam-webster.com/dictionary/personality}{Merriam-Webster Dictionary}, - ‘personality’} \footnote{\href{https://babel.hathitrust.org/cgi/pt?id=mdp.39015050663213&view=1up&seq=780}{Hunter Robert, The American encyclopaedic dictionary}, ‘\textit{personality}’ - “\textit{the quality or state of being personal}”; Mencionado diccionario estaba en posesión de Ellen White (ver \href{https://repo.adventistdigitallibrary.org/PDFs/adl-22/adl-22251050.pdf?_ga=2.116010630.1065317374.1621993520-1506151612.1617862694&fbclid=IwAR3vwmp8jxtnpPEKv0KD9mCv8dJpmRGoyIXW0CkbQAjbU0h6YaBGqhgBzbk}{EGW Private and Office Libraries})}. Leemos esta definición como la definición principal de la palabra ‘\textit{personalidad}’ del Diccionario Merriam-Webster\footnote{\href{https://www.merriam-webster.com/dictionary/personality\#word-history}{Merriam-Webster Dictionary} señala que el primer registro de la definición “la cualidad o estado de ser una persona” se registró en el siglo XV.}. Cuando la hermana White y nuestros pioneros escribieron sobre la \emcap{personalidad de Dios}, se refirieron a \textit{la cualidad o estado de Dios siendo una persona}. En otras palabras, ellos trataron con la pregunta, “\textit{¿es Dios una persona?}”, y, “\textit{¿qué es lo que lo hace una persona?}” o “\textit{¿cuál es la cualidad o estado de Dios siendo una persona?}”. Trata de recordar la última vez que tuviste un estudio bíblico sobre la pregunta, “\textit{¿es Dios una persona?}” Piensa en cómo puedes demostrarte a ti mismo, a partir de la Biblia, que Dios es una persona. Piensa en ello. Es una pregunta importante. De esta pregunta depende tu visión de Dios y tu relación con Él. La \emcap{personalidad de Dios} es fundamental para la verdadera espiritualidad; la verdadera espiritualidad se basa en tu relación personal con Dios. No se puede establecer ninguna relación real de ningún tipo con nadie a menos que sea una persona. Tal vez nunca te has hecho esta pregunta porque nunca has sentido la necesidad de cuestionar si Dios es una persona, y qué es (la cualidad o estado) lo que lo hace una persona. O tal vez te abstuviste de esta pregunta porque sentiste que podía ser un misterio que Dios no tenía intención de revelar. Tal vez te sorprenda saber que Dios ha dado una respuesta definitiva y afirmativa en Su Palabra a la pregunta “\textit{cuál es la cualidad o estado de Dios siendo una persona}”. Lo que fue aún más sorprendente para nosotros, fue que los pioneros adventistas, incluyendo a la hermana White, tuvieron una luz definida con respecto a este tema, y lo sostuvieron como un pilar de nuestra fe, como parte del fundamento de la fe adventista del séptimo día. Cuando la \emcap{personalidad de Dios} se entiende correctamente a la luz de nuestro pasado histórico, las viejas citas brillan con una nueva luz y se presentan nuevas evidencias, que profundizarán la comprensión de nuestra historia pasada y la crisis actual.


The root problem of the Kellogg crisis was about the \emcap{personality of God}. It is certainly important to evaluate Kellogg's crisis over the \emcap{personality of God} using the meaning intended at that time; that is, using the definition of ‘\textit{personality},’ as the quality or state of God being a person. With this definition in mind, the Kellogg crisis comes into a new light and new relevant evidence is brought forth for us today. In light of this evidence, we see how God has led us in the past; thus, we should not fear for the future. Knowing and understanding this, as well as its importance, helps us to not be shaken by any wave of deception in present controversies. When Sister White was drawing Kellogg’s attention to the importance of this subject, she was drawing our attention also, as it is everything to us as a people.


El problema de fondo de la crisis de Kellogg era sobre la \emcap{personalidad de Dios}. Es ciertamente importante evaluar la crisis de Kellogg sobre la \emcap{personalidad de Dios} utilizando el significado que se le daba en aquella época; es decir, utilizando la definición de ‘\textit{personalidad}’, como la cualidad o estado de Dios siendo una persona. Teniendo en cuenta esta definición, la crisis de Kellogg adquiere una nueva luz y se aportan nuevas pruebas relevantes para nosotros hoy. A la luz de esta evidencia, vemos cómo Dios nos ha guiado en el pasado; por lo tanto, no debemos temer por el futuro. Conociendo y comprendiendo esto, así como su importancia, nos ayuda a no ser sacudidos por ninguna ola de engaño en las controversias actuales. Cuando la hermana White llamaba la atención de Kellogg sobre la importancia de este tema, estaba llamando también nuestra atención, ya que lo es todo para nosotros como pueblo.


[Writing to Kellogg] \egw{You are not definitely clear on \textbf{the personality of God}, which is \textbf{\underline{everything} to us as a people}.}[Lt300-1903.7][https://egwwritings.org/read?panels=p14068.7705013]


[Escribiendo a Kellogg] \egw{Usted no está definitivamente claro sobre \textbf{la personalidad de Dios}, que lo es \textbf{\underline{todo} para nosotros como pueblo}.}[Lt300-1903.7][https://egwwritings.org/read?panels=p14068.7705013]


These studies on the \emcap{personality of God} will prompt a lot of new and hard questions. We do not promise to answer all of them, and perhaps you won’t be satisfied with the answers provided, but we pray, hope and believe that this book will fulfill the three objectives proposed in the beginning of this introduction. Through the reviving of the doctrine on the \emcap{personality of God}, we believe that your confidence in the Spirit of Prophecy will strengthen, and that you’ll find yourself rooted deeper in the Adventist message—where we find our identity as people—making you a more faithful Seventh-day Adventist. Most importantly, we want you to become more aware of God as your personal God. This will surely strengthen and deepen your relationship with Him.


Estos estudios sobre la \emcap{personalidad de Dios} darán lugar a muchas preguntas nuevas y difíciles. No prometemos responder a todas ellas, y tal vez no queden satisfechos con las respuestas ofrecidas, pero oramos, esperamos y creemos que este libro cumplirá los tres objetivos propuestos al principio de esta introducción. A través del reavivamiento de la doctrina sobre la \emcap{personalidad de Dios}, creemos que su confianza en el Espíritu de Profecía se fortalecerá, y que se encontrará más arraigado en el mensaje adventista—donde encontramos nuestra identidad como pueblo—haciéndote un adventista del séptimo día más fiel. Y lo que es más importante, queremos que seas más consciente de que Dios es tu Dios personal. Esto seguramente fortalecerá y profundizará tu relación con Él.


We find answers to the issue on the \emcap{personality of God} in examining the Kellogg crisis, where Sister White gave the most definite light on the \emcap{personality of God} and on the foundation of Seventh-day Adventist faith. The following is the complete tenth chapter from the book, \textit{Testimonies for the Church Containing Letters to Physicians and Ministers Instruction to Seventh-Day Adventists}. This chapter, \textit{The Foundation of our Faith}, contains deep insight into the history of Kellogg’s crisis. It gives a historical overview of the truths God gave as the foundation of our faith and in these truths we find our identity as Seventh-day Adventists— keeping the commandments of God and having the faith of Jesus.


Encontramos respuestas a la cuestión sobre la \emcap{personalidad de Dios} al examinar la crisis de Kellogg, donde la hermana White dio la luz más definitiva sobre la \emcap{personalidad de Dios} y sobre el fundamento de la fe adventista del séptimo día. El siguiente es el décimo capítulo completo del libro, \textit{Testimonies for the Church Containing Letters to Physicians and Ministers Instruction to Seventh-Day Adventists}. Este capítulo, \textit{El fundamento de nuestra fe}, contiene una visión profunda de la historia de la crisis de Kellogg. Da una visión histórica de las verdades que Dios dio como fundamento de nuestra fe y en estas verdades encontramos nuestra identidad como adventistas del séptimo día: guardar los mandamientos de Dios y tener la fe de Jesús.
