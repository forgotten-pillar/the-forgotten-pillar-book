\qrchapter{https://forgottenpillar.com/rsc/en-fp-chapter25}{Setting up the wrong Fundamental Principles}


\qrchapter{https://forgottenpillar.com/rsc/en-fp-chapter25}{Establecimiento de los principios fundamentales erróneos}


You might ask yourself: how could it be possible that we, as a church, have gone astray from the light God gave us in the beginning? The answer to this question is the same answer to the question why the Jews went astray from the light God gave them concerning His Son. Please, take a look at the driving force behind the church in Apostolic times and our time.


Te preguntarás: ¿cómo es posible que nosotros, como iglesia, nos hayamos desviado de la luz que Dios nos dio al principio? La respuesta a esta pregunta es la misma respuesta a la pregunta de por qué los judíos se desviaron de la luz que Dios les dio con respecto a su Hijo. Por favor, eche un vistazo a la fuerza impulsora de la iglesia en los tiempos apostólicos y en nuestro tiempo.


\egw{‘The angel of the Lord by night opened the prison doors, and brought them forth, and said, Go, stand and speak in the temple to the people all the words of this life.’ [Acts 5:19, 20.] We see here that the men in authority are not always obeyed, even though they may profess to be teachers of Bible doctrines. \textbf{There are many today who feel indignant and aggrieved that any voice should be raised presenting ideas that differ from their own in regard to points of religious belief}. \textbf{Have they not long advocated their ideas as truth?} So the priests and rabbis reasoned in apostolic days. What mean these men who are unlearned, some of them mere fishermen, who are presenting ideas contrary to the doctrines which the learned priests and rulers are teaching the people? \textbf{They have no right to meddle with the fundamental principles of our faith}.}[Lt38-1896.23; 1896][https://egwwritings.org/read?panels=p5631.29]


\egw{‘El ángel del Señor, de noche, abrió las puertas de la cárcel, y los sacó, y dijo: Id, poneos en pie y hablad al pueblo todas las palabras de esta vida.’ [Hechos 5:19, 20.] Vemos aquí que no siempre se obedece a los hombres con autoridad, aunque profesen ser maestros de las doctrinas bíblicas. \textbf{Hay muchos hoy en día que se sienten indignados y agraviados de que se alce cualquier voz presentando ideas que difieren de las suyas en cuanto a puntos de creencia religiosa}. \textbf{¿Acaso no han defendido durante mucho tiempo sus ideas como verdad?} Así razonaban los sacerdotes y rabinos en los días apostólicos. ¿Qué significan estos hombres indoctos, algunos de ellos simples pescadores, que presentan ideas contrarias a las doctrinas que los sacerdotes y gobernantes doctos enseñan al pueblo? \textbf{No tienen derecho a entrometerse en los principios fundamentales de nuestra fe}.}[Lt38-1896.23; 1896][https://egwwritings.org/read?panels=p5631.29]


\egwnogap{“\textbf{But we see that the God of heaven sometimes commissions men to \underline{teach that which is regarded as contrary to the established doctrines}. Because those who were once the depositaries of truth \underline{became unfaithful to their sacred trust}, the Lord chose others who would receive the bright beams of the Sun of Righteousness, and would advocate truths that were not in accordance with the ideas of the religious leaders. And then these leaders, in the blindness of their minds, give full sway to what is supposed to be righteous indignation against the ones who have set aside cherished fables. They act like men that have lost their reason. They do not consider the possibility that they themselves have not rightly understood the Word. They will not open their eyes to discern the fact that they have misinterpreted and misapplied the Scriptures, and have built up false theories, \underline{calling them fundamental doctrines of the faith}}.“}[Lt38-1896.24; 1896][https://egwwritings.org/read?panels=p5631.30]


\egwnogap{“\textbf{Pero vemos que el Dios del cielo a veces encarga a los hombres que \underline{enseñen lo que se considera contrario a las doctrinas establecidas}. Debido a que los que una vez fueron depositarios de la verdad \underline{se volvieron infieles a su sagrada confianza}, el Señor eligió a otros que recibirían los brillantes rayos del Sol de Justicia, y defenderían verdades que no estaban de acuerdo con las ideas de los líderes religiosos. Y entonces estos líderes, en la ceguera de sus mentes, dan rienda suelta a lo que se supone que es una justa indignación contra los que han dejado de lado fábulas apreciadas. Actúan como hombres que han perdido la razón. No consideran la posibilidad de que ellos mismos no hayan entendido bien la Palabra. No abren sus ojos para discernir el hecho de que han interpretado y aplicado mal las Escrituras, y han construido teorías falsas, \underline{llamándolas doctrinas fundamentales de la fe}}.”}[Lt38-1896.24; 1896][https://egwwritings.org/read?panels=p5631.30]


\egwnogap{\textbf{But the Holy Spirit will from time to time reveal the truth through its own chosen agencies; and no man, not even a priest or ruler, has a right to say, You shall not give publicity to your opinions, because I do not believe them. That wonderful ‘I’ may attempt to put down the Holy Spirit’s teaching. Men may, for a time, attempt to smother it and kill it; but that will not make error truth or truth error. The inventive minds of men have advanced speculative opinions in various lines, and when the Holy Spirit lets light shine into human minds, it does not respect every point of man’s application of the word. God impressed his servants to speak the truth irrespective of what men had taken for granted as truth}.}[Lt38-1896.25; 1896][https://egwwritings.org/read?panels=p5631.31]


\egwnogap{\textbf{Pero el Espíritu Santo revelará de vez en cuando la verdad por medio de sus propios organismos elegidos; y ningún hombre, ni siquiera un sacerdote o gobernante, tiene derecho a decir: No darás publicidad a tus opiniones, porque yo no las creo. Ese maravilloso ‘yo’ puede intentar sofocar la enseñanza del Espíritu Santo. Los hombres pueden, por un tiempo, intentar sofocarla y matarla; pero eso no hará que el error sea verdad ni la verdad error. Las mentes inventivas de los hombres han avanzado opiniones especulativas en varias líneas, y cuando el Espíritu Santo deja que la luz brille en las mentes humanas, no respeta cada punto de la aplicación de la palabra por parte del hombre. Dios impresionó a sus siervos para que dijeran la verdad sin tener en cuenta lo que los hombres habían dado por sentado como verdad}.}[Lt38-1896.25; 1896][https://egwwritings.org/read?panels=p5631.31]


\egwnogap{\textbf{\underline{Even Seventh-day Adventists are in danger of closing their eyes to truth as it is in Jesus}, because it contradicts something which they have taken for granted as truth, but which the Holy Spirit teaches is not truth. Let all be very modest, and seek most earnestly to put self out of the question, and to exalt Jesus.} \textbf{In most of the religious controversies, the foundation of the trouble is that self is striving for the supremacy}. About what? About matters which are not vital points at all, and which are regarded as such only because men have given importance to them. See Matthew 12:31-37; Mark 14:56; Luke 5:21; Matthew 9:3.}[Lt38-1896.26; 1896][https://egwwritings.org/read?panels=p5631.32]


\egwnogap{\textbf{\underline{Incluso los adventistas del séptimo día corren el peligro de cerrar los ojos a la verdad tal como está en Jesús}, porque contradice algo que han dado por sentado como verdad, pero que el Espíritu Santo enseña que no es verdad. Que todos sean muy modestos, y busquen con gran empeño poner el yo fuera de la cuestión, y exaltar a Jesús.} \textbf{En la mayoría de las controversias religiosas, el fundamento del problema es que el yo está luchando por la supremacía}. ¿Sobre qué? Sobre asuntos que no son puntos vitales en absoluto, y que se consideran como tales sólo porque los hombres les han dado importancia. Véase Mateo 12:31-37; Marcos 14:56; Lucas 5:21; Mateo 9:3.}[Lt38-1896.26; 1896][https://egwwritings.org/read?panels=p5631.32]


The proud state of the heart resists the will of God and is the driving force behind apostasy; the humble heart is obedient to the will of God and is the driving force behind true reformation. The following quotations express future, concrete prophecies where the fanciful ideas of God will be brought in and \egwinline{many things of like character will in the future arise}[Ms137-1903.10; 1903][https://egwwritings.org/read?panels=p9939.17]. These ideas are of like character to the ideas contained in the Living Temple. They will do away with the \emcap{personality of God}. Ellen White gives warning after warning to adhere to the \emcap{Fundamental Principles}, and to be aware of the leaders who will tear down the old foundation.


El estado orgulloso del corazón se resiste a la voluntad de Dios y es el motor de la apostasía; el corazón humilde es obediente a la voluntad de Dios y es el motor de la verdadera reforma. Las siguientes citas expresan profecías futuras y concretas en las que se introducirán las ideas fantasiosas de Dios y \egwinline{muchas cosas de carácter semejante surgirán en el futuro}[Ms137-1903.10; 1903][https://egwwritings.org/read?panels=p9939.17]. Estas ideas son de carácter similar a las ideas contenidas en el Templo Viviente. Acabarán con la \emcap{personalidad de Dios}. Elena G. de White da advertencia tras advertencia para adherirse a los \emcap{Principios Fundamentales}, y para estar al tanto de los líderes que derribarán los viejos cimientos.


\egw{In view of these Scriptures, who will dare to interpret God and place in the minds of others the sentiments regarding Him that are contained in Living Temple? \textbf{These theories are the theories of the great deceiver, and in the lives of \underline{those who receive them there will be sad chapters}}. \textbf{This is Satan’s device \underline{to unsettle the foundation of our faith}, to shake our confidence in the Lord’s guidance and in the experience that He has given us. \underline{Many things of like character will in the future arise}}. I entreat our medical missionary workers to be afraid to trust the suppositions and devising of any human being who entertains the thought that \textbf{the path over which the people of God have been led for the last fifty years is a wrong path}. \textbf{\underline{Beware of those who}, not having had any decided experience in the leading of the Lord’s Spirit, \underline{would suppose that this leading is all a fallacy}; that we have not the truth}; that we are not the people of the Lord, gathered by Him from all countries and nations. \textbf{\underline{Beware of those who would tear down the foundation, upon which we have been building for the last fifty years, to establish a new doctrine}}. \textbf{I know that these new theories are from the enemy}.}[Ms137-1903.10; 1903][https://egwwritings.org/read?panels=p9939.17]


\egw{En vista de estas Escrituras, ¿quién se atreverá a interpretar a Dios y a poner en la mente de los demás los sentimientos respecto a Él que están contenidos en el Templo Viviente? \textbf{Estas teorías son las del gran engañador, y en la vida de \underline{quienes las reciban habrá tristes capítulos}}. \textbf{Esta es la estratagema de Satanás \underline{para desestabilizar el fundamento de nuestra fe}, para hacer tambalear nuestra confianza en la guía del Señor y en la experiencia que nos ha dado. \underline{En el futuro surgirán muchas cosas de este tipo}}. Ruego a nuestros obreros médicos misioneros que tengan miedo de confiar en las suposiciones e invenciones de cualquier ser humano que albergue el pensamiento de que \textbf{el camino por el que el pueblo de Dios ha sido guiado durante los últimos cincuenta años es un camino equivocado}. \textbf{\underline{Cuidado con aquellos que}, sin haber tenido ninguna experiencia decidida en la conducción del Espíritu del Señor, \underline{supondrían que esta conducción es toda una falacia}; que no tenemos la verdad; que no somos el pueblo del Señor, reunido por Él de todos los países y naciones. \textbf{\underline{Cuidado con los que quieren derribar los cimientos, sobre los que hemos estado construyendo durante los últimos cincuenta años, para establecer una nueva doctrina}}. \textbf{Sé que estas nuevas teorías provienen del enemigo}.}[Ms137-1903.10; 1903][https://egwwritings.org/read?panels=p9939.17]


\egwnogap{\textbf{Let those who would \underline{bring in} fanciful ideas of God awake to a sense of their danger. This is too solemn a subject to be trifled with}.}[Ms137-1903.11; 1903][https://egwwritings.org/read?panels=p9939.18]


\egwnogap{\textbf{Que aquellos que quieren \underline{introducir} ideas fantasiosas de Dios despierten al sentido de su peligro. Este es un tema demasiado solemne para que se juegue con él}.}[Ms137-1903.11; 1903][https://egwwritings.org/read?panels=p9939.18]


\egwnogap{The root of idolatry is an evil heart of unbelief in departing from the living God. It is because men have not faith in the presence and power of God \textbf{that they have been putting their trust in their own wisdom}. They have been devising and planning to exalt themselves and find salvation in their own works. \textbf{\underline{A deceptive influence from satanic agencies is coming in}, because leaders whom the Lord has warned and entreated and counseled are choosing their own wisdom in the place of the wisdom of God}. To such ones the warning comes, ‘Talk no more exceedingly proudly; let not arrogancy come out of your mouth; for the Lord is a God of knowledge, and by Him actions are weighed.’}[Ms137-1903.12; 1903][https://egwwritings.org/read?panels=p9939.19]


\egwnogap{La raíz de la idolatría es un corazón malvado de incredulidad al apartarse del Dios vivo. Es porque los hombres no tienen fe en la presencia y el poder de Dios \textbf{que han estado poniendo su confianza en su propia sabiduría}. Han estado ideando y planeando exaltarse a sí mismos y encontrar la salvación en sus propias obras. \textbf{\underline{Una influencia engañosa de las agencias satánicas está entrando}, porque los líderes a quienes el Señor ha advertido y suplicado y aconsejado están eligiendo su propia sabiduría en lugar de la sabiduría de Dios}. A los tales viene la advertencia: ‘No habléis más con soberbia; no salga de vuestra boca la arrogancia; porque el Señor es un Dios de conocimiento, y por él se pesan las acciones’.}[Ms137-1903.12; 1903][https://egwwritings.org/read?panels=p9939.19]


The difference between the old \emcap{Fundamental Principles} and the new Fundamental Beliefs is in our \egwinline{ideas of God.} The Trinitarian idea of God was not part of the foundation of our faith, which Sister White defended. How did this change take place? It was done through the leaders who chose \egwinline{their own wisdom in the place of the wisdom of God.} We should \egwinline{Beware of those who would tear down the foundation, upon which we have been building for the last fifty years, to establish a new doctrine.} In this observation, we recognize that this new Trinitarian idea of God was \egwinline{a deceptive influence from satanic agency} that came into our ranks.


La diferencia entre los antiguos \emcap{Principios Fundamentales} y las nuevas Creencias Fundamentales está en nuestras \egwinline{ideas de Dios.} La idea trinitaria de Dios no formaba parte del fundamento de nuestra fe, que la hermana White defendió. ¿Cómo se produjo este cambio? Se hizo a través de los líderes que eligieron \egwinline{su propia sabiduría en lugar de la sabiduría de Dios.} Debemos \egwinline{Cuidarnos de aquellos que quieren derribar los cimientos, sobre los que hemos estado construyendo durante los últimos cincuenta años, para establecer una nueva doctrina.} En esta observación, reconocemos que esta nueva idea trinitaria de Dios fue \egwinline{una influencia engañosa de las agencias satánicas} que entró en nuestras filas.


% Setting up the wrong Fundamental Principles

\begin{titledpoem}
    
    \stanza{
        In the silence of our own church’s walls, \\
        We’ve strayed from the light that first on us falls. \\
        An ancient query resounds with age: \\
        Why from sacred paths do we disengage?
    }

    \stanza{
        Ellen spoke of priests and rabbis of old, \\
        Men of cloth, with hearts yet so cold. \\
        They spurned the truth when it brightly shone, \\
        Proud hearts rejecting what was divinely shown.
    }

    \stanza{
        Behold the fishermen with unlearned tongues, \\
        Who dared to challenge the learned ones. \\
        "The Holy Spirit guides," they boldly claimed, \\
        While leaders scorned and fiercely blamed.
    }

    \stanza{
        Do not the humble hearts perceive \\
        The quiet whispers they should believe? \\
        While proud hearts preach their own command, \\
        True faith slips like fine sand from hand.
    }

    \stanza{
        Even now, echoes of the past, \\
        Warn us of shadows that leaders cast. \\
        God’s own voice, through years does span, \\
        Yet resisted by the inventions of man.
    }

    \stanza{
        To the foundations we must hold fast, \\
        Not swayed by shadows that leaders cast. \\
        For in the Scripture’s unerring light, \\
        Lies the path that is just and right.
    }

    \stanza{
        Beware of new doctrines, thinly veiled, \\
        On old, firm rocks they have not sailed. \\
        Let not man’s wisdom lead astray, \\
        But God’s own Spirit show the way.
    }

    \stanza{
        In faith, let us each day commence, \\
        With Bible as shield, no human pretense. \\
        For truth in Christ alone is found, \\
        And on this rock, our faith is sound.
    }
    
\end{titledpoem}


% Setting up the wrong Fundamental Principles

\begin{titledpoem}
    
    \stanza{
        In the silence of our own church’s walls, \\
        We’ve strayed from the light that first on us falls. \\
        An ancient query resounds with age: \\
        Why from sacred paths do we disengage?
    }

    \stanza{
        Ellen spoke of priests and rabbis of old, \\
        Men of cloth, with hearts yet so cold. \\
        They spurned the truth when it brightly shone, \\
        Proud hearts rejecting what was divinely shown.
    }

    \stanza{
        Behold the fishermen with unlearned tongues, \\
        Who dared to challenge the learned ones. \\
        "The Holy Spirit guides," they boldly claimed, \\
        While leaders scorned and fiercely blamed.
    }

    \stanza{
        Do not the humble hearts perceive \\
        The quiet whispers they should believe? \\
        While proud hearts preach their own command, \\
        True faith slips like fine sand from hand.
    }

    \stanza{
        Even now, echoes of the past, \\
        Warn us of shadows that leaders cast. \\
        God’s own voice, through years does span, \\
        Yet resisted by the inventions of man.
    }

    \stanza{
        To the foundations we must hold fast, \\
        Not swayed by shadows that leaders cast. \\
        For in the Scripture’s unerring light, \\
        Lies the path that is just and right.
    }

    \stanza{
        Beware of new doctrines, thinly veiled, \\
        On old, firm rocks they have not sailed. \\
        Let not man’s wisdom lead astray, \\
        But God’s own Spirit show the way.
    }

    \stanza{
        In faith, let us each day commence, \\
        With Bible as shield, no human pretense. \\
        For truth in Christ alone is found, \\
        And on this rock, our faith is sound.
    }
    
\end{titledpoem}
