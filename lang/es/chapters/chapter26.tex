\qrchapter{https://forgottenpillar.com/rsc/en-fp-chapter26}{The steps to Omega}


\qrchapter{https://forgottenpillar.com/rsc/es-fp-chapter26}{Los pasos hacia Omega}


In our study so far, we have seen evidence that Kellogg’s controversy was connected to the Trinity doctrine and the \emcap{personality of God} expressed in the first point of the \emcap{Fundamental Principles}. Unfortunately, today we do not stand on that foundation regarding the \emcap{personality of God}; we have built another foundation that has changed the truth on the \emcap{personality of God} to a mysterious Triune God. Sister White was clearly against this reorganization and she prophesied that in the closing of His work, God will rehearse the history of the Advent movement and re-establish every pillar of our faith that was held in the beginning.


En nuestro estudio hasta ahora, hemos visto evidencia de que la controversia de Kellogg estaba conectada con la doctrina trinitaria y la \emcap{personalidad de Dios} expresada en el primer punto de los \emcap{Principios Fundamentales}. Desgraciadamente, hoy no nos apoyamos en ese fundamento respecto a la \emcap{personalidad de Dios}; hemos construido otro fundamento que ha cambiado la verdad sobre la \emcap{personalidad de Dios} a un misterioso Dios Trino. La hermana White estaba claramente en contra de esta reorganización y profetizó que en el cierre de su obra, Dios ensayará la historia del movimiento adventista y restablecerá cada pilar de nuestra fe que se sostenía en el principio.


\egw{\textbf{\underline{The Lord has declared that the history of the past shall be rehearsed as we enter upon the closing work}. \underline{Every truth} that He has given for these last days is to be proclaimed to the world. \underline{Every pillar} that He has established \underline{is to be strengthened}. We cannot now step off the foundation that God has established. We cannot now enter into any new organization; for this would mean apostasy from the truth}.}[Ms129-1905.6; 1905][https://egwwritings.org/read?panels=p9797.13]


\egw{\textbf{\underline{El Señor ha declarado que la historia del pasado será ensayada cuando entremos en la obra de clausura}. \underline{Cada verdad} que Él ha dado para estos últimos días debe ser proclamada al mundo. \underline{Cada pilar} que Él ha establecido \underline{debe ser fortalecido}. Ahora no podemos salirnos del fundamento que Dios ha establecido. No podemos ahora entrar en ninguna nueva organización; porque esto significaría apostasía de la verdad}.}[Ms129-1905.6; 1905][https://egwwritings.org/read?panels=p9797.13]


Comparing the \emcap{Fundamental Principles} with the current Fundamental Beliefs of Seventh-day Adventists, we see that we have entered into a new organization. God’s warning, given through Sister White, to re-establish all pillars of our faith in these last days, is becoming imperative. As we traced the Trinity doctrine from Kellogg's controversy, we came across Ellen White’s warnings against alpha and omega apostasy, which will enter into our church.


Comparando los \emcap{Principios Fundamentales} con las actuales Creencias Fundamentales de los Adventistas del Séptimo Día, vemos que hemos entrado en una nueva organización. La advertencia de Dios, dada a través de la hermana White, de restablecer todos los pilares de nuestra fe en estos últimos días, se hace imperativa. Al trazar la doctrina trinitaria desde la controversia de Kellogg, nos encontramos con las advertencias de Ellen White contra la apostasía alfa y omega, que entrará en nuestra iglesia.


\egw{\textbf{‘Living Temple’ contains the alpha of these theories. I knew that \underline{the omega would follow in a little while}; and I trembled for our people}. I knew that \textbf{I must warn our brethren and sisters not to enter into controversy \underline{over the presence and personality of God}. The statements made in ‘Living Temple’ \underline{in regard to this point are incorrect}. }The scripture used to substantiate the doctrine there set forth, is scripture misapplied.}[SpTB02 53.2; 1904][https://egwwritings.org/read?panels=p417.271]


\egw{\textbf{‘The Living Temple’ contiene la alfa de estas teorías. Sabía que \underline{la omega seguiría en poco tiempo}; y temblé por nuestro pueblo}. Sabía que \textbf{debía advertir a nuestros hermanos y hermanas que no entraran en controversia \underline{sobre la presencia y personalidad de Dios}. Las declaraciones hechas en ‘Living Temple’ \underline{con respecto a este punto son incorrectas}. }La escritura utilizada para fundamentar la doctrina allí expuesta, es una escritura mal aplicada.}[SpTB02 53.2; 1904][https://egwwritings.org/read?panels=p417.271]


In the context of Seventh-day Adventist reorganization, we identify several steps that were necessary to accomplish this reorganization and are necessary to uphold it.


En el contexto de la reorganización adventista del séptimo día, identificamos varios pasos que fueron necesarios para lograr esta reorganización y que son necesarios para mantenerla.


\subsection*{Step 1: Deny the Fundamental Principles to be the foundation of our faith and the official, and accurate, representation of Seventh-day Adventist beliefs}


\subsection*{Paso 1: Negar que los Principios Fundamentales sean el fundamento de nuestra fe y la representación oficial, y exacta, de las creencias adventistas del séptimo día}


The first step necessary is to hide the original foundation of our faith by unlinking it with the \emcap{Fundamental Principles}.


El primer paso necesario es ocultar el fundamento original de nuestra fe desvinculándolo de los \emcap{Principios Fundamentales}.


\egw{\textbf{As a people, we are to \underline{stand firm on the platform of eternal truth} that has withstood test and trial. We are to \underline{hold to the sure pillars of our faith}. \underline{The principles of truth} that God has revealed to us \underline{are our only true foundation}. They have made us what we are. The lapse of time has not lessened their value. \underline{It is the constant effort of the enemy to remove these truths from their setting}, and to put in their place \underline{spurious theories}. He \underline{will bring in} everything that he possibly can to carry out his deceptive designs.}}[SpTB02 51.2; 1904][https://egwwritings.org/read?panels=p417.261]


\egw{\textbf{Como pueblo, debemos \underline{mantenernos firmes en la plataforma de la verdad eterna} que ha resistido la prueba y el juicio. Debemos \underline{aferrarnos a los pilares seguros de nuestra fe}. \underline{Los principios de la verdad} que Dios nos ha revelado \underline{son nuestro único y verdadero fundamento}. Ellos nos han hecho lo que somos. El paso del tiempo no ha disminuido su valor. \underline{Es el esfuerzo constante del enemigo para quitar estas verdades de su lugar}, y poner en su lugar \underline{teorías espurias}. Él \underline{traerá} todo lo que pueda para llevar a cabo sus engañosos designios.}}[SpTB02 51.2; 1904][https://egwwritings.org/read?panels=p417.261]


\egw{\textbf{Messages of every order and kind have been urged upon Seventh-day Adventists, to take the place of the truth which, \underline{point by point}, has been sought out by prayerful study, and testified to by the miracle-working power of the Lord}. \textbf{But \underline{the way-marks} \underline{which have made us what we are}, \underline{are to be preserved}, and they \underline{will be preserved}, as God has signified through His word and the testimony of His Spirit}. \textbf{He calls upon us to \underline{hold firmly}, with the grip of faith, to \underline{the fundamental principles} that are \underline{based upon unquestionable authority}}.}[SpTB02 59.1; 1904][https://egwwritings.org/read?panels=p417.299]


\egw{\textbf{Se ha instado a los Adventistas del Séptimo Día a que reciban mensajes de toda clase y orden, para que ocupen el lugar de la verdad que, \underline{punto por punto}, ha sido buscada por medio del estudio en oración, y testificada por el poder milagroso del Señor}. \textbf{Pero \underline{los hitos} \underline{que nos han convertido en lo que somos}, \underline{deben ser preservados}, y \underline{serán preservados}, tal como Dios lo ha indicado mediante su palabra y el testimonio de su Espíritu}. \textbf{Él nos pide que \underline{nos aferremos firmemente}, con la garra de la fe, a \underline{los principios fundamentales} que se \underline{basan en una autoridad incuestionable}}.}[SpTB02 59.1; 1904][https://egwwritings.org/read?panels=p417.299]


The \emcap{Fundamental Principles} were the truths God revealed to the pioneers after the passing of time in 1844. We have seen the testimonies of our pioneers, including Ellen White, regarding the first point of the \emcap{Fundamental Principles}. All of them were in harmony regarding these particular points of our faith. In 1863, Seventh-day Adventists organized themselves into a church, as an organized body. Since then, many were misrepresenting the position of the Seventh-day Adventist Church and the pioneers found it necessary to meet inquiries, \others{and sometimes to correct false statements circulated against} the church’s beliefs and practices. Consequently, in 1872, the pioneers issued the document called “\textit{A Declaration of the Fundamental Principles, Taught and Practiced by the Seventh-Day Adventists}”\footnote{“A Declaration of the Fundamental Principles, Taught and Practiced by the Seventh-Day Adventists (1872) : MVT : Free Download, Borrow, and Streaming : Internet Archive.” Internet Archive, 2025, \href{https://archive.org/details/ADeclarationOfTheFundamentalPrinciplesTaughtAndPracticedByThe}{archive.org/details/ADeclarationOfTheFundamentalPrinciplesTaughtAndPracticedByThe}. Accessed 3 Feb. 2025.}. This declaration presented the public with \others{a brief statement of what is, and has been, with great unanimity, held by}[The preface of the Fundamental Principles in 1872.] Seventh-day Adventists.


Los \emcap{Principios Fundamentales} fueron las verdades que Dios reveló a los pioneros después del paso del tiempo en 1844. Hemos visto los testimonios de nuestros pioneros, incluyendo a Ellen White, con respecto al primer punto de los \emcap{Principios Fundamentales}. Todos ellos estaban en armonía sobre estos puntos particulares de nuestra fe. En 1863, los adventistas del séptimo día se organizaron en una iglesia, como un cuerpo organizado. Desde entonces, muchos tergiversaron la posición de la Iglesia Adventista del Séptimo Día y los pioneros consideraron necesario responder a las preguntas, \others{y a veces corregir las declaraciones falsas que circulaban contra} las creencias y prácticas de la iglesia. En consecuencia, en 1872, los pioneros publicaron el documento llamado “\textit{Declaración de los Principios Fundamentales, Enseñados y Practicados por los Adventistas del Séptimo Día}”\footnote{“A Declaration of the Fundamental Principles, Taught and Practiced by the Seventh-Day Adventists (1872) : MVT : Free Download, Borrow, and Streaming : Internet Archive.” Internet Archive, 2025, \href{https://archive.org/details/ADeclarationOfTheFundamentalPrinciplesTaughtAndPracticedByThe}{archive.org/details/ADeclarationOfTheFundamentalPrinciplesTaughtAndPracticedByThe}. Accessed 3 Feb. 2025.}. Esta declaración presentaba al público \others{una breve declaración de lo que es, y ha sido, con gran unanimidad, sostenido por}[The preface of the Fundamental Principles in 1872.] los adventistas del séptimo día.


In the chapter “\hyperref[chap:authority]{The Authority of the Fundamental Principles}”, we discussed how pro-Trinitarian scholars have been compromising the authority of the \emcap{Fundamental Principles}, denying their true value in our Adventist history.


En el capítulo “\hyperref[chap:authority]{La Autoridad de los Principios Fundamentales}”, discutimos cómo los eruditos pro-trinitarios han estado comprometiendo la autoridad de los \emcap{Principios Fundamentales}, negando su verdadero valor en nuestra historia adventista.


Pro-trinitarian scholars argue that this declaration was not what it claims to be—a declaration of the \emcap{fundamental principles}, taught and practiced by the Seventh-day Adventists. This declaration was a summary of the principal features of Adventist’s faith, and no point is really as problematic or objectionable as the first point, dealing with the \emcap{personality of God} and where His presence is. But the evidence in favor of the \emcap{Fundamental Principles}, especially to the first point, is overwhelming.


Los eruditos pro-trinitarios argumentan que esta declaración no era lo que dice ser—una declaración de los \emcap{principios fundamentales}, enseñados y practicados por los Adventistas del Séptimo Día. Esta declaración era un resumen de las principales características de la fe adventista, y ningún punto es realmente tan problemático u objetable como el primero, que trata de la \emcap{personalidad de Dios} y de dónde está su presencia. Pero la evidencia a favor de los \emcap{Principios Fundamentales}, especialmente el primer punto, es abrumadora.


All of these claims are easily refuted by the fact that the \emcap{Fundamental Principles} have been regularly issued and reprinted over the course of the entire life of Sister White, until 1914. If they were mere private opinions of a few individuals, as claimed by scholars\footnote{Ministry Magazine “Our Declaration of Fundamental Beliefs”: January 1958, Roy Anderson, J. Arthur Buckwalter, Louise Kleuser, Earl Cleveland and Walter Schubert}, would they have been consistently reprinted over the course of 42 years\footnote{For a detailed list of publications throughout these years, see the Appendix.}, publicly claiming to represent the synopsis of Seventh-day Adventist faith? If they had been issued only once, we could deem it a conspiracy by some individuals to purposely misrepresent Seventh-day Adventist faith. On the contrary, the \emcap{Fundamental Principles} were regularly reprinted, and they truly represented the official Seventh-day Adventist faith and practice.


Todas estas afirmaciones son fácilmente refutadas por el hecho de que los \emcap{Principios Fundamentales} han sido publicados y reimpresos regularmente durante toda la vida de la hermana White, hasta 1914. Si fueran simplemente las opiniones privadas de unos pocos individuos, como afirman los académicos\footnote{Ministry Magazine “Our Declaration of Fundamental Beliefs”: January 1958, Roy Anderson, J. Arthur Buckwalter, Louise Kleuser, Earl Cleveland and Walter Schubert}, ¿se habrían reimpreso constantemente durante 42 años\footnote{Para una lista detallada de publicaciones a lo largo de estos años, véase el Apéndice.}, afirmando públicamente que representan la sinopsis de la fe adventista del séptimo día? Si sólo se hubieran publicado una vez, podríamos considerar que se trata de una conspiración de algunos individuos para tergiversar a propósito la fe adventista del séptimo día. Por el contrario, los \emcap{Principios Fundamentales} se reimprimieron con regularidad, y representaron realmente la fe y la práctica oficial adventista del séptimo día.


Another argument is that Sister White approved the \emcap{Fundamental Principles} in her writings by explicitly referring to them, and also by teaching the same truths taught in the \emcap{Fundamental Principles}. The works of our pioneers are also in harmony with the statements in this Declaration of the \emcap{Fundamental Principles}. Considering all of these facts, it is inevitable that this declaration was truthful in its claims. This document was indeed a declaration of the \emcap{fundamental principles}, taught and practiced by the Seventh-day Adventist Church, representing a public \others{synopsis of our faith}, \others{a brief statement of what is, and has been, with great unanimity, held by} Seventh-day Adventists.\footnote{The preface of the Fundamental Principles in 1872.} As such, it accurately represents the Seventh-day Adventist belief and practice, and represents the foundation of Seventh-day Adventist faith in the time of Ellen White.


Otro argumento es que la hermana White respaldó los \emcap{Principios Fundamentales} en sus escritos al referirse explícitamente a ellos, y también al enseñar las mismas verdades que se enseñan en los \emcap{Principios Fundamentales}. Las obras de nuestros pioneros también están en armonía con las afirmaciones de esta Declaración de los \emcap{Principios Fundamentales}. En vista de todos estos hechos, es inevitable que esta declaración sea cierta en sus afirmaciones. Este documento fue, en efecto, una declaración de los \emcap{principios fundamentales}, enseñados y practicados por la Iglesia Adventista del Séptimo Día, representando una \others{sinopsis pública de nuestra fe}, \others{una breve declaración de lo que es, y ha sido, con gran unanimidad, sostenido por} los adventistas del séptimo día.\footnote{The preface of the Fundamental Principles in 1872.} Como tal, representa con exactitud la creencia y la práctica adventistas del séptimo día, y representa el fundamento de la fe adventista del séptimo día en la época de Ellen White.


Today, in defense of the Trinity doctrine, Adventist historians boldly claim that when our pioneers were studying Adventist truths such as the sanctuary, investigative judgment, the Sabbath and other doctrines, they \others{did not study the subject of the doctrine of God}. These Adventist historians falsely claim that the doctrine of God \others{was not the question that they dealt at that time}[Denis Kaiser. “From Antitrinitarianism to Trinitarianism: The Adventist story” and Panelist. The God We Worship: A Godhead Symposium. Central California Conference, Dinuba, CA. March 23-24, 2018.]. Following this false claim, they present historical data on how Adventist doctrine gradually moved toward Trinitarian understanding. The truth is, there are some instances early on\footnote{The earliest mention of the Trinity doctrine, in a positive sense, was when M.C. Wilcox reprinted a non-Adventist article by Samuel Spear in Signs of the Times, December 7th, 1891 and December 14th, 1891} when the Trinity doctrine is mentioned in a positive light in our literature. But when you consider the fact that the Adventist church did have a positive position on the subject of the doctrine of God, as it was expressed in the \emcap{Fundamental Principles}, these instances cannot be interpreted as progressiveness in understanding, but rather an intrusion of the Trinity doctrine into the Seventh-day Adventist Church.


Hoy, en defensa de la doctrina trinitaria, los historiadores adventistas afirman audazmente que cuando nuestros pioneros estudiaban las verdades adventistas como el santuario, el juicio investigador, el sábado y otras doctrinas, ellos \others{no estudiaron el tema de la doctrina de Dios}. Estos historiadores adventistas afirman falsamente que la doctrina de Dios \others{no era la cuestión que trataban en ese momento}[Denis Kaiser. “From Antitrinitarianism to Trinitarianism: The Adventist story” and Panelist. The God We Worship: A Godhead Symposium. Central California Conference, Dinuba, CA. March 23-24, 2018.]. Siguiendo esta falsa afirmación, presentan datos históricos sobre cómo la doctrina adventista se fue moviendo gradualmente hacia la comprensión trinitaria. La verdad es que hay algunos casos tempranos\footnote{La mención más temprana de la doctrina de la trinidad, en un sentido positivo, fue cuando M.C. Wilcox reimprimió un artículo no adventista de Samuel Spear en Signs of the Times, el 7 de diciembre de 1891 y el 14 de diciembre de 1891} en los que la doctrina de la trinidad se menciona de forma positiva en nuestra literatura. Pero cuando se considera el hecho de que la iglesia adventista tenía una posición positiva sobre el tema de la doctrina de Dios, tal como se expresaba en los \emcap{Principios Fundamentales}, estos casos no pueden interpretarse como una progresión en la comprensión, sino más bien como una intrusión de la doctrina de la trinidad en la Iglesia Adventista del Séptimo Día.


It is easy to refute the claim that Adventist pioneers did not understand the doctrine of God. If they did not understand it, they would have failed to proclaim the first angel’s message. We discussed this point in detail in the chapter “\hyperref[chap:remembering-the-beginning]{Remembering the beginning}”. The Seventh-day Adventist movement was not a failure, but a God-led, prophetic movement.


Es fácil refutar la afirmación de que los pioneros adventistas no entendían la doctrina de Dios. Si no la entendieran, no habrían proclamado el mensaje del primer ángel. Discutimos este punto en detalle en el capítulo “\hyperref[chap:remembering-the-beginning]{Recordando el principio}”. El movimiento adventista del séptimo día no fue un fracaso, sino un movimiento profético dirigido por Dios.


\subsection*{Step 2: Ignore the warnings of building a new foundation}


\subsection*{Paso 2: Ignorar las advertencias de construir un nuevo fundamento}


When the \emcap{Fundamental Principles} are removed from the equation, many of Ellen White’s warnings fail to shine in their true light and their true meaning does not resonate with the reader.


Cuando se eliminan los \emcap{Principios Fundamentales} de la ecuación, muchas de las advertencias de Ellen White no brillan en su verdadera luz y su verdadero significado no resuena con el lector.


We have cited many quotations where Sister White warned the church not to step off the \emcap{Fundamental Principles}. We dealt with them in the chapter “\hyperref[chap:apostasy]{The great apostasy is soon to be realized}”, but we will mention one of the most prominent quotations again.


Hemos citado muchas citas en las que la hermana White advirtió a la iglesia que no se saliera de los \emcap{Principios Fundamentales}. Nos ocupamos de ellas en el capítulo “\hyperref[chap:apostasy]{La gran apostasía pronto se hará realidad}”, pero volveremos a mencionar una de las citas más destacadas.


\egw{\textbf{The enemy of souls has sought to bring in the supposition that a great reformation was to take place among Seventh-day Adventists, and that this reformation would \underline{consist in giving up the doctrines which stand as the pillars of our faith} and engaging in a process of reorganization}. Were this reformation to take place, what would result? \textbf{The principles of truth that God in His wisdom has given to the remnant church would be discarded. Our religion would be changed. \underline{The fundamental principles that have sustained the work for the last fifty years would be accounted as error}}. \textbf{A new organization would be established. Books of a new order would be written. A system of intellectual philosophy would be introduced}...}[Lt242-1903.13; 1903][https://egwwritings.org/read?panels=p7767.20]


\egw{\textbf{El enemigo de las almas ha tratado de introducir la suposición de que una gran reforma iba a tener lugar entre los Adventistas del Séptimo Día, y que esta reforma consistiría en \underline{abandonar las doctrinas que se mantienen como pilares de nuestra fe} y emprender un proceso de reorganización}. Si esta reforma tuviera lugar, ¿qué resultaría? \textbf{Los principios de la verdad que Dios, en su sabiduría, ha dado a la iglesia remanente serían descartados. Nuestra religión cambiaría. \underline{Los principios fundamentales que han sostenido la obra durante los últimos cincuenta años serían considerados como un error}}. \textbf{Se establecería una nueva organización. Se escribirían libros de un nuevo orden. Se introduciría un sistema de filosofía intelectual}...}[Lt242-1903.13; 1903][https://egwwritings.org/read?panels=p7767.20]


\egwnogap{Who has authority to begin such a movement? \textbf{We have our Bibles. We have our experience, attested to by the miraculous working of the Holy Spirit}. \textbf{We have a truth that admits of no compromise.} \textbf{\underline{Shall we not repudiate everything that is not in harmony with this truth}?}}[Lt242-1903.14; 1903][https://egwwritings.org/read?panels=p7767.21]


\egwnogap{¿Quién tiene autoridad para iniciar tal movimiento? \textbf{Tenemos nuestras Biblias. Tenemos nuestra experiencia, atestiguada por la obra milagrosa del Espíritu Santo}. \textbf{Tenemos una verdad que no admite concesiones.} \textbf{\underline{¿No deberíamos repudiar todo lo que no esté en armonía con esta verdad}?}}[Lt242-1903.14; 1903][https://egwwritings.org/read?panels=p7767.21]


\subsection*{Step 3: Deny that the personality of God was the pillar of our faith and a part of the foundation of our faith}


\subsection*{Paso 3: Negar que la personalidad de Dios era el pilar de nuestra fe y una parte del fundamento de nuestra fe}


There is one Ellen White statement that apparently supports the claim that the \emcap{personality of God} was not a pillar of our faith. Another expression for “\textit{pillars of our faith}” is “\textit{landmarks}”. In the following quotations, Sister White lists several landmarks: the cleansing of the sanctuary, the three angels’ messages, the temple of God, the Sabbath and the non-immortality of the wicked.


Hay una declaración de Ellen White que aparentemente apoya la afirmación de que la \emcap{personalidad de Dios} no era un pilar de nuestra fe. Otra expresión para “\textit{pilares de nuestra fe}” es “\textit{hitos}”. En las siguientes citas, la hermana White enumera varios hitos: la purificación del santuario, los mensajes de los tres ángeles, el templo de Dios, el sábado y la no inmortalidad de los impíos.


\egw{The passing of the time in 1844 was a period of great events, opening to our astonished eyes \textbf{the cleansing of the sanctuary transpiring in heaven}, and having decided relation to God’s people upon the earth, [also] \textbf{the first and second angels’ messages and the third}, unfurling the banner on which was inscribed, ‘The commandments of God and the faith of Jesus.’ [Revelation 14:12.] One of the landmarks under this message was \textbf{the temple of God}, seen by His truth-loving people in heaven, and the ark containing the law of God. The light of \textbf{the Sabbath} of the fourth commandment flashed its strong rays in the pathway of the transgressors of God’s law. The \textbf{non-immortality of the wicked} is an old landmark. \textbf{I can call to mind nothing more that can come under the head of the old landmarks}. All this cry about changing the old landmarks is all imaginary.}[Ms13-1889.9; 1889][https://egwwritings.org/read?panels=p4179.14]


\egw{El paso del tiempo en 1844 fue un período de grandes acontecimientos, abriendo a nuestros asombrados ojos \textbf{la purificación del santuario que ocurría en el cielo}, y que tenía una relación decidida con el pueblo de Dios en la tierra, [también] \textbf{el primer y segundo mensajes de los ángeles y el tercero}, desplegando el estandarte en el que estaba inscrito, ‘Los mandamientos de Dios y la fe de Jesús.’ [Apocalipsis 14:12.] Uno de los hitos de este mensaje fue \textbf{el templo de Dios}, visto por su pueblo amante de la verdad en el cielo, y el arca que contenía la ley de Dios. La luz de \textbf{el sábado} del cuarto mandamiento lanzó sus fuertes rayos en el camino de los transgresores de la ley de Dios. La \textbf{no inmortalidad de los malvados} es un viejo hito. \textbf{No puedo recordar nada más que pueda caer bajo el título de los antiguos hitos}. Todo este clamor sobre el cambio de los antiguos hitos es imaginario.}[Ms13-1889.9; 1889][https://egwwritings.org/read?panels=p4179.14]


At the end of this list of landmarks, or pillars of our faith, she states that she can recall nothing else that would fall under the category of the old landmarks. For many, this quotation serves as proof that the \emcap{personality of God} was neither an old landmark nor a pillar. It is true that in this quotation, Sister White did not explicitly mention the \emcap{personality of God}, but it would be implicitly included under the first angel’s message, as well as being an underlying doctrine of the Sanctuary message. Furthermore, there are other quotations from Sister White that explicitly include the \emcap{personality of God} as an old landmark or pillar of our faith.


Al final de esta lista de hitos, o pilares de nuestra fe, ella dice que no puede recordar nada más que pueda estar bajo el título de los antiguos hitos. Para muchos, esta cita es una prueba de que la \emcap{personalidad de Dios} no era un antiguo hito ni un pilar. Es cierto que en esta cita la hermana White no mencionó explícitamente la \emcap{personalidad de Dios}, pero estaría implícitamente incluida bajo el mensaje del primer ángel, así como siendo una doctrina subyacente del mensaje del Santuario. Además, hay otras citas de la hermana White que incluyen explícitamente la \emcap{personalidad de Dios} como un antiguo hito o pilar de nuestra fe.


\egw{Those who seek to remove the \textbf{old landmarks} are not holding fast; they \textbf{are not remembering how they have received and heard}. Those who try to \textbf{\underline{bring in} theories that would remove \underline{the pillars of our faith}} \textbf{concerning the sanctuary}, \textbf{\underline{or concerning the personality of God or of Christ}, are working as blind men}. They are seeking to bring in uncertainties and to set the people of God \textbf{adrift}, without an anchor.}[Ms62-1905.14; 1905][https://egwwritings.org/read?panels=p10026.20]


\egw{Los que tratan de eliminar los \textbf{antiguos hitos} no se mantienen firmes; \textbf{no recuerdan cómo han recibido y oído}. Los que tratan de \textbf{\underline{introducir} teorías que eliminen \underline{los pilares de nuestra fe}} \textbf{en relación con el santuario}, \textbf{\underline{o en relación con la personalidad de Dios o de Cristo}, están obrando como ciegos}. Intentan introducir incertidumbres y dejar al pueblo de Dios \textbf{a la deriva}, sin ancla.}[Ms62-1905.14; 1905][https://egwwritings.org/read?panels=p10026.20]


Sister White also teaches us that the pillars of our faith constitute the foundation of our faith.


La hermana White también nos enseña que los pilares de nuestra fe constituyen el fundamento de nuestra fe.


\egw{\textbf{What influence is it that would lead men at this stage of our history to work in an underhanded, powerful way \underline{to tear down the foundation of our faith},—the foundation that was laid at the beginning of our work by prayerful study of the word and by revelation? Upon \underline{this foundation} we have been building for \underline{the past fifty years}. Do you wonder that when I see the beginning of a work that would \underline{remove some of the pillars of our faith}, I have something to say? I must obey the command, ‘Meet it!’}}[SpTB02 58.1; 1904][https://egwwritings.org/read?panels=p417.295]


\egw{\textbf{¿Qué influencia es la que llevaría a los hombres, en esta etapa de nuestra historia, a trabajar de manera solapada y poderosa \underline{para derribar el fundamento de nuestra fe},—el fundamento que fue colocado al comienzo de nuestra obra por el estudio en oración de la palabra y por la revelación? Sobre \underline{este fundamento} hemos estado construyendo durante \underline{los últimos cincuenta años}. ¿Se extrañan de que cuando veo el comienzo de una obra que \underline{removería algunos de los pilares de nuestra fe}, tenga algo que decir? Debo obedecer el mandato, ‘¡Enfréntalo!’}}[SpTB02 58.1; 1904][https://egwwritings.org/read?panels=p417.295]


Removing some of the pillars of our faith means tearing down the foundation of our faith. Elsewhere, Sister White said that tearing down or undermining the foundation of our faith is done by indoctrination of the sentiments regarding the \emcap{personality of God}.


Quitar algunos de los pilares de nuestra fe significa derribar el fundamento de nuestra fe. En otra parte, la hermana White dijo que derribar o socavar el fundamento de nuestra fe se hace mediante el adoctrinamiento de los sentimientos respecto a la \emcap{personalidad de Dios}.


\egw{The college was taken out of Battle Creek; yet students are still called there, and there they \textbf{become indoctrinated with the very sentiments regarding the personality of God and Christ that would undermine the foundation of our faith}.}[Lt72-1906.5; 1906][https://egwwritings.org/read?panels=p10013.11]


\egw{El colegio fue sacado de Battle Creek; sin embargo, los estudiantes siguen siendo llamados allí, y allí \textbf{se adoctrinan con los mismos sentimientos respecto a la personalidad de Dios y de Cristo que socavarían el fundamento de nuestra fe}.}[Lt72-1906.5; 1906][https://egwwritings.org/read?panels=p10013.11]


In light of these quotations we see positive testimony that the \emcap{personality of God} was part of the foundation of our faith. Furthermore, in chapter 10 of the special testimonies, entitled “\textit{The foundation of our faith}”, Sister White mentioned “\textit{Fundamental Principles}” using the synonyms “\textit{pillars of our faith}”, “\textit{waymarks}”, and “\textit{landmarks}”, when addressing the foundation of our faith.


A la luz de estas citas vemos un testimonio positivo de que la personalidad de Dios era parte del fundamento de nuestra fe. Además, en el capítulo 10 de los testimonios especiales, titulado “\textit{El fundamento de nuestra fe}”, la hermana White mencionó “\textit{Principios Fundamentales}” utilizando los sinónimos “\textit{pilares de nuestra fe}”, “\textit{hitos}”, y “\textit{marcas del camino}”, al referirse al fundamento de nuestra fe.


\subsection*{Step 4: Alter the meaning of the term “the personality of God”}


\subsection*{Paso 4: Alterar el significado del término “la personalidad de Dios”}


The term ‘\textit{personality}’ has two different applications and the most common definition in everyday use is in the area of psychology. ‘\textit{Personality}’ is defined as “\textit{the characteristic sets of behaviors, cognitions, and emotional patterns that evolve from biological and environmental factors}”\footnote{Wikipedia Contributors. “Personality.” Wikipedia, Wikimedia Foundation, 19 Apr. 2019, \href{https://en.wikipedia.org/wiki/Personality}{en.wikipedia.org/wiki/Personality}.}. It is of utmost importance to recognize that when we are dealing with the pillar of our faith—“\textit{the personality of God}”—we are not in the realms of psychology. The accurate application of the word ‘\textit{personality}’ within the doctrine on the \emcap{personality of God} is found in the Merriam-Webster Dictionary: “\textit{the quality or state of being a person}”\footnote{\href{https://www.merriam-webster.com/dictionary/personality}{Merriam-Webster Dictionary} - ‘\textit{personality}’}. According to the Merriam-Webster Dictionary, this definition has been in use since the 15th century\footnote{See “\href{https://www.merriam-webster.com/dictionary/personality\#word-history}{First known use}” of the word ‘personality’ in Merriam Webster Dictionary}. In the 1828 edition of the Merriam Webster Dictionary we read definition of the word ‘\textit{personality}’ as: “\textit{that which constitutes an individual a distinct person}”\footnote{\href{https://archive.org/details/americandictiona02websrich/page/272/mode/2up}{Merriam-Webster Dictionary, 1828 edition} - ‘\textit{personality}’} \footnote{\href{https://archive.org/details/websterscomplete00webs/page/974/mode/2up}{The 1886 edition of Merriam-Webster Dictionary} defines the word ‘\textit{personality}’ as: “\textit{that which constitutes, or pertains to, a person}”}. Both of the definitions are found in The Encyclopaedic Dictionary, by Hunter Robert\footnote{\href{https://babel.hathitrust.org/cgi/pt?id=mdp.39015050663213&view=1up&seq=780}{Hunter Robert, The Encyclopaedic Dictionary} - ‘\textit{personality}’}—dictionary owned by Ellen White. The use of these definitions can be seen from the articles written on the \emcap{personality of God}.


El término ‘\textit{personalidad}’ tiene dos aplicaciones diferentes y la definición más común en el uso cotidiano es en el área de la psicología. ‘\textit{Personalidad}’ se define como “\textit{los conjuntos característicos de comportamientos, cogniciones y patrones emocionales que evolucionan a partir de factores biológicos y ambientales}”\footnote{Wikipedia Contributors. “Personality.” Wikipedia, Wikimedia Foundation, 19 Apr. 2019, \href{https://en.wikipedia.org/wiki/Personality}{en.wikipedia.org/wiki/Personality}.}. Es de suma importancia reconocer que cuando tratamos con el pilar de nuestra fe—“\textit{la personalidad de Dios}”—no estamos en el ámbito de la psicología. La aplicación exacta de la palabra ‘\textit{personalidad}’ dentro de la doctrina sobre la \emcap{personalidad de Dios} se encuentra en el Diccionario Merriam-Webster: “\textit{la cualidad o estado de ser una persona}”\footnote{\href{https://www.merriam-webster.com/dictionary/personality}{Merriam-Webster Dictionary} - ‘\textit{personalidad}’}. Según el diccionario Merriam-Webster, esta definición se utiliza desde el siglo XV\footnote{Ver “\href{https://www.merriam-webster.com/dictionary/personality\#word-history}{Primer uso conocido}” de la palabra ‘personalidad’ en el Diccionario Merriam Webster}. En la edición de 1828 del Diccionario Merriam Webster leemos la definición de la palabra ‘\textit{personalidad}’ como: “\textit{lo que constituye a un individuo como persona distinta}”\footnote{\href{https://archive.org/details/americandictiona02websrich/page/272/mode/2up}{Merriam-Webster Dictionary, edición de 1828} - ‘\textit{personalidad}’} \footnote{\href{https://archive.org/details/websterscomplete00webs/page/974/mode/2up}{La edición de 1886 del Diccionario Merriam-Webster} define la palabra ‘\textit{personalidad}’ como: “\textit{lo que constituye, o pertenece a, una persona}”}. Ambas definiciones se encuentran en The Encyclopaedic Dictionary, por Hunter Robert\footnote{\href{https://babel.hathitrust.org/cgi/pt?id=mdp.39015050663213&view=1up&seq=780}{Hunter Robert, The Encyclopaedic Dictionary} - ‘\textit{personalidad}’}—diccionario que era propiedad de Elena G. de White. El uso de estas definiciones puede verse en los artículos escritos sobre la \emcap{personalidad de Dios}.


In 1903, when Sister White wrote to Dr. Kellogg, \egwinline{I have \textbf{ever }had the same testimony to bear which I now bear \textbf{regarding the personality of God}}[Lt253-1903.9; 1903][https://egwwritings.org/read?panels=p9980.15], she recalled her vision when she saw the Father and the Son.


En 1903, cuando la hermana White escribió al Dr. Kellogg, \egwinline{Siempre \textbf{he }tenido el mismo testimonio que doy ahora \textbf{respecto a la personalidad de Dios}}[Lt253-1903.9; 1903][https://egwwritings.org/read?panels=p9980.15], recordó su visión cuando vio al Padre y al Hijo.


\egw{‘I have often seen the lovely Jesus, that\textbf{ He is a person}.\textbf{ I asked Him if His Father was a person, }and \textbf{had \underline{a form} like Himself}. Said Jesus, ‘\textbf{I am the express image of My Father’s person!}’ [Hebrews 1:3.]}[Lt253-1903.12; 1903][https://egwwritings.org/read?panels=p9980.18]


\egw{‘He visto a menudo al encantador Jesús, que\textbf{ Él es una persona}.\textbf{ Le pregunté si su Padre era una persona, }y \textbf{tenía \underline{una forma} como Él mismo}. Dijo Jesús, ‘\textbf{¡Yo soy la imagen misma de la sustancia de mi Padre!}’ [Hebreos 1:3.]}[Lt253-1903.12; 1903][https://egwwritings.org/read?panels=p9980.18]


The quality or state that Sister White defines God to be a person is to have \textit{a form}—\textit{a physical appearance}. Dr. Kellogg follows the same application of the word \textit{‘personality’}, although through speculation.


La cualidad o estado que la hermana White define que Dios sea una persona es tener \textit{una forma}—\textit{una apariencia física}. El Dr. Kellogg sigue la misma aplicación de la palabra \textit{‘personalidad’}, aunque a través de la especulación.


\others{The fact that God is so great that we cannot form a clear mental picture of \textbf{his physical appearance} need not lessen in our minds the reality of \textbf{His personality}...}[John H. Kellogg, The Living Temple, p. 31][https://archive.org/details/J.H.Kellogg.TheLivingTemple1903/page/n31/mode/2up]


\others{El hecho de que Dios sea tan grande que no podamos formarnos una imagen mental clara de \textbf{su apariencia física} no tiene por qué disminuir en nuestras mentes la realidad de \textbf{Su personalidad}...}[John H. Kellogg, The Living Temple, p. 31][https://archive.org/details/J.H.Kellogg.TheLivingTemple1903/page/n31/mode/2up]


As we have previously seen, our Adventist pioneers also pinpointed the physical appearance as a quality that makes God a person. James White wrote, \others{Those who deny \textbf{the personality of God}, say that ‘image’ here does not mean \textbf{physical form}, but moral image...}[James S. White, PERGO 1.1; 1861][https://egwwritings.org/read?panels=p1471.3]. J. B. Frisbie wrote, \others{Some seem to suppose it argues against \textbf{the personality of God}, because he is a Spirit, and say that he is without \textbf{body, or parts}...}[\href{https://documents.adventistarchives.org/Periodicals/RH/RH18540307-V05-07.pdf}{Adventist Review and Sabbath Herald, March 7, 1854}, J. B. Frisbie, “The Seventh-Day Sabbath Not Abolished”, p. 50]


Como hemos visto anteriormente, nuestros pioneros adventistas también señalaron la apariencia física como una cualidad que hace de Dios una persona. James White escribió, \others{Los que niegan \textbf{la personalidad de Dios}, dicen que ‘imagen’ aquí no significa \textbf{forma física}, sino imagen moral...}[James S. White, PERGO 1.1; 1861][https://egwwritings.org/read?panels=p1471.3]. J. B. Frisbie escribió, \others{Algunos parecen suponer que esto argumenta en contra de \textbf{la personalidad de Dios}, porque es un Espíritu, y dicen que no tiene \textbf{cuerpo, ni partes}...}[\href{https://documents.adventistarchives.org/Periodicals/RH/RH18540307-V05-07.pdf}{Adventist Review and Sabbath Herald, 7 de marzo de 1854}, J. B. Frisbie, “The Seventh-Day Sabbath Not Abolished”, p. 50]


In light of the facts, we recognize the application of the word ‘\textit{personality}’. When the subject on the \emcap{personality of God} is presented in its connection to the Trinity doctrine, there is often a tendency to alter the meaning of the word ‘\textit{personality}’. It is also important to mention that the subject on the \emcap{personality of God} deals with the personality of the Father. This is clearly seen from the presented data.


A la luz de los hechos, reconocemos la aplicación de la palabra ‘\textit{personalidad}’. Cuando el tema sobre la \emcap{personalidad de Dios} se presenta en su conexión con la doctrina trinitaria, a menudo hay una tendencia a alterar el significado de la palabra ‘\textit{personalidad}’. También es importante mencionar que el tema sobre la \emcap{personalidad de Dios} trata de la personalidad del Padre. Esto se ve claramente en los datos presentados.


\subsection*{Step 5: In examining the Kellogg crisis, shifting the main focus from the personality of God to pantheism}


\subsection*{Paso 5: Al examinar la crisis de Kellogg, cambiar el enfoque principal de la personalidad de Dios al panteísmo}


The data on the Kellogg crisis, in connection with the Trinity doctrine, is overwhelming if the \emcap{personality of God} is accounted for in the equation. The only way to not connect the dots is to ignore the \emcap{personality of God} and shift focus to pantheism exclusively. We do not deny the pantheistic nature of Kellogg's controversy. We believe that the pantheistic nature of Kellogg's controversy cannot be rightly understood if it is not examined in the true light of the \emcap{personality of God}. But, unfortunately, in examination of the Kellogg crisis, the attention that pantheism receives supersedes the examination of the truth on the \emcap{personality of God}.


Los datos sobre la crisis de Kellogg, en relación con la doctrina trinitaria, son abrumadores si se tiene en cuenta la \emcap{personalidad de Dios} en la ecuación. La única manera de no conectar los puntos es ignorar la \emcap{personalidad de Dios} y cambiar el enfoque hacia el panteísmo exclusivamente. No negamos la naturaleza panteísta de la controversia de Kellogg. Creemos que la naturaleza panteísta de la controversia de Kellogg no puede entenderse correctamente si no se examina a la verdadera luz de la \emcap{personalidad de Dios}. Pero, desafortunadamente, en el examen de la crisis de Kellogg, la atención que recibe el panteísmo sustituye el examen de la verdad sobre la \emcap{personalidad de Dios}.


You can do a search of Ellen White’s compilations to see just how much more attention pantheism received than the \emcap{personality of God}. If you were to search her writings for ‘pantheism’ or ‘pantheistic’, excluding the compilations after her death, you would find 36 occurrences. Among them are several repetitive quotations that Sister White copied from one letter to another, or to the special testimonies for the church. If you were to count the distinct occurrences you would only find 12 distinct quotations containing words like ‘\textit{pantheism}’ or ‘\textit{pantheistic}’\footnote{On the \href{https://egwwritings.org/}{https://egwwritings.org/} search bar, input the word “\textit{pantheis*} ”; this will include all words beginning with the ‘\textit{pantheis...}’, (including ‘\textit{pantheism}’ and ‘\textit{pantheistic}’). The results can be compared in subsetting the corpus of Ellen White writings by including or excluding compilations after her death. This option is available in the dropdown menu under the search bar.}. If you conducted the same search, but only in the compilations issued after her death, you would find 140 occurrences! All of these fall into one of the twelve distinct instances Sister White wrote on the subject of pantheism.


Se puede hacer una búsqueda en las compilaciones de Ellen White para ver cuánta más atención recibió el panteísmo que la \emcap{personalidad de Dios}. Si se buscara en sus escritos ‘panteísmo’ o ‘panteísta’, excluyendo las compilaciones posteriores a su muerte, se encontrarían 36 ocurrencias. Entre ellas hay varias citas repetitivas que la hermana White copió de una carta a otra, o a los testimonios especiales para la iglesia. Si usted contara las ocurrencias distintas, sólo encontraría 12 citas distintas que contienen palabras como ‘\textit{panteísmo}’ o ‘\textit{panteísta}’\footnote{En la barra de búsqueda de \href{https://egwwritings.org/}{https://egwwritings.org/}, ingrese la palabra “\textit{pantheis*} “; esto incluirá todas las palabras que comienzan con ‘\textit{pantheis...}’, (incluyendo ‘\textit{panteísmo}’ y ‘\textit{panteísta}’). Los resultados pueden compararse al subdividir el corpus de escritos de Ellen White incluyendo o excluyendo compilaciones después de su muerte. Esta opción está disponible en el menú desplegable debajo de la barra de búsqueda.}. Si se realiza la misma búsqueda, pero sólo en las compilaciones publicadas después de su muerte, ¡se encontrarían 140 ocurrencias! Todas ellas caen en una de las doce instancias distintas que la hermana White escribió sobre el tema del panteísmo.


In a search of Ellen White writings on the phrase “\textit{personality of God}”, excluding the compilations after her death, you would find 58 occurrences. Among them are also several repetitive quotations that Sister White copied to several different letters and to the testimonies for the church. Yet, if you were to search this phrase within the compilations that were issued after her death you would only find 52 occurrences.


En una búsqueda de los escritos de Ellen White sobre la frase “\textit{personalidad de Dios}”, excluyendo las compilaciones posteriores a su muerte, se encontrarían 58 ocurrencias. Entre ellas hay también varias citas repetitivas que la hermana White copió en varias cartas diferentes y en los testimonios para la iglesia. Sin embargo, si se buscara esta frase en las compilaciones que se publicaron después de su muerte, sólo se encontrarían 52 apariciones.


These simple statistics demonstrate the focus of the compilators after the death of Sister White. Such emphasis on pantheism changed our public opinion regarding Kellogg’s crisis. Forty-three, out of fifty-eight, quotations on the phrase “\textit{personality of God}” are found in letters and manuscripts, available to the public from 2015 onwards. This means that three quarters (\textit{74 percent}) of the quotation regarding the \emcap{personality of God}, prior to 2015, was not available to the public. Prior to 2015 we did not have much available data to study Kellogg's crisis in light of the \emcap{personality of God} and in its context.


Estas simples estadísticas demuestran el enfoque de los compiladores después de la muerte de la hermana White. Este énfasis en el panteísmo cambió nuestra opinión pública sobre la crisis de Kellogg. Cuarenta y tres, de cincuenta y ocho, citas sobre la frase “\textit{personalidad de Dios}” se encuentran en cartas y manuscritos, disponibles al público desde 2015 en adelante. Esto significa que tres cuartos (\textit{74 por ciento}) de las citas sobre la \emcap{personalidad de Dios}, antes de 2015, no estaban disponibles para el público. Antes de 2015 no teníamos muchos datos disponibles para estudiar la crisis de Kellogg a la luz de la \emcap{personalidad de Dios} y en su contexto.


% Steps to Omega

\begin{titledpoem}
    
    \stanza{
        On pillars now, the shadows cast— \\
        A truth forsaken, from the past. \\
        In steps they chart the silent drift, \\
        Five marks of change, through sacred rift.
    }

    \stanza{
        Denial blooms when once truth stood, \\
        Foundations are not understood, \\
        The fundamentals, once held dear \\
        Obscured, as new creeds appear.
    }

    \stanza{
        Prophetic warnings have been dimmed, \\
        Pioneers are shunned, old hymns are trimmed. \\
        The testimonies once rang out \\
        But now they’re often tinged with doubt.
    }

    \stanza{
        “God is a person” cast aside, \\
        And now His essence they deride. \\
        Forgotten pillar once was strong \\
        Now a new pillar, which is wrong!
    }

    \stanza{
        Scholars now twist the sacred term, \\
        Words redefined, they now affirm. \\
        Gone is the quest to see God’s face, \\
        Dim the desire for His embrace.
    }

    \stanza{
        The Kellogg crisis point is missed, \\
        The alpha given untrue twist \\
        And thus, the lessons are not learned \\
        The church toward omega turned.
    }

    \stanza{
        Confusion reigns, we can’t perceive \\
        It is not clear what we believe \\
        Our history has been revised \\
        We wanted truth, but then they lied.
    }
    
\end{titledpoem}


% \chapter{The steps to Omega}


\chapter{Kroki do Omegi}


In our study so far, we have seen evidence that Kellogg’s controversy was connected to the Trinity doctrine and the \emcap{personality of God} expressed in the first point of the \emcap{Fundamental Principles}. Unfortunately, today we do not stand on that foundation regarding the \emcap{personality of God}; we have built another foundation that has changed the truth on the \emcap{personality of God} to a mysterious Triune God. Sister White was clearly against this reorganization and she prophesied that in the closing of His work, God will rehearse the history of the Advent movement and re-establish every pillar of our faith that was held in the beginning.


W naszym dotychczasowym studium widzieliśmy dowody, że kontrowersja Kellogga była związana z doktryną o Trójcy i \emcap{osobowością Boga} wyrażoną w pierwszym punkcie \emcap{Fundamentalnych Zasad}. Niestety, dzisiaj nie stoimy na tym fundamencie dotyczącym \emcap{osobowości Boga}; zbudowaliśmy inny fundament, który zmienił prawdę o \emcap{osobowości Boga} w tajemniczego Trójjedynego Boga. Siostra White była wyraźnie przeciwna tej reorganizacji i przepowiedziała, że przy końcu Swojego dzieła Bóg powtórzy historię ruchu adwentowego i przywróci każdy filar naszej wiary, który był utrzymywany na początku.


\egw{\textbf{\underline{The Lord has declared that the history of the past shall be rehearsed as we enter upon the closing work}. \underline{Every truth} that He has given for these last days is to be proclaimed to the world. \underline{Every pillar} that He has established \underline{is to be strengthened}. We cannot now step off the foundation that God has established. We cannot now enter into any new organization; for this would mean apostasy from the truth}.}[Ms129-1905.6; 1905][https://egwwritings.org/?ref=en\_Ms129-1905.6&para=9797.13]


\egw{\textbf{\underline{Pan oświadczył, że historia przeszłości zostanie powtórzona, gdy wkraczamy w końcowe dzieło}. \underline{Każda prawda}, którą dał na te ostatnie dni, ma być ogłoszona światu. \underline{Każdy filar}, który ustanowił, \underline{ma być wzmocniony}. Nie możemy teraz zejść z fundamentu, który Bóg ustanowił. Nie możemy teraz wejść w żadną nową organizację; oznaczałoby to odstępstwo od prawdy}.}[Ms129-1905.6; 1905][https://egwwritings.org/?ref=en\_Ms129-1905.6&para=9797.13]


Comparing the \emcap{Fundamental Principles} with the current Fundamental Beliefs of Seventh-day Adventists, we see that we have entered into a new organization. God’s warning, given through Sister White, to re-establish all pillars of our faith in these last days, is becoming imperative. As we traced the Trinity doctrine from Kellogg's controversy, we came across Ellen White’s warnings against alpha and omega apostasy, which will enter into our church.


Porównując \emcap{Fundamentalne Zasady} z obecnymi Fundamentalnymi Wierzeniami Adwentystów Dnia Siódmego, widzimy, że weszliśmy w nową organizację. Boże ostrzeżenie, przekazane przez Siostrę White, aby przywrócić wszystkie filary naszej wiary w tych ostatnich dniach, staje się naglące. Śledząc doktrynę o Trójcy od kontrowersji Kellogga, natrafiliśmy na ostrzeżenia Ellen White przed odstępstwem alfa i omega, które wejdzie do naszego kościoła.


\egw{\textbf{‘Living Temple’ contains the alpha of these theories. I knew that \underline{the omega would follow in a little while}; and I trembled for our people}. I knew that \textbf{I must warn our brethren and sisters not to enter into controversy \underline{over the presence and personality of God}. The statements made in ‘Living Temple’ \underline{in regard to this point are incorrect}. }The scripture used to substantiate the doctrine there set forth, is scripture misapplied.}[SpTB02 53.2; 1904][https://egwwritings.org/?ref=en\_SpTB02.53.2&para=417.271]


\egw{\textbf{‘Living Temple’ zawiera alfę tych teorii. Wiedziałam, że \underline{omega nastąpi wkrótce}; i drżałam o nasz lud}. Wiedziałam, że \textbf{muszę ostrzec naszych braci i siostry, aby nie wchodzili w spór \underline{dotyczący obecności i osobowości Boga}. Stwierdzenia zawarte w ‘Living Temple’ \underline{odnośnie tego punktu są niepoprawne}. }Pismo Święte użyte do poparcia doktryny tam przedstawionej jest błędnie zastosowanym Pismem.}[SpTB02 53.2; 1904][https://egwwritings.org/?ref=en\_SpTB02.53.2&para=417.271]


In the context of Seventh-day Adventist reorganization, we identify several steps that were necessary to accomplish this reorganization and are necessary to uphold it.


W kontekście reorganizacji Adwentystów Dnia Siódmego, identyfikujemy kilka kroków, które były niezbędne do dokonania tej reorganizacji i są konieczne do jej podtrzymania.


\subsection*{Step 1: Deny the Fundamental Principles to be the foundation of our faith and the official, and accurate, representation of Seventh-day Adventist beliefs}


\subsection*{Krok 1: Zaprzeczenie, że Fundamentalne Zasady są fundamentem naszej wiary oraz oficjalną i dokładną reprezentacją wierzeń Adwentystów Dnia Siódmego}


The first step necessary is to hide the original foundation of our faith by unlinking it with the \emcap{Fundamental Principles}.


Pierwszym niezbędnym krokiem jest ukrycie oryginalnego fundamentu naszej wiary poprzez odłączenie go od \emcap{Fundamentalnych Zasad}.


\egw{\textbf{As a people, we are to \underline{stand firm on the platform of eternal truth} that has withstood test and trial. We are to \underline{hold to the sure pillars of our faith}. \underline{The principles of truth} that God has revealed to us \underline{are our only true foundation}. They have made us what we are. The lapse of time has not lessened their value. \underline{It is the constant effort of the enemy to remove these truths from their setting}, and to put in their place \underline{spurious theories}. He \underline{will bring in} everything that he possibly can to carry out his deceptive designs.}}[SpTB02 51.2; 1904][https://egwwritings.org/?ref=en\_SpTB02.51.2&para=417.261]


\egw{\textbf{Jako lud, mamy \underline{stać mocno na platformie wiecznej prawdy}, która wytrzymała próby i testy. Mamy \underline{trzymać się pewnych filarów naszej wiary}. \underline{Zasady prawdy}, które Bóg nam objawił, \underline{są naszym jedynym prawdziwym fundamentem}. One uczyniły nas tym, kim jesteśmy. Upływ czasu nie zmniejszył ich wartości. \underline{Nieustannym wysiłkiem wroga jest usunięcie tych prawd z ich kontekstu} i umieszczenie na ich miejscu \underline{fałszywych teorii}. On \underline{wprowadzi} wszystko, co tylko może, aby zrealizować swoje zwodnicze plany.}}[SpTB02 51.2; 1904][https://egwwritings.org/?ref=en\_SpTB02.51.2&para=417.261]


\egw{\textbf{Messages of every order and kind have been urged upon Seventh-day Adventists, to take the place of the truth which, \underline{point by point}, has been sought out by prayerful study, and testified to by the miracle-working power of the Lord}. \textbf{But \underline{the way-marks} \underline{which have made us what we are}, \underline{are to be preserved}, and they \underline{will be preserved}, as God has signified through His word and the testimony of His Spirit}. \textbf{He calls upon us to \underline{hold firmly}, with the grip of faith, to \underline{the fundamental principles} that are \underline{based upon unquestionable authority}}.}[SpTB02 59.1; 1904][https://egwwritings.org/?ref=en\_SpTB02.59.1&para=417.299]


\egw{\textbf{Przesłania wszelkiego rodzaju i typu były narzucane Adwentystom Dnia Siódmego, aby zająć miejsce prawdy, która, \underline{punkt po punkcie}, została odkryta przez modlitewne studium i potwierdzona przez cudowną moc Pana}. \textbf{Ale \underline{znaki}, które \underline{uczyniły nas tym, kim jesteśmy}, \underline{mają być zachowane} i \underline{będą zachowane}, jak Bóg oznajmił przez Swoje słowo i świadectwo Swojego Ducha}. \textbf{Wzywa nas, abyśmy \underline{trzymali się mocno}, z uściskiem wiary, \underline{fundamentalnych zasad}, które \underline{opierają się na niepodważalnym autorytecie}}.}[SpTB02 59.1; 1904][https://egwwritings.org/?ref=en\_SpTB02.59.1&para=417.299]


The \emcap{Fundamental Principles} were the truths God revealed to the pioneers after the passing of time in 1844. We have seen the testimonies of our pioneers, including Ellen White, regarding the first point of the \emcap{Fundamental Principles}. All of them were in harmony regarding these particular points of our faith. In 1863, Seventh-day Adventists organized themselves into a church, as an organized body. Since then, many were misrepresenting the position of the Seventh-day Adventist Church and the pioneers found it necessary to meet inquiries, \others{and sometimes to correct false statements circulated against} the church’s beliefs and practices. Consequently, in 1872, the pioneers issued the document called “\textit{A Declaration of the Fundamental Principles, Taught and Practiced by the Seventh-Day Adventists}”\footnote{“A Declaration of the Fundamental Principles, Taught and Practiced by the Seventh-Day Adventists (1872) : MVT : Free Download, Borrow, and Streaming : Internet Archive.” Internet Archive, 2025, \href{https://archive.org/details/ADeclarationOfTheFundamentalPrinciplesTaughtAndPracticedByThe}{archive.org/details/ADeclarationOfTheFundamentalPrinciplesTaughtAndPracticedByThe}. Accessed 3 Feb. 2025.}. This declaration presented the public with \others{a brief statement of what is, and has been, with great unanimity, held by}[The preface of the Fundamental Principles in 1872.] Seventh-day Adventists.


\emcap{Fundamentalne zasady} były prawdami, które Bóg objawił pionierom po upływie czasu w 1844 roku. Widzieliśmy świadectwa naszych pionierów, w tym Ellen White, dotyczące pierwszego punktu \emcap{Fundamentalnych zasad}. Wszyscy oni byli zgodni co do tych konkretnych punktów naszej wiary. W 1863 roku Adwentyści Dnia Siódmego zorganizowali się w kościół, jako zorganizowane ciało. Od tego czasu wielu błędnie przedstawiało stanowisko Kościoła Adwentystów Dnia Siódmego, a pionierzy uznali za konieczne odpowiadanie na zapytania, \others{a czasami korygowanie fałszywych stwierdzeń rozpowszechnianych przeciwko} wierzeniom i praktykom kościoła. W konsekwencji, w 1872 roku, pionierzy wydali dokument zatytułowany “\textit{Oświadczenie o fundamentalnych zasadach nauczanych i wyznawanych przez Adwentystów Dnia Siódmego}”\footnote{“A Declaration of the Fundamental Principles, Taught and Practiced by the Seventh-Day Adventists (1872) : MVT : Free Download, Borrow, and Streaming : Internet Archive.” Internet Archive, 2025, \href{https://archive.org/details/ADeclarationOfTheFundamentalPrinciplesTaughtAndPracticedByThe}{archive.org/details/ADeclarationOfTheFundamentalPrinciplesTaughtAndPracticedByThe}. Accessed 3 Feb. 2025.}. Ta deklaracja przedstawiła opinii publicznej \others{krótkie oświadczenie o tym, co jest i było, z wielką jednomyślnością, wyznawane przez}[Przedmowa do Fundamentalnych zasad z 1872 roku.] Adwentystów Dnia Siódmego.


In the chapter “\hyperref[chap:authority]{The Authority of the Fundamental Principles}”, we discussed how pro-Trinitarian scholars have been compromising the authority of the \emcap{Fundamental Principles}, denying their true value in our Adventist history.


W rozdziale “\hyperref[chap:authority]{Autorytet Fundamentalnych zasad}”, omówiliśmy, jak uczeni popierający Trójcę kompromitują autorytet \emcap{Fundamentalnych zasad}, zaprzeczając ich prawdziwej wartości w naszej adwentystycznej historii.


Pro-trinitarian scholars argue that this declaration was not what it claims to be—a declaration of the \emcap{fundamental principles}, taught and practiced by the Seventh-day Adventists. This declaration was a summary of the principal features of Adventist’s faith, and no point is really as problematic or objectionable as the first point, dealing with the \emcap{personality of God} and where His presence is. But the evidence in favor of the \emcap{Fundamental Principles}, especially to the first point, is overwhelming.


Uczeni popierający Trójcę twierdzą, że ta deklaracja nie była tym, co twierdzi - deklaracją \emcap{fundamentalnych zasad}, nauczanych i praktykowanych przez Adwentystów Dnia Siódmego. Ta deklaracja była podsumowaniem głównych cech wiary adwentystycznej, a żaden punkt nie jest naprawdę tak problematyczny lub kontrowersyjny jak pierwszy punkt, dotyczący \emcap{osobowości Boga} i tego, gdzie jest Jego obecność. Ale dowody na korzyść \emcap{Fundamentalnych zasad}, szczególnie pierwszego punktu, są przytłaczające.


All of these claims are easily refuted by the fact that the \emcap{Fundamental Principles} have been regularly issued and reprinted over the course of the entire life of Sister White, until 1914. If they were mere private opinions of a few individuals, as claimed by scholars\footnote{Ministry Magazine “Our Declaration of Fundamental Beliefs”: January 1958, Roy Anderson, J. Arthur Buckwalter, Louise Kleuser, Earl Cleveland and Walter Schubert}, would they have been consistently reprinted over the course of 42 years\footnote{For a detailed list of publications throughout these years, see the Appendix.}, publicly claiming to represent the synopsis of Seventh-day Adventist faith? If they had been issued only once, we could deem it a conspiracy by some individuals to purposely misrepresent Seventh-day Adventist faith. On the contrary, the \emcap{Fundamental Principles} were regularly reprinted, and they truly represented the official Seventh-day Adventist faith and practice.


Wszystkie te twierdzenia są łatwo obalone przez fakt, że \emcap{Fundamentalne zasady} były regularnie wydawane i przedrukowywane przez całe życie Siostry White, aż do 1914 roku. Gdyby były jedynie prywatnymi opiniami kilku osób, jak twierdzą uczeni\footnote{Ministry Magazine “Our Declaration of Fundamental Beliefs”: January 1958, Roy Anderson, J. Arthur Buckwalter, Louise Kleuser, Earl Cleveland and Walter Schubert}, czy byłyby konsekwentnie przedrukowywane przez 42 lata\footnote{Szczegółową listę publikacji w tych latach można znaleźć w Załączniku.}, publicznie twierdząc, że reprezentują streszczenie wiary Adwentystów Dnia Siódmego? Gdyby zostały wydane tylko raz, moglibyśmy uznać to za spisek kilku osób, które celowo błędnie przedstawiały wiarę Adwentystów Dnia Siódmego. Przeciwnie, \emcap{Fundamentalne zasady} były regularnie przedrukowywane i naprawdę reprezentowały oficjalną wiarę i praktykę Adwentystów Dnia Siódmego.


Another argument is that Sister White approved the \emcap{Fundamental Principles} in her writings by explicitly referring to them, and also by teaching the same truths taught in the \emcap{Fundamental Principles}. The works of our pioneers are also in harmony with the statements in this Declaration of the \emcap{Fundamental Principles}. Considering all of these facts, it is inevitable that this declaration was truthful in its claims. This document was indeed a declaration of the \emcap{fundamental principles}, taught and practiced by the Seventh-day Adventist Church, representing a public \others{synopsis of our faith}, \others{a brief statement of what is, and has been, with great unanimity, held by} Seventh-day Adventists.\footnote{The preface of the Fundamental Principles in 1872.} As such, it accurately represents the Seventh-day Adventist belief and practice, and represents the foundation of Seventh-day Adventist faith in the time of Ellen White.


Innym argumentem jest to, że Siostra White zatwierdziła \emcap{Fundamentalne zasady} w swoich pismach, wyraźnie się do nich odnosząc, a także nauczając tych samych prawd, które są nauczane w \emcap{Fundamentalnych zasadach}. Prace naszych pionierów są również zgodne z oświadczeniami w tej Deklaracji \emcap{Fundamentalnych zasad}. Biorąc pod uwagę wszystkie te fakty, jest nieuniknione, że ta deklaracja była prawdziwa w swoich twierdzeniach. Ten dokument był rzeczywiście deklaracją \emcap{fundamentalnych zasad}, nauczanych i praktykowanych przez Kościół Adwentystów Dnia Siódmego, reprezentującą publiczne \others{streszczenie naszej wiary}, \others{krótkie oświadczenie o tym, co jest i było, z wielką jednomyślnością, wyznawane przez} Adwentystów Dnia Siódmego.\footnote{Przedmowa do Fundamentalnych zasad z 1872 roku.} Jako taki, dokładnie reprezentuje wierzenia i praktyki Adwentystów Dnia Siódmego i stanowi fundament wiary Adwentystów Dnia Siódmego w czasach Ellen White.


Today, in defense of the Trinity doctrine, Adventist historians boldly claim that when our pioneers were studying Adventist truths such as the sanctuary, investigative judgment, the Sabbath and other doctrines, they \others{did not study the subject of the doctrine of God}. These Adventist historians falsely claim that the doctrine of God \others{was not the question that they dealt at that time}[Denis Kaiser. “From Antitrinitarianism to Trinitarianism: The Adventist story” and Panelist. The God We Worship: A Godhead Symposium. Central California Conference, Dinuba, CA. March 23-24, 2018.]. Following this false claim, they present historical data on how Adventist doctrine gradually moved toward Trinitarian understanding. The truth is, there are some instances early on\footnote{The earliest mention of the Trinity doctrine, in a positive sense, was when M.C. Wilcox reprinted a non-Adventist article by Samuel Spear in Signs of the Times, December 7th, 1891 and December 14th, 1891} when the Trinity doctrine is mentioned in a positive light in our literature. But when you consider the fact that the Adventist church did have a positive position on the subject of the doctrine of God, as it was expressed in the \emcap{Fundamental Principles}, these instances cannot be interpreted as progressiveness in understanding, but rather an intrusion of the Trinity doctrine into the Seventh-day Adventist Church.


Dziś, w obronie doktryny o Trójcy, historycy adwentystyczni śmiało twierdzą, że kiedy nasi pionierzy studiowali adwentystyczne prawdy, takie jak świątynia, sąd śledczy, Sabat i inne doktryny, \others{nie studiowali tematu doktryny o Bogu}. Ci historycy adwentystyczni fałszywie twierdzą, że doktryna o Bogu \others{nie była kwestią, którą zajmowali się w tamtym czasie}[Denis Kaiser. “From Antitrinitarianism to Trinitarianism: The Adventist story” and Panelist. The God We Worship: A Godhead Symposium. Central California Conference, Dinuba, CA. March 23-24, 2018.]. Po tym fałszywym twierdzeniu przedstawiają dane historyczne o tym, jak doktryna adwentystyczna stopniowo zmierzała w kierunku trynitarnego zrozumienia. Prawda jest taka, że istnieją pewne wczesne przypadki\footnote{Najwcześniejsza wzmianka o doktrynie o Trójcy, w pozytywnym sensie, miała miejsce, gdy M.C. Wilcox przedrukował nieadwentystyczny artykuł Samuela Speara w Signs of the Times, 7 grudnia 1891 i 14 grudnia 1891}, kiedy doktryna o Trójcy jest wspominana w pozytywnym świetle w naszej literaturze. Ale gdy weźmie się pod uwagę fakt, że Kościół Adwentystów miał pozytywne stanowisko w kwestii doktryny o Bogu, wyrażone w \emcap{Fundamentalnych zasadach}, tych przypadków nie można interpretować jako postępu w zrozumieniu, ale raczej jako wtargnięcie doktryny o Trójcy do Kościoła Adwentystów Dnia Siódmego.


It is easy to refute the claim that Adventist pioneers did not understand the doctrine of God. If they did not understand it, they would have failed to proclaim the first angel’s message. We discussed this point in detail in the chapter “\hyperref[chap:remembering-the-beginning]{Remembering the beginning}”. The Seventh-day Adventist movement was not a failure, but a God-led, prophetic movement.


Łatwo jest obalić twierdzenie, że pionierzy adwentystyczni nie rozumieli doktryny o Bogu. Gdyby jej nie rozumieli, nie udałoby im się głosić poselstwa pierwszego anioła. Omówiliśmy ten punkt szczegółowo w rozdziale “\hyperref[chap:remembering-the-beginning]{Pamiętając początek}”. Ruch Adwentystów Dnia Siódmego nie był porażką, ale prowadzonym przez Boga, prorockim ruchem.


\subsection*{Step 2: Ignore the warnings of building a new foundation}


\subsection*{Krok 2: Ignorowanie ostrzeżeń przed budowaniem nowego fundamentu}


When the \emcap{Fundamental Principles} are removed from the equation, many of Ellen White’s warnings fail to shine in their true light and their true meaning does not resonate with the reader.


Kiedy \emcap{Fundamentalne zasady} są usunięte z równania, wiele ostrzeżeń Ellen White nie świeci w swoim prawdziwym świetle, a ich prawdziwe znaczenie nie rezonuje z czytelnikiem.


We have cited many quotations where Sister White warned the church not to step off the \emcap{Fundamental Principles}. We dealt with them in the chapter “\hyperref[chap:apostasy]{The great apostasy is soon to be realized}”, but we will mention one of the most prominent quotations again.


Cytowaliśmy wiele wypowiedzi, w których Siostra White ostrzegała kościół, aby nie odstępował od \emcap{Fundamentalnych zasad}. Zajmowaliśmy się nimi w rozdziale “\hyperref[chap:apostasy]{Wielkie odstępstwo wkrótce się zrealizuje}”, ale wspomnimy ponownie jeden z najbardziej znaczących cytatów.


\egw{\textbf{The enemy of souls has sought to bring in the supposition that a great reformation was to take place among Seventh-day Adventists, and that this reformation would \underline{consist in giving up the doctrines which stand as the pillars of our faith} and engaging in a process of reorganization}. Were this reformation to take place, what would result? \textbf{The principles of truth that God in His wisdom has given to the remnant church would be discarded. Our religion would be changed. \underline{The fundamental principles that have sustained the work for the last fifty years would be accounted as error}}. \textbf{A new organization would be established. Books of a new order would be written. A system of intellectual philosophy would be introduced}...}[Lt242-1903.13; 1903][https://egwwritings.org/?ref=en\_Lt242-1903.13&para=7767.20]


\egw{\textbf{Wróg dusz starał się wprowadzić przypuszczenie, że wśród Adwentystów Dnia Siódmego miała nastąpić wielka reforma, i że ta reforma \underline{polegałaby na porzuceniu doktryn, które stoją jako filary naszej wiary} i zaangażowaniu się w proces reorganizacji}. Gdyby ta reforma miała miejsce, co by z tego wynikło? \textbf{Zasady prawdy, które Bóg w swojej mądrości dał Kościołowi ostatków, zostałyby odrzucone. Nasza religia zostałaby zmieniona. \underline{Fundamentalne zasady, które podtrzymywały dzieło przez ostatnie pięćdziesiąt lat, zostałyby uznane za błąd}}. \textbf{Zostałaby ustanowiona nowa organizacja. Zostałyby napisane książki nowego porządku. Zostałby wprowadzony system filozofii intelektualnej}...}[Lt242-1903.13; 1903][https://egwwritings.org/?ref=en\_Lt242-1903.13&para=7767.20]


\egwnogap{Who has authority to begin such a movement? \textbf{We have our Bibles. We have our experience, attested to by the miraculous working of the Holy Spirit}. \textbf{We have a truth that admits of no compromise.} \textbf{\underline{Shall we not repudiate everything that is not in harmony with this truth}?}}[Lt242-1903.14; 1903][https://egwwritings.org/?ref=en\_Lt242-1903.14&para=7767.21]


\egwnogap{Kto ma upoważnienie do rozpoczęcia takiego ruchu? \textbf{Mamy nasze Biblie. Mamy nasze doświadczenie, potwierdzone cudownym działaniem Ducha Świętego}. \textbf{Mamy prawdę, która nie dopuszcza żadnego kompromisu.} \textbf{\underline{Czy nie powinniśmy odrzucić wszystkiego, co nie jest w harmonii z tą prawdą}?}}[Lt242-1903.14; 1903][https://egwwritings.org/?ref=en\_Lt242-1903.14&para=7767.21]


\subsection*{Step 3: Deny that the personality of God was the pillar of our faith and a part of the foundation of our faith}


\subsection*{Krok 3: Zaprzeczenie, że osobowość Boga była filarem naszej wiary i częścią fundamentu naszej wiary}


There is one Ellen White statement that apparently supports the claim that the \emcap{personality of God} was not a pillar of our faith. Another expression for “\textit{pillars of our faith}” is “\textit{landmarks}”. In the following quotations, Sister White lists several landmarks: the cleansing of the sanctuary, the three angels’ messages, the temple of God, the Sabbath and the non-immortality of the wicked.


Istnieje jedno stwierdzenie Ellen White, które pozornie popiera twierdzenie, że \emcap{osobowość Boga} nie była filarem naszej wiary. Innym wyrażeniem na “\textit{filary naszej wiary}” jest “\textit{znaki}”. W poniższych cytatach Siostra White wymienia kilka znaków: oczyszczenie świątyni, poselstwa trzech aniołów, świątynię Boga, Sabat i nieśmiertelność bezbożnych.


\egw{The passing of the time in 1844 was a period of great events, opening to our astonished eyes \textbf{the cleansing of the sanctuary transpiring in heaven}, and having decided relation to God’s people upon the earth, [also] \textbf{the first and second angels’ messages and the third}, unfurling the banner on which was inscribed, ‘The commandments of God and the faith of Jesus.’ [Revelation 14:12.] One of the landmarks under this message was \textbf{the temple of God}, seen by His truth-loving people in heaven, and the ark containing the law of God. The light of \textbf{the Sabbath} of the fourth commandment flashed its strong rays in the pathway of the transgressors of God’s law. The \textbf{non-immortality of the wicked} is an old landmark. \textbf{I can call to mind nothing more that can come under the head of the old landmarks}. All this cry about changing the old landmarks is all imaginary.}[Ms13-1889.9; 1889][https://egwwritings.org/?ref=en\_Ms13-1889.9&para=4179.14]


\egw{Upływ czasu w 1844 roku był okresem wielkich wydarzeń, otwierających naszym zdumionym oczom \textbf{oczyszczenie świątyni odbywające się w niebie} i mające zdecydowany związek z ludem Bożym na ziemi, [także] \textbf{poselstwa pierwszego i drugiego anioła oraz trzeciego}, rozwijające sztandar, na którym było napisane: ‘Przykazania Boże i wiara Jezusa.’ [Objawienie 14:12.] Jednym ze znaków pod tym poselstwem była \textbf{świątynia Boża}, widziana przez Jego miłujący prawdę lud w niebie, i arka zawierająca prawo Boże. Światło \textbf{Sabatu} z czwartego przykazania rzucało swoje silne promienie na ścieżkę przestępców prawa Bożego. \textbf{Nieśmiertelność bezbożnych} jest starym znakiem. \textbf{Nie mogę przypomnieć sobie niczego więcej, co mogłoby wchodzić pod nagłówek starych znaków}. Całe to wołanie o zmianę starych znaków jest całkowicie wyimaginowane.}[Ms13-1889.9; 1889][https://egwwritings.org/?ref=en\_Ms13-1889.9&para=4179.14]


At the end of this list of landmarks, or pillars of our faith, she states that she can recall nothing else that would fall under the category of the old landmarks. For many, this quotation serves as proof that the \emcap{personality of God} was neither an old landmark nor a pillar. It is true that in this quotation, Sister White did not explicitly mention the \emcap{personality of God}, but it would be implicitly included under the first angel’s message, as well as being an underlying doctrine of the Sanctuary message. Furthermore, there are other quotations from Sister White that explicitly include the \emcap{personality of God} as an old landmark or pillar of our faith.


Na końcu tej listy znaków, czyli filarów naszej wiary, stwierdza, że nie może przypomnieć sobie niczego innego, co wchodziłoby do kategorii starych znaków. Dla wielu ten cytat służy jako dowód, że \emcap{osobowość Boga} nie była ani starym znakiem, ani filarem. To prawda, że w tym cytacie Siostra White nie wspomniała wyraźnie o \emcap{osobowości Boga}, ale byłaby ona domyślnie zawarta w poselstwie pierwszego anioła, a także jako podstawowa doktryna poselstwa o Świątyni. Ponadto istnieją inne cytaty od Siostry White, które wyraźnie włączają \emcap{osobowość Boga} jako stary znak lub filar naszej wiary.


\egw{Those who seek to remove the \textbf{old landmarks} are not holding fast; they \textbf{are not remembering how they have received and heard}. Those who try to \textbf{\underline{bring in} theories that would remove \underline{the pillars of our faith}} \textbf{concerning the sanctuary}, \textbf{\underline{or concerning the personality of God or of Christ}, are working as blind men}. They are seeking to bring in uncertainties and to set the people of God \textbf{adrift}, without an anchor.}[Ms62-1905.14; 1905][https://egwwritings.org/?ref=en\_Ms62-1905.14&para=10026.20]


\egw{Ci, którzy starają się usunąć \textbf{stare znaki}, nie trzymają się mocno; \textbf{nie pamiętają, jak otrzymali i słyszeli}. Ci, którzy próbują \textbf{\underline{wprowadzić} teorie, które usunęłyby \underline{filary naszej wiary}} \textbf{dotyczące świątyni}, \textbf{\underline{lub dotyczące osobowości Boga lub Chrystusa}, działają jak ślepi ludzie}. Starają się wprowadzić niepewności i postawić lud Boży \textbf{na dryfie}, bez kotwicy.}[Ms62-1905.14; 1905][https://egwwritings.org/?ref=en\_Ms62-1905.14&para=10026.20]


Sister White also teaches us that the pillars of our faith constitute the foundation of our faith.


Siostra White uczy nas również, że filary naszej wiary stanowią fundament naszej wiary.


\egw{\textbf{What influence is it that would lead men at this stage of our history to work in an underhanded, powerful way \underline{to tear down the foundation of our faith},—the foundation that was laid at the beginning of our work by prayerful study of the word and by revelation? Upon \underline{this foundation} we have been building for \underline{the past fifty years}. Do you wonder that when I see the beginning of a work that would \underline{remove some of the pillars of our faith}, I have something to say? I must obey the command, ‘Meet it!’}}[SpTB02 58.1; 1904][https://egwwritings.org/?ref=en\_SpTB02.58.1&para=417.295]


\egw{\textbf{Jaki wpływ prowadzi ludzi na tym etapie naszej historii do działania w podstępny, potężny sposób, \underline{aby zburzyć fundament naszej wiary},—fundament, który został położony na początku naszej pracy przez modlitewne studiowanie słowa i przez objawienie? Na \underline{tym fundamencie} budujemy przez \underline{ostatnie pięćdziesiąt lat}. Czy dziwisz się, że kiedy widzę początek pracy, która \underline{usunęłaby niektóre z filarów naszej wiary}, mam coś do powiedzenia? Muszę być posłuszna rozkazowi: ‘Przeciwstaw się temu!’}}[SpTB02 58.1; 1904][https://egwwritings.org/?ref=en\_SpTB02.58.1&para=417.295]


Removing some of the pillars of our faith means tearing down the foundation of our faith. Elsewhere, Sister White said that tearing down or undermining the foundation of our faith is done by indoctrination of the sentiments regarding the \emcap{personality of God}.


Usunięcie niektórych filarów naszej wiary oznacza zburzenie fundamentu naszej wiary. W innym miejscu Siostra White powiedziała, że burzenie lub podważanie fundamentu naszej wiary odbywa się poprzez indoktrynację poglądów dotyczących \emcap{osobowości Boga}.


\egw{The college was taken out of Battle Creek; yet students are still called there, and there they \textbf{become indoctrinated with the very sentiments regarding the personality of God and Christ that would undermine the foundation of our faith}.}[Lt72-1906.5; 1906][https://egwwritings.org/?ref=en\_Lt72-1906.5&para=10013.11]


\egw{Kolegium zostało przeniesione z Battle Creek; jednak studenci nadal są tam wzywani i tam \textbf{zostają indoktrynowani poglądami dotyczącymi osobowości Boga i Chrystusa, które podważyłyby fundament naszej wiary}.}[Lt72-1906.5; 1906][https://egwwritings.org/?ref=en\_Lt72-1906.5&para=10013.11]


In light of these quotations we see positive testimony that the \emcap{personality of God} was part of the foundation of our faith. Furthermore, in chapter 10 of the special testimonies, entitled “\textit{The foundation of our faith}”, Sister White mentioned “\textit{Fundamental Principles}” using the synonyms “\textit{pillars of our faith}”, “\textit{waymarks}”, and “\textit{landmarks}”, when addressing the foundation of our faith.


W świetle tych cytatów widzimy pozytywne świadectwo, że \emcap{osobowość Boga} była częścią fundamentu naszej wiary. Ponadto, w rozdziale 10 specjalnych świadectw, zatytułowanym “\textit{Fundament naszej wiary}”, Siostra White wspomniała o “\textit{Fundamentalnych Zasadach}” używając synonimów “\textit{filary naszej wiary}”, “\textit{znaki}” i “\textit{punkty orientacyjne}”, odnosząc się do fundamentu naszej wiary.


\subsection*{Step 4: Alter the meaning of the term “the personality of God”}


\subsection*{Krok 4: Zmiana znaczenia terminu “osobowość Boga”}


The term ‘\textit{personality}’ has two different applications and the most common definition in everyday use is in the area of psychology. ‘\textit{Personality}’ is defined as “\textit{the characteristic sets of behaviors, cognitions, and emotional patterns that evolve from biological and environmental factors}”\footnote{Wikipedia Contributors. “Personality.” Wikipedia, Wikimedia Foundation, 19 Apr. 2019, \href{https://en.wikipedia.org/wiki/Personality}{en.wikipedia.org/wiki/Personality}.}. It is of utmost importance to recognize that when we are dealing with the pillar of our faith—“\textit{the personality of God}”—we are not in the realms of psychology. The accurate application of the word ‘\textit{personality}’ within the doctrine on the \emcap{personality of God} is found in the Merriam-Webster Dictionary: “\textit{the quality or state of being a person}”\footnote{\href{https://www.merriam-webster.com/dictionary/personality}{Merriam-Webster Dictionary} - ‘\textit{personality}’}. According to the Merriam-Webster Dictionary, this definition has been in use since the 15th century\footnote{See “\href{https://www.merriam-webster.com/dictionary/personality\#word-history}{First known use}” of the word ‘personality’ in Merriam Webster Dictionary}. In the 1828 edition of the Merriam Webster Dictionary we read definition of the word ‘\textit{personality}’ as: “\textit{that which constitutes an individual a distinct person}”\footnote{\href{https://archive.org/details/americandictiona02websrich/page/272/mode/2up}{Merriam-Webster Dictionary, 1828 edition} - ‘\textit{personality}’} \footnote{\href{https://archive.org/details/websterscomplete00webs/page/974/mode/2up}{The 1886 edition of Merriam-Webster Dictionary} defines the word ‘\textit{personality}’ as: “\textit{that which constitutes, or pertains to, a person}”}. Both of the definitions are found in The Encyclopaedic Dictionary, by Hunter Robert\footnote{\href{https://babel.hathitrust.org/cgi/pt?id=mdp.39015050663213&view=1up&seq=780}{Hunter Robert, The Encyclopaedic Dictionary} - ‘\textit{personality}’}—dictionary owned by Ellen White. The use of these definitions can be seen from the articles written on the \emcap{personality of God}.


Termin ‘\textit{osobowość}’ ma dwa różne zastosowania, a najczęstsza definicja w codziennym użyciu dotyczy psychologii. ‘\textit{Osobowość}’ jest definiowana jako “\textit{charakterystyczny zestaw zachowań, procesów poznawczych i wzorców emocjonalnych, które ewoluują z czynników biologicznych i środowiskowych}”\footnote{Wikipedia Contributors. “Personality.” Wikipedia, Wikimedia Foundation, 19 Apr. 2019, \href{https://en.wikipedia.org/wiki/Personality}{en.wikipedia.org/wiki/Personality}.}. Niezwykle ważne jest, aby zdać sobie sprawę, że gdy zajmujemy się filarem naszej wiary—“\textit{osobowością Boga}”—nie poruszamy się w dziedzinie psychologii. Dokładne zastosowanie słowa ‘\textit{osobowość}’ w doktrynie o \emcap{osobowości Boga} znajduje się w Słowniku Merriam-Webster: “\textit{właściwość lub stan jako osoby}”\footnote{\href{https://www.merriam-webster.com/dictionary/personality}{Merriam-Webster Dictionary} - ‘\textit{personality}’}. Według Słownika Merriam-Webster, ta definicja jest używana od XV wieku\footnote{Zobacz “\href{https://www.merriam-webster.com/dictionary/personality\#word-history}{First known use}” słowa ‘personality’ w Słowniku Merriam Webster}. W wydaniu Słownika Merriam Webster z 1828 roku czytamy definicję słowa ‘\textit{osobowość}’ jako: “\textit{to, co stanowi jednostkę odrębną osobą}”\footnote{\href{https://archive.org/details/americandictiona02websrich/page/272/mode/2up}{Merriam-Webster Dictionary, 1828 edition} - ‘\textit{personality}’} \footnote{\href{https://archive.org/details/websterscomplete00webs/page/974/mode/2up}{Wydanie Słownika Merriam-Webster z 1886 roku} definiuje słowo ‘\textit{personality}’ jako: “\textit{to, co stanowi lub odnosi się do osoby}”}. Obie definicje znajdują się w The Encyclopaedic Dictionary autorstwa Huntera Roberta\footnote{\href{https://babel.hathitrust.org/cgi/pt?id=mdp.39015050663213&view=1up&seq=780}{Hunter Robert, The Encyclopaedic Dictionary} - ‘\textit{personality}’}—słowniku należącym do Ellen White. Użycie tych definicji można zobaczyć w artykułach napisanych na temat \emcap{osobowości Boga}.


In 1903, when Sister White wrote to Dr. Kellogg, \egwinline{I have \textbf{ever }had the same testimony to bear which I now bear \textbf{regarding the personality of God}}[Lt253-1903.9; 1903][https://egwwritings.org/?ref=en\_Lt253-1903.9&para=9980.15], she recalled her vision when she saw the Father and the Son.


W 1903 roku, kiedy Siostra White napisała do dr. Kellogga, \egwinline{Zawsze \textbf{miałam }to samo świadectwo do złożenia, które teraz składam \textbf{odnośnie osobowości Boga}}[Lt253-1903.9; 1903][https://egwwritings.org/?ref=en\_Lt253-1903.9&para=9980.15], przypomniała sobie swoją wizję, w której widziała Ojca i Syna.


\egw{‘I have often seen the lovely Jesus, that\textbf{ He is a person}.\textbf{ I asked Him if His Father was a person, }and \textbf{had \underline{a form} like Himself}. Said Jesus, ‘\textbf{I am the express image of My Father’s person!}’ [Hebrews 1:3.]}[Lt253-1903.12; 1903][https://egwwritings.org/?ref=en\_Lt253-1903.12&para=9980.18]


\egw{‘Często widziałam umiłowanego Jezusa, że\textbf{ jest On osobą}.\textbf{ Zapytałam Go, czy Jego Ojciec jest osobą, }i \textbf{czy ma \underline{postać} podobną do Niego}. Jezus powiedział: ‘\textbf{Jestem wyrazem istoty Mojego Ojca!}’ [Hebrajczyków 1:3.]}[Lt253-1903.12; 1903][https://egwwritings.org/?ref=en\_Lt253-1903.12&para=9980.18]


The quality or state that Sister White defines God to be a person is to have \textit{a form}—\textit{a physical appearance}. Dr. Kellogg follows the same application of the word \textit{‘personality’}, although through speculation.


Właściwość lub stan, który Siostra White definiuje, że Bóg jest osobą, to posiadanie \textit{postaci}—\textit{fizycznego wyglądu}. Dr Kellogg stosuje to samo zastosowanie słowa \textit{‘osobowość’}, choć poprzez spekulację.


\others{The fact that God is so great that we cannot form a clear mental picture of \textbf{his physical appearance} need not lessen in our minds the reality of \textbf{His personality}...}[John H. Kellogg, The Living Temple, p. 31][https://archive.org/details/J.H.Kellogg.TheLivingTemple1903/page/n31/mode/2up]


\others{Fakt, że Bóg jest tak wielki, że nie możemy stworzyć jasnego mentalnego obrazu \textbf{jego fizycznego wyglądu}, nie musi umniejszać w naszych umysłach rzeczywistości \textbf{Jego osobowości}...}[John H. Kellogg, The Living Temple, s. 31][https://archive.org/details/J.H.Kellogg.TheLivingTemple1903/page/n31/mode/2up]


As we have previously seen, our Adventist pioneers also pinpointed the physical appearance as a quality that makes God a person. James White wrote, \others{Those who deny \textbf{the personality of God}, say that ‘image’ here does not mean \textbf{physical form}, but moral image...}[James S. White, PERGO 1.1; 1861][https://egwwritings.org/?ref=en\_PERGO.1.1&para=1471.3]. J. B. Frisbie wrote, \others{Some seem to suppose it argues against \textbf{the personality of God}, because he is a Spirit, and say that he is without \textbf{body, or parts}...}[\href{https://documents.adventistarchives.org/Periodicals/RH/RH18540307-V05-07.pdf}{Adventist Review and Sabbath Herald, March 7, 1854}, J. B. Frisbie, “The Seventh-Day Sabbath Not Abolished”, p. 50]


Jak wcześniej widzieliśmy, nasi adwentystyczni pionierzy również wskazywali na fizyczny wygląd jako cechę, która czyni Boga osobą. James White napisał, \others{Ci, którzy zaprzeczają \textbf{osobowości Boga}, mówią, że ‘obraz’ tutaj nie oznacza \textbf{fizycznej formy}, ale obraz moralny...}[James S. White, PERGO 1.1; 1861][https://egwwritings.org/?ref=en\_PERGO.1.1&para=1471.3]. J. B. Frisbie napisał, \others{Niektórzy wydają się zakładać, że to argumentuje przeciwko \textbf{osobowości Boga}, ponieważ jest On Duchem, i mówią, że jest On bez \textbf{ciała lub części}...}[\href{https://documents.adventistarchives.org/Periodicals/RH/RH18540307-V05-07.pdf}{Adventist Review and Sabbath Herald, 7 marca 1854}, J. B. Frisbie, “The Seventh-Day Sabbath Not Abolished”, s. 50]


In light of the facts, we recognize the application of the word ‘\textit{personality}’. When the subject on the \emcap{personality of God} is presented in its connection to the Trinity doctrine, there is often a tendency to alter the meaning of the word ‘\textit{personality}’. It is also important to mention that the subject on the \emcap{personality of God} deals with the personality of the Father. This is clearly seen from the presented data.


W świetle faktów, rozpoznajemy zastosowanie słowa ‘\textit{osobowość}’. Kiedy temat \emcap{osobowości Boga} jest przedstawiany w związku z doktryną o Trójcy, często istnieje tendencja do zmiany znaczenia słowa ‘\textit{osobowość}’. Ważne jest również, aby wspomnieć, że temat \emcap{osobowości Boga} dotyczy osobowości Ojca. Jest to wyraźnie widoczne z przedstawionych danych.


\subsection*{Step 5: In examining the Kellogg crisis, shifting the main focus from the personality of God to pantheism}


\subsection*{Krok 5: W badaniu kryzysu Kellogga, przesunięcie głównego punktu uwagi z osobowości Boga na panteizm}


The data on the Kellogg crisis, in connection with the Trinity doctrine, is overwhelming if the \emcap{personality of God} is accounted for in the equation. The only way to not connect the dots is to ignore the \emcap{personality of God} and shift focus to pantheism exclusively. We do not deny the pantheistic nature of Kellogg's controversy. We believe that the pantheistic nature of Kellogg's controversy cannot be rightly understood if it is not examined in the true light of the \emcap{personality of God}. But, unfortunately, in examination of the Kellogg crisis, the attention that pantheism receives supersedes the examination of the truth on the \emcap{personality of God}.


Dane dotyczące kryzysu Kellogga, w związku z doktryną o Trójcy, są przytłaczające, jeśli \emcap{osobowość Boga} jest uwzględniona w równaniu. Jedynym sposobem, aby nie połączyć kropek, jest zignorowanie \emcap{osobowości Boga} i przesunięcie uwagi wyłącznie na panteizm. Nie zaprzeczamy panteistycznej naturze sporu Kellogga. Wierzymy, że panteistyczna natura sporu Kellogga nie może być właściwie zrozumiana, jeśli nie jest badana w prawdziwym świetle \emcap{osobowości Boga}. Niestety jednak, w badaniu kryzysu Kellogga, uwaga, jaką otrzymuje panteizm, przewyższa badanie prawdy o \emcap{osobowości Boga}.


You can do a search of Ellen White’s compilations to see just how much more attention pantheism received than the \emcap{personality of God}. If you were to search her writings for ‘pantheism’ or ‘pantheistic’, excluding the compilations after her death, you would find 36 occurrences. Among them are several repetitive quotations that Sister White copied from one letter to another, or to the special testimonies for the church. If you were to count the distinct occurrences you would only find 12 distinct quotations containing words like ‘\textit{pantheism}’ or ‘\textit{pantheistic}’\footnote{On the \href{https://egwwritings.org/}{https://egwwritings.org/} search bar, input the word “\textit{pantheis*} ”; this will include all words beginning with the ‘\textit{pantheis...}’, (including ‘\textit{pantheism}’ and ‘\textit{pantheistic}’). The results can be compared in subsetting the corpus of Ellen White writings by including or excluding compilations after her death. This option is available in the dropdown menu under the search bar.}. If you conducted the same search, but only in the compilations issued after her death, you would find 140 occurrences! All of these fall into one of the twelve distinct instances Sister White wrote on the subject of pantheism.


Możesz przeszukać kompilacje Ellen White, aby zobaczyć, o ile więcej uwagi poświęcono panteizmowi niż \emcap{osobowości Boga}. Jeśli przeszukałbyś jej pisma pod kątem słów ‘panteizm’ lub ‘panteistyczny’, wykluczając kompilacje po jej śmierci, znalazłbyś 36 wystąpień. Wśród nich jest kilka powtarzających się cytatów, które Siostra White kopiowała z jednego listu do drugiego lub do specjalnych świadectw dla kościoła. Gdybyś policzył odrębne wystąpienia, znalazłbyś tylko 12 odrębnych cytatów zawierających słowa takie jak ‘\textit{panteizm}’ lub ‘\textit{panteistyczny}’\footnote{W pasku wyszukiwania na stronie \href{https://egwwritings.org/}{https://egwwritings.org/} wpisz słowo “\textit{pantheis*} “; obejmie to wszystkie słowa zaczynające się od ‘\textit{pantheis...}’, (w tym ‘\textit{panteizm}’ i ‘\textit{panteistyczny}’). Wyniki można porównać, dzieląc korpus pism Ellen White poprzez włączenie lub wykluczenie kompilacji po jej śmierci. Ta opcja jest dostępna w menu rozwijanym pod paskiem wyszukiwania.}. Jeśli przeprowadziłbyś to samo wyszukiwanie, ale tylko w kompilacjach wydanych po jej śmierci, znalazłbyś 140 wystąpień! Wszystkie one należą do jednego z dwunastu odrębnych przypadków, w których Siostra White pisała na temat panteizmu.


In a search of Ellen White writings on the phrase “\textit{personality of God}”, excluding the compilations after her death, you would find 58 occurrences. Among them are also several repetitive quotations that Sister White copied to several different letters and to the testimonies for the church. Yet, if you were to search this phrase within the compilations that were issued after her death you would only find 52 occurrences.


W wyszukiwaniu pism Ellen White na temat frazy “\textit{osobowość Boga}”, z wyłączeniem kompilacji po jej śmierci, znalazłbyś 58 wystąpień. Wśród nich również znajduje się kilka powtarzających się cytatów, które Siostra White kopiowała do kilku różnych listów i do świadectw dla kościoła. Jednak gdybyś przeszukał tę frazę w kompilacjach, które zostały wydane po jej śmierci, znalazłbyś tylko 52 wystąpienia.


These simple statistics demonstrate the focus of the compilators after the death of Sister White. Such emphasis on pantheism changed our public opinion regarding Kellogg’s crisis. Forty-three, out of fifty-eight, quotations on the phrase “\textit{personality of God}” are found in letters and manuscripts, available to the public from 2015 onwards. This means that three quarters (\textit{74 percent}) of the quotation regarding the \emcap{personality of God}, prior to 2015, was not available to the public. Prior to 2015 we did not have much available data to study Kellogg's crisis in light of the \emcap{personality of God} and in its context.


Te proste statystyki pokazują, na czym skupiali się kompilatorzy po śmierci Siostry White. Taki nacisk na panteizm zmienił naszą opinię publiczną dotyczącą kryzysu Kellogga. Czterdzieści trzy z pięćdziesięciu ośmiu cytatów na temat frazy “\textit{osobowość Boga}” znajduje się w listach i rękopisach, dostępnych publicznie od 2015 roku. Oznacza to, że trzy czwarte (\textit{74 procent}) cytatów dotyczących \emcap{osobowości Boga}, przed 2015 rokiem, nie było dostępnych publicznie. Przed 2015 rokiem nie mieliśmy wielu dostępnych danych, aby badać kryzys Kellogga w świetle \emcap{osobowości Boga} i w jego kontekście.


% Steps to Omega

\begin{titledpoem}
    
    \stanza{
        On pillars now, the shadows cast— \\
        A truth forsaken, from the past. \\
        In steps they chart the silent drift, \\
        Five marks of change, through sacred rift.
    }

    \stanza{
        Denial blooms when once truth stood, \\
        Foundations are not understood, \\
        The fundamentals, once held dear \\
        Obscured, as new creeds appear.
    }

    \stanza{
        Prophetic warnings have been dimmed, \\
        Pioneers are shunned, old hymns are trimmed. \\
        The testimonies once rang out \\
        But now they’re often tinged with doubt.
    }

    \stanza{
        “God is a person” cast aside, \\
        And now His essence they deride. \\
        Forgotten pillar once was strong \\
        Now a new pillar, which is wrong!
    }

    \stanza{
        Scholars now twist the sacred term, \\
        Words redefined, they now affirm. \\
        Gone is the quest to see God’s face, \\
        Dim the desire for His embrace.
    }

    \stanza{
        The Kellogg crisis point is missed, \\
        The alpha given untrue twist \\
        And thus, the lessons are not learned \\
        The church toward omega turned.
    }

    \stanza{
        Confusion reigns, we can’t perceive \\
        It is not clear what we believe \\
        Our history has been revised \\
        We wanted truth, but then they lied.
    }
    
\end{titledpoem}

% \begin{titledpoem}

    \stanza{
        Dziś na filary padły cienie, \\
        A prawda idzie w zapomnienie. \\
        Przez kroków pięć w ciszy prowadzą, \\
        Aż rozłam wśród ludu wprowadzą.
    }

    \stanza{
        Fałsz kwitnie tam, gdzie prawda stała, \\
        Którą zrozumieć trzódka miała. \\
        A to, co fundamentem było, \\
        W wyznanie wiary się zmieniło.
    }

    \stanza{
        Prorocze słowa porzucone, \\
        Pionierzy też, pieśni zmienione. \\
        Świadectwa kiedyś tak dźwięczały, \\
        Lecz teraz respekt do nich mały.
    }

    \stanza{
        „Bóg jest osobą” odrzucono, \\
        Jego istotę znieważono. \\
        Prawdziwy filar zapomniany, \\
        A nowy, błędny jest nam dany.
    }

    \stanza{
        Uczeni przekręcają słowa, \\
        By zmieniła znaczenie mowa. \\
        Nikt nie chce widzieć Bożej twarzy, \\
        O Jego uścisku nie marzy.
    }

    \stanza{
        Kellogga kryzys przekręcany, \\
        Krok alfa nie jest zrozumiany. \\
        A że się prawdzie nie dowierza, \\
        To Kościół ku omedze zmierza.
    }

    \stanza{
        Wśród wiernych zamieszanie rządzi \\
        I wielu dziś w wierzeniach błądzi. \\
        Historię naszą przepisano, \\
        Miała być prawda, lecz skłamano.
    }

\end{titledpoem}

