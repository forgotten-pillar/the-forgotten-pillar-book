\qrchapter{https://forgottenpillar.com/rsc/en-fp-chapter7}{The authority of the Fundamental Principles} \label{chap:authority}


\qrchapter{https://forgottenpillar.com/rsc/en-fp-chapter7}{La autoridad de los Principios Fundamentales} \label{chap:authority}


In the 10th chapter of the Special Testimonies, we read how God established the foundation of our faith. Sister White used several different expressions for the foundation of our faith. Her references included: “\textit{a platform of eternal truth}, \textit{“pillars of our faith”}, \textit{“principles of truth”}, \textit{“principal points”}, \textit{“waymarks”}, and “\textit{foundation principles}—all of these refer to the \emcap{Fundamental Principles}. At the end of the chapter, she affirmed the will of God that \egwinline{He calls upon us to hold firmly, with the grip of faith, to \textbf{the fundamental principles} that are \textbf{based upon unquestionable \underline{authority}}.}[SpTB02 59.1; 1904][https://egwwritings.org/read?panels=p417.299]


En el capítulo 10 de los Testimonios Especiales, leemos cómo Dios estableció el fundamento de nuestra fe. La hermana White utilizó varias expresiones diferentes para referirse al fundamento de nuestra fe. Sus referencias incluyeron: “\textit{una plataforma de verdad eterna}, \textit{“pilares de nuestra fe”}, \textit{“principios de verdad”}, \textit{“puntos principales”}, \textit{“hitos”}, y “\textit{principios fundacionales}—todas ellas se refieren a los \emcap{Principios Fundamentales}. Al final del capítulo, afirmó la voluntad de Dios de que \egwinline{Él nos llama a aferrarnos firmemente, con la garra de la fe, a \textbf{los principios fundamentales} que se \textbf{basan en una \underline{autoridad} incuestionable}}.}[SpTB02 59.1; 1904][https://egwwritings.org/read?panels=p417.299]


The authority on which the \emcap{fundamental principles} are established is unquestionable. They were the result of deep, earnest study in the time of great disappointment, when \egwinline{\textbf{\underline{point by point}, has been sought out by prayerful study, and testified to by the \underline{miracle-working power of the Lord}}}\footnote{Ibid.}. \egwinline{\textbf{Thus \underline{the leading points of our faith} as we hold them today were firmly established}. \textbf{\underline{Point after point} was clearly defined, and all the brethren came into harmony}.}[Lt253-1903.4; 1903][https://egwwritings.org/read?panels=p14068.9980010]


La autoridad sobre la que se establecen los \emcap{principios fundamentales} es incuestionable. Fueron el resultado de un estudio profundo y serio en el tiempo de la gran decepción, cuando \egwinline{\textbf{\underline{punto por punto}, ha sido buscado por el estudio en oración, y testificado por el \underline{poder milagroso del Señor}}}\footnote{Ibid.}. \egwinline{\textbf{Así \underline{los puntos principales de nuestra fe} tal como los sostenemos hoy fueron firmemente establecidos}. \textbf{\underline{Punto tras punto} fue claramente definido, y todos los hermanos entraron en armonía}.}[Lt253-1903.4; 1903][https://egwwritings.org/read?panels=p14068.9980010]


They were the result of the earnest Bible studies of our pioneers, after the passing of time in 1844. As the Seventh-day Adventist movement progressed, there came a need for instituting the organization, which was realized in 1863. In 1872, the Seventh-day Adventist Church issued the document called “\textit{A Declaration of the Fundamental Principles, Taught and Practiced by the Seventh-Day Adventists}. This was the first written document declaring the \emcap{fundamental principles} as public statements of the Seventh-day Adventist faith. This document was the public synopsis of Seventh-day Adventist faith and it declared \others{what is, and has been, with great unanimity, held by} the Seventh-day Adventist people. It was written \others{to meet inquiries} as to what was believed by Seventh-day Adventists, \others{to correct false statements circulated} and to \others{remove erroneous impressions}[FP1872 3.1; 1872][https://egwwritings.org/read?panels=p928.8].


Fueron el resultado de los serios estudios bíblicos de nuestros pioneros, después del paso del tiempo en 1844. A medida que el movimiento adventista del séptimo día progresaba, surgió la necesidad de instituir la organización, lo cual se realizó en 1863. En 1872, la Iglesia Adventista del Séptimo Día publicó el documento llamado “\textit{Declaración de los Principios Fundamentales, Enseñados y Practicados por los Adventistas del Séptimo Día}. Este fue el primer documento escrito que declaraba los \emcap{principios fundamentales} como declaraciones públicas de la fe adventista del séptimo día. Este documento era la sinopsis pública de la fe adventista del séptimo día y declaraba \others{lo que es, y ha sido, con gran unanimidad, sostenido por} el pueblo adventista del séptimo día. Fue escrito \others{para responder a las preguntas} sobre lo que creían los adventistas del séptimo día, \others{para corregir las falsas declaraciones que circulaban} y para \others{eliminar las impresiones erróneas}[FP1872 3.1; 1872][https://egwwritings.org/read?panels=p928.8].


Today it is still debated who authored the synopsis because originally, in 1872, it was left anonymous. In 1874, James White issued it in Signs of the Times\footnote{\href{https://adventistdigitallibrary.org/adl-364148/signs-times-june-4-1874}{Signs of the Times, June 4, 1874}} and Uriah Smith in the Review and Herald\footnote{\href{http://documents.adventistarchives.org/Periodicals/RH/RH18741124-V44-22.pdf}{The Advent Review and Herald of the Sabbath, November 24, 1874}}—both signing with their own signatures. In 1889, Uriah Smith revised it by adding three points; it was issued in the Adventist Yearbook with his signature on it. Uriah Smith died in 1903 and all successive printings of the \emcap{Fundamental Principles} were printed under his name. They were printed in the Yearbooks—each year from 1905 until 1914\footnote{For more detailed timeline of Fundamental Principles, see \hyperref[appendix:timeline]{Appendix: Fundamental Principles - Timeline}}. Sister White died in 1915 and, for the next 17 years, the \emcap{fundamental principles} were not printed. Their next appearance was in the 1931 Yearbook when they received significant changes.


Hoy en día todavía se discute quién fue el autor de la sinopsis porque originalmente, en 1872, se dejó en el anonimato. En 1874, James White la publicó en Signs of the Times\footnote{\href{https://adventistdigitallibrary.org/adl-364148/signs-times-june-4-1874}{Signs of the Times, June 4, 1874}} y Uriah Smith en el Review and Herald\footnote{\href{http://documents.adventistarchives.org/Periodicals/RH/RH18741124-V44-22.pdf}{The Advent Review and Herald of the Sabbath, November 24, 1874}}—ambos firmando con sus propias firmas. En 1889, Uriah Smith lo revisó añadiendo tres puntos; se publicó en el Anuario Adventista con su firma. Uriah Smith murió en 1903 y todas las impresiones sucesivas de los \emcap{Principios Fundamentales} se imprimieron bajo su nombre. Se imprimieron en los Anuarios—cada año desde 1905 hasta 1914\footnote{Para una cronología más detallada de los Principios Fundamentales, véase \hyperref[appendix:timeline]{Apéndice: Principios Fundamentales - Cronología}}. La hermana White murió en 1915 y, durante los siguientes 17 años, los \emcap{principios fundamentales} no se imprimieron. Su siguiente aparición fue en el Anuario de 1931, cuando recibieron cambios significativos.


In 1971, LeRoy Froom wrote about a statement from 1872: \others{Though appearing anonymously, it was actually composed by Smith}[Edwin Froom, LeRoy. Movement of Destiny. 1971., p. 160]. Unfortunately, he didn’t provide any data to support his claim. It is unfortunate to see how pro-trinitarian scholars consider the \emcap{Fundamental Principles} to be of very little importance. Their true value is starkly diminished by attributing these beliefs to those of a small group of people, mostly to James White’s or Uriah Smith’s personal belief, rather than belief which was \others{with great unanimity, held by}[Preface of the Fundamental Principles 1872] the Seventh-day Adventist people. In 1958, Ministry Magazine described the \emcap{Fundamental Principles} as follows:


En 1971, LeRoy Froom escribió sobre una declaración de 1872: \others{Aunque aparece de forma anónima, en realidad fue compuesta por Smith}[Edwin Froom, LeRoy. Movement of Destiny. 1971., p. 160]. Desgraciadamente, no aportó ningún dato que apoyara su afirmación. Es lamentable ver cómo los académicos pro-trinitarios consideran que los \emcap{Principios Fundamentales} tienen muy poca importancia. Su verdadero valor se ve muy disminuido al atribuir estas creencias a las de un pequeño grupo de personas, principalmente a la creencia personal de James White o Uriah Smith, en lugar de una creencia que \others{con gran unanimidad, sostenía}[Preface of the Fundamental Principles 1872] el pueblo adventista del séptimo día. En 1958, la revista Ministry Magazine describió los \emcap{Principios Fundamentales} de la siguiente manera:


\others{It is true that in 1872 a ‘Declaration of the Fundamental Principles Taught and Practiced by Seventhday Adventists’ was printed, \textbf{but it was never adopted by the denomination and therefore cannot be considered official}. Evidently a small group, \textbf{perhaps even one or two, endeavored to put into words what they thought were the views of the entire church…}}[Ministry Magazine “\textit{Our Declaration of Fundamental Beliefs}”, January 1958, Roy Anderson, J. Arthur Buckwalter, Louise Kleuser, Earl Cleveland and Walter Schubert]


\others{Es cierto que en 1872 se imprimió una ‘Declaración de los Principios Fundamentales Enseñados y Practicados por los Adventistas del Séptimo Día’, \textbf{pero nunca fue adoptada por la denominación y, por tanto, no puede considerarse oficial}. Evidentemente, un pequeño grupo, \textbf{tal vez incluso uno o dos, se esforzaron por poner en palabras lo que pensaban que eran los puntos de vista de toda la iglesia...}}[Ministry Magazine “\textit{Our Declaration of Fundamental Beliefs}”, January 1958, Roy Anderson, J. Arthur Buckwalter, Louise Kleuser, Earl Cleveland and Walter Schubert]


Problematically, there is no evidence to support the claim that the \emcap{Fundamental Principles} were not the representation of faith of the whole body. We certainly know that Sister White endorsed them and, from her influence alone, we know that these beliefs were indeed accepted by the denomination—this is in addition to the fact that they were printed multiple times over the course of 42 years, during the life of Ellen White.


Problemáticamente, no hay pruebas que apoyen la afirmación de que los \emcap{Principios Fundamentales} no eran la representación de la fe de todo el cuerpo. Ciertamente sabemos que la hermana White los respaldó y, solo por su influencia, sabemos que estas creencias fueron efectivamente aceptadas por la denominación—además del hecho de que se imprimieron varias veces en el transcurso de 42 años, durante la vida de Ellen White.


But there should be no controversy over the authorship of the \emcap{Fundamental Principles}. We have a quotation from Sister White about who authored them. When speaking of Uriah Smith, Sister White wrote:


Pero no debería haber controversia sobre la autoría de los \emcap{Principios Fundamentales}. Tenemos una cita de la hermana White sobre quién fue su autor. Al hablar de Uriah Smith, la hermana White escribió:


\egw{\textbf{Brother Smith was with us in the rise of this work. He understands how \underline{we—my husband and myself}—have carried the work forward and upward step by step and have borne the hardships, the poverty, and the want of means. With us were those early workers. Elder Smith, especially, was one with my husband in his early manhood}. …}[Ms54-1890.6; 1890][https://egwwritings.org/read?panels=p7213.15]


\egw{\textbf{El hermano Smith estuvo con nosotros en el surgimiento de esta obra. Él comprende cómo \underline{nosotros—mi esposo y yo}—hemos llevado la obra adelante y hacia arriba paso a paso y hemos soportado las dificultades, la pobreza y la falta de medios. Con nosotros estaban aquellos primeros obreros. El élder Smith, especialmente, fue uno con mi esposo en su temprana edad}. …}[Ms54-1890.6; 1890][https://egwwritings.org/read?panels=p7213.15]


\egwnogap{\textbf{\underline{We have stood shoulder to shoulder with Elder Smith in this work while the Lord was laying the foundation principles}}. \textbf{We had to work constantly against one-idea men}, who thought correct business relations in regard to the work which had to be done were an evidence of worldly-mindedness, and the cranky ones who would present themselves as capable of bearing responsibilities, but could not be trusted to be connected with the work lest they swing it in wrong lines. \textbf{Step after step has had to be taken, \underline{not after the wisdom of men} but after the wisdom and instruction of One who is too wise to err and too good to do us harm}. \textbf{There have been so many elements that would have to be proved and tried. I thank the Lord that Elders Smith, Amadon, and Batchellor still live. They composed the members of our family in the most trying parts of our history}.}[Ms54-1890.7; 1890][https://egwwritings.org/read?panels=p7213.16]


\egwnogap{\textbf{\underline{Hemos permanecido hombro con hombro con el anciano Smith en esta obra mientras el Señor establecía los principios fundamentales}}. \textbf{Tuvimos que trabajar constantemente contra los hombres de una sola idea}, que pensaban que las relaciones comerciales correctas con respecto a la obra que había que hacer eran una evidencia de mentalidad mundana, y los malhumorados que se presentaban como capaces de asumir responsabilidades, pero en los que no se podía confiar para que se relacionaran con la obra, a fin de que no la hicieran girar en líneas equivocadas. \textbf{Ha habido que dar un paso tras otro, \underline{no según la sabiduría de los hombres} sino según la sabiduría y la instrucción de Uno que es demasiado sabio para equivocarse y demasiado bueno para hacernos daño}. \textbf{Ha habido muchos elementos que tendrían que ser probados y comprobados. Doy gracias al Señor porque los ancianos Smith, Amadon y Batchellor aún viven. Ellos compusieron los miembros de nuestra familia en las partes más difíciles de nuestra historia}.}[Ms54-1890.7; 1890][https://egwwritings.org/read?panels=p7213.16]


According to this quotation, who laid down the foundation principles?


Según esta cita, ¿quién estableció los principios fundamentales?


\egwinline{\textbf{\underline{We have stood shoulder to shoulder with Elder Smith in this work while the Lord was laying the foundation principles}}.} \textbf{It was the Lord}! But who wrote them down as a declaration of our faith? It was Elder Smith with James White and Sister White; we see that where Sister White says\egwinline{\textbf{we} have stood shoulder to shoulder with Elder Smith}. This \textit{‘we’} is explained in the previous paragraph: \egwinline{He \normaltext{[Elder Smith]} understands how\textbf{ we—my husband and myself}—have carried the work forward}. With this quotation, Sister White was clearly involved when the Lord was laying the \emcap{Fundamental Principles}.


\egwinline{\textbf{\underline{Hemos permanecido hombro con hombro con el anciano Smith en esta obra mientras el Señor establecía los principios fundamentales}}.} \textbf{¡Fue el Señor!} Pero ¿quién los escribió como una declaración de nuestra fe? Fue el anciano Smith con Jaime White y la hermana White; lo vemos cuando la hermana White dice\egwinline{\textbf{nosotros} hemos permanecido hombro con hombro con el anciano Smith}. Este \textit{‘nosotros’} se explica en el párrafo anterior: \egwinline{Él \normaltext{[el anciano Smith]} comprende cómo\textbf{ nosotros—mi esposo y yo}—hemos llevado adelante la obra}. Con esta cita, la hermana White estaba claramente involucrada cuando el Señor estaba estableciendo los \emcap{Principios Fundamentales}.


It is true that the Declaration of the \emcap{Fundamental Principles} was written by a small group of people, namely Elder Smith, James White and Ellen White, but they endeavored to put into words what was the true view of the entire church body. They accurately represented the \emcap{fundamental principles}—the truths received in the beginning of our work. If that were not the case, then this declaration is the very opposite of what it claims to be. They were written \others{to meet inquiries} as to what was believed by Seventh-day Adventists, \others{to correct false statements circulated} and to \others{remove erroneous impressions.}[FP1872 3.1; 1872][https://egwwritings.org/read?panels=p928.8] If this document misrepresented the Adventist position, why was its continual reprinting, over the course of 42 years, permitted? It was reprinted until the death of Ellen White. If this document misrepresented the church’s position, wouldn’t Ellen White have raised her voice against it? She always raised her voice against the misrepresentation of the Seventh-day Adventist position, as she did with D. M. Canright and Dr. Kellogg. If the \emcap{Fundamental Principles} were misrepresenting the Seventh-day Adventist position, then all subsequent reprinting should be attributed to a conspiracy theory. That would be the greatest conspiracy theory within the Seventh-day Adventist Church. Ever. The harmony between the writings of Ellen White, Adventist pioneers, and the claims made in the Declaration of the \emcap{Fundamental Principles}, testify of the fact that this declaration is an accurate \others{summary of the principal features of} Seventh-day Adventist \others{faith, upon which there is, so far as we know, entire unanimity throughout the body}[The preface of the Fundamental Principles 1889].


Es cierto que la Declaración de los \emcap{Principios Fundamentales} fue escrita por un pequeño grupo de personas, a saber, el anciano Smith, James White y Ellen White, pero se esforzaron por poner en palabras lo que era el verdadero punto de vista de todo el cuerpo de la iglesia. Representaron con exactitud los \emcap{principios fundamentales}—las verdades recibidas en el comienzo de nuestra obra. Si ese no fuera el caso, entonces esta declaración es todo lo contrario de lo que pretende ser. Fueron escritas \others{para responder a las preguntas} sobre lo que creían los adventistas del séptimo día, \others{para corregir las falsas declaraciones que circulaban} y para \others{eliminar las impresiones erróneas.}[FP1872 3.1; 1872][https://egwwritings.org/read?panels=p928.8] Si este documento tergiversaba la posición adventista, ¿por qué se permitió su continua reimpresión, a lo largo de 42 años? Se reimprimió hasta la muerte de Ellen White. Si este documento tergiversaba la posición de la iglesia, ¿no habría levantado Ellen White su voz contra él? Siempre alzó la voz contra la tergiversación de la posición adventista del séptimo día, como hizo con D. M. Canright y el Dr. Kellogg. Si los \emcap{Principios Fundamentales} estaban tergiversando la posición adventista del séptimo día, entonces todas las reimpresiones sucesivas deberían atribuirse a una teoría conspiratoria. Esa sería la mayor teoría conspiratoria dentro de la Iglesia Adventista del Séptimo Día. Jamás. La armonía entre los escritos de Ellen White, los pioneros adventistas, y las afirmaciones hechas en la Declaración de los \emcap{Principios Fundamentales}, testifican el hecho de que esta declaración es un \others{resumen exacto de los rasgos principales de la fe} adventista del séptimo día, \others{sobre la cual hay, hasta donde sabemos, una completa unanimidad en todo el cuerpo}[El prefacio de los Principios Fundamentales 1889].


With the death of Sister White in 1915, printing of the \emcap{Fundamental Principles} ceased. From 1915 onward, the Yearbook did not print any statement of belief until 1931. At this time, the \emcap{Fundamental Principles} received substantial changes. For the first time, the Trinity was introduced to the \emcap{fundamental principles}. In points’ 2 and 3 we read:


Con la muerte de la hermana White en 1915, se dejó de imprimir los \emcap{Principios Fundamentales}. A partir de 1915, el Anuario no imprimió ninguna declaración de creencias hasta 1931. En esta época, los \emcap{Principios Fundamentales} recibieron cambios sustanciales. Por primera vez, se introdujo la Trinidad en los \emcap{principios fundamentales}. En los puntos 2 y 3 se lee:


\others{2. \textbf{That the Godhead, or Trinity, consists of the Eternal Father, a \underline{personal, spiritual Being}}, omnipotent, \textbf{\underline{omnipresent}}, omniscient, infinite in wisdom and love; \textbf{the Lord Jesus Christ, the Son of the Eternal Father}, \textbf{through whom all things were created} and through whom the salvation of the redeemed hosts will be accomplished; \textbf{the Holy Spirit, the third person of the Godhead}, the great regenerating power in the work of redemption. Matt. 28:19}


\others{2. \textbf{Que la Divinidad, o Trinidad, consiste en el Padre Eterno, un \underline{Ser personal y espiritual}}, omnipotente, \textbf{\underline{omnipresente}}, omnisciente, infinito en sabiduría y amor; \textbf{el Señor Jesucristo, el Hijo del Padre Eterno}, \textbf{por medio del cual fueron creadas todas las cosas} y por medio del cual se cumplirá la salvación de las huestes redimidas; \textbf{el Espíritu Santo, la tercera persona de la Divinidad}, el gran poder regenerador en la obra de la redención. Mateo 28:19}


\others{3. \textbf{That Jesus Christ is very God, being of the same nature and essence as the Eternal Father}…}[Yearbook of the Seventh-day Adventist Denomination, 1931, page. 377][https://static1.squarespace.com/static/554c4998e4b04e89ea0c4073/t/59d17eec12abd9c6194cd26d/1506901758727/SDA-YB1931-22+\%28P.+377-380\%29.pdf]


\others{3. \textbf{Que Jesucristo es muy Dios, siendo de la misma naturaleza y esencia que el Padre Eterno}…}[Anuario de la Denominación Adventista del Séptimo Día, 1931, página 377][https://static1.squarespace.com/static/554c4998e4b04e89ea0c4073/t/59d17eec12abd9c6194cd26d/1506901758727/SDA-YB1931-22+\%28P.+377-380\%29.pdf]


This change, in favor of the Trinity, appeared sixteen years after the death of Sister White. A comparison of this statement with the original \emcap{Fundamental Principles} presents several striking differences. The Father is still a personal, spiritual Being, the creator of all things, but is not addressed as “\textit{one God}” any longer. Jesus Christ is still the Son of the Eternal Father, through whom the Father created all things; Jesus is, also, of the very same nature and essence of the Father. Although these were the same terms to describe the doctrine on the \emcap{personality of God} in the original \emcap{Fundamental Principles}, we ask about the meaning of the term “\textit{personal, spiritual being}” applied to the Father, if He is, by new statement, omnipresent by Himself? The Holy Spirit is not an instrument, or means of the Father’s omnipresence anymore. Although this statement uses similar rhetoric of the original \emcap{Fundamental Principles}, it steps away from the original doctrine on the presence and the \emcap{personality of God}.


Este cambio, a favor de la Trinidad, apareció dieciséis años después de la muerte de la hermana White. Una comparación de esta declaración con los \emcap{Principios Fundamentales} originales presenta varias diferencias sorprendentes. El Padre sigue siendo un Ser personal y espiritual, el creador de todas las cosas, pero ya no se le llama “\textit{un Dios}”. Jesucristo sigue siendo el Hijo del Padre Eterno, a través del cual el Padre creó todas las cosas; Jesús es, además, de la misma naturaleza y esencia del Padre. Aunque estos fueron los mismos términos para describir la doctrina sobre la \emcap{personalidad de Dios} en los \emcap{Principios Fundamentales} originales, nos preguntamos sobre el significado del término “\textit{Ser personal y espiritual}” aplicado al Padre, si Él es, según la nueva declaración, omnipresente por sí mismo. El Espíritu Santo ya no es un instrumento o medio de la omnipresencia del Padre. Aunque esta declaración utiliza una retórica similar a la de los \emcap{Principios Fundamentales} originales, se aleja de la doctrina original sobre la presencia y la \emcap{personalidad de Dios}.


According to LeRoy Froom, this statement was written entirely by Francis Wilcox, with the approval of three other brothers (C.H. Watson, M.E. Kern and E.R. Palmer).\footnote{Edwin Froom, LeRoy. Movement of Destiny. 1971., p. 411, 413, 414} In the unpublished paper of \textit{The Seventh-day Adventist Church in Mission: 1919-1979}, we read how Elder Wilcox made this statement contrary to the belief of the church body and published it without their approval.


Según LeRoy Froom, esta declaración fue escrita enteramente por Francis Wilcox, con la aprobación de otros tres hermanos (C.H. Watson, M.E. Kern y E.R. Palmer).\footnote{Edwin Froom, LeRoy. Movement of Destiny. 1971., p. 411, 413, 414} En el documento inédito de \textit{The Seventh-day Adventist Church in Mission: 1919-1979}, leemos cómo el anciano Wilcox hizo esta declaración contraria a la creencia del cuerpo de la iglesia y la publicó sin su aprobación.


\others{\textbf{Realizing that the General Conference Committee or any other church body would never accept the document in the form in which it was written}, Elder Wilcox, with full knowledge of the group \normaltext{[C.H. Watson, M.E. Kern and E.R. Palmer]}, handed the Statement directly to Edson Rogers, the General Conference statistician, who published it in the 1931 edition of the Yearbook, where it has appeared ever since. It was without the official approval of the General Conference Committee, therefore, and without any formal denominational adoption, that Elder Wilcox's statement became the accepted declaration of our faith.}[Dwyer, Bonnie. “A New Statement of Fundamental Beliefs (1980) - Spectrum Magazine.” \textit{Spectrum Magazine}, 7 June 2009, \href{https://spectrummagazine.org/news/new-statement-fundamental-beliefs-1980/}{spectrummagazine.org/news/new-statement-fundamental-beliefs-1980/}. Accessed 30 Jan. 2025.]


\others{\textbf{Al darse cuenta de que el Comité de la Conferencia General o cualquier otro cuerpo de la iglesia nunca aceptaría el documento en la forma en que fue escrito}, el anciano Wilcox, con pleno conocimiento del grupo \normaltext{[C.H. Watson, M.E. Kern y E.R. Palmer]}, entregó la Declaración directamente a Edson Rogers, el estadístico de la Conferencia General, quien la publicó en la edición de 1931 del Anuario, donde ha aparecido desde entonces. Fue sin la aprobación oficial del Comité de la Conferencia General, por lo tanto, y sin ninguna adopción denominacional formal, que la declaración del anciano Wilcox se convirtió en la declaración aceptada de nuestra fe.}[Dwyer, Bonnie. “A New Statement of Fundamental Beliefs (1980) - Spectrum Magazine.” \textit{Spectrum Magazine}, 7 June 2009, \href{https://spectrummagazine.org/news/new-statement-fundamental-beliefs-1980/}{spectrummagazine.org/news/new-statement-fundamental-beliefs-1980/}. Accessed 30 Jan. 2025.]


In 1980, the final change to the public synopsis of the Seventh-day Adventist faith was made. The General Conference voted to adopt today’s official statement:


En 1980, se realizó el cambio final en la sinopsis pública de la fe adventista del séptimo día. La Conferencia General votó para adoptar la declaración oficial actual:


\others{\textbf{There is one God: Father, Son and Holy Spirit, a unity of three coeternal Persons}. God is immortal, all-powerful, all-knowing, above all, and \textbf{ever present}. He is infinite and beyond human comprehension, yet known through His self-revelation. He is forever worthy of worship, adoration, and service by the whole creation.}[Seventh-day Adventists Believe: A Biblical Exposition of 27 Fundamental Doctrines, p. 16]


\others{\textbf{Hay un solo Dios: Padre, Hijo y Espíritu Santo, una unidad de tres Personas coeternas}. Dios es inmortal, todopoderoso, omnisciente, por encima de todo y \textbf{siempre presente}. Es infinito y está más allá de la comprensión humana, pero se conoce a través de su autorrevelación. Él es siempre digno de la adoración, el culto y el servicio de toda la creación.}[Seventh-day Adventists Believe: A Biblical Exposition of 27 Fundamental Doctrines, p. 16]


In this brief historical overview we see that the 1931 statement is a “middle step” between the original Adventist belief to the full trinitarian belief.


En este breve resumen histórico vemos que la declaración de 1931 es un “paso intermedio” entre la creencia adventista original y la creencia trinitaria completa.


The change in our beliefs has occurred over time with many discussions. Our Adventist history has left a trace of these changes. If we are honest truth seekers we should study this matter in detail. Can we see, in our Adventist history, why we have left the first point of the \emcap{Fundamental Principles} in favor of the Trinity doctrine? Most certainly! In the following studies we will look at some of the historical documents that show why we have moved from the first point of the \emcap{Fundamental Principles}, held in the early years, to accept the Trinity doctrine. During these studies, we bid you to prayerfully evaluate the changes with your own beliefs.


El cambio en nuestras creencias ha ocurrido a lo largo del tiempo con muchas discusiones. Nuestra historia adventista ha dejado un rastro de estos cambios. Si somos honestos buscadores de la verdad debemos estudiar este asunto en detalle. ¿Podemos ver, en nuestra historia adventista, por qué hemos dejado el primer punto de los \emcap{Principios Fundamentales} a favor de la doctrina trinitaria? ¡Ciertamente! En los siguientes estudios veremos algunos de los documentos históricos que muestran por qué hemos pasado del primer punto de los \emcap{Principios Fundamentales}, sostenido en los primeros años, a aceptar la doctrina trinitaria. Durante estos estudios, le pedimos que evalúe en oración los cambios con sus propias creencias.


% The authority of the Fundamental Principles

\begin{titledpoem}
\stanza{
    Our principles of faith stand firm and true, \\
    Established by the Lord through chosen few. \\
    A platform built on unquestionable might, \\
    Waymarks that guide us through the darkest night.
}

\stanza{
    The pioneers sought truth with earnest prayer, \\
    Point after point laid down with godly care. \\
    Yet modern minds have altered what was clear, \\
    Changing foundations held for many a year.
}

\stanza{
    Return, O church, to truths that God ordained, \\
    Not to revised beliefs that men have claimed. \\
    Stand firm upon the rock that cannot move, \\
    In Fundamental Principles approved.
}

\stanza{
    Let not new scholars lead your faith astray, \\
    From paths our founders walked in heaven's way. \\
    The Lord Himself laid down these truths of old, \\
    Embrace their power with faith both strong and bold.
}
\end{titledpoem}


% The authority of the Fundamental Principles

\begin{titledpoem}
\stanza{
    Our principles of faith stand firm and true, \\
    Established by the Lord through chosen few. \\
    A platform built on unquestionable might, \\
    Waymarks that guide us through the darkest night.
}

\stanza{
    The pioneers sought truth with earnest prayer, \\
    Point after point laid down with godly care. \\
    Yet modern minds have altered what was clear, \\
    Changing foundations held for many a year.
}

\stanza{
    Return, O church, to truths that God ordained, \\
    Not to revised beliefs that men have claimed. \\
    Stand firm upon the rock that cannot move, \\
    In Fundamental Principles approved.
}

\stanza{
    Let not new scholars lead your faith astray, \\
    From paths our founders walked in heaven's way. \\
    The Lord Himself laid down these truths of old, \\
    Embrace their power with faith both strong and bold.
}
\end{titledpoem}
