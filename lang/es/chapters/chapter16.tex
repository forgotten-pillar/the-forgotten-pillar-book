\qrchapter{https://forgottenpillar.com/rsc/en-fp-chapter16}{Dr. Kellogg and pantheism}


\qrchapter{https://forgottenpillar.com/rsc/en-fp-chapter16}{Dr. Kellogg y el panteísmo}


From her personal diary, on January 5, 1902, Sister White wrote that Kellogg’s \egwinline{science of God in nature is \textbf{true}}.


De su diario personal, el 5 de enero de 1902, la hermana White escribió que la \egwinline{ciencia de Dios en la naturaleza} de Kellogg \egwinline{es \textbf{verdadera}}.


\egw{I am having things presented to me that worry my mind. Dr. Kellogg is traveling the same road that he did soon after taking up his responsibilities in the Sanitarium. \textbf{Human science is a lie in regard to God not having a personality}. I know this is a falsehood, and yet if we can in any way help the doctor we must try to do this. What can be said? There is such an exaltation given him that he is about to topple over the precipice. What can any of us do? The Lord alone can save Dr. Kellogg. \textbf{\underline{His science of God in nature is true}}, but he has placed nature where God should be. Nature is not God, but God created nature. \textbf{\underline{This science of God in nature is correct in one sense}}. \textbf{God gives to nature its life, its living properties, its beauty}. [He] is the author of all nature’s loveliness, and while He gives us this evidence of mighty power, \textbf{He is a personal God and Christ is a personal Saviour}.}[Ms236-1902.1; 1902][https://egwwritings.org/read?panels=p12779.6]


\egw{Se me están presentando cosas que preocupan a mi mente. El Dr. Kellogg está recorriendo el mismo camino que recorrió poco después de asumir sus responsabilidades en el Sanatorio. \textbf{La ciencia humana es una mentira con respecto a que Dios no tiene personalidad}. Sé que esto es una falsedad, y sin embargo, si podemos ayudar de alguna manera al doctor, debemos intentar hacerlo. ¿Qué se puede decir? Se le ha dado tal exaltación que está a punto de caer por el precipicio. ¿Qué podemos hacer nosotros? Sólo el Señor puede salvar al Dr. Kellogg. \textbf{\underline{Su ciencia de Dios en la naturaleza es verdadera}}, pero ha colocado a la naturaleza donde debería estar Dios. La naturaleza no es Dios, pero Dios creó la naturaleza. \textbf{\underline{Esta ciencia de Dios en la naturaleza es correcta en un sentido}}. \textbf{Dios da a la naturaleza su vida, sus propiedades vivas, su belleza}. [Él] es el autor de toda la belleza de la naturaleza, y mientras nos da esta evidencia de poderoso poder, \textbf{Él es un Dios personal y Cristo es un Salvador personal}.}[Ms236-1902.1; 1902][https://egwwritings.org/read?panels=p12779.6]


\egwnogap{\textbf{We take not the fallacies of man but the Word of God that man was created after the image of God and Christ}, for the Word declares ‘God, who at sundry times and in divers manners spake in time past unto the fathers by the prophets, hath in these last days spoken unto us by his son, whom he hath appointed heir of all things, \textbf{by whom also he made the worlds; who being the brightness of his glory, and \underline{the express image of his person}}, and upholding all things by the word of his power, when he had by himself purged our sins, \textbf{sat down on the right hand of the Majesty of heaven}.’ Hebrews 1:1-3.}[Ms236-1902.4; 1902][https://egwwritings.org/read?panels=p12779.9]


\egwnogap{\textbf{No tomamos las falacias del hombre, sino la Palabra de Dios de que el hombre fue creado a imagen de Dios y de Cristo}, pues la Palabra declara ‘Dios, habiendo hablado muchas veces y de muchas maneras en otro tiempo a los padres por los profetas, en estos postreros días nos ha hablado por el Hijo, a quien constituyó heredero de todo, \textbf{y por quien asimismo hizo el universo; el cual siendo el resplandor de su gloria, y \underline{la imagen misma de su sustancia}}, y quien sustenta todas las cosas con la palabra de su poder, habiendo efectuado la purificación de nuestros pecados por medio de sí mismo, \textbf{se sentó a la diestra de la Majestad en las alturas}.’ Hebreos 1:1-3.}[Ms236-1902.4; 1902][https://egwwritings.org/read?panels=p12779.9]


Interestingly, Sister White also claimed that God is in nature, and He is giving life and the living properties. Kellogg is correct on this point and his claim is definitely supported by her writings. Based on this point, Kellogg defended himself, saying that The Living Temple is in harmony with Sister White’s writings. He wrote to brother G. I. Butler precisely where Sister White advocated the same sentiment as he did.


Es interesante que la hermana White también afirmara que Dios está en la naturaleza, y que Él da la vida y las propiedades vivas. Kellogg tiene razón en este punto y su afirmación está definitivamente apoyada por los escritos de ella. Basándose en este punto, Kellogg se defendió diciendo que The Living Temple está en armonía con los escritos de la hermana White. Escribió al hermano G. I. Butler precisamente donde la hermana White defendía el mismo sentimiento que él.


\others{Sister White has clearly taken the same position with reference to this matter which I have taken. You will find it, in her little work on \textbf{Education }in the chapters ‘\textbf{God in Nature}’ and ‘\textbf{Science and the Bible.}’ You will find it all through ‘\textbf{Desire of Ages,}’ and ‘\textbf{Patriarchs and Prophets.}’}[Letter from Dr. Kellogg to Eld. Butler, February 21, 1904]


\others{La hermana White ha tomado claramente la misma posición con referencia a este asunto que yo he tomado. La encontraréis en su pequeña obra sobre \textbf{Educación} en los capítulos ‘\textbf{Dios en la naturaleza}’ y ‘\textbf{La ciencia y la Biblia.}’ Lo encontraréis en todo ‘\textbf{El Deseo de Todas las Gentes,}’ y ‘\textbf{Patriarcas y Profetas.}’}[Letter from Dr. Kellogg to Eld. Butler, February 21, 1904]


Let’s take a look at “\textit{God in Nature}”, in the book Education, where we can find the same sentiment regarding God in Nature that Kellogg promoted.


Echemos un vistazo a “\textit{Dios en la Naturaleza}”, en el libro Educación, donde podemos encontrar el mismo sentimiento con respecto a Dios en la Naturaleza que Kellogg promovió.


\egw{\textbf{Upon all created things is seen the impress of the Deity}. Nature testifies of God. The susceptible mind, brought in contact with the miracle and mystery of the universe, cannot but recognize \textbf{the working of infinite power}. \textbf{\underline{Not by its own inherent energy} does the earth produce its bounties}, and year by year continue its motion around the sun. \textbf{An unseen hand guides the planets in their circuit of the heavens}. \textbf{\underline{A mysterious life pervades all nature—a life that sustains the unnumbered worlds throughout immensity}}, \textbf{that lives in the insect atom which floats in the summer breeze, that wings the flight of the swallow and feeds the young ravens which cry, that brings the bud to blossom and the flower to fruit}.}[Ed 99.1; 1903][https://egwwritings.org/read?panels=p29.470]


\egw{\textbf{Sobre todas las cosas creadas se ve la impresión de la Deidad}. La naturaleza da testimonio de Dios. La mente susceptible, puesta en contacto con el milagro y el misterio del universo, no puede sino reconocer \textbf{la obra del poder infinito}. \textbf{\underline{No es por su propia energía inherente} que la tierra produce sus bondades}, y año tras año continúa su movimiento alrededor del sol. \textbf{Una mano invisible guía a los planetas en su recorrido por los cielos}. \textbf{\underline{Una vida misteriosa impregna toda la naturaleza—una vida que sostiene los mundos innumerables a través de la inmensidad}}, \textbf{que vive en el átomo de insecto que flota en la brisa de verano, que alienta el vuelo de la golondrina y alimenta a los jóvenes cuervos que lloran, que hace florecer el capullo y fructificar la flor}.}[Ed 99.1; 1903][https://egwwritings.org/read?panels=p29.470]


\egwnogap{\textbf{The same \underline{power} that upholds nature, is working also in man}. \textbf{The same great laws that guide alike the star and the atom control human life}. \textbf{The laws that govern the heart’s action, regulating the flow of the current of life to the body, are the laws of the mighty Intelligence that has the jurisdiction of the soul}. \textbf{\underline{From Him all life proceeds}}. Only in harmony with Him can be found its true sphere of action. For all the objects of His creation the condition is the same—\textbf{a life sustained by receiving the life of God}, a life exercised in harmony with the Creator’s will...}[Ed 99.2; 1903][https://egwwritings.org/read?panels=p29.471]


\egwnogap{\textbf{El mismo \underline{poder} que sostiene la naturaleza, actúa también en el hombre}. \textbf{Las mismas grandes leyes que guían tanto a la estrella como al átomo, controlan la vida humana}. \textbf{Las leyes que gobiernan la acción del corazón, regulando el flujo de la corriente de vida al cuerpo, son las leyes de la poderosa Inteligencia que tiene la jurisdicción del alma}. \textbf{\underline{De Él procede toda la vida}}. Sólo en armonía con Él puede encontrarse su verdadera esfera de acción. Para todos los objetos de Su creación la condición es la misma—\textbf{una vida sostenida al recibir la vida de Dios}, una vida ejercida en armonía con la voluntad del Creador...}[Ed 99.2; 1903][https://egwwritings.org/read?panels=p29.471]


\egw{…The heart not yet hardened by contact with evil is quick to \textbf{recognize the \underline{Presence} that pervades all created things}…}[Ed 100.2; 1903][https://egwwritings.org/read?panels=p29.475]


\egw{...El corazón que aún no se ha endurecido por el contacto con el mal se apresura a \textbf{reconocer la \underline{Presencia} que impregna todas las cosas creadas}...}[Ed 100.2; 1903][https://egwwritings.org/read?panels=p29.475]


In his defense, Kellogg was also referring to the Patriarchs and Prophets. There we read the following:


En su defensa, Kellogg también se refería a los Patriarcas y Profetas. Allí leemos lo siguiente:


\egw{Many teach that matter possesses vital power,—that certain properties are imparted to matter, and it is then left to act through its own inherent energy; and that the operations of nature are conducted in harmony with fixed laws, with which God himself cannot interfere. \textbf{This is false science, and is not sustained by the word of God}. Nature is the servant of her Creator. God does not annul his laws, or work contrary to them; \textbf{but he is continually using them as his instruments. Nature testifies of an intelligence, \underline{a presence}, \underline{an active energy}, that works in and through her laws. There is in nature the continual working of \underline{the Father and the Son}.} Christ says, ‘My Father worketh hitherto, and I work.’ John 5:17.}[PP 114.4; 1980][https://egwwritings.org/read?panels=p84.445]


\egw{Muchos enseñan que la materia posee poder vital,—que ciertas propiedades son impartidas a la materia, y que entonces se le deja actuar a través de su propia energía inherente; y que las operaciones de la naturaleza son conducidas en armonía con leyes fijas, con las cuales Dios mismo no puede interferir. \textbf{Esta es una ciencia falsa, y no está sustentada por la palabra de Dios}. La naturaleza es sierva de su Creador. Dios no anula sus leyes, ni obra en contra de ellas; \textbf{pero las utiliza continuamente como instrumentos suyos. La naturaleza da testimonio de una inteligencia, \underline{una presencia}, \underline{una energía activa}, que obra en sus leyes y a través de ellas. Hay en la naturaleza la obra continua \underline{del Padre y del Hijo}.} Cristo dice: ‘Mi Padre trabaja hasta ahora, y yo trabajo’. Juan 5:17.}[PP 114.4; 1980][https://egwwritings.org/read?panels=p84.445]


These quotations are in harmony with the quotations from The Living Temple.


Estas citas están en armonía con las citas de El Templo Viviente.


\others{The manifestations of life are as varied as the different individual animals and plants, and parts of animated things. Every leaf, every blade of grass, every flower, every bird, even every insect, as well as every beast or every tree, bears witness to the infinite versatility and inexhaustible resources of \textbf{the one all-pervading, all-creating, all-sustaining Life}.}[John H. Kellogg, The Living Temple p. 16][https://archive.org/details/J.H.Kellogg.TheLivingTemple1903/page/n15/]


\others{Las manifestaciones de la vida son tan variadas como los diferentes animales y plantas individuales, y las partes de las cosas animadas. Cada hoja, cada brizna de hierba, cada flor, cada pájaro, incluso cada insecto, así como cada bestia o cada árbol, dan testimonio de la infinita versatilidad y de los inagotables recursos de \textbf{la sola vida todo-permeante, que todo lo crea y que todo lo mantiene}.}[John H. Kellogg, The Living Temple p. 16][https://archive.org/details/J.H.Kellogg.TheLivingTemple1903/page/n15/]


\others{Intelligence is one of the forces of the universe, one of the manifestations of the \textbf{\underline{all-pervading life which} created and creates, \underline{animates and sustains}}.}[John H. Kellogg, The Living Temple p. 396][https://archive.org/details/J.H.Kellogg.TheLivingTemple1903/page/n425/]


\others{La inteligencia es una de las fuerzas del universo, una de las manifestaciones de la \textbf{\underline{vida todo-permeante que} creó y crea, \underline{anima y sostiene}}.}[John H. Kellogg, The Living Temple p. 396][https://archive.org/details/J.H.Kellogg.TheLivingTemple1903/page/n425/]


If Kellogg’s understanding of God as the source that sustains and animates nature is correct, then where is his error? Why is he called a pantheist? Is it fair to call him a pantheist? He definitely doesn’t think so. Take a look at what he wrote to Elder Butler:


Si la comprensión de Kellogg de Dios como la fuente que sostiene y anima la naturaleza es correcta, entonces ¿dónde está su error? ¿Por qué se le llama panteísta? ¿Es justo llamarle panteísta? Definitivamente, él no lo cree así. Mire lo que le escribió a Elder Butler:


\others{\textbf{I abhor pantheism} as much as you do. \textbf{I have endeavored in my book to simply teach the fact that man is dependent upon God for everything, and that without the divine power working in him the Spirit of God operating upon the elements which compose his body, he would be dust}.}[Letter from Dr. Kellogg to Eld. Butler, February 21, 1904]


\others{\textbf{Aborrezco el panteísmo} tanto como usted. \textbf{Me he esforzado en mi libro por enseñar simplemente el hecho de que el hombre depende de Dios para todo, y que sin el poder divino que obra en él el Espíritu de Dios operando sobre los elementos que componen su cuerpo, sería polvo}.}[Letter from Dr. Kellogg to Eld. Butler, February 21, 1904]


\others{I am willing to renounce all the awful doctrines you and others attribute to me. I am willing to confess that \textbf{I am not a pantheist} nor a spiritualist, and that I believe none of the doctrines taught by these people or \textbf{by pantheistic or spiritualistic writings}. I never read a pantheistic book in my life. I never read a book on ‘New Thought,’ or anything of that kind. Anybody who will read carefully the ‘Living Temple’ from the first page right straight through to the last, and will give the matter fair and consistent consideration, ought to see very clearly that \textbf{I have no accord whatever with these pantheistic and spiritualistic theories}.}[Ibid.]


\others{Estoy dispuesto a renunciar a todas las horribles doctrinas que usted y otros me atribuyen. Estoy dispuesto a confesar que \textbf{no soy panteísta} ni espiritista, y que no creo en ninguna de las doctrinas enseñadas por estas personas o \textbf{por los escritos panteístas o espiritistas}. Nunca leí un libro panteísta en mi vida. Nunca leí un libro de ‘Nuevo Pensamiento’, ni nada de ese tipo. Cualquiera que lea cuidadosamente el ‘Templo Viviente’ desde la primera página hasta la última, y que considere el asunto de manera justa y coherente, debería ver muy claramente que \textbf{no tengo ningún acuerdo con estas teorías panteístas y espiritualistas}.}[Ibid.]


This is a very hard puzzle to solve unless you encounter the truth on the \emcap{personality of God}, which we covered in the beginning of this book. Yes, God sustains life in nature. In nature, we \egwinline{\textbf{recognize \underline{the Presence} that pervades all created things}}[Ed 100.2; 1903][https://egwwritings.org/read?panels=p29.475]. But God \textit{Himself}—in His personality—is not in nature, nor is nature God. God is a \textit{personal being}, and He is in His holy temple, sitting on His throne. God is everywhere present by His \textit{representative}, the Holy Spirit.


Este es un rompecabezas muy difícil de resolver, a menos que se encuentre la verdad sobre la \emcap{personalidad de Dios}, lo cual cubrimos al principio de este libro. Sí, Dios sostiene la vida en la naturaleza. En la naturaleza, \egwinline{\textbf{reconocemos \underline{la Presencia} que impregna todas las cosas creadas}}[Ed 100.2; 1903][https://egwwritings.org/read?panels=p29.475]. Pero Dios \textit{mismo}—en su personalidad—no está en la naturaleza, ni la naturaleza es Dios. Dios es un \textit{ser personal}, y está en su santo templo, sentado en su trono. Dios está presente en todas partes por medio de su \textit{representante}, el Espíritu Santo.


When Sister White said \egwinline{Human science is a lie in regard to God \textbf{not having a personality},}[Ms236-1902; 1902][https://egwwritings.org/read?panels=p12779.6] she was particularly referencing God having a physical form of a person, as could be seen in the context of that quotation. But when Dr. Kellogg was addressing ‘\textit{personality},’ he was not addressing the form or shape of a person. In 1936 in his lecture, he expressed the same sentiments he held in the Living Temple, only more vividly:


Cuando la hermana White dijo \egwinline{La ciencia humana es una mentira con respecto a que Dios \textbf{no tiene personalidad},}[Ms236-1902; 1902][https://egwwritings.org/read?panels=p12779.6] se refería particularmente a que Dios tiene una forma física de persona, como podía verse en el contexto de esa cita. Pero cuando el Dr. Kellogg abordaba la ‘\textit{personalidad}’, no se refería a la forma o figura de una persona. En 1936, en su conferencia, expresó los mismos sentimientos que sostenía en El Templo Viviente, solo que de manera más vívida:


\others{So you see it is impossible to conceive of infinite things. They are beyond us. They are \textbf{outside of comprehension} and the same thing is true of \textbf{the \underline{infinite personality}}. \textbf{We can not form any conception of its shape or its size or any limitations of any sort because it is infinite}. Now, perhaps that is a difficult idea for you to take in and \textbf{the difficulty of accepting this idea is the fact that \underline{we have not a clear idea of personality}}. \textbf{We think of personality \underline{as connected with form}}.}


\others{Así que ves que es imposible concebir cosas infinitas. Están más allá de nosotros. Están \textbf{fuera de la comprensión} y lo mismo es cierto de \textbf{la \underline{personalidad infinita}}. \textbf{No podemos formar ninguna concepción de su forma o su tamaño o cualquier limitación de ningún tipo porque es infinita}. Ahora, quizás esa es una idea difícil de asimilar y \textbf{la dificultad de aceptar esta idea es el hecho de que \underline{no tenemos una idea clara de la personalidad}}. \textbf{Pensamos en la personalidad \underline{como conectada con la forma}}.}


\others{…\textbf{It gave me a new conception of personality}. \textbf{\underline{Personality does not mean a person, a man or a woman}}. It does not mean that sort of thing at all. \textbf{It means the possession of the power to will and to do and to think and to plan}.}[\href{https://forgotten-pillar.s3.us-east-2.amazonaws.com/Sanitarium+Lecture+1936.pdf}{Dr. Kellogg Sanitarium Lectures, 1936}; For transcript see \href{https://notefp.link/1938-kellogg-lecture}{https://notefp.link/1938-kellogg-lecture}]


\others{…\textbf{Me dio una nueva concepción de personalidad}. \textbf{\underline{Personalidad no significa una persona, un hombre o una mujer}}. No significa ese tipo de cosa en absoluto. \textbf{Significa la posesión del poder para querer y hacer y pensar y planificar}.}[\href{https://forgotten-pillar.s3.us-east-2.amazonaws.com/Sanitarium+Lecture+1936.pdf}{Dr. Kellogg Sanitarium Lectures, 1936}; Para la transcripción ver \href{https://notefp.link/1938-kellogg-lecture}{https://notefp.link/1938-kellogg-lecture}]


Such a view of personality applied to God led Dr. Kellogg into pantheism. The doctrine of the \emcap{personality of God} deals with the correct perception of God. Dr. Kellogg's perception of God was a trinitarian perception.


Tal visión de la personalidad aplicada a Dios llevó al Dr. Kellogg al panteísmo. La doctrina de la \emcap{personalidad de Dios} trata con la percepción correcta de Dios. La percepción de Dios del Dr. Kellogg era una percepción trinitaria.


\others{All I wanted to explain in Living Temple was that this work that is going on in the man here \textbf{is not going on by itself \underline{like a clock wound up}; but it is the power of God and \underline{the Spirit of God that is carrying it on}}. \textbf{Now, I thought I had cut out entirely the theological side of questions of \underline{the trinity and all that sort of things}}. \textbf{I didn't mean to put it in at all}, and I took pains to state in the preface that I did not. I never dreamed \textbf{of such a thing} as any theological question being \textbf{brought into it}. I only wanted to show that \textbf{\underline{the heart does not beat of its own motion} but that it is \underline{the power of God that keeps it going}}.}[Interview, J. H. Kellogg, G. W. Amadon and A. C. Bourdeau, October 7th 1907 held at Kellogg’s residence][https://archive.org/details/KelloggVs.TheBrethrenHisLastInterviewAsAnAdventistoct71907/page/n37]


\others{Todo lo que quería explicar en el Templo Viviente era que esta obra que se está llevando a cabo en el hombre \textbf{no funciona por sí misma \underline{como un reloj al que se le da cuerda}; sino que es el poder de Dios y \underline{el Espíritu de Dios los que la llevan a cabo}}. \textbf{Ahora, pensé que había eliminado por completo el aspecto teológico de las cuestiones de \underline{la trinidad y todo ese tipo de cosas}}. \textbf{No quise incluirlo en absoluto}, y me esforcé en declarar en el prefacio que no lo hice. Nunca soñé \textbf{con tal cosa} como cualquier cuestión teológica \textbf{siendo incluida}. Sólo quería mostrar que \textbf{\underline{el corazón no late por sí mismo} sino que es \underline{el poder de Dios el que lo mantiene en marcha}}.}[Entrevista, J. H. Kellogg, G. W. Amadon y A. C. Bourdeau, 7 de octubre de 1907 realizada en la residencia de Kellogg][https://archive.org/details/KelloggVs.TheBrethrenHisLastInterviewAsAnAdventistoct71907/page/n37]


The heart does not beat of its own motion; it is the power of God that keeps it going. In this, Kellogg was absolutely right.


El corazón no late por su propio movimiento; es el poder de Dios el que lo mantiene en marcha. En esto, Kellogg tenía toda la razón.


\egw{\textbf{The physical organism of man is under the supervision of God, but \underline{it is not like a clock which is set in operation and must go of itself}}. \textbf{The heart beats, pulse succeeds pulse, breath succeeds breath, but bear in mind that the being is under the supervision of God}. Ye are God's husbandry, ye are God's building. \textbf{In God we live and move and have our being}. \textbf{Each heartbeat, each breath is the inspiration of that God who breathed into the nostrils of Adam the breath of life}, the inspiration of the ever present God, the great I AM.}[13LtMs, Ms 92, 1898, par. 7][https://egwwritings.org/read?panels=p14063.7342012&index=0]


\egw{\textbf{El organismo físico del hombre está bajo la supervisión de Dios, pero \underline{no es como un reloj que se pone en funcionamiento y debe funcionar por sí mismo}}. \textbf{El corazón late, el pulso sucede al pulso, la respiración sucede a la respiración, pero ten en cuenta que el ser está bajo la supervisión de Dios}. Sois labranza de Dios, edificio de Dios. \textbf{En Dios vivimos, nos movemos y somos}. \textbf{Cada latido del corazón, cada respiración es la inspiración de aquel Dios que sopló en las narices de Adán el aliento de vida}, la inspiración del Dios siempre presente, el gran YO SOY.}[13LtMs, Ms 92, 1898, par. 7][https://egwwritings.org/read?panels=p14063.7342012&index=0]


Dr. Kellogg's \egwinline{science of God in nature is true.}[Ms236-1902; 1902][https://egwwritings.org/read?panels=p12779.6] The Scriptures clearly teach it: \bible{If he \normaltext{[God]} set his heart upon man, \textbf{if he gather unto himself \underline{his spirit} and his breath}; \textbf{\underline{All flesh shall perish together}, and man shall turn again unto dust}.}[Job 34:14-15] \bible{…thy judgments are a great deep: \textbf{O Lord, thou \underline{preservest} man and beast}… \textbf{For with thee is the fountain of life}: in thy light shall we see light.}[Psalm 36:6b,9]


La \egwinline{ciencia de Dios en la naturaleza}[Ms236-1902; 1902][https://egwwritings.org/read?panels=p12779.6] del Dr. Kellogg es verdadera. Las Escrituras lo enseñan claramente: \bible{Si él \normaltext{[Dios]} pusiese sobre el hombre su corazón, \textbf{y \underline{su espíritu} y su aliento recogiese}; \textbf{\underline{Toda carne perecería juntamente}, y el hombre volvería al polvo}.}[Job 34:14-15] \bible{...tus juicios son un abismo grande: \textbf{Oh Jehová, \underline{conservas} al hombre y la bestia}... \textbf{Porque contigo está el manantial de la vida}: en tu luz veremos la luz.}[Salmo 36:6b,9]


This evidence testifies that Dr. Kellogg's science of God in nature is true, but his problems were erroneous views on the personality of God, which were trinitarian views. Even when he clarified that \others{God the Father sits upon his throne in heaven where God the Son is also; while God's life, or spirit or presence is the all-pervading power which is carrying out the will of God in all the universe,}[Letter: Dr. Kellogg to W. W. Prescott, October 25, 1903][https://forgotten-pillar.s3.us-east-2.amazonaws.com/1903-10-25-JHKellogg-to-W.W.Prescott.pdf] still he held erroneous views on the personality of God—God in \others{comprehensive sense} as \others{the Godhead… God the Father, God the Son, and God the Holy Spirit}[Ibid.][https://forgotten-pillar.s3.us-east-2.amazonaws.com/1903-10-25-JHKellogg-to-W.W.Prescott.pdf]. His Trinitarian view could \textit{not} \others{clear the matter up satisfactorily.}[Letter: A. G. Daniells to W. C. White, October 29, 1903][https://forgotten-pillar.s3.us-east-2.amazonaws.com/Letter-A-G-Daniells-to-W-C-White-October-29-1903.pdf]


Esta evidencia testifica que la ciencia de Dios en la naturaleza del Dr. Kellogg es verdadera, pero sus problemas eran puntos de vista erróneos sobre la personalidad de Dios, que eran puntos de vista trinitarios. Incluso cuando aclaró que \others{Dios el Padre se sienta en su trono en el cielo donde también está Dios el Hijo; mientras que la vida, o espíritu o presencia de Dios es el poder omnipresente que está llevando a cabo la voluntad de Dios en todo el universo,}[Carta: Dr. Kellogg a W. W. Prescott, 25 de octubre de 1903][https://forgotten-pillar.s3.us-east-2.amazonaws.com/1903-10-25-JHKellogg-to-W.W.Prescott.pdf] todavía mantenía puntos de vista erróneos sobre la personalidad de Dios—Dios en \others{sentido integral} como \others{la Deidad... Dios el Padre, Dios el Hijo y Dios el Espíritu Santo}[Ibid.][https://forgotten-pillar.s3.us-east-2.amazonaws.com/1903-10-25-JHKellogg-to-W.W.Prescott.pdf]. Su visión trinitaria \textit{no} podía \others{aclarar el asunto satisfactoriamente.}[Carta: A. G. Daniells a W. C. White, 29 de octubre de 1903][https://forgotten-pillar.s3.us-east-2.amazonaws.com/Letter-A-G-Daniells-to-W-C-White-October-29-1903.pdf]


The conclusion is frightening. If you believe that the heart does not beat of its own motion but that it is the power of God that keeps it going, and you combine it with the belief that God Himself is not a tangible being but a spirit present everywhere, then in the eyes of the Spirit of Prophecy, you are a pantheist. The perception of the quality or state of God being a person makes the difference between the true believer and the pantheist.


La conclusión es aterradora. Si crees que el corazón no late por su propio movimiento, sino que es el poder de Dios el que lo mantiene en marcha, y lo combinas con la creencia de que Dios mismo no es un ser tangible sino un espíritu presente en todas partes, entonces a los ojos del Espíritu de Profecía, eres un panteísta. La percepción de la cualidad o estado de Dios siendo una persona marca la diferencia entre el verdadero creyente y el panteísta.


% Dr. Kellogg and pantheism

\begin{titledpoem}
    
    \stanza{
        In nature’s vast, a truth untold, \\
        He said God was in every fold. \\
        The trees, the breeze, the soil, the sea, \\
        God’s presence there, for all to see.
    }

    \stanza{
        Yet, in this truth where we concur, \\
        A deeper error did occur. \\
        The Trinity, unsacred bond, \\
        As pantheism and beyond.
    }

    \stanza{
        God’s personality is clear, \\
        Beyond those frontiers, we revere. \\
        For God, who’s more than nature’s face, \\
        Is personal, in sacred space.
    }

    \stanza{
        The doctor’s path did lead astray, \\
        On trinity, we cannot sway. \\
        His view of God, misunderstood, \\
        A misstep from the path of good.
    }

    \stanza{
        In nature, power does reside, \\
        It’s not God’s body that presides. \\
        Beside Him, Christ stands as our guide, \\
        And by His Spirit, life abides.
    }

    \stanza{
        In nature’s charm, God’s hand we see, \\
        Beyond the vastness, He must be. \\
        A precious God, with love so wide, \\
        In whom, in peace, we can confide.  
    }
    
\end{titledpoem}


% Dr. Kellogg and pantheism

\begin{titledpoem}
    
    \stanza{
        In nature’s vast, a truth untold, \\
        He said God was in every fold. \\
        The trees, the breeze, the soil, the sea, \\
        God’s presence there, for all to see.
    }

    \stanza{
        Yet, in this truth where we concur, \\
        A deeper error did occur. \\
        The Trinity, unsacred bond, \\
        As pantheism and beyond.
    }

    \stanza{
        God’s personality is clear, \\
        Beyond those frontiers, we revere. \\
        For God, who’s more than nature’s face, \\
        Is personal, in sacred space.
    }

    \stanza{
        The doctor’s path did lead astray, \\
        On trinity, we cannot sway. \\
        His view of God, misunderstood, \\
        A misstep from the path of good.
    }

    \stanza{
        In nature, power does reside, \\
        It’s not God’s body that presides. \\
        Beside Him, Christ stands as our guide, \\
        And by His Spirit, life abides.
    }

    \stanza{
        In nature’s charm, God’s hand we see, \\
        Beyond the vastness, He must be. \\
        A precious God, with love so wide, \\
        In whom, in peace, we can confide.  
    }
    
\end{titledpoem}
