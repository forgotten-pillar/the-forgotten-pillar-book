\qrchapter{https://forgottenpillar.com/rsc/en-fp-chapter27}{Steps to apostasy}


\qrchapter{https://forgottenpillar.com/rsc/es-fp-chapter27}{Pasos hacia la apostasía}


In the following quotation, brother J. N. Loughborough, who was one of the pioneers of the Seventh-day Adventist Church, warned us about the five steps to apostasy.


En la siguiente cita, el hermano J. N. Loughborough, que fue uno de los pioneros de la Iglesia Adventista del Séptimo Día, nos advirtió sobre los cinco pasos hacia la apostasía.


\others{\textbf{The} \textbf{first step} of apostasy is to \textbf{get up a creed}, telling us what we shall believe. \textbf{The second} is to \textbf{make that creed a test of fellowship}. \textbf{The third} is to \textbf{try members by that creed}. \textbf{The fourth} is to \textbf{denounce as heretics those who do not believe that creed}. And \textbf{fifth}, to \textbf{commence persecution against such}. I plead that we are not patterning after the churches in any unwarrantable sense in the step proposed.}[John N. Loughborough, Review and Herald, Oct. 8, 1861.][https://egwwritings.org/read?panels=p1685.5326]


\others{\textbf{El} \textbf{primer paso} de la apostasía es \textbf{conseguir un credo} que nos diga lo que debemos creer. \textbf{El segundo} es \textbf{hacer de ese credo una prueba de compañerismo}. \textbf{El tercero} es \textbf{juzgar a los miembros por ese credo}. \textbf{El cuarto} es \textbf{denunciar como herejes a los que no creen en ese credo}. Y \textbf{quinto}, \textbf{comenzar la persecución contra ellos}. Ruego que no estemos imitando a las iglesias en ningún sentido injustificado en el paso propuesto.}[John N. Loughborough, Review and Herald, Oct. 8, 1861.][https://egwwritings.org/read?panels=p1685.5326]


These principles are important to have in mind, and we ought to ask ourselves if we, today, are patterning after the churches in any unwarrantable sense in the step proposed. What would happen to a Seventh-day Adventist who would reject the Trinity doctrine in favor of the \emcap{Fundamental Principles}? Do we have a creed set up in our church? Do we test our membership by it?


Es importante tener en cuenta estos principios, y debemos preguntarnos si, hoy en día, estamos imitando a las iglesias en algún sentido injustificable en el paso propuesto. ¿Qué pasaría con un adventista del séptimo día que rechazara la doctrina trinitaria en favor de los \emcap{Principios Fundamentales}? ¿Tenemos un credo establecido en nuestra iglesia? ¿Ponemos a prueba a nuestros miembros por medio de él?


The \emcap{Fundamental Principles} had a different nature and role in the Seventh-day Adventist Church contrary to that of the pattern held by other churches. The \emcap{Fundamental Principles} were not designed as a creed. In the preface of the 1872 statement, we read about their nature:


Los \emcap{Principios Fundamentales} tenían una naturaleza y un papel diferentes en la Iglesia Adventista del Séptimo Día, contrario al patrón seguido por otras iglesias. Los \emcap{Principios Fundamentales} no fueron diseñados como un credo. En el prefacio de la declaración de 1872, leemos sobre su naturaleza:


\others{In presenting to the \textbf{public} this \textbf{synopsis of our faith}, we wish to have it distinctly understood that \textbf{\underline{we have no articles of faith, creed}, or discipline, \underline{aside from the Bible}}. We \textbf{do not} put forth this \textbf{\underline{as having any authority with our people}}, \textbf{nor is it designed to secure uniformity among them}, \textbf{as a system of faith}, \textbf{but is a brief statement of what is, and has been, with great unanimity, held by them}.}[A Declaration of the Fundamental Principles, Taught and Practiced by the Seventh-Day Adventists, 1872]


\others{Al presentar al \textbf{público} esta \textbf{sinopsis de nuestra fe}, deseamos que se entienda claramente que \textbf{\underline{no tenemos artículos de fe, credo}, o disciplina, \underline{aparte de la Biblia}}. \textbf{No} presentamos esto \textbf{\underline{como si tuviera alguna autoridad entre nuestra gente}}, \textbf{ni está diseñado para asegurar la uniformidad entre ellos}, \textbf{como un sistema de fe}, \textbf{sino que es una breve declaración de lo que es, y ha sido, con gran unanimidad, sostenido por ellos}.}[Declaración de los Principios Fundamentales, Enseñados y Practicados por los Adventistas del Séptimo Día, 1872]


In the preface of the 1889 statement, we read similar sentiments:


En el prefacio de la declaración de 1889, leemos sentimientos similares:


\others{As elsewhere stated, Seventh-day Adventists \textbf{have no creed but the Bible}; but they hold to \textbf{certain well-defined points of faith}, for which they \textbf{feel prepared to give a reason ‘to every man that asketh’ them}. The following propositions may be taken as a summary of \textbf{the principal features of their religious faith}, upon which there is, so far as we know, \textbf{entire unanimity throughout the body}.}[Seventh-day Adventist Year Book of statistics for 1889, pg. 147, The Fundamental Principles of Seventh-day Adventists]


\others{Como se ha dicho en otra parte, los Adventistas del Séptimo Día \textbf{no tienen más credo que la Biblia}; pero sostienen \textbf{ciertos puntos de fe bien definidos}, para los cuales \textbf{se sienten preparados para dar una razón ‘a todo el que se lo pida’}. Las siguientes proposiciones pueden tomarse como un resumen de \textbf{los principales rasgos de su fe religiosa}, sobre los que hay, por lo que sabemos, \textbf{total unanimidad en todo el cuerpo}.}[Seventh-day Adventist Year Book of statistics for 1889, pg. 147, The Fundamental Principles of Seventh-day Adventists]


The \emcap{Fundamental Principles} were not designed to dictate someone’s faith. The believers, led by the Holy Spirit, freely rendered their consciences to the Word of God; under the influence of the Holy Spirit, they came to the same conclusions. There was entire unanimity throughout the body. All believers felt “\textit{prepared to give a reason to every man that asketh them}” regarding their faith.


Los \emcap{Principios Fundamentales} no fueron diseñados para dictar la fe de alguien. Los creyentes, guiados por el Espíritu Santo, sometieron libremente sus conciencias a la Palabra de Dios; bajo la influencia del Espíritu Santo, llegaron a las mismas conclusiones. Había total unanimidad en todo el cuerpo. Todos los creyentes se sentían “\textit{preparados para dar razón a todo el que se la pidiera}” respecto a su fe.


Today we see a striking difference in the principles and practice of Adventist beliefs compared to our pioneers. We are keeping the spirit of unity by disciplining our members for the denial of the Fundamental Beliefs. In our church manual, under the section “\textit{Reason for Disciplines}”, we read the first point which states the discipline for denial of faith in the Seventh-day Adventist Fundamental Beliefs.


Hoy vemos una diferencia sorprendente en los principios y la práctica de las creencias adventistas en comparación con nuestros pioneros. Mantenemos el espíritu de unidad al disciplinar a nuestros miembros por la negación de las Creencias Fundamentales. En nuestro manual de la iglesia, bajo la sección “\textit{Motivos de disciplinas}”, leemos el primer punto que establece la disciplina por la negación de la fe en las Creencias Fundamentales de los Adventistas del Séptimo Día.


\others{Reasons for Discipline}


\others{Razones para la disciplina}


\others{1. \textbf{Denial of faith} in the fundamentals of the gospel and \textbf{in the Fundamental Beliefs of the Church} or \textbf{teaching doctrines contrary to the same}.}[SDA Church Manual, 20th edition, Revised 2022, p. 67][https://www.adventist.org/wp-content/uploads/2023/07/2022-Seventh-day-Adventist-Church-Manual.pdf]


\others{1. \textbf{Negación de la fe} en los fundamentos del evangelio y \textbf{en las Creencias Fundamentales de la Iglesia} o \textbf{enseñanza de doctrinas contrarias a los mismos}.}[SDA Church Manual, 20th edition, Revised 2022, p. 67][https://www.adventist.org/wp-content/uploads/2023/07/2022-Seventh-day-Adventist-Church-Manual.pdf]


To discipline someone over their faith is nothing else than coercion of conscience. We are to render our conscience to the Bible alone—not to any man, councils or church creed(s). Disciplining members for their denial of the Fundamental Beliefs is clear evidence that we, indeed, have a creed besides the Bible. We cannot exercise the freedom of our conscience in subjection to the Word of God while confined to a set of beliefs that, if questioned with the authority of the Bible, will be disciplined. In our practice we have forgotten the foundation of protestantism and reformation. All reformers have had their conscience coerced to the extent of their lives. Martin Luther had famously put this principle in action in his defense before the Diet of Worms.


Disciplinar a alguien por su fe no es más que una coerción de conciencia. Debemos rendir nuestra conciencia solo a la Biblia—no a ningún hombre, concilios o credo(s) de la iglesia. Disciplinar a los miembros por su negación de las Creencias Fundamentales es una clara evidencia de que, de hecho, tenemos un credo además de la Biblia. No podemos ejercer la libertad de nuestra conciencia en sujeción a la Palabra de Dios mientras estemos confinados a un conjunto de creencias que, si son cuestionadas con la autoridad de la Biblia, serán disciplinadas. En nuestra práctica hemos olvidado el fundamento del protestantismo y la reforma. Todos los reformadores han tenido su conciencia coercionada hasta el punto de sus vidas. Martín Lutero había puesto famosamente este principio en acción en su defensa ante la Dieta de Worms.


\others{Unless I am \textbf{convicted by Scripture} and plain reason—I do not accept the authority of popes and councils, for they have contradicted each other—\textbf{\underline{my conscience is captive to the Word of God}}. I cannot and I will not recant anything, for \textbf{to go against conscience is neither right nor safe}. Here I stand, I cannot do otherwise. God help me. Amen.}[Bainton, 182]


\others{A menos que me \textbf{convenzan las Escrituras} y la razón pura—no acepto la autoridad de papas y concilios, pues se han contradicho—\textbf{\underline{mi conciencia es cautiva de la Palabra de Dios}}. No puedo y no me retractaré de nada, porque \textbf{ir en contra de la conciencia no es ni correcto ni seguro}. Aquí estoy, no puedo hacer otra cosa. Que Dios me ayude. Amén.}[Bainton, 182]


If one Seventh-day Adventist member has his conscience captive to the Word of God and is not in harmony with the Seventh-day Adventist Fundamental Beliefs, his conscience should not be coerced by church discipline. We know that in the end of time, the whole Seventh-day Adventist Church will be coerced over the issue of the Sabbath. We have been fighting for religious freedom, yet we’re allowing ourselves to coerce the conscience of those who are not in harmony with the Fundamental Beliefs. If today we discipline our members for not subjecting their consciences to men, councils and creeds, how shall we act tomorrow when the government will discipline their citizens for not subjecting their conscience to its power, when they will force obedience to legislation contrary to the Scriptures?


Si un miembro adventista del séptimo día tiene su conciencia cautiva a la Palabra de Dios y no está en armonía con las Creencias Fundamentales Adventistas del Séptimo Día, su conciencia no debe ser coercionada por la disciplina de la iglesia. Sabemos que al final de los tiempos, toda la Iglesia Adventista del Séptimo Día será coercionada por la cuestión del sábado. Hemos estado luchando por la libertad religiosa, pero nos permitimos coercionar la conciencia de aquellos que no están en armonía con las Creencias Fundamentales. Si hoy disciplinamos a nuestros miembros por no someter sus conciencias a hombres, concilios y credos, ¿cómo actuaremos mañana cuando el gobierno disciplinará a sus ciudadanos por no someter su conciencia a su poder, cuando obligará a obedecer una legislación contraria a las Escrituras?


Adventist pioneers were very much aware of the dangers of extorting church members’ consciences. The expression of their beliefs was not designed to form unity. They were ready to justify their faith, from the Bible, when asked. The Bible was their only creed and article of faith.


Los pioneros adventistas eran muy conscientes de los peligros de extorsionar las conciencias de los miembros de la iglesia. La expresión de sus creencias no estaba destinada a formar la unidad. Estaban dispuestos a justificar su fe, a partir de la Biblia, cuando se les pedía. La Biblia era su único credo y artículo de fe.


In 1883, there was a suggestion to introduce the church manual into the Seventh-day Adventist Church. This proposal was rejected after close investigation of the committee appointed by the General Conference. In the article “\textit{No Church Manual}”, we read their reasons for not accepting the proposed church manual.


En 1883, hubo una sugerencia de introducir el manual de la iglesia en la Iglesia Adventista del Séptimo Día. Esta propuesta fue rechazada después de una minuciosa investigación del comité nombrado por la Conferencia General. En el artículo “\textit{No Church Manual}”, leemos sus razones para no aceptar el manual de la iglesia propuesto.


\others{\textbf{While brethren who have favored a manual have ever contended that such a work was not to be anything like a creed or a discipline, or to have authority to settle disputed points}, but was only to be considered as a book containing hints for the help of those of little experience, \textbf{yet it must be evident that such a work, issued under the auspices of the General Conference, would at once carry with it much weight of authority, and would be consulted by most of our younger ministers}. \textbf{\underline{It would gradually shape and mold the whole body}}; \textbf{and those who did not follow it would be considered out of harmony with established principles of church order}. \textbf{And, really, is this not the object of the manual?} And what would be the use of one if not to accomplish such a result? But would this result, on the whole, be a benefit? Would our ministers be broader, more original, more self-reliant men? Could they be better depended on in great emergencies? Would their spiritual experiences likely be deeper and their judgment more reliable? \textbf{We think the tendency all the other way}.}[No Church Manual, The Review and Herald, November 27, 1883, pg. 745][https://documents.adventistarchives.org/Periodicals/RH/RH18831127-V60-47.pdf]


\others{\textbf{Aunque los hermanos que han estado a favor de un manual han sostenido siempre que tal obra no debía ser nada parecido a un credo o una disciplina, o tener autoridad para resolver puntos disputados}, sino que sólo debía considerarse como un libro que contenía sugerencias para ayudar a los que tienen poca experiencia, \textbf{sin embargo, debe ser evidente que tal obra, publicada bajo los auspicios de la Asociación General, tendría de inmediato mucho peso de autoridad, y sería consultada por la mayoría de nuestros ministros más jóvenes}. \textbf{\underline{Poco a poco daría forma y moldearía a todo el cuerpo}}; \textbf{y aquellos que no la siguieran serían considerados fuera de armonía con los principios establecidos del orden eclesiástico}. \textbf{Y, realmente, ¿no es éste el objeto del manual?} ¿Y de qué serviría uno si no fuera para lograr tal resultado? Pero, ¿sería este resultado, en general, un beneficio? ¿Serían nuestros ministros hombres más amplios, más originales, más autosuficientes? ¿Se podría depender mejor de ellos en las grandes emergencias? ¿Serían sus experiencias espirituales más profundas y su juicio más fiable? \textbf{Creemos que la tendencia es la contraria}.}[No Church Manual, The Review and Herald, November 27, 1883, pg. 745][https://documents.adventistarchives.org/Periodicals/RH/RH18831127-V60-47.pdf]


\others{\textbf{The Bible contains our creed and discipline. It \underline{thoroughly} furnishes the man of God unto all good works}. What it has not revealed relative to church organization and management, the duties of officers and ministers, and kindred subjects, should not be strictly defined and drawn out into minute specifications for the sake of uniformity, \textbf{but rather be left to individual judgment under the guidance of the Holy Spirit}. \textbf{Had it been best to have a book of directions of this sort, the Spirit would doubtless have gone further, and left one on record with the stamp of inspiration upon it}.}[Ibid.][https://documents.adventistarchives.org/Periodicals/RH/RH18831127-V60-47.pdf]


\others{\textbf{La Biblia contiene nuestro credo y disciplina. Equipa \underline{completamente} al hombre de Dios para todas las buenas obras}. Lo que no ha revelado en relación con la organización y administración de la iglesia, los deberes de los funcionarios y ministros, y otros temas afines, no debe ser estrictamente definido y elaborado en especificaciones minuciosas en aras de la uniformidad, \textbf{sino que debe dejarse al juicio individual bajo la guía del Espíritu Santo}. \textbf{Si hubiera sido mejor tener un libro de instrucciones de este tipo, el Espíritu sin duda habría ido más allá, y habría dejado uno registrado con el sello de la inspiración}.}[Ibid.][https://documents.adventistarchives.org/Periodicals/RH/RH18831127-V60-47.pdf]


Since 1883, the Seventh-day Adventist Church had grown considerably; so, for the sake of convenience, in 1931, the General Conference Committee voted to publish a church manual.\footnote{Maratas, Prince. “Church Manual.” General Conference of Seventh-Day Adventists, 20 Aug. 2023, \href{https://gc.adventist.org/church-manual/}{gc.adventist.org/church-manual/}. Accessed 3 Feb. 2025.} The church, as an organized body, should exercise order and discipline, in the matters of organization and plans of the prosperity of the Church's mission. But no committee should exercise authority over someone’s conscience and someone’s belief. Only God holds the right to this authority. This is why the Bible is our only creed. We render our conscience to the Word of God, not a man, nor a group of men or committee. Contrary to this, many believe that God vested this authority to the general assembly of the General Conference. But such an idea is based on misrepresentation of one particular quotation. Let us read this quotation carefully.


Desde 1883, la Iglesia Adventista del Séptimo Día había crecido considerablemente; así que, por conveniencia, en 1931, el Comité de la Conferencia General votó la publicación de un manual de la iglesia.\footnote{Maratas, Prince. “Church Manual.” General Conference of Seventh-Day Adventists, 20 Aug. 2023, \href{https://gc.adventist.org/church-manual/}{gc.adventist.org/church-manual/}. Accessed 3 Feb. 2025.} La iglesia, como cuerpo organizado, debe ejercer el orden y la disciplina, en los asuntos de organización y planes de la prosperidad de la misión de la Iglesia. Pero ningún comité debe ejercer autoridad sobre la conciencia y la creencia de alguien. Sólo Dios tiene derecho a esta autoridad. Por eso la Biblia es nuestro único credo. Rendimos nuestra conciencia a la Palabra de Dios, no a un hombre, ni a un grupo de hombres o comité. Contrario a esto, muchos creen que Dios confirió esta autoridad a la asamblea general de la Conferencia General. Pero tal idea se basa en la tergiversación de una cita particular. Leamos atentamente esta cita.


\egw{At times, when a small group of men entrusted with \textbf{the general management of the work} have, in the name of the General Conference, sought to carry out unwise plans and to restrict God’s work, I have said that I could no longer regard the voice of the General Conference, represented by these few men, as the voice of God. \textbf{But this is not saying that the decisions of a General Conference composed of an assembly of duly appointed, representative men from all parts of the field should not be respected}. \textbf{God has ordained that the representatives of His church from all parts of the earth, when assembled in a General Conference, \underline{shall have authority}}. The error that some are in danger of committing is in giving to the mind and judgment of one man, or of a small group of men, \textbf{the full measure of authority and influence that God has vested in His church in the judgment and voice of the General Conference assembled \underline{to plan for the prosperity and advancement of His work}}.}[9T 260.2; 1909][https://egwwritings.org/read?panels=p115.1474]


\egw{A veces, cuando un pequeño grupo de hombres a los que se les ha confiado \textbf{la dirección general de la obra} han tratado, en nombre de la Conferencia General, de llevar a cabo planes imprudentes y de restringir la obra de Dios, he dicho que ya no podía considerar la voz de la Conferencia General, representada por esos pocos hombres, como la voz de Dios. \textbf{Pero esto no quiere decir que no deban respetarse las decisiones de una Conferencia General compuesta por una asamblea de hombres debidamente designados y representativos de todas las partes del campo}. \textbf{Dios ha ordenado que los representantes de su iglesia de todas las partes de la tierra, cuando se reúnen en una Conferencia General, \underline{tengan autoridad}}. El error que algunos corren el peligro de cometer es el de dar a la mente y al juicio de un solo hombre, o de un pequeño grupo de hombres, \textbf{toda la medida de autoridad e influencia que Dios ha conferido a su iglesia en el juicio y la voz de la Conferencia General reunida \underline{para planear la prosperidad y el avance de su obra}}.}[9T 260.2; 1909][https://egwwritings.org/read?panels=p115.1474]


Sister White pointed out that the world wide assembly of the General Conference meeting does have authority as the voice of God, yet she is very particular over what matters it has this authority. The authority God vested in the assembly of the General Conference is \egwinline{to plan for the prosperity and advancement of His work}. It is about making mission plans, not about managing beliefs or the conscience. God’s church does have His voice regarding beliefs; the voice of God pertaining to the faith is the Bible. The Bible is fully sufficient for us and we are free to render our conscience to it. No synopsis of any denominational faith has authority to dictate someone's faith; neither do \emcap{Fundamental Principles}, or current Fundamental Beliefs.\footnote{Although the Fundamental Principles were not designed to have authority over the people, nor were they designed to secure uniformity among them, as a system of faith, there is some evidence to the contrary. In his article, “\textit{Seventh-day Adventists and the Doctrine of the Trinity}”, of the “\textit{Christian Workers Magazine}”, 1915, D.M. Canright gave evidence that a Conference president used the \emcap{Fundamental Principles} as a test of fellowship in 1911. Such practice is not constructive to the Truth, neither is it beneficial for believers.} Sister White was very clear about the Bible being the only rule of faith, and every doctrine should be questioned with Scripture. In the Great Controversy, we read the following:


La hermana White señaló que la asamblea mundial de la reunión de la Asociación General tiene autoridad como la voz de Dios, pero es muy particular sobre qué asuntos tiene esta autoridad. La autoridad que Dios confirió a la asamblea de la Conferencia General es \egwinline{para planear la prosperidad y el avance de su obra}. Se trata de hacer planes de misión, no de administrar las creencias o la conciencia. La iglesia de Dios sí tiene su voz con respecto a las creencias; la voz de Dios relativa a la fe es la Biblia. La Biblia es totalmente suficiente para nosotros y somos libres de rendir nuestra conciencia a ella. Ninguna sinopsis de ninguna fe denominacional tiene autoridad para dictar la fe de alguien; tampoco los \emcap{Principios Fundamentales}, ni las creencias fundamentales actuales.\footnote{Aunque los Principios Fundamentales no fueron diseñados para tener autoridad sobre las personas, ni fueron diseñados para asegurar la uniformidad entre ellas, como un sistema de fe, hay alguna evidencia de lo contrario. En su artículo, “\textit{Seventh-day Adventists and the Doctrine of the Trinity}”, del “\textit{Christian Workers Magazine}”, 1915, D.M. Canright dio evidencia de que un presidente de Conferencia usó los \emcap{Principios Fundamentales} como una prueba de comunión en 1911. Tal práctica no es constructiva para la Verdad, ni es beneficiosa para los creyentes.} La hermana White fue muy clara en cuanto a que la Biblia es la única regla de fe, y toda doctrina debe ser cuestionada con las Escrituras. En el Gran Conflicto, leemos lo siguiente:


\egw{But God will have a people upon the earth \textbf{to maintain the Bible, and \underline{the Bible only}}, \textbf{as the standard of all doctrines and the basis of all reforms}. \textbf{The opinions of learned men, the deductions of science, \underline{the creeds or decisions of ecclesiastical councils}, as numerous and discordant as are the churches which they represent, the voice of the majority - not one nor all of these should be regarded as evidence for or against any point of religious faith.} \textbf{Before accepting any doctrine or precept, we should demand a plain ‘Thus saith the Lord’ in its support.}}[GC 595.1; 1888][https://egwwritings.org/read?panels=p132.2689]


\egw{Pero Dios tendrá un pueblo sobre la tierra \textbf{que mantenga la Biblia, y \underline{sólo la Biblia}}, \textbf{como la norma de todas las doctrinas y la base de todas las reformas}. \textbf{Las opiniones de los hombres eruditos, las deducciones de la ciencia, \underline{los credos o las decisiones de los concilios eclesiásticos}, tan numerosos y discordantes como lo son las iglesias que representan, la voz de la mayoría - ni uno ni todos ellos deben ser considerados como evidencia a favor o en contra de cualquier punto de la fe religiosa.} \textbf{Antes de aceptar cualquier doctrina o precepto, deberíamos exigir un claro ‘Así dice el Señor’ en su apoyo.}}[GC 595.1; 1888][https://egwwritings.org/read?panels=p132.2689]


The liberty of conscience is the basics of protestantism and reformation. We hope and believe that every Seventh-day Adventist can exercise freedom to render his conscience to the Bible without being coerced by discipline, or any other means. The issue of the church's creed and discipline becomes more relevant today, when we have the promise that God will re-establish the original foundation of our faith. We hope and pray that the evidence brought up here will bring light to the church leadership and encourage them to eradicate the false practices in our midst. As the religious leaders in Christ’s time were entrusted with the duty to preserve the Truth and to recognize the time of God’s visitation, so it is today with the leaders of the Seventh-day Adventist Church. In what follows, we will present the prophecies God specifically gave to the Seventh-day Adventist Church. In our time, the end-time, all the pillars of our faith that were held in the beginning will be re-established. May every member of the Seventh-day Adventist Church recognize the importance of the revival that God is about to establish.


La libertad de conciencia es la base del protestantismo y de la reformación. Esperamos y creemos que cada adventista del séptimo día puede ejercer la libertad de rendir su conciencia a la Biblia sin ser extorsionado por disciplina, o cualquier otro medio. La cuestión del credo y la disciplina de la iglesia adquiere mayor relevancia hoy en día, cuando tenemos la promesa de que Dios restablecerá el fundamento original de nuestra fe. Esperamos y oramos para que las evidencias aquí expuestas traigan luz a los dirigentes de la iglesia y les animen a erradicar las falsas prácticas en nuestro medio. Así como a los líderes religiosos de la época de Cristo se les encomendó el deber de preservar la Verdad y reconocer el tiempo de la visitación de Dios, lo mismo ocurre hoy con los líderes de la Iglesia Adventista del Séptimo Día. En lo que sigue, presentaremos las profecías que Dios dio específicamente a la Iglesia Adventista del Séptimo Día. En nuestro tiempo, el tiempo del fin, se restablecerán todos los pilares de nuestra fe que se sostenían en el principio. Que cada miembro de la Iglesia Adventista del Séptimo Día reconozca la importancia del reavivamiento que Dios está a punto de establecer.


% Steps to Apostasy

\begin{titledpoem}
    \stanza{
        A creed established beyond God's Word, \\
        The voice of conscience no longer heard. \\
        Fellowship tested by human decree, \\
        From Bible authority we slowly flee.
    }

    \stanza{
        Those who dissent labeled heretics, lost, \\
        Their faith and conviction at terrible cost. \\
        Persecution follows for standing apart, \\
        When creeds replace Scripture within the heart. \\
    }

    \stanza{
        The Bible alone should guide our belief, \\
        All other authorities bringing grief. \\
        Our conscience surrenders to God's Word divine, \\
        Not to councils of men who draw the line.
    }

    \stanza{
        The pioneers knew this freedom well, \\
        Against human creeds they chose to rebel. \\
        For truth must flourish where conscience is free, \\
        As God intended His church to be.
    }
\end{titledpoem}


% % Steps to Apostasy

\begin{titledpoem}
    \stanza{
        A creed established beyond God's Word, \\
        The voice of conscience no longer heard. \\
        Fellowship tested by human decree, \\
        From Bible authority we slowly flee.
    }

    \stanza{
        Those who dissent labeled heretics, lost, \\
        Their faith and conviction at terrible cost. \\
        Persecution follows for standing apart, \\
        When creeds replace Scripture within the heart. \\
    }

    \stanza{
        The Bible alone should guide our belief, \\
        All other authorities bringing grief. \\
        Our conscience surrenders to God's Word divine, \\
        Not to councils of men who draw the line.
    }

    \stanza{
        The pioneers knew this freedom well, \\
        Against human creeds they chose to rebel. \\
        For truth must flourish where conscience is free, \\
        As God intended His church to be.
    }
\end{titledpoem}
