\qrchapter{https://forgottenpillar.com/rsc/en-fp-chapter18}{The Heavenly Trio}


\qrchapter{https://forgottenpillar.com/rsc/es-fp-chapter18}{El Trío Celestial}


So far we have seen the evidence that Ellen White knew about Dr. Kellogg's trinitarian sentiments, and we have seen how she responded to it. She always uplifted the truth on the presence and the \emcap{personality of God}, and called to come back to the foundation of our faith—\emcap{Fundamental Principles}. However, when Adventist scholars discuss the doctrine of the Trinity and Ellen White, they do not approach it in the same manner as Ellen White did. The \emcap{Fundamental Principles} together with the doctrine on the \emcap{personality of God} is downplayed, and the twisted story is presented that Ellen White was trinitarian and responsible for the church's acceptance of the Trinity doctrine into our ranks. We want to challenge this twisted story by looking at the evidence that is often used to support this false narrative.


Hasta ahora hemos visto la evidencia de que Elena White conocía los sentimientos trinitarios del Dr. Kellogg, y hemos visto cómo respondió a ello. Ella siempre exaltó la verdad sobre la presencia y la \emcap{personalidad de Dios}, y llamó a volver al fundamento de nuestra fe—\emcap{Principios Fundamentales}. Sin embargo, cuando los académicos adventistas discuten la doctrina de la Trinidad y Elena White, no la abordan de la misma manera que Elena White lo hizo. Los \emcap{Principios Fundamentales} junto con la doctrina sobre la \emcap{personalidad de Dios} son minimizados, y se presenta una historia distorsionada de que Elena White era trinitaria y responsable de la aceptación de la doctrina trinitaria en nuestras filas. Queremos desafiar esta historia distorsionada examinando la evidencia que a menudo se utiliza para apoyar esta falsa narrativa.


One of the most prominent quotations to support the claim that Sister White was responsible for accepting the Trinity doctrine into our ranks is her writings and comments on Matthew 28:19\footnote{\bible{Go ye therefore, and teach all nations, baptizing them in the name of the Father, and of the Son, and of the Holy Ghost}[Matthew 28:19]}. The most prominent quotation to stand out in defense of the Trinity doctrine is “\textit{the Heavenly Trio}” quotation:


Una de las citas más prominentes para apoyar la afirmación de que la hermana White fue responsable de aceptar la doctrina trinitaria en nuestras filas son sus escritos y comentarios sobre Mateo 28:19\footnote{\bible{Id, pues, y haced discípulos a todas las naciones, bautizándolos en el nombre del Padre, y del Hijo, y del Espíritu Santo}[Mateo 28:19]}. La cita más prominente que se destaca en defensa de la doctrina trinitaria es la cita del “\textit{trío celestial}”:


\egw{\textbf{There are \underline{three living persons} of the \underline{heavenly trio}}; in the name of these three great powers—\textbf{the Father, the Son, and the Holy Spirit}—those who receive Christ by living faith are baptized, and these powers will co-operate with the obedient subjects of heaven in their efforts to live the new life in Christ...}[Ev 615.1; 1946][https://egwwritings.org/read?panels=p30.3407]


\egw{\textbf{Hay \underline{tres personas vivas} del \underline{trío celestial}}; en el nombre de estos tres grandes poderes—\textbf{el Padre, el Hijo y el Espíritu Santo}—se bautizan los que reciben a Cristo por fe viva, y estos poderes cooperarán con los súbditos obedientes del cielo en sus esfuerzos por vivir la nueva vida en Cristo...}[Ev 615.1; 1946][https://egwwritings.org/read?panels=p30.3407]


To reiterate, this quotation is often cited to argue that Sister White defended and advocated the Trinity doctrine. But, if we take a look at this quotation in its literary context, we see that within the quotation itself she actually \textit{refuted} this doctrine and exalted the truth on the \emcap{personality of God}. To some this is a ludicrous claim, but we invite you to make your judgment based on presented data. Let us examine the context of this quotation.


Para reiterar, esta cita se cita a menudo para argumentar que la hermana White defendió y abogó por la doctrina trinitaria. Pero, si echamos un vistazo a esta cita en su contexto literario, vemos que dentro de la propia cita ella realmente \textit{refutó} esta doctrina y exaltó la verdad sobre la \emcap{personalidad de Dios}. Para algunos esto es una afirmación absurda, pero te invitamos a hacer tu juicio basado en los datos presentados. Examinemos el contexto de esta cita.


\egw{I am instructed to say, \textbf{The sentiments} of those who are searching for advanced scientific ideas \textbf{\underline{are not to be trusted}}. Such representations as the following are made: ‘\textbf{The Father is as the light invisible; the Son is as the light embodied; the Spirit as the light shed abroad.}’ ‘\textbf{The Father is like the dew, invisible vapor; the Son is like the dew gathered in beauteous form; the Spirit is like the dew fallen to the seat of life.}’ Another representation: ‘\textbf{The Father is like the invisible vapor. The Son is like the leaden cloud. The Spirit is rain fallen and working in refreshing power.}’}[Ms21-1906.8; 1906][https://egwwritings.org/read?panels=p9754.15]


\egw{Se me ha instruido decir, \textbf{Los sentimientos} de los que buscan ideas científicas avanzadas \textbf{\underline{no son de fiar}}. Se hacen representaciones como las siguientes: ‘\textbf{El Padre es como la luz invisible; el Hijo es como la luz encarnada; el Espíritu como la luz derramada.}’ ‘\textbf{El Padre es como el rocío, vapor invisible; el Hijo es como el rocío recogido en forma hermosa; el Espíritu es como el rocío caído en la sede de la vida.}’ Otra representación: ‘\textbf{El Padre es como el vapor invisible. El Hijo es como la nube de plomo. El Espíritu es la lluvia caída y obrando con poder refrescante.}’}[Ms21-1906.8; 1906][https://egwwritings.org/read?panels=p9754.15]


What sentiments are not to be trusted? The data suggest that those sentiments are trinitarian ideas of \textit{one God in three persons}. How do we know that? We see in the literary context of the representations Sister White was quoting. Contrary to the popular belief that she was referencing the “\textit{false}” trinity expressed by Dr. Kellogg,\footnote{Whidden, Woodrow W, et al. \textit{The Trinity : Understanding God’s Love, His Plan of Salvation, and Christian Relationships}. Hagerstown, Md, Review And Herald Pub. Association, 2002, p. 216.} she was actually referencing trinitarian idea of \textit{three living persons of one living God}, advocated by William Boardman, in his book “Higher Christian Life”, which she quoted. The context matters. The context of the quotations she quoted, shows that the representations of the Father, the Son, and the Holy Spirit are serving to illustrate the sentiment of three living persons of one God. That is the sentiment we have been clearly instructed by God, not to trust. Let the data be its own interpreter.


¿Qué sentimientos no son de fiar? Los datos sugieren que esos sentimientos son ideas trinitarias de \textit{un Dios en tres personas}. ¿Cómo lo sabemos? Lo vemos en el contexto literario de las representaciones que la hermana White estaba citando. Contrario a la creencia popular de que ella se refería a la “\textit{falsa}” trinidad expresada por el Dr. Kellogg,\footnote{Whidden, Woodrow W, et al. \textit{The Trinity : Understanding God's Love, His Plan of Salvation, and Christian Relationships}. Hagerstown, Md, Review And Herald Pub. Association, 2002, p. 216.} ella en realidad se refería a la idea trinitaria de \textit{tres personas vivas de un Dios vivo}, defendida por William Boardman, en su libro “Higher Christian Life”, que ella citó. El contexto importa. El contexto de las citas que ella citó, muestra que las representaciones del Padre, del Hijo y del Espíritu Santo sirven para ilustrar el sentimiento de tres personas vivas de un solo Dios. Ese es el sentimiento que Dios nos ha instruido claramente a no confiar. Dejemos que los datos sean su propio intérprete.


\section*{The Higher Christian Life, William Boardman}


\section*{The Higher Christian Life, William Boardman}


Ellen White owned William Boardman's book “Higher Christian Life.” It was a good book about Christian sanctification, but in it there was trinitarian sentiment, which Sister White was particularly instructed by God to call out. This is another instance of evidence where we see that Ellen White was familiar with the trinitarian stance, and she was addressing it directly. Let's get familiar with the trinitarian sentiments promoted by William Boardman.


Elena White poseía el libro de William Boardman “Higher Christian Life”. Era un buen libro sobre la santificación cristiana, pero en él había un sentimiento trinitario, que la hermana White fue particularmente instruida por Dios a señalar. Esta es otra instancia de evidencia donde vemos que Elena White estaba familiarizada con la postura trinitaria, y la estaba abordando directamente. Conozcamos los sentimientos trinitarios promovidos por William Boardman.


Speaking of Triune God, William Boardman writes:


Hablando del Dios Trino, William Boardman escribe:


\othersQuote{And then, again, the Father is the author and planner of salvation through faith in his Son; and when we trust in his Son we honor the Father, because we accept of his plan of salvation for us, justify his wisdom, and act in accordance with his will in the matter. \textbf{A glance at the official and essential relations of the persons of the Holy Trinity to each other and to us, may throw additional light upon our pathway}. Upon this subject flippancy would border upon blasphemy. It is holy ground. He who ventures upon it may well tread with unshod foot, and uncovered head bowed low.}[William Boardman, The Higher Christian Life, p. 99; 1858][https://archive.org/details/higherchristian02boargoog/page/n106/]


\othersQuote{Y entonces, de nuevo, el Padre es el autor y planificador de la salvación por medio de la fe en su Hijo; y cuando confiamos en su Hijo honramos al Padre, porque aceptamos su plan de salvación para nosotros, justificamos su sabiduría y actuamos de acuerdo con su voluntad en el asunto. \textbf{Una mirada a las relaciones oficiales y esenciales de las personas de la Santa Trinidad entre sí y con nosotros, puede arrojar luz adicional sobre nuestro camino}. En este tema la ligereza rozaría la blasfemia. Es un terreno sagrado. El que se aventura en él puede pisar con los pies descalzos y la cabeza descubierta inclinada hacia abajo.}[William Boardman, The Higher Christian Life, p. 99; 1858][https://archive.org/details/higherchristian02boargoog/page/n106/]


Brother Boardman wants us to take \others{a glance at the official and essential relations} of the three persons of the Holy Trinity. He asserts that \textit{God is one but also three}–\textit{Triune}–by presenting official and essential relations of the persons of the Holy Trinity. His fundamental statement and outline for his thesis is as follows:


El hermano Boardman quiere que echemos \others{una mirada a las relaciones oficiales y esenciales} de las tres personas de la Santa Trinidad. Afirma que \textit{Dios es uno pero también tres}–\textit{Triuno}–presentando las relaciones oficiales y esenciales de las personas de la Santísima Trinidad. Su declaración fundamental y el esquema de su tesis son los siguientes:


\othersQuote{\textbf{The Father is fullness of the Godhead \underline{invisibly}, without form, whom no creature hath seen or can see}. \\
\textbf{The Son is the fullness of the Godhead \underline{embodied}, that his creatures may see him, and know him, and trust him}. \\
\textbf{The Spirit is the fullness of the Godhead \underline{in all active workings}, whether of creation, providence, revelation, or salvation, by which God manifests himself to and through the universe}.}[William Boardman, The Higher Christian Life, p. 100][https://archive.org/details/higherchristian02boargoog/page/n108/]


\othersQuote{\textbf{El Padre es la plenitud de la Divinidad \underline{invisible}, sin forma, a quien ninguna criatura ha visto ni puede ver}. \\
\textbf{El Hijo es la plenitud de la Divinidad \underline{encarnada}, para que sus criaturas puedan verlo, conocerlo y confiar en él}. \\
\textbf{El Espíritu es la plenitud de la Divinidad \underline{en todas las obras activas}, ya sea de la creación, la providencia, la revelación o la salvación, por las que Dios se manifiesta al universo y a través de él}.}[William Boardman, The Higher Christian Life, p. 100][https://archive.org/details/higherchristian02boargoog/page/n108/]


This statement is foundational to his following statements and illustrations. In the following paragraphs, William Boardman gives the biblical motives to illustrate \others{the official and essential relations of the Holy Trinity}—\textit{that is, God being one, but yet three}. He writes:


Esta declaración es fundamental para sus siguientes declaraciones e ilustraciones. En los siguientes párrafos, William Boardman da los motivos bíblicos para ilustrar \others{las relaciones oficiales y esenciales de la Santa Trinidad}—\textit{es decir, que Dios es uno, pero a la vez tres}. Escribe:


\othersQuote{Another of the names of Jesus will give the same analogies in a light not less striking - \textbf{The Sun of Righteousness}. \\
All the light of the sun in the heavens was once hidden in the invisibility of primal darkness; and after this, the light now blazing in the orb of day was, when first the command when forth, Let light be! and light was, at most only the diffused haze of the gray dawn of the morn of creation out of the darkness of chaotic night, without form, or body, or centre, or radiance, or glory. But when separated from the darkness and centered in the sun, then in its glorious glitter it became so resplendent that none but the eagle eye could bear to look it in the face. \\
But then again its rays falling aslant through earth’s atmosphere and vapors, gladdens all the world with the same light, dispelling the winter, and the cold, and the darkness; starting Spring forth in floral beauty, and Summer in vernal luxuriance, and Autumn laden with golden treasures for the garner.
\textbf{The Father is as the Light invisible}. \\
\textbf{The Son is as the Light embodied}. \\
\textbf{The Spirit is as the Light shed down}.}[William Boardman, The Higher Christian Life, p. 101,102][https://archive.org/details/higherchristian02boargoog/page/n108/]


\othersQuote{Otro de los nombres de Jesús dará las mismas analogías bajo una luz no menos llamativa - \textbf{El Sol de Justicia}. \\
Toda la luz del sol en los cielos estuvo una vez oculta en la invisibilidad de las tinieblas primigenias; y después de esto, la luz que ahora resplandece en el orbe del día era, cuando se dio la primera orden, ¡Sea la luz! y la luz era, a lo sumo, sólo la neblina difusa del gris amanecer de la mañana de la creación a partir de las tinieblas de la noche caótica, sin forma, ni cuerpo, ni centro, ni esplendor, ni gloria. Pero cuando se separó de las tinieblas y se centró en el sol, entonces en su glorioso brillo se hizo tan resplandeciente que nadie más que el ojo del águila podía soportar mirarlo a la cara. \\
Pero luego, de nuevo, sus rayos que caen de forma oblicua a través de la atmósfera y los vapores de la tierra, alegran a todo el mundo con la misma luz, disipando el invierno, el frío y las tinieblas; iniciando la primavera en belleza floral, y el verano en exuberancia vernal, y el otoño cargado de tesoros dorados para el granero.
\textbf{El Padre es como la Luz invisible}. \\
\textbf{El Hijo es como la Luz encarnada}. \\
\textbf{El Espíritu es como la Luz derramada}.}[William Boardman, The Higher Christian Life, p. 101,102][https://archive.org/details/higherchristian02boargoog/page/n108/]


This illustration of the Sun of Righteousness shows that God the Father, who is \textit{the fullness of the Godhead invisible,} can be symbolically illustrated as a Light that \others{was once hidden in the invisibility of primal darkness}. The Son, who is \textit{the fullness of the Godhead embodied}, is like a Light that is embodied in \others{the morn of creation}. The Holy Spirit, who is \textit{the fullness of the Godhead in all active workings}, is like a \others{Light shed down}. William Boardman gives us another similar illustration to clarify the \others{official relations of the persons of the Godhead}:


Esta ilustración del Sol de Justicia muestra que Dios el Padre, que es \textit{la plenitud de la Divinidad invisible,} puede ser simbólicamente ilustrado como una Luz que \others{estaba oculta en la invisibilidad de las tinieblas primigenias}. El Hijo, que es \textit{la plenitud de la Divinidad encarnada}, es como una Luz que se encarna en \others{la mañana de la creación}. El Espíritu Santo, que es \textit{la plenitud de la Divinidad en toda obra activa}, es como una \others{Luz derramada}. William Boardman nos da otra ilustración similar para aclarar las \others{relaciones oficiales de las personas de la Divinidad}:


\othersQuote{One of the similies for blessed influences of the Spirit, \textbf{while giving the self-same official relations of the persons of the Godhead, to each other and to us}, may illustrate them still further,—\textbf{The Dew},—\textbf{The dew of Hermon} - the dew on the mown meadow. Before the dew gathers at all in drops, it hangs over all the landscape in visible vapor, omnipresent but unseen. By and by as the light wanes into morning, and as the temperature sinks and touches the dew point the invisible becomes the visible, the embodied; and, as the sun rises, it stands in diamond drops trembling and glittering in the sun’s young beams in pearly beauty upon leaf and flower, over all the face of nature. \\
But now again, a breeze springs up, the breath of heaven is wafted gently along, shaking leaf and flower, and in a moment the pearly drops are invisible angina. But where now? Fallen at the root of herb and flower to impart new life, freshness, vigor to all it touches. \\
\textbf{The Father is like the dew in invisible vapor}. \\
\textbf{The Son is like the dew gathered in beauteous form}. \\
\textbf{The Spirit is like the dew fallen to the seat of life}.}[William Boardman, The Higher Christian Life, p. 102,103][https://archive.org/details/higherchristian02boargoog/page/n110/]


\othersQuote{Una de las similitudes para las benditas influencias del Espíritu, \textbf{al mismo tiempo que da las mismas relaciones oficiales de las personas de la Divinidad, entre sí y con nosotros}, puede ilustrarlas aún más,—\textbf{El Rocío},—\textbf{El rocío de Hermón} - el rocío en el prado segado. Antes de que el rocío se convierta en gotas, se cierne sobre todo el paisaje en forma de vapor invisible, omnipresente pero invisible. A medida que la luz se desvanece en la mañana, y cuando la temperatura desciende y roza el punto de rocío, lo invisible se convierte en visible, encarnado; y, cuando sale el sol, se presenta en gotas de diamante que tiemblan y brillan bajo los jóvenes rayos del sol en una belleza de perla sobre las hojas y las flores, sobre toda la cara de la naturaleza. \\
Pero ahora, de nuevo, se levanta una brisa, el soplo del cielo se mueve suavemente a lo largo, agitando la hoja y la flor, y en un momento las gotas perladas son invisibles de nuevo. Pero, ¿dónde están ahora? Caen en la raíz de la hierba y la flor para impartir nueva vida, frescura, vigor a todo lo que toca. \\
\textbf{El Padre es como el rocío en el vapor invisible}. \\
\textbf{El Hijo es como el rocío recogido en forma hermosa}. \\
\textbf{El Espíritu es como el rocío caído en la sede de la vida}.}[William Boardman, The Higher Christian Life, p. 102,103][https://archive.org/details/higherchristian02boargoog/page/n110/]


The Father, who is \textit{the fullness of the Godhead invisible,} is illustrated by the \others{dew in invisible vapor}. The Son, who is \textit{the fullness of the Godhead embodied}, is illustrated by \others{the dew gathered in beauteous form}. The Spirit, who is \textit{the fullness of the Godhead in all active works}, is illustrated by \others{the dew fallen to the seat of life}. The next illustration that exemplifies the official relations of the three personalities of one God is by another Bible likening—the Rain.


El Padre, que es \textit{la plenitud de la Divinidad invisible,} es ilustrado por el \others{rocío en vapor invisible}. El Hijo, que es \textit{la plenitud de la Divinidad encarnada}, es ilustrado por \others{el rocío recogido en forma hermosa}. El Espíritu, que es \textit{la plenitud de la Divinidad en todas las obras activas}, es ilustrado por \others{el rocío caído en la sede de la vida}. La siguiente ilustración que ejemplifica las relaciones oficiales de las tres personalidades de un solo Dios es por medio de otra semejanza bíblica—la Lluvia.


\othersQuote{\textbf{Yet one more of these Bible likenings} – by no means exhausting them – will not be unwelcome, or useless, - \textbf{the Rain}. \\
Rain, like the dew, floats in invisibility, and omnipresence at the first, over all, around all. Seen by none. While it remains in its invisibility, the earth parches, clods cleave together, the ground cracks open, the sun pours down his burning heat, the winds lift up the dust in circling whirls, and rolling clouds, and famine gaunt and greedy stalks through the land, followed by pestilence and death. By and by, the eager watcher sees the little hand-like cloud rising far out over the sea. It gathers, gathers, gathers; comes and spreads as it comes, in majesty over the whole heavens: - But all is parched and dry and dead yet, upon earth. \\
But now comes a drop, and drop after drop, quicker, faster – the shower, the rain – sweeping on, and giving to earth all the treasures of the clouds – clods open, furrows soften, springs, rivulets, rivers, swell and fill, and all the land is gladdened again with restored abundance. \\
\textbf{The Father is like to the invisible vapor}. \\
\textbf{The Son is as the laden cloud and falling rain}. \\
\textbf{The Spirit is the Rain – fallen and working in refreshing power}.}[William Boardman, The Higher Christian Life, p. 103,104][https://archive.org/details/higherchristian02boargoog/page/n110/]


\othersQuote{\textbf{Sin embargo, una más de estas comparaciones bíblicas} – que no las agota en absoluto – no será inoportuna ni inútil, - \textbf{la Lluvia}. \\
La lluvia, como el rocío, flota en la invisibilidad y la omnipresencia al principio, sobre todo, alrededor de todo. No es vista por nadie. Mientras permanece en su invisibilidad, la tierra se reseca, los terrones se parten, el suelo se abre, el sol derrama su calor abrasador, los vientos levantan el polvo en remolinos y nubes ondulantes, y el hambre, demacrada y codiciosa, recorre la tierra, seguida de la peste y la muerte. El observador ansioso ve la pequeña nube en forma de mano que se eleva sobre el mar. Se reúne, se reúne, se reúne; viene y se extiende, con majestuosidad, sobre todo el cielo: - Pero todo está reseco y seco y muerto aún, sobre la tierra. \\
Pero ahora viene una gota, y gota tras gota, más rápida, más veloz - la lluvia, la lluvia - barriendo, y dando a la tierra todos los tesoros de las nubes - los terrones se abren, los surcos se suavizan, los manantiales, los riachuelos, los ríos, se hinchan y se llenan, y toda la tierra se alegra de nuevo con la abundancia restaurada. \\
\textbf{El Padre es como el vapor invisible}. \\
\textbf{El Hijo es como la nube cargada y la lluvia que cae}. \\
\textbf{El Espíritu es la Lluvia - caída y obrando con poder refrescante}.}[William Boardman, The Higher Christian Life, p. 103,104][https://archive.org/details/higherchristian02boargoog/page/n110/]


Let's give William Boardman a fair hearing. He is not saying that the Father is \others{invisible vapor}; rather, he uses a metaphor of rain and \others{invisible vapor} to illustrate his main point that the Father is the invisible fullness of the Godhead. So it is with the Son, who, just like rain manifested in leaden clouds, is all the fullness of the Godhead manifested. To ensure his sentiments are not potentially misrepresented, William Boardman clarified his sentiment. This was the very sentiment that Ellen White was instructed by God not to trust:


Démosle a William Boardman una audiencia justa. Él no está diciendo que el Padre es \others{vapor invisible}; más bien, utiliza una metáfora de la lluvia y el \others{vapor invisible} para ilustrar su punto principal de que el Padre es la plenitud invisible de la Divinidad. Así es con el Hijo, quien, al igual que la lluvia manifestada en nubes cargadas, es toda la plenitud de la Divinidad manifestada. Para asegurarse de que sus sentimientos no sean potencialmente tergiversados, William Boardman aclaró su sentimiento. Este fue el mismo sentimiento que a Elena White se le instruyó por Dios que no confiara:


\othersQuote{\textbf{These likenings are all imperfect. They rather hide than illustrate \underline{the tri-personality of the one God}, for they are not persons but things, poor and earthly at best, to represent the living personalities of the living God. So much they may do, however, as to illustrate the official relations of each to the others and of each and all to us. And more. They may also illustrate the truth that all the fulness of Him who filleth all in all, dwells in each person of \underline{the Triune God}}. \\
\textbf{The Father is all the fulness of the Godhead INVISIBLE}. \\
\textbf{The Son is all the fulness of the Godhead MANIFESTED}. \\
\textbf{The Spirit is all the fulness of the Godhead MAKING MANIFEST}. \\
\textbf{The persons are not mere offices, or modes of revelation, but living persons of the living God}.}[William Boardman, The Higher Christian Life, p. 104,105][https://archive.org/details/higherchristian02boargoog/page/n112/]


\othersQuote{\textbf{Estas comparaciones son todas imperfectas. Más bien ocultan que ilustran \underline{la tri-personalidad del Dios único}, porque no son personas sino cosas, pobres y terrenales en el mejor de los casos, para representar las personalidades vivas del Dios vivo. Sin embargo, pueden servir para ilustrar las relaciones oficiales de cada uno con los demás y de cada uno y todos con nosotros. Y más aún. También pueden ilustrar la verdad de que toda la plenitud de Aquel que llena todo en todos, habita en cada persona del \underline{Dios Trino}}. \\
\textbf{El Padre es toda la plenitud de la Divinidad INVISIBLE}. \\
\textbf{El Hijo es toda la plenitud de la Divinidad MANIFESTADA}. \\
\textbf{El Espíritu es toda la plenitud de la Divinidad MANIFESTÁNDOSE}. \\
\textbf{Las personas no son meros oficios, o modos de revelación, sino personas vivas del Dios vivo}.}[William Boardman, The Higher Christian Life, p. 104,105][https://archive.org/details/higherchristian02boargoog/page/n112/]


It is crucial to emphasize that when Boardman uses these Bible likenings from nature, he speaks of the illustrations, and not reality. These representations are illustrating his sentiments. In his own admission, that was the sentiment of three \others{living personalities of the living God.} Though these illustrations are imperfect, they may \others{illustrate the official relations} of \others{the tri-personality of the one God} and \others{the truth that all the fullness of Him who filleth all in all dwells in each person of the Triune God.} One God in three persons is the sentiment in question, and that sentiment is common to all types and versions of the trinity doctrine—including our current trinitarian stance in the second point of the Fundamental Beliefs.\footnote{\others{There is \textbf{one God}: Father, Son, and Holy Spirit, \textbf{a unity of three} coeternal \textbf{Persons}…} 2nd point of the Fundamental Beliefs}


Es crucial enfatizar que cuando Boardman utiliza estas comparaciones bíblicas de la naturaleza, habla de las ilustraciones, y no de la realidad. Estas representaciones están ilustrando sus sentimientos. En su propia admisión, ese era el sentimiento de tres \others{personalidades vivas del Dios vivo.} Aunque estas ilustraciones son imperfectas, pueden \others{ilustrar las relaciones oficiales} de \others{la tri-personalidad del Dios único} y \others{la verdad de que toda la plenitud de Aquel que llena todo en todos, habita en cada persona del Dios Trino.} Un Dios en tres personas es el sentimiento en cuestión, y ese sentimiento es común a todos los tipos y versiones de la doctrina trinitaria—incluyendo nuestra actual posición trinitaria en el segundo punto de las Creencias Fundamentales.\footnote{\others{Hay \textbf{un Dios}: Padre, Hijo y Espíritu Santo, \textbf{una unidad de tres} \textbf{Personas} coeternas...} 2do punto de las Creencias Fundamentales}


In this brief look at William Boardman's sentiments, it is clear that the sentiments in question which Ellen White was instructed by God to call out, were the sentiments of the Triune God, or \textit{three living persons in the Trinity}. With that data in mind, let's examine Ellen White's response.


En este breve vistazo a los sentimientos de William Boardman, queda claro que los sentimientos en cuestión que Ellen White fue instruida por Dios para señalar, eran los sentimientos del Dios Trino, o \textit{tres personas vivas en la Trinidad}. Con esos datos en mente, examinemos la respuesta de Ellen White.


\section*{Ellen White on William Boardman’s sentiment}


\section*{Ellen White sobre el sentimiento de William Boardman}


With the Heavenly Trio quotation, it has been asserted that Ellen White was trinitarian. This is done by ignorantly or sometimes purposely ignoring the context of this valuable quotation. When reading Ellen White’s response, in which she defends our perceptions of God, try to recognize whom she is addressing when she speaks of God. Was the God she defended the Trinity or the Father? Referencing William Boardmans illustrations she said:


Con la cita del Trío Celestial, se ha afirmado que Ellen White era trinitaria. Esto se hace ignorando ingenuamente o a veces a propósito el contexto de esta valiosa cita. Al leer la respuesta de Ellen White, en la que defiende nuestras percepciones de Dios, trate de reconocer a quién se dirige cuando habla de Dios. ¿Era el Dios que ella defendía la Trinidad o el Padre? Refiriéndose a las ilustraciones de William Boardman, ella dijo:


\egw{\textbf{All these \underline{spiritualistic} representations are simply nothingness}. They are imperfect, untrue. They weaken and diminish the Majesty which no earthly likeness can be compared to. \textbf{God cannot be compared with the things His hands have made}. These are mere earthly things, suffering under the curse of God because of the sins of man. \textbf{The Father cannot be described by the things of earth}. \textbf{The Father is all the fulness of the Godhead \underline{bodily} and is \underline{invisible to mortal sight}}.}[Ms21-1906.9; 1906][https://egwwritings.org/read?panels=p9754.15]


\egw{\textbf{Todas estas representaciones \underline{espiritualistas} son simplemente la nada}. Son imperfectas, falsas. Debilitan y disminuyen la Majestad a la que ninguna semejanza terrenal puede compararse. \textbf{Dios no puede ser comparado con las cosas que sus manos han hecho}. Estas son meras cosas terrenales, que sufren bajo la maldición de Dios a causa de los pecados del hombre. \textbf{El Padre no puede ser descrito por las cosas de la tierra}. \textbf{El Padre es toda la plenitud de la Divinidad \underline{corporalmente} y es \underline{invisible a la vista de los mortales}}.}[Ms21-1906.9; 1906][https://egwwritings.org/read?panels=p9754.15]


By observing the context, it is obvious that Sister White follows Boardman’s line of reasoning and corrects the mistakes. For better comparison, let us look at their writings side by side:


Al observar el contexto, es obvio que la hermana White sigue la línea de razonamiento de Boardman y corrige los errores. Para una mejor comparación, veamos sus escritos lado a lado:


\begin{table}[H]
\centering
\renewcommand{\arraystretch}{1.5}
\setlength{\tabcolsep}{15pt}
\resizebox{\textwidth}{!}{
\begin{tabular}{|p{0.4\textwidth}|p{0.4\textwidth}|}
\hline
\multicolumn{1}{|c|}{\textbf{William Boardman}} & \multicolumn{1}{c|}{\textbf{Ellen G. White}} \\ \hline
\othersQuote{These likenings are all imperfect. They rather hide than \textbf{illustrate the tri-personality of the \underline{one God}}, for they are not persons but things, poor and earthly at best, to represent \textbf{the living personalities of the living God}. \textbf{So much they may do, however, as to illustrate the official relations of each to the other and of each and all to us. And more. They may also illustrate the truth that all the fulness of Him who filleth all in all, dwells in \underline{each person of Triune God}}.}[p. 104,105][https://archive.org/details/higherchristian02boargoog/page/n112] & 
\egw{\textbf{All these \underline{spiritualistic} representations are simply nothingness}. They are imperfect, untrue. They weaken and diminish the Majesty which no earthly likeness can be compared to. \textbf{God cannot be compared with the things His hands have made}. These are mere earthly things, suffering under the curse of God because of the sins of man. \textbf{The Father cannot be described by the things of earth}.}[Ms21-1906.9; 1906][https://egwwritings.org/read?panels=p9754.15] \\ \hline
\end{tabular}
}
\end{table}


\begin{table}[H]
\centering
\renewcommand{\arraystretch}{1.5}
\setlength{\tabcolsep}{15pt}
\resizebox{\textwidth}{!}{
\begin{tabular}{|p{0.4\textwidth}|p{0.4\textwidth}|}
\hline
\multicolumn{1}{|c|}{\textbf{William Boardman}} & \multicolumn{1}{c|}{\textbf{Ellen G. White}} \\ \hline
\othersQuote{Estas comparaciones son todas imperfectas. Más bien ocultan que \textbf{ilustran la tri-personalidad del \underline{Dios único}}, porque no son personas sino cosas, pobres y terrenales en el mejor de los casos, para representar \textbf{las personalidades vivas del Dios vivo}. \textbf{Sin embargo, pueden servir para ilustrar las relaciones oficiales de cada uno con el otro y de cada uno y todos con nosotros. Y más aún. También pueden ilustrar la verdad de que toda la plenitud de Aquel que llena todo en todos, habita en \underline{cada persona del Dios Trino}}.}[p. 104,105][https://archive.org/details/higherchristian02boargoog/page/n112] & 
\egw{\textbf{Todas estas representaciones \underline{espiritualistas} son simplemente la nada}. Son imperfectas, falsas. Debilitan y disminuyen la Majestad a la que ninguna semejanza terrenal puede compararse. \textbf{Dios no puede ser comparado con las cosas que sus manos han hecho}. Estas son meras cosas terrenales, que sufren bajo la maldición de Dios a causa de los pecados del hombre. \textbf{El Padre no puede ser descrito por las cosas de la tierra}.}[Ms21-1906.9; 1906][https://egwwritings.org/read?panels=p9754.15] \\ \hline
\end{tabular}
}
\end{table}


In this comparison, it is clear who God is for William Boardman, and who He is for Sister White. For Boardman, God is the Triune God, a tri-personality of the one God. For Sister White, God is the Father. For Boardman, these representations are imperfect because they \others{rather hide than illustrate the tri-personality of the one God}, and for Sister White these representations are imperfect because \egw{The Father cannot be described by the things of earth}. For Boardman, God is the \textit{Triune God}; for Sister White, God is \textit{the Father}.


En esta comparación, queda claro quién es Dios para William Boardman, y quién es para la hermana White. Para Boardman, Dios es el Dios Trino, una tri-personalidad del Dios único. Para la hermana White, Dios es el Padre. Para Boardman, estas representaciones son imperfectas porque \others{más bien ocultan que ilustran la tri-personalidad del Dios único}, y para la hermana White estas representaciones son imperfectas porque \egw{El Padre no puede ser descrito por las cosas de la tierra}. Para Boardman, Dios es el \textit{Dios Trino}; para la hermana White, Dios es \textit{el Padre}.


Boardman’s only point that Ellen White affirms is that these representations are imperfect. Surely, William Boardman would not agree with Ellen White that these representations are \textit{spiritualistic} and \textit{untrue}. On the contrary, he believes that these illustrations \others{illustrate the truth that all the fulness of Him who filleth all in all, dwells in each person of Triune God}. To say that Ellen White agreed with such sentiment is gross misrepresentation.


El único punto de Boardman que Ellen White afirma es que estas representaciones son imperfectas. Seguramente, William Boardman no estaría de acuerdo con Ellen White en que estas representaciones son \textit{espiritualistas} y \textit{falsas}. Por el contrario, él cree que estas ilustraciones \others{ilustran la verdad de que toda la plenitud de Aquel que llena todo en todos, habita en cada persona del Dios Trino}. Decir que Ellen White estaba de acuerdo con tal sentimiento es una grave tergiversación.


The context of this important quotation prompts important questions. Why does the prophet of God refer to the representations that illustrate the \others{tri-personality of the one God} as \egwinline{spiritualistic representations}, which illustrate the sentiment that \egwinline{is not to be trusted}? Or why does the prophet of God refer to the representations that \others{represent the living personalities of the living God} as \egwinline{spiritualistic representations}? Or why does the prophet of God, when referring to the representations that \others{illustrate the truth that all the fullness of Him who filleth all in all, dwells in each person of Triune God}, refer to them as \egwinline{spiritualistic representations}? All of these spiritualistic representations illustrate the sentiment that \egwinline{is not to be trusted}. This sentiment is clearly the trinitarian sentiment.


El contexto de esta importante cita suscita importantes preguntas. ¿Por qué la profeta de Dios se refiere a las representaciones que ilustran la \others{tripersonalidad del Dios único} como \egwinline{representaciones espiritualistas}, que ilustran el sentimiento que \egwinline{no es de fiar}? ¿O por qué la profeta de Dios se refiere a las representaciones que \others{representan las personalidades vivas del Dios vivo} como \egwinline{representaciones espiritualistas}? ¿O por qué la profeta de Dios, al referirse a las representaciones que \others{ilustran la verdad de que toda la plenitud de Aquel que llena todo en todos, habita en cada persona del Dios Trino}, se refiere a ellas como \egwinline{representaciones espiritualistas}? Todas estas representaciones espiritualistas ilustran el sentimiento que \egwinline{no es de fiar}. Este sentimiento es claramente el sentimiento trinitario.


Sister White continues to follow Boardman’s line of reasoning and corrects the error.


La hermana White continúa siguiendo la línea de razonamiento de Boardman y corrige el error.


\begin{table}[H]
\centering
\renewcommand{\arraystretch}{1.5}
\setlength{\tabcolsep}{15pt}
\resizebox{\textwidth}{!}{
\begin{tabular}{|p{0.4\textwidth}|p{0.4\textwidth}|}
\hline
\multicolumn{1}{|c|}{\textbf{William Boardman}} & \multicolumn{1}{c|}{\textbf{Ellen G. White}} \\ \hline
\othersQuote{The Father is fullness of the Godhead \textbf{invisibly}, \textbf{\underline{without form}}, whom \textbf{no creature hath seen \underline{or can see}}.}[p.100][https://archive.org/details/higherchristian02boargoog/page/n108/]

\othersQuote{The Father is all the fullness of the Godhead \textbf{INVISIBLE}.}[p.105][https://archive.org/details/higherchristian02boargoog/page/n112/] & 
\egw{The Father is all the fulness of the Godhead \textbf{\underline{bodily}}, and is \textbf{invisible to mortal sight}.}[Ms21-1906.9; 1906][https://egwwritings.org/read?panels=p9754.15] \\ \hline
\end{tabular}
}
\end{table}


\begin{table}[H]
\centering
\renewcommand{\arraystretch}{1.5}
\setlength{\tabcolsep}{15pt}
\resizebox{\textwidth}{!}{
\begin{tabular}{|p{0.4\textwidth}|p{0.4\textwidth}|}
\hline
\multicolumn{1}{|c|}{\textbf{William Boardman}} & \multicolumn{1}{c|}{\textbf{Ellen G. White}} \\ \hline
\othersQuote{El Padre es la plenitud de la Divinidad \textbf{invisible}, \textbf{\underline{sin forma}}, a quien \textbf{ninguna criatura ha visto \underline{ni puede ver}}.}[p.100][https://archive.org/details/higherchristian02boargoog/page/n108/]

\othersQuote{El Padre es toda la plenitud de la Divinidad \textbf{INVISIBLE}.}[p.105][https://archive.org/details/higherchristian02boargoog/page/n112/] & 
\egw{El Padre es toda la plenitud de la Divinidad \textbf{\underline{corporalmente}}, y es \textbf{invisible a la vista de los mortales}.}[Ms21-1906.9; 1906][https://egwwritings.org/read?panels=p9754.15] \\ \hline
\end{tabular}
}
\end{table}


For Boardman, the Father does not have a form nor body and is invisible to all creatures. For Sister White, the Father has a form and body and is invisible only to mortal human beings.\footnote{When Sister White talks about mortals, she talks about sin polluted humanity. After the restoration of humanity, at the resurrection, Christ will give His immortal life to His children. For more information read \href{https://egwwritings.org/?ref=en_RH.July.5.1887.par.5}{EGW, RH July 5, 1887, par. 5; 1887}.}


Para Boardman, el Padre no tiene forma ni cuerpo y es invisible para todas las criaturas. Para la hermana White, el Padre tiene forma y cuerpo y es invisible sólo para los seres humanos mortales.\footnote{Cuando la hermana White habla de mortales, habla de la humanidad contaminada por el pecado. Después de la restauración de la humanidad, en la resurrección, Cristo dará Su vida inmortal a Sus hijos. Para más información lea \href{https://egwwritings.org/?ref=en_RH.July.5.1887.par.5}{EGW, RH July 5, 1887, par. 5; 1887}.}


This quotation is one of the most direct quotations regarding the \emcap{personality of God}. \egwinline{The Father is all the fullness of the Godhead \textbf{bodily}}[Ms21-1906.9; 1906][https://egwwritings.org/read?panels=p9754.16].


Esta cita es una de las más directas sobre la \emcap{personalidad de Dios}. \egwinline{El Padre es toda la plenitud de la Divinidad \textbf{corporalmente}}[Ms21-1906.9; 1906][https://egwwritings.org/read?panels=p9754.16].


It might be confusing to someone that the Father is all the fullness of the Godhead bodily because in \textit{Colossians 2:9}, when referring to Jesus, it is written that \bible{in him dwelleth all the fulness of the Godhead bodily.} Scripture does not contradict itself. \textit{Colossians 2:9} does not exclude the Father to be all the fulness of the Godhead bodily. Various places in the Bible describe the Father having a body (\textit{a form: Daniel 7:9,10; Revelation 4:2,3; 1 Kings 22:19-22; a shape: John 5:37}). He has the appearance of a man (\textit{Ezekiel 1:26-28}). He has a face (\textit{Exodus 33:20; Matthew 18:10; Revelation 22:3, 4}). However, the Bible is completely silent about the nature of its substance. The Bible teaches us that \bible{\textbf{The secret things belong unto the LORD our God}: \textbf{but those things which \underline{are revealed} belong unto us and to our children for ever}, that we may do all the words of this law}[Deuteronomy 29:29]. It is revealed to us that the Father has body, He is all the fulness of the Godhead bodily. Also, it is revealed that in Jesus also dwells all the fulness of the Godhead bodily, because \bible{it pleased the Father that in him should all fulness dwell}[Colossians 1:19]. This is not a contradiction whatsoever because the Son is \bible{the \textbf{express image of \underline{His person}}}[Hebrews 1:3].


Podría ser confuso para alguien que el Padre sea toda la plenitud de la Divinidad corporalmente porque en \textit{Colosenses 2:9}, al referirse a Jesús, está escrito que \bible{en él habita toda la plenitud de la Divinidad corporalmente.} La Escritura no se contradice. \textit{Colosenses 2:9} no excluye que el Padre sea toda la plenitud de la Divinidad corporalmente. Varios lugares en la Biblia describen al Padre teniendo un cuerpo (\textit{una forma: Daniel 7:9,10; Apocalipsis 4:2,3; 1 Reyes 22:19-22; una figura: Juan 5:37}). Tiene apariencia de hombre (\textit{Ezequiel 1:26-28}). Tiene un rostro (\textit{Éxodo 33:20; Mateo 18:10; Apocalipsis 22:3, 4}). Sin embargo, la Biblia guarda completo silencio sobre la naturaleza de su sustancia. La Biblia nos enseña que \bible{\textbf{Las cosas secretas pertenecen a Jehová nuestro Dios}: \textbf{mas las reveladas son para nosotros y para nuestros hijos para siempre}, para que cumplamos todas las palabras de esta ley}[Deuteronomio 29:29]. Se nos revela que el Padre tiene cuerpo, Él es toda la plenitud de la Divinidad corporalmente. Además, se revela que en Jesús también habita toda la plenitud de la Divinidad corporalmente, porque \bible{al Padre le agradó que en él habitara toda la plenitud}[Colosenses 1:19]. Esto no es una contradicción en absoluto porque el Hijo es \bible{la \textbf{imagen misma de \underline{su sustancia}}}[Hebreos 1:3].


\begin{table}[H]
\centering
\renewcommand{\arraystretch}{1.5}
\setlength{\tabcolsep}{15pt}
\resizebox{\textwidth}{!}{
\begin{tabular}{|p{0.4\textwidth}|p{0.4\textwidth}|}
\hline
\multicolumn{1}{|c|}{\textbf{William Boardman}} & \multicolumn{1}{c|}{\textbf{Ellen G. White}} \\ \hline
\othersQuote{The Son is the fullness of the Godhead \textbf{embodied, that his creatures may see him, and know him, and trust him}.}[p.100][https://archive.org/details/higherchristian02boargoog/page/n108/]

\othersQuote{The Son is all the fulness of the Godhead \textbf{MANIFESTED}.}[p.105][https://archive.org/details/higherchristian02boargoog/page/n112/] & 
\egw{The Son is all the fulness of the Godhead \textbf{manifested}. The Word of God declares Him to be ‘\textbf{the express image of His person}’. ‘God so loved the world that He gave \textbf{His only begotten Son}, that whosoever believeth in Him should not perish, but have everlasting life’. \textbf{Here is shown \underline{the personality of the Father}}.}[Ms21-1906.10; 1906][https://egwwritings.org/read?panels=p9754.17] \\ \hline
\end{tabular}
}
\end{table}


\begin{table}[H]
\centering
\renewcommand{\arraystretch}{1.5}
\setlength{\tabcolsep}{15pt}
\resizebox{\textwidth}{!}{
\begin{tabular}{|p{0.4\textwidth}|p{0.4\textwidth}|}
\hline
\multicolumn{1}{|c|}{\textbf{William Boardman}} & \multicolumn{1}{c|}{\textbf{Ellen G. White}} \\ \hline
\othersQuote{El Hijo es la plenitud de la Divinidad \textbf{encarnada, para que sus criaturas puedan verlo, conocerlo y confiar en él}.}[p.100][https://archive.org/details/higherchristian02boargoog/page/n108/]

\othersQuote{El Hijo es toda la plenitud de la Divinidad \textbf{MANIFESTADA}.}[p.105][https://archive.org/details/higherchristian02boargoog/page/n112/] & 
\egw{El Hijo es toda la plenitud de la Divinidad \textbf{manifestada}. La Palabra de Dios declara que Él es ‘\textbf{la imagen misma de su sustancia}’. ‘Tanto amó Dios al mundo que dio a \textbf{su Hijo unigénito}, para que todo el que crea en él no perezca, sino que tenga vida eterna’. \textbf{Aquí se muestra \underline{la personalidad del Padre}}.}[Ms21-1906.10; 1906][https://egwwritings.org/read?panels=p9754.17] \\ \hline
\end{tabular}
}
\end{table}


Sister White focused on the \emcap{personality of God}, which is the personality of the Father. In Christ, who is \egwinline{begotten in the express image of the Father’s person}[ST May 30, 1895, par. 3; 1895][https://egwwritings.org/read?panels=p820.12891], is shown the personality of the Father. In the same way that Jesus is a person, so is the Father. The quality or state of Christ being a person is the same quality or state of the Father being a person. As Christ is a personal being, so is the Father. Just as all the fullness of the Godhead bodily dwells in Christ, so it does in the Father, because Christ is begotten in the express image of the Father’s person. In Him is shown the personality of the Father. These simple conclusions have been asserted by Scripture in John 3:16 and Hebrews 1:3.


La hermana White se enfocó en la \emcap{personalidad de Dios}, que es la personalidad del Padre. En Cristo, que es \egwinline{engendrado a la imagen expresa de la persona del Padre}[ST May 30, 1895, par. 3; 1895][https://egwwritings.org/read?panels=p820.12891], se muestra la personalidad del Padre. Del mismo modo que Jesús es una persona, el Padre también lo es. La cualidad o estado de Cristo como persona es la misma cualidad o estado del Padre como persona. Así como Cristo es un ser personal, también lo es el Padre. Así como toda la plenitud de la Divinidad habita corporalmente en Cristo, así lo hace en el Padre, porque Cristo es engendrado a la imagen expresa de la persona del Padre. En Él se muestra la personalidad del Padre. Estas simples conclusiones han sido afirmadas por la Escritura en Juan 3:16 y Hebreos 1:3.


Does the same reasoning, of the personality of the Father and Son, apply to the Holy Spirit? Speaking of the Holy Spirit, Sister White continues:


¿Se aplica el mismo razonamiento, de la personalidad del Padre y del Hijo, al Espíritu Santo? Hablando del Espíritu Santo, la hermana White continúa:


\egw{\textbf{The Comforter that Christ} promised to send after He ascended to heaven, \textbf{is the Spirit \underline{in} all the fulness of the Godhead}, making manifest the power of divine grace to all who receive and believe in Christ as a personal Saviour.}[Ms21-1906.11; 1906][https://egwwritings.org/read?panels=p9754.18]


\egw{\textbf{El Consolador que Cristo} prometió enviar después de ascender al cielo, \textbf{es el Espíritu \underline{en} toda la plenitud de la Deidad}, que manifiesta el poder de la gracia divina a todos los que reciben y creen en Cristo como Salvador personal.}[Ms21-1906.11; 1906][https://egwwritings.org/read?panels=p9754.18]


Sister White draws a distinction between Father and Son who \textbf{are}, individually, \textbf{all} the fullness of the Godhead, and the Spirit that is \textbf{in all} the fullness of the Godhead. This is a marked contrast to William Boardman’s reasoning, where all three are the fullness of the Godhead. Sister White does not follow this trinitarian fashion. The explanation is simple in light of the \emcap{personality of God} and of Christ. The Holy Spirit is a spirit, and the spirit dwells \textbf{in} the flesh/body. The Holy Spirit is \textbf{in all} the fullness of the Godhead\footnote{Take a look at the quotation from \href{https://egwwritings.org/?ref=en_Ms128-1897.13&para=5426.19}{{EGW, Ms128-1897.13; 1897}}, where Sister White states that the Father and the Son are the absolute Godhead.}.


La hermana White establece una distinción entre el Padre y el Hijo que \textbf{son}, individualmente, \textbf{toda} la plenitud de la Divinidad, y el Espíritu que está \textbf{en toda} la plenitud de la Divinidad. Esto es un marcado contraste con el razonamiento de William Boardman, donde los tres son la plenitud de la Divinidad. La hermana White no sigue esta moda trinitaria. La explicación es sencilla a la luz de la \emcap{personalidad de Dios} y de Cristo. El Espíritu Santo es un espíritu, y el espíritu mora \textbf{en} la carne/cuerpo. El Espíritu Santo está \textbf{en toda} la plenitud de la Divinidad\footnote{Eche un vistazo a la cita de \href{https://egwwritings.org/?ref=en_Ms128-1897.13&para=5426.19}{{EGW, Ms128-1897.13; 1897}}, donde la hermana White afirma que el Padre y el Hijo son la Divinidad absoluta.}.


Finally, the quotation continues to its most renowned part:


Finalmente, la cita continúa con su parte más conocida:


\begin{table}[H]
    \centering
    \renewcommand{\arraystretch}{1.5}
    \setlength{\tabcolsep}{15pt}
    \resizebox{\textwidth}{!}{
    \begin{tabular}{|p{0.4\textwidth}|p{0.4\textwidth}|}
    \hline
    \multicolumn{1}{|c|}{\textbf{William Boardman}} & \multicolumn{1}{c|}{\textbf{Ellen G. White}} \\ \hline
    \othersQuote{\textbf{The Father} is all the fulness of the Godhead INVISIBLE.}

    \othersQuote{\textbf{The Son} is all the fulness of the Godhead MANIFESTED.}

    \othersQuote{\textbf{The Spirit} is all the fulness of the Godhead MAKING MANIFEST.}


    \othersQuote{\textbf{The persons} are not mere offices, or modes of revelation, \textbf{but living persons of the living God}.}[p.105][https://archive.org/details/higherchristian02boargoog/page/n112/] & 
    \egw{There are \textbf{three living persons of the heavenly trio}; in the name of these three great powers—\textbf{the Father, the Son, and the Holy Spirit}—those who receive Christ by living faith are baptized, and these powers will co-operate with the obedient subjects of heaven in their efforts to live the new life in Christ.}[Ms21-1906.11; 1906][https://egwwritings.org/read?panels=p9754.18] \\ \hline
    \end{tabular}
    }
\end{table}

\begin{table}[H]
    \centering
    \renewcommand{\arraystretch}{1.5}
    \setlength{\tabcolsep}{15pt}
    \resizebox{\textwidth}{!}{
    \begin{tabular}{|p{0.4\textwidth}|p{0.4\textwidth}|}
    \hline
    \multicolumn{1}{|c|}{\textbf{William Boardman}} & \multicolumn{1}{c|}{\textbf{Ellen G. White}} \\ \hline
    \othersQuote{\textbf{El Padre} es toda la plenitud de la Divinidad INVISIBLE.}

    \othersQuote{\textbf{El Hijo} es toda la plenitud de la Divinidad MANIFESTADA.}

    \othersQuote{\textbf{El Espíritu} es toda la plenitud de la Divinidad MANIFESTANDO.}

    \othersQuote{\textbf{Las personas} no son meros oficios, o modos de revelación, \textbf{sino personas vivas del Dios vivo}.}[p.105][https://archive.org/details/higherchristian02boargoog/page/n112/] & 
    \egw{Hay \textbf{tres personas vivas del trío celestial}; en el nombre de estos tres grandes poderes—\textbf{el Padre, el Hijo y el Espíritu Santo}—se bautizan los que reciben a Cristo por fe viva, y estos poderes cooperarán con los súbditos obedientes del cielo en sus esfuerzos por vivir la nueva vida en Cristo.}[Ms21-1906.11; 1906][https://egwwritings.org/read?panels=p9754.18] \\ \hline
    \end{tabular}
    }
\end{table}


In light of the context of William Boardman’s book, we see a marked contrast between \others{three living persons of \textbf{one living God}}, which is the trinitarian sentiment, and \egwinline{the three living persons of \textbf{the heavenly trio}}, which is in accordance with the truth on the \emcap{personality of God}.


A la luz del contexto del libro de William Boardman, vemos un marcado contraste entre \others{tres personas vivas de \textbf{un Dios vivo}}, que es el sentimiento trinitario, y \egwinline{las tres personas vivas de \textbf{el trío celestial}}, que está de acuerdo con la verdad sobre la \emcap{personalidad de Dios}.


The word ‘\textit{trio}’ simply indicates the group of three. The \textit{“heavenly trio}” is represented by the Father, the Son, and the Holy Spirit. But, contrary to popular assumption, they do not make one living God. Three-in-one and one-in-three are concepts that do away with the \emcap{personality of God}. This is why Sister White referred to trinitarian sentiments as sentiments that \egwinline{are not to be trusted}[Ms21-1906.8; 1906][https://egwwritings.org/read?panels=p9754.15].


La palabra ‘\textit{trío}’ simplemente indica el grupo de tres. El \textit{“trío celestial”} está representado por el Padre, el Hijo y el Espíritu Santo. Pero, contrariamente a la suposición popular, no hacen un Dios vivo. Tres-en-uno y uno-en-tres son conceptos que eliminan la \emcap{personalidad de Dios}. Por eso la hermana White se refirió a los sentimientos trinitarios como sentimientos que \egwinline{no son de fiar}[Ms21-1906.8; 1906][https://egwwritings.org/read?panels=p9754.15].


Sister White never followed any trinitarian fashion—neither in words and expressions, nor in sentiments. There is an almost effortless research endeavor we encourage you to take: in the writings of Ellen White, search for standard trinitarian terms like “\textit{three are one},” “\textit{one are three},” “\textit{one in three},” “\textit{three in one},” or any of the permutations possible. In her impressive oeuvre you will not find a single occurrence of any of these, let alone the word ‘\textit{trinity}’ describing our God\footnote{There is but one occurrence, in the writings of Ellen White, of the word ‘\textit{trinity}’ referring to \egw{the lust of the flesh, the lust of the eyes and the pride of life}[Lt43-1898.25; 1898][https://egwwritings.org/read?panels=p4806.31]}. She never used these phrases that are necessary to explain the trinitarian sentiment. Examining the following quote, we can see why she never said that God is trinity.


La hermana White nunca siguió ninguna moda trinitaria—ni en palabras y expresiones, ni en sentimientos. Hay un esfuerzo de investigación casi sin esfuerzo que te alentamos a emprender: en los escritos de Ellen White, busca términos trinitarios estándar como “\textit{tres son uno},” “\textit{uno son tres},” “\textit{uno en tres},” “\textit{tres en uno},” o cualquiera de las permutaciones posibles. En su impresionante obra no encontrarás ni una sola ocurrencia de ninguno de estos, y mucho menos la palabra ‘\textit{trinidad}’ describiendo a nuestro Dios\footnote{Hay solo una ocurrencia, en los escritos de Ellen White, de la palabra ‘\textit{trinity}’ refiriéndose a \egw{la concupiscencia de la carne, la concupiscencia de los ojos y la soberbia de la vida}[Lt43-1898.25; 1898][https://egwwritings.org/read?panels=p4806.31]}. Ella nunca usó estas frases que son necesarias para explicar el sentimiento trinitario. Examinando la siguiente cita, podemos ver por qué ella nunca dijo que Dios es trinidad.


\egw{The subject of \textbf{\underline{speculation} regarding \underline{God’s personality} \underline{we will not venture} to express}, \textbf{\underline{except in the language of the Word which represents His personality}}. There is to be no discussion over this question \textbf{lest God would give unmistakable revelation of \underline{what He is}} that would extinguish the one who dares venture on the holy ground in \textbf{his speculative theories}, as some ventured to do in opening the ark to see what was in it as its power and how God was manifested. The men were slain for their curiosity science.}[17LtMs, Ms 223, 1902, par. 16][https://egwwritings.org/read?panels=p14067.9124037&index=0]


\egw{El tema de \textbf{\underline{especulación} respecto a la \underline{personalidad de Dios} \underline{no nos aventuraremos} a expresar}, \textbf{\underline{excepto en el lenguaje de la Palabra que representa Su personalidad}}. No debe haber discusión sobre esta cuestión \textbf{no sea que Dios dé una revelación inequívoca de \underline{lo que Él es}} que extinguiría a quien se atreva a aventurarse en el terreno santo con \textbf{sus teorías especulativas}, como algunos se aventuraron a hacer al abrir el arca para ver qué había en ella como su poder y cómo se manifestaba Dios. Los hombres fueron muertos por su curiosidad científica.}[17LtMs, Ms 223, 1902, par. 16][https://egwwritings.org/read?panels=p14067.9124037&index=0]


Did you catch that? There is to be no discussion over the question of what God is, \egwinline{lest God would give unmistakable revelation} of \egwinline{what He is}. To say “God is \_\_\_\_\_\_\_”, the blank must be filled with \egwinline{the language of the Word which represents His personality.} The Bible clearly teaches that God is a personal, spiritual being—a truth confirmed by Christ Himself in His revelations to Ellen White. This fits within the biblical language that describes God’s personality. However, according to above statement, can we say “\textit{God is trinity}?” No! That is not \egwinline{the language of the Word which represents His personality.} Therefore, within explored context, we can safely conclude that, the Trinitarian view of God is part of \egwinline{speculative theories} of \egwinline{what He is}.


¿Lo captaste? No debe haber discusión sobre la cuestión de lo que Dios es, \egwinline{no sea que Dios dé una revelación inequívoca} de \egwinline{lo que Él es}. Para decir “Dios es \_\_\_\_\_\_\_“, el espacio en blanco debe llenarse con \egwinline{el lenguaje de la Palabra que representa Su personalidad.} La Biblia enseña claramente que Dios es un ser personal y espiritual—una verdad confirmada por Cristo mismo en Sus revelaciones a Ellen White. Esto encaja dentro del lenguaje bíblico que describe la personalidad de Dios. Sin embargo, según la declaración anterior, ¿podemos decir “\textit{Dios es trinidad}?” ¡No! Eso no es \egwinline{el lenguaje de la Palabra que representa Su personalidad.} Por lo tanto, dentro del contexto explorado, podemos concluir con seguridad que la visión Trinitaria de Dios es parte de las \egwinline{teorías especulativas} de \egwinline{lo que Él es}.


This being said, the phrase \egwinline{Heavenly Trio} is not a definition of what God is. Our God is the Father—not \egwinline{the Heavenly Trio.} The term Heavenly Trio does not serve as a replacement for the Trinitarian idea of \textit{three living persons of one God}. This becomes obvious, when we examine the context. Ellen White was instructed to warn us against Trinitarian sentiments, not to trust them. She was not endorsing them.


Dicho esto, la frase \egwinline{Trío Celestial} no es una definición de lo que Dios es. Nuestro Dios es el Padre—no \egwinline{el Trío Celestial.} El término Trío Celestial no sirve como reemplazo para la idea Trinitaria de \textit{tres personas vivas de un solo Dios}. Esto se vuelve obvio cuando examinamos el contexto. A Ellen White se le instruyó advertirnos contra los sentimientos trinitarios, no confiar en ellos. Ella no los estaba respaldando.


Although the illustrations Ellen White quoted were not from Dr. Kellogg, it seems that Kellogg's proponents, if not Kellogg himself, were defending him with William Boardman's sentiments. We do not have direct data to confirm this, but we do know that Dr. Kellogg raised \others{the theological side of questions of \textbf{the trinity and all that sort of things}.}[Interview, J. H. Kellogg, G. W. Amadon and A. C. Bourdeau, October 7th 1907 held at Kellogg’s residence][https://archive.org/details/KelloggVs.TheBrethrenHisLastInterviewAsAnAdventistoct71907/page/n37] The last three paragraphs in the heavenly trio manuscript \href{https://egwwritings.org/?ref=en_Ms21-1906&para=9754.1}{(Ms21-1906; 1906)} reveal the connection with Dr. Kellogg, which is another “smoking gun” of Dr. Kellogg's trinitarian stance.


Aunque las ilustraciones que Ellen White citó no eran del Dr. Kellogg, parece que los defensores de Kellogg, si no el mismo Kellogg, lo estaban defendiendo con los sentimientos de William Boardman. No tenemos datos directos para confirmar esto, pero sabemos que el Dr. Kellogg planteó \others{el lado teológico de cuestiones de \textbf{la trinidad y todo ese tipo de cosas}.}[Entrevista, J. H. Kellogg, G. W. Amadon y A. C. Bourdeau, 7 de octubre de 1907 celebrada en la residencia de Kellogg][https://archive.org/details/KelloggVs.TheBrethrenHisLastInterviewAsAnAdventistoct71907/page/n37] Los últimos tres párrafos en el manuscrito del trío celestial \href{https://egwwritings.org/?ref=en_Ms21-1906&para=9754.1}{(Ms21-1906; 1906)} revelan la conexión con el Dr. Kellogg, lo cual es otra “prueba contundente” de la postura trinitaria del Dr. Kellogg.


\egw{I write this because any moment my life may be ended. \textbf{Unless there is a breaking away from the influence that Satan has prepared, and a \underline{reviving of the testimonies that God has given, souls will perish in their delusion}. They will accept fallacy after fallacy and will thus keep up a disunion that will always exist until those who have been deceived take \underline{their stand on the right platform}}. All this higher education that is being planned will be extinguished; for it is spurious. The more simple the education of our workers, the less connection they have with the men whom God is not leading, the more will be accomplished. \textbf{Work will be done in the \underline{simplicity} of true godliness, and the old, old times will be back when, under the Holy Spirit’s guidance, thousands were converted in a day. When the truth in its simplicity is lived in every place, then God will work through His angels as He worked on the day of Pentecost, and hearts will be changed so decidedly that there will be a manifestation of the influence of genuine truth, as is represented in the descent of the Holy Spirit}.}[Ms21-1906.18; 1906][https://egwwritings.org/read?panels=p9754.25]


\egw{Escribo esto porque en cualquier momento mi vida puede terminar. \textbf{A menos que haya un rompimiento con la influencia que Satanás ha preparado, y un \underline{reavivamiento de los testimonios que Dios ha dado}, las almas perecerán en su engaño. Aceptarán falacia tras falacia y así mantendrán una desunión que siempre existirá hasta que aquellos que han sido engañados \underline{tomen su posición en la plataforma correcta}}. Toda esta educación superior que se está planeando se extinguirá; porque es espuria. Cuanto más sencilla sea la educación de nuestros obreros, cuanto menos relación tengan con los hombres a quienes Dios no está guiando, más se logrará. \textbf{Se trabajará en la \underline{sencillez} de la verdadera piedad, y volverán los viejos tiempos cuando, bajo la guía del Espíritu Santo, se convertían miles en un día. Cuando la verdad en su sencillez sea vivida en cada lugar, entonces Dios obrará por medio de sus ángeles como obró en el día de Pentecostés, y los corazones serán cambiados tan decididamente que habrá una manifestación de la influencia de la verdad genuina, como se representa en el descenso del Espíritu Santo}.}[Ms21-1906.18; 1906][https://egwwritings.org/read?panels=p9754.25]


\egwnogap{The Holy Spirit never has and never will in the future divorce the medical missionary work from the gospel ministry. They cannot be divorced. Bound up with Jesus Christ, the ministry of the Word and the healing of the sick are one.}[Ms21-1906.19; 1906][https://egwwritings.org/read?panels=p9754.26]


\egwnogap{El Espíritu Santo nunca ha divorciado ni divorciará en el futuro la obra médica misionera del ministerio evangélico. No pueden estar divorciados. Ligados a Jesucristo, el ministerio de la Palabra y la curación de los enfermos son uno.}[Ms21-1906.19; 1906][https://egwwritings.org/read?panels=p9754.26]


\egwnogap{The fifty-eighth chapter of Isaiah contains instruction for today. \textbf{‘Cry aloud, spare not, lift up thy voice like a trumpet, and show My people their transgression, and the house of Jacob their sin.’ God does not accept \underline{Dr. Kellogg as His laborer}, unless he will now break with Satan}. The work would not have been hindered, as it has been for the past several years, \textbf{if Dr. Kellogg were a converted man. ‘Come,’ I call, ‘come ye out and be separate from him and his associates whom he has leavened.’ I am now giving the message God has given me, to give to all who claim to believe the truth, \underline{‘Come ye out from among them, and be separate},’ else their sin in justifying wrongs and framing deceits will continue to be the ruin of souls. We cannot afford to be on the wrong side. We cannot afford to cover the truth with scientific problems. We urge that decided changes be made and no more stumbling blocks be placed before the feet of the people of God}. Let every soul put on the gospel shoes. \textbf{Let every soul pray and work, placing their feet upon \underline{the foundation Christ laid} in giving His life for the life of the world}.}[Ms21-1906.20; 1906][https://egwwritings.org/read?panels=p9754.27]


\egwnogap{El capítulo cincuenta y ocho de Isaías contiene instrucción para hoy. \textbf{‘Clama a voz en cuello, no te detengas; alza tu voz como trompeta, y anuncia a mi pueblo su rebelión, y a la casa de Jacob su pecado’. Dios no acepta \underline{al Dr. Kellogg como Su obrero}, a menos que ahora rompa con Satanás}. La obra no habría sido obstaculizada, como lo ha sido durante los últimos años, \textbf{si el Dr. Kellogg fuera un hombre convertido. ‘Venid’, llamo, ‘salid y separaos de él y de sus asociados, a quienes ha fermentado’. Ahora estoy dando el mensaje que Dios me ha dado, para dar a todos los que afirman creer la verdad, \underline{‘Salid de entre ellos, y separaos}’, o de lo contrario su pecado al justificar males y enmarcar engaños continuará siendo la ruina de las almas. No podemos permitirnos estar en el lado equivocado. No podemos permitirnos cubrir la verdad con problemas científicos. Instamos a que se hagan cambios decididos y no se pongan más piedras de tropiezo ante los pies del pueblo de Dios}. Que cada alma se ponga los zapatos del evangelio. \textbf{Que cada alma ore y trabaje, poniendo sus pies sobre \underline{el fundamento que Cristo puso} al dar su vida por la vida del mundo}.}[Ms21-1906.20; 1906][https://egwwritings.org/read?panels=p9754.27]


The heavenly trio quotation was part of Kellogg's controversy. This is evidence that Kellogg’s controversy included the Trinity doctrine. We are told to break \egwinline{away from the influence of Satan} and to revive the \egw{testimony that God has given} us, or else our souls will perish in delusions. These influences and delusions come from trinitarians such as \textit{William Boardman} and \textit{Dr. John H. Kellogg}. She is pointing us back to place our feet upon the foundation that was built by the Masterworker.\footnote{\href{https://egwwritings.org/?ref=en_SpTB02.54.2&para=417.276}{EGW, SpTB02 54.2; 1904}}


La cita del trío celestial fue parte de la controversia de Kellogg. Esto es evidencia de que la controversia de Kellogg incluía la doctrina de la Trinidad. Se nos dice que nos separemos \egwinline{de la influencia de Satanás} y que reavivemos los \egw{testimonios que Dios ha dado}, o de lo contrario nuestras almas perecerán en engaños. Estas influencias y engaños provienen de trinitarios como \textit{William Boardman} y el \textit{Dr. John H. Kellogg}. Ella nos está señalando que volvamos a poner nuestros pies sobre el fundamento que fue construido por el Maestro Obrero.\footnote{\href{https://egwwritings.org/?ref=en_SpTB02.54.2&para=417.276}{EGW, SpTB02 54.2; 1904}}


We hope that this context exposes the false narrative of Ellen White's endorsement of the Trinity doctrine, propagated by our Adventist scholars. Dr. Kellogg was in apostasy for stepping off from the foundation of our faith, and the Trinity doctrine was his justification. With such data in mind, one must ask: If the Trinity was true, and Ellen White endorsed it, and this “true” Trinity was mixed with Dr. Kellogg's error, we should expect her to separate the Trinity from error. But this is not what she did. Instead, she consistently pointed us back to the foundation of our faith, where we had a clear teaching on the presence and the \emcap{personality of God}. But for the case of Trinity, she faithfully bore the message from Heaven: “\textit{\textbf{I am instructed to say}, the sentiments of those who are searching for \textbf{trinitarian ideas are not to be trusted}}.”


Esperamos que este contexto exponga la falsa narrativa del respaldo de Ellen White a la doctrina de la Trinidad, propagada por nuestros académicos adventistas. El Dr. Kellogg estaba en apostasía por apartarse del fundamento de nuestra fe, y la doctrina de la Trinidad era su justificación. Con tales datos en mente, uno debe preguntarse: Si la Trinidad fuera verdadera, y Ellen White la respaldara, y esta Trinidad “verdadera” estuviera mezclada con el error del Dr. Kellogg, deberíamos esperar que ella separara la Trinidad del error. Pero esto no es lo que hizo. En cambio, ella consistentemente nos señaló de vuelta al fundamento de nuestra fe, donde teníamos una enseñanza clara sobre la presencia y la \emcap{personalidad de Dios}. Pero en el caso de la Trinidad, ella fielmente transmitió el mensaje del Cielo: “\textit{\textbf{Se me ha instruido decir}, los sentimientos de aquellos que buscan \textbf{ideas trinitarias no son de fiar}}.”


% The Heavenly Trio

\begin{titledpoem}
    
    \stanza{
        In heaven’s realm, where truths unfold, \\
        A message clear, so brave and bold. \\
        God spoke through Ellen, clear and bright, \\
        Revealing depths of heavenly light.
    }

    \stanza{
        Misunderstood by some who read, \\
        Her words of God that all must heed. \\
        Not as triune, but trio three \\
        Distinct as persons, heavenly.
    }

    \stanza{
        The Father, not a formless feel, \\
        Invisible to us, yet real. \\
        He is the fullness, all complete, \\
        The Godhead, bodily, concrete.
    }

    \stanza{
        The Son, God’s fullness, manifest \\
        In Him, divinity does rest. \\
        God’s character, seen in His face, \\
        In Christ, we see His Father’s grace.
    }

    \stanza{
        The Spirit, in all fullness dwells, \\
        A mystery nature, Ellen tells. \\
        With forms, the Father and His Son \\
        With Them, in Spirit, we are one.
    }

    \stanza{
        Distinct and clear, Their roles unfold, \\
        The Father, Son, in form behold. \\
        Yet present everywhere we find, \\
        Their Spirit shows Their heart and mind.
    }

    \stanza{
        God’s message true, from up above. \\
        Reveals to us the Father’s love. \\
        To know this truth about our God— \\
        It lights the path that we must trod.
    }

    \stanza{
        Dear Ellen’s words, in context found, \\
        Reveal a truth that’s so profound \\
        Not trinity did she embrace, \\
        But trio persons in their place.
    }

    \stanza{
        The pillar stands, our platform firm, \\
        God’s personality we learn. \\
        The trio that is heavenly, \\
        Exposes falsehood—trinity.
    }
    
\end{titledpoem}


% \chapter{The Heavenly Trio}

So far we have seen the evidence that Ellen White knew about Dr. Kellogg's trinitarian sentiments, and we have seen how she responded to it. She always uplifted the truth on the presence and the \emcap{personality of God}, and called to come back to the foundation of our faith—\emcap{Fundamental Principles}. However, when Adventist scholars discuss the doctrine of the Trinity and Ellen White, they do not approach it in the same manner as Ellen White did. The \emcap{Fundamental Principles} together with the doctrine on the \emcap{personality of God} is downplayed, and the twisted story is presented that Ellen White was trinitarian and responsible for the church's acceptance of the Trinity doctrine into our ranks. We want to challenge this twisted story by looking at the evidence that is often used to support this false narrative.

One of the most prominent quotations to support the claim that Sister White was responsible for accepting the Trinity doctrine into our ranks is her writings and comments on Matthew 28:19\footnote{\bible{Go ye therefore, and teach all nations, baptizing them in the name of the Father, and of the Son, and of the Holy Ghost}[Matthew 28:19]}. The most prominent quotation to stand out in defense of the Trinity doctrine is “\textit{the Heavenly Trio}” quotation:

\egw{\textbf{There are \underline{three living persons} of the \underline{heavenly trio}}; in the name of these three great powers—\textbf{the Father, the Son, and the Holy Spirit}—those who receive Christ by living faith are baptized, and these powers will co-operate with the obedient subjects of heaven in their efforts to live the new life in Christ...}[Ev 615.1; 1946][https://egwwritings.org/read?panels=p30.3407]

To reiterate, this quotation is often cited to argue that Sister White defended and advocated the Trinity doctrine. But, if we take a look at this quotation in its literary context, we see that within the quotation itself she actually \textit{refuted} this doctrine and exalted the truth on the \emcap{personality of God}. To some this is a ludicrous claim, but we invite you to make your judgment based on presented data. Let us examine the context of this quotation.

\egw{I am instructed to say, \textbf{The sentiments} of those who are searching for advanced scientific ideas \textbf{\underline{are not to be trusted}}. Such representations as the following are made: ‘\textbf{The Father is as the light invisible; the Son is as the light embodied; the Spirit as the light shed abroad.}’ ‘\textbf{The Father is like the dew, invisible vapor; the Son is like the dew gathered in beauteous form; the Spirit is like the dew fallen to the seat of life.}’ Another representation: ‘\textbf{The Father is like the invisible vapor. The Son is like the leaden cloud. The Spirit is rain fallen and working in refreshing power.}’}[Ms21-1906.8; 1906][https://egwwritings.org/read?panels=p9754.15]

What sentiments are not to be trusted? The data suggest that those sentiments are trinitarian ideas of \textit{one God in three persons}. How do we know that? We see in the literary context of the representations Sister White was quoting. Contrary to the popular belief that she was referencing the “\textit{false}” trinity expressed by Dr. Kellogg,\footnote{Whidden, Woodrow W, et al. \textit{The Trinity : Understanding God’s Love, His Plan of Salvation, and Christian Relationships}. Hagerstown, Md, Review And Herald Pub. Association, 2002, p. 216.} she was actually referencing trinitarian idea of \textit{three living persons of one living God}, advocated by William Boardman, in his book “Higher Christian Life”, which she quoted. The context matters. The context of the quotations she quoted, shows that the representations of the Father, the Son, and the Holy Spirit are serving to illustrate the sentiment of three living persons of one God. That is the sentiment we have been clearly instructed by God, not to trust. Let the data be its own interpreter.

\section*{The Higher Christian Life, William Boardman}

Ellen White owned William Boardman's book “Higher Christian Life.” It was a good book about Christian sanctification, but in it there was trinitarian sentiment, which Sister White was particularly instructed by God to call out. This is another instance of evidence where we see that Ellen White was familiar with the trinitarian stance, and she was addressing it directly. Let's get familiar with the trinitarian sentiments promoted by William Boardman.

Speaking of Triune God, William Boardman writes:

\othersQuote{And then, again, the Father is the author and planner of salvation through faith in his Son; and when we trust in his Son we honor the Father, because we accept of his plan of salvation for us, justify his wisdom, and act in accordance with his will in the matter. \textbf{A glance at the official and essential relations of the persons of the Holy Trinity to each other and to us, may throw additional light upon our pathway}. Upon this subject flippancy would border upon blasphemy. It is holy ground. He who ventures upon it may well tread with unshod foot, and uncovered head bowed low.}[William Boardman, The Higher Christian Life, p. 99; 1858][https://archive.org/details/higherchristian02boargoog/page/n106/]

Brother Boardman wants us to take \others{a glance at the official and essential relations} of the three persons of the Holy Trinity. He asserts that \textit{God is one but also three}–\textit{Triune}–by presenting official and essential relations of the persons of the Holy Trinity. His fundamental statement and outline for his thesis is as follows:

\othersQuote{\textbf{The Father is fullness of the Godhead \underline{invisibly}, without form, whom no creature hath seen or can see}. \\
\textbf{The Son is the fullness of the Godhead \underline{embodied}, that his creatures may see him, and know him, and trust him}. \\
\textbf{The Spirit is the fullness of the Godhead \underline{in all active workings}, whether of creation, providence, revelation, or salvation, by which God manifests himself to and through the universe}.}[William Boardman, The Higher Christian Life, p. 100][https://archive.org/details/higherchristian02boargoog/page/n108/]

This statement is foundational to his following statements and illustrations. In the following paragraphs, William Boardman gives the biblical motives to illustrate \others{the official and essential relations of the Holy Trinity}—\textit{that is, God being one, but yet three}. He writes:

\othersQuote{Another of the names of Jesus will give the same analogies in a light not less striking - \textbf{The Sun of Righteousness}. \\
All the light of the sun in the heavens was once hidden in the invisibility of primal darkness; and after this, the light now blazing in the orb of day was, when first the command when forth, Let light be! and light was, at most only the diffused haze of the gray dawn of the morn of creation out of the darkness of chaotic night, without form, or body, or centre, or radiance, or glory. But when separated from the darkness and centered in the sun, then in its glorious glitter it became so resplendent that none but the eagle eye could bear to look it in the face. \\
But then again its rays falling aslant through earth’s atmosphere and vapors, gladdens all the world with the same light, dispelling the winter, and the cold, and the darkness; starting Spring forth in floral beauty, and Summer in vernal luxuriance, and Autumn laden with golden treasures for the garner.
\textbf{The Father is as the Light invisible}. \\
\textbf{The Son is as the Light embodied}. \\
\textbf{The Spirit is as the Light shed down}.}[William Boardman, The Higher Christian Life, p. 101,102][https://archive.org/details/higherchristian02boargoog/page/n108/]

This illustration of the Sun of Righteousness shows that God the Father, who is \textit{the fullness of the Godhead invisible,} can be symbolically illustrated as a Light that \others{was once hidden in the invisibility of primal darkness}. The Son, who is \textit{the fullness of the Godhead embodied}, is like a Light that is embodied in \others{the morn of creation}. The Holy Spirit, who is \textit{the fullness of the Godhead in all active workings}, is like a \others{Light shed down}. William Boardman gives us another similar illustration to clarify the \others{official relations of the persons of the Godhead}:

\othersQuote{One of the similies for blessed influences of the Spirit, \textbf{while giving the self-same official relations of the persons of the Godhead, to each other and to us}, may illustrate them still further,—\textbf{The Dew},—\textbf{The dew of Hermon} - the dew on the mown meadow. Before the dew gathers at all in drops, it hangs over all the landscape in visible vapor, omnipresent but unseen. By and by as the light wanes into morning, and as the temperature sinks and touches the dew point the invisible becomes the visible, the embodied; and, as the sun rises, it stands in diamond drops trembling and glittering in the sun’s young beams in pearly beauty upon leaf and flower, over all the face of nature. \\
But now again, a breeze springs up, the breath of heaven is wafted gently along, shaking leaf and flower, and in a moment the pearly drops are invisible angina. But where now? Fallen at the root of herb and flower to impart new life, freshness, vigor to all it touches. \\
\textbf{The Father is like the dew in invisible vapor}. \\
\textbf{The Son is like the dew gathered in beauteous form}. \\
\textbf{The Spirit is like the dew fallen to the seat of life}.}[William Boardman, The Higher Christian Life, p. 102,103][https://archive.org/details/higherchristian02boargoog/page/n110/]

The Father, who is \textit{the fullness of the Godhead invisible,} is illustrated by the \others{dew in invisible vapor}. The Son, who is \textit{the fullness of the Godhead embodied}, is illustrated by \others{the dew gathered in beauteous form}. The Spirit, who is \textit{the fullness of the Godhead in all active works}, is illustrated by \others{the dew fallen to the seat of life}. The next illustration that exemplifies the official relations of the three personalities of one God is by another Bible likening—the Rain.

\othersQuote{\textbf{Yet one more of these Bible likenings} – by no means exhausting them – will not be unwelcome, or useless, - \textbf{the Rain}. \\
Rain, like the dew, floats in invisibility, and omnipresence at the first, over all, around all. Seen by none. While it remains in its invisibility, the earth parches, clods cleave together, the ground cracks open, the sun pours down his burning heat, the winds lift up the dust in circling whirls, and rolling clouds, and famine gaunt and greedy stalks through the land, followed by pestilence and death. By and by, the eager watcher sees the little hand-like cloud rising far out over the sea. It gathers, gathers, gathers; comes and spreads as it comes, in majesty over the whole heavens: - But all is parched and dry and dead yet, upon earth. \\
But now comes a drop, and drop after drop, quicker, faster – the shower, the rain – sweeping on, and giving to earth all the treasures of the clouds – clods open, furrows soften, springs, rivulets, rivers, swell and fill, and all the land is gladdened again with restored abundance. \\
\textbf{The Father is like to the invisible vapor}. \\
\textbf{The Son is as the laden cloud and falling rain}. \\
\textbf{The Spirit is the Rain – fallen and working in refreshing power}.}[William Boardman, The Higher Christian Life, p. 103,104][https://archive.org/details/higherchristian02boargoog/page/n110/]

Let's give William Boardman a fair hearing. He is not saying that the Father is \others{invisible vapor}; rather, he uses a metaphor of rain and \others{invisible vapor} to illustrate his main point that the Father is the invisible fullness of the Godhead. So it is with the Son, who, just like rain manifested in leaden clouds, is all the fullness of the Godhead manifested. To ensure his sentiments are not potentially misrepresented, William Boardman clarified his sentiment. This was the very sentiment that Ellen White was instructed by God not to trust:

\othersQuote{\textbf{These likenings are all imperfect. They rather hide than illustrate \underline{the tri-personality of the one God}, for they are not persons but things, poor and earthly at best, to represent the living personalities of the living God. So much they may do, however, as to illustrate the official relations of each to the others and of each and all to us. And more. They may also illustrate the truth that all the fulness of Him who filleth all in all, dwells in each person of \underline{the Triune God}}. \\
\textbf{The Father is all the fulness of the Godhead INVISIBLE}. \\
\textbf{The Son is all the fulness of the Godhead MANIFESTED}. \\
\textbf{The Spirit is all the fulness of the Godhead MAKING MANIFEST}. \\
\textbf{The persons are not mere offices, or modes of revelation, but living persons of the living God}.}[William Boardman, The Higher Christian Life, p. 104,105][https://archive.org/details/higherchristian02boargoog/page/n112/]

It is crucial to emphasize that when Boardman uses these Bible likenings from nature, he speaks of the illustrations, and not reality. These representations are illustrating his sentiments. In his own admission, that was the sentiment of three \others{living personalities of the living God.} Though these illustrations are imperfect, they may \others{illustrate the official relations} of \others{the tri-personality of the one God} and \others{the truth that all the fullness of Him who filleth all in all dwells in each person of the Triune God.} One God in three persons is the sentiment in question, and that sentiment is common to all types and versions of the trinity doctrine—including our current trinitarian stance in the second point of the Fundamental Beliefs.\footnote{\others{There is \textbf{one God}: Father, Son, and Holy Spirit, \textbf{a unity of three} coeternal \textbf{Persons}…} 2nd point of the Fundamental Beliefs}

In this brief look at William Boardman's sentiments, it is clear that the sentiments in question which Ellen White was instructed by God to call out, were the sentiments of the Triune God, or \textit{three living persons in the Trinity}. With that data in mind, let's examine Ellen White's response.

\section*{Ellen White on William Boardman’s sentiment}

With the Heavenly Trio quotation, it has been asserted that Ellen White was trinitarian. This is done by ignorantly or sometimes purposely ignoring the context of this valuable quotation. When reading Ellen White’s response, in which she defends our perceptions of God, try to recognize whom she is addressing when she speaks of God. Was the God she defended the Trinity or the Father? Referencing William Boardmans illustrations she said:

\egw{\textbf{All these \underline{spiritualistic} representations are simply nothingness}. They are imperfect, untrue. They weaken and diminish the Majesty which no earthly likeness can be compared to. \textbf{God cannot be compared with the things His hands have made}. These are mere earthly things, suffering under the curse of God because of the sins of man. \textbf{The Father cannot be described by the things of earth}. \textbf{The Father is all the fulness of the Godhead \underline{bodily} and is \underline{invisible to mortal sight}}.}[Ms21-1906.9; 1906][https://egwwritings.org/read?panels=p9754.15]

By observing the context, it is obvious that Sister White follows Boardman’s line of reasoning and corrects the mistakes. For better comparison, let us look at their writings side by side:

\begin{table}[H]
\centering
\renewcommand{\arraystretch}{1.5}
\setlength{\tabcolsep}{15pt}
\begin{tabular}{|p{0.4\textwidth}|p{0.4\textwidth}|}
\hline
\multicolumn{1}{|c|}{\textbf{William Boardman}} & \multicolumn{1}{c|}{\textbf{Ellen G. White}} \\ \hline
\othersQuote{These likenings are all imperfect. They rather hide than \textbf{illustrate the tri-personality of the \underline{one God}}, for they are not persons but things, poor and earthly at best, to represent \textbf{the living personalities of the living God}. \textbf{So much they may do, however, as to illustrate the official relations of each to the other and of each and all to us. And more. They may also illustrate the truth that all the fulness of Him who filleth all in all, dwells in \underline{each person of Triune God}}.}[p. 104,105][https://archive.org/details/higherchristian02boargoog/page/n112] & 
\egw{\textbf{All these \underline{spiritualistic} representations are simply nothingness}. They are imperfect, untrue. They weaken and diminish the Majesty which no earthly likeness can be compared to. \textbf{God cannot be compared with the things His hands have made}. These are mere earthly things, suffering under the curse of God because of the sins of man. \textbf{The Father cannot be described by the things of earth}.}[Ms21-1906.9; 1906][https://egwwritings.org/read?panels=p9754.15] \\ \hline
\end{tabular}
\end{table}

In this comparison, it is clear who God is for William Boardman, and who He is for Sister White. For Boardman, God is the Triune God, a tri-personality of the one God. For Sister White, God is the Father. For Boardman, these representations are imperfect because they \others{rather hide than illustrate the tri-personality of the one God}, and for Sister White these representations are imperfect because \egw{The Father cannot be described by the things of earth}. For Boardman, God is the \textit{Triune God}; for Sister White, God is \textit{the Father}.

Boardman’s only point that Ellen White affirms is that these representations are imperfect. Surely, William Boardman would not agree with Ellen White that these representations are \textit{spiritualistic} and \textit{untrue}. On the contrary, he believes that these illustrations \others{illustrate the truth that all the fulness of Him who filleth all in all, dwells in each person of Triune God}. To say that Ellen White agreed with such sentiment is gross misrepresentation.

The context of this important quotation prompts important questions. Why does the prophet of God refer to the representations that illustrate the \others{tri-personality of the one God} as \egwinline{spiritualistic representations}, which illustrate the sentiment that \egwinline{is not to be trusted}? Or why does the prophet of God refer to the representations that \others{represent the living personalities of the living God} as \egwinline{spiritualistic representations}? Or why does the prophet of God, when referring to the representations that \others{illustrate the truth that all the fullness of Him who filleth all in all, dwells in each person of Triune God}, refer to them as \egwinline{spiritualistic representations}? All of these spiritualistic representations illustrate the sentiment that \egwinline{is not to be trusted}. This sentiment is clearly the trinitarian sentiment.

Sister White continues to follow Boardman’s line of reasoning and corrects the error.

\begin{table}[H]
\centering
\renewcommand{\arraystretch}{1.5}
\setlength{\tabcolsep}{15pt}
\begin{tabular}{|p{0.4\textwidth}|p{0.4\textwidth}|}
\hline
\multicolumn{1}{|c|}{\textbf{William Boardman}} & \multicolumn{1}{c|}{\textbf{Ellen G. White}} \\ \hline
\othersQuote{The Father is fullness of the Godhead \textbf{invisibly}, \textbf{\underline{without form}}, whom \textbf{no creature hath seen \underline{or can see}}.}[p.100][https://archive.org/details/higherchristian02boargoog/page/n108/]

\othersQuote{The Father is all the fullness of the Godhead \textbf{INVISIBLE}.}[p.105][https://archive.org/details/higherchristian02boargoog/page/n112/] & 
\egw{The Father is all the fulness of the Godhead \textbf{\underline{bodily}}, and is \textbf{invisible to mortal sight}.}[Ms21-1906.9; 1906][https://egwwritings.org/read?panels=p9754.15] \\ \hline
\end{tabular}
\end{table}

For Boardman, the Father does not have a form nor body and is invisible to all creatures. For Sister White, the Father has a form and body and is invisible only to mortal human beings.\footnote{When Sister White talks about mortals, she talks about sin polluted humanity. After the restoration of humanity, at the resurrection, Christ will give His immortal life to His children. For more information read \href{https://egwwritings.org/?ref=en_RH.July.5.1887.par.5}{EGW, RH July 5, 1887, par. 5; 1887}.}

This quotation is one of the most direct quotations regarding the \emcap{personality of God}. \egwinline{The Father is all the fullness of the Godhead \textbf{bodily}}[Ms21-1906.9; 1906][https://egwwritings.org/read?panels=p9754.16].

It might be confusing to someone that the Father is all the fullness of the Godhead bodily because in \textit{Colossians 2:9}, when referring to Jesus, it is written that \bible{in him dwelleth all the fulness of the Godhead bodily.} Scripture does not contradict itself. \textit{Colossians 2:9} does not exclude the Father to be all the fulness of the Godhead bodily. Various places in the Bible describe the Father having a body (\textit{a form: Daniel 7:9,10; Revelation 4:2,3; 1 Kings 22:19-22; a shape: John 5:37}). He has the appearance of a man (\textit{Ezekiel 1:26-28}). He has a face (\textit{Exodus 33:20; Matthew 18:10; Revelation 22:3, 4}). However, the Bible is completely silent about the nature of its substance. The Bible teaches us that \bible{\textbf{The secret things belong unto the LORD our God}: \textbf{but those things which \underline{are revealed} belong unto us and to our children for ever}, that we may do all the words of this law}[Deuteronomy 29:29]. It is revealed to us that the Father has body, He is all the fulness of the Godhead bodily. Also, it is revealed that in Jesus also dwells all the fulness of the Godhead bodily, because \bible{it pleased the Father that in him should all fulness dwell}[Colossians 1:19]. This is not a contradiction whatsoever because the Son is \bible{the \textbf{express image of \underline{His person}}}[Hebrews 1:3].

\begin{table}[H]
\centering
\renewcommand{\arraystretch}{1.5}
\setlength{\tabcolsep}{15pt}
\begin{tabular}{|p{0.4\textwidth}|p{0.4\textwidth}|}
\hline
\multicolumn{1}{|c|}{\textbf{William Boardman}} & \multicolumn{1}{c|}{\textbf{Ellen G. White}} \\ \hline
\othersQuote{The Son is the fullness of the Godhead \textbf{embodied, that his creatures may see him, and know him, and trust him}.}[p.100][https://archive.org/details/higherchristian02boargoog/page/n108/]

\othersQuote{The Son is all the fulness of the Godhead \textbf{MANIFESTED}.}[p.105][https://archive.org/details/higherchristian02boargoog/page/n112/] & 
\egw{The Son is all the fulness of the Godhead \textbf{manifested}. The Word of God declares Him to be ‘\textbf{the express image of His person}’. ‘God so loved the world that He gave \textbf{His only begotten Son}, that whosoever believeth in Him should not perish, but have everlasting life’. \textbf{Here is shown \underline{the personality of the Father}}.}[Ms21-1906.10; 1906][https://egwwritings.org/read?panels=p9754.17] \\ \hline
\end{tabular}
\end{table}

Sister White focused on the \emcap{personality of God}, which is the personality of the Father. In Christ, who is \egwinline{begotten in the express image of the Father’s person}[ST May 30, 1895, par. 3; 1895][https://egwwritings.org/read?panels=p820.12891], is shown the personality of the Father. In the same way that Jesus is a person, so is the Father. The quality or state of Christ being a person is the same quality or state of the Father being a person. As Christ is a personal being, so is the Father. Just as all the fullness of the Godhead bodily dwells in Christ, so it does in the Father, because Christ is begotten in the express image of the Father’s person. In Him is shown the personality of the Father. These simple conclusions have been asserted by Scripture in John 3:16 and Hebrews 1:3.

Does the same reasoning, of the personality of the Father and Son, apply to the Holy Spirit? Speaking of the Holy Spirit, Sister White continues:

\egw{\textbf{The Comforter that Christ} promised to send after He ascended to heaven, \textbf{is the Spirit \underline{in} all the fulness of the Godhead}, making manifest the power of divine grace to all who receive and believe in Christ as a personal Saviour.}[Ms21-1906.11; 1906][https://egwwritings.org/read?panels=p9754.18]

Sister White draws a distinction between Father and Son who \textbf{are}, individually, \textbf{all} the fullness of the Godhead, and the Spirit that is \textbf{in all} the fullness of the Godhead. This is a marked contrast to William Boardman’s reasoning, where all three are the fullness of the Godhead. Sister White does not follow this trinitarian fashion. The explanation is simple in light of the \emcap{personality of God} and of Christ. The Holy Spirit is a spirit, and the spirit dwells \textbf{in} the flesh/body. The Holy Spirit is \textbf{in all} the fullness of the Godhead\footnote{Take a look at the quotation from \href{https://egwwritings.org/?ref=en_Ms128-1897.13&para=5426.19}{{EGW, Ms128-1897.13; 1897}}, where Sister White states that the Father and the Son are the absolute Godhead.}.

Finally, the quotation continues to its most renowned part:

\begin{table}[H]
\centering
\renewcommand{\arraystretch}{1.5}
\setlength{\tabcolsep}{15pt}
\begin{tabular}{|p{0.4\textwidth}|p{0.4\textwidth}|}
\hline
\multicolumn{1}{|c|}{\textbf{William Boardman}} & \multicolumn{1}{c|}{\textbf{Ellen G. White}} \\ \hline
\othersQuote{\textbf{The Father} is all the fulness of the Godhead INVISIBLE.}

\othersQuote{\textbf{The Son} is all the fulness of the Godhead MANIFESTED.}

\othersQuote{\textbf{The Spirit} is all the fulness of the Godhead MAKING MANIFEST.}

\othersQuote{\textbf{The persons} are not mere offices, or modes of revelation, \textbf{but living persons of the living God}.}[p.105][https://archive.org/details/higherchristian02boargoog/page/n112/] & 
\egw{There are \textbf{three living persons of the heavenly trio}; in the name of these three great powers—\textbf{the Father, the Son, and the Holy Spirit}—those who receive Christ by living faith are baptized, and these powers will co-operate with the obedient subjects of heaven in their efforts to live the new life in Christ.}[Ms21-1906.11; 1906][https://egwwritings.org/read?panels=p9754.18] \\ \hline
\end{tabular}
\end{table}

In light of the context of William Boardman’s book, we see a marked contrast between \others{three living persons of \textbf{one living God}}, which is the trinitarian sentiment, and \egwinline{the three living persons of \textbf{the heavenly trio}}, which is in accordance with the truth on the \emcap{personality of God}.

The word ‘\textit{trio}’ simply indicates the group of three. The \textit{“heavenly trio}” is represented by the Father, the Son, and the Holy Spirit. But, contrary to popular assumption, they do not make one living God. Three-in-one and one-in-three are concepts that do away with the \emcap{personality of God}. This is why Sister White referred to trinitarian sentiments as sentiments that \egwinline{are not to be trusted}[Ms21-1906.8; 1906][https://egwwritings.org/read?panels=p9754.15].

Sister White never followed any trinitarian fashion—neither in words and expressions, nor in sentiments. There is an almost effortless research endeavor we encourage you to take: in the writings of Ellen White, search for standard trinitarian terms like “\textit{three are one},” “\textit{one are three},” “\textit{one in three},” “\textit{three in one},” or any of the permutations possible. In her impressive oeuvre you will not find a single occurrence of any of these, let alone the word ‘\textit{trinity}’ describing our God\footnote{There is but one occurrence, in the writings of Ellen White, of the word ‘\textit{trinity}’ referring to \egw{the lust of the flesh, the lust of the eyes and the pride of life}[Lt43-1898.25; 1898][https://egwwritings.org/read?panels=p4806.31]}. She never used these phrases that are necessary to explain the trinitarian sentiment. Examining the following quote, we can see why she never said that God is trinity.

\egw{The subject of \textbf{\underline{speculation} regarding \underline{God’s personality} \underline{we will not venture} to express}, \textbf{\underline{except in the language of the Word which represents His personality}}. There is to be no discussion over this question \textbf{lest God would give unmistakable revelation of \underline{what He is}} that would extinguish the one who dares venture on the holy ground in \textbf{his speculative theories}, as some ventured to do in opening the ark to see what was in it as its power and how God was manifested. The men were slain for their curiosity science.}[17LtMs, Ms 223, 1902, par. 16][https://egwwritings.org/read?panels=p14067.9124037&index=0]

Did you catch that? There is to be no discussion over the question of what God is, \egwinline{lest God would give unmistakable revelation} of \egwinline{what He is}. To say “God is \_\_\_\_\_\_\_”, the blank must be filled with \egwinline{the language of the Word which represents His personality.} The Bible clearly teaches that God is a personal, spiritual being—a truth confirmed by Christ Himself in His revelations to Ellen White. This fits within the biblical language that describes God’s personality. However, according to above statement, can we say “\textit{God is trinity}?” No! That is not \egwinline{the language of the Word which represents His personality.} Therefore, within explored context, we can safely conclude that, the Trinitarian view of God is part of \egwinline{speculative theories} of \egwinline{what He is}.

This being said, the phrase \egwinline{Heavenly Trio} is not a definition of what God is. Our God is the Father—not \egwinline{the Heavenly Trio.} The term Heavenly Trio does not serve as a replacement for the Trinitarian idea of \textit{three living persons of one God}. This becomes obvious, when we examine the context. Ellen White was instructed to warn us against Trinitarian sentiments, not to trust them. She was not endorsing them.

Although the illustrations Ellen White quoted were not from Dr. Kellogg, it seems that Kellogg's proponents, if not Kellogg himself, were defending him with William Boardman's sentiments. We do not have direct data to confirm this, but we do know that Dr. Kellogg raised \others{the theological side of questions of \textbf{the trinity and all that sort of things}.}[Interview, J. H. Kellogg, G. W. Amadon and A. C. Bourdeau, October 7th 1907 held at Kellogg’s residence][https://archive.org/details/KelloggVs.TheBrethrenHisLastInterviewAsAnAdventistoct71907/page/n37] The last three paragraphs in the heavenly trio manuscript \href{https://egwwritings.org/?ref=en_Ms21-1906&para=9754.1}{(Ms21-1906; 1906)} reveal the connection with Dr. Kellogg, which is another “smoking gun” of Dr. Kellogg's trinitarian stance.

\egw{I write this because any moment my life may be ended. \textbf{Unless there is a breaking away from the influence that Satan has prepared, and a \underline{reviving of the testimonies that God has given, souls will perish in their delusion}. They will accept fallacy after fallacy and will thus keep up a disunion that will always exist until those who have been deceived take \underline{their stand on the right platform}}. All this higher education that is being planned will be extinguished; for it is spurious. The more simple the education of our workers, the less connection they have with the men whom God is not leading, the more will be accomplished. \textbf{Work will be done in the \underline{simplicity} of true godliness, and the old, old times will be back when, under the Holy Spirit’s guidance, thousands were converted in a day. When the truth in its simplicity is lived in every place, then God will work through His angels as He worked on the day of Pentecost, and hearts will be changed so decidedly that there will be a manifestation of the influence of genuine truth, as is represented in the descent of the Holy Spirit}.}[Ms21-1906.18; 1906][https://egwwritings.org/read?panels=p9754.25]

\egwnogap{The Holy Spirit never has and never will in the future divorce the medical missionary work from the gospel ministry. They cannot be divorced. Bound up with Jesus Christ, the ministry of the Word and the healing of the sick are one.}[Ms21-1906.19; 1906][https://egwwritings.org/read?panels=p9754.26]

\egwnogap{The fifty-eighth chapter of Isaiah contains instruction for today. \textbf{‘Cry aloud, spare not, lift up thy voice like a trumpet, and show My people their transgression, and the house of Jacob their sin.’ God does not accept \underline{Dr. Kellogg as His laborer}, unless he will now break with Satan}. The work would not have been hindered, as it has been for the past several years, \textbf{if Dr. Kellogg were a converted man. ‘Come,’ I call, ‘come ye out and be separate from him and his associates whom he has leavened.’ I am now giving the message God has given me, to give to all who claim to believe the truth, \underline{‘Come ye out from among them, and be separate},’ else their sin in justifying wrongs and framing deceits will continue to be the ruin of souls. We cannot afford to be on the wrong side. We cannot afford to cover the truth with scientific problems. We urge that decided changes be made and no more stumbling blocks be placed before the feet of the people of God}. Let every soul put on the gospel shoes. \textbf{Let every soul pray and work, placing their feet upon \underline{the foundation Christ laid} in giving His life for the life of the world}.}[Ms21-1906.20; 1906][https://egwwritings.org/read?panels=p9754.27]

The heavenly trio quotation was part of Kellogg's controversy. This is evidence that Kellogg’s controversy included the Trinity doctrine. We are told to break \egwinline{away from the influence of Satan} and to revive the \egw{testimony that God has given} us, or else our souls will perish in delusions. These influences and delusions come from trinitarians such as \textit{William Boardman} and \textit{Dr. John H. Kellogg}. She is pointing us back to place our feet upon the foundation that was built by the Masterworker.\footnote{\href{https://egwwritings.org/?ref=en_SpTB02.54.2&para=417.276}{EGW, SpTB02 54.2; 1904}}

We hope that this context exposes the false narrative of Ellen White's endorsement of the Trinity doctrine, propagated by our Adventist scholars. Dr. Kellogg was in apostasy for stepping off from the foundation of our faith, and the Trinity doctrine was his justification. With such data in mind, one must ask: If the Trinity was true, and Ellen White endorsed it, and this “true” Trinity was mixed with Dr. Kellogg's error, we should expect her to separate the Trinity from error. But this is not what she did. Instead, she consistently pointed us back to the foundation of our faith, where we had a clear teaching on the presence and the \emcap{personality of God}. But for the case of Trinity, she faithfully bore the message from Heaven: “\textit{\textbf{I am instructed to say}, the sentiments of those who are searching for \textbf{trinitarian ideas are not to be trusted}}.”

% The Heavenly Trio

\begin{titledpoem}
    
    \stanza{
        In heaven’s realm, where truths unfold, \\
        A message clear, so brave and bold. \\
        God spoke through Ellen, clear and bright, \\
        Revealing depths of heavenly light.
    }

    \stanza{
        Misunderstood by some who read, \\
        Her words of God that all must heed. \\
        Not as triune, but trio three \\
        Distinct as persons, heavenly.
    }

    \stanza{
        The Father, not a formless feel, \\
        Invisible to us, yet real. \\
        He is the fullness, all complete, \\
        The Godhead, bodily, concrete.
    }

    \stanza{
        The Son, God’s fullness, manifest \\
        In Him, divinity does rest. \\
        God’s character, seen in His face, \\
        In Christ, we see His Father’s grace.
    }

    \stanza{
        The Spirit, in all fullness dwells, \\
        A mystery nature, Ellen tells. \\
        With forms, the Father and His Son \\
        With Them, in Spirit, we are one.
    }

    \stanza{
        Distinct and clear, Their roles unfold, \\
        The Father, Son, in form behold. \\
        Yet present everywhere we find, \\
        Their Spirit shows Their heart and mind.
    }

    \stanza{
        God’s message true, from up above. \\
        Reveals to us the Father’s love. \\
        To know this truth about our God— \\
        It lights the path that we must trod.
    }

    \stanza{
        Dear Ellen’s words, in context found, \\
        Reveal a truth that’s so profound \\
        Not trinity did she embrace, \\
        But trio persons in their place.
    }

    \stanza{
        The pillar stands, our platform firm, \\
        God’s personality we learn. \\
        The trio that is heavenly, \\
        Exposes falsehood—trinity.
    }
    
\end{titledpoem}
