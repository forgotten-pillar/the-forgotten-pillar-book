\qrchapter{https://forgottenpillar.com/rsc/en-fp-chapter5}{The patchwork theories - Lt253-1903}


\qrchapter{https://forgottenpillar.com/rsc/en-fp-chapter5}{Las teorías de parches - Lt253-1903}


\egw{Dear Brother,—}


\egw{Querido hermano,—}


\egwnogap{\textbf{I must tell you that your ideas in regard to some things \underline{have been decidedly wrong}.} I would that you could see your errors. \textbf{The book Living Temple \underline{is not to be patched up}, a few changes made in it, and then advertised and praised as a valuable production}. It would be better to present the physiological parts in another book under another title. \textbf{When you wrote that book}, \textbf{you were not under the inspiration of God}. There was by your side the one who inspired Adam to look at God in a false light. Your whole heart needs to be changed, thoroughly and entirely cleansed.}[Lt253-1903.1; 1903][https://egwwritings.org/read?panels=p9980.7]


\egwnogap{\textbf{Debo decirle que sus ideas con respecto a algunas cosas \underline{han sido decididamente erróneas}.} Desearía que pudiera ver sus errores. \textbf{El libro Templo Viviente \underline{no es para ser remendado}, hacer unos pocos cambios en él, y luego publicitarlo y alabarlo como una producción valiosa}. Sería mejor presentar las partes fisiológicas en otro libro bajo otro título. \textbf{Cuando escribió ese libro}, \textbf{no estaba bajo la inspiración de Dios}. Estaba a su lado el que inspiró a Adán a mirar a Dios bajo una luz falsa. Todo su corazón necesita ser cambiado, limpiado a fondo y por completo.}[Lt253-1903.1; 1903][https://egwwritings.org/read?panels=p9980.7]


\egwnogap{\textbf{My brother, do not allow yourself to be alienated from your ministering brethren who tell you of your dangers. Those who faithfully and frankly tell you of your errors are your best friends.} I am sorry, very sorry, for your medical associates. They have been unfaithful to God and untrue to you in failing to tell you kindly but firmly where you were not working righteously.}[Lt253-1903.2; 1903][https://egwwritings.org/read?panels=p9980.8]


\egwnogap{\textbf{Hermano mío, no permita que se aleje de sus hermanos ministros que le hablan de sus peligros. Aquellos que le dicen fiel y francamente sus errores son sus mejores amigos.} Lo siento, lo siento mucho, por sus asociados médicos. Han sido infieles a Dios y falsos contigo al no decirte amable pero firmemente dónde no estabas trabajando rectamente.}[Lt253-1903.2; 1903][https://egwwritings.org/read?panels=p9980.8]


\egwnogap{There are many things that you must overcome before you can be saved. In the heart that is not led by God, there is a something that leads it to desire to be sustained in its wrong course. The men who faithfully tell you the truth, pointing out your mistakes, you have regarded as your enemies. But often they are your best friends and, in telling you wherein you were walking in strange paths, were doing a very disagreeable duty. The Lord’s servants are not to flatter your pride; they are not to stand silent, fearing to say, ‘Why do ye thus?’ They are faithfully to warn you of your danger.}[Lt253-1903.3; 1903][https://egwwritings.org/read?panels=p9980.9]


\egwnogap{Hay muchas cosas que debe superar antes de poder ser salvo. En el corazón que no es guiado por Dios, hay algo que lo lleva a desear ser sostenido en su mal curso. A los hombres que le dicen fielmente la verdad, señalando sus errores, los ha considerado sus enemigos. Pero a menudo son sus mejores amigos y, al decirle por dónde andaba en caminos extraños, estaban cumpliendo un deber muy desagradable. Los siervos del Señor no han de halagar su orgullo; no han de callar, temiendo decir: “¿Por qué hacéis así?”. Han de advertirle fielmente de su peligro.}[Lt253-1903.3; 1903][https://egwwritings.org/read?panels=p9980.9]


\egwnogap{\textbf{My husband, Elder Joseph Bates, Father Pierce, Elder Edson, and many others who were keen, noble, and true were among those who, after the passing of the time in 1844, searched for truth}. \textbf{At our important meetings, these men would meet together and search for the truth as for hidden treasure}. I met with them, and we studied and prayed earnestly; for we felt that we must learn God’s truth. Often we remained together until late at night, and sometimes through the entire night, praying for light and studying the Word. As we fasted and prayed, great power came upon us. But I could not understand the reasoning of the brethren. My mind was locked, as it were, and I could not comprehend what we were studying. Then the Spirit of God would come upon me, I would be taken off in vision, and a clear explanation of the passages we had been studying would be given me with instruction as to the position we were to take regarding truth and duty. Again and again this happened. \textbf{A line of truth extending from that time to the time when we shall enter the city of God was plainly marked out before me}, and I gave my brethren and sisters the instruction that the Lord had given me. They knew that when not in vision, I could not understand these matters, and they accepted as light direct from heaven the revelations given me. \textbf{Thus the leading points of our faith as we hold them today were firmly established}. \textbf{\underline{Point after point} was clearly defined, and all the brethren came into harmony}.}[Lt253-1903.4; 1903][https://egwwritings.org/read?panels=p14068.9980010]


\egwnogap{\textbf{Mi esposo, el élder Joseph Bates, el padre Pierce, el élder Edson y muchos otros que eran agudos, nobles y verdaderos, se encontraban entre los que, después del paso del tiempo en 1844, buscaban la verdad}. \textbf{En nuestras reuniones importantes, estos hombres se reunían y buscaban la verdad como un tesoro escondido}. Yo me reunía con ellos, y estudiábamos y orábamos fervientemente; porque sentíamos que debíamos aprender la verdad de Dios. A menudo permanecíamos juntos hasta altas horas de la noche, y a veces durante toda la noche, orando por luz y estudiando la Palabra. Mientras ayunábamos y orábamos, nos invadía un gran poder. Pero yo no podía entender el razonamiento de los hermanos. Mi mente estaba bloqueada, por así decirlo, y no podía comprender lo que estábamos estudiando. Entonces el Espíritu de Dios venía a mí, me llevaba en visión, y me daba una clara explicación de los pasajes que habíamos estado estudiando, con instrucciones sobre la posición que debíamos tomar respecto a la verdad y el deber. Esto sucedió una y otra vez. \textbf{Una línea de verdad que se extendía desde ese tiempo hasta el momento en que entraríamos en la ciudad de Dios estaba claramente marcada ante mí}, y yo daba a mis hermanos y hermanas la instrucción que el Señor me había dado. Ellos sabían que cuando no estaba en visión, yo no podía entender estos asuntos, y aceptaron como luz directa del cielo las revelaciones que me fueron dadas. \textbf{Así, los puntos principales de nuestra fe, tal como los sostenemos hoy, quedaron firmemente establecidos}. \textbf{\underline{Punto tras punto} fue claramente definido, y todos los hermanos entraron en armonía}.}[Lt253-1903.4; 1903][https://egwwritings.org/read?panels=p14068.9980010]


\egwnogap{\textbf{The whole company of believers were united in the truth}. \textbf{There were those who came in with strange doctrines, but we were never afraid to meet them. Our experience was wonderfully established by the revelations of the Holy Spirit}.}[Lt253-1903.5; 1903][https://egwwritings.org/read?panels=p9980.11]


\egwnogap{\textbf{Toda la compañía de creyentes estaba unida en la verdad}. \textbf{Hubo quienes vinieron con doctrinas extrañas, pero nunca tuvimos miedo de enfrentarnos a ellos. Nuestra experiencia fue maravillosamente establecida por las revelaciones del Espíritu Santo}.}[Lt253-1903.5; 1903][https://egwwritings.org/read?panels=p9980.11]


\egwnogap{For two or three years my mind continued to be locked to the Scriptures. In 1846 I was married to Elder James White. It was some time after my second son was born that we were in great perplexity regarding certain points of doctrine. I was praying to the Lord to unlock my mind, that I might understand His Word. Suddenly I seemed to be enshrouded in clear, beautiful light, and ever since, \textbf{the Scriptures have been an open book to me}.}[Lt253-1903.6; 1903][https://egwwritings.org/read?panels=p14068.9980012]


\egwnogap{Durante dos o tres años mi mente continuó encerrada en las Escrituras. En 1846 me casé con el anciano James White. Fue algún tiempo después de que naciera mi segundo hijo que nos encontramos en gran perplejidad con respecto a ciertos puntos de la doctrina. Estaba orando al Señor para que desbloqueara mi mente, para que pudiera entender su Palabra. De repente me pareció que me envolvía una luz clara y hermosa, y desde entonces, \textbf{las Escrituras han sido un libro abierto para mí}.}[Lt253-1903.6; 1903][https://egwwritings.org/read?panels=p14068.9980012]


\egwnogap{I was at that time in Paris, Maine. Old Father Andrews was very sick. For some time he had been a great sufferer from inflammatory rheumatism. He could not move without intense pain. We prayed for him. I laid my hands on his head, and said, “Father Andrews, the Lord Jesus maketh thee whole.” He was healed instantly. He got up and walked about the room, praising God, and saying, “I never saw it on this wise before. Angels of God are in this room.” The glory of God was revealed. \textbf{Light seemed to shine all through the house, and an angel’s hand was laid upon my head. From that time to this I have been able to understand the Word of God.}}[Lt253-1903.7; 1903][https://egwwritings.org/read?panels=p9980.13]


\egwnogap{Me encontraba en ese momento en París, Maine. El viejo padre Andrews estaba muy enfermo. Desde hacía algún tiempo sufría mucho de reumatismo inflamatorio. No podía moverse sin un intenso dolor. Rezamos por él. Puse mis manos sobre su cabeza y dije: “Padre Andrews, el Señor Jesús te sana”. Se curó al instante. Se levantó y caminó por la habitación, alabando a Dios, y diciendo: “Nunca lo había visto de esta manera. Los ángeles de Dios están en esta habitación”. La gloria de Dios se reveló. \textbf{La luz parecía brillar por toda la casa, y la mano de un ángel se posó sobre mi cabeza. Desde entonces hasta ahora he podido entender la Palabra de Dios.}}[Lt253-1903.7; 1903][https://egwwritings.org/read?panels=p9980.13]


\egwnogap{\textbf{After the passing of the time, we were opposed and cruelly falsified. Erroneous theories were pressed in upon us by men and women who had gone into fanaticism}. I was directed to go to the places where these people were advocating these erroneous theories, and as I went, the power of the Spirit was wonderfully displayed in rebuking the errors that were creeping in. \textbf{\underline{Satan himself, in the person of a man}, was working to make of no effect my testimony regarding the position that we now know to be substantiated by Scripture.}}[Lt253-1903.8; 1903][https://egwwritings.org/read?panels=p9980.14]


\egwnogap{\textbf{Después del paso del tiempo, se nos opuso y falsificó cruelmente. Teorías erróneas fueron presionadas sobre nosotros por hombres y mujeres que habían caído en el fanatismo}. Se me indicó que fuera a los lugares donde estas personas defendían estas teorías erróneas, y mientras iba, el poder del Espíritu se manifestó maravillosamente al reprender los errores que se estaban introduciendo. \textbf{\underline{El propio Satanás, en la persona de un hombre}, estaba obrando para que mi testimonio sobre la posición que ahora sabemos que está corroborada por las Escrituras no tuviera ningún efecto.}}[Lt253-1903.8; 1903][https://egwwritings.org/read?panels=p9980.14]


\egwnogap{\textbf{Just such theories as you have presented in Living Temple were presented then}. \textbf{These subtle, deceiving sophistries have again and again sought to find place amongst us. \underline{But I have ever had the same testimony to bear which I now bear regarding the personality of God}}.}[Lt253-1903.9; 1903][https://egwwritings.org/read?panels=p9980.15]


\egwnogap{\textbf{Justo estas teorías como las que has presentado en Templo Viviente fueron presentadas entonces}. \textbf{Estos sutiles y engañosos sofismas han intentado una y otra vez encontrar lugar entre nosotros. \underline{Pero siempre he tenido que dar el mismo testimonio que ahora doy sobre la personalidad de Dios}}.}[Lt253-1903.9; 1903][https://egwwritings.org/read?panels=p9980.15]


\egwnogap{In (Early Writings, 60, 66, 67)\footnote{It appears that the pages are incorrect. The mentioned paragraphs can be found in Early Writings on pages \href{https://egwwritings.org/read?panels=p28.462&index=0}{70.2}, \href{https://egwwritings.org/read?panels=p28.490&index=0}{77}, and \href{https://egwwritings.org/read?panels=p28.390&index=0}{54.2}.}, are the following statements:}[Lt253-1903.10; 1903][https://egwwritings.org/read?panels=p9980.16]


\egwnogap{En (Primeros Escritos, 60, 66, 67)\footnote{Parece que las páginas son incorrectas. Los párrafos mencionados se pueden encontrar en Primeros Escritos en las páginas \href{https://egwwritings.org/read?panels=p28.462&index=0}{70.2}, \href{https://egwwritings.org/read?panels=p28.490&index=0}{77}, y \href{https://egwwritings.org/read?panels=p28.390&index=0}{54.2}.}, están las siguientes declaraciones:}[Lt253-1903.10; 1903][https://egwwritings.org/read?panels=p9980.16]


\egwnogap{‘May 14, 1851, I saw the beauty and loveliness of Jesus. As I beheld His glory, the thought did not occur to me that I should ever be separated from His presence. \textbf{I saw a light coming from the glory that encircled the Father}, and as it approached near to me, my body shook and trembled like a leaf. I thought that if it should come near me, I would be struck out of existence; but the light passed me. \textbf{Then could I have some sense of the great and terrible \underline{God} with whom we have to do}.’}[Lt253-1903.11; 1903][https://egwwritings.org/read?panels=p9980.17]


\egwnogap{‘El 14 de mayo de 1851, vi la belleza y hermosura de Jesús. Al contemplar su gloria, no se me ocurrió pensar que jamás me separaría de su presencia. \textbf{Vi una luz que salía de la gloria que rodeaba al Padre}, y al acercarse a mí, mi cuerpo se estremeció y tembló como una hoja. Pensé que si se acercaba a mí, quedaría fulminada; pero la luz pasó de largo. \textbf{Entonces pude tener algún sentido del grande y terrible \underline{Dios} con el que tenemos que ver}.’}[Lt253-1903.11; 1903][https://egwwritings.org/read?panels=p9980.17]


\egwnogap{‘I have often seen \textbf{the lovely Jesus, that He is a person}. \textbf{I asked Him if His Father was a person, and had \underline{a form} like Himself}. Said Jesus, ‘\textbf{I am the express image of My Father’s person!}’ [Hebrews 1:3.]}[Lt253-1903.12; 1903][https://egwwritings.org/read?panels=p9980.18]


\egwnogap{‘He visto a menudo \textbf{al encantador Jesús, que es una persona}. \textbf{Le pregunté si su Padre era una persona y tenía \underline{una forma} como Él mismo}. Dijo Jesús: ‘\textbf{¡Yo soy la imagen misma de la sustancia de mi Padre!}’ [Hebreos 1:3.]}[Lt253-1903.12; 1903][https://egwwritings.org/read?panels=p9980.18]


\egwnogap{‘\textbf{I have often seen that the spiritual view took away all the glory of heaven, and that in many minds the throne of David and the lovely person of Jesus have been burned up in the fire of spiritualism}. I have seen that some who have been deceived and led into this error, will be brought out into the light of truth, \textbf{but it will be almost impossible for them to get entirely rid of the deceptive power of spiritualism. Such should make thorough work in confessing their errors, and leaving them forever}.’}[Lt253-1903.13; 1903][https://egwwritings.org/read?panels=p9980.19]


\egwnogap{‘\textbf{He visto a menudo que la visión espiritual quitó toda la gloria del cielo, y que en muchas mentes el trono de David y la hermosa persona de Jesús han sido quemados en el fuego del espiritualismo}. He visto que algunos que han sido engañados y conducidos a este error, serán sacados a la luz de la verdad, \textbf{pero les será casi imposible librarse completamente del poder engañoso del espiritualismo. Los tales deben hacer un trabajo minucioso para confesar sus errores, y dejarlos para siempre}.’}[Lt253-1903.13; 1903][https://egwwritings.org/read?panels=p9980.19]


\egwnogap{\textbf{There is a strain of spiritualism \underline{coming in} among our people, and \underline{it will undermine the faith} of those who give place to it, leading them to give heed to seducing spirits and doctrines of devils}. Errors will be presented in a pleasing and flattering manner. The enemy desires to divert the minds of our brethren and sisters from the work of preparing a people to stand in these last days.}[Lt253-1903.14; 1903][https://egwwritings.org/read?panels=p9980.21]


\egwnogap{\textbf{Hay una corriente de espiritualismo \underline{que se está introduciendo} entre nuestro pueblo, y \underline{minará la fe} de los que le den lugar, llevándolos a prestar atención a los espíritus seductores y a las doctrinas de los demonios}. Los errores se presentarán de manera agradable y halagadora. El enemigo desea desviar las mentes de nuestros hermanos y hermanas de la obra de preparar un pueblo que esté de pie en estos últimos días.}[Lt253-1903.14; 1903][https://egwwritings.org/read?panels=p9980.21]


\egwnogap{I am instructed to warn our brethren and sisters \textbf{not to discuss the nature of our God}. Many of the curious who attempted to open the ark of the testament, to see what was inside, were punished for their presumption. \textbf{We are not to say that the Lord God of heaven is in a leaf, or in a tree; for He is not there. \underline{He sitteth upon His throne in the heavens}.}}[Lt253-1903.15; 1903][https://egwwritings.org/read?panels=p9980.22]


\egwnogap{Se me ha ordenado advertir a nuestros hermanos y hermanas \textbf{que no discutan la naturaleza de nuestro Dios}. Muchos de los curiosos que intentaron abrir el arca del testamento, para ver lo que había dentro, fueron castigados por su presunción. \textbf{No debemos decir que el Señor Dios del cielo está en una hoja o en un árbol, porque no está allí. \underline{Está sentado en su trono en los cielos}.}}[Lt253-1903.15; 1903][https://egwwritings.org/read?panels=p9980.22]


\egwnogap{The work of the Creator as seen in nature reveals His power. But nature is not above God, nor is God in nature as some represent Him to be. God made the world, but the world is not God; it is but the work of His hands. \textbf{Nature reveals the work of a positive, \underline{personal God}, showing that God is, and that He is a rewarder of those who diligently seek Him}.}[Lt253-1903.16, 1903][https://egwwritings.org/read?panels=p9980.23]


\egwnogap{La obra del Creador vista en la naturaleza revela su poder. Pero la naturaleza no está por encima de Dios, ni Dios está en la naturaleza como algunos lo representan. Dios hizo el mundo, pero el mundo no es Dios; no es más que la obra de sus manos. \textbf{La naturaleza revela la obra de un Dios positivo y \underline{personal}}, mostrando que Dios es, y que es un recompensador de aquellos que lo buscan diligentemente.}[Lt253-1903.16, 1903][https://egwwritings.org/read?panels=p9980.23]


\egwnogap{I could say much regarding the sanctuary; the ark containing the law of God; the cover of the ark, which is the mercy seat; the angels at either end of the ark; and other things connected with the heavenly sanctuary and with the great day of atonement. I could say much regarding the mysteries of heaven; but my lips are closed. I have no inclination to try to describe them.}[Lt253-1903.17; 1903][https://egwwritings.org/read?panels=p9980.25]


\egwnogap{Podría decir mucho respecto al santuario; el arca que contiene la ley de Dios; la cubierta del arca, que es el propiciatorio; los ángeles a cada extremo del arca; y otras cosas relacionadas con el santuario celestial y con el gran día de la expiación. Podría decir mucho sobre los misterios del cielo, pero mis labios están cerrados. No tengo ninguna inclinación a tratar de describirlos.}[Lt253-1903.17; 1903][https://egwwritings.org/read?panels=p9980.25]


\egwnogap{\textbf{I would not dare to speak of God as you have spoken of Him}. He is high and lifted up, and His glory fills the heavens. “The voice of the Lord is mighty; it shaketh the cedars of Lebanon. \textbf{The Lord is in His holy temple}; let all the earth keep silence before Him.” [See Psalm 29:5; Habakkuk 2:20.]}[Lt253-1903.18; 1903][https://egwwritings.org/read?panels=p9980.26]


\egwnogap{\textbf{No me atrevería a hablar de Dios como tú lo has hecho}. Él es alto y elevado, y su gloria llena los cielos. “La voz del Señor es poderosa; hace temblar los cedros del Líbano. \textbf{El Señor está en su santo templo}; que toda la tierra guarde silencio ante Él”. [Véase Salmo 29:5; Habacuc 2:20.]}[Lt253-1903.18; 1903][https://egwwritings.org/read?panels=p9980.26]


\egwnogap{\textbf{My brother, when you are tempted to speak of God, \underline{where He is, or what He is}, remember that on this point silence is eloquence}. Take off your shoes from off your feet; for the ground on which you are placing your careless, unsanctified feet is holy ground.}[Lt253-1903.19; 1903][https://egwwritings.org/read?panels=p14068.9980027]


\egwnogap{\textbf{Hermano mío, cuando tengas la tentación de hablar de Dios, \underline{dónde está o qué es}, recuerda que en este punto el silencio es elocuencia}. Quítate los zapatos de los pies; porque el suelo sobre el que estás poniendo tus pies descuidados y no santificados es suelo sagrado.}[Lt253-1903.19; 1903][https://egwwritings.org/read?panels=p14068.9980027]


\egwnogap{\textbf{I am instructed to say that there is nothing in the Word of God to substantiate your spiritualistic theories. Will you not renounce these theories at once? Upon them your mind has been dwelling for a long time, but they have had no sanctifying, refining, ennobling influence upon your life. The Lord has no use for these theories, and He would not have His people vindicate or propagate them.}}[Lt253-1903.20; 1903][https://egwwritings.org/read?panels=p9980.28]


\egwnogap{\textbf{Se me instruye para decir que no hay nada en la Palabra de Dios que corrobore sus teorías espiritualistas. ¿No va a renunciar a estas teorías de inmediato? Tu mente ha estado pensando en ellas durante mucho tiempo, pero no han tenido ninguna influencia santificadora, refinadora y ennoblecedora en tu vida. El Señor no le sirven estas teorías, y no quiere que su pueblo las reivindique o las propague.}}[Lt253-1903.20; 1903][https://egwwritings.org/read?panels=p9980.28]


\egwnogap{\textbf{The Father, the omniscient One, created the world \underline{through} Christ Jesus}. Christ is the light of the world, the way to eternal life. He, the anointed One, God gave to make an atonement for the sins of the world. You need to understand that unless you believe \textbf{in that atonement}, and know that you are bought with the price of the blood of \textbf{the only begotten Son of God}, you will assuredly be bound up with the wicked one. \textbf{If you continue to cherish the theories that you have been cherishing, you will be left to become the sport of Satan’s temptations}. He is playing the game of life for your soul. Remain for a little longer linked up with him, and be assured that you will lose your soul.}[Lt253-1903.21; 1903][https://egwwritings.org/read?panels=p9980.29]


\egwnogap{\textbf{El Padre, el omnisciente, creó el mundo \underline{por medio de} Cristo Jesús}. Cristo es la luz del mundo, el camino a la vida eterna. A Él, el ungido, Dios lo dio para hacer una expiación por los pecados del mundo. Debes entender que a menos que creas \textbf{en esa expiación}, y sepas que has sido comprado con el precio de la sangre de \textbf{el Hijo unigénito de Dios}, seguramente estarás atado al malvado. \textbf{Si continúas acariciando las teorías que has estado acariciando, quedarás para convertirte en el deporte de las tentaciones de Satanás}. Él está jugando el juego de la vida por tu alma. Permanece un poco más vinculado a él, y ten por seguro que perderás tu alma.}[Lt253-1903.21; 1903][https://egwwritings.org/read?panels=p9980.29]


\egwnogap{By declaring that our institutions are undenominational, you have put our people and our work in a false position. You have been led over a terrible path, the dangers of which you have not known, but may sometime see. It is not yet too late for wrongs to be righted. There is hope for you. \textbf{You have followed the enemy step by step, striving to look into mysteries too high and holy for your comprehension}. \textbf{Then in your teaching the Holy One has been brought down to man’s \underline{scientific, spiritualistic ideas}}. You have been walking in crooked paths. You have lost the moral image of God. But there is hope for you. You may still turn your feet into the right path. Will you not now make straight paths for your feet, lest the lame be turned out of the way? Will you now refuse to sow one more seed of skepticism and sophistry in the minds of others? Will you now come to Christ and be healed?}[Lt253-1903.22; 1903][https://egwwritings.org/read?panels=p14068.9980030]


\egwnogap{Al declarar que nuestras instituciones no son confesionales, habéis puesto a nuestro pueblo y a nuestra obra en una posición falsa. Habéis sido conducidos por un camino terrible, cuyos peligros no habéis conocido, pero que algún día podréis ver. Todavía no es demasiado tarde para corregir los errores. Hay esperanza para ti. \textbf{Has seguido al enemigo paso a paso, esforzándote por mirar en misterios demasiado altos y santos para tu comprensión}. \textbf{Luego, en tu enseñanza el Santo ha sido rebajado a las \underline{ideas científicas y espiritualistas del hombre}}. Has andado por caminos torcidos. Has perdido la imagen moral de Dios. Pero hay esperanza para ti. Todavía puedes volver tus pies al camino correcto. ¿No harás ahora sendas rectas para tus pies, para que los cojos no se aparten del camino? ¿Te negarás ahora a sembrar una semilla más de escepticismo y sofisma en la mente de los demás? ¿Vendrás ahora a Cristo y serás curado?}[Lt253-1903.22; 1903][https://egwwritings.org/read?panels=p14068.9980030]


\egwnogap{\textbf{I have hesitated and delayed about the sending out of that which the Spirit of the Lord has impelled me to write}. I did not want to be compelled to present the satanic influence of these sophistries. But unless there is a decided change, in yourself and your associates, I shall have to do this, to save others from following the path that you have been following. I shall have to obey the command given me of God, “\textbf{Meet it}.” This is the only thing that I can do.}[Lt253-1903.23; 1903][https://egwwritings.org/read?panels=p9980.31]


\egwnogap{\textbf{He dudado y retrasado el envío de lo que el Espíritu del Señor me ha impulsado a escribir}. No quería verme obligada a presentar la influencia satánica de estos sofismas. Pero a menos que haya un cambio decidido, en usted y en sus asociados, tendré que hacer esto, para salvar a otros de seguir el camino que usted ha estado siguiendo. Tendré que obedecer el mandato que me ha dado Dios, “\textbf{Enfréntalo}”. Esto es lo único que puedo hacer.}[Lt253-1903.23; 1903][https://egwwritings.org/read?panels=p9980.31]


\egwnogap{I present to you the things that the Lord has presented to me. There is a great work to be done. We are to take hold of the work understandingly, praying, believing, and receiving the Holy Spirit. Thus only can we do the work given us. \textbf{I am required by God to bear testimony against Living Temple}. Whatever your associates may say concerning this book,\textbf{ I take the position now and forever that it is a snare}. \textbf{No union will be formed by our people as a whole upon the \underline{theories} that you have begun to present in that book}. \textbf{You may regard this as forever decided}. \textbf{As a people we shall stand firm \underline{on the platform that has withstood test and trial}. We shall hold to the \underline{sure pillars of our faith}. \underline{The principles of truth} that God has revealed to us are our only foundation. They have made us what we are. These new, fanciful theories are fascinating and misleading. They endanger the eternal interests of the soul. The Scriptures do not sustain them}. Clothed with the Christian armor, shod with the preparation of the gospel of peace, we shall stand \textbf{firm against these misleading theories}. You may turn and wrest the Word of God to your own destruction, but I entreat you not to do this.}[Lt253-1903.24; 1903][https://egwwritings.org/read?panels=p9980.32]


\egwnogap{Te presento las cosas que el Señor me ha presentado. Hay una gran obra por hacer. Debemos tomar la obra con entendimiento, orando, creyendo y recibiendo el Espíritu Santo. Sólo así podremos hacer la obra que se nos ha dado. \textbf{Dios me exige que dé testimonio contra el Templo Viviente}. Independientemente de lo que digan sus asociados con respecto a este libro,\textbf{ tomo la posición ahora y para siempre de que es una trampa}. \textbf{Nuestro pueblo no formará ninguna unión en su conjunto sobre las \underline{teorías} que usted ha comenzado a presentar en ese libro}. \textbf{Puede considerar esto como decidido para siempre}. \textbf{Como pueblo, nos mantendremos firmes \underline{en la plataforma que ha resistido la prueba y el juicio}. Nos aferramos a los \underline{pilares seguros de nuestra fe}. \underline{Los principios de la verdad} que Dios nos ha revelado son nuestro único fundamento. Ellos nos han convertido en lo que somos. Estas nuevas y fantasiosas teorías son fascinantes y engañosas. Ponen en peligro los intereses eternos del alma. Las Escrituras no las sostienen}. Vestidos con la armadura cristiana, calzados con la preparación del evangelio de la paz, nos mantendremos \textbf{firmes contra estas teorías engañosas}. Puedes volverte y retorcer la Palabra de Dios para tu propia destrucción, pero te ruego que no lo hagas.}[Lt253-1903.24; 1903][https://egwwritings.org/read?panels=p9980.32]


\egwnogap{\textbf{Heaven is not a vapor. It is a place}. \textbf{Christ has gone to prepare mansions for those who love Him}, those who, in obedience to His commands, come out from the world and are separate. The principles of heaven must be brought into our experience, that we may be distinguished from the world. \textbf{There must be a marked contrast between us and the world; for we are God’s denominated people}.}[Lt253-1903.25; 1903][https://egwwritings.org/read?panels=p9980.33]


\egwnogap{\textbf{El cielo no es un vapor. Es un lugar}. \textbf{Cristo ha ido a preparar mansiones para los que le aman}, los que, en obediencia a sus mandatos, salen del mundo y se separan. Los principios del cielo deben ser llevados a nuestra experiencia, para que podamos ser distinguidos del mundo. \textbf{Debe haber un marcado contraste entre nosotros y el mundo; porque somos el pueblo denominado de Dios}.}[Lt253-1903.25; 1903][https://egwwritings.org/read?panels=p9980.33]


\egwnogap{The Lord has given you an opportunity to make things right. \textbf{I rejoice that you have made a beginning. Do not think that we have no right to try to correct your errors and the results of these errors. As long as God gives me breath, and commissions me to use pen and voice in beating back this evil thing that has come in among us, I shall act my part in the warfare. Ever since I was seventeen years old, I have had to fight this battle against false theories, in defense of the truth}. \textbf{The history of our past experience is indelibly fixed in my mind, and I am determined that \underline{no theories of the order that you have been accepting} shall come into our ranks}. If you refuse to change, and labor to lead your associates after you, and they venture to follow your leading, the accountability rests with you and with them, not on my soul.}[Lt253-1903.26, 1903][https://egwwritings.org/read?panels=p9980.34]


\egwnogap{El Señor les ha dado la oportunidad de hacer las cosas bien. \textbf{Me alegro de que hayas empezado. No pienses que no tenemos derecho a tratar de corregir tus errores y los resultados de estos errores. Mientras Dios me dé aliento, y me encargue que use la pluma y la voz para rechazar esta cosa maligna que ha entrado entre nosotros, actuaré mi parte en la guerra. Desde que tenía diecisiete años, he tenido que librar esta batalla contra las falsas teorías, en defensa de la verdad}. \textbf{La historia de nuestra experiencia pasada está indeleblemente fijada en mi mente, y estoy decidida a que \underline{ninguna teoría del orden que ustedes han estado aceptando} entre en nuestras filas}. Si te niegas a cambiar, y te esfuerzas por guiar a tus asociados tras de ti, y ellos se aventuran a seguir tu dirección, la responsabilidad recae en ti y en ellos, no en mi alma.}[Lt253-1903.26, 1903][https://egwwritings.org/read?panels=p9980.34]


\egwnogap{\textbf{I speak decidedly, in order that you may know, that unless there is a decided change in you, there can be no hope of a union between you and those who are holding the beginning of their confidence firm unto the end.} You have made the division. \textbf{\underline{We must stand firm for the truths that the Lord has given us as the pillars of our faith}}.}[Lt253-1903.27; 1903][https://egwwritings.org/read?panels=p9980.35]


\egwnogap{\textbf{Hablo con decisión, para que sepas que, a menos que haya un cambio decidido en ti, no puede haber esperanza de una unión entre tú y los que mantienen firme el principio de su confianza hasta el final.} Tú has hecho la división. \textbf{\underline{Debemos mantenernos firmes en las verdades que el Señor nos ha dado como pilares de nuestra fe}}.}[Lt253-1903.27; 1903][https://egwwritings.org/read?panels=p9980.35]


\egwnogap{I entreat you to turn to the Lord with full purpose of heart, before it is forever too late. Separate yourself from the influences which have separated you from your brethren who are engaged in the gospel ministry and from the people whom God is leading. \textbf{\underline{Patchwork theories} cannot be accepted by those who are loyal to the faith and to \underline{the principles} that have withstood all the opposition of satanic influences}.}[Lt253-1903.28; 1903][https://egwwritings.org/read?panels=p9980.36]


\egwnogap{Te ruego que te vuelvas al Señor con pleno propósito de corazón, antes de que sea demasiado tarde para siempre. Sepárate de las influencias que te han separado de tus hermanos que están comprometidos en el ministerio del evangelio y del pueblo que Dios está guiando. \textbf{\underline{Las teorías de parches} no pueden ser aceptadas por quienes son leales a la fe y a \underline{los principios} que han resistido toda la oposición de las influencias satánicas}.}[Lt253-1903.28; 1903][https://egwwritings.org/read?panels=p9980.36]


\egwnogap{If you will empty yourself of all that has separated you from Christ, and receive the Saviour into your heart, you will be transformed in character. Lay off responsibilities for a time, and go away somewhere with a few of your brethren, and with them search the Scriptures. Humble your heart before the Lord, and make thorough work for repentance. \textbf{The religion of Christ is the spiritual leaven that is to be introduced into the heart. This changes the life and character}. This religion is a heavenly principle, seen in the Christian’s life and conversation. It is revealed in Christian purity. The love of Christ is seen in the tenderness and grace of sanctified humanity. It is by the Word made flesh that we are saved. Our redemption was wrought out, \textbf{not by the Son of God’s remaining in heaven, but by the Son of God’s becoming incarnate—taking humanity upon Him and coming to this world}. Thus eternal life was brought to us. That which authority, commands, and promises could not do, God did by coming to this world in the likeness of sinful flesh.}[Lt253-1903.29; 1903][https://egwwritings.org/read?panels=p9980.37]


\egwnogap{Si te despojas de todo lo que te ha separado de Cristo, y recibes al Salvador en tu corazón, serás transformado en carácter. Deja las responsabilidades por un tiempo, y ve a algún lugar con algunos de tus hermanos, y con ellos escudriña las Escrituras. Humilla tu corazón ante el Señor, y haz un trabajo minucioso de arrepentimiento. \textbf{La religión de Cristo es la levadura espiritual que debe ser introducida en el corazón. Esto cambia la vida y el carácter}. Esta religión es un principio celestial, que se ve en la vida y la conversación del cristiano. Se revela en la pureza cristiana. El amor de Cristo se ve en la ternura y la gracia de la humanidad santificada. Es por la Palabra hecha carne que somos salvados. Nuestra redención fue llevada a cabo, \textbf{no por el Hijo de Dios permaneciendo en el cielo, sino por el Hijo de Dios encarnándose—tomando la humanidad sobre Él y viniendo a este mundo}. Así se nos trajo la vida eterna. Lo que la autoridad, los mandatos y las promesas no podían hacer, Dios lo hizo al venir a este mundo en semejanza de carne de pecado.}[Lt253-1903.29; 1903][https://egwwritings.org/read?panels=p9980.37]


\egwnogap{Christ came to the earth to live as a man among men, not to be spoiled by human frailty, but to place in the minds of men principles of truth that could never be obliterated, because they are eternally true. He came to bring a new life to fallen human beings—an excellence that could not be stained or deteriorated by sin.}[Lt253-1903.30; 1903][https://egwwritings.org/read?panels=p9980.38]


\egwnogap{Cristo vino a la tierra para vivir como un hombre entre los hombres, no para ser estropeado por la fragilidad humana, sino para colocar en las mentes de los hombres principios de verdad que nunca podrían ser borrados, porque son eternamente verdaderos. Vino a traer una nueva vida a los seres humanos caídos—una excelencia que no podía ser manchada o deteriorada por el pecado.}[Lt253-1903.30; 1903][https://egwwritings.org/read?panels=p9980.38]


\egwnogap{\textbf{My brother, I must tell you that you have little realization of whither your feet have been tending}. You have been binding yourself up with those who belong to the army of the great apostate. \textbf{Your mind has been as dark as Egypt}. \textbf{If you will fall on the Rock and be broken}, Christ will accept you. But you have been standing on the enemy’s ground, doing his work. \textbf{The religious world is fast going over the same road that you have been following. If you continue to follow this road, you will have plenty of company. But what will the end be?}}[Lt253-1903.31; 1903][https://egwwritings.org/read?panels=p14068.9980039]


\egwnogap{\textbf{Hermano mío, debo decirte que tienes poca conciencia de hacia dónde han estado tendiendo tus pies}. Te has estado uniendo a los que pertenecen al ejército del gran apóstata. \textbf{Tu mente ha sido tan oscura como Egipto}. \textbf{Si caes sobre la Roca y eres quebrantado}, Cristo te aceptará. Pero has estado parado en el terreno del enemigo, haciendo su trabajo. \textbf{El mundo religioso está pasando rápidamente por el mismo camino que tú has estado siguiendo. Si continúas siguiendo este camino, tendrás mucha compañía. Pero, ¿cuál será el final?}}[Lt253-1903.31; 1903][https://egwwritings.org/read?panels=p14068.9980039]


\egwnogap{So long have you been walking in darkness, so long have you followed your own way, that you may be strongly tempted to resist this appeal that I make. If it were not that your \textbf{eternal interests are involved}, I would not speak to you on this subject. It would seem that I have written enough, that there is no need of my urging this subject upon you further. \textbf{But I tell you in truth that I clearly understand what I am doing}. Sufficient light has been given you. But for several years you have not heeded this light. If you had wished to know what the Lord has said, you could have known; \textbf{for you have the books that have been written under the guidance of His Spirit}. You have had all the directions that could be asked for to point out the right way. Direct light has been sent you. But you have looked upon this as of less importance than your own plans and devisings. If you had heeded the testimonies sent you, Living Temple would never have been written.}[Lt253-1903.32; 1903][https://egwwritings.org/read?panels=p9980.40]


\egwnogap{Tanto tiempo has estado caminando en la oscuridad, tanto tiempo has seguido tu propio camino, que puedes estar fuertemente tentado a resistir este llamamiento que te hago. Si no fuera porque tus \textbf{intereses eternos están en juego}, no te hablaría de este tema. Parecería que he escrito lo suficiente, que no hay necesidad de que insista más en este tema. \textbf{Pero te digo en verdad que entiendo claramente lo que estoy haciendo}. Se te ha dado suficiente luz. Pero durante varios años no has prestado atención a esta luz. Si hubieras querido saber lo que el Señor ha dicho, podrías haberlo sabido; \textbf{porque tienes los libros que han sido escritos bajo la guía de Su Espíritu}. Has tenido todas las indicaciones que se podían pedir para señalar el camino correcto. Se te ha enviado luz directa. Pero has considerado esto como de menor importancia que tus propios planes e ideaciones. Si hubieras prestado atención a los testimonios que se te enviaron, Living Temple nunca habría sido escrito.}[Lt253-1903.32; 1903][https://egwwritings.org/read?panels=p9980.40]


\egwnogap{Will you not make a thorough, determined, Christlike effort to break the spell that Satan has cast over you? He has had great power over your mind and has swayed you in wrong lines. He thinks that he can hold you now. Will you not defeat and disappoint him?}[Lt253-1903.33; 1903][https://egwwritings.org/read?panels=p9980.41]


\egwnogap{¿No harás un esfuerzo minucioso, determinado y semejante a Cristo para romper el hechizo que Satanás ha lanzado sobre ti? Él ha tenido gran poder sobre tu mente y te ha influido en líneas equivocadas. Cree que puede retenerte ahora. ¿No lo derrotarás y lo decepcionarás?}[Lt253-1903.33; 1903][https://egwwritings.org/read?panels=p9980.41]


\egwnogap{I write to you as I would to a son. Break away from the enemy—the accuser of the brethren. Say to him, “Get thee behind me Satan. I have committed a grievous sin in heeding your suggestions. I will no longer listen to them.” I beg of you, for your soul’s sake, to resist the tempter, that he may flee from you. Draw near to God, and He will draw near to you. \textbf{You will lose heaven unless you fall on the Rock and are broken}.}[Lt253-1903.34; 1903][https://egwwritings.org/read?panels=p9980.42]


\egwnogap{Te escribo como a un hijo. Sepárate del enemigo—el acusador de los hermanos. Dile: “Apártate de mí, Satanás. He cometido un grave pecado al escuchar tus sugerencias. No las escucharé más”. Te ruego, por el bien de tu alma, que resistas al tentador, para que huya de ti. Acércate a Dios, y Él se acercará a ti. \textbf{Perderás el cielo a menos que caigas sobre la Roca y seas quebrantado}.}[Lt253-1903.34; 1903][https://egwwritings.org/read?panels=p9980.42]


Many things in this letter to Dr. Kellogg go without being said, yet are explained when the context is understood. Ellen White read the letter from Brother Daniells expressing how Dr. Kellogg wanted to revise the Living Temple because he\others{had been thinking the matter over, and began to see that he had made a slight mistake in \textbf{expressing }his views}, and\others{that within a short time \textbf{he had come to believe in the trinity} and could now see pretty clearly where all the difficulty was, and believed that he could clear the matter up satisfactorily}. Kellogg confessed,\others{that he now believed \textbf{in God the Father, God the Son, and God the Holy Ghost}}. In answer to that, Sister White personally wrote to him:\egwinline{The book Living Temple \textbf{is not to be patched up}, a few changes made in it, and then advertised and praised as a valuable production}. How did Kellogg want to patch up his book? According to A. G. Daniells’ testimony, he thought to change a few expressions by explicitly stating his trinitarian sentiment. But the expression of the views was not the real problem—it was the views themselves. Sister White did not spare rebuking him for his views of God, which were \textit{trinitarian} views. She told him that she is\egwinline{\textbf{determined that \underline{no theories of the order that you have been accepting} shall come into our ranks}}. This is a very strong statement. Could it be that, since Kellogg confessed that he was accepting the Trinity doctrine, Sister White was also including it in her statement? It seems unthinkable because this doctrine is in our ranks today. But her statement actually pinpoints the Trinity when she said:\egwinline{\textbf{Patchwork theories} cannot be accepted by those who are loyal \textbf{to the faith and to the principles} that have withstood all the opposition of satanic influences}. Kellogg wanted to patch up “\textit{Living Temple}” by explicitly mentioning the Trinity doctrine. Why was Sister White determined to keep this doctrine out of our ranks, yet it is in our ranks today? It is fair to point out that the Trinity was not part of Seventh-day Adventist faith in her time and it came into our ranks later. Today, many argue that it was because of her works that the Trinity is a part of our beliefs, but Ellen White’s reaction, and her answer to Kellogg’s belief in it, showcases how she dealt with such doctrine. What can we learn from that?


Muchas cosas en esta carta al Dr. Kellogg no se dicen, pero se explican cuando se entiende el contexto. Elena White leyó la carta del hermano Daniells expresando cómo el Dr. Kellogg quería revisar el Living Temple porque\others{había estado pensando en el asunto, y empezó a ver que había cometido un ligero error al \textbf{expresar }sus opiniones}, y\others{que en poco tiempo \textbf{había llegado a creer en la trinidad} y ahora podía ver con bastante claridad dónde estaba toda la dificultad, y creía que podía aclarar el asunto satisfactoriamente}. Kellogg confesó,\others{que ahora creía \textbf{en Dios Padre, Dios Hijo y Dios Espíritu Santo}}. En respuesta a eso, la hermana White le escribió personalmente:\egwinline{El libro Living Temple \textbf{no es para ser remendado}, hacer unos pocos cambios en él, y luego publicitarlo y alabarlo como una producción valiosa}. ¿Cómo quería Kellogg remendar su libro? Según el testimonio de A. G. Daniells, pensó en cambiar algunas expresiones declarando explícitamente su sentimiento trinitario. Pero la expresión de los puntos de vista no era el verdadero problema—eran los puntos de vista mismos. La hermana White no escatimó en reprenderle por sus puntos de vista sobre Dios, que eran puntos de vista \textit{trinitarios}. Ella le dijo que estaba\egwinline{\textbf{decidida a que \underline{ninguna teoría del orden que has estado aceptando} entre en nuestras filas}}. Esta es una declaración muy fuerte. ¿Podría ser que, ya que Kellogg confesó que estaba aceptando la doctrina de la Trinidad, la hermana White también la estaba incluyendo en su declaración? Parece impensable porque esta doctrina está en nuestras filas hoy. Pero su declaración en realidad señala a la Trinidad cuando dijo:\egwinline{\textbf{Las teorías de parches} no pueden ser aceptadas por aquellos que son leales \textbf{a la fe y a los principios} que han resistido toda la oposición de las influencias satánicas}. Kellogg quería remendar el “\textit{Living Temple}” mencionando explícitamente la doctrina de la Trinidad. ¿Por qué la hermana White estaba decidida a mantener esta doctrina fuera de nuestras filas, y sin embargo está en nuestras filas hoy? Es justo señalar que la Trinidad no formaba parte de la fe adventista del séptimo día en su época y llegó a nuestras filas más tarde. Hoy, muchos argumentan que fue gracias a sus obras que la Trinidad forma parte de nuestras creencias, pero la reacción de Elena White, y su respuesta a la creencia de Kellogg en ella, muestra cómo trataba dicha doctrina. ¿Qué podemos aprender de eso?


Taken in its context, this letter sheds new light on Kellogg’s controversy and demonstrates how we should deal with the Trinity doctrine. The first thing we question is why Sister White never used the word “Trinity” in her writings, even when she was directly dealing with this doctrine? Elsewhere, she answers:


Tomada en su contexto, esta carta arroja nueva luz sobre la controversia de Kellogg y demuestra cómo debemos tratar la doctrina de la Trinidad. Lo primero que nos preguntamos es por qué la hermana White nunca usó la palabra “Trinidad” en sus escritos, incluso cuando trataba directamente con esta doctrina. En otra parte, ella responde:


\egw{I was cautioned not to enter into controversy \textbf{regarding the question} that \textbf{\underline{will come up}} over \textbf{these things, because controversy \underline{might lead men to resort to subterfuges, and their minds would be led away from the truth of the Word of God to assumption and guesswork}}. \textbf{The more that fanciful theories are discussed, the \underline{less men will know of God and of the truth that sanctifies the soul}}.}[Lt232-1903.41; 1903][https://egwwritings.org/read?panels=p14068.10197050]


\egw{Se me advirtió que no entrara en controversia \textbf{con respecto a la cuestión} que \textbf{\underline{surgirá}} sobre \textbf{estas cosas, porque la controversia \underline{podría llevar a los hombres a recurrir a subterfugios, y sus mentes serían desviadas de la verdad de la Palabra de Dios hacia suposiciones y conjeturas}}. \textbf{Cuanto más se discutan las teorías extravagantes, \underline{menos conocerán los hombres de Dios y de la verdad que santifica el alma}}.}[Lt232-1903.41; 1903][https://egwwritings.org/read?panels=p14068.10197050]


This is a very important lesson and principle that Sister White is teaching us here. When the controversy over Kellogg’s theories arose, she did not venture into the theories themselves, because this would lead the minds of men away from the truth of the Word of God to assumption and guesswork. Rather, she led the minds of men into the truth, which sanctifies the soul. She led by example, evident here in her letter to Dr. Kellogg. This truth that she led the minds of men to, was the truth on the \emcap{personality of God}. She rebuked Kellogg for his theories but, very importantly, we properly identify these theories by their context and her implicit expression of them.


Esta es una lección y un principio muy importantes que la hermana White nos enseña aquí. Cuando surgió la controversia sobre las teorías de Kellogg, ella no se aventuró en las teorías mismas, porque esto alejaría las mentes de los hombres de la verdad de la Palabra de Dios hacia suposiciones y conjeturas. Más bien, condujo las mentes de los hombres hacia la verdad, que santifica el alma. Ella guió con el ejemplo, evidente aquí en su carta al Dr. Kellogg. Esta verdad a la que condujo las mentes de los hombres, fue la verdad sobre la \emcap{personalidad de Dios}. Ella reprendió a Kellogg por sus teorías pero, muy importante, identificamos adecuadamente estas teorías por su contexto y su expresión implícita de ellas.


We see that she made a contrast between the Trinity and the \emcap{personality of God}. She made a contrast between the old principles of our faith and the new theories. First, she drew our minds back to the beginning of our spiritual heritage,\egwinline{after the passing of the time in 1844}, when her husband James White, Joseph Bates, Father Pierce, Elder Edson, and many others who were keen, noble, and true, searched for truth. She pointed back to the wonderful and mighty experiences of how the leading points of our faith, held in 1903, were firmly established. \egwinline{\textbf{Thus \underline{the leading points of our faith}} as we hold them today were firmly established.} \egwinline{\textbf{\underline{Point after point} was clearly defined, and all the brethren came into harmony}.} \egwinline{\textbf{The whole company of believers were united in the truth}}. Obviously, from the context of chapter 10 of the Special Testimonies, we know that these experiences explain \egwinline{\textbf{how firmly the foundation of our faith has been laid}}[SpTB02 56.4; 1904][https://egwwritings.org/read?panels=p417.288]. This foundation is expressed in the \emcap{Fundamental Principles}\footnote{\href{https://static1.squarespace.com/static/554c4998e4b04e89ea0c4073/t/59d17e24c027d84167e17617/1506901547915/SDA-YB1905+\%28P.+188-192\%29.pdf}{Yearbook Of Seventh-day Adventist denomination 1905, p. 188-192}}. This foundation is the truth which,\egwinline{\textbf{\underline{point by point}}, \textbf{has been sought out by prayerful study, and testified to by the miracle-working power of the Lord}}. God \egwinline{\textbf{calls upon us to \underline{hold firmly}, with the grip of faith, to \underline{the fundamental principles} that are \underline{based upon unquestionable authority}}.}[SpTB02 59.1; 1904][https://egwwritings.org/read?panels=p417.299] In light of these experiences and the truth expressed in the \emcap{fundamental principles}, \egwinline{\textbf{\underline{Patchwork theories} cannot be accepted by those who are loyal \underline{to the faith} and \underline{to the principles} that have withstood all the opposition of satanic influences}}[Lt253-1903.28; 1903][https://egwwritings.org/read?panels=p14068.9980036]. From the historical record of these brethren who were keen, noble and true, we have evidence that they, too, have contrasted the Trinity doctrine with the truth on the \emcap{personality of God}. James White, in the Review and Herald article, listed \others{some of the popular fables of the age}, saying: \others{Here we might mention \textbf{the Trinity, which \underline{does away the personality of God, and of his Son Jesus Christ}}}[James White, Review \& Herald, December 11, 1855, p. 85.15][http://documents.adventistarchives.org/Periodicals/RH/RH18551211-V07-11.pdf]. J. N. Andrews said, \others{\textbf{The doctrine of the Trinity which was established in the church by the council of Nicea, A. D. 325}. \textbf{This doctrine \underline{destroys the personality of God, and his Son Jesus Christ our Lord}}...}[J. N. Andrews, Review \& Herald, March 6, 1855, p. 185][http://documents.adventistarchives.org/Periodicals/RH/RH18550306-V06-24.pdf] J. B. Frisbie, in his article “\textit{Seventh-day Sabbath not abolished}”, compares the Sabbath God to the Sunday god; he describes the Sabbath God in light of the \emcap{personality of God} expressed in the first point of the \emcap{Fundamental Principles}. The Sunday god is described by the \others{unity of this God-head, there are three persons of one substance, power and eternity; the Father, the Son, and the Holy Ghost}[J. B. Frisbie, Review \& Herald March 7, 1854. p. 50][http://documents.adventistarchives.org/Periodicals/RH/RH18540307-V05-07.pdf]. He explained how the doctrine on the \emcap{personality of God} stands in conflict with the doctrine of Trinity, in the same way the Holy Sabbath stands in conflict with pagan Sunday worship. Also, brother J. N. Loughborough wrote the objections to the Trinity doctrine in the Adventist Review and Sabbath Herald\footnote{\href{https://adventistdigitallibrary.org/adl-349160/advent-review-and-sabbath-herald-november-5-1861}{J. N. Loughborough, November 5, 1861, Review \& Herald, vol. 18, p. 184, par. 1-11}}. In the other publication of the Review and Herald, he published the article “\textit{Is God a person?}”, explaining the position of Seventh-day Adventist belief on the \emcap{personality of God}, expressed in the first point of the \emcap{Fundamental Principles}\footnote{\href{http://documents.adventistarchives.org/Periodicals/RH/RH18550918-V07-06.pdf}{J. N. Loughborough, September 18. 1855, Review \& Herald, vol. 7, p. 6.}}. James White was also explaining the same position in his multiple print pamphlet, “\textit{The Personality of God}”\footnote{\href{https://egwwritings.org/?ref=en_PERGO.1.1&para=1471.3}{J. White, The Personality of God, June 18. 1861.}}. These are just a few examples where the Adventist pioneers explained the position on the \emcap{personality of God} expressed by the first point of the \emcap{fundamental principles}.


Vemos que ella hizo un contraste entre la Trinidad y la \emcap{personalidad de Dios}. Hizo un contraste entre los antiguos principios de nuestra fe y las nuevas teorías. En primer lugar, dirigió nuestras mentes al comienzo de nuestra herencia espiritual,\egwinline{después del paso del tiempo en 1844}, cuando su esposo James White, Joseph Bates, el Padre Pierce, el Anciano Edson, y muchos otros que eran agudos, nobles y verdaderos, buscaron la verdad. Señaló las maravillosas y poderosas experiencias de cómo los puntos principales de nuestra fe, sostenidos en 1903, fueron firmemente establecidos. \egwinline{\textbf{Así \underline{los puntos principales de nuestra fe}} tal como los sostenemos hoy fueron firmemente establecidos.} \egwinline{\textbf{\underline{Punto tras punto} fue claramente definido, y todos los hermanos entraron en armonía}.} \egwinline{\textbf{Toda la compañía de creyentes estaba unida en la verdad}}. Obviamente, por el contexto del capítulo 10 de los Testimonios Especiales, sabemos que estas experiencias explican \egwinline{\textbf{cuán firmemente se ha establecido el fundamento de nuestra fe}}[SpTB02 56.4; 1904][https://egwwritings.org/read?panels=p417.288]. Este fundamento está expresado en los \emcap{Principios Fundamentales}\footnote{\href{https://static1.squarespace.com/static/554c4998e4b04e89ea0c4073/t/59d17e24c027d84167e17617/1506901547915/SDA-YB1905+\%28P.+188-192\%29.pdf}{Yearbook Of Seventh-day Adventist denomination 1905, p. 188-192}}. Este fundamento es la verdad que,\egwinline{\textbf{\underline{punto por punto}}, \textbf{ha sido buscada por el estudio en oración, y testificada por el poder milagroso del Señor}}. Dios \egwinline{\textbf{nos llama a \underline{aferrarnos firmemente}, con la garra de la fe, a los \underline{principios fundamentales} que están \underline{basados en una autoridad incuestionable}}.}[SpTB02 59.1; 1904][https://egwwritings.org/read?panels=p417.299] A la luz de estas experiencias y de la verdad expresada en los \emcap{principios fundamentales}, \egwinline{\textbf{Las \underline{teorías de parches} no pueden ser aceptadas por aquellos que son leales a \underline{la fe} y a \underline{los principios} que han resistido toda la oposición de las influencias satánicas}}[Lt253-1903.28; 1903][https://egwwritings.org/read?panels=p14068.9980036]. Del registro histórico de estos hermanos que eran agudos, nobles y verdaderos, tenemos evidencia de que ellos también contrastaron la doctrina trinitaria con la verdad sobre la \emcap{personalidad de Dios}. James White, en el artículo de la Review and Herald, enumeró \others{algunas de las fábulas populares de la época}, diciendo: \others{Aquí podríamos mencionar \textbf{la Trinidad, que \underline{elimina la personalidad de Dios, y de su Hijo Jesucristo}}}[James White, Review \& Herald, December 11, 1855, p. 85.15][http://documents.adventistarchives.org/Periodicals/RH/RH18551211-V07-11.pdf]. J. N. Andrews dijo, \others{\textbf{La doctrina de la Trinidad que fue establecida en la iglesia por el concilio de Nicea, A. D. 325}. \textbf{Esta doctrina \underline{destruye la personalidad de Dios, y de su Hijo Jesucristo nuestro Señor}}...}[J. N. Andrews, Review \& Herald, March 6, 1855, p. 185][http://documents.adventistarchives.org/Periodicals/RH/RH18550306-V06-24.pdf] J. B. Frisbie, en su artículo “\textit{Seventh-day Sabbath not abolished}”, compara al Dios del sábado con el dios del domingo; describe al Dios del sábado a la luz de la \emcap{personalidad de Dios} expresada en el primer punto de los \emcap{Principios Fundamentales}. El dios dominical es descrito por la \others{unidad de esta Deidad, hay tres personas de una sustancia, poder y eternidad; el Padre, el Hijo y el Espíritu Santo}[J. B. Frisbie, Review \& Herald March 7, 1854. p. 50][http://documents.adventistarchives.org/Periodicals/RH/RH18540307-V05-07.pdf]. Explicó cómo la doctrina sobre la \emcap{personalidad de Dios} está en conflicto con la doctrina de la Trinidad, de la misma manera que el Santo Sábado está en conflicto con la adoración pagana del domingo. También, el hermano J. N. Loughborough escribió las objeciones a la doctrina trinitaria en la Adventist Review and Sabbath Herald\footnote{\href{https://adventistdigitallibrary.org/adl-349160/advent-review-and-sabbath-herald-november-5-1861}{J. N. Loughborough, November 5, 1861, Review \& Herald, vol. 18, p. 184, par. 1-11}}. En la otra publicación de la Review and Herald, publicó el artículo “\textit{Is God a person?}”, explicando la posición de la creencia adventista del séptimo día sobre la \emcap{personalidad de Dios}, expresada en el primer punto de los \emcap{Principios Fundamentales}\footnote{\href{http://documents.adventistarchives.org/Periodicals/RH/RH18550918-V07-06.pdf}{J. N. Loughborough, September 18. 1855, Review \& Herald, vol. 7, p. 6.}}. James White también explicaba la misma posición en su folleto de impresión múltiple, “\textit{The Personality of God}”\footnote{\href{https://egwwritings.org/?ref=en_PERGO.1.1&para=1471.3}{J. White, The Personality of God, June 18. 1861.}}. Estos son sólo algunos ejemplos donde los pioneros adventistas explicaron la posición sobre la \emcap{personalidad de Dios} expresada por el primer punto de los \emcap{principios fundamentales}.


Sister White rebuked Kellogg:\egwinline{\textbf{But I tell you in truth that I clearly understand what I am doing}. \textbf{Sufficient light has been given you}. But for several years you have not heeded this light. If you had wished to know what the Lord has said, you could have known; \textbf{for \underline{you have the books} that have been written under the guidance of His Spirit}. You have had all the directions that could be asked for to point out the right way. Direct light has been sent you. But you have looked upon this as of less importance than your own plans and devisings. If you had heeded the testimonies sent you, Living Temple would never have been written.}[Lt253-1903.32; 1903][https://egwwritings.org/read?panels=p14068.9980040] The core issue of Dr. Kellogg’s controversy was \egwinline{the personality of God and where His presence is}[SpTB02 51.3; 1904][https://egwwritings.org/read?panels=p417.262]. Dr. Kellogg had access to the pioneer writings, books and the church's \emcap{Fundamental Principles} that were testified to by the miracle working power of the Holy Spirit.


La hermana White reprendió a Kellogg:\egwinline{\textbf{Pero te digo en verdad que entiendo claramente lo que estoy haciendo}. \textbf{Se te ha dado suficiente luz}. Pero durante varios años no has prestado atención a esta luz. Si hubieras querido saber lo que el Señor ha dicho, podrías haberlo sabido; \textbf{porque \underline{tienes los libros} que han sido escritos bajo la guía de Su Espíritu}. Has tenido todas las indicaciones que se podían pedir para señalar el camino correcto. Se te ha enviado luz directa. Pero has considerado esto como de menor importancia que tus propios planes e ideaciones. Si hubieras prestado atención a los testimonios que se te enviaron, el Templo Viviente nunca se habría escrito.}[Lt253-1903.32; 1903][https://egwwritings.org/read?panels=p14068.9980040] El tema central de la controversia del Dr. Kellogg era \egwinline{la personalidad de Dios y dónde está su presencia}[SpTB02 51.3; 1904][https://egwwritings.org/read?panels=p417.262]. El Dr. Kellogg tuvo acceso a los escritos de los pioneros, libros y los \emcap{Principios Fundamentales} de la iglesia que fueron testificados por el poder milagroso del Espíritu Santo.


Sister White recalled the experiences of how the \textit{leading points of our faith}, as were held in former times, were firmly established.\egwinline{\textbf{\underline{Point after point} was clearly defined, and all the brethren came into harmony}}[Lt253-1903.4; 1903][https://egwwritings.org/read?panels=p14068.9980010]. These leading points were the \emcap{Fundamental Principles}, of which the \emcap{personality of God} was one. This point, and Sister White’s testimony of it, remained the same during the course of her life.  She said\egwinline{\textbf{\underline{I have ever had the same testimony to bear which I now bear regarding the personality of God}}}[Lt253-1903.9; 1903][https://egwwritings.org/read?panels=p14068.9980015]. From Early Writings, she then quoted her visions of the Heavenly reality. She recalled how she had had the privilege to be in the presence of God, how God, encircled by the light of His glory, passed by her side. She did not see God from the light He was encircled by; she was afraid of Him, thinking that if He were to approach her she\egwinline{would be struck out of existence}. Then she saw\egwinline{\textbf{the lovely Jesus, that He is a person}. \textbf{I asked Him if His Father was a person, and had \underline{a form like} Himself}. Said Jesus, ‘\textbf{I am the express image of My Father’s person!}’}[Lt253-1903.12; 1903][https://egwwritings.org/read?panels=p14068.9980018]. The question she had was: \textit{is God a person, having a form like Jesus}? The answer was affirmative—with a strong biblical foundation. Her visions were not the source of the truth on the \emcap{personality of God}; rather, they confirmed the truth the pioneers had discovered through diligent study of God’s word.


La hermana White recordó las experiencias de cómo los \textit{puntos principales de nuestra fe}, tal como se sostenían en tiempos anteriores, fueron firmemente establecidos.\egwinline{\textbf{\underline{Punto tras punto} fue claramente definido, y todos los hermanos entraron en armonía}}[Lt253-1903.4; 1903][https://egwwritings.org/read?panels=p14068.9980010]. Estos puntos principales eran los \emcap{Principios Fundamentales}, de los cuales la \emcap{personalidad de Dios} era uno. Este punto, y el testimonio de la hermana White al respecto, permaneció igual durante el curso de su vida. Ella dijo\egwinline{\textbf{\underline{Siempre he tenido el mismo testimonio que ahora doy sobre la personalidad de Dios}}}[Lt253-1903.9; 1903][https://egwwritings.org/read?panels=p14068.9980015]. De los Primeros Escritos, citó entonces sus visiones de la realidad celestial. Recordó cómo había tenido el privilegio de estar en la presencia de Dios, cómo Dios, rodeado por la luz de Su gloria, pasó a su lado. Ella no veía a Dios desde la luz que lo rodeaba; tenía miedo de Él, pensando que si se acercaba a ella\egwinline{sería fulminada}. Entonces vio\egwinline{\textbf{al encantador Jesús, que es una persona}. \textbf{Le pregunté si Su Padre era una persona, y tenía \underline{una forma como} Él mismo}. Dijo Jesús, ‘\textbf{¡Yo soy la imagen misma de la sustancia de Mi Padre!}’}[Lt253-1903.12; 1903][https://egwwritings.org/read?panels=p14068.9980018]. La pregunta que tenía era: \textit{¿es Dios una persona, que tiene una forma como la de Jesús?} La respuesta fue afirmativa—con un fuerte fundamento bíblico. Sus visiones no eran la fuente de la verdad sobre la \emcap{personalidad de Dios}; más bien, confirmaban la verdad que los pioneros habían descubierto a través del estudio diligente de la palabra de Dios.


Therefore, their final conclusion on the \emcap{personality of God} was,\others{That there is \textbf{one God}, \textbf{a personal, spiritual \underline{being}}, \textbf{the creator of all things}, omnipotent, omniscient, and eternal, infinite in wisdom, holiness, justice, goodness, truth, and mercy; unchangeable, and \textbf{everywhere present by his representative, the Holy Spirit}. Ps. 139:7; That there is one Lord Jesus Christ, \textbf{the Son of the Eternal Father, the one by whom he created all things, and by whom they do consist} …and as the closing portion of his work as priest, before he takes his throne as king, he will make \textbf{the great atonement} for the sins of all such, and their sins will then be blotted out (Acts 3:19) and borne away from the sanctuary, as shown in the service of the Levitical priesthood, which foreshadowed and prefigured the ministry of our Lord in heaven. See Lev. 16; Heb. 8: 4, 5; 9: 6, 7; etc.}[The first, and part of the second, point of the Fundamental Principles, 1905.]


Por lo tanto, su conclusión final sobre la \emcap{personalidad de Dios} fue,\others{Que hay \textbf{un solo Dios}, \textbf{un \underline{ser} personal y espiritual}, \textbf{el creador de todas las cosas}, omnipotente, omnisciente y eterno, infinito en sabiduría, santidad, justicia, bondad, verdad y misericordia; inmutable, y \textbf{presente en todas partes por su representante, el Espíritu Santo}. Sal. 139:7; Que hay un solo Señor Jesucristo, \textbf{el Hijo del Padre Eterno, aquel por quien creó todas las cosas, y por quien éstas consisten} …y como parte final de su obra como sacerdote, antes de tomar su trono como rey, hará \textbf{la gran expiación} por los pecados de todos ellos, y sus pecados serán entonces borrados (Hechos 3:19) y llevados fuera del santuario, como se muestra en el servicio del sacerdocio levítico, que prefiguraba y prefigura el ministerio de nuestro Señor en el cielo. Véase Lev. 16; Heb. 8: 4, 5; 9: 6, 7; etc.}[El primer punto, y parte del segundo, de los Principios Fundamentales, 1905.]


Ellen White reminded Dr. Kellogg on this point of the \emcap{fundamental principles} by stating:\egwinline{\textbf{The Father, the omniscient One, created the world \underline{through} Christ Jesus}. Christ is the light of the world, the way to eternal life. \textbf{He, the anointed One, God gave to make an atonement for the sins of the world}...}[Lt253-1903.21; 1903][https://egwwritings.org/read?panels=p14068.9980029]


Ellen White recordó al Dr. Kellogg sobre este punto de los \emcap{principios fundamentales} al afirmar:\egwinline{\textbf{El Padre, el omnisciente, creó el mundo \underline{por medio de} Cristo Jesús}. Cristo es la luz del mundo, el camino a la vida eterna. \textbf{A Él, el ungido, Dios lo dio para hacer expiación por los pecados del mundo}...}[Lt253-1903.21; 1903][https://egwwritings.org/read?panels=p14068.9980029]


The question on the \emcap{personality of God} deals with the quality or state of God being a person. The Adventist pioneers gave an answer to it and God approved it through the writings of Ellen White: God is a \textit{personal spiritual Being} and He is our heavenly Father. Where is His presence?\egwinline{\textbf{We are not to say that the Lord God of heaven is in a leaf, or in a tree; for He is not there. \underline{He sitteth upon His throne in the heavens}}.}[Lt253-1903.15; 1903][https://egwwritings.org/read?panels=p14068.9980022] \\
His presence is on the throne in heaven. \\
\egwinline{\textbf{Heaven is not a vapor. It is a place}. \textbf{Christ has gone to prepare mansions for those who love Him}, those who, in obedience to His commands, come out from the world and are separate...}[EGW, Lt253-1903.25; 1903][https://egwwritings.org/read?panels=p14068.9980033]. \\
“...\egwinline{‘The voice of the Lord is mighty; it shaketh the cedars of Lebanon. \textbf{The Lord is in His holy temple}; let all the earth keep silence before Him.’ [See Psalm 29:5; Habakkuk 2:20.]}[Lt253-1903.18; 1903][https://egwwritings.org/read?panels=p14068.9980026]


La pregunta sobre la \emcap{personalidad de Dios} trata de la cualidad o estado de Dios siendo una persona. Los pioneros adventistas dieron una respuesta a ella y Dios la aprobó a través de los escritos de Ellen White: Dios es un \textit{Ser espiritual personal} y es nuestro Padre celestial. ¿Dónde está Su presencia?\egwinline{\textbf{No debemos decir que el Señor Dios del cielo está en una hoja o en un árbol; porque no está allí. \underline{Está sentado en Su trono en los cielos}}.}[Lt253-1903.15; 1903][https://egwwritings.org/read?panels=p14068.9980022] \\
Su presencia está en el trono en el cielo. \\
\egwinline{\textbf{El cielo no es un vapor. Es un lugar}. \textbf{Cristo ha ido a preparar mansiones para los que le aman}, aquellos que, en obediencia a sus mandatos, salen del mundo y se separan...}[EGW, Lt253-1903.25; 1903][https://egwwritings.org/read?panels=p14068.9980033]. \\
“...\egwinline{‘La voz del Señor es poderosa; hace temblar los cedros del Líbano. \textbf{El Señor está en Su santo templo}; que toda la tierra guarde silencio ante Él.’ [Véase Salmo 29:5; Habacuc 2:20.]}[Lt253-1903.18; 1903][https://egwwritings.org/read?panels=p14068.9980026]


According to Adventist pioneers and Sister White, our heavenly Father is one God. He is a personal Spiritual Being, present in heaven, on His throne. The throne of heaven is a real, physical throne, upon which sits a real Person (Being, having a form, just like Jesus)—our heavenly Father. That place is a real place; it is not a vapor, or any other spiritual view.


Según los pioneros adventistas y la hermana White, nuestro Padre celestial es un solo Dios. Es un Ser espiritual personal, presente en el cielo, en Su trono. El trono del cielo es un trono real, físico, sobre el cual se sienta una Persona real (Ser, que tiene una forma, igual que Jesús)—nuestro Padre celestial. Ese lugar es un lugar real; no es un vapor, o cualquier otra vista espiritual.


\egwinline{\textbf{I have often seen that the spiritual view took away all the glory of heaven, and that in many minds the throne of David and the lovely person of Jesus have been burned up in the fire of spiritualism}. I have seen that some who have been deceived and led into this error, will be brought out into the light of truth, \textbf{but it will be almost impossible for them to get entirely rid of the deceptive power of spiritualism. Such should make thorough work in confessing their errors, and leaving them forever}.}[Lt253-1903.13; 1903][https://egwwritings.org/read?panels=p14068.9980019]


\egwinline{\textbf{He visto a menudo que la vista espiritual quitó toda la gloria del cielo, y que en muchas mentes el trono de David y la hermosa persona de Jesús han sido quemados en el fuego del espiritualismo}. He visto que algunos que han sido engañados y conducidos a este error, serán sacados a la luz de la verdad, \textbf{pero será casi imposible que se libren completamente del poder engañoso del espiritualismo. Los tales deben hacer un trabajo minucioso para confesar sus errores, y dejarlos para siempre}.}[Lt253-1903.13; 1903][https://egwwritings.org/read?panels=p14068.9980019]


The spiritual view of God’s person is an erroneous view. In the Bible we have testimonies of heaven, the heavenly throne, and God who is sitting upon it. If we accept these testimonies in their obvious meaning, then the Trinity doctrine cannot be sustained. The Bible and Spirit of Prophecy present one God in heaven, as a personal being, having a body and form just as Jesus has. This view is not in harmony with the doctrine of the Triune God, since it requires the Holy Spirit to be a Being\footnote{Please look at \hyperref[appendix:unauthenticated-reports]{the appendix} for more quotations which exclude the Holy Spirit to be a being, possessing physical body and form.}, having a body and form—this idea would compromise the Holy Spirit to be a means of the Father and Son by which They are everywhere present. In order to sustain the Trinity doctrine, the testimonies regarding the throne of God and of God’s person, need to be understood by some spiritual view. Here we have seen that Sister White contrasted the truth of the \emcap{personality of God} with the doctrine of Trinity. She contrasted the doctrine of Trinity with the first two points of the \emcap{Fundamental Principles}, which were the results of our pioneers studying the Word of God. Referring to the pioneers and the \emcap{Fundamental Principles}, she said: \egwinline{\textbf{\underline{Patchwork theories} cannot be accepted by those who are \underline{loyal to the faith and to the principles} that have withstood all the opposition of satanic influences.}}[Lt253-1903.28; 1903][https://egwwritings.org/read?panels=p14068.9980036]


La vista espiritual de la persona de Dios es una visión errónea. En la Biblia tenemos testimonios del cielo, del trono celestial y de Dios que está sentado en él. Si aceptamos estos testimonios en su significado obvio, entonces la doctrina trinitaria no puede ser sostenida. La Biblia y el Espíritu de Profecía presentan a un solo Dios en el cielo, como un ser personal, que tiene un cuerpo y una forma igual a la de Jesús. Este punto de vista no está en armonía con la doctrina del Dios Trino, ya que requiere que el Espíritu Santo sea un Ser\footnote{Por favor, consulta \hyperref[appendix:unauthenticated-reports]{el apéndice} para más citas que excluyen que el Espíritu Santo sea un ser, que posea cuerpo físico y forma.}, que tenga un cuerpo y una forma—esta idea comprometería al Espíritu Santo a ser un medio del Padre y del Hijo por el cual Ellos están presentes en todas partes. Para sostener la doctrina trinitaria, los testimonios relativos al trono de Dios y a la persona de Dios, necesitan ser entendidos por alguna vista espiritual. Aquí hemos visto que la hermana White contrastó la verdad de la \emcap{personalidad de Dios} con la doctrina trinitaria. Ella contrastó la doctrina trinitaria con los dos primeros puntos de los \emcap{Principios Fundamentales}, que fueron los resultados de nuestros pioneros estudiando la Palabra de Dios. Refiriéndose a los pioneros y a los \emcap{Principios Fundamentales}, dijo: \egwinline{\textbf{Las \underline{teorías de parches} no pueden ser aceptadas por aquellos que son \underline{leales a la fe y a los principios} que han resistido toda la oposición de las influencias satánicas.}}[Lt253-1903.28; 1903][https://egwwritings.org/read?panels=p14068.9980036]


The conclusion is straightforward and simple. Those who are loyal to the faith, and to the principles received in the beginning of the work, cannot accept patchwork theories. Put into context, the patchwork theory, which is the Trinity doctrine, cannot be accepted by those who are holding fast \egwinline{\textbf{to \underline{the fundamental principles} that are \underline{based upon unquestionable authority}}}[SpTB02 59.1; 1904][https://egwwritings.org/read?panels=p417.299]. This conclusion leads us back to our first proposed test of the foundation of our faith.


La conclusión es directa y sencilla. Los que son fieles a la fe, y a los principios recibidos en el inicio de la obra, no pueden aceptar teorías de parches. En su contexto, la teoría de parche, que es la doctrina trinitaria, no puede ser aceptada por aquellos que se aferran \egwinline{\textbf{a \underline{los principios fundamentales} que están \underline{basados en una autoridad incuestionable}}}[SpTB02 59.1; 1904][https://egwwritings.org/read?panels=p417.299]. Esta conclusión nos lleva de nuevo a nuestra primera prueba propuesta del fundamento de nuestra fe.


% The Patchwork Theories

\begin{titledpoem}
    
    \stanza{
        Truth established through earnest prayer, \\
        Points of faith discovered with care. \\
        Principles tested by time and trial, \\
        Stand firm against Satan's denial.
    }

    \stanza{
        Patchwork theories seek to sway, \\
        Those from the ancient, proven way. \\
        No revisions of truth we'll accept, \\
        The faithful path must be kept.
    }

    \stanza{
        The personality of God, a sacred revelation, \\
        Not subject to human innovation. \\
        Loyal hearts stand on ground that's sure, \\
        Where foundations eternal endure.
    }
    
\end{titledpoem}


Las teorías de parches - Lt253-1903
