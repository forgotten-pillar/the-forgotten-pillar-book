\qrchapter{https://forgottenpillar.com/rsc/en-fp-chapter20}{Dr. Kellogg and Ellen White writings}


\qrchapter{https://forgottenpillar.com/rsc/es-fp-chapter20}{Dr. Kellogg y los escritos de Ellen White}


Dr. Kellogg asserted that in the Living Temple he represented the same sentiments advocated by Sister White. Likewise, today many claim that Sister White was trinitarian and was responsible for the church's acceptance of the Trinity doctrine\footnote{William Johnsson, Adventist Review, January 6th, 1994, ‘\textit{Present Truth –Walking in God’s Light}’}. Sister White, herself, declared such claims to be false.


El Dr. Kellogg afirmó que en el Living Temple representaba los mismos sentimientos defendidos por la hermana White. Asimismo, hoy en día muchos afirman que la hermana White era trinitaria y fue responsable de la aceptación por parte de la iglesia de la doctrina trinitaria\footnote{William Johnsson, Adventist Review, 6 de enero de 1994, ‘\textit{Present Truth –Walking in God's Light}’}. La propia hermana White declaró que tales afirmaciones son falsas.


\egw{\textbf{The enemy is seeking to \underline{bring in} among the people of God spiritualistic theories, which \underline{if accepted, would undermine the foundation of the faith} that has made us what we are}. He leads men to present fables clothed with Scripture. \textbf{There are those who assert that Sister White’s writings are in harmony with these teachings}.\textbf{ \underline{I declare this to be false}. Men may misapply Scripture; they may misinterpret my words; but God understands their devising}. How thankful I am for this! When the enemy comes in like a flood, \textbf{the Spirit of the Lord will lift up a standard for us against him}.}[Ms137-1903.21; 1903][https://egwwritings.org/read?panels=p9939.30]


\egw{\textbf{El enemigo está tratando de \underline{introducir} entre el pueblo de Dios teorías espiritualistas, que \underline{si son aceptadas, socavarían el fundamento de la fe} que nos ha hecho lo que somos}. Lleva a los hombres a presentar fábulas revestidas de las Escrituras. \textbf{Hay quienes afirman que los escritos de la hermana White están en armonía con estas enseñanzas}.\textbf{ \underline{Yo declaro que esto es falso}. Los hombres pueden aplicar mal la Escritura; pueden malinterpretar mis palabras; pero Dios entiende sus designios}. ¡Qué agradecida estoy por esto! Cuando el enemigo venga como una avalancha, \textbf{el Espíritu del Señor nos levantará un estandarte contra él}.}[Ms137-1903.21; 1903][https://egwwritings.org/read?panels=p9939.30]


Dr. Kellogg advocated the theories that, if accepted, would undermine the foundation of our faith. It is crucial to correctly understand what constitutes the foundation of our faith, which Sister White referred to. We have seen that it refers to the \emcap{Fundamental Principles}. Looking at her writings, and the writings of our pioneers, we see that the Trinity doctrine contradicts the \emcap{personality of God} and the truth about God’s presence. Today, with the Trinity doctrine as part of our belief, we recognize that we have moved away from the \emcap{Fundamental Principles} and formed another foundation. Sister White was not responsible for this transition. It is purely a misinterpretation of her works. Her writings do not undermine the foundation of the faith that has made us what we are. Her later work is completely in harmony with the truth given in the beginning.


El Dr. Kellogg defendió las teorías que, de ser aceptadas, socavarían el fundamento de nuestra fe. Es crucial entender correctamente lo que constituye el fundamento de nuestra fe, al que se refirió la hermana White. Hemos visto que se refiere a los \emcap{Principios Fundamentales}. Al examinar sus escritos, y los escritos de nuestros pioneros, vemos que la doctrina trinitaria contradice la \emcap{personalidad de Dios} y la verdad sobre la presencia de Dios. Hoy, con la doctrina trinitaria como parte de nuestra creencia, reconocemos que nos hemos alejado de los \emcap{Principios Fundamentales} y hemos formado otro fundamento. La hermana White no fue responsable de esta transición. Es puramente una mala interpretación de sus obras. Sus escritos no socavan el fundamento de la fe que nos ha convertido en lo que somos. Su obra posterior está completamente en armonía con la verdad dada en el principio.


\egw{\textbf{The past fifty years have not dimmed one jot or principle of our faith as we received the great and wonderful evidences that were made certain to us in 1844, after the passing of the time.} ... \textbf{\underline{Not a word is changed or denied}. That which the Holy Spirit testified to as truth after the passing of the time, in our great disappointment, \underline{is the solid foundation of truth}. \underline{Pillars of truth were revealed}, and we accepted \underline{the foundation principles} that have made us what we are—Seventh-day Adventists, keeping the commandments of God and having the faith of Jesus.}}[Lt326-1905.3; 1905][https://egwwritings.org/read?panels=p7678.9]


\egw{\textbf{Los últimos cincuenta años no han atenuado ni una pizca ni un principio de nuestra fe al recibir las grandes y maravillosas evidencias que se nos hicieron ciertas en 1844, después del paso del tiempo.} ... \textbf{\underline{Ni una sola palabra ha sido cambiada o negada}. Lo que el Espíritu Santo testificó como verdad después del paso del tiempo, en nuestro gran chasco, \underline{es el sólido fundamento de la verdad}. \underline{Se revelaron los pilares de la verdad}, y aceptamos \underline{los principios fundacionales} que nos han convertido en lo que somos—adventistas del séptimo día, guardando los mandamientos de Dios y teniendo la fe de Jesús.}}[Lt326-1905.3; 1905][https://egwwritings.org/read?panels=p7678.9]


\section*{Misrepresentation of the church standpoint}


\section*{Tergiversación del punto de vista de la iglesia}


By misrepresenting Sister White’s writings, Dr. Kellogg did not only misrepresent her work, but also the church’s official standpoint expressed in the \emcap{Fundamental Principles}. Ellen White rebuked Kellogg for misrepresenting the church’s standpoint. As we read this rebuke, let us keep in mind the church’s current standpoint on the \emcap{personality of God} as it compares to the first point of the \emcap{Fundamental Principles}.


Al tergiversar los escritos de la hermana White, el Dr. Kellogg no sólo tergiversó su obra, sino también el punto de vista oficial de la iglesia expresado en los \emcap{Principios Fundamentales}. Ellen White reprendió a Kellogg por tergiversar el punto de vista de la iglesia. Al leer esta reprimenda, tengamos en cuenta el punto de vista actual de la iglesia sobre la \emcap{personalidad de Dios} en comparación con el primer punto de los \emcap{Principios Fundamentales}.


\egw{You \textbf{are not sound in the truth}. Your statements made to believers and unbelievers \textbf{misrepresent us as a people who have not changed the truth for error}. They detract from the influence \textbf{God would have us possess before the world in revealing in plain, unmistakable language that we are \underline{true to the principles of our faith} and that we hold the beginning of our confidence firm unto the end}. We are strictly denominational. \textbf{We believe in 1903 the same truths we did believe when we established the Sanitarium and the College in Battle Creek, and \underline{we know that we had no ifs or ands about this matter}}.}[Lt300-1903.4; 1903][https://egwwritings.org/read?panels=p7705.10]


\egw{Usted \textbf{no está sano en la verdad}. Sus declaraciones hechas a los creyentes y a los incrédulos \textbf{nos desprestigian como un pueblo que no ha cambiado la verdad por el error}. Restan la influencia \textbf{que Dios quiere que tengamos ante el mundo al revelar en lenguaje claro e inequívoco que somos \underline{fieles a los principios de nuestra fe} y que mantenemos firme el principio de nuestra confianza hasta el final}. Somos estrictamente denominacionales. \textbf{Creemos en 1903 las mismas verdades que creíamos cuando establecimos el Sanatorio y el Colegio en Battle Creek, y \underline{sabemos que no tenemos peros en este asunto}}.}[Lt300-1903.4; 1903][https://egwwritings.org/read?panels=p7705.10]


\egwnogap{While you have told the things that you have and made the statements you have before unbelievers, my heart has been sad indeed. \textbf{You have evidenced that you have departed from the faith}. The very statements you have made before worldly men of influence, as the papers have reported your words, have been presented to me distinctly from your lips as you have spoken them. We cannot labor to give you influence as one whom we can trust with the sacred work connected with our institutions, for you need first to be converted and led.}[Lt300-1903.5; 1903][https://egwwritings.org/read?panels=p7705.11]


\egwnogap{Mientras usted ha dicho las cosas que ha dicho y ha hecho las declaraciones que ha hecho ante los incrédulos, mi corazón ha estado realmente triste. \textbf{Habéis evidenciado que os habéis apartado de la fe}. Las mismas declaraciones que has hecho ante hombres influyentes del mundo, tal como los periódicos han reportado tus palabras, se me han presentado claramente de tus labios tal como las has dicho. No podemos trabajar para darle influencia como alguien a quien podemos confiar con el trabajo sagrado relacionado con nuestras instituciones, pues primero hay que convertirse y ser conducido.}[Lt300-1903.5; 1903][https://egwwritings.org/read?panels=p7705.11]


\egwnogap{You are not sound in the faith. I have stated this in my diary months ago. \textbf{You have certainly placed the people of God, whom the Lord has led step by step in the ways of truth and placed upon \underline{a solid foundation}, in a false showing before unbelievers. Some have departed from the faith and \underline{will continue to misrepresent the work God has given me}}.}[Lt300-1903.6; 1903][https://egwwritings.org/read?panels=p7705.12]


\egwnogap{Usted no es sano en la fe. Lo he declarado en mi diario hace meses. \textbf{Ciertamente habéis colocado al pueblo de Dios, al que el Señor ha guiado paso a paso por los caminos de la verdad y ha colocado sobre \underline{un sólido fundamento}, en una falsa muestra ante los incrédulos. Algunos se han apartado de la fe y \underline{seguirán tergiversando la obra que Dios me ha dado}}.}[Lt300-1903.6; 1903][https://egwwritings.org/read?panels=p7705.12]


\egwnogap{\textbf{The sanctuary question is a clear and definite doctrine as we have held it as a people. \underline{You are not definitely clear on the personality of God, which is everything to us as a people}. \underline{You have virtually destroyed the Lord God Himself}}.}[Lt300-1903.7; 1903][https://egwwritings.org/read?panels=p7705.13]


\egwnogap{\textbf{La cuestión del santuario es una doctrina clara y definida tal como la hemos sostenido como pueblo. \underline{No estás definitivamente claro en cuanto a la personalidad de Dios, que lo es todo para nosotros como pueblo}. \underline{Has destruido virtualmente al Señor Dios mismo}}.}[Lt300-1903.7; 1903][https://egwwritings.org/read?panels=p7705.13]


\egwnogap{Why should you take the liberty to make the statements which you have made, as though you had authority for thus stating, when they are falsehoods? \textbf{You have made the facts of our faith of none effect before unbelievers,} \textbf{and the truth which should ever be kept prominent and exalted with this people you have virtually denied and ignored in your many statements. How dared you to do this?} \textbf{It necessitates us now to present our true position which constitutes us Seventh-day Adventists}. Whatever influence God has given you in the past has been in mercy to you, letting the light shine upon you.}[Lt300-1903.8; 1903][https://egwwritings.org/read?panels=p7705.14]


\egwnogap{¿Por qué te tomaste la libertad de hacer las afirmaciones que has hecho, como si tuvieras autoridad para ello, cuando son falsedades? \textbf{Has hecho que los hechos de nuestra fe no tengan ningún efecto ante los incrédulos,} \textbf{y la verdad que debería mantenerse siempre prominente y exaltada con este pueblo la has negado e ignorado virtualmente en tus numerosas declaraciones. ¿Cómo te atreviste a hacer esto?} \textbf{Es necesario que ahora presentemos nuestra verdadera posición que nos constituye como adventistas del séptimo día}. Cualquier influencia que Dios te haya dado en el pasado ha sido por misericordia hacia ti, dejando que la luz brille sobre ti.}[Lt300-1903.8; 1903][https://egwwritings.org/read?panels=p7705.14]


\egwnogap{\textbf{We cannot for a moment have any misrepresentation upon these solemn and important subjects of truth which have been the faith of our people since 1844. This means much to us.} The Lord would have me say to you that the enemy has, through his specious deceptions, placed his unbelief in your mind, and you have been working it out. \textbf{\underline{All who receive your presentations will enter upon strange paths if they connect with you}}. \textbf{You are \underline{bringing in} strange, common fire}, \textbf{but not the fire of God’s own kindling}; and now \textbf{I must speak plainly to our people that the Lord has led us step by step and shown us clear light upon the heavenly sanctuary in the most holy of holies where \underline{God revealed Himself} to His appointed ones.}}[Lt300-1903.9; 1903][https://egwwritings.org/read?panels=p7705.15]


\egwnogap{\textbf{No podemos tener ni por un momento ninguna tergiversación sobre estos solemnes e importantes temas de la verdad que han sido la fe de nuestro pueblo desde 1844. Esto significa mucho para nosotros.} El Señor quiere que te diga que el enemigo, por medio de sus engaños especiosos, ha colocado su incredulidad en tu mente, y has estado trabajando en ella. \textbf{\underline{Todos los que reciban tus presentaciones entrarán en caminos extraños si se conectan contigo}}. \textbf{Estás \underline{trayendo} fuego extraño y común}, \textbf{pero no el fuego del propio encendido de Dios}; y ahora \textbf{debo hablar claramente a nuestro pueblo de que el Señor nos ha guiado paso a paso y nos ha mostrado una luz clara sobre el santuario celestial en el santísimo de los santos donde \underline{Dios se reveló a Sí mismo} a sus designados.}}[Lt300-1903.9; 1903][https://egwwritings.org/read?panels=p7705.15]


Dr. Kellogg misrepresented the truth that constituted the foundation of our faith; most specifically, he misrepresented the truth on the \emcap{personality of God}, which was everything to us as people. If in 1903, it necessitated \egwinline{\textbf{to present our true position which constitutes us Seventh-day Adventists}}, how much more important is it for us today? Sister White did her part in upholding the foundation of our faith in the beginning, but it seems like we have forgotten.


El Dr. Kellogg tergiversó la verdad que constituía el fundamento de nuestra fe; más específicamente, tergiversó la verdad sobre la \emcap{personalidad de Dios}, que lo era todo para nosotros como pueblo. Si en 1903, era necesario \egwinline{\textbf{presentar nuestra verdadera posición que nos constituye como adventistas del séptimo día}}, ¿cuánto más importante es para nosotros hoy? La hermana White hizo su parte al sostener el fundamento de nuestra fe en el principio, pero parece que lo hemos olvidado.


% Dr. Kellogg and Ellen White writings

\begin{titledpoem}
    
    \stanza{
        In faith’s foundation, once so clear, \\
        Dear Ellen’s words, we should revere. \\
        J.H. agreed, yet in deceit, \\
        And twisted truth, in his conceit.
    }

    \stanza{
        The enemy—dear Ellen warned— \\
        Will twist beliefs, till they are scorned. \\
        These dangerous theories, wrongly dressed, \\
        In Scripture’s garb, the false impressed.
    }

    \stanza{
        "False!" she declared, against the tide, \\
        His crafty statements were denied. \\
        Defense was strong, her vision, broad \\
        The Fundamental were from God.
    }

    \stanza{    
        No word was changed, not peg nor pin, \\
        From when the pillars did begin. \\
        Now who would dare to move a board— \\
        This platform built up by the Lord.
    }

    \stanza{
        For Kellogg’s stance did not agree, \\
        Foundation was not trinity. \\
        God’s personality is true; \\
        Confusion came through Kellogg’s view.
    }

    \stanza{   
        Yet, Ellen stood, unyielding, firm, \\
        The early truth she did confirm. \\
        Against the tide of Trinity \\
        She held the truth of deity.
    }

    \stanza{    
        Today, as then, let’s hold the line, \\
        The early pioneer faith, divine. \\
        Truth’s legacy, let’s rightly claim, \\
        Unchanging, solid, still the same.
    }

    \stanza{    
        The way God led, let’s not forget, \\
        These principles are firmly set. \\
        Against the changing winds of doubt, \\
        Her writings guide, within, without.
    }
    
\end{titledpoem}


% \qrchapter{https://forgottenpillar.com/rsc/en-fp-chapter20}{Dr. Kellogg and Ellen White writings}


\qrchapter{https://forgottenpillar.com/rsc/es-fp-chapter20}{Dr. Kellogg y los escritos de Ellen White}


Dr. Kellogg asserted that in the Living Temple he represented the same sentiments advocated by Sister White. Likewise, today many claim that Sister White was trinitarian and was responsible for the church's acceptance of the Trinity doctrine\footnote{William Johnsson, Adventist Review, January 6th, 1994, ‘\textit{Present Truth –Walking in God’s Light}’}. Sister White, herself, declared such claims to be false.


El Dr. Kellogg afirmó que en el Living Temple representaba los mismos sentimientos defendidos por la hermana White. Asimismo, hoy en día muchos afirman que la hermana White era trinitaria y fue responsable de la aceptación por parte de la iglesia de la doctrina trinitaria\footnote{William Johnsson, Adventist Review, 6 de enero de 1994, ‘\textit{Present Truth –Walking in God's Light}’}. La propia hermana White declaró que tales afirmaciones son falsas.


\egw{\textbf{The enemy is seeking to \underline{bring in} among the people of God spiritualistic theories, which \underline{if accepted, would undermine the foundation of the faith} that has made us what we are}. He leads men to present fables clothed with Scripture. \textbf{There are those who assert that Sister White’s writings are in harmony with these teachings}.\textbf{ \underline{I declare this to be false}. Men may misapply Scripture; they may misinterpret my words; but God understands their devising}. How thankful I am for this! When the enemy comes in like a flood, \textbf{the Spirit of the Lord will lift up a standard for us against him}.}[Ms137-1903.21; 1903][https://egwwritings.org/read?panels=p9939.30]


\egw{\textbf{El enemigo está tratando de \underline{introducir} entre el pueblo de Dios teorías espiritualistas, que \underline{si son aceptadas, socavarían el fundamento de la fe} que nos ha hecho lo que somos}. Lleva a los hombres a presentar fábulas revestidas de las Escrituras. \textbf{Hay quienes afirman que los escritos de la hermana White están en armonía con estas enseñanzas}.\textbf{ \underline{Yo declaro que esto es falso}. Los hombres pueden aplicar mal la Escritura; pueden malinterpretar mis palabras; pero Dios entiende sus designios}. ¡Qué agradecida estoy por esto! Cuando el enemigo venga como una avalancha, \textbf{el Espíritu del Señor nos levantará un estandarte contra él}.}[Ms137-1903.21; 1903][https://egwwritings.org/read?panels=p9939.30]


Dr. Kellogg advocated the theories that, if accepted, would undermine the foundation of our faith. It is crucial to correctly understand what constitutes the foundation of our faith, which Sister White referred to. We have seen that it refers to the \emcap{Fundamental Principles}. Looking at her writings, and the writings of our pioneers, we see that the Trinity doctrine contradicts the \emcap{personality of God} and the truth about God’s presence. Today, with the Trinity doctrine as part of our belief, we recognize that we have moved away from the \emcap{Fundamental Principles} and formed another foundation. Sister White was not responsible for this transition. It is purely a misinterpretation of her works. Her writings do not undermine the foundation of the faith that has made us what we are. Her later work is completely in harmony with the truth given in the beginning.


El Dr. Kellogg defendió las teorías que, de ser aceptadas, socavarían el fundamento de nuestra fe. Es crucial entender correctamente lo que constituye el fundamento de nuestra fe, al que se refirió la hermana White. Hemos visto que se refiere a los \emcap{Principios Fundamentales}. Al examinar sus escritos, y los escritos de nuestros pioneros, vemos que la doctrina trinitaria contradice la \emcap{personalidad de Dios} y la verdad sobre la presencia de Dios. Hoy, con la doctrina trinitaria como parte de nuestra creencia, reconocemos que nos hemos alejado de los \emcap{Principios Fundamentales} y hemos formado otro fundamento. La hermana White no fue responsable de esta transición. Es puramente una mala interpretación de sus obras. Sus escritos no socavan el fundamento de la fe que nos ha convertido en lo que somos. Su obra posterior está completamente en armonía con la verdad dada en el principio.


\egw{\textbf{The past fifty years have not dimmed one jot or principle of our faith as we received the great and wonderful evidences that were made certain to us in 1844, after the passing of the time.} ... \textbf{\underline{Not a word is changed or denied}. That which the Holy Spirit testified to as truth after the passing of the time, in our great disappointment, \underline{is the solid foundation of truth}. \underline{Pillars of truth were revealed}, and we accepted \underline{the foundation principles} that have made us what we are—Seventh-day Adventists, keeping the commandments of God and having the faith of Jesus.}}[Lt326-1905.3; 1905][https://egwwritings.org/read?panels=p7678.9]


\egw{\textbf{Los últimos cincuenta años no han atenuado ni una pizca ni un principio de nuestra fe al recibir las grandes y maravillosas evidencias que se nos hicieron ciertas en 1844, después del paso del tiempo.} ... \textbf{\underline{Ni una sola palabra ha sido cambiada o negada}. Lo que el Espíritu Santo testificó como verdad después del paso del tiempo, en nuestro gran chasco, \underline{es el sólido fundamento de la verdad}. \underline{Se revelaron los pilares de la verdad}, y aceptamos \underline{los principios fundacionales} que nos han convertido en lo que somos—adventistas del séptimo día, guardando los mandamientos de Dios y teniendo la fe de Jesús.}}[Lt326-1905.3; 1905][https://egwwritings.org/read?panels=p7678.9]


\section*{Misrepresentation of the church standpoint}


\section*{Tergiversación del punto de vista de la iglesia}


By misrepresenting Sister White’s writings, Dr. Kellogg did not only misrepresent her work, but also the church’s official standpoint expressed in the \emcap{Fundamental Principles}. Ellen White rebuked Kellogg for misrepresenting the church’s standpoint. As we read this rebuke, let us keep in mind the church’s current standpoint on the \emcap{personality of God} as it compares to the first point of the \emcap{Fundamental Principles}.


Al tergiversar los escritos de la hermana White, el Dr. Kellogg no sólo tergiversó su obra, sino también el punto de vista oficial de la iglesia expresado en los \emcap{Principios Fundamentales}. Ellen White reprendió a Kellogg por tergiversar el punto de vista de la iglesia. Al leer esta reprimenda, tengamos en cuenta el punto de vista actual de la iglesia sobre la \emcap{personalidad de Dios} en comparación con el primer punto de los \emcap{Principios Fundamentales}.


\egw{You \textbf{are not sound in the truth}. Your statements made to believers and unbelievers \textbf{misrepresent us as a people who have not changed the truth for error}. They detract from the influence \textbf{God would have us possess before the world in revealing in plain, unmistakable language that we are \underline{true to the principles of our faith} and that we hold the beginning of our confidence firm unto the end}. We are strictly denominational. \textbf{We believe in 1903 the same truths we did believe when we established the Sanitarium and the College in Battle Creek, and \underline{we know that we had no ifs or ands about this matter}}.}[Lt300-1903.4; 1903][https://egwwritings.org/read?panels=p7705.10]


\egw{Usted \textbf{no está sano en la verdad}. Sus declaraciones hechas a los creyentes y a los incrédulos \textbf{nos desprestigian como un pueblo que no ha cambiado la verdad por el error}. Restan la influencia \textbf{que Dios quiere que tengamos ante el mundo al revelar en lenguaje claro e inequívoco que somos \underline{fieles a los principios de nuestra fe} y que mantenemos firme el principio de nuestra confianza hasta el final}. Somos estrictamente denominacionales. \textbf{Creemos en 1903 las mismas verdades que creíamos cuando establecimos el Sanatorio y el Colegio en Battle Creek, y \underline{sabemos que no tenemos peros en este asunto}}.}[Lt300-1903.4; 1903][https://egwwritings.org/read?panels=p7705.10]


\egwnogap{While you have told the things that you have and made the statements you have before unbelievers, my heart has been sad indeed. \textbf{You have evidenced that you have departed from the faith}. The very statements you have made before worldly men of influence, as the papers have reported your words, have been presented to me distinctly from your lips as you have spoken them. We cannot labor to give you influence as one whom we can trust with the sacred work connected with our institutions, for you need first to be converted and led.}[Lt300-1903.5; 1903][https://egwwritings.org/read?panels=p7705.11]


\egwnogap{Mientras usted ha dicho las cosas que ha dicho y ha hecho las declaraciones que ha hecho ante los incrédulos, mi corazón ha estado realmente triste. \textbf{Habéis evidenciado que os habéis apartado de la fe}. Las mismas declaraciones que has hecho ante hombres influyentes del mundo, tal como los periódicos han reportado tus palabras, se me han presentado claramente de tus labios tal como las has dicho. No podemos trabajar para darle influencia como alguien a quien podemos confiar con el trabajo sagrado relacionado con nuestras instituciones, pues primero hay que convertirse y ser conducido.}[Lt300-1903.5; 1903][https://egwwritings.org/read?panels=p7705.11]


\egwnogap{You are not sound in the faith. I have stated this in my diary months ago. \textbf{You have certainly placed the people of God, whom the Lord has led step by step in the ways of truth and placed upon \underline{a solid foundation}, in a false showing before unbelievers. Some have departed from the faith and \underline{will continue to misrepresent the work God has given me}}.}[Lt300-1903.6; 1903][https://egwwritings.org/read?panels=p7705.12]


\egwnogap{Usted no es sano en la fe. Lo he declarado en mi diario hace meses. \textbf{Ciertamente habéis colocado al pueblo de Dios, al que el Señor ha guiado paso a paso por los caminos de la verdad y ha colocado sobre \underline{un sólido fundamento}, en una falsa muestra ante los incrédulos. Algunos se han apartado de la fe y \underline{seguirán tergiversando la obra que Dios me ha dado}}.}[Lt300-1903.6; 1903][https://egwwritings.org/read?panels=p7705.12]


\egwnogap{\textbf{The sanctuary question is a clear and definite doctrine as we have held it as a people. \underline{You are not definitely clear on the personality of God, which is everything to us as a people}. \underline{You have virtually destroyed the Lord God Himself}}.}[Lt300-1903.7; 1903][https://egwwritings.org/read?panels=p7705.13]


\egwnogap{\textbf{La cuestión del santuario es una doctrina clara y definida tal como la hemos sostenido como pueblo. \underline{No estás definitivamente claro en cuanto a la personalidad de Dios, que lo es todo para nosotros como pueblo}. \underline{Has destruido virtualmente al Señor Dios mismo}}.}[Lt300-1903.7; 1903][https://egwwritings.org/read?panels=p7705.13]


\egwnogap{Why should you take the liberty to make the statements which you have made, as though you had authority for thus stating, when they are falsehoods? \textbf{You have made the facts of our faith of none effect before unbelievers,} \textbf{and the truth which should ever be kept prominent and exalted with this people you have virtually denied and ignored in your many statements. How dared you to do this?} \textbf{It necessitates us now to present our true position which constitutes us Seventh-day Adventists}. Whatever influence God has given you in the past has been in mercy to you, letting the light shine upon you.}[Lt300-1903.8; 1903][https://egwwritings.org/read?panels=p7705.14]


\egwnogap{¿Por qué te tomaste la libertad de hacer las afirmaciones que has hecho, como si tuvieras autoridad para ello, cuando son falsedades? \textbf{Has hecho que los hechos de nuestra fe no tengan ningún efecto ante los incrédulos,} \textbf{y la verdad que debería mantenerse siempre prominente y exaltada con este pueblo la has negado e ignorado virtualmente en tus numerosas declaraciones. ¿Cómo te atreviste a hacer esto?} \textbf{Es necesario que ahora presentemos nuestra verdadera posición que nos constituye como adventistas del séptimo día}. Cualquier influencia que Dios te haya dado en el pasado ha sido por misericordia hacia ti, dejando que la luz brille sobre ti.}[Lt300-1903.8; 1903][https://egwwritings.org/read?panels=p7705.14]


\egwnogap{\textbf{We cannot for a moment have any misrepresentation upon these solemn and important subjects of truth which have been the faith of our people since 1844. This means much to us.} The Lord would have me say to you that the enemy has, through his specious deceptions, placed his unbelief in your mind, and you have been working it out. \textbf{\underline{All who receive your presentations will enter upon strange paths if they connect with you}}. \textbf{You are \underline{bringing in} strange, common fire}, \textbf{but not the fire of God’s own kindling}; and now \textbf{I must speak plainly to our people that the Lord has led us step by step and shown us clear light upon the heavenly sanctuary in the most holy of holies where \underline{God revealed Himself} to His appointed ones.}}[Lt300-1903.9; 1903][https://egwwritings.org/read?panels=p7705.15]


\egwnogap{\textbf{No podemos tener ni por un momento ninguna tergiversación sobre estos solemnes e importantes temas de la verdad que han sido la fe de nuestro pueblo desde 1844. Esto significa mucho para nosotros.} El Señor quiere que te diga que el enemigo, por medio de sus engaños especiosos, ha colocado su incredulidad en tu mente, y has estado trabajando en ella. \textbf{\underline{Todos los que reciban tus presentaciones entrarán en caminos extraños si se conectan contigo}}. \textbf{Estás \underline{trayendo} fuego extraño y común}, \textbf{pero no el fuego del propio encendido de Dios}; y ahora \textbf{debo hablar claramente a nuestro pueblo de que el Señor nos ha guiado paso a paso y nos ha mostrado una luz clara sobre el santuario celestial en el santísimo de los santos donde \underline{Dios se reveló a Sí mismo} a sus designados.}}[Lt300-1903.9; 1903][https://egwwritings.org/read?panels=p7705.15]


Dr. Kellogg misrepresented the truth that constituted the foundation of our faith; most specifically, he misrepresented the truth on the \emcap{personality of God}, which was everything to us as people. If in 1903, it necessitated \egwinline{\textbf{to present our true position which constitutes us Seventh-day Adventists}}, how much more important is it for us today? Sister White did her part in upholding the foundation of our faith in the beginning, but it seems like we have forgotten.


El Dr. Kellogg tergiversó la verdad que constituía el fundamento de nuestra fe; más específicamente, tergiversó la verdad sobre la \emcap{personalidad de Dios}, que lo era todo para nosotros como pueblo. Si en 1903, era necesario \egwinline{\textbf{presentar nuestra verdadera posición que nos constituye como adventistas del séptimo día}}, ¿cuánto más importante es para nosotros hoy? La hermana White hizo su parte al sostener el fundamento de nuestra fe en el principio, pero parece que lo hemos olvidado.


% Dr. Kellogg and Ellen White writings

\begin{titledpoem}
    
    \stanza{
        In faith’s foundation, once so clear, \\
        Dear Ellen’s words, we should revere. \\
        J.H. agreed, yet in deceit, \\
        And twisted truth, in his conceit.
    }

    \stanza{
        The enemy—dear Ellen warned— \\
        Will twist beliefs, till they are scorned. \\
        These dangerous theories, wrongly dressed, \\
        In Scripture’s garb, the false impressed.
    }

    \stanza{
        "False!" she declared, against the tide, \\
        His crafty statements were denied. \\
        Defense was strong, her vision, broad \\
        The Fundamental were from God.
    }

    \stanza{    
        No word was changed, not peg nor pin, \\
        From when the pillars did begin. \\
        Now who would dare to move a board— \\
        This platform built up by the Lord.
    }

    \stanza{
        For Kellogg’s stance did not agree, \\
        Foundation was not trinity. \\
        God’s personality is true; \\
        Confusion came through Kellogg’s view.
    }

    \stanza{   
        Yet, Ellen stood, unyielding, firm, \\
        The early truth she did confirm. \\
        Against the tide of Trinity \\
        She held the truth of deity.
    }

    \stanza{    
        Today, as then, let’s hold the line, \\
        The early pioneer faith, divine. \\
        Truth’s legacy, let’s rightly claim, \\
        Unchanging, solid, still the same.
    }

    \stanza{    
        The way God led, let’s not forget, \\
        These principles are firmly set. \\
        Against the changing winds of doubt, \\
        Her writings guide, within, without.
    }
    
\end{titledpoem}


% \qrchapter{https://forgottenpillar.com/rsc/en-fp-chapter20}{Dr. Kellogg and Ellen White writings}


\qrchapter{https://forgottenpillar.com/rsc/es-fp-chapter20}{Dr. Kellogg y los escritos de Ellen White}


Dr. Kellogg asserted that in the Living Temple he represented the same sentiments advocated by Sister White. Likewise, today many claim that Sister White was trinitarian and was responsible for the church's acceptance of the Trinity doctrine\footnote{William Johnsson, Adventist Review, January 6th, 1994, ‘\textit{Present Truth –Walking in God’s Light}’}. Sister White, herself, declared such claims to be false.


El Dr. Kellogg afirmó que en el Living Temple representaba los mismos sentimientos defendidos por la hermana White. Asimismo, hoy en día muchos afirman que la hermana White era trinitaria y fue responsable de la aceptación por parte de la iglesia de la doctrina trinitaria\footnote{William Johnsson, Adventist Review, 6 de enero de 1994, ‘\textit{Present Truth –Walking in God's Light}’}. La propia hermana White declaró que tales afirmaciones son falsas.


\egw{\textbf{The enemy is seeking to \underline{bring in} among the people of God spiritualistic theories, which \underline{if accepted, would undermine the foundation of the faith} that has made us what we are}. He leads men to present fables clothed with Scripture. \textbf{There are those who assert that Sister White’s writings are in harmony with these teachings}.\textbf{ \underline{I declare this to be false}. Men may misapply Scripture; they may misinterpret my words; but God understands their devising}. How thankful I am for this! When the enemy comes in like a flood, \textbf{the Spirit of the Lord will lift up a standard for us against him}.}[Ms137-1903.21; 1903][https://egwwritings.org/read?panels=p9939.30]


\egw{\textbf{El enemigo está tratando de \underline{introducir} entre el pueblo de Dios teorías espiritualistas, que \underline{si son aceptadas, socavarían el fundamento de la fe} que nos ha hecho lo que somos}. Lleva a los hombres a presentar fábulas revestidas de las Escrituras. \textbf{Hay quienes afirman que los escritos de la hermana White están en armonía con estas enseñanzas}.\textbf{ \underline{Yo declaro que esto es falso}. Los hombres pueden aplicar mal la Escritura; pueden malinterpretar mis palabras; pero Dios entiende sus designios}. ¡Qué agradecida estoy por esto! Cuando el enemigo venga como una avalancha, \textbf{el Espíritu del Señor nos levantará un estandarte contra él}.}[Ms137-1903.21; 1903][https://egwwritings.org/read?panels=p9939.30]


Dr. Kellogg advocated the theories that, if accepted, would undermine the foundation of our faith. It is crucial to correctly understand what constitutes the foundation of our faith, which Sister White referred to. We have seen that it refers to the \emcap{Fundamental Principles}. Looking at her writings, and the writings of our pioneers, we see that the Trinity doctrine contradicts the \emcap{personality of God} and the truth about God’s presence. Today, with the Trinity doctrine as part of our belief, we recognize that we have moved away from the \emcap{Fundamental Principles} and formed another foundation. Sister White was not responsible for this transition. It is purely a misinterpretation of her works. Her writings do not undermine the foundation of the faith that has made us what we are. Her later work is completely in harmony with the truth given in the beginning.


El Dr. Kellogg defendió las teorías que, de ser aceptadas, socavarían el fundamento de nuestra fe. Es crucial entender correctamente lo que constituye el fundamento de nuestra fe, al que se refirió la hermana White. Hemos visto que se refiere a los \emcap{Principios Fundamentales}. Al examinar sus escritos, y los escritos de nuestros pioneros, vemos que la doctrina trinitaria contradice la \emcap{personalidad de Dios} y la verdad sobre la presencia de Dios. Hoy, con la doctrina trinitaria como parte de nuestra creencia, reconocemos que nos hemos alejado de los \emcap{Principios Fundamentales} y hemos formado otro fundamento. La hermana White no fue responsable de esta transición. Es puramente una mala interpretación de sus obras. Sus escritos no socavan el fundamento de la fe que nos ha convertido en lo que somos. Su obra posterior está completamente en armonía con la verdad dada en el principio.


\egw{\textbf{The past fifty years have not dimmed one jot or principle of our faith as we received the great and wonderful evidences that were made certain to us in 1844, after the passing of the time.} ... \textbf{\underline{Not a word is changed or denied}. That which the Holy Spirit testified to as truth after the passing of the time, in our great disappointment, \underline{is the solid foundation of truth}. \underline{Pillars of truth were revealed}, and we accepted \underline{the foundation principles} that have made us what we are—Seventh-day Adventists, keeping the commandments of God and having the faith of Jesus.}}[Lt326-1905.3; 1905][https://egwwritings.org/read?panels=p7678.9]


\egw{\textbf{Los últimos cincuenta años no han atenuado ni una pizca ni un principio de nuestra fe al recibir las grandes y maravillosas evidencias que se nos hicieron ciertas en 1844, después del paso del tiempo.} ... \textbf{\underline{Ni una sola palabra ha sido cambiada o negada}. Lo que el Espíritu Santo testificó como verdad después del paso del tiempo, en nuestro gran chasco, \underline{es el sólido fundamento de la verdad}. \underline{Se revelaron los pilares de la verdad}, y aceptamos \underline{los principios fundacionales} que nos han convertido en lo que somos—adventistas del séptimo día, guardando los mandamientos de Dios y teniendo la fe de Jesús.}}[Lt326-1905.3; 1905][https://egwwritings.org/read?panels=p7678.9]


\section*{Misrepresentation of the church standpoint}


\section*{Tergiversación del punto de vista de la iglesia}


By misrepresenting Sister White’s writings, Dr. Kellogg did not only misrepresent her work, but also the church’s official standpoint expressed in the \emcap{Fundamental Principles}. Ellen White rebuked Kellogg for misrepresenting the church’s standpoint. As we read this rebuke, let us keep in mind the church’s current standpoint on the \emcap{personality of God} as it compares to the first point of the \emcap{Fundamental Principles}.


Al tergiversar los escritos de la hermana White, el Dr. Kellogg no sólo tergiversó su obra, sino también el punto de vista oficial de la iglesia expresado en los \emcap{Principios Fundamentales}. Ellen White reprendió a Kellogg por tergiversar el punto de vista de la iglesia. Al leer esta reprimenda, tengamos en cuenta el punto de vista actual de la iglesia sobre la \emcap{personalidad de Dios} en comparación con el primer punto de los \emcap{Principios Fundamentales}.


\egw{You \textbf{are not sound in the truth}. Your statements made to believers and unbelievers \textbf{misrepresent us as a people who have not changed the truth for error}. They detract from the influence \textbf{God would have us possess before the world in revealing in plain, unmistakable language that we are \underline{true to the principles of our faith} and that we hold the beginning of our confidence firm unto the end}. We are strictly denominational. \textbf{We believe in 1903 the same truths we did believe when we established the Sanitarium and the College in Battle Creek, and \underline{we know that we had no ifs or ands about this matter}}.}[Lt300-1903.4; 1903][https://egwwritings.org/read?panels=p7705.10]


\egw{Usted \textbf{no está sano en la verdad}. Sus declaraciones hechas a los creyentes y a los incrédulos \textbf{nos desprestigian como un pueblo que no ha cambiado la verdad por el error}. Restan la influencia \textbf{que Dios quiere que tengamos ante el mundo al revelar en lenguaje claro e inequívoco que somos \underline{fieles a los principios de nuestra fe} y que mantenemos firme el principio de nuestra confianza hasta el final}. Somos estrictamente denominacionales. \textbf{Creemos en 1903 las mismas verdades que creíamos cuando establecimos el Sanatorio y el Colegio en Battle Creek, y \underline{sabemos que no tenemos peros en este asunto}}.}[Lt300-1903.4; 1903][https://egwwritings.org/read?panels=p7705.10]


\egwnogap{While you have told the things that you have and made the statements you have before unbelievers, my heart has been sad indeed. \textbf{You have evidenced that you have departed from the faith}. The very statements you have made before worldly men of influence, as the papers have reported your words, have been presented to me distinctly from your lips as you have spoken them. We cannot labor to give you influence as one whom we can trust with the sacred work connected with our institutions, for you need first to be converted and led.}[Lt300-1903.5; 1903][https://egwwritings.org/read?panels=p7705.11]


\egwnogap{Mientras usted ha dicho las cosas que ha dicho y ha hecho las declaraciones que ha hecho ante los incrédulos, mi corazón ha estado realmente triste. \textbf{Habéis evidenciado que os habéis apartado de la fe}. Las mismas declaraciones que has hecho ante hombres influyentes del mundo, tal como los periódicos han reportado tus palabras, se me han presentado claramente de tus labios tal como las has dicho. No podemos trabajar para darle influencia como alguien a quien podemos confiar con el trabajo sagrado relacionado con nuestras instituciones, pues primero hay que convertirse y ser conducido.}[Lt300-1903.5; 1903][https://egwwritings.org/read?panels=p7705.11]


\egwnogap{You are not sound in the faith. I have stated this in my diary months ago. \textbf{You have certainly placed the people of God, whom the Lord has led step by step in the ways of truth and placed upon \underline{a solid foundation}, in a false showing before unbelievers. Some have departed from the faith and \underline{will continue to misrepresent the work God has given me}}.}[Lt300-1903.6; 1903][https://egwwritings.org/read?panels=p7705.12]


\egwnogap{Usted no es sano en la fe. Lo he declarado en mi diario hace meses. \textbf{Ciertamente habéis colocado al pueblo de Dios, al que el Señor ha guiado paso a paso por los caminos de la verdad y ha colocado sobre \underline{un sólido fundamento}, en una falsa muestra ante los incrédulos. Algunos se han apartado de la fe y \underline{seguirán tergiversando la obra que Dios me ha dado}}.}[Lt300-1903.6; 1903][https://egwwritings.org/read?panels=p7705.12]


\egwnogap{\textbf{The sanctuary question is a clear and definite doctrine as we have held it as a people. \underline{You are not definitely clear on the personality of God, which is everything to us as a people}. \underline{You have virtually destroyed the Lord God Himself}}.}[Lt300-1903.7; 1903][https://egwwritings.org/read?panels=p7705.13]


\egwnogap{\textbf{La cuestión del santuario es una doctrina clara y definida tal como la hemos sostenido como pueblo. \underline{No estás definitivamente claro en cuanto a la personalidad de Dios, que lo es todo para nosotros como pueblo}. \underline{Has destruido virtualmente al Señor Dios mismo}}.}[Lt300-1903.7; 1903][https://egwwritings.org/read?panels=p7705.13]


\egwnogap{Why should you take the liberty to make the statements which you have made, as though you had authority for thus stating, when they are falsehoods? \textbf{You have made the facts of our faith of none effect before unbelievers,} \textbf{and the truth which should ever be kept prominent and exalted with this people you have virtually denied and ignored in your many statements. How dared you to do this?} \textbf{It necessitates us now to present our true position which constitutes us Seventh-day Adventists}. Whatever influence God has given you in the past has been in mercy to you, letting the light shine upon you.}[Lt300-1903.8; 1903][https://egwwritings.org/read?panels=p7705.14]


\egwnogap{¿Por qué te tomaste la libertad de hacer las afirmaciones que has hecho, como si tuvieras autoridad para ello, cuando son falsedades? \textbf{Has hecho que los hechos de nuestra fe no tengan ningún efecto ante los incrédulos,} \textbf{y la verdad que debería mantenerse siempre prominente y exaltada con este pueblo la has negado e ignorado virtualmente en tus numerosas declaraciones. ¿Cómo te atreviste a hacer esto?} \textbf{Es necesario que ahora presentemos nuestra verdadera posición que nos constituye como adventistas del séptimo día}. Cualquier influencia que Dios te haya dado en el pasado ha sido por misericordia hacia ti, dejando que la luz brille sobre ti.}[Lt300-1903.8; 1903][https://egwwritings.org/read?panels=p7705.14]


\egwnogap{\textbf{We cannot for a moment have any misrepresentation upon these solemn and important subjects of truth which have been the faith of our people since 1844. This means much to us.} The Lord would have me say to you that the enemy has, through his specious deceptions, placed his unbelief in your mind, and you have been working it out. \textbf{\underline{All who receive your presentations will enter upon strange paths if they connect with you}}. \textbf{You are \underline{bringing in} strange, common fire}, \textbf{but not the fire of God’s own kindling}; and now \textbf{I must speak plainly to our people that the Lord has led us step by step and shown us clear light upon the heavenly sanctuary in the most holy of holies where \underline{God revealed Himself} to His appointed ones.}}[Lt300-1903.9; 1903][https://egwwritings.org/read?panels=p7705.15]


\egwnogap{\textbf{No podemos tener ni por un momento ninguna tergiversación sobre estos solemnes e importantes temas de la verdad que han sido la fe de nuestro pueblo desde 1844. Esto significa mucho para nosotros.} El Señor quiere que te diga que el enemigo, por medio de sus engaños especiosos, ha colocado su incredulidad en tu mente, y has estado trabajando en ella. \textbf{\underline{Todos los que reciban tus presentaciones entrarán en caminos extraños si se conectan contigo}}. \textbf{Estás \underline{trayendo} fuego extraño y común}, \textbf{pero no el fuego del propio encendido de Dios}; y ahora \textbf{debo hablar claramente a nuestro pueblo de que el Señor nos ha guiado paso a paso y nos ha mostrado una luz clara sobre el santuario celestial en el santísimo de los santos donde \underline{Dios se reveló a Sí mismo} a sus designados.}}[Lt300-1903.9; 1903][https://egwwritings.org/read?panels=p7705.15]


Dr. Kellogg misrepresented the truth that constituted the foundation of our faith; most specifically, he misrepresented the truth on the \emcap{personality of God}, which was everything to us as people. If in 1903, it necessitated \egwinline{\textbf{to present our true position which constitutes us Seventh-day Adventists}}, how much more important is it for us today? Sister White did her part in upholding the foundation of our faith in the beginning, but it seems like we have forgotten.


El Dr. Kellogg tergiversó la verdad que constituía el fundamento de nuestra fe; más específicamente, tergiversó la verdad sobre la \emcap{personalidad de Dios}, que lo era todo para nosotros como pueblo. Si en 1903, era necesario \egwinline{\textbf{presentar nuestra verdadera posición que nos constituye como adventistas del séptimo día}}, ¿cuánto más importante es para nosotros hoy? La hermana White hizo su parte al sostener el fundamento de nuestra fe en el principio, pero parece que lo hemos olvidado.


% Dr. Kellogg and Ellen White writings

\begin{titledpoem}
    
    \stanza{
        In faith’s foundation, once so clear, \\
        Dear Ellen’s words, we should revere. \\
        J.H. agreed, yet in deceit, \\
        And twisted truth, in his conceit.
    }

    \stanza{
        The enemy—dear Ellen warned— \\
        Will twist beliefs, till they are scorned. \\
        These dangerous theories, wrongly dressed, \\
        In Scripture’s garb, the false impressed.
    }

    \stanza{
        "False!" she declared, against the tide, \\
        His crafty statements were denied. \\
        Defense was strong, her vision, broad \\
        The Fundamental were from God.
    }

    \stanza{    
        No word was changed, not peg nor pin, \\
        From when the pillars did begin. \\
        Now who would dare to move a board— \\
        This platform built up by the Lord.
    }

    \stanza{
        For Kellogg’s stance did not agree, \\
        Foundation was not trinity. \\
        God’s personality is true; \\
        Confusion came through Kellogg’s view.
    }

    \stanza{   
        Yet, Ellen stood, unyielding, firm, \\
        The early truth she did confirm. \\
        Against the tide of Trinity \\
        She held the truth of deity.
    }

    \stanza{    
        Today, as then, let’s hold the line, \\
        The early pioneer faith, divine. \\
        Truth’s legacy, let’s rightly claim, \\
        Unchanging, solid, still the same.
    }

    \stanza{    
        The way God led, let’s not forget, \\
        These principles are firmly set. \\
        Against the changing winds of doubt, \\
        Her writings guide, within, without.
    }
    
\end{titledpoem}


% \qrchapter{https://forgottenpillar.com/rsc/en-fp-chapter20}{Dr. Kellogg and Ellen White writings}


\qrchapter{https://forgottenpillar.com/rsc/es-fp-chapter20}{Dr. Kellogg y los escritos de Ellen White}


Dr. Kellogg asserted that in the Living Temple he represented the same sentiments advocated by Sister White. Likewise, today many claim that Sister White was trinitarian and was responsible for the church's acceptance of the Trinity doctrine\footnote{William Johnsson, Adventist Review, January 6th, 1994, ‘\textit{Present Truth –Walking in God’s Light}’}. Sister White, herself, declared such claims to be false.


El Dr. Kellogg afirmó que en el Living Temple representaba los mismos sentimientos defendidos por la hermana White. Asimismo, hoy en día muchos afirman que la hermana White era trinitaria y fue responsable de la aceptación por parte de la iglesia de la doctrina trinitaria\footnote{William Johnsson, Adventist Review, 6 de enero de 1994, ‘\textit{Present Truth –Walking in God's Light}’}. La propia hermana White declaró que tales afirmaciones son falsas.


\egw{\textbf{The enemy is seeking to \underline{bring in} among the people of God spiritualistic theories, which \underline{if accepted, would undermine the foundation of the faith} that has made us what we are}. He leads men to present fables clothed with Scripture. \textbf{There are those who assert that Sister White’s writings are in harmony with these teachings}.\textbf{ \underline{I declare this to be false}. Men may misapply Scripture; they may misinterpret my words; but God understands their devising}. How thankful I am for this! When the enemy comes in like a flood, \textbf{the Spirit of the Lord will lift up a standard for us against him}.}[Ms137-1903.21; 1903][https://egwwritings.org/read?panels=p9939.30]


\egw{\textbf{El enemigo está tratando de \underline{introducir} entre el pueblo de Dios teorías espiritualistas, que \underline{si son aceptadas, socavarían el fundamento de la fe} que nos ha hecho lo que somos}. Lleva a los hombres a presentar fábulas revestidas de las Escrituras. \textbf{Hay quienes afirman que los escritos de la hermana White están en armonía con estas enseñanzas}.\textbf{ \underline{Yo declaro que esto es falso}. Los hombres pueden aplicar mal la Escritura; pueden malinterpretar mis palabras; pero Dios entiende sus designios}. ¡Qué agradecida estoy por esto! Cuando el enemigo venga como una avalancha, \textbf{el Espíritu del Señor nos levantará un estandarte contra él}.}[Ms137-1903.21; 1903][https://egwwritings.org/read?panels=p9939.30]


Dr. Kellogg advocated the theories that, if accepted, would undermine the foundation of our faith. It is crucial to correctly understand what constitutes the foundation of our faith, which Sister White referred to. We have seen that it refers to the \emcap{Fundamental Principles}. Looking at her writings, and the writings of our pioneers, we see that the Trinity doctrine contradicts the \emcap{personality of God} and the truth about God’s presence. Today, with the Trinity doctrine as part of our belief, we recognize that we have moved away from the \emcap{Fundamental Principles} and formed another foundation. Sister White was not responsible for this transition. It is purely a misinterpretation of her works. Her writings do not undermine the foundation of the faith that has made us what we are. Her later work is completely in harmony with the truth given in the beginning.


El Dr. Kellogg defendió las teorías que, de ser aceptadas, socavarían el fundamento de nuestra fe. Es crucial entender correctamente lo que constituye el fundamento de nuestra fe, al que se refirió la hermana White. Hemos visto que se refiere a los \emcap{Principios Fundamentales}. Al examinar sus escritos, y los escritos de nuestros pioneros, vemos que la doctrina trinitaria contradice la \emcap{personalidad de Dios} y la verdad sobre la presencia de Dios. Hoy, con la doctrina trinitaria como parte de nuestra creencia, reconocemos que nos hemos alejado de los \emcap{Principios Fundamentales} y hemos formado otro fundamento. La hermana White no fue responsable de esta transición. Es puramente una mala interpretación de sus obras. Sus escritos no socavan el fundamento de la fe que nos ha convertido en lo que somos. Su obra posterior está completamente en armonía con la verdad dada en el principio.


\egw{\textbf{The past fifty years have not dimmed one jot or principle of our faith as we received the great and wonderful evidences that were made certain to us in 1844, after the passing of the time.} ... \textbf{\underline{Not a word is changed or denied}. That which the Holy Spirit testified to as truth after the passing of the time, in our great disappointment, \underline{is the solid foundation of truth}. \underline{Pillars of truth were revealed}, and we accepted \underline{the foundation principles} that have made us what we are—Seventh-day Adventists, keeping the commandments of God and having the faith of Jesus.}}[Lt326-1905.3; 1905][https://egwwritings.org/read?panels=p7678.9]


\egw{\textbf{Los últimos cincuenta años no han atenuado ni una pizca ni un principio de nuestra fe al recibir las grandes y maravillosas evidencias que se nos hicieron ciertas en 1844, después del paso del tiempo.} ... \textbf{\underline{Ni una sola palabra ha sido cambiada o negada}. Lo que el Espíritu Santo testificó como verdad después del paso del tiempo, en nuestro gran chasco, \underline{es el sólido fundamento de la verdad}. \underline{Se revelaron los pilares de la verdad}, y aceptamos \underline{los principios fundacionales} que nos han convertido en lo que somos—adventistas del séptimo día, guardando los mandamientos de Dios y teniendo la fe de Jesús.}}[Lt326-1905.3; 1905][https://egwwritings.org/read?panels=p7678.9]


\section*{Misrepresentation of the church standpoint}


\section*{Tergiversación del punto de vista de la iglesia}


By misrepresenting Sister White’s writings, Dr. Kellogg did not only misrepresent her work, but also the church’s official standpoint expressed in the \emcap{Fundamental Principles}. Ellen White rebuked Kellogg for misrepresenting the church’s standpoint. As we read this rebuke, let us keep in mind the church’s current standpoint on the \emcap{personality of God} as it compares to the first point of the \emcap{Fundamental Principles}.


Al tergiversar los escritos de la hermana White, el Dr. Kellogg no sólo tergiversó su obra, sino también el punto de vista oficial de la iglesia expresado en los \emcap{Principios Fundamentales}. Ellen White reprendió a Kellogg por tergiversar el punto de vista de la iglesia. Al leer esta reprimenda, tengamos en cuenta el punto de vista actual de la iglesia sobre la \emcap{personalidad de Dios} en comparación con el primer punto de los \emcap{Principios Fundamentales}.


\egw{You \textbf{are not sound in the truth}. Your statements made to believers and unbelievers \textbf{misrepresent us as a people who have not changed the truth for error}. They detract from the influence \textbf{God would have us possess before the world in revealing in plain, unmistakable language that we are \underline{true to the principles of our faith} and that we hold the beginning of our confidence firm unto the end}. We are strictly denominational. \textbf{We believe in 1903 the same truths we did believe when we established the Sanitarium and the College in Battle Creek, and \underline{we know that we had no ifs or ands about this matter}}.}[Lt300-1903.4; 1903][https://egwwritings.org/read?panels=p7705.10]


\egw{Usted \textbf{no está sano en la verdad}. Sus declaraciones hechas a los creyentes y a los incrédulos \textbf{nos desprestigian como un pueblo que no ha cambiado la verdad por el error}. Restan la influencia \textbf{que Dios quiere que tengamos ante el mundo al revelar en lenguaje claro e inequívoco que somos \underline{fieles a los principios de nuestra fe} y que mantenemos firme el principio de nuestra confianza hasta el final}. Somos estrictamente denominacionales. \textbf{Creemos en 1903 las mismas verdades que creíamos cuando establecimos el Sanatorio y el Colegio en Battle Creek, y \underline{sabemos que no tenemos peros en este asunto}}.}[Lt300-1903.4; 1903][https://egwwritings.org/read?panels=p7705.10]


\egwnogap{While you have told the things that you have and made the statements you have before unbelievers, my heart has been sad indeed. \textbf{You have evidenced that you have departed from the faith}. The very statements you have made before worldly men of influence, as the papers have reported your words, have been presented to me distinctly from your lips as you have spoken them. We cannot labor to give you influence as one whom we can trust with the sacred work connected with our institutions, for you need first to be converted and led.}[Lt300-1903.5; 1903][https://egwwritings.org/read?panels=p7705.11]


\egwnogap{Mientras usted ha dicho las cosas que ha dicho y ha hecho las declaraciones que ha hecho ante los incrédulos, mi corazón ha estado realmente triste. \textbf{Habéis evidenciado que os habéis apartado de la fe}. Las mismas declaraciones que has hecho ante hombres influyentes del mundo, tal como los periódicos han reportado tus palabras, se me han presentado claramente de tus labios tal como las has dicho. No podemos trabajar para darle influencia como alguien a quien podemos confiar con el trabajo sagrado relacionado con nuestras instituciones, pues primero hay que convertirse y ser conducido.}[Lt300-1903.5; 1903][https://egwwritings.org/read?panels=p7705.11]


\egwnogap{You are not sound in the faith. I have stated this in my diary months ago. \textbf{You have certainly placed the people of God, whom the Lord has led step by step in the ways of truth and placed upon \underline{a solid foundation}, in a false showing before unbelievers. Some have departed from the faith and \underline{will continue to misrepresent the work God has given me}}.}[Lt300-1903.6; 1903][https://egwwritings.org/read?panels=p7705.12]


\egwnogap{Usted no es sano en la fe. Lo he declarado en mi diario hace meses. \textbf{Ciertamente habéis colocado al pueblo de Dios, al que el Señor ha guiado paso a paso por los caminos de la verdad y ha colocado sobre \underline{un sólido fundamento}, en una falsa muestra ante los incrédulos. Algunos se han apartado de la fe y \underline{seguirán tergiversando la obra que Dios me ha dado}}.}[Lt300-1903.6; 1903][https://egwwritings.org/read?panels=p7705.12]


\egwnogap{\textbf{The sanctuary question is a clear and definite doctrine as we have held it as a people. \underline{You are not definitely clear on the personality of God, which is everything to us as a people}. \underline{You have virtually destroyed the Lord God Himself}}.}[Lt300-1903.7; 1903][https://egwwritings.org/read?panels=p7705.13]


\egwnogap{\textbf{La cuestión del santuario es una doctrina clara y definida tal como la hemos sostenido como pueblo. \underline{No estás definitivamente claro en cuanto a la personalidad de Dios, que lo es todo para nosotros como pueblo}. \underline{Has destruido virtualmente al Señor Dios mismo}}.}[Lt300-1903.7; 1903][https://egwwritings.org/read?panels=p7705.13]


\egwnogap{Why should you take the liberty to make the statements which you have made, as though you had authority for thus stating, when they are falsehoods? \textbf{You have made the facts of our faith of none effect before unbelievers,} \textbf{and the truth which should ever be kept prominent and exalted with this people you have virtually denied and ignored in your many statements. How dared you to do this?} \textbf{It necessitates us now to present our true position which constitutes us Seventh-day Adventists}. Whatever influence God has given you in the past has been in mercy to you, letting the light shine upon you.}[Lt300-1903.8; 1903][https://egwwritings.org/read?panels=p7705.14]


\egwnogap{¿Por qué te tomaste la libertad de hacer las afirmaciones que has hecho, como si tuvieras autoridad para ello, cuando son falsedades? \textbf{Has hecho que los hechos de nuestra fe no tengan ningún efecto ante los incrédulos,} \textbf{y la verdad que debería mantenerse siempre prominente y exaltada con este pueblo la has negado e ignorado virtualmente en tus numerosas declaraciones. ¿Cómo te atreviste a hacer esto?} \textbf{Es necesario que ahora presentemos nuestra verdadera posición que nos constituye como adventistas del séptimo día}. Cualquier influencia que Dios te haya dado en el pasado ha sido por misericordia hacia ti, dejando que la luz brille sobre ti.}[Lt300-1903.8; 1903][https://egwwritings.org/read?panels=p7705.14]


\egwnogap{\textbf{We cannot for a moment have any misrepresentation upon these solemn and important subjects of truth which have been the faith of our people since 1844. This means much to us.} The Lord would have me say to you that the enemy has, through his specious deceptions, placed his unbelief in your mind, and you have been working it out. \textbf{\underline{All who receive your presentations will enter upon strange paths if they connect with you}}. \textbf{You are \underline{bringing in} strange, common fire}, \textbf{but not the fire of God’s own kindling}; and now \textbf{I must speak plainly to our people that the Lord has led us step by step and shown us clear light upon the heavenly sanctuary in the most holy of holies where \underline{God revealed Himself} to His appointed ones.}}[Lt300-1903.9; 1903][https://egwwritings.org/read?panels=p7705.15]


\egwnogap{\textbf{No podemos tener ni por un momento ninguna tergiversación sobre estos solemnes e importantes temas de la verdad que han sido la fe de nuestro pueblo desde 1844. Esto significa mucho para nosotros.} El Señor quiere que te diga que el enemigo, por medio de sus engaños especiosos, ha colocado su incredulidad en tu mente, y has estado trabajando en ella. \textbf{\underline{Todos los que reciban tus presentaciones entrarán en caminos extraños si se conectan contigo}}. \textbf{Estás \underline{trayendo} fuego extraño y común}, \textbf{pero no el fuego del propio encendido de Dios}; y ahora \textbf{debo hablar claramente a nuestro pueblo de que el Señor nos ha guiado paso a paso y nos ha mostrado una luz clara sobre el santuario celestial en el santísimo de los santos donde \underline{Dios se reveló a Sí mismo} a sus designados.}}[Lt300-1903.9; 1903][https://egwwritings.org/read?panels=p7705.15]


Dr. Kellogg misrepresented the truth that constituted the foundation of our faith; most specifically, he misrepresented the truth on the \emcap{personality of God}, which was everything to us as people. If in 1903, it necessitated \egwinline{\textbf{to present our true position which constitutes us Seventh-day Adventists}}, how much more important is it for us today? Sister White did her part in upholding the foundation of our faith in the beginning, but it seems like we have forgotten.


El Dr. Kellogg tergiversó la verdad que constituía el fundamento de nuestra fe; más específicamente, tergiversó la verdad sobre la \emcap{personalidad de Dios}, que lo era todo para nosotros como pueblo. Si en 1903, era necesario \egwinline{\textbf{presentar nuestra verdadera posición que nos constituye como adventistas del séptimo día}}, ¿cuánto más importante es para nosotros hoy? La hermana White hizo su parte al sostener el fundamento de nuestra fe en el principio, pero parece que lo hemos olvidado.


\input{lang/en/poems/chapter20.tex}


% \input{lang/es/poems/chapter20.tex}



