\qrchapter{https://forgottenpillar.com/rsc/en-fp-chapter6}{Examining the test}


\qrchapter{https://forgottenpillar.com/rsc/es-fp-chapter6}{Examinando la prueba}


In Sister White's reply, to Dr. Kellogg's belief on the Trinity doctrine and his attempts to \textit{patch up} the Living Temple, we see that she viewed the Trinity doctrine as contradicting the light given her regarding \emcap{the personality of God}. If she had actually embraced the Trinity doctrine, we would expect her to carefully separate it from pantheism and preserve its legitimate aspects. However, this is not what we see in her response. Instead, her response was to contrast the Trinity doctrine with the truth about the \emcap{personality of God}, recalling her past visions which showed that this doctrine would rob God's people of their past experiences. In her reactive recalling of how God established the \emcap{fundamental principles}, she indicated that the Trinity doctrine \textit{tears down the pillars of our faith} and \textit{leads us astray from the foundation principles}. This stark difference can be clearly seen by comparing our current Fundamental Beliefs with the \emcap{Fundamental Principles} held in the past.


En la respuesta de la hermana White a la creencia del Dr. Kellogg sobre la doctrina trinitaria y sus intentos de \textit{emparejar} el Templo Viviente, vemos que ella consideraba la doctrina trinitaria como contradictoria a la luz que le fue dada respecto a \emcap{la personalidad de Dios}. Si ella realmente hubiera adoptado la doctrina trinitaria, esperaríamos que la separara cuidadosamente del panteísmo y preservara sus aspectos legítimos. Sin embargo, esto no es lo que vemos en su respuesta. En cambio, su respuesta fue contrastar la doctrina trinitaria con la verdad sobre \emcap{la personalidad de Dios}, recordando sus visiones pasadas que mostraban que esta doctrina robaría al pueblo de Dios sus experiencias pasadas. En su recuerdo reactivo de cómo Dios estableció los \emcap{principios fundamentales}, ella indicó que la doctrina trinitaria \textit{derriba los pilares de nuestra fe} y \textit{nos desvía de los principios fundacionales}. Esta marcada diferencia puede verse claramente al comparar nuestras Creencias Fundamentales actuales con los \emcap{Principios Fundamentales} sostenidos en el pasado.


Keeping in mind Sister White’s reply to Dr. Kellogg's belief on the Trinity doctrine, let us review the characteristics of the theories she described in the chapter “\textit{The Foundation of our Faith}”. When Sister White is speaking of Kellogg’s theories of God, our question should be, “do her quotations make sense if the Trinity doctrine is applied to their context?” Let’s examine each characteristic.


Teniendo en cuenta la respuesta de la hermana White a la creencia del Dr. Kellogg sobre la doctrina trinitaria, revisemos las características de las teorías que ella describió en el capítulo “\textit{La Fundación de nuestra Fe}”. Cuando la hermana White habla de las teorías de Kellogg sobre Dios, nuestra pregunta debería ser, “¿tienen sentido sus citas si se aplica la doctrina trinitaria a su contexto?” Examinemos cada característica.


\subsection*{Does the Trinity “rob the people of God of their past experience”?}


\subsection*{¿La Trinidad “roba al pueblo de Dios su experiencia pasada”?}


\egw{They \normaltext{[the spiritualistic theories]} make of no effect the truth of heavenly origin, and \textbf{rob the people of God of their past experience}, giving them instead a false science.}[SpTB02 54.1; 1904][https://egwwritings.org/read?panels=p417.275]


\egw{Ellos \normaltext{[las teorías espiritualistas]} dejan sin efecto la verdad del origen celestial, y \textbf{roban al pueblo de Dios su experiencia pasada}, dándole en cambio una ciencia falsa.}[SpTB02 54.1; 1904][https://egwwritings.org/read?panels=p417.275]


\egw{This foundation was built by the Masterworker, and will stand storm and tempest. Will they permit this man \normaltext{[Kellogg]} to present \textbf{doctrines that deny the past experience of the people of God}? The time has come to take decided action.}[SpTB02 54.2; 1904][https://egwwritings.org/read?panels=p417.276]


\egw{Este fundamento fue construido por el Maestro Trabajador, y resistirá la tormenta y la tempestad. ¿Permitirán que este hombre \normaltext{[Kellogg]} presente \textbf{doctrinas que niegan la experiencia pasada del pueblo de Dios}? Ha llegado el momento de tomar una acción decidida.}[SpTB02 54.2; 1904][https://egwwritings.org/read?panels=p417.276]


\egw{\textbf{What influence is it that would lead men at this stage of our history to work in an underhanded, powerful way to \underline{tear down the foundation of our faith},—the foundation that was laid \underline{at the beginning of our work} by prayerful study of the word and by revelation? Upon this foundation \underline{we have been building for the past fifty years}}. Do you wonder that when I see the beginning of a work \textbf{that would \underline{remove some of the pillars of our faith},} I have something to say? I must obey the command, ‘Meet it!’}[SpTB02 58.1; 1904][https://egwwritings.org/read?panels=p417.295]


\egw{\textbf{¿Qué influencia es la que llevaría a los hombres, en esta etapa de nuestra historia, a trabajar de manera sutil y poderosa para \underline{derribar la fundación de nuestra fe},—la fundación que fue colocada \underline{al comienzo de nuestra obra} mediante el estudio en oración de la palabra y la revelación? Sobre este fundamento \underline{hemos estado construyendo durante los últimos cincuenta años}}. ¿Se extrañan de que cuando veo el comienzo de una obra \textbf{que \underline{removería algunos de los pilares de nuestra fe},} tenga algo que decir? Debo obedecer el mandato: ‘¡Confrontalo!’}[SpTB02 58.1; 1904][https://egwwritings.org/read?panels=p417.295]


According to Sister White’s testimony, the foundation of our faith was the \emcap{Fundamental Principles}. Currently, these do not represent our beliefs. Most objectionable is the first point, concerning who God is. Instead of the belief that there is one God—the Father, a personal spiritual being, we have a new belief that there is one God—Father, Son, and Holy Spirit, a unity of three Persons. From the light and the experiences of how God established the first point of the \emcap{Fundamental Principles}, does the newly formed doctrine about who God is and what He is, has robbed the people of God of their past experience?


Según el testimonio de la hermana White, el fundamento de nuestra fe eran los \emcap{Principios Fundamentales}. Actualmente, estos no representan nuestras creencias. Lo más objetable es el primer punto, referente a quién es Dios. En lugar de la creencia de que hay un solo Dios—el Padre, un ser personal y espiritual, tenemos una nueva creencia de que hay un solo Dios—Padre, Hijo y Espíritu Santo, una unidad de tres Personas. A la luz de las experiencias de cómo Dios estableció el primer punto de los \emcap{Principios Fundamentales}, ¿la doctrina recién formada sobre quién es Dios y qué es, ha robado al pueblo de Dios su experiencia pasada?


\subsection*{Does the Trinity tear down the pillars of our faith, or lead astray from foundation principles?}


\subsection*{¿Acaso la trinidad derriba los pilares de nuestra fe, o desvía de los principios fundamentales?}


\egw{I have been instructed by the heavenly messenger that some of the reasoning in the book, ‘Living Temple,’ is unsound and that \textbf{this reasoning would lead astray the minds of those who are not thoroughly established on the foundation principles of present truth.}}[SpTB02 51.3; 1904][https://egwwritings.org/read?panels=p417.262]


\egw{He sido instruida por el mensajero celestial de que algunos de los razonamientos en el libro, ‘Templo Viviente’, no son sólidos y que \textbf{este razonamiento desviaría las mentes de aquellos que no están completamente establecidos en los principios fundamentales de la verdad presente.}}[SpTB02 51.3; 1904][https://egwwritings.org/read?panels=p417.262]


\egw{About the time that ‘Living Temple’ was published, there passed before me in the night season, representations indicating that some \textbf{danger was approaching}, and that I must prepare for it by writing out the things God has revealed to me \textbf{regarding the foundation principles of our faith}.}[SpTB02 52.3; 1904][https://egwwritings.org/read?panels=p417.267]


\egw{Alrededor del tiempo en que se publicó ‘El Templo Viviente’, pasaron ante mí, en la estación nocturna, representaciones que indicaban que \textbf{se acercaba algún peligro}, y que debía prepararme para él, escribiendo las cosas que Dios me había revelado \textbf{con respecto a los principios fundamentales de nuestra fe}.}[SpTB02 52.3; 1904][https://egwwritings.org/read?panels=p417.267]


\egw{\textbf{The enemy of souls has sought to bring in the supposition that a great reformation was to take place among Seventh-day Adventists, and that this reformation would consist in \underline{giving up the doctrines which stand as the pillars of our faith,} and engaging in a process of reorganization}. Were this reformation to take place, what would result? \textbf{The principles of truth} that God in His wisdom has given to the remnant church, \textbf{would be discarded}. Our religion would be changed. \textbf{The fundamental principles} that have sustained the work for the last fifty years \textbf{would be accounted as error}. A new organization would be established. Books of a new order would be written. A system of intellectual philosophy would be introduced.}[SpTB02 54.3; 1904][https://egwwritings.org/read?panels=p417.277]


\egw{\textbf{El enemigo de las almas ha tratado de introducir la suposición de que iba a tener lugar una gran reforma entre los adventistas del séptimo día, y que esta reforma consistiría en \underline{abandonar las doctrinas que son los pilares de nuestra fe,} y en emprender un proceso de reorganización}. Si esta reforma tuviera lugar, ¿qué resultaría? \textbf{Los principios de la verdad} que Dios, en su sabiduría, ha dado a la iglesia remanente, \textbf{serían descartados}. Nuestra religión cambiaría. \textbf{Los principios fundamentales} que han sostenido la obra durante los últimos cincuenta años \textbf{se considerarían un error}. Se establecería una nueva organización. Se escribirían libros de un nuevo orden. Se introduciría un sistema de filosofía intelectual.}[SpTB02 54.3; 1904][https://egwwritings.org/read?panels=p417.277]


Dr. Kellogg’s theories on the \emcap{personality of God}, if accepted, would ignite a reformation within the Seventh-day Adventist Church. Based on intellectual philosophy, they would cause us to renounce some of the doctrines that stand as the pillars of our faith, condemning the \emcap{Fundamental Principles} as error. Could it be that by adhering to the Trinity doctrine we entered into a new organization?


Las teorías del Dr. Kellogg sobre la \emcap{personalidad de Dios}, de ser aceptadas, encenderían una reforma dentro de la Iglesia Adventista del Séptimo Día. Basadas en la filosofía intelectual, nos harían renunciar a algunas de las doctrinas que se erigen como pilares de nuestra fe, condenando los \emcap{Principios Fundamentales} como error. ¿Será que al adherirnos a la doctrina trinitaria entramos en una nueva organización?


\egw{Shortly before I sent out the testimonies \textbf{regarding the efforts of the enemy to undermine the foundation of our faith through the dissemination of seductive theories}, I had read an incident about a ship in a fog meeting an iceberg…}[SpTB02 55.3; 1904][https://egwwritings.org/read?panels=p417.282]


\egw{Poco antes de enviar los testimonios \textbf{relativos a los esfuerzos del enemigo por socavar el fundamento de nuestra fe mediante la difusión de teorías seductoras}, había leído un incidente sobre un barco en la niebla que se encontró con un iceberg...}[SpTB02 55.3; 1904][https://egwwritings.org/read?panels=p417.282]


\egw{Messages of every order and kind have been \textbf{urged upon Seventh-day Adventists, to take the place of the truth which, \underline{point by point}, has been sought out by prayerful study, and testified to by the miracle-working power of the Lord}. \textbf{But the way-marks which have made us what we are, are to be preserved, and they will be preserved}, as God has signified through His word and the testimony of His Spirit. \textbf{He calls upon us to hold firmly}, with the grip of faith, \textbf{to \underline{the fundamental principles} that are based upon \underline{unquestionable authority}}.}[SpTB02 59.1; 1904][https://egwwritings.org/read?panels=p417.299]


\egw{Se ha instado a los adventistas del séptimo día a que reciban mensajes de todo tipo y clase, \textbf{para que ocupen el lugar de la verdad que, \underline{punto por punto}, ha sido buscada por medio del estudio en oración, y testificada por el poder milagroso del Señor}. \textbf{Pero las marcas del camino que nos han convertido en lo que somos, deben ser preservadas, y serán preservadas}, tal como Dios lo ha indicado mediante su palabra y el testimonio de su Espíritu. \textbf{Él nos pide que nos aferramos firmemente}, con la garra de la fe, \textbf{a los \underline{principios fundamentales} que se basan en una \underline{autoridad incuestionable}}.}[SpTB02 59.1; 1904][https://egwwritings.org/read?panels=p417.299]


The \emcap{personality of God} was the pillar of our faith\footnote{\href{https://egwwritings.org/?ref=en_Ms62-1905.14}{EGW, Ms62-1905.14; 1905}}. The \emcap{personality of God} was expressed in the first point of the \emcap{Fundamental Principles}. Could it be that by adhering to the Trinity doctrine we have torn down this particular pillar of our faith? Is it possible that by accepting the Trinity doctrine we were led astray from this foundation principle—the \emcap{personality of God}?


La \emcap{personalidad de Dios} era el pilar de nuestra fe\footnote{\href{https://egwwritings.org/?ref=en_Ms62-1905.14}{EGW, Ms62-1905.14; 1905}}. La \emcap{personalidad de Dios} se expresaba en el primer punto de los \emcap{Principios Fundamentales}. ¿Será que al adherirnos a la doctrina trinitaria hayamos derribado este pilar particular de nuestra fe? ¿Es posible que al aceptar la doctrina trinitaria nos hayamos desviado de este principio fundamental–  la personalidad de Dios?


\subsection*{Does the Trinity do away with the personality of God?}


\subsection*{¿Acaba la Trinidad con la personalidad de Dios?}


\egw{\textbf{It \normaltext{[The Living Temple]} introduces that which is naught but \underline{speculation} in regard to \underline{the personality of God} and where His presence is.}}[SpTB02 51.3; 1904][https://egwwritings.org/read?panels=p417.262]


\egw{\textbf{Esto \normaltext{[El Templo Viviente]} introduce lo que no es más que \underline{especulación} con respecto a \underline{la personalidad de Dios} y donde está Su presencia.}}[SpTB02 51.3; 1904][https://egwwritings.org/read?panels=p417.262]


\egw{\textbf{The spiritualistic theories \underline{regarding the personality of God}, followed to their logical conclusion, sweep away the whole Christian economy.}}[SpTB02 54.1; 1904][https://egwwritings.org/read?panels=p417.275]


\egw{\textbf{Las teorías espiritualistas \underline{respecto a la personalidad de Dios}, seguidas hasta su conclusión lógica, barren con toda la economía cristiana.}}[SpTB02 54.1; 1904][https://egwwritings.org/read?panels=p417.275]


\egw{‘Living Temple’ contains the alpha of these theories. I knew that the omega would follow in a little while; and I trembled for our people. I knew that \textbf{I must warn our brethren and sisters not to enter into controversy over \underline{the presence} and \underline{personality of God}. The statements made in ‘Living Temple’ \underline{in regard to this point are incorrect}. The scripture used to substantiate the doctrine there set forth, is scripture misapplied}.}[SpTB02 53.2; 1904][https://egwwritings.org/read?panels=p417.271]


\egw{‘Templo Viviente’ contiene el alfa de estas teorías. Sabía que la omega vendría dentro de poco; y temí por nuestro pueblo. Sabía que \textbf{debía advertir a nuestros hermanos y hermanas que no entraran en controversias sobre \underline{la presencia} y \underline{la personalidad de Dios}. Las afirmaciones hechas en ‘Living Temple’ \underline{con respecto a este punto son incorrectas}. La escritura utilizada para fundamentar la doctrina allí expuesta, es una escritura mal aplicada}.}[SpTB02 53.2; 1904][https://egwwritings.org/read?panels=p417.271]


The theories Kellogg presented in the Living Temple are speculative in regard to the \emcap{personality of God} and where His presence is. These theories deal with the question of the quality or state of God being a person\footnote{The Merriam-Webster definition of ‘\textit{personality}’ - “\textit{the quality or state of being a person}”}. God has given us definite light regarding this issue in our \emcap{Fundamental Principles}. Could it be that the Trinity doctrine is casting doubt on this definite light regarding the \emcap{personality of God}?


Las teorías que Kellogg presentó en el Living Temple son especulativas con respecto a la \emcap{personalidad de Dios} y dónde está su presencia. Estas teorías tratan la cuestión de la cualidad o estado de Dios siendo una persona\footnote{La definición de Merriam-Webster de ‘\textit{personalidad}’ - “\textit{la cualidad o estado de ser una persona}”}. Dios nos ha dado una luz definitiva sobre esta cuestión en nuestros \emcap{Principios Fundamentales}. ¿Podría ser que la doctrina trinitaria esté poniendo en duda esta luz definitiva con respecto a la \emcap{personalidad de Dios}?


\subsection*{Is the Trinity doctrine presented as if Mrs. White supported it?}


\subsection*{¿Se presenta la doctrina trinitaria como si la Sra. White la apoyara?}


\egw{In the controversy that arose among our brethren \textbf{regarding the teachings of this book,} those in favor of giving it a wide circulation \textbf{declared: ‘It contains the very sentiments that Sister White has been teaching.’ This assertion struck right to my heart. I felt heart-broken; for I knew that this representation of the matter was not true}.}[SpTB02 53.1; 1904][https://egwwritings.org/read?panels=p417.270]


\egw{En la controversia que surgió entre nuestros hermanos \textbf{respecto a las enseñanzas de este libro,} los que estaban a favor de darle una amplia circulación \textbf{declararon: ‘Contiene los mismos sentimientos que la hermana White ha estado enseñando’. Esta afirmación me llegó al corazón. Me sentí con el corazón destrozado, porque sabía que esta representación del asunto no era cierta}.}[SpTB02 53.1; 1904][https://egwwritings.org/read?panels=p417.270]


\egw{\textbf{I am compelled to speak in denial of the claim that the teachings of ‘Living Temple’ can be sustained by statements from my writings}. There may be in this book expressions and sentiments that are in harmony with my writings. And there may be in my writings many statements which, taken from their connection, and interpreted according to the mind of the writer of ‘Living Temple,’ would seem to be in harmony with the teachings of this book. This may give apparent support to the assertion that the sentiments in ‘Living Temple’ are in harmony with my writings. \textbf{But God forbid that this sentiment should prevail}.}[SpTB02 53.3; 1904][https://egwwritings.org/read?panels=p417.272]


\egw{\textbf{Me veo obligada a hablar para negar la afirmación de que las enseñanzas del ‘Living Temple’ pueden ser sostenidas por declaraciones de mis escritos}. Puede haber en este libro expresiones y sentimientos que estén en armonía con mis escritos. Y puede haber en mis escritos muchas declaraciones que, tomadas de su conexión, e interpretadas según la mente del escritor de ‘Living Temple’, parecerían estar en armonía con las enseñanzas de este libro. Esto puede dar un apoyo aparente a la afirmación de que los sentimientos en ‘Living Temple’ están en armonía con mis escritos. \textbf{Pero Dios no permita que este sentimiento prevalezca}.}[SpTB02 53.3; 1904][https://egwwritings.org/read?panels=p417.272]


At this point, we have many unanswered questions. But, as we continue to study the first point of the \emcap{Fundamental Principles}, we will find answers to all of these questions. So far, in light of the \emcap{Fundamental Principles}, belief in the Trinity doctrine—as a Seventh-day Adventist—becomes very questionable. In order to defend the Trinity doctrine, the authority of the \emcap{Fundamental Principles} must be compromised. In what follows, we will briefly study their authority, context in Adventist history, and God’s purpose in giving them. We will also look at the true authorship of the \emcap{Fundamental Principles} and their role in present days.


En este punto, tenemos muchas preguntas sin respuesta. Pero, a medida que continuemos estudiando el primer punto de los \emcap{Principios Fundamentales}, encontraremos respuestas a todas estas preguntas. Hasta ahora, a la luz de los \emcap{Principios Fundamentales}, la creencia en la doctrina trinitaria—como adventista del séptimo día—resulta muy cuestionable. Para defender la doctrina trinitaria, la autoridad de los \emcap{Principios Fundamentales} debe ser comprometida. En lo que sigue, estudiaremos brevemente su autoridad, su contexto en la historia adventista y el propósito de Dios al darlos. También examinaremos la verdadera autoría de los \emcap{Principios Fundamentales} y su papel en la actualidad.


% Examining Test

\begin{titledpoem}
    
    \stanza{
        Sister White's vision stands against the tide, \\
        As Trinity's doctrine she firmly denied. \\
        Her words a warning, clear and bright, \\
        Against teachings that dimmed revealed light.
    }

    \stanza{
        The pillars of faith, established with care, \\
        Now face a challenge, a doctrine to beware. \\
        For what was built through prayer and revelation, \\
        Faces change through doctrinal innovation.
    }

    \stanza{
        The personality of God, once clearly known, \\
        Now wrapped in theories not heaven's own. \\
        Past experiences of God's people at stake, \\
        When foundations new doctrines would break.
    }

    \stanza{
        The waymarks that made us what we are, \\
        Should guide us still, like a guiding star. \\
        Hold firmly to principles with faith's strong grip, \\
        Lest in confusion's fog we lose our ship.
    }
\end{titledpoem}


% Looking at the Polish text, I found only one minor grammar issue:

# Corrected spelling of “jet” to “jest”
[Wprowadza to \normaltext{[The Living Temple]}, co jest niczym innym jak \underline{spekulacją} w odniesieniu do \underline{osobowości Boga} i tego, gdzie jet Jego obecność]
->
[Wprowadza to \normaltext{[The Living Temple]}, co jest niczym innym jak \underline{spekulacją} w odniesieniu do \underline{osobowości Boga} i tego, gdzie jest Jego obecność]
---------
